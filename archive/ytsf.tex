% Start: 2021:05:09 21:20:02+07:00 

 \documentclass[a4paper,10pt,fleqn]{book}
\usepackage{polyglossia} 
\setdefaultlanguage[babelshorthands=true]{russian}
\setotherlanguage{english}
\defaultfontfeatures{Ligatures=TeX,Mapping=tex-text}

\setmainfont{Linux Libertine}

\usepackage{xltxtra}
\usepackage{microtype}

\newcommand{\mulang}[3]{#2}

\usepackage{xcolor}
\definecolor{darkblue}{HTML}{003153}
\usepackage{graphicx}
\usepackage {textcomp}
\usepackage[twoside,left=2.5cm,right=3cm,top=3cm,bottom=4cm,bindingoffset=0cm]{geometry}
\usepackage{epigraph}
\renewcommand{\epigraphsize}{\footnotesize}
\epigraphrule=0pt
\epigraphwidth=8cm
\usepackage{soul, xspace}
\xspaceaddexceptions{ < ) }
\sodef\so{}{.1em}{1em}{.3em plus.05em minus.05em}
\newcommand{\ldotst}{\so{...}\xspace}
\newcommand{\ldotsq}{\so{?\hbox{\hspace{-.212em}}..}\xspace}
\newcommand{\ldotse}{\so{!..}\xspace}
\newcommand{\razd}{~\\{\centering\Large\bfseries$\ast \ast \ast$\par}~\\}
\newcommand{\spacing}{\textcolor{red}{\textbf{(разрыв)}}}
\newcommand{\authornote}{\emph{"--- Прим. авт.}}

\usepackage{etoolbox}
\AtBeginEnvironment{quote}{\itshape}
\makeatletter
\newlength\episourceskip
\pretocmd{\@episource}{\em}{}{}
\apptocmd{\@episource}{\em}{}{}
\patchcmd{\epigraph}{\@episource{#1}\\}{\@episource{#1}\\[\episourceskip]}{}{}
\makeatother

\usepackage{amsmath}
\usepackage{amssymb}
\usepackage{amsfonts}

\usepackage{hyperref}
\hypersetup{colorlinks=true, linkcolor=darkblue, citecolor=darkblue, filecolor=darkblue, urlcolor=darkblue}

\usepackage{fancyhdr}
\pagestyle{fancy}
\fancyhead[LE,RO]{\thepage}
\fancyhead[LO]{\small
ЁТЛАНД, КАМЕННЫЙ ЛЕС}
\fancyhead[RE]{\small
ЭМИЛЬ ВЕСНА}
\fancyfoot{}
\fancypagestyle{plain}
{
\fancyhead{}
\renewcommand{\headrulewidth}{0mm}
\fancyfoot{}
}
\fancypagestyle{requisit}
{
\fancyhead{}
\renewcommand{\headrulewidth}{0mm}
}

\begin{document}

И я "--- Лия.
Я по старой памяти всё ещё называю себя Серёгой, но новое прозвище тоже сумел принять.
Давно утерянный паспорт с тупой уверенностью утверждал, что меня зовут Шахович Лилия Борисовна.
Серёгой я стал в четырнадцать лет, когда осознал себя мужчиной.
Лией меня в двадцать два окрестили люди "--- ночным кошмаром, крадущимся в ночи убийцей, которым пугали маленьких детей.

\spacing

Фурия "--- с ударением на последний слог "--- были единственным в нашей компании небинаром.
В отличие от обычных трансов, небинары часто в ходе трансформации теряли человеческий облик, за что и были ненавидимы во всех без исключения человеческих сообществах.
Трансов часто воспитывали как своих "--- даже без использования Силы они становились отличными лекарями и бойцами, но небинаров пускали под нож сразу, едва появлялись признаки того, что люди называли <<мутацией>>.
Фурия повезло "--- они повзрослели ещё до ядерной войны.
За два года трансформации Фурия отрастили кожистые крылья, тонкий хвост с перепонкой и два комплекта грудей;
всё их мощное гибкое тело, за исключением живота, плеч, бёдер и части лица, покрывала тонкая гладкая чешуя.
Глаза Фурия то светились жёлтым, то становились чернее ночи.

\spacing

Девочка снова промолчала.

Я вздохнул.
Лицо девочки оставалось каменно-неподвижным;
Чиргон не оставлял детям ни единого шанса побыть детьми.
Как, впрочем, и вся Орда.

Территория Орды ныне охватывала большую часть страны, которая когда-то называлась Россией.
Передвигаться по Орде можно было двумя способами: по-человечески, пешком, рискуя погибнуть от радиации, голода или жары, либо так, как это делаем мы.
Наш лагерь располагался в живописном месте "--- Бахал, бывший Бакалинский район;
там была незараженная питьевая вода, которую могли пить даже люди, там были заброшенные поля, на которых росла вкуснейшая одичавшая свёкла, там были мустанги, косули и вепри.
Впрочем, крупное зверьё мы по неписанному соглашению старались не трогать.
Правильно приготовленная крыса ничем не хуже.
Если же очень хотелось жирного мяса и яиц, мы отправлялись в рейд на тетеревов и квочек "--- диких кур, которые во время ядерной зимы совершенно неожиданно научились летать, давая фору своим лесным собратьям.

Чиргону повезло куда меньше.
Территория бывшей Черметпереработки находилась в самом сердце Орды "--- неподалёку от Безволги, каменистого провала на месте старого русла угадайте-какой-реки.
Руины завода были в рекордные сроки превращены людьми в бункер.
Оплавленные стены, наскоро обшитые листовым металлом, щерились пулемётами и перископами, на главной башне были даже видеокамеры "--- те, что сумели пережить суровый ордынский климат.
У Чиргона не хватало ресурсов расширить жалкий лоскуток леса, чтобы закрыть себя от палящих ветров.
Лес облюбовали злые как черти волкособы, которые регулярно проверяли оборону Чиргона на прочность.
Сжечь немногие уцелевшие деревья у людей не поднималась рука.

<<Но проблема для них, разумеется, не волкособы, а мы>>.

"--*Слушай, Лия, нам надо двигаться дальше, "--- сказал Лорд.

Тут он прав.
Нашей целью были земли восточнее Орды "--- Ётланд, Каменный лес.
Мёртвая тайга, которую богатые солями дождь и пепел превратили в камень.
Говорят, что там можно было найти места с уцелевшей экосистемой.
Но девочка\ldotst

"--*Девчонку отведём к воротам Чиргона, "--- заявил Лорд.

"--*Ага, и её покрошат вместе с нами, "--- флегматично парировали чешуйчатые Фурия.

"--*Ты Силой владеешь или как?

"--*Лорд, заткнись, "--- Саманта была единственной, кого Лорд слушался.
"--- Очевидно же, что её выбросили.

"--*Она цис!

"--*А чиргонцы только трансов выбрасывают за ворота?
Всё, хватить душнить.
Девчонку возьмем в Бахал.

"--*Мы договорились, что до Ётланда пойдём впятером!

"--*В таком случае возвращаемся.

"--*Ну заебись, блядь.

Лорд схватил миску с едой и демонстративно опорожнил в огонь.
Запахло горелым.

"--*Я согласна с Лордом.

Ясень подала голос впервые за вечер.
Мы все уставились на неё.

"--*Вы можете считать меня предвзятой, но я не вижу в девочке вообще ничего особенного.
Ничего такого, за что её могли бы выкинуть наружу.
Наиболее очевидный для меня вывод "--- с Чиргоном что-то не так.

"--*Так, согласная ты моя, "--- начал Лорд.
"--- Я не имел в виду, что нам надо лезть в этот муравейник и выяснять\ldotst

"--*А я имела.
Лия?
Фурия?

"--*Окей, завтра, "--- кивнул я.
"--- Давайте спать.
Солнышко, ты что-нибудь ещё хочешь?

Девочка помотала головой.

"--*Ну, кроме того, что она хочет к маме, вроде бы действительно всё в порядке, "--- пожали плечами Фурия.
Сканирующие желтые глаза мигнули и угасли.
"--- Будешь спать со мной?
Я тёплые.

Девочка бросила испуганный взгляд на чешуйчатые шестипалые руки и прижалась ко мне.

"--*Да ладно, ладно, я не в обиде, "--- слегка обиженно проворчали Фурия.
"--- Ох уж эти люди\ldotst

\razd

Ночь выдалась холодной.
К утру заморосил ледяной дождь.
Девочка, оставив попытки согреться под моим одеялом, залезла под горячее, покрытое сеткой огненных артерий крыло Фурия.
К завтраку они были лучшими подругами.

"--*Я тебя возьму с собой на квочек охотиться.
Видела квочек?

Фурия дунули чем-то похожим на дым "--- и в воздухе повисла толстенькая курочка.
Девочка засмеялась.

"--*А это квочкина ножка из нашего пайка.
Жареная.
Она немножко подсохла, свежие гораздо сочнее.
Ешь.
Вкусно?
Конечно, вкусно!

Девочка за это время не произнесла ни слова.
Фурия читали мысли напрямую из её головы.

\end{document}
