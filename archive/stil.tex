\documentclass[a4paper,10pt,fleqn]{book}
\usepackage{polyglossia} 
\setdefaultlanguage[babelshorthands=true]{russian}
\setotherlanguage{english}
\defaultfontfeatures{Ligatures=TeX,Mapping=tex-text}

\setmainfont{Linux Libertine}

\usepackage{xltxtra}
\usepackage{microtype}
 
\usepackage{xcolor}
\definecolor{darkblue}{HTML}{003153}
\usepackage{graphicx}
\usepackage {textcomp}
\usepackage[twoside,left=2.5cm,right=3cm,top=3cm,bottom=4cm,bindingoffset=0cm]{geometry}
\usepackage{epigraph}
\renewcommand{\epigraphsize}{\footnotesize}
\epigraphrule=0pt
\epigraphwidth=8cm
\usepackage{soul, xspace}
\xspaceaddexceptions{ < ) }
\sodef\so{}{.1em}{1em}{.3em plus.05em minus.05em}
\newcommand{\ldotst}{\so{...}\xspace}
\newcommand{\ldotsq}{\so{?\hbox{\hspace{-.212em}}..}\xspace}
\newcommand{\ldotse}{\so{!..}\xspace}
\newcommand{\razd}{~\\{\centering\Large\bfseries$\ast \ast \ast$\par}~\\}
\newcommand{\spacing}{\textcolor{red}{\textbf{(разрыв)}}}
\newcommand{\authornote}{\emph{"--- Прим. авт.}}

\usepackage{etoolbox}
\AtBeginEnvironment{quote}{\itshape}
\makeatletter
\newlength\episourceskip
\pretocmd{\@episource}{\em}{}{}
\apptocmd{\@episource}{\em}{}{}
\patchcmd{\epigraph}{\@episource{#1}\\}{\@episource{#1}\\[\episourceskip]}{}{}
\makeatother

\usepackage{amsmath}
\usepackage{amssymb}
\usepackage{amsfonts}

\usepackage{hyperref}
\hypersetup{colorlinks=true, linkcolor=darkblue, citecolor=darkblue, filecolor=darkblue, urlcolor=darkblue}

\usepackage{fancyhdr}
\pagestyle{fancy}
\fancyhead[LE,RO]{\thepage}
\fancyhead[LO]{{\footnotesize ЗАСТЫВШИЙ}}
\fancyhead[RE]{{\footnotesize ЭМИЛЬ ХЕЛЬГАСОН}} 
\fancyfoot{}
\fancypagestyle{plain}
{
\fancyhead{}
\renewcommand{\headrulewidth}{0mm}
\fancyfoot{}
}
\fancypagestyle{requisit}
{
\fancyhead{}
\renewcommand{\headrulewidth}{0mm}
}

\begin{document}

\author{Эмиль Хельгасон}
\title{Застывший}
\date{21.05.2017}
\maketitle

Он никогда не менял позу.
Кто-то посчитал бы его йогом или адептом другой кататонической практики, если бы поза походила на те, вычурные, из книжек.
Но нет, старик просто сидел.
В его позе не было ни патетической обречённости, свойственной бездомным, ни других эмоций.
Живая статуя без какого-либо философского посыла.

<<Рассказчик, перестань.
Зачем вам, плывущим в потоке жизни, необходим философский посыл?
Почему вы находите интересными совершенно одинаковые человеческие трагедии, но зеваете, глядя на никогда не повторяющийся рисунок неба?
Впрочем, неважно.
Сегодня этот рисунок совсем другой, и я не хочу ничего пропустить>>.

\spacing

Она была из тех женщин, которых в древности продавали и покупали.
Из тех женщин, которые безропотно ложились под победителя и всю оставшуюся жизнь рожали ему детей, накрывали на стол и штопали рубахи, не ропща и не задавая вопросов.
Она была воплощением выносливого безволия, которому жестокое общество дало свободу выбора, взамен потребовав индивидуальность и собственное мнение. 

Она ходила по этой улице каждый день.
Вначале на работу, противусолонь, затем посолонь к полоумной матери-алкоголичке, которой природа дала незаслуженно много жизненной силы.
Даже блики отражались в узких очках совершенно одинаково, кое-кто мог бы в этом поклясться.
Однажды её оскорбили слова сослуживца;
он сравнил её с застывшим в янтаре насекомым.

<<Он был прав, рассказчик.
В моём теле шевелятся молекулы, она шевелит ногами, а по сути мы оба стоим\dots>>

\ldotst Женщина отогнала от себя странные, как будто чужие мысли и отвернулась от старого бродяги, как делала это уже девять тысяч восемьсот семьдесят один раз.

\spacing

Старик снова сидел на том же месте.
Может быть, на меня посмотрит кто-то ещё\ldotsq

<<Посмотрит, Максимка.
Но как же это странно\ldotst >>

<<Потерпи, Рита.
Мы видим одними глазами, слышим одними ушами, чувствуем одной кожей.
Очень скоро наши мысли станут эхом друг друга, ведь мысль "--- результат чувств>>.

<<Посмотрит, Рита.
Потерпи, Максимка>>.

Кажется, это уже когда-то происходило.
Да, такое было.
Кажется, даже не один раз.
Старик уже не различал в мысленном голосе составляющие его голоса.

В России каждый день пропадают без вести триста человек.
Это написали в газете.
Вжух "--- страшная цифра свернулась в кулёк и исчезла под ароматным щёлкающим ручьём семечек.

"--*Двадцать, "--- сказала старуха, и газета окончательно исчезла в чьём-то кармане.

<<Сколько же нас здесь?>> --- изумлённо выдохнул слабый женский голос.
Или мужской.
Или детский.
Он на секунду задержался над океаном сознания, словно капля, готовая слиться с мириадами таких же.

Он никогда не менял позу.
Кто-то посчитал бы его йогом или адептом другой кататонической практики, если бы поза походила на те, вычурные, из книжек.
Но нет, старик просто сидел.
В его позе не было ни патетической обречённости, свойственной бездомным, ни других эмоций.
Живая статуя без какого-либо философского посыла.

<<Я собрал всех достойных рая и в свой час передам их следующему>>.

А что же такое рай, старик?

<<Жизнь, рассказчик, жизнь>>.

\end{document}
