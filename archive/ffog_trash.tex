\chapter*{Неразобранное}

\subsection{Кидалово}

\ml{$0$}
{--- Эти пидоры меня вообще без штанов оставили, ты понял, Варёныч?}
{``That faggots left me behind with my pants down, got it, Vary\'onych?''}

\ml{$0$}
{--- Ну я-то тут при чём?}
{``What's it got to do with me?''}

\ml{$0$}
{--- Так ты, блядь, мне их цифры скинул!}
{``It's you who wrote me the fucking numbers!''}

\ml{$0$}
{--- Откуда я знал-то?}
{``How would I know?}
Объява висела в подъезде у нас, я сфотал и всё, не звонил, не смотрел чё как...

--- Лады, слуш, --- Киря изо всех сил старался перейти на дружелюбный тон.
--- Скинь мне по-братски на карту.
На билет.
Даже не на весь, на половину.
Не половину, фиолет, круглая цифра, лады, наскребу как-нить остатки.
Бля буду, Варёныч, отдам, как приеду, всё отдам, тазик продам, и комнату в общаге материну продам, если надо.
Слово пацана.
Слово пацана, Варёныч, бля буду, всё отдам!

--- Нечего, Кирь, --- громко и отчётливо сказал Вареник.
\ml{$0$}
{--- Реально, блядь, нечего.}
{``I bloody can't.}
У меня жена в больнице, у неё обнаружили какую-то болезнь, лавэ ушли.

--- Тьфу, --- плюнул Киря.
--- Думал, ты ровный пацан, а ты...
Зажал фиолет и бабой отпизделся?

--- Да, --- без обиняков сказал Вареник и бросил трубку.

Киря позвонил ещё два раза, потом написал СМС с какими-то угрозами.
\ml{$0$}
{Вареник удалил его не читая и бросил Кирю в чёрный список.}
{\Varenik\ deleted it without reading and blacklisted K\'{\i}rya.}

\subsection{Оптимизм}

Пессимист --- это всегда человек среднего ума.
Оптимист же либо очень умён, либо очень глуп.

\subsection{Ум}

Он представлял собой самый нелюбимый работодателями типаж --- живой и изобретательный ум, который совершенно бесполезен в коммерческом плане.

--- Если ты такой умный, почему ты такой бедный? --- спрашивали его друзья.

--- Это к Богу, --- отвечал он.
Друзья смеялись и возвращались к своим иллюзиям.

\subsection{Грех}

--- Знаешь, во времена Гражданской был один такой комиссар, из обедневших дворян...
Прапрадед мой со стороны отца.
В тридцатые его из-за происхождения в расход пустили по накатанной, но делов натворить он успел.
Особенно любил духовенство троллить.
Поймает очередного попа, поставит у стенки и задаёт контрольный вопрос --- куда вошло копьё Лонгина Сотника?
И если ответит поп --- в сердце, комиссар ему тут же пулю в печень.
А всё потому, что читать Писание бездумно, не зная анатомии --- грех.

--- А если поп отвечал правильно?

--- В голову стрелял.
Дворянин всё-таки был, не пролетарий голожопый.

\subsection{Кошка и мышь}

--- Загадка: как найти чёрную кошку и чёрную мышь в тёмной комнате?

--- И как?

--- По хрусту.

--- Не понял.

--- Скоро поймёшь.


\subsection{Карьера}

Своей карьерой Вареник был обязан порезанному пальцу.
В тот знаменательный день он окончательно поругался с матерью, и та отказалась высылать ему деньги.
Вареник не особо расстроился.
Он расстроился бы, если бы у него были на это душевные силы --- поиск работы шёл не ахти как, деньги заканчивались, да и ссора с матерью не была первой в его жизни.
<<Заебала>>, --- вздохнул Вареник и пошёл мыться.

Палец резануло неожиданно.
Вареник всегда гордился своей реакцией;
падающая зубная щётка не успела коснуться пола.
Однако стеклянная полочка имела своё видение ситуации.
Палец залил кровью все поверхности, словно обезглавленный неумелым палачом преступник;
на срезе проступили сухожилия --- к счастью, целые.

Вареник собирался после душа поискать работу, но палец окончательно его доконал.
Он залил рану какой-то дрянью, попытался свести воедино края кожи и перемотал злосчастный палец бинтом;
затем, чтобы успокоиться, без особых надежд занялся старым проектом, который безуспешно пытался раскрутить последние два года.

Проект взлетел в тот же день.
Финансовые проблемы на время ушли в прошлое.


\subsection{4}

Лучше не стало.
Но стало легче.

\subsection{Жадность}

Нет, не прав был Иешуа Га-Ноцри.
Страшнейший из пороков --- отнюдь не трусость.
Жадность Иуды до денег, жадность Синедриона до власти отправили Иешуа на крест, а отнюдь не старый, сломленный службой, одиночеством и болезнью прокуратор.
Жадность --- вот самый страшный порок, и едва ли кто-то усомнится в этом, глядя на сегодняшнюю Россию --- безмерно богатую и бесконечно нищую, словно профессиональный попрошайка.

\subsection{Пытки}

--- Володьке с сердцем поплохело, он сейчас в реанимации, --- сказал Вареник.
--- Его в ментовке пытали, хотели выведать, что именно он вынюхивал в больнице.
Если это правда, он тебя не выдал.

\subsection{Вероника}

--- Я уволилась из <<Вектора>>, --- пробормотала Вероника.
--- Сказала, что устала и устроюсь в институте, поближе к дому и поспокойнее.

--- Они не знают?

--- Нет.

\subsection{6}

Если за вечную жизнь приходится платить тем, что сам ты сотворить не в состоянии --- это вечное рабство, а не вечная жизнь.
