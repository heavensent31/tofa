\documentclass[a4paper,12pt,fleqn]{book}\usepackage{polyglossia}\setdefaultlanguage[babelshorthands=true]{russian}\setotherlanguage{english}\defaultfontfeatures{Ligatures=TeX,Mapping=tex-text}\usepackage{xcolor}\newcommand{\ml}[3]{#2}

% \documentclass[a4paper,12pt,fleqn]{book}\usepackage{cooltooltips}\usepackage{polyglossia}\setdefaultlanguage[babelshorthands=true]{russian}\setotherlanguage{english}\defaultfontfeatures{Ligatures=TeX,Mapping=tex-text} \usepackage{xcolor}\definecolor{lightgray}{HTML}{bbbbbb}\color{lightgray}\newcommand{\ml}[3]{\textenglish{\textcolor{black}{#3}} }

% ----------------------

\usepackage{amsmath,amssymb,amsfonts,xltxtra,microtype,graphicx,textcomp}
\usepackage{svg}

% ------ GEOMETRY ------

\usepackage[twoside,left=2.5cm,right=3cm,top=3cm,bottom=4cm,bindingoffset=0cm]{geometry}

% ------ FONT ------

\usepackage{ebgaramond}
\definecolor{darkblue}{HTML}{003153}

% ------ HYPERLINKS ------

\usepackage{hyperref}
\hypersetup{colorlinks=true, linkcolor=darkblue, citecolor=darkblue, filecolor=darkblue, urlcolor=darkblue}

% ------ EPIGRAPH ------

\usepackage{epigraph}
\renewcommand{\epigraphsize}{\footnotesize}
\epigraphrule=0pt
\epigraphwidth=8cm

\usepackage{etoolbox}
\AtBeginEnvironment{quote}{\itshape}
\makeatletter
\newlength\episourceskip
\pretocmd{\@episource}{\em}{}{}
\apptocmd{\@episource}{\em}{}{}
\patchcmd{\epigraph}{\@episource{#1}\\}{\@episource{#1}\\[\episourceskip]}{}{}
\makeatother

% ------ METADATA ------

\newcommand{\tofaauthor}{\ml{$0$}{Эмиль~Весна}{Emil~Viesn\'{a}}}
\newcommand{\tofatitle}{\ml{$0$}{BSRK}{BSRK}}
\newcommand{\tofastarted}{11.03.2022}

% ------ FANCY PAGE STYLE ------

\usepackage{fancyhdr}
\pagestyle{fancy}
\fancyhead[LE,RO]{\thepage}
\fancyhead[LO]{{\small\textsc{\tofatitle}}}
\fancyhead[RE]{{\small\textsc{\tofaauthor}}}
\fancyfoot{}
\fancypagestyle{plain}
{\fancyhead{}
\renewcommand{\headrulewidth}{0mm}
\fancyfoot{}}

% ------ NEW COMMANDS ------

\newcommand{\asterism}{\vspace{1em}{\centering\Large\bfseries$\ast~\ast~\ast$\par}\vspace{1em}}
\newcommand{\textspace}{\vspace{1em}{\centering\Large\bfseries<...>\par}\vspace{1em}}
\newcommand{\FM}{\footnotemark}
\newcommand{\FL}[2]{\footnotetext{См. \textit{\hyperlink{#1}{#2}}.}}
\newcommand{\FA}[1]{\footnotetext{#1 \emph{\ml{$0$}{---~Прим.~авт.}{---~Author.}}}}

\newcommand{\theterm}[3]{\textbf{\hypertarget{#1}{#2}} --- #3}
\newcommand{\thesynonim}[3]{\textbf{#2} --- см. \textit{\hyperlink{#1}{#3}}.}
\newcommand{\theorigin}[3]{\textit{#1:} #2 --- #3}

\begin{document}
 
% ------ TITLE PAGE ------

\begin{titlepage}
{\centering{~\par}\vspace{0.25\textheight}
{\LARGE\tofaauthor}\par
\vspace{1.0cm}\rule{17em}{1pt}\par\vspace{0.3cm}
{\Huge\textsc{\tofatitle}\par}
\vspace{0.3cm}\rule{17em}{2pt}\par\vspace{1.0cm}
{\Large\textit{\ml{$0$}{Фэнтези-фанфик}{Fantasy fanfiction}}\par}
\vspace{0.5cm}\asterism\par\vspace{1.0cm}
{\textbf{\ml{$0$}{Начато:}{Started:}}~\tofastarted\par}\vfill
{\Large\ml{$0$}{Создано~в}{Created~by}~\XeLaTeX}\par}
\end{titlepage}

\tableofcontents

\chapter{BSRK}

Памяти Миура Кэнтаро

Это фанфик, посвящённый творчеству Миура Кэнтаро.
Он не претендует на серьёзное произведение и не будет использован для получения прибыли --- я отдаю его в общественное достояние (лицензия CC0).

\section{Пролог}

--- А вон та сучка беременна, --- ухмыльнулся Кварт, показав окровавленным клинком в сторону столба.
--- Девятый месяц.

--- Сейчас посмотрим, на что похож ублюдок, --- Донован поиграл мечом.
--- Я никогда не видел брюхатую шлюху изнутри.
А ты, Гамбино?

--- Гамбино, пожалуйста, --- тихо сказала Сисс.

Гамбино промолчал.
Донован воспринял это как согласие и одним взмахом меча распорол беременной живот.

Сисс, едва успев увидеть то, что вывалилось из трупа, упала на колени.
Её вырвало.

--- Глаз-алмаз, Донован, --- сказал Кварт под смех наёмников.

Ребёнок повис на пуповине.
Она сжала его шею, как верёвка сжала шею его матери.
Сисс с рыданиями подбежала к нему и подхватила на руки, пачкаясь в крови, моче и плодной жидкости.

--- Всё в порядке, маленький, --- Сисс освободила шею младенца от пуповины и вгрызлась в неё зубами.

--- Сисс, пошли, --- сказал Гамбино.
--- Он сдохнет.
Ворон его знает, сколько она висела на столбе.

--- Всё в порядке, маленький, --- Сисс баюкала безмолвного младенца.
--- Мама рядом.

Гамбино, вздохнув, подошёл к Сисс, приподнял младенца за ножку и ударил жёсткой, как подошва, ладонью по ягодице.
И младенец закричал.

Всех поразил этот крик.
Он не был похож на детский.
Младенец не плакал, а ревел, как дикий зверь, во всю мощь своих крохотных лёгких.

--- Жить будет, --- констатировал Гамбино.

--- Дай ему имя, дорогой, --- тихо сказала Сисс.

--- Пусть будет тем, кто он есть --- потрохами, --- сказал Гамбино, морщась от рёва младенца.
--- Его имя --- Гатс.

--- Всё хорошо, Гатс, --- зашептала Сисс, успокаиваивая малыша.
--- Мамочка рядом.
Всё хорошо.

--- Иди в хвосте, --- сказал Гамбино, хлопнув Сисс по заднице.
--- От тебя воняет как от скотобойни.
Как разобьём лагерь, помоешься и помоешь ублюдка.

Сисс не ответила.
Она продолжала влюблённо смотреть на ребёнка.

--- Всё хорошо, Гатс, --- говорила она.
--- Мамочка рядом.
Всё хорошо...

\section{Чёрный мечник}

Отдохнуть в этот раз не вышло.
Рядом всё ещё дымился труп обнажённой молодой женщины с развороченным черепом.
Кролик был съеден, прихваченная бутылка разбавленного вина выпита, но острая боль в паху не проходила.
Подумав, Гатс встал и отправился в лес.

Коростель-трава, барсучья смерть, чернобыльник...
Гатс опустошил котелок и наполнил его не очень свежим козлиным жиром --- остатком пиршества одичавших собак.
Мазь должна была покипеть ровно сто три удара сердца.
Затем быстренько охладить котелок, сунув в текущий голубоватый ручей и, шипя, наложить жгучую как огонь грязноватую субстанцию на член и мошонку.

<<Даже смазка Апостола опасна, --- угрюмо думал Гатс, перебинтовывая хозяйство.
--- Будем знать>>.

Остатки мази Гатс сцедил в треснувшую бутылочку и положил в дорожный мешок.
Последний год бутылочки в мешке множились бесконтрольно, и это тревожило Гатса.
Его не волновали раны, синяки, ссадины.
Гораздо больше силы отнимала ноющая боль в суставах, гнойники и сыпь под пластинами доспехов, постоянно воспалённые пальцы ног и жжение в изуродованном глазу.

<<Я не Апостол.
Моя жизнь конечна.
Именно поэтому у меня нет времени на раздумья>>.

Боль постепенно стихала, и Гатс нашёл в себе силы припомнить события часовой давности.

--- Ещё!
Ещё! --- стонала красотка, невесть откуда появившаяся у привала Гатса.

--- Попался! --- рыкнула Апостол, впиваясь внезапно удлинившимися ногтями в его спину.
--- Попробовав рая, отведай ад!

Вместо ответа Гатс засунул ей в зубастую пасть металлическую руку.

Бум.

Гатс еще раз осмотрел труп.
В его взгляде не было ни ненависти, ни сожаления.
Апостолы после смерти превращаются в людей, которыми были, чтобы их сущность не была раскрыта.
Так говорили маги.
На самом деле, как подозревал Гатс, причина была более прозаичной --- Идея Зла просто забирала назад силы, которые отдала в долг своему слуге.
В конце концов, Апостолы бродят в истинном обличье по городам и весям, о какой вообще скрытности может идти речь?

<<Темнеет.
До замка ещё два часа ходьбы>>.

\asterism

Д\`{у}хи, подобно дух\`{а}м, имеют запах.
Это кажется невероятным для любого, кто ни разу не попадал в Расщелину.
Гатс знал, как пахнут голодные тени.
Воздух начинал отдавать плесенью и гнилью --- запах заполненного трупами погреба.

Лёгкий запах гнили настиг его у ворот города.
Заря ещё пылала в стеклянно-чистом, похожем на шар предсказателя небе.
На горизонте ни облачка.
Ночь обещала быть тёплой, лунной и звёздной --- в такую комфортно дежать оборону.
Правда, не в городе.

--- Пода-ааайте на хлебушек, добрый господин, --- унылым механическим голосом прогнусил попрошайка, сжимая чиненный-перечиненный костыль.
--- На хле-ееебушек...
Ну и пёс с тобой, жлоб.
Чтоб ты сдох.

Мимо проехала телега с клеткой, в которой сидели подростки.
Телегу сопровождали несколько всадников с пышными плюмажами и напомаженными усами --- явно декоративных.
Сумрачным, потухшим взглядом посмотрел на Гатса красивый молодой парень.
Его карие глаза, вздёрнутый веснушчатый нос и прямые пшеничные волосы вызывали опасные непрошенные воспоминания.
Парня за руку держала девушка --- смуглая, черноглазая, коротко стриженная...

Гатс отвернулся, поправил капюшон и попытался выкинуть морок из головы.

\asterism

--- Сколько стоишь, красавица? --- вопрошал вдрызг пьяный стражник.
Он был при полном параде --- плащ, латы, начищенный шлем.
Правда, только выше пояса.
Ниже полный парад годился разве что в баню.

--- Как и все --- час десять, ночь тридцать, --- кокетливо отвечала красавица, кутаясь в шерстяную шаль.

<<Приличная таверна, --- отметил Гатс.
--- Может, здесь и...
Впрочем, похоже, я сглазил>>.

--- Да не вертись ты! --- орал лысый наёмник, распространяя крепкий запах пива и сожжённого луком желудка.
В его не очень верной руке вращался метательный нож.
Его товарищи вопили и улюлюкали, подначивая метателя.

Нож свистнул и вонзился в дощатую стену таверны.
Цель --- а целью оказался привязанный за шею истинный эльф --- упорхнула и показала лысому язык.

--- Прома-аазал! --- захохотали наёмники.

--- Ставлю ещё четыре, что и в третий раз мимо!

--- Десять!
Ставлю сегодняшний паёк шлюшатины, что Игор попадёт!

--- А Дин ставит?

--- Он отлить пошёл.
Ставлю два от его имени, ты знаешь на что!

--- Подождите, парни, я записываю.
Грин, бабки, бабки на бочку.
Никаких ставок под честное слово...
Сорес, сраный недоносок, ещё раз стащишь монету --- выбью тебе глаз вместе с мозгами.

Местная публика явно была от развлечения не в восторге.
Служанки кутались в шали и натягивали чепцы на глаза, несмотря на духоту, жались по углам, выпивку разносили быстро и не поднимая головы.
Они знали, что уже участвуют в ставках, и знали, что стоит на кону: как только эльф исчерпает свою занятность, возьмутся за них.
Мужчины тоже болели за то, чтобы эльф продержался как можно дольше, но уже по другим причинам.

--- Недолго будет у нас спокойный разговор, чувствую, --- пробормотал стриженный под горшок меняла своему товарищу.
Тот нервно оглядывался, проверяя, жив ли эльф.

--- Твоя правда, Гер.
Давай побыстрее.
С этими наёмниками Кока не связывается даже мэр, нам тем паче не с руки...

Оба вздрогнули и проводили Гатса взглядом, когда он прошёл мимо.

\asterism

--- Заказывать будем?

--- Бутыль воды, --- Гатс положил на прилавок семь серебряных монет.

--- Тут на целую бочку, --- спокойно прогудел трактирщик из-под усов.
--- Вода у нас не золотая, забери лишнее.

--- Остальное на ремонт, --- Гатс один за другим вложил восемь болтов в массивный чжугэну на запястье.
Трактирщик выпучил глаза.

--- Ты чего...

--- ...сейчас я заткну твой поганый маленький рот, крылатая крыса! --- орал Игор.
Его тираду прервал стальной болт, пробив висок и выйдя из глазницы с другой стороны.

На таверну пала тишина.

--- Кто это, раздрочи меня Господь, сделал, сука? --- один из наёмников начал подниматься, вытаскивая меч.
В следующий миг чжугэну снова защёлкал.

Гатс снял задвижку, освободил тетиву и закрыл магазин арбалета.
Использованные болты отправились в <<грязную>> сумку на левом бедре --- один из них погнулся от удара о шлем и требовал правки, все без исключения следовало отмыть от крови, мозгов и соплей.
Наёмники так и лежали, раскрыв невидящие глаза в ужасе и удивлении.
Местные в страхе спрятались за столами и притихли.
За столом подвывал последний наёмник --- худой, с немытыми чёрными волосами писарь.
Пергамент со ставками так и лежал перед ним, залитый кровью.
Из носа у писаря торчал болт.

Гатс схватился за болт рукой.
Писарь взвыл и запыхтел, вытаращив глаза и высунув обмётанный жёлтым язык.

--- Ты один из людей замка Кока? --- спокойно спросил Гатс.

Писарь задышал и заскулил чуть громче.
Гатс пошевелил болт, вызвав новый поток визга и стенаний.

--- Спрашиваю ещё раз.

--- Ва, ва, --- торопливо ответил писарь.

Гатс лениво сбил металлическоим пальцем блоху, перебежавшую с сальных чёрных волос на его руку.
Затем пошевелил пергамент со ставками.

--- Вижу, что человек грамотный.
Писать-читать умеешь, значит, и памятью не обижен.
Передай своему хозяину: пришёл Чёрный Мечник.

--- Фофоифо? --- дрожащим голосом поинтересовался писарь.

--- Больше ничего.

--- Сзади! --- завопил тонким голоском эльф.

Гатс схватил рукоять притороченного к спине меча --- и таверна взорвалась криками ужаса.
Кровь заплескала потолок, пол, столы, размозжённые кишки мягко шлёпались на рассохшиеся деревянные поверхности и тут же пристывали к ним.
Разрубленный напополам наёмник клацнул челюстями и замер.

Язык не поворачивался назвать эту громадину мечом.
Чересчур тяжёлый, чересчур тупой, похожий на необработанную стальную плиту.
По поверхности шли жилы ржавчины и трещины от длительного использования.
Гатс взмахом отряхнул меч и убрал его за спину.

--- Рассчитываю на тебя, --- кивнул он писарю.
Тот всё ещё лежал щекой на столе, боясь пошевелиться.

--- Эй! --- крикнул эльф.
Тонкие крылышки в отчаянии зажужжали.
--- Освободи меня!

Но Гатс уже был в дверях таверны.
Его плащ в последний раз колыхнулся и слился с вечерними сумерками.

\asterism

--- Гады, --- эльф присел на скрипучую вывеску таверны и принялся выковыривать из зубов пеньковые волокна.
--- Хоть бы один помог...

Всё, чего обычно желает невольник --- это свобода.
Свободой он может жить годами и десятилетиями, грезя во сне и наяву, подчиняя своей мечте весь жизненный уклад, насколько это позволяет неволя.
Но когда свобода всё-таки приходит, возникает закономерный вопрос --- а что дальше?
Для большинства бывших невольников он так и остаётся нерешённым.

К счастью, эльф был в плену не очень долго по меркам своего народа.
Кроме того, ему весьма помогала квазиполиморфность ментального тела, делающая некоторых существ практически неуязвимыми к душевным травмам --- способность сколь полезная, столь и редкая.
Глупые люди, так и не разобравшись, как это работает, спутали это явление с гиперактивным расстройством и занесли в список заболеваний.

Клацанье доспехов всё дальше уходило в сумерки.
В кроне вяза тихо пел одинокий соловей, словно о чём-то спрашивая каждые полминуты.
Эльф быстренько прикинул --- чужой город, бродячий театр давно уехал невозвратной тропой, сородичей и даже просто сочувствующих нет на сотни миль вокруг, крылья не железные.
А гигант выглядел если не добрым, то хотя бы надёжным и...

--- Эй, подожди!
Подожди меня!

\asterism

--- Начал спасать, так доводи дело до конца! --- пищал эльф в ухо Чёрного Мечника.
Ухо на пищание никак не реагировало.

--- Ладно, я погорячился, --- эльф откашлялся.
--- Здравствуй, я Пак.
Рад знакомству.
Я путешествовал с труппой артистов, но эти бандиты напали на нас и заперли меня в птичью клетку.
Обливали выпивкой, крылья угрожали поджечь!

Ответа не последовало.
Пак прикинул, насколько невежливо будет на данной стадии знакомства садиться на плечо.
Вывод очень опечалил его уставшие крылья.

--- Красивый меч, --- попытался он ещё раз.
--- Наверное, тяжёлый.

Ноль внимания.

Пак облетел Гатса и задумчиво оглядел меч.
Чересчур тяжёлый, чересчур тупой, похожий на необработанную стальную плиту.
Таким невозможно размахивать и тем более разрезать кого-то на ровные половинки.
Однако в свете первых зажигающихся звёзд Пак ясно видел идеальные призрачные грани и острую как бритва кромку.

<<Меч на самом деле достаточно лёгкий, не тяжелее обычного двуручного меча, и очень острый.
Но свои свойства он должен проявлять только в Расщелине.
Как же он?..
А, вон оно что>>.

Знак на шее.
Таинственная руническая лигатура горела желтоватым огнём и трепетала, словно абстрактная, но вполне живая оса.
Подобие младенческой пуповины стягивало шею Гатса, связывая его физическое тело с миром духов.

Обычные методы знакомства --- рассказ о себе и лесть --- не сработали, и Пак решил перейти к практической стороне вопроса.

--- Слушай, тебе лучше бежать отсюда, --- проникновенно начал он.
--- Эти наёмники много раз нападали и на город, несмотря на то, что мэр платит им дань.
Тебя покромсают на ломтики, если ты попадёшься.
И покромсают не только наёмники, но и стража, потому что они все повязаны...
Ай!
Ты мне чуть крылья не переломал!
Ты вообще представляешь, каково эльфу получать шлепки такой огромной рукой?
А давай я тебя телегой ударю?

--- Лети отсюда, --- лаконично посоветовал Гатс.
--- В следующий раз прихлопну насмерть.

Из тёмного узкого проулка медленно вышли закованные в железо стражники.
В грудь Гатсу упёрлись копья, заскрипели взводимые арбалеты.

--- Я пытался, --- развёл руками Пак и шмыгнул под ближайшую крышу.

\asterism

<<Самый лучший способ проникнуть в крепость --- это проникнуть в крепость в качестве пленника.
Проблема в том, что тысяча и одна вещь могут пойти не по плану>>.

Хлысь.
Тяжёлый бич сорвал ещё лоскуток кожи с груди Гатса.

--- Молчит, --- констатировал палач.
--- Молчит, зараза.
Что бы ему эдакое задеть, чтобы подал голос?
Э, глыба!
Может, тебе соски вырезать?
Или очко прочистить горяченькой кочергой?
За что голосуем?

--- Решай сам, --- Гатс пожал плечами, насколько позволяли цепи.
--- Солнце уже почти зашло, час-два --- и полночь наступит.
Твои заплечные дела покажутся детской игрой по сравнению с тем, что принесёт сюда полночь.

--- А ну-ка повтори! --- угрожающе рыкнул палач.
Половины сказанного он не понял, поэтому выбрал универсальную тактику ведения беседы.
Но умудрённый опытом Гатс больше не стал тратить время на беседу и просто смачно плюнул в лицо палача смесью загустевшей слюны, крови и кусочков сегодняшнего кролика.

--- Ах ты!..

--- Хватит, --- прервал его дребезжащий старческий голос.

--- Господин мэр, --- палач забормотал и согнулся в неуклюжем поклоне, насколько позволял круглый как шарик живот.

Мэр нервно оглянулся на сопровождавших его стражников и подошёл к столу.

--- Его оружие?

--- Так точно, господин мэр, всё до единого его самого.
Чжугэну, тэцубиси, метательные ножи, яды, три вида арбалетных болтов, и вот это...
Что это вообще, меч?
Как не помер под таким весом ублюдок.
И на какого зверя только с таким арсеналом ходят, м?

Мэр снова оглянулся на стражу и подошёл ближе к Гатсу.

--- Ты наёмник?

Гатс промолчал.

--- Ты вообще понимаешь, что случится из-за твоей выходки? --- дрожащим голосом продолжал мэр.
--- Мы долгие месяцы обустраивали нашу жизнь в соответствии с новым... правлением.
А сейчас...
Сейчас может статься, что город просто уничтожат.
Изведут под корень всех до последнего жителя.
Детишек, женщин, стариков.
И всё из-за того, что ты убил людей Кока!

--- Надо же, про детишек вспомнил, --- хмыкнул Гатс.
--- А что за пугало за тобой?

--- Что ты сказал? --- рыкнул один из стражников, подняв копьё.

--- Стой!
Стой! --- мэр замахал руками.
--- Пожалуйста, остановись.
Я пытаюсь послужить лорду как можно лучше!

Мэр подошёл ещё ближе к пленнику.

--- Эти люди опасны, --- сказал он.
--- Их хозяин опасен.
Ходят слухи... --- мэр снова оглянулся, --- ...истинная правда, что его нельзя убить.
Он не человек, он нечто более страшное, неизведанное.
Создание самой преисподней!

--- Если хочешь продать меня Кока --- продавай, --- перебил Гатс.
--- И так понятно, что ты с ним в сделке.
К чему всё это нытьё про детишек, стариков и преисподнюю?
В добряка решил поиграть?

--- Да что ты знаешь! --- прошипел мэр.

--- Я знаю, --- оскалился Гатс.
--- Я знаю больше, чем ты, чем громилы за твоей спиной.
Если вкратце --- он жрёт людей.
А ты, народный избранник, поставляешь ему корм.

Мэр выхватил из рук у палача бич.
Костяшки старческий рук побелели.

--- Кстати, о детишках и женщинах, --- продолжал Гатс.
--- Я видел повозку у ворот города не далее как час назад.

--- Я сделаю всё, чтобы защитить мой город! --- рявкнул мэр.

--- И свою шкуру, --- присовокупил Гатс.

--- Ах ты... --- мэр захлебнулся, схватившись за сердце, и упал на колени.
Бич выпал из ослабевших пальцев на древний каменный пол.

--- Господин мэр!
Господин мэр! --- палач скакал вокруг как кролик.
--- Али поплохело вам?
Сердечко схватило, господин мэр?

--- Сдери с него всё, что допустимо по пятому приказу, --- хрипло и зло ответил мэр.
--- Не калечить.
Этого наёмника надо доставить в замок в надлежащем состоянии.
Сквайр, приготовьте экипаж.
Я сам поеду в замок Кока и попробую вымолить прощение для города...

\asterism

Глаза сложно было назвать глазами.
Ни зрачков, ни глазных яблок в привычном понимании там не было.
Глазницы мужчины были заполнены двумя шарообразными сгустками красного тумана.
Эти сгустки темнели, светлели, переливались, следуя мельчайшим изменениям настроения их хозяина.
Стоящий на одном колене писарь был осведомлён об этом свойстве --- он украдкой наблюдал за глазами сюзерена, пытаясь прочитать в них свой приговор.

--- Чёрный Мечник, значит?

--- Да, милорд, --- писарь изо всех сил пытался говорить не в нос.

--- Как выглядит?

--- Был одет во всё чёрное, милорд.

--- Звучит логично.
Поэтому он и Чёрный Мечник.
Не Зелёный, не Синий.
Важные детали, Освальд.
Важные.

--- Рука железная, милорд, в ней самозаряд с пружинкой.
Им он наших и порешил в один миг.

--- Он убил шесть --- шесть! --- латников с помощью пружинного чжугэну?

--- Я бы сам в такое не поверил, милорд.
Но я сам видел, милорд.
Он стрелял очень быстро и точно.
Шесть стрел, шесть парней, одна Игору, последняя мне.
Дин только отлить пошёл, и...

--- Меня не интересует, сколько раз Дин посещал уборную.
Что ещё?

--- Меч, милорд.
Большой, толстый, как дверь.
Меч ржавый, тупой как хер конский, однако ж он им Дина напополам разрубил, как поросёнка...

--- Достаточно, --- хозяин махнул рукой.
--- Десерт.

Пожилая служанка аккуратно подняла тарелку с объедками и приборы.
Приборы он держала, прихватив салфеткой за самый кончик, словно боясь ненароком коснуться грязной части.
Судя по всему, страх был обоснованным --- на нескольких пальцах у служанки красовались бородавки и странные ожоги, как от кислоты.

Одетый с иголочки слуга принёс серебряное блюдо с крышкой и с поклоном поставил его перед хозяином.
Ещё одна служанка расстелила затейливо сложенную салфетку и выложила на неё приборы в строгой очерёдности.

\asterism

--- Меня не интересует золото и людские жертвы, господин мэр, --- прервал его хозяин.
--- У нас с вами был договор.
Вы обеспечиваете порядок, я обеспечиваю спокойствие.
Скажите, мэр, вы обеспечили порядок?

--- Я-я пытался, милорд, наша стража...

--- Ваша стража не справилась.
То есть договор можно считать аннулированным.
Ваши извинения мне не интересны.

Хозяин щёлкнул пальцами.
Мэра тут же подхватили под руки и повели из зала.

--- Что всё это значит?!
Милорд!
Милорд, умоляю вас!..

Хозяин встал из-за стола и подошёл к окну.
Его острые зубы сверкнули в последних лучах зари.

--- Как будто мне нужен был повод, --- усмехнулся он.
--- Я просто хочу увидеть вселенский пожар.
Я хочу видеть, как людишки метаются в огне.
Я хочу слышать, как ломаются их кости под копытами моего коня.
Это лучшее, что мне приходилось слышать.
Это то, ради чего стоит жить.
Но люди любят простые и понятные им правила.
Веди себя хорошо --- и ты будешь вознаграждён.
Будешь вести себя плохо --- тебя ждёт наказание.
Извинись --- и будешь прощён.

Хозяин захохотал.
Его хохот метался по зале, точно испуганная птица, ломая крылья о стены и застревая в изящной старинной лепнине.

--- В этом городе лишь один закон --- моё желание, --- хозяин сощурил алые туманные глаза.
--- И сейчас ты в этом убедишься, Чёрный Мечник.

\section{Превращение}

Гатс уже научился по внешним признакам определять историю Апостола.
Чем больше Апостол во плоти похож на человека --- тем он сильнее.
Особенно сильны те, чьи глаза лишены обычного для Апостолов красного цвета.
Размытые границы тела, туманные глаза выдают психопатов-убийц, совершенно лишенных моральных принципов.
Несмотря на то, что такие уступают в бою другим апостолам, они крайне коварны и выжимают из своих не очень впечатляющих сил максимум эффекта.

Едва увидев алые туманные глаза и дворянскую стать, Гатс понял --- бой будет не из лёгких.

\section{Плата}

Донован пнул Гатса в бок.

--- Твоя оплата, --- на землю упали три серебряных монеты.
--- Гамбино свою долю получил, из щедрости добавлю и тебе.
Твой первый заработок.

Гатс подобрал монеты.
Он долго смотрел на них.
Потемневший, со следами зубов мидландский талер, квадратная <<мотыга>> из Кушана, с которой стесали углы фальшивомонетчики, недавно отчеканенный сребреник с профилем какого-то толстяка...
Гатс с хрустом сжал их в кулаке.
По запястью потекла тоненькая струйка крови.

\section{Месть}

Гатс смотрел на насильника сверху вниз и понимал, что мир больше никогда не будет прежним.
Он не мог вернуть того себя, для которого задница была если не святилищем, то хотя бы достойным уважения местом.
Всё, что он мог --- отнять жизнь у того, кто грубо вломился в его тело, чтобы тот больше не мог этого сделать никогда.

Гатс вытащил из мешочка заветные монеты.
Одна, две, три...
Квадратная из Кушана, старый мидландский талер, новенький самодел какого-то новоявленного диктатора, решившего вписать себя в историю.
Одну за другой Гатс бросил монеты Доновану в раскрытый рот.

--- Твоя оплата.

\section{Дуэль}

<<Сейчас я покажу тебе, как действительно используется рот в бою>>, --- подумал Гатс.
Он зубами сорвал с наплечника стёганую накладку и, молниеносно сжевав её, захватил зубами клинок противника.

<<Сумасшедший, что ли?..>> --- Гатс всё больше заинтриговывал Гриффита.
Клинок тот схватил основательно, укус у него медвежий, ещё и эта тряпка...
Но медленно, по пшеничному зерну, шпага всё-таки продвигалась к горлу.

<<Если продвину дальше --- он умрёт.
Если попытаюсь выдернуть --- выбью ему зубы или сильно рассеку щёку.
Люди умирают и от меньшего на поле боя, если не остановить кровь.
Поддаться?
А, это не нужно.
Его хватки вполне достаточно, чтобы...>>

В следующий миг из глаз у Гриффита полетели искры.

\section{Бехелит}

Странный амулет, не похожий ни на что --- словно творение авангардного художника.

--- У того, кто это сделал, точно было не всё в порядке с головой, --- резюмировал Гатс.
--- Как будто...
Ах ты!..

Гриффит едва успел подхватить друга за руку и втащить обратно на лестницу.

--- Ты чего?

Гатс нервно сглотнул, пристально разглядывая бехелит.
На секунду ему почудилось, что один из разбросанных по красному яйцу глаз открылся и пронзил его острым, как швейная игла, взглядом.

--- Ну и дрянь ты с собой носишь, --- буркнул он и всучил амулет Гриффиту.
Тот ухмыльнулся.

\section{Шарлотта}

Гриффит чувствовал себя изнасилованным --- с той лишь разницей, что изнасиловал он сам себя, посредством этой девчонки.

\section{Древние руины}

--- Я вырвал его вот этими клещами, --- продолжал глумиться палач.
--- Это произведение искусства --- ни один язык ещё от них не убе...

Тираду прервал грохот петель и треск дерева.
Дверь вылетела, придавив палача.
Гатс и Пиппин молча стряхивали с одежды щепки.

--- Дверь чересчур тонкая, --- сообщил Гатс палачу.
--- Может, она и годится для искалеченного Гриффита, но не для трёх крепких наёмников.
Дай-ка эти клещи.
Говоришь, произведение искусства?

Гатс отвалил тяжёлую дверь и поднял палача за ворот.

--- Пожалуйста, не делай этого! --- взмолился палач.

--- Почему? --- спросил Гатс.
--- Почему я не должен этого делать?

--- Ну, это... ммм... я не знаю...

--- Ну вот и всё, --- Гатс зажал клещами язык палача и одним движением вырвал его.
Палач издал булькающий стон и упал ничком.

Гатс подошёл к Каске.
Та задумчиво стояла, скрестив руки на груди, и смотрела вниз, в темноту гигантского колодца.

--- Как думаешь, здесь действительно есть развалины древнего города? --- отсутствующим тоном спросила она.
--- Того, о котором говорила леди Шарлотта.

--- Сейчас посмотрим, --- ответил Гатс.
Он схватил за шкирку захлёбывающегося кровью палача, облил его горящим маслом из лампы и столкнул пылающее тело в пропасть.

Палач летел недолго.
Его тело шмякнулось о каменный пол, разбрызгав капли огня.
Каска ахнула, увидев осветившиеся изящные статуи и колонны, скрывавшиеся на дне колодца.

--- Гатс, смотри!

--- Видимо, насчёт города всё правда, --- буркнул Гатс.
--- Пиппин, бери Гриффита и уходим.
Больше нам здесь делать нечего.

\section{Разговор}

--- Гатс, насчет вчерашнего.

--- Прости меня.

--- Не надо.
У тебя нервы сдали, у меня сдали.
Ты меня ударил, я на тебя накричала...

--- Ничего нового.

--- В этом всё и дело.
Я не хочу, чтобы у нас было всё по-старому.

--- А как ты хочешь?

--- Не знаю.
Мне пока что не из чего выбирать, нечему следовать.
Но для начала, если ты захочешь поговорить со мной --- давай поговорим.

--- О чём?

--- О чём хочешь.

\section{Деревенская благодарность}

--- А хаты ты нам починишь? --- раздался голос. 
Все замолчали и посмотрели на Ширке.

--- Что? --- растерянно спросила она.

--- Деточка, ты посмотри, что с домами нашими, --- терпеливо сказала женщина.
--- Мусор мы и сами-то уберём, ты хаты нам почини.

--- Ты же ведьма, тебе раз плюнуть!

--- А церковь? --- поддержал священник. --- Шестьсот лет стояла, и смотри что теперь.
У нас мужиков не хватит её починить.

--- Сделай милость, госпожа ведьма!
Почини хаты!

--- Но я... не могу... --- пискнула Ширке.

\asterism

--- Как они на меня смотрели, --- всхлипнула Ширке.
--- Словно это я троллей напустила.

--- Обычное дело, --- ответил Гатс.
--- На будущее, если ты еще хочешь делать добро: люди за него не платят ничем, кроме того, что было оговорено в деньгах.
Если хочешь делать добро бескорыстно --- платой должно быть осознание, что ты поступаешь правильно.
Ничего другого ты не получишь.
По крайней мере, хорошего.

\section{При смерти}

--- ...Госпожа Фарнезе, мы не можем снять с него доспехи!

--- Его нельзя оставлять так, Серпико!

--- У нас нет выбора.
Позаботьтесь о госпоже Каске.
Пак, займись поверхностными ранами.
Там, где у доспехов шипы, есть перелом.

--- Я не смогу пролезть туда, Серпико!
Крыльям нужно пространство!
Сделай что-нибудь!

--- Я расстегну ремни.
Госпожа Ширке, подсветите, пожалуйста...
Благодарю.
А, чёрт, ремни присохли из-за крови...
Нагрейте воды!
Госпожа Ширке, нужны ваши способности.
Посмотрите, что у него повреждено.

--- Я не врач, Серпико.
Я не умею.

--- Если вы не сможете, не сможет никто.
Пожалуйста, сосредоточьтесь.

Ширке вдохнула.

--- У него внутри кровь.
Я чувствую, как она тёчет вот отсюда, слева.
Медленно, как ручеёк, но не останавливается.

Серпико крякнул.

--- Скорее всего, разрыв селезёнки.
Я видел, как солдат от такого умер за десять минут...
Пак!
Брось поверхностные раны, тут дело посерьёзнее!

--- С селезёнкой я ничего не сделаю!
И пыльца у меня не бесконечная!

--- Нужно добавить пыльцу в кровь.

--- Я никогда этого не делал.
Я что, похож на Амбруаза Паре?
А если он умрёт?

--- Он в любом случае не жилец.
Если мы хотим дать ему шанс, нам всем придётся делать то, что мы никогда не делали.
Держите его вот так...

\section{Нечто}

Ширке часами смотрела на Каску.

--- Что ты видишь? --- спросил Гатс.

--- Ничего определенного, --- уклончиво сказала Ширке.
--- Наставница сказала, что госпоже Каске можно помочь, и я хочу понять, что она имела в виду.

--- Хочешь превзойти наставницу.

--- Нет, что ты! --- в благоговейном ужасе сказала Ширке.

--- Хочешь закончить то, что она начала, --- поправился Гатс.
--- По сути то же самое.

--- Нет, это другое!

--- Как скажешь.
Не нервничай и займись делом.
Это сможешь только ты.
Просто продолжай.

\section{Корочка}

Эти люди --- как засохшая корочка на ранке: не тревожат, легко отходят и исчезают, оставляя после себя чистую розовую кожу --- место для новых ран.

\section{Затмение}

Все повторялось как в страшном сне.
Телега катилась по кочкам, и каждый скачок взрывался в голове Гатса приступом боли.
Но вот колесо подвернулось, и Гатса выбросило наружу, прямо на камни.

<<Смертные не могут изменить ход празднования>>.

Гатс лежал и отплёвывался от крови.
Она затекала в горло, и каждый миг грозил стать последним.
Гатс слабо откашливался, и это единственное, что он мог сделать.

Вдруг у его уха что-то глухо зажужжало.
Гатс скорее понял, чем увидел, что это бехелит.
Разумеется, сумка расстегнулась, и каменная слеза выкатилась наружу.
Кровь, безнадёжность, бехелит --- всё это уже видели, всё это уже знакомо...

--- Гатс! --- Каска уже неслась к нему со всех ног.
Почему она бежит так безоглядно?
Она что, не прочитала знаки?
Неужели она не видит очевидное?..
Она единственная, кто может их прочитать, кто может остановить прочих.

--- Гатс, всё хорошо, --- Каска вывернула ему голову и очистила рот платком.
--- Всё хорошо.

--- Бехелит...

--- Я знаю.
Я уже сообщила остальным.

--- Что?!

--- Они готовы рискнуть.
Чего мы, по-твоему, ждали все эти дни?

Гатс посмотрел на товарищей.
Их лица были суровы.
Ширке успела обвешаться каким-то ведьминским прибором с головы до ног.
Серпико и Фарнезе надели изящные доспехи с гербом дома Валленштайн.
Исидоро тоже надел шлем, чем-то похожий на старый, который носил в молодости сам Гатс.
Эльфы предусмотрительно попрятались под доспехи товарищей, чтобы их ненароком не раздавили.

--- Гатс, --- тихо сказала Каска.
--- Мы один раз это уже пережили.
В этот раз сюрпризов не будет, нас не застанут беспомощными и ничего не понимающими.
Давай, заверши начатое.

Каска поднесла вибрирующий бехелит к губам Гатса, и тот кашлянул, выплюнув на камень ещё горсть крови.
Глаза на поверхности бехелита открылись.

--- Приготовиться к бою, --- спокойно сказала Каска.
Прочие молча встали в круг вокруг распростёртого на камнях Гатса.
Небеса вокруг кровавой луны медленно начали темнеть.

\section{Жертва}

--- Я приношу в жертву Длань Господа.

Под чёрной луной воцарилось молчание.
Возможно, с самого начала времён не было так тихо.
Если прислушаться, можно было услышать тихое биение сердца Идеи Зла.

Вдруг Слан захохотала.
Ей вторил Убик.
Фемто с улыбкой покачал головой, и только Войд остался недвижим.

--- Гатс, ты не изменился, --- сказал Фемто.
--- Всегда идёшь ва-банк.

--- Мальчишка, ты не можешь принести нас в жертву, --- сказал Убик.
--- Мы --- Длань Господа.
Мы --- ближайшие слуги Господа.
Мы исполняем волю Господа.

--- Идея Зла бесстрастна, --- сказала Ширке.
--- Она не делает различий между слугами и их жертвами.
Вы можете быть жертвой, как и любой из нас.

--- В жертву можно принести лишь самое дорогое! --- сказал Фемто.

--- Я и приношу в жертву самое дорогое, --- ответил Гатс.
--- Тебя, моего первого друга.
Прочих членов Длани Господа, ради встречи с которыми я жил все эти годы после Затмения.
Вас всех --- существ, которые придавали мне сил.

--- Лишь мне дарована тайна нанесения Клейма Жертвы, --- прогудел Войд.

--- Я обращаюсь не к тебе, Войд.
Я обращаюсь к твоему дарителю.
Идея Зла!
Господь, владыка этого мира!

Голос Гатса отозвался эхом под бесконечными красными небесами.

--- Мой выбор --- Длань Господа!
Я приношу в жертву Длань Господа!

И снова молчание.
Слан снова захохотала.

--- Кажется, тебя никто не слышит, маль...

Вдруг Слан вскрикнула от боли.
На её груди медленно, дымясь и тлея, проступил знакомый знак.

--- Не может быть! --- взвизгнул в ужасе Убик.
Вспышка --- и на его лбу тоже отпечаталось Клеймо.

--- Ты за это ответишь, --- спокойно сказал Войд.
Печать выжглась прямо на его лице --- на изуродованном носу, оскаленных зубах.

--- Наши повелители...
Вкусно...
Вкусно... --- забормотали Апостолы.

--- Кто меня тронет --- умрёт, --- рявкнула Слан.

--- Да будет пир, --- тихо сказал Гатс.

И полчища Апостолов ринулись вперёд --- на штурм огромной, воздетой к чёрной луне длани.

Каска бросила взгляд на Гатса --- на месте, где он стоял, уже возвышался знакомый кокон.

--- Слушайте все! --- крикнула она товарищам.
--- Гатс сейчас в стадии превращения.
Если падёт хоть одна его жертва --- он обретёт силу Апостола!

--- Если нет прямого приказа, Апостолы всегда выбирают из жертв слабейшую, --- сказал Серпико.
--- Посмотрим...

Наверху тем временем творился кошмар.
Апостолы гибли десятками тысяч.
Длань использовала все свои силы, истребляя нападавших.
Но медленно, пыдь за пядью, их теснили.
Убик в ужасе метался и выл, пытаясь спастись от чудовищных зубов.

--- Вон тот, --- Серпико указал на Убика.
--- Похоже, он только говорить хорошо умеет.

--- Вперёд, --- Каска вытащила крючья и проверила тросы.
--- Серпико, закрепи крючья в надёжном месте.

--- Так точно, --- Серпико схватил крюк и взлетел.
Сильфовый плащ мягко колыхался за его спиной.

\section{Камень}

--- Его нужно заставить снизиться.

--- Дайте мне камень, --- ухмыльнулся Исидоро.
--- А, вот же он.

Бехелит подозрительно зыркнул на парня, но тот уже подхватил камень и невежливо подбросил его в воздух.
До бехелита, кажется, дошло, что дело запахло жареным.
Он зажмурил кривые глаза и даже сжался, насколько вообще может сжаться камень.

\textspace

Убик рухнул на землю и тут же исчез под клацающими челюстями Апостолов.

--- Поздравляю, ты убил члена Длани одним камнем, --- Каска толкнула Исидоро.
--- Нам его даже добивать не придётся.

--- Да ладно?! --- заорал воришка и рванулся посмотреть.
Серпико подхватил его за шиворот.

--- Ты чего? --- сказал он.
--- Держим строй!

--- О тебе будут слагать песни, Исидоро, но до того момента нам нужно, чтобы нас не сожрали!

\textspace

--- Он не изменился! ---- возмущенно сказал Исидоро.
--- Где крылья, шипы и всё прочее?

--- Чем больше Апостол похож на человека, тем он сильнее, --- ответила Ширке.

--- Жалко, что не наоборот.

\textspace

Гатс продирался сквозь толпы Апостолов, оставляя за собой горы трупов.
За ним волочилось нечто, похожее на плаценту --- Гатс, не дожидаясь окончания метаморфозы, вырвался из кокона.
Улучив момент, он подхватил послед и обмотал пуповину вокруг шеи, чтобы она не мешала.

--- Пять минут подождать не может, --- буркнул Серпико.

--- Сколько его знаю, он всегда такой, --- сообщила Каска. 

\textspace

--- Мальчишка, --- голос Войда рокотал под кровавыми небесами, --- твой путь окончен.
Есть только одна Судьба, и это я.

Пуповина всё сильнее сдавливала шею Гатса.

--- Не сможет, --- крякнул Серпико и сорвался с места.
Когти Войда рванулись, преграждая ему путь, но лезвие пера успело надрезать пуповину.
Серпико отлетел в сторону и брякнулся о череп мёртвого Апостола.
Едва ноги Гатса коснулись земли, из спины Войда высунулся окровавленный клинок Убийцы Драконов.

--- Нет никакой Судьбы, Войд, --- прорычал Гатс.
--- Нет никакой Судьбы.

--- Серпико!
Серпико! --- рыдала Фарнезе, дрожащими руками пытаясь зажать огромную рану на его груди.

--- Оставь, госпожа, --- слабо шептал Серпико.
--- Оставь.
Такова моя Судьба --- умереть на руках сестры.

--- Что ты говоришь, Серпико!

--- Ты не знала, конечно, --- Серпико положил голову ей на руки и закрыл глаза.
--- Отец тебе ничего не сказал.
Он никому ничего не сказал.
Теперь ты знаешь.

--- Ты поэтому всё это время был рядом?..

--- Разумеется, нет.
Я был рядом, потому что я принадлежу тебе, моя госпожа и сестра.
Я принадлежал тебе со дня нашей встречи.
Я принадлежу тебе до самой смерти.

Голова Серпико откинулась, и его взгляд остановился.

\end{document}
