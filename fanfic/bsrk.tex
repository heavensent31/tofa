Памяти Миура Кэнтаро

Это фанфик, посвящённый творчеству Миура Кэнтаро.
Он не претендует на серьёзное произведение и не будет использован для получения прибыли --- я отдаю его в общественное достояние (лицензия CC0).

\section{Пролог}

--- А вон та сучка беременна, --- ухмыльнулся Кварт, показав окровавленным клинком в сторону столба.
--- Девятый месяц.

--- Сейчас посмотрим, на что похож ублюдок, --- Донован поиграл мечом.
--- Я никогда не видел брюхатую шлюху изнутри.
А ты, Гамбино?

--- Гамбино, пожалуйста, --- тихо сказала Сисс.

Гамбино промолчал.
Донован воспринял это как согласие и одним взмахом меча распорол беременной живот.

Сисс, едва успев увидеть то, что вывалилось из трупа, упала на колени.
Её вырвало.

--- Глаз-алмаз, Донован, --- сказал Кварт под смех наёмников.

Ребёнок повис на пуповине.
Она сжала его шею, как верёвка сжала шею его матери.
Сисс с рыданиями подбежала к нему и подхватила на руки, пачкаясь в крови, моче и плодной жидкости.

--- Всё в порядке, маленький, --- Сисс освободила шею младенца от пуповины и вгрызлась в неё зубами.

--- Сисс, пошли, --- сказал Гамбино.
--- Он сдохнет.
Ворон его знает, сколько она висела на столбе.

--- Всё в порядке, маленький, --- Сисс баюкала безмолвного младенца.
--- Мама рядом.

Гамбино, вздохнув, подошёл к Сисс, приподнял младенца за ножку и ударил жёсткой, как подошва, ладонью по ягодице.
И младенец закричал.

Всех поразил этот крик.
Он не был похож на детский.
Младенец не плакал, а ревел, как дикий зверь, во всю мощь своих крохотных лёгких.

--- Жить будет, --- констатировал Гамбино.

--- Дай ему имя, дорогой, --- тихо сказала Сисс.

--- Пусть будет тем, кто он есть --- потрохами, --- сказал Гамбино, морщась от рёва младенца.
--- Его имя --- Гатс.

--- Всё хорошо, Гатс, --- зашептала Сисс, успокаиваивая малыша.
--- Мамочка рядом.
Всё хорошо.

--- Иди в хвосте, --- сказал Гамбино, хлопнув Сисс по заднице.
--- От тебя воняет как от скотобойни.
Как разобьём лагерь, помоешься и помоешь ублюдка.

Сисс не ответила.
Она продолжала влюблённо смотреть на ребёнка.

--- Всё хорошо, Гатс, --- говорила она.
--- Мамочка рядом.
Всё хорошо...

\section{Чёрный мечник}

Отдохнуть в этот раз не вышло.
Рядом всё ещё дымился труп обнажённой молодой женщины с развороченным черепом.
Кролик был съеден, прихваченная бутылка разбавленного вина выпита, но острая боль в паху не проходила.
Подумав, Гатс встал и отправился в лес.

Коростель-трава, барсучья смерть, чернобыльник...
Гатс опустошил котелок и наполнил его не очень свежим козлиным жиром --- остатком пиршества одичавших собак.
Мазь должна была покипеть ровно сто три удара сердца.
Затем быстренько охладить котелок, сунув в текущий голубоватый ручей и, шипя, наложить жгучую как огонь грязноватую субстанцию на член и мошонку.

<<Даже смазка Апостола опасна, --- угрюмо думал Гатс, перебинтовывая хозяйство.
--- Будем знать>>.

Боль постепенно стихала, и Гатс нашёл в себе силы припомнить события часовой давности.

--- Ещё!
Ещё! --- стонала красотка, невесть откуда появившаяся у привала Гатса.

--- Попался! --- рыкнула Апостол, впиваясь внезапно удлинившимися ногтями в его спину.
--- Попробовав рая, отведай ад!

Вместо ответа Гатс засунул ей в зубастую пасть металлическую руку.

Бум.

<<Темнеет.
До замка ещё два часа ходьбы>>.



\section{Дуэль}

<<Сейчас я покажу тебе, как действительно используется рот в бою>>, --- подумал Гатс.
Он зубами сорвал с наплечника стёганую накладку и, молниеносно сжевав её, захватил зубами клинок противника.

<<Сумасшедший, что ли?..>> --- Гатс всё больше заинтриговывал Гриффита.
Клинок тот схватил основательно, укус у него медвежий, ещё и эта тряпка...
Но медленно, по пшеничному зерну, шпага всё-таки продвигалась к горлу.

<<Если продвину дальше --- он умрёт.
Если попытаюсь выдернуть --- выбью ему зубы или сильно рассеку щёку.
Люди умирают и от меньшего на поле боя, если не остановить кровь.
Поддаться?
А, это не нужно.
Его хватки вполне достаточно, чтобы...>>

В следующий миг из глаз у Гриффита полетели искры.
