\documentclass[a4paper,12pt,fleqn]{book}\usepackage{polyglossia}\setdefaultlanguage[babelshorthands=true]{russian}\setotherlanguage{english}\defaultfontfeatures{Ligatures=TeX,Mapping=tex-text}\usepackage{xcolor}\newcommand{\ml}[3]{#2}

% \documentclass[a4paper,12pt,fleqn]{book}\usepackage{cooltooltips}\usepackage{polyglossia}\setdefaultlanguage[babelshorthands=true]{russian}\setotherlanguage{english}\defaultfontfeatures{Ligatures=TeX,Mapping=tex-text} \usepackage{xcolor}\definecolor{lightgray}{HTML}{bbbbbb}\color{lightgray}\newcommand{\ml}[3]{\textenglish{\textcolor{black}{#3}} }

% ----------------------

\usepackage{amsmath,amssymb,amsfonts,xltxtra,microtype,graphicx,textcomp}
\usepackage{svg}

% ------ GEOMETRY ------

\usepackage[twoside,left=2.5cm,right=3cm,top=3cm,bottom=4cm,bindingoffset=0cm]{geometry}

% ------ FONT ------

\usepackage{ebgaramond}
\definecolor{darkblue}{HTML}{003153}

% ------ HYPERLINKS ------

\usepackage{hyperref}
\hypersetup{colorlinks=true, linkcolor=darkblue, citecolor=darkblue, filecolor=darkblue, urlcolor=darkblue}

% ------ EPIGRAPH ------

\usepackage{epigraph}
\renewcommand{\epigraphsize}{\footnotesize}
\epigraphrule=0pt
\epigraphwidth=8cm

\usepackage{etoolbox}
\AtBeginEnvironment{quote}{\itshape}
\makeatletter
\newlength\episourceskip
\pretocmd{\@episource}{\em}{}{}
\apptocmd{\@episource}{\em}{}{}
\patchcmd{\epigraph}{\@episource{#1}\\}{\@episource{#1}\\[\episourceskip]}{}{}
\makeatother

% ------ METADATA ------

\newcommand{\tofaauthor}{\ml{$0$}{Эмиль~Весна}{Emil~Viesn\'{a}}}
\newcommand{\tofatitle}{\ml{$0$}{BSRK}{Flowers~for~Granny}}
\newcommand{\tofastarted}{11.03.2022}

% ------ FANCY PAGE STYLE ------

\usepackage{fancyhdr}
\pagestyle{fancy}
\fancyhead[LE,RO]{\thepage}
\fancyhead[LO]{{\small\textsc{\tofatitle}}}
\fancyhead[RE]{{\small\textsc{\tofaauthor}}}
\fancyfoot{}
\fancypagestyle{plain}
{\fancyhead{}
\renewcommand{\headrulewidth}{0mm}
\fancyfoot{}}

% ------ NEW COMMANDS ------

\newcommand{\asterism}{\vspace{1em}{\centering\Large\bfseries$\ast~\ast~\ast$\par}\vspace{1em}}
\newcommand{\textspace}{\vspace{1em}{\centering\Large\bfseries<...>\par}\vspace{1em}}
\newcommand{\FM}{\footnotemark}
\newcommand{\FL}[2]{\footnotetext{См. \textit{\hyperlink{#1}{#2}}.}}
\newcommand{\FA}[1]{\footnotetext{#1 \emph{\ml{$0$}{---~Прим.~авт.}{---~Author.}}}}

\newcommand{\theterm}[3]{\textbf{\hypertarget{#1}{#2}} --- #3}
\newcommand{\thesynonim}[3]{\textbf{#2} --- см. \textit{\hyperlink{#1}{#3}}.}
\newcommand{\theorigin}[3]{\textit{#1:} #2 --- #3}

\begin{document}
 
% ------ TITLE PAGE ------

\begin{titlepage}
{\centering{~\par}\vspace{0.25\textheight}
{\LARGE\tofaauthor}\par
\vspace{1.0cm}\rule{17em}{1pt}\par\vspace{0.3cm}
{\Huge\textsc{\tofatitle}\par}
\vspace{0.3cm}\rule{17em}{2pt}\par\vspace{1.0cm}
{\Large\textit{\ml{$0$}{Фэнтези-фанфик}{Fantasy fanfiction}}\par}
\vspace{0.5cm}\asterism\par\vspace{1.0cm}
{\textbf{\ml{$0$}{Начато:}{Started:}}~\tofastarted\par}\vfill
{\Large\ml{$0$}{Создано~в}{Created~by}~\XeLaTeX}\par}
\end{titlepage}

\tableofcontents

\chapter{BSRK}

Памяти Миура Кэнтаро

Это фанфик, посвящённый творчеству Миура Кэнтаро.
Он не претендует на серьёзное произведение и не будет использован для получения прибыли --- я отдаю его в общественное достояние (лицензия CC0).

\section{Пролог}

--- А вон та сучка беременна, --- ухмыльнулся Кварт, показав окровавленным клинком в сторону столба.
--- Девятый месяц.

--- Сейчас посмотрим, на что похож ублюдок, --- Донован поиграл мечом.
--- Я никогда не видел брюхатую шлюху изнутри.
А ты, Гамбино?

--- Гамбино, пожалуйста, --- тихо сказала Сисс.

Гамбино промолчал.
Донован воспринял это как согласие и одним взмахом меча распорол беременной живот.

Сисс, едва успев увидеть то, что вывалилось из трупа, упала на колени.
Её вырвало.

--- Глаз-алмаз, Донован, --- сказал Кварт под смех наёмников.

Ребёнок повис на пуповине.
Она сжала его шею, как верёвка сжала шею его матери.
Сисс с рыданиями подбежала к нему и подхватила на руки, пачкаясь в крови, моче и плодной жидкости.

--- Всё в порядке, маленький, --- Сисс освободила шею младенца от пуповины и вгрызлась в неё зубами.

--- Сисс, пошли, --- сказал Гамбино.
--- Он сдохнет.
Ворон его знает, сколько она висела на столбе.

--- Всё в порядке, маленький, --- Сисс баюкала безмолвного младенца.
--- Мама рядом.

Гамбино, вздохнув, подошёл к Сисс, приподнял младенца за ножку и ударил жёсткой, как подошва, ладонью по ягодице.
И младенец закричал.

Всех поразил этот крик.
Он не был похож на детский.
Младенец не плакал, а ревел, как дикий зверь, во всю мощь своих крохотных лёгких.

--- Жить будет, --- констатировал Гамбино.

--- Дай ему имя, дорогой, --- тихо сказала Сисс.

--- Пусть будет тем, кто он есть --- потрохами, --- сказал Гамбино, морщась от рёва младенца.
--- Его имя --- Гатс.

--- Всё хорошо, Гатс, --- зашептала Сисс, успокаиваивая малыша.
--- Мамочка рядом.
Всё хорошо.

--- Иди в хвосте, --- сказал Гамбино, хлопнув Сисс по заднице.
--- От тебя воняет как от скотобойни.
Как разобьём лагерь, помоешься и помоешь ублюдка.

Сисс не ответила.
Она продолжала влюблённо смотреть на ребёнка.

--- Всё хорошо, Гатс, --- говорила она.
--- Мамочка рядом.
Всё хорошо...

\section{Чёрный мечник}

Отдохнуть в этот раз не вышло.
Рядом всё ещё дымился труп обнажённой молодой женщины с развороченным черепом.
Кролик был съеден, прихваченная бутылка разбавленного вина выпита, но острая боль в паху не проходила.
Подумав, Гатс встал и отправился в лес.

Коростель-трава, барсучья смерть, чернобыльник...
Гатс опустошил котелок и наполнил его не очень свежим козлиным жиром --- остатком пиршества одичавших собак.
Мазь должна была покипеть ровно сто три удара сердца.
Затем быстренько охладить котелок, сунув в текущий голубоватый ручей и, шипя, наложить жгучую как огонь грязноватую субстанцию на член и мошонку.

<<Даже смазка Апостола опасна, --- угрюмо думал Гатс, перебинтовывая хозяйство.
--- Будем знать>>.

Боль постепенно стихала, и Гатс нашёл в себе силы припомнить события часовой давности.

--- Ещё!
Ещё! --- стонала красотка, невесть откуда появившаяся у привала Гатса.

--- Попался! --- рыкнула Апостол, впиваясь внезапно удлинившимися ногтями в его спину.
--- Попробовав рая, отведай ад!

Вместо ответа Гатс засунул ей в зубастую пасть металлическую руку.

Бум.

<<Темнеет.
До замка ещё два часа ходьбы>>.

\asterism

Д\`{у}хи, подобно дух\`{а}м, имеют запах.
Это кажется невероятным для любого, кто ни разу не попадал в Расщелину.
Гатс знал, как пахнут голодные тени.
Воздух начинал отдавать плесенью и гнилью --- запах заполненного трупами погреба.

Лёгкий запах гнили настиг его у ворот города.
Заря ещё пылала в стеклянно-чистом, похожем на шар предсказателя небе.
На горизонте ни облачка.
Ночь обещала быть тёплой, лунной и звёздной --- в такую комфортно дежать оборону.
Правда, не в городе.

--- Пода-ааайте на хлебушек, добрый господин, --- унылым механическим голосом прогнусил попрошайка, сжимая чиненный-перечиненный костыль.
--- На хле-ееебушек...
Ну и пёс с тобой, жлоб.
Чтоб ты сдох.

Мимо проехала телега с клеткой, в которой сидели подростки.
Телегу сопровождали несколько всадников с пышными плюмажами и напомаженными усами --- явно декоративных.
Сумрачным, потухшим взглядом посмотрел на Гатса красивый молодой парень.
Его карие глаза, вздёрнутый веснушчатый нос и прямые пшеничные волосы вызывали опасные непрошенные воспоминания.
Парня за руку держала девушка --- смуглая, черноглазая, коротко стриженная...

Гатс отвернулся, поправил капюшон и попытался выкинуть морок из головы.

\asterism

--- Сколько стоишь, красавица? --- вопрошал вдрызг пьяный стражник.
Он был при полном параде --- плащ, латы, начищенный шлем.
Правда, только выше пояса.
Ниже полный парад годился разве что в баню.

--- Как и все --- час десять, ночь тридцать, --- кокетливо отвечала красавица, кутаясь в шерстяную шаль.

<<Приличная таверна, --- отметил Гатс.
--- Может, здесь и...
Впрочем, похоже, я сглазил>>.

--- Да не вертись ты! --- орал лысый наёмник, распространяя крепкий запах пива и сожжённого луком желудка.
В его не очень верной руке вращался метательный нож.
Его товарищи вопили и улюлюкали, подначивая метателя.

Нож свистнул и вонзился в дощатую стену таверны.
Цель --- а целью оказался привязанный за шею истинный эльф --- упорхнула и показала лысому язык.

--- Прома-аазал! --- захохотали наёмники.

--- Ставлю ещё четыре, что и в третий раз мимо!

--- Десять!
Ставлю сегодняшний паёк шлюшатины, что Игор попадёт!

--- А Дин ставит?

--- Он отлить пошёл.
Ставлю два от его имени, ты знаешь на что!

--- Подождите, парни, я записываю.
Грин, бабки, бабки на бочку.
Никаких ставок под честное слово...
Сорес, сраный недоносок, ещё раз стащишь монету --- выбью тебе глаз вместе с мозгами.

Местная публика явно была от развлечения не в восторге.
Служанки кутались в шали и натягивали чепцы на глаза, несмотря на духоту, жались по углам, выпивку разносили быстро и не поднимая головы.
Они знали, что уже участвуют в ставках, и знали, что стоит на кону: как только эльф исчерпает свою занятность, возьмутся за них.
Мужчины тоже болели за то, чтобы эльф продержался как можно дольше, но уже по другим причинам.

--- Недолго будет у нас спокойный разговор, чувствую, --- пробормотал стриженный под горшок меняла своему товарищу.
Тот нервно оглядывался, проверяя, жив ли эльф.

--- Твоя правда, Гер.
Давай побыстрее.
С этими наёмниками Кока не связывается даже мэр, нам тем паче не с руки...

Оба вздрогнули и проводили Гатса взглядом, когда он прошёл мимо.

\asterism

--- Заказывать будем?

--- Бутыль воды, --- Гатс положил на прилавок семь серебряных монет.

--- Тут на целую бочку, --- спокойно прогудел трактирщик из-под усов.
--- Вода у нас не золотая, забери лишнее.

--- Остальное на ремонт, --- Гатс один за другим вложил восемь болтов в массивный чжугэну на запястье.
Трактирщик выпучил глаза.

--- Ты чего...

--- ...сейчас я заткну твой поганый маленький рот, крылатая крыса! --- орал Игор.
Его тираду прервал стальной болт, пробив висок и выйдя из глазницы с другой стороны.

На таверну пала тишина.

--- Кто это, раздрочи меня Господь, сделал, сука? --- один из наёмников начал подниматься, вытаскивая меч.
В следующий миг чжугэну снова защёлкал.

Гатс снял задвижку, освободил тетиву и закрыл магазин арбалета.
Использованные болты отправились в <<грязную>> сумку на левом бедре --- один из них погнулся от удара о шлем и требовал правки, все без исключения следовало отмыть от крови, мозгов и соплей.
Наёмники так и лежали, раскрыв невидящие глаза в ужасе и удивлении.
Местные в страхе спрятались за столами и притихли.
За столом подвывал последний наёмник --- худой, с немытыми чёрными волосами писарь.
Пергамент со ставками так и лежал перед ним, залитый кровью.
Из носа у писаря торчал болт.

Гатс схватился за болт рукой.
Писарь взвыл и запыхтел, вытаращив глаза и высунув обмётанный жёлтым язык.

--- Ты один из людей замка Кока? --- спокойно спросил Гатс.

Писарь задышал и заскулил чуть громче.
Гатс пошевелил болт, вызвав новый поток визга и стенаний.

--- Спрашиваю ещё раз.

--- Ва, ва, --- торопливо ответил писарь.

Гатс лениво сбил металлическоим пальцем блоху, перебежавшую с сальных чёрных волос на его руку.
Затем пошевелил пергамент со ставками.

--- Вижу, что человек грамотный.
Писать-читать умеешь, значит, и памятью не обижен.
Передай своему хозяину: пришёл Чёрный Мечник.

--- Фофоифо? --- дрожащим голосом поинтересовался писарь.

--- Больше ничего.

--- Сзади! --- завопил тонким голоском эльф.

Гатс схватил рукоять притороченного к спине меча --- и таверна взорвалась криками ужаса.
Кровь заплескала потолок, пол, столы, размозжённые кишки мягко шлёпались на рассохшиеся деревянные поверхности и тут же пристывали к ним.
Разрубленный напополам наёмник клацнул челюстями и замер.

Язык не поворачивался назвать эту громадину мечом.
Чересчур тяжёлый, чересчур тупой, похожий на необработанную стальную плиту.
По поверхности шли жилы ржавчины и трещины от длительного использования.
Гатс взмахом отряхнул меч и убрал его за спину.

--- Рассчитываю на тебя, --- кивнул он писарю.
Тот всё ещё лежал щекой на столе, боясь пошевелиться.

--- Эй! --- крикнул эльф.
Тонкие крылышки в отчаянии зажужжали.
--- Освободи меня!

Но Гатс уже был в дверях таверны.
Его плащ в последний раз колыхнулся и слился с вечерними сумерками.

\asterism

--- Гады, --- эльф присел на скрипучую вывеску таверны и принялся выковыривать из зубов пеньковые волокна.
--- Хоть бы один помог...

Всё, чего обычно желает невольник --- это свобода.
Свободой он может жить годами и десятилетиями, грезя во сне и наяву, подчиняя своей мечте весь жизненный уклад, насколько это позволяет неволя.
Но когда свобода всё-таки приходит, возникает закономерный вопрос --- а что дальше?
Для большинства бывших невольников он так и остаётся нерешённым.

К счастью, эльф был в плену не очень долго по меркам своего народа.
Кроме того, ему весьма помогала квазиполиморфность ментального тела, делающая некоторых существ практически неуязвимыми к душевным травмам --- способность сколь полезная, столь и редкая.
Глупые люди, так и не разобравшись, как это работает, спутали это явление с гиперактивным расстройством и занесли в список заболеваний.

Клацанье доспехов всё дальше уходило в темноту.
В кроне вяза тихо пел одинокий соловей, словно о чём-то спрашивая каждые полминуты.
Эльф быстренько прикинул --- чужой город, бродячий театр давно уехал невозвратной тропой, сородичей и даже просто сочувствующих нет на сотни миль вокруг, крылья не железные.
А гигант выглядел если не добрым, то хотя бы надёжным и...

--- Эй, подожди!
Подожди меня!

\asterism

--- Начал спасать, так доводи дело до конца! --- пищал эльф в ухо Чёрного Мечника.
Ухо на пищание никак не реагировало.

--- Ладно, я погорячился, --- эльф откашлялся.
--- Здравствуй, я Пак.
Рад знакомству.
Я путешествовал с труппой артистов, но эти бандиты напали на нас и заперли меня в птичью клетку.
Обливали выпивкой, крылья угрожали поджечь!

Ответа не последовало.
Пак прикинул, насколько невежливо будет на данной стадии знакомства садиться на плечо.
Вывод очень опечалил его уставшие крылья.

--- Красивый меч, --- попытался он ещё раз.
--- Наверное, тяжёлый.

Ноль внимания.

Пак облетел Гатса и задумчиво оглядел меч.
Чересчур тяжёлый, чересчур тупой, похожий на необработанную стальную плиту.
Таким невозможно размахивать и тем более разрезать кого-то на ровные половинки.
Однако в свете зажигающихся звёзд Пак ясно видел идеальные призрачные грани и острую как бритва кромку.

<<Меч на самом деле достаточно лёгкий, не тяжелее обычного двуручного меча, и очень острый.
Но свои свойства он должен проявлять только в Расщелине.
Как же он?..
А, вон оно что>>.

Знак на шее.
Таинственная руническая лигатура горела желтоватым огнём и трепетала, словно абстрактная, но вполне живая оса.
Подобие младенческой пуповины стягивало шею Гатса, связывая его физическое тело с миром духов.

Обычные методы знакомства --- рассказ о себе и лесть --- не сработали, и Пак решил перейти к практической стороне вопроса.

--- Слушай, тебе лучше бежать отсюда, --- проникновенно начал он.
--- Эти наёмники много раз нападали и на город, несмотря на то, что мэр платит им дань.
Тебя покромсают на ломтики, если ты попадёшься.
И покромсают не только наёмники, но и стража, потому что они все повязаны...
Ай!
Ты мне чуть крылья не переломал!
Ты вообще представляешь, каково эльфу получать шлепки такой огромной рукой?
А давай я тебя телегой ударю?

--- Лети отсюда, --- лаконично посоветовал Гатс.
--- В следующий раз прихлопну насмерть.

Из темноты медленно вышли закованные в железо стражники.
В грудь Гатсу упёрлись копья, заскрипели взводимые арбалеты.

--- Я пытался, --- развёл руками Пак и шмыгнул под ближайшую крышу.

\asterism

<<Самый лучший способ проникнуть в крепость --- это проникнуть в крепость в качестве пленника.
Проблема в том, что тысяча и одна вещь могут пойти не по плану>>.

Хлысь.
Тяжёлый бич сорвал ещё лоскуток кожи с груди Гатса.

--- Молчит, --- констатировал палач.
--- Молчит, зараза.
Что бы ему эдакое задеть, чтобы подал голос?
Э, глыба!
Может, тебе соски вырезать?
Или очко прочистить горяченькой кочергой?
За что голосуем?

--- Хватит.

--- Господин мэр, --- палач забормотал и согнулся в неуклюжем поклоне, насколько позволял круглый как шарик живот.

Мэр нервно оглянулся на сопровождавших его стражников и подошёл к столу.

--- Его оружие?

--- Так точно, господин мэр, всё до единого его самого.
Чжугэну, тэцубиси, метательные ножи, яды, три вида арбалетных болтов, и вот это...
Что это вообще, меч?
Как не помер под таким весом ублюдок.
И на какого зверя только с таким арсеналом ходят, м?

Мэр снова оглянулся на стражу и подошёл ближе к Гатсу.

--- Ты наёмник?

Гатс промолчал.

--- Ты вообще понимаешь, что случится из-за твоей выходки? --- дрожащим голосом продолжал мэр.
--- Мы долгие месяцы обустраивали нашу жизнь в соответствии с новым... правлением.
А сейчас...
Сейчас может статься, что город просто уничтожат.
Изведут под корень всех до последнего жителя.
Детишек, женщин, стариков.
И всё из-за того, что ты убил людей Кока!

--- Надо же, про детишек вспомнил, --- хмыкнул Гатс.
--- А что за пугало за тобой?

--- Что ты сказал? --- рыкнул один из стражников, подняв копьё.

--- Стой!
Стой! --- мэр замахал руками.
--- Пожалуйста, остановись.
Я пытаюсь послужить лорду как можно лучше!

Мэр подошёл ещё ближе к пленнику.

--- Эти люди опасны, --- сказал он.
--- Их хозяин опасен.
Ходят слухи... --- мэр снова оглянулся, --- ...истинная правда, что его нельзя убить.
Он не человек, он нечто более страшное, неизведанное.
Создание самой преисподней!

--- Если хочешь продать меня Кока --- продавай, --- перебил Гатс.
--- И так понятно, что ты с ним в сделке.
К чему всё это нытьё про детишек, стариков и преисподнюю?
В добряка решил поиграть?

--- Да что ты знаешь! --- прошипел мэр.

--- Я знаю, --- оскалился Гатс.
--- Я знаю больше, чем ты, чем громилы за твоей спиной.
Если вкратце --- он жрёт людей.
А ты, народный избранник, поставляешь ему корм.

Мэр выхватил из рук у палача бич.
Костяшки старческий рук побелели.

--- Кстати, о детишках и женщинах, --- продолжал Гатс.
--- Я видел повозку у ворот города не далее как час назад.

--- Я сделаю всё, чтобы защитить мой город! --- рявкнул мэр.

--- И свою шкуру, --- присовокупил Гатс.

--- Ах ты... --- мэр захлебнулся, схватившись за сердце, и упал на колени.
Бич выпал из ослабевших пальцев на древний каменный пол.

--- Господин мэр!
Господин мэр! --- палач скакал вокруг как кролик.
--- Али поплохело вам?
Сердечко схватило, господин мэр?

--- Сдери с него всё, что допустимо по пятому приказу, --- хрипло и зло ответил мэр.
--- Не калечить.
Этого наёмника надо доставить в замок в надлежащем состоянии.
Сквайр, приготовьте экипаж.
Я сам поеду в замок Кока и попробую вымолить прощение для города...

\asterism

Глаза сложно было назвать глазами.
Ни зрачков, ни глазных яблок в привычном понимании там не было.
Глазницы мужчины были заполнены двумя шарообразными сгустками красного тумана.
Эти сгустки темнели, светлели, переливались, следуя мельчайшим изменениям настроения их хозяина.
Стоящий на одном колене писарь был осведомлён об этом свойстве --- он украдкой наблюдал за глазами сюзерена, пытаясь прочитать в них свой приговор.

--- Чёрный Мечник, значит?

--- Да, милорд, --- писарь изо всех сил пытался говорить не в нос.

--- Как выглядит?

--- Был одет во всё чёрное, милорд.

--- Звучит логично.
Поэтому он и Чёрный Мечник.
Не Зелёный, не Синий.
Важные детали, Освальд.
Важные.

--- Рука железная, милорд, в ней самозаряд с пружинкой.
Им он наших и порешил в один миг.

--- Он убил шесть --- шесть! --- латников с помощью пружинного чжугэну?

--- Я бы сам в такое не поверил, милорд.
Но я сам видел, милорд.
Он стрелял очень быстро и точно.
Шесть стрел, шесть парней, одна Игору, последняя мне.
Дин только отлить пошёл, и...

--- Меня не интересует, сколько раз Дин посещал уборную.
Что ещё?

--- Меч, милорд.
Большой, толстый, как дверь.
Меч ржавый, тупой как хер конский, однако ж он им Дина напополам разрубил, как поросёнка...

--- Достаточно, --- хозяин махнул рукой.
--- Десерт.

Пожилая служанка аккуратно подняла тарелку с объедками и приборы.
Приборы он держала, прихватив салфеткой за самый кончик, словно боясь ненароком коснуться грязной части.
Судя по всему, страх был обоснованным --- на нескольких пальцах у служанки красовались бородавки и странные ожоги, как от кислоты.

Одетый с иголочки слуга принёс серебряное блюдо с крышкой и с поклоном поставил его перед хозяином.
Ещё одна служанка расстелила затейливо сложенную салфетку и выложила на неё приборы в строгой очерёдности.

\asterism

--- Меня не интересует золото и людские жертвы, господин мэр, --- прервал его хозяин.
--- У нас с вами был договор.
Вы обеспечиваете порядок, я обеспечиваю спокойствие.
Скажите, мэр, вы обеспечили порядок?

--- Я-я пытался, милорд, наша стража...

--- Ваша стража не справилась.
То есть договор можно считать аннулированным.
Ваши извинения мне не интересны.

Хозяин щёлкнул пальцами.
Мэра тут же подхватили под руки и повели из зала.

--- Что всё это значит?!
Милорд!
Милорд, умоляю вас!..

Хозяин встал из-за стола и подошёл к окну.
Его острые зубы сверкнули в последних лучах зари.

--- Как будто мне нужен был повод, --- усмехнулся он.
--- Я просто хочу увидеть вселенский пожар.
Я хочу видеть, как людишки метаются в огне.
Я хочу слышать, как ломаются их кости под копытами моего коня.
Это лучшее, что мне приходилось слышать.
Это то, ради чего стоит жить.
Но люди любят простые и понятные им правила.
Веди себя хорошо --- и ты будешь вознаграждён.
Будешь вести себя плохо --- тебя ждёт наказание.
Извинись --- и будешь прощён.

Хозяин захохотал.
Его хохот метался по зале, точно испуганная птица, ломая крылья о стены и застревая в изящной старинной лепнине.

--- В этом городе лишь один закон --- моё желание, --- хозяин сощурил алые туманные глаза.
--- И сейчас ты в этом убедишься, Чёрный Мечник.

\section{Дуэль}

<<Сейчас я покажу тебе, как действительно используется рот в бою>>, --- подумал Гатс.
Он зубами сорвал с наплечника стёганую накладку и, молниеносно сжевав её, захватил зубами клинок противника.

<<Сумасшедший, что ли?..>> --- Гатс всё больше заинтриговывал Гриффита.
Клинок тот схватил основательно, укус у него медвежий, ещё и эта тряпка...
Но медленно, по пшеничному зерну, шпага всё-таки продвигалась к горлу.

<<Если продвину дальше --- он умрёт.
Если попытаюсь выдернуть --- выбью ему зубы или сильно рассеку щёку.
Люди умирают и от меньшего на поле боя, если не остановить кровь.
Поддаться?
А, это не нужно.
Его хватки вполне достаточно, чтобы...>>

В следующий миг из глаз у Гриффита полетели искры.

\end{document}

