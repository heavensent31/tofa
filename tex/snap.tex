% \documentclass[a4paper,12pt,fleqn]{book}\usepackage{polyglossia}\setdefaultlanguage[babelshorthands=true]{russian}\setotherlanguage{english}\defaultfontfeatures{Ligatures=TeX,Mapping=tex-text}\usepackage{xcolor}\newcommand{\ml}[3]{#2}

\documentclass[a4paper,12pt,fleqn]{book}\usepackage{cooltooltips}\usepackage{polyglossia}\setdefaultlanguage{russian}\setotherlanguage{english}\defaultfontfeatures{Ligatures=TeX,Mapping=tex-text} \usepackage{xcolor}\definecolor{lightgray}{HTML}{bbbbbb}\color{lightgray}\newcommand{\ml}[3]{\textenglish{\textcolor{black}{#3}}}

\setlength{\headheight}{15pt}

% ----------------------

\usepackage{amsmath,amssymb,amsfonts,xltxtra,microtype,graphicx,textcomp}
\usepackage{svg}

% ------ GEOMETRY ------

\usepackage[twoside,left=2.5cm,right=3cm,top=3cm,bottom=4cm,bindingoffset=0cm]{geometry}

% ------ FONT ------

\usepackage{ebgaramond}
\definecolor{darkblue}{HTML}{003153}

% ------ HYPERLINKS ------

\usepackage{hyperref}
\hypersetup{colorlinks=true, linkcolor=darkblue, citecolor=darkblue, filecolor=darkblue, urlcolor=darkblue}

% ------ EPIGRAPH ------

\usepackage{epigraph}
\renewcommand{\epigraphsize}{\footnotesize}
\epigraphrule=0pt
\epigraphwidth=8cm

\usepackage{etoolbox}
\AtBeginEnvironment{quote}{\itshape}
\makeatletter
\newlength\episourceskip
\pretocmd{\@episource}{\em}{}{}
\apptocmd{\@episource}{\em}{}{}
\patchcmd{\epigraph}{\@episource{#1}\\}{\@episource{#1}\\[\episourceskip]}{}{}
\makeatother

% ------ METADATA ------

\newcommand{\tofaauthor}{\ml{$0$}{Эмиль~Весна}{Emil~Viesn\'{a}}}
\newcommand{\tofatitle}{\ml{$0$}{ЗМЕИНАЯ~ЯМА}{Snake~Pit}}
\newcommand{\tofastarted}{11.11.2021}

% ------ FANCY PAGE STYLE ------

\usepackage{fancyhdr}
\pagestyle{fancy}
\fancyhead[LE,RO]{\thepage}
\fancyhead[LO]{{\small\textsc{\tofatitle}}}
\fancyhead[RE]{{\small\textsc{\tofaauthor}}}
\fancyfoot{}
\fancypagestyle{plain}
{\fancyhead{}
\renewcommand{\headrulewidth}{0mm}
\fancyfoot{}}

% ------ NEW COMMANDS ------

\newcommand{\asterism}{\vspace{1em}{\centering\Large\bfseries$\ast~\ast~\ast$\par}\vspace{1em}}
\newcommand{\textspace}{\vspace{1em}{\centering\Large\bfseries<...>\par}\vspace{1em}}
\newcommand{\FM}{\footnotemark}
\newcommand{\FL}[2]{\footnotetext{См. \textit{\hyperlink{#1}{#2}}.}}
\newcommand{\FA}[1]{\footnotetext{#1 \emph{\ml{$0$}{---~Прим.~авт.}{---~Author.}}}}

\newcommand{\theterm}[3]{\textbf{\hypertarget{#1}{#2}} --- #3}
\newcommand{\thesynonim}[3]{\textbf{#2} --- см. \textit{\hyperlink{#1}{#3}}.}
\newcommand{\theorigin}[3]{\textit{#1:} #2 --- #3}

% ------ DIFFICULT TO WRITE TERMS ------

\newcommand{\Aatris}{\"{A}\={a}tr\v{\i}s}
\newcommand{\Akchsar}{\`{A}kchs\r{a}r}
\newcommand{\Chhammitrai}{Chh\`{a}mm\={\i}tr\^{a}i}
\newcommand{\Chhanei}{Chh\r{a}n\^{e}i}
\newcommand{\Chitram}{Ch\"{\i}tr\'{a}m}
\newcommand{\Choe}{Cho\^{e}}
\newcommand{\choe}{cho\^{e}}
\newcommand{\Harrmatr}{H\r{a}rrm\`{a}tr}
\newcommand{\tHat}{H\={a}t}
\newcommand{\Hei}{H\r{e}i}
\newcommand{\hei}{h\r{e}i}
\newcommand{\Hoesitr}{Ho\`{e}s\={\i}tr}
\newcommand{\hoesitr}{ho\`{e}s\={\i}tr}
\newcommand{\Kchaagotr}{Kch\^{a}\={a}g\~{o}tr}
\newcommand{\kchaagotr}{kch\^{a}\={a}g\~{o}tr}
\newcommand{\Kcharas}{Kch\'{a}r\v{a}s}
\newcommand{\Kchatrim}{Kch\r{a}tr\"{\i}m}
\newcommand{\Kchenoel}{Kch\={e}no\^{e}}
\newcommand{\kchenoel}{kch\={e}no\^{e}}
\newcommand{\Kchenoet}{Kch\"{e}no\^{e}}
\newcommand{\kchenoet}{kch\"{e}no\^{e}}
\newcommand{\Kchoho}{Kch\`{o}h\^{o}}
\newcommand{\Kchotlam}{Kch\={o}tl\'{a}m}
\newcommand{\Kchotris}{Kch\={o}tr\v{\i}s}
\newcommand{\Kihotr}{K\^{\i}h\~{o}tr}
\newcommand{\kihotr}{k\^{\i}h\~{o}tr}
\newcommand{\Kukchuatr}{K\`{u}kchu\={a}tr}
\newcommand{\kukchuatr}{k\`{u}kchu\={a}tr}
\newcommand{\kutraph}{k\r{u}tr\={a}\`{a}ph}
\newcommand{\Laaka}{L\={a}\"{a}k\^{a}}
\newcommand{\laaka}{l\={a}\"{a}k\^{a}}
\newcommand{\Lechoe}{L\={e}cho\`{e}}
\newcommand{\lechoe}{l\={e}cho\`{e}}
\newcommand{\Likas}{L\^{\i}k\v{a}s}
\newcommand{\Likchmas}{L\={\i}kchm\r{a}s}
\newcommand{\Likchoe}{L\^{\i}kcho\^{e}}
\newcommand{\Loem}{Lo\~{e}m}
\newcommand{\Maaras}{M\"{a}\={a}r\v{a}s}
\newcommand{\Mitchoe}{M\={\i}tcho\^{e}}
\newcommand{\Mitlikch}{M\={\i}tl\={\i}kch}
\newcommand{\Mitris}{M\={\i}tr\={\i}s}
\newcommand{\Oerchoe}{O\r{e}rcho\^{e}}
\newcommand{\Oerlikch}{O\r{e}rl\'{\i}kch}
\newcommand{\Sat}{S\={a}t}
\newcommand{\Satchir}{S\={a}tch\"{\i}r}
\newcommand{\Satrakch}{S\={a}tr\`{a}kch}
\newcommand{\Seli}{S\r{e}l\={\i}}
\newcommand{\Sirtchu}{S\r{\i}rtch\'{u}}
\newcommand{\Sitris}{S\~{\i}tr\v{\i}s}
\newcommand{\Siusiu}{S\~{\i}u-s\~{\i}u}
\newcommand{\siusiu}{s\~{\i}u-s\~{\i}u}
\newcommand{\Sogcho}{S\"{o}gch\={o}}
\newcommand{\Sotron}{S\~{o}tr\`{o}n}
\newcommand{\Tchalas}{Tch\r{a}l\v{a}s}
\newcommand{\Tchammitr}{Tch\`{a}mm\={\i}tr}
\newcommand{\Tchanoe}{Tch\r{a}no\^{e}}
\newcommand{\Tchartchaahitr}{Tch\~{a}rtch\"{a}\={a}h\r{\i}tr}
\newcommand{\Tchartchu}{Tch\~{a}rtch\'{u}}
\newcommand{\Tchitron}{Tch\"{\i}tr\`{o}n}
\newcommand{\Tchu}{Tch\`{u}}
\newcommand{\tchu}{tch\`{u}}
\newcommand{\Technku}{T\`{e}chnk\r{u}}
\newcommand{\Tesarrokch}{Te's\'{a}rr\r{o}kch}
\newcommand{\Tesatron}{Te's\'{a}tr\v{o}n}
\newcommand{\Traa}{Tr\={a}\"{a}}
\newcommand{\traa}{tr\={a}\"{a}}
\newcommand{\Trai}{Tr\r{a}i}
\newcommand{\trai}{tr\r{a}i}
\newcommand{\TraRenkchal}{Tr\r{a}-R\={e}nkch\'{a}l}
\newcommand{\Trukchual}{Tr\`{u}kchu\r{a}l}


\begin{document}

% ------ TITLE PAGE ------

\begin{titlepage}
{\centering{~\par}\vspace{0.25\textheight}
{\LARGE\tofaauthor}\par
\vspace{1.0cm}\rule{17em}{1pt}\par\vspace{0.3cm}
{\Huge\textsc{\tofatitle}\par}
\vspace{0.3cm}\rule{17em}{2pt}\par\vspace{1.0cm}
{\Large\textit{\ml{$0$}{Фантастический~роман}{Science~fiction}}\par}
\vspace{0.5cm}\asterism\par\vspace{1.0cm}
{\textbf{\ml{$0$}{Начато:}{Started:}}~\tofastarted\par}\vfill
{\Large\ml{$0$}{Создано~в}{Created~by}~\XeLaTeX}\par}
\end{titlepage}

\tableofcontents

\part{Змеиная яма}

\chapter{Предисловие}

О ранней жизни Кхотлам ар'Люм э'Кахрахан известно немногое.
Несмотря на её имя, многие кахраханцы клялись, что в жизни не знали женщину с таким именем и никогда она в городе не жила.
Это породило многочисленные слухи о ее криминальном прошлом.
И только к концу жизни Кхотлам рассказала питомцам некоторые факты, о которых до этого знал лишь ее мужчина, Хитрам ар'Кхир.

Родилась Кхотлам в неизвестном хуторе под Кахраханом.
Её кормилец, воспитывавший её и братьев в одиночку, был обвинён в Разрушении, но оправдан Советами.
Это не помешало хуторским устроить над ним самосуд.
Перед смертью Сатрилам ар'Сатр попросил убийц позволить ему проститься с детьми.
Он собрал им дорожные котомки и вытолкнул за дверь.
Со своими братьями Кхотси, едва перевалившая возраст Отбора, разминулась в джунглях и ничего не знает об их судьбе.
Скорее всего, братья решили бросить ребёнка, который был для них обузой.

Дальше случилось то, что иначе как чудом назвать нельзя.
Голодного и измождённого ребёнка приютила кормящая самка ягуара.
Несколько рассветов Кхотси сосала её грудь и грелась под её боком.
Так их и нашли охотники, почуявшие какой-то необычный запах от ягуарьего следа.
Чуда они не увидели --- у деревенских каждого второго, по их словам, выкормил дикий зверь --- но ей нашли семью в Кахрахане.
Дом ар'Люм не поверил ни истории про самосуд, ни истории про ягуара, несмотря на то, что Кхотлам сама неоднократно их рассказывала, но приняли её как родную.
Так Кхотлам получила образование купца.
Её образование прервалось радужным безумием, которое забрало треть горожан, в том числе её семью и многих друзей.
После этого Кхотлам ушла на берег моря и села на первый попавшийся корабль, который довез ее до Тхитрона.
С Тхитрона началась её карьера торговца.

\chapter{Лорика}

Митхэ поджала губы.
Эта торгашка начала раздражать её с первой михнет.
Впрочем, выбирать не приходится --- командир дал чёткий приказ.

--- Я передам твои слова, --- она коротко поклонилась и вышла.

\asterism

--- Ты не будешь так обращаться со своими воинами при мне.

--- А кто мне запретит?

--- Я --- хозяйка каравана, и я тебя наняла.

--- В таком случае мне стоит разорвать договор, как думаешь?

--- Как только ты вернешь мне золото --- мы его разорвём.

--- Тхэай, красавица, так дела не де...

--- Я тебе не красавица, Юстиция Марвий.
Обращайся ко мне по имени.
Возвращаешь деньги --- свободен.
Если, конечно, ты ещё не проиграл свой аванс.

По выражению лица Марвия Кхотлам поняла, что попала в точку.

--- Я не буду тебе ничего возвращать.
Я оказал тебе услуги на выплаченное золото.

--- Мы прошли три дня из тех восемнадцати, на которые мы договаривались.
Я выплатила тебе авансом половину.
Тебе посчитать на пергаменте точную сумму, которую ты мне должен?

--- Я тебе ничего не должен.
Ты платила мне за то, чтобы тебя не убили и не ограбили.
Тебя не убили и не ограбили.

--- Тебе показать контракт и зачитать, за что именно я платила?

--- Убери это топливо, Кхотлам.
Я платить не буду.

--- Не хочешь платить золотом --- будешь платить репутацией.
Я позабочусь о том, чтобы все узнали, что у Марвиев нет чести.

--- Аккуратнее выбирай слова, Кхотлам.
Мы посреди сельвы, и жилья нет в десятках кхене пути.
Мало ли что может случиться с караваном...
Я сказал что-то смешное?

--- Я делала полные сводки о караване во всех поселениях.
Все знают, что я везу, кого я наняла и в каком количестве.
Во всех поселениях на пути следования знают, кого ждать и когда.

--- И как тебе это поможет?

--- Если ты тронешь меня или кого-то из моих людей, заказов у тебя больше не будет.
И это в лучшем случае.

--- Кхотлам, не провоцируй меня.

--- Выбор у тебя простой, уно Марвий.
Либо насилие и дебош прекращается --- либо ты платишь, золотом или репутацией.
Иди пообедай и обдумай мои слова.

Взбешённый Марвий вылетел из палатки отравленной стрелой, даже забыв попрощаться.

\asterism

--- Это правда? --- начала с порога Митхэ.

--- Здравствуй, Митхэ ар'Кахр.
Что именно ты хотела узнать?

--- Командир сказал, что мы разрываем контракт и уходим.

--- Я ему уже передала, что контракт он сможет разорвать только после того, как выплатит мне то, что должен --- рассветами или золотом.

--- То есть мы ещё и должны тебе остались?

--- Боюсь, что так.

--- Глупость какая-то, --- Митхэ подсела к Кхотлам, не спрашивая разрешения, и отхлебнула из чаши.
--- Я понимаю, что ты не со мной дела ведёшь, но не могла бы ты объяснить, что случилось?
Марвий нам толком ничего не рассказал...

--- Я настоятельно порекомендовала ему прекратить пьянство и насилие.
Но, видимо, привычки для твоего командира важнее контракта.

Митхэ закрыла глаза и молча осушила чашу.

--- Слушай, мы с ним давно это решили.
Есть его палатка, его кострище, он там жрец, воин и купец, и всё делается по его словам.
Игры, драки, всё прочее не выходит за границы его удела.

--- <<Драки и всё прочее>> я бы назвала <<избиение и унижение>>.

--- Будь по-твоему.
В любом случае: кто хочет --- приходит к нему, кто не хочет --- сидит поодаль.
\ml{$0$}
{Всем хорошо, все счастливы.}
{Everybody likes that, everybody's happy about that.''}

--- Во-первых, это нервирует караванщиков, --- буркнула Кхотлам.
--- Во-вторых, это бьёт по нашей репутации в городах.
\ml{$0$}
{В-третьих, тебе самой это не нравится.}
{Third, you are not happy about that.}
\ml{$0$}
{И правильно не нравится.}
{And you're right not to be.}
Храм --- это не рабский барак.

Эти слова пронзили Митхэ до самой глубины души.
Храм.
Она почти начала забывать это слово.

\asterism

--- Вы хотите сказать, что вот это было ради драгоценностей? --- Митхэ подняла рубаху, обнажив шрам на месте груди.
--- Сколько стоит мой шрам?
Сколько золота вы дадите мне за то, чего я лишилась?

Наёмники притихли.

--- Мы начали забывать, кто мы такие, --- продолжала Митхэ.
--- Кто-то из вас просто хотел лучшей жизни.
Кто-то хотел золота.
Кто-то вышел из Храма и ещё помнит слова клятвы.
Но мы все равны.
Все мы --- отряд чести, Бродячий Храм.
И я вам об этом напомню.

Митхэ вытащила из кармана плоскую коробочку и тремя ритуальными движениями нанесла краску.
Над толпой торговцев и наёмников прокатился вздох --- частью ошеломлённый, частью восхищённый.

--- Мы дали Кхотлам ар'Люм обещание и забрали её золото, --- сказала Митхэ.
--- Кхотлам предъявила разумные требования, Марвий принял их как оскорбление на свой счёт и решил впутать всех нас в свои личные обиды.
Те, кто считает, что он прав, пусть уходят с ним.
Для кого же обещание дороже золота --- идите за мной, и я покажу вам, чем Храм отличается от отряда наёмников.

Марвий захохотал.

--- Это было очень пафосно, Миция.
Но есть одна небольшая проблема.
Здесь не Храм.
Здесь люди, которые нуждаются в моём таланте и играют по правилам, которые написал я.

--- Больше нет, --- отрезала Митхэ.

Марвий выхватил саблю, вышел вперёд и вонзил её в землю.
Эфес Перидотовой Лианы вибрировал, словно крыло пчелы.

--- В Храме спор не решается боем, уно Марвий, --- холодно сказала Митхэ.

--- Здесь не Храм, --- ухмыльнулся Марвий.
--- Правила здесь пока что мои.
И согласно моим правилам, я сейчас умоюсь слезами этого Ягуара и сниму с него шкуру.
В следующий раз, Миция, если кто-то вздумает нести чушь про Храм, я просто укажу на лоскут лорики на моей груди.

--- Да будет так, --- Митхэ стянула рубаху и бросила на землю.

\asterism

Вдруг на поле между дуэлянтами выбежал молодой белый олень.
Он встал мордой к Марвию, расставил тонкие ноги и угрожающе опустил голову, увенчанную тремя ажурными золотистыми рогами.

--- Чёрточка, уведи Серебряного, --- быстро попросила Митхэ старого торговца с палкой, увидев, что Марвий с ухмылкой потянулся за саблей.

Старик кивнул и, проковыляв в центр круга, взял оленя под уздцы.
Марвий разочарованно хмыкнул.

У Митхэ внезапно заныло сердце и ослабли руки.
Олень волновался, не хотел уходить, взвизгивал и щёлкал языком.
Его белый, похожий на флажок хвост трепетал без остановки, его упирающиеся копытца оставляли в грязи длинные двойные полосы.
Старик еле-еле успокоил животное, закрыв шоры и сунув под острую мордочку дурманящих солеросов.

<<Всё будет хорошо, --- пыталась убедить себя Митхэ.
--- Умру --- найдёт себе хозяина получше>>.

Но ведь не найдёт же, твердил ей внутренний голос.
С Серебряным всё было по-другому, не так, как с другими.
Он сам подошел, сам встал под седло.
Он сам унес её, раненную, с поля боя, когда она потеряла грудь.
Сам бежал по нужному пути, пока она стояла на седле с луком и отстреливалась от погони.
Никакой дрессировки, чистое понимание.
Он не просто ездовое животное, она не просто всадник.
Они --- команда.
Они --- напарники.
Такие отношения не имеют цены.

Митхэ уже открыла рот, чтобы сказать старику об этом --- и тут же закрыла.
Сабля Марвия звякнула.
Бой начался.

\asterism

Вдруг из толпы вылетела стрела.
Митхэ едва успела отклониться;
стрела засела в боку.

--- Твой Храм втыкает в тебя булавки, Ягуар? --- ухмыльнулся Марвий, поигрывая саблей.

--- Булавки втыкают в твою честь, Юстиция, --- парировала Митхэ.

Митхэ видела, что Кхотлам занервничала.
Она беспомощно оглядывала ряды наёмников, ища поддержки.
Ситуация выходит из-под контроля --- из-под контроля Кхотлам, Марвия, самой Митхэ...
Ещё чуть-чуть --- и дело закончится свалкой и жертвами.
Кхотлам это видела так же ясно, как и Митхэ, но, в отличие от Митхэ, Кхотлам была бессильна перед угрожающей ей опасностью.

<<Заварила похлёбку, нюхай теперь, --- думала Митхэ с некоторым злорадством.
--- В следующий раз думать будешь.
Если этот следующий раз наступит.
Понять бы ещё, кто меня выцеливает...>>

Кхотлам видела нападавшего.
Митхэ проследила за её взглядом.
Конечно же, это Люэлам --- она не отличалась умом.

Наёмница в толпе тем временем наложила на тетиву вторую стрелу.
Митхэ напряглась, пытаясь одновременно следить за Марвием и Люэлам.
Но тут стрелу перехватила жилистая рука.
Лучница увидела перед собой спутанные золотистые волосы, ленточки, погремушки... и зелёные лезвия глаз под ними.

--- Не вмешивайся, --- шевельнулись тонкие губы.

--- А то что? --- осведомилась женщина.

Наёмник молча вонзил ей в шею гриф-клинок, с хрустом пробив позвонок.
Люэлам рухнула как подкошенная.
Наёмник встал на колено, набрал в рот воды из мехов и, прислонив свои губы к губам убитой, влил воду ей в рот.
Затем, приложив узловатую кисть к левому уху, что-то прошептал.
Любой сели узнал бы эти движения --- похоронный ритуал Ближнерек, притаившихся у отрогов Серебряных гор.
Но все были поглощены дуэлью Марвия и Митхэ, и на разыгравшуюся за кулисами сцену никто не обратил внимания.

\asterism

--- Что ты делаешь? --- просипел Марвий.

--- <<Оставь побеждённому жизнь, но забери его оружие>>.

\ml{$0$}
{--- <<...ибо путь Ягуара --- путь слёз>>, --- закончила Согхо.}
{```... for the path of Jaguar is the path of tears','' \Sogcho\ finished.}
В глазах молодой женщины светилось восхищение, пухлые, почти детские щёки раскраснелись.
Она вышла вперёд и встала на колено.
\ml{$0$}
{--- Меня не обучали в Храме, но я хочу принести воинскую клятву.}
{``I've never been taught in the Temple, but I want to swear the warrior's oath.}
\ml{$0$}
{Примешь ли ты её, Митхэ ар'Кахр?}
{Shalt thou take it, \Mitchoe\ ar'Kachr?''}

Митхэ убрала фалангу в ножны и кивнула.

\ml{$0$}
{--- Я тебя слушаю, Согхо.}
{``I hear thee, \Sogcho.''}

Большинство наёмников хмуро, скрестив руки на груди, наблюдали за происходящим.
Некоторые презрительно ухмылялись.
Кое-кто уходил в сторону повозок --- собирать вещи.
Но один за другим из толпы выходили люди и повторяли древние слова.
Их губы дрожали, они запинались и путали порядок фраз, но их глаза сияли, словно звёзды.
Последним подошел зеленоглазый наёмник со спутанными золотыми волосами.
Он помолчал десяток секхар, а затем проговорил:

\ml{$0$}
{--- Я не запомнил ни слова из клятвы.}
{``I couldn't remember any of those words.}
\ml{$0$}
{И кланяться не хочу.}
{And I won't bow.''}

--- Ты принят, Акхсар ар'Лотр, --- ответила Митхэ.
Наёмник кивнул и вернулся в строй.

Марвий поднялся с земли и, плюнув, пошёл по мощёной дороге к ближайшему городу.
\ml{$0$}
{За ним не последовал никто.}
{Nobody followed him.}
Перидотовая Лиана так и осталась лежать в дорожной пыли.

\asterism

Кхотлам улыбнулась --- за вечер Митхэ ни разу не рассталась с Перидотовой Лианой.
Кто-то из наёмников (теперь уже воинов) сказал, что малышка Митхэ давно положила на неё глаз.
Да и кто бы не положил на такую красоту...

--- Я собрала пятнадцать человек, --- отчиталась Митхэ.
--- Из них восемь --- это ветераны, остриё копья.
Но этого недостаточно, чтобы...

\ml{$0$}
{--- Достаточно.}
{``It is enough.''}

--- Кхотлам, правила предписывают: один обоз должны охранять как минимум...

\ml{$0$}
{--- Пятнадцати достаточно, Митхэ.}
{``Fifteen is enough, \Mitchoe.}
\ml{$0$}
{Они пошли за тобой не ради золота.}
{Not the gold made them follow you.}
\ml{$0$}
{А значит, и сражаться будут не ради золота.}
{Not the gold will make them fight.}
Правила, про которые ты говоришь, рассчитаны на наёмников, а не на Храм.

Митхэ церемонно поклонилась.

--- Я послала письмо в Миситр, что на Короне ещё один воин принял обет Плачущего Ягуара, --- сказала Кхотлам.
--- Думаю, ты можешь предстать перед Винтовым Храмом по прибытии.
Там тебя и твоё оружие очистят перед тем, как завершить обряд посвящения.
Наше соглашение с Марвием, конечно же, аннулируется, так что...

--- Я беру на себя все его обязательства, --- перебила Митхэ.
--- Это не обсуждается.
Сделай копии документов с поправкой на сегодня.
Мы выплатим тебе неустойку за беспокойство.

--- В этом точно нет никакой нужды, --- улыбнулась Кхотлам.
--- Обычные дорожные неурядицы, не привыкать.
Бывало и хуже.

--- Я не знаю, зачем я это сделала, --- Митхэ снова посмотрела на себя в зеркало и закрыла лицо рукой.
--- Такая ответственность...
Зачем это мне?
Почему я это сделала?

--- Может быть, не нужно искать ни цели, ни причин, --- сказала Кхотлам.
--- Это просто порыв твоей души, который невозможно остановить.
Он не пришёл из ниоткуда.
Он зрел внутри тебя, как семя внутри фрукта.
Ему пришло время прорасти.

--- Мне страшно, --- прошептала Митхэ.
Из её глаз покатились слёзы.
--- Мне так страшно, Кхотлам.

--- Это нормально, Золото.
Любому было бы страшно на твоём месте.

Митхэ хихикнула.
Кхотлам впервые назвала её домашним именем.
От молодой женщины веяло чем-то взрослым, нежным, ласковым.
Митхэ вдруг посетило детское желание подбежать и спрятать лицо у неё на груди, схватиться руками за длинные волосы, зарыться в её рубаху, назвать её кормилицей, которой у неё никогда не было...

--- Я пойду подготовлю отряд к дороге, --- Митхэ встала и поклонилась.
Невысказанное, непрочувствованное, неосуществлённое, но яростное как буря желание застряло у неё где-то в районе рваного шрама на месте груди.

\chapter{Костяная погремушка}

\section{Расчёска}

Кхотлам взяла гребень и начала расчёсывать спутанные золотые волосы Акхсара.
Она аккуратно вытаскивала костяные погремушки, ниточки, ленточки и складывала рядом.

--- Я потом вплету обратно, --- пояснила она на вопросительный взгляд воина.
--- Я знаю, что они для тебя много значат.

Он улыбнулся и молча кивнул.
Сердце Кхотлам подпрыгнуло.

--- Хочешь лечь мне на колени?

--- Угу.

Кхотлам гладила Акхсару голову, тонкий нос, тонкие губы, густые брови.
Вскоре ей показалось, что он заснул --- его дыхание стало ровным и глубоким.
Она не могла видеть, что из-под полуприкрытых век воина светилась всё та же настороженная кинжальная зелень, не угасая ни на секхар.

\asterism

--- И долго ты ей собираешься голову морочить? --- спросила Митхэ.

--- Я не морочу ей голову, командир, --- буркнул Акхсар.
--- Хочет расчёсывать --- пусть расчёсывает.

--- Помимо расчёсывания, она тебе таскает лучшие куски еды, показывает красивые листья, плетёт браслеты.
Ты же понимаешь, что это значит?

--- Я всё прекрасно понимаю, Митхэ.
Просто хочу, чтобы всё оставалось как есть.
Мне с ней хорошо и так.

--- Может, стоит ей об этом сказать?

--- Митхэ, --- Акхсар понизил голос.
--- Я примкнул к твоему Храму.
У меня были свои причины.
Но это не значит, что ты можешь лезть в мои дела.

Воин встал и пошёл к шатрам.

--- Бедная, бедная Кхотлам, --- грустно пробормотала Митхэ.
--- Повезло же тебе.

\asterism

--- Не надо, --- Акхсар мягко оттолкнул женщину.

--- Прости.
Ты мне очень нравишься, Снежок.

--- Кхотлам, ты очень милая, и я...
Нет, Меркхалон тебя раздери!

Кхотлам не сразу поняла, от чего она свалилась с бревна --- от ощутимого толчка в грудь или глухого рёва Акхсара.
Он вскочил, рассыпав пайки и опрокинув чашу.
Его рука успела инстинктивно двинуться к сабле, но вовремя остановилась;
Акхсар отряхнулся и начал лихорадочно завязывать рубаху на тугие узлы, перепутывая и стягивая шнурки.
В его глазах застыл детский испуг.

--- Нет, --- повторил он уже спокойнее.
--- Нет, нет, нет.
Нет.

--- Прости, --- Кхотлам встала и отряхнула платье.
--- Я не знала, что для тебя это...

--- Спрашивай разрешения в следующий раз.

--- Хорошо.
Можно я тебя расчешу?

--- Не сегодня, --- Акхсар шумно вдохнул воздух, выдохнул и подхватил с земли чашу.
--- Я спать.

--- Храни твой сон лесные духи, --- тихо сказала Кхотлам.
Ответом было молчание.

\chapter{Буря}

\section{Клятва пирата}

--- Поклянись, Ферро! --- рявкнула Кхотлам.
--- Дай клятву!

Ферро оскалил жёлтые зубы.
Вражеские корабли уже подняли чёрные боевые спинакеры с крабом, яростно растопырившим клешни.
Один жест --- и...

--- Клянусь, --- шёпотом произнёс он.

--- Голубой спинакер! --- крикнула Кхотлам.
Пираты сломя голову бросились выполнять приказ.

--- Дай мне лодку и пару людей, --- сказала Кхотлам.
Ферро нервно кивнул и махнул рукой.

\asterism

--- Что ты им сказала?

--- Неважно, --- ответила Кхотлам.
--- У тебя есть ещё декада на то, чтобы закончить свои дела, и Ризомера ждёт тебя в известной тебе лодочной мастерской на Острове Неудачников.
Но это, сам понимаешь, в последний раз.

--- Ничего себе, --- удивился Ферро.
--- Язык у тебя змеиный.

--- Помнишь, что ты мне обещал?

--- Помню, Кхотлам.
Я плеча не трону у членов твоей команды.
Они останутся живы и невредимы.

--- А ещё что ты обещал?

--- А я ещё что-то обещал?

--- Ферро!

--- Помолчи, Кхотлам.
Я всегда держу свои обещания.
Только море таит в себе множество опасностей, и некоторые из них понимают речь.
Вдруг ты понадобишься мне вновь?

--- Я не собираюсь больше тебе помогать.

--- Я не спрашивал, собираешься ты или нет.
Что касается обещания --- я освобожу тебя, твою команду и твоё судно, когда мы прибудем на Остров Невезучих.
Надеюсь, ты знаешь дорогу домой.

Пираты дружно захохотали шутке своего капитана.

\section{За смехом таится молчание}

--- Мне всегда нравились мудрые женщины, --- она схватила Кхотлам за волосы и притянула к себе.
Кхотлам не издала ни звука.
Сидела, закрыв глаза и сжав губы.

--- В чём дело?

\ml{$0$}
{--- Ты действительно хочешь испортить инструмент переговоров твоего капитана?}
{``Do you really want to spoil a negotiation tool of your Captain?''}

\ml{$0$}
{--- Я не испорчу, я сделаю лучше.}
{``I won't spoil, I'll make better.''}

\ml{$0$}
{--- Тогда продолжай.}
{``Go on, then.''}

Кхотлам ощутила, что хватка ослабла.
Наконец пиратка отпустила её волосы.

--- Молодец, --- холодно сказала Кхотлам, открыв глаза.
--- Пока это будет моей тайной.
Но это тут же перестанет быть тайной, если ты ещё раз тронешь меня или кого-то из моей команды.
И поверь, Люка Агнозис, я об этом узнаю.

Пиратка прошипела что-то оскорбительное и вышла за дверь.

\section{В молчании зреет бунт}

--- Тебе нравится капитан? --- спросила Кхотлам.

Пират продолжал молча драить пол каюты.

--- Я к тебе обращаюсь, Лепесток-Пустынной-Вишни!

--- Закрой рот! --- прорычал пират.
--- Не смей произносить это имя!

--- Тебе нравится капитан, который нарушает слово? --- продолжала Кхотлам.
--- Ответь мне, Лепесток-Пустынной-Вишни.
Нравится ли тебе капитан, который может урезать твою долю или выгнать тебя на берег, нарушая свои клятвы и договоры?

Пират молча треснул ей по губам грязной губкой.

--- Он и тебя предаст, если ему будет удобно.

--- Поплавай, рыбина, --- пират схватил ведро и вылил воду на лежанку Кхотлам.
--- Может, и молчать научишься.

\asterism

--- Вот что с тобой делать, а, --- капитан вздохнул и тронул пальцами цепи Кхотлам.
--- Рот тебе завязать твой грязный или как?

--- С чего бы был грязным мой рот?
Его не осквернили ложь или клятвопреступление.

\ml{$0$}
{--- Кхотлам, если ты ещё будешь подстрекать мою команду, я пущу тебя по кругу.}
{``\Kchotlam, if you provoke my crew any more, I will have you screwed.''}

\ml{$0$}
{--- Ты уже пустил по кругу свою клятву.}
{``You have already had your oath screwed.}
\ml{$0$}
{Пустишь и свою честь.}
{So your honor you will.''}

--- С этого дня мыть твою каюту буду я.
Раз в декаду.
И тебе это не понравится, поверь мне.

\section{Шёпот духов}

На судне стояла тишина.
Было слышно, как плещутся за бортом мелкие рыбёшки, объедая с бортов моллюсков и водоросли.

Кхотлам закрыла глаза и набрала в грудь воздух.
По судну разнеслось тихое шипение.
Оно проникало повсюду, пузырьками лопалось на поверхности ночного моря.

--- Клятвопрессссступник...
Проклятая душшшшша...
Ссссудно пуссссстить на дно...
Дни сссссочтены...

Кхотлам остановилась и тихо прочистила горло.

--- Клятвопресссступник...

За дверью послышались голоса.
Кхотлам тут же легла и накрылась одеялом.

--- Капитан, вы слышали это?

--- Слышал что? --- осведомился заспанный голос Ферро.

--- Голос.

--- Это пленники, идиотка.
Ты забыла, что у нас полный трюм товара?

--- Но это другой голос, капитан!
Он какой-то... как будто из джунглевых жил.

Дверь каюты распахнулась.
Слабый свет фонаря выхватил мирно спящую Кхотлам.

--- Идиотка, --- фыркнул Ферро.
--- Перепила --- твоё дело, но держи при себе своих розовых дельфинов!

Вскоре суета улеглась.
Кхотлам дождалась склянок и снова приникла к стальной банке:

--- Клятвопрессссступник...
Проклятая душшшшша...

\section{Уговор}

Ферро спустился в трюм через пару михнет и молча, гремя ключами, снял кандалы с Кхотлам.
У него ощутимо дёргался глаз.

--- Я хочу, чтобы ты знала --- ты ошиблась, --- сказал пират.
--- У меня есть честь.
Даже у меня.
Ты и твои моряки свободны, как мы и договорились.
Судно я вам возвращаю, как мы и договорились.

--- Как к этому отнесутся твои люди?

--- Мои люди в курсе, что по долгам платить необходимо.
Тем более, о товаре речи не шло, поэтому товар остаётся у нас.

--- Благодарю тебя, Ферро.

--- Попутного ветра, --- пират кивнул и поднялся наверх, стуча твёрдыми каблуками.

\section{Свобода}

Кхотлам улыбалась и молчала.
Моряки тоже улыбались и молчали.
Она знала, что многие уже успели мысленно распрощаться со свободой и надеждой увидеть близких.
И тут --- снова под ногами твёрдая палуба, снова в руках верный руль...

--- Что прикажешь, Кхотлам? --- подошла наконец старая темнокожая шкиперша.

--- Домой, Костелла, --- коротко сказала купчица.
Команда молча принялась за дело.

<<Нет, --- думала Кхотлам.
--- Хватит с меня такой жизни.
Все эти морские опасности, пиратство, предательские ветра, коварные воды...
Я же никогда не любила всю эту непредсказуемость.
Никогда.
Отныне только на берегу.
Отобью команде неустойку за беспокойства --- и осяду где-нибудь>>.

Взглянув на шкипершу, Кхотлам поняла, что та, всё это время следила за ней.

--- Знакомое лицо, --- сказала шкиперша.
--- У меня такое бывало в молодости после каждого шторма, после каждого блуждания в тумане.
Оно означает: <<В бездну всё это, ухожу на берег навсегда>>.

Кхотлам захохотала от всей души впервые за долгое время, и ответом ей были дружный хохот команды и крики альбатросов.

\chapter{Песок и ракушки}

\section{Перцовка}

--- Живая, --- пробормотал Хитрам.
--- Точно живая.
Метвецы так не дерутся...

--- Думать надо, --- буркнула Кхотлам, скрестив руки на груди.

--- А я что, по-твоему, делал?
Ты несколько кхамит без дыхания лежала.
От глубокого морского сна только это и помогает.

Кхотлам промолчала.

--- Где я, для начала? --- наконец спросила она.

--- А ты как думаешь?

--- На Короне или на Ките?

--- Ааа, --- протянул Хитрам.
--- Всё настолько плохо?
Это Корона.
На юго-запад --- Валенсия.

--- Меня выбросило к Валенсии? --- ужаснулась Кхотлам.

--- Как видишь.

--- Сколько же я тогда без сознания-то была, --- пробормотала женщина.
--- А моя команда?

--- Больше никто не очнулся.
Сочувствую.

--- Я должна их осмотреть.

Кхотлам попыталась встать и тут же села, вскрикнув и схватившись за лодыжку.

--- У тебя обе ноги покалечены, куда ты собралась? --- Хитрам подскочил и поправил шину.
--- Похоронил я их уже.

--- Как --- уже? --- Кхотлам удивлённо оглядела берег и, увидев свежие могилы, зажала рот рукой.

Хитрам смутился.

--- Да я бы и тебя давно похоронил, только сомнения были.

--- Так зачем ты их закопал?
Может, и они живы!..

Кхотлам проползла на руках до ближайшей могилы и начала с остервенением раскапывать песок.

--- Хай, женщина, --- Хитрам попылался перехватить её руки.
--- Да подожди ты.
Женщина!
Да стой!
Они мертвы, хватит!

Кхотлам остановилась и обратила к Хитраму лицо в слезах.
Тот без сил опустился рядом.

--- Я нашёл вас ранним утром, --- объяснил он.
--- Сейчас уже к вечеру клонится.
На всех, кроме тебя и ещё двоих, были следы тления.
Перцовки я дал всем, очнулась только ты.
Те двое вон там лежат, как и лежали.

Кхотлам легла ничком в песок и зарыдала во весь голос.

\asterism

--- И долго мы тут сидеть будем? --- буркнула Кхотлам.
--- Если у тебя нет сил меня нести --- я поползу.
Только скажи, куда.

--- Я всегда возвращаюсь к вечеру, --- пояснил Хитрам.
--- Сёстры это знают.
Если я не вернусь --- меня пойдут искать.

--- Голова болит?

--- Если бы не болела --- я бы тебя понёс.

--- Какие мы сердитые.

--- От тебя одни проблемы, --- прямо сказал Хитрам.
--- Ноги заживут --- иди к свиньям.

--- А если не заживут?

--- Тогда ползи к свиньям.
Ладно, ладно, не плачь, это я зря сказал...

Вдали показались два бумажных фонаря.
Вскоре из темноты вынырнули и их обладательницы --- две женщины, по обычаю местных ноа закутанные в плащи на голое тело.
Лица женщин скрывались под круглыми травяными шляпами;
в руках они держали зонты-копья.

--- Хитрам, мне это надоело, --- без предисловий обратилась старшая к брату.
--- Это четвёртый корабль за дождь.
Все рыбаки приносят домой рыбу, и только ты приносишь домой трупы моряков.

--- В этот раз одна живая, Огонёк, --- весело откликнулся Хитрам.

--- И толку с неё?
В котелке не сваришь и на рынке не продашь.
Всё, со следующей декады идёшь на рынок, там от тебя больше пользы.

--- Братик, тебя ранили? --- вторая подбежала и стала ощупывать Хитраму голову.
--- Огонёк, у него голова разбита!

--- Была бы у него голова цела, нас бы здесь не было на ночь глядя.
Курочка, иди нарежь веток для носилок...

--- Для двоих?

--- Для одного, глупая!
Этот дурак пешком пойдёт, заслужил!

--- Ничего нового, --- поделился Хитрам с Кхотлам.
Та сконфуженно промолчала.

Впрочем, угроза оказалась пустой.
Когда носилки были готовы и Кхотлам устроилась на них, Огонёк без лишних слов взвалила брата на плечи и потащила домой.

\asterism

--- У тебя волосы пахнут морем, --- прошептал Хитрам.
--- А ещё в них песок.
И у меня на лежанке теперь тоже песок...

--- У тебя на лежанке и был песок! --- тихо возмутилась Кхотлам.

\ml{$0$}
{--- Нет, не было, пока ты не приползла!}
{``No, there wasn't before you have crawled here!}
\ml{$0$}
{Кто вообще разрешил тебе ползать?}
{Who gave you permission to crawl?}
\ml{$0$}
{У тебя ножки больные!}
{You've got sore legs!''}

\ml{$0$}
{--- Я не спрашиваю чужого разрешения, если хочу поползать!}
{``I never ask anyone's permission if I want to crawl!''}

--- Кхотлам, мы не уснём тут вдвоём.
\ml{$0$}
{Лежанка чересчур маленькая.}
{The lounger is too small.''}

--- Притащи мою и положи рядом, тогда всё будет в порядке!

Хитрам вздохнул и отправился за лежанкой Кхотлам.
Сёстры мужественно притворялись спящими.

--- Надо было тебя помыть, что ли, --- буркнул мужчина, разглаживая простыни.
--- Всё в песке...
Ай!
Это что, ракушка?
Кхотлам, лесные духи!
Ракушек мне только и не хватало в постели!

--- Спи уже.
Обними меня и спи.
Только ноги осторожно.

Хитрам молча обнял женщину и прижал к себе.

--- Ты тоже это чувствуешь, да? --- поинтересовалась Кхотлам.

--- Что мы как будто знаем друг друга с рождения? --- хмыкнул Хитрам.
--- Сразу почувствовал.
Как будто всё это время я высматривал твой корабль, а сейчас моё ожидание подошло к концу.
Если честно, думал, что все эти мысли из-за удара по голове...

--- Но ты-то меня по голове не бил!

--- Тебя море побило.

--- Разумное объяснение, --- признала Кхотлам и устроилась поудобнее, уткнувшись в Хитрама носом.

\section{И снова платки}

--- Слушай, Кхотлам, --- промямлил Ситрис, --- я тут это...

Кхотлам оторвалась от записей.

--- Что такое, Ситрис?

--- Когда ты, хай, сказала <<тысячу платков>>, я всё-таки думал, что это фигура речи...

--- Нет, это не была фигура речи.
Давай-давай, вышивай, не отвлекайся.
У тебя получается гораздо лучше, чем вчера.

--- Я иглу в руках держу второй раз в жизни!

--- А завтра будет третий.
Вышивай.

Ситрис угрюмо ковырнул иглой ткань и вдруг бросил хитрый взгляд на хозяйку жилища.

--- Кстати, ты обещала, что введёшь меня в Храм.

\ml{$0$}
{--- Я тебе ничего не обещала, Ситрис.}
{``I promised nothing, \Sitris.}
\ml{$0$}
{Тебе, наверное, приснилось.}
{Maybe you dreamt it.''}

--- Обещала, точно обещала.

--- Я сказала <<Посмотрим>>.
Посмотрим --- это не обещание.

--- Ты сказала <<Посмотрим, и может быть>>...

--- Ситрис, --- Кхотлам отодвинула в сторону чернильницу немного резче, чем следовало, --- тебе мало того, что я наложила вето на решение Советов и сделала Тхитрон единственным местом, где ты можешь жить без перспективы быть принесённым в жертву?

--- Я тебе очень-очень благодарен, --- быстро и испуганно протараторил Ситрис.

Увидев отголоски ужаса в чёрных как уголь глазах, Кхотлам смягчилась.

--- Ты здесь не раб, --- сказала она.
--- Не хочешь вышивать --- не вышивай.

--- <<Но тогда никакого Храма>>, --- вкрадчиво закончил за неё Ситрис.

\ml{$0$}
{--- Я этого не говорила.}
{``I've never told that.''}

\ml{$0$}
{--- Но ты ведь имела это в виду, верно?}
{``But you meant that, didn't you?''}

\ml{$0$}
{--- Хватит говорить мне, что я имела в виду.}
{``Stop telling me what I meant.''}

\ml{$0$}
{--- Разве это не очевидно?}
{``Isn't that obvious?}
\ml{$0$}
{Ты --- мне, я --- тебе.}
{It's give and take.''}

Кхотлам вздохнула и встала, торжественно сложив руки.

--- Я клянусь, что порекомендую тебя Храму вне зависимости от того, вышьешь ты платки или нет.
Также я освобождаю тебя от любых моральных обязательств, в свете которых вышивка платков может считаться возвращением морального долга.

Кхотлам села на циновку, задумалась и обмакнула перо в чернила.
Ситрис с несчастным видом посмотрел на стопку хлопковых лоскутков.

--- Ну и рыбина же ты, Кхотлам, --- буркнул он.
--- Помру я с этими платками.

--- Весьма неплохая смерть, --- ухмыльнулась женщина.

Ситрис вздохнул и снова принялся за работу.

\chapter{Комната без окон}

\section{Последнее желание}

Увидев Хитрама, Кхотлам зарыдала впервые с момента, когда её завели в тёмную комнату без окон.
Мужчина бросился к ней.

--- Кхотлам, я клянусь --- я найду способ тебя вытащить.
Не падай духом.

--- Нет, --- всхлипнула Кхотлам.
--- Если с тобой что-то случится, Лисёнок и Хмурчик останутся совсем одни.
Хмурчик только-только пришёл в Храм, а Лисёнок совсем сосунок.
Поклянись, что ты не будешь меня спасать.

--- Ты с ума сошла?
Я не могу пообещать такое.

--- Хитрам! --- сказала купец.
--- Это моё последнее желание!
Ты откажешь мне?

Хитрам долго смотрел в полные слёз зелёные глаза.

--- Хорошо, --- наконец сказал он.
--- Я клянусь, что не буду тебя спасать.
Но если кто-то захочет тебя вытащить, я помогу чем смогу.

--- Если тебя поймают, я тебя не прощу никогда.
Ты слышишь?
А теперь иди отсюда.

Кхотлам оттолкнула Хитрама.

--- Пёрышко, жизнь моя...

--- Уйди к свиньям, Пловец!
Мне даже смотреть на тебя больно!
Уйди!

Хитрам едва успел увернуться от тяжёлой деревянной скамеечки.
Скамеечка глухо треснула о каменную стену.

--- Я тебя не оставлю, --- тихо сказал Хитрам и вышел из комнаты.

\section{Молоко}

За дверью была едва слышная перебранка.

--- Возьми это золото.
Тебе лишнее, что ли?

--- Я не возьму.
Плевать на обычай.
Если хотите, можете его выкинуть.
Всё, что мне нужно --- это побыть с ней.

--- Хитрам, хватит.
И возьми пирамидку этого дерьма.
Я не могу оставить её себе, не могу выкинуть, ты должен её взять!

--- Я просто посижу с ней, пожалуйста, --- убеждал Хитрам.
--- Какой это может причинить вред?

--- Хитрам, ты знаешь правила, --- устало говорил Первый.
--- Никаких посиделок.
Я вам свидание устроил просто из уважения к Кхотлам, хотя должен был отказать.

--- Я никуда не уйду.
Да не нужна мне эта пирамидка!

Что-то громко звякнуло о стену.
Похоже, Хитрам швырнул тяжёлый кусочек золота подальше.

--- Всё, я взял, доволен?
А теперь дай мне побыть с ней.
Если надо, я буду сидеть перед дверью.

--- Не будешь!
Если ты не прекратишь бесчинствовать, я велю Кхохо тебя вывести!
Она тебя жалеть не будет, ты её знаешь!

--- Пусть попробует.

--- Кхохо!
Иди-ка сюда.
Проводи городского на выход.
И скажи остальным, что на ближайший дождь он отлучён от здания храма, за исключением лазарета.
Если будет сопротивляться --- обеспечь ему лазарет.

--- Слушай, парень, не дури, --- проворчала Кхохо.
--- Я ведь тебе правда сейчас руки переломаю.
Давай, давай, не зли меня, у меня плохой день сегодня.

--- У тебя всю жизнь плохой день, --- буркнул в ответ Хитрам, но, судя по удаляющимся шагам, всё-таки послушался.
Кхотлам выдохнула.

\asterism

Сейчас, когда ей принесли новую лампу, Кхотлам разглядела каморку.
Над лежанкой на уровне головы были круглые пятна, протёртые сотнями затылков.
Те, что пониже, принадлежали идолам и человеческим детям, что повыше --- взрослым людям.
Пол и стены были исчерчены тысячами посланий.
Рисунки, имена любимых, собственные имена, молитвы, проклятия, исповеди.

<<Передайте Митхэ, что я её люблю>>.

Цветочек.
Бабочка.
Неуклюже расставивший руки человечек.

<<Будь ты проклят, предатель>>.

(неразборчивая надпись на абисе)

<<Невезение чистой воды>>.
Классический цатрон, тоновая поэтика, изящный саркастический слог, незнакомое, но явно каллиграфическое начертание, выдающее образованную персону.
Черты очень глубокие, сами иероглифы крупные, и... это что, когти?
Точно ркхве-хор.
Как его только занесло так далеко на восток...

Какая-то молитва змеистым письмом на хесетроне --- скорее всего, идол.

<<Я невиновен>>.
При виде этого послания у Кхотлам ёкнуло сердце, и она прекратила разглядывать надписи.

Болела набухшая грудь.
Кхотлам тёрла её, но боль не проходила.
<<Молоко пропадёт, --- с грустью думала она.
--- Я ведь даже выкормить не успела ребёнка.
Что я скажу Митхэ, когда встречу её \emph{там}?
Наобещала и не выполнила>>.

\section{Три мучительных декады}

Кхотлам сидела, повернувшись лицом в угол и прислонившись лбом к стене.
Она сильно осунулась --- глаза запали, нос заострился, волосы поредели, в них прибавилось седины.
Трукхвал с грустью посмотрел на нетронутый поднос с едой.

--- Кхотлам, поешь, --- робко сказал он.
--- Мидии --- это то немногое, что я действительно умею готовить.
Хочешь, принесу свежие?

--- Когда меня принесут в жертву? --- глухо спросила женщина.

--- Полагаю, когда будет нужно.

--- Трукхвал, я уже три декады сижу здесь.

--- Подожди, --- нахмурился библиотекарь.
--- Как это?

--- А вот так.
Мимо меня прошло пятеро детей.
Каждого я утешала, укладывала спать и провожала на крышу.
\ml{$0$}
{Я больше так не могу.}
{I can't take it anymore.}
\ml{$0$}
{С меня хватит.}
{I have drank sick.}
Если вы не принесете меня в жертву в ближайшее время, я перережу себе жилы.

--- Это твоё право, --- сказал Трукхвал.
--- Как ты знаешь, мы...

--- Вы серьёзно ждёте, пока я покончу с собой? --- Кхотлам обратила страшные, запавшие, с синяками глаза к жрецу.
Тот вздрогнул.
--- Это ваш план, как не марать руки в моей крови?

--- Кхотлам, я не совсем в курсе ситуации, но я тебя понял.
Обещаю, я поговорю со жрецами и узнаю, что случилось.
Если никто не вызовется, в следующий знак кирпича я сам поведу тебя на крышу.
Поешь хоть немного, это правда вкусно.

--- Убери поднос, пожалуйста.
Меня тошнит от запаха мидий.

Трукхвал поднял поднос и задумчиво слопал пару жареных моллюсков.

\ml{$0$}
{--- Ты же ведь невиновна, да?}
{``You're innocent, aren't you?''}

\ml{$0$}
{--- Какая теперь разница?}
{``Who cares?''}

\ml{$0$}
{--- Для твоего мужчины, видимо, есть разница.}
{``Your man cares, apparently.}
Он уже ввязался в три драки из-за того, что кто-то назвал тебя Разрушителем.
Пока что отделался сломанным носом и десятком синяков.
Крестьяне судачат, что он ненормальный и что ты его приворожила.
Я видел его сегодня на поле.
Он кормил грудью твоего ребёнка в перерывах между пахотой.
Если честно, немного тебе завидую --- я так и не встретил человека, который любил бы меня так же.

\ml{$0$}
{--- Он с тобой говорил, да?}
{``He talked to you, didn't he?''}

\ml{$0$}
{--- Хай, он спросил, не могу ли я тебе как-то помочь.}
{``Well, he asked me if I can help you somehow.}
\ml{$0$}
{Похоже, он у всех прохожих это спрашивает.}
{Looks like he asks everyone he sees.''}

Кхотлам промолчала.

--- Ситрис сказал, что тебя, скорее всего, подставили.
Твои домашние в один голос твердят, что ты никогда бы этого не сделала.

Молчание.

--- Кхотлам, я понимаю, что ты меня почти не знаешь, --- проникновенно сказал Трукхвал.
--- Я простой жрец.
<<Я не оставлю без внимания ни ложь, ни заблуждение, и скажу во всеуслышание то, что считаю истиной>>.
Я здесь не ради тебя, я здесь ради истины.
Просто посмотри мне в глаза и ответь --- ты действительно сделала то, за что тебя осудили?

--- Ищи истину в другом месте, жрец, --- ответила Кхотлам.

Трукхвал поклонился и вышел за дверь.

\section{Охотничий заговор}

--- Ты чего такой радостный, Трукхвал?

--- Я это, хай, давно днём по лесу не гулял, --- смущённо сказал жрец.

--- Ты вообще из храма выходишь?

--- Вечерком, на храмовые земли.
Больше люблю на крыше сидеть с чашей отвара или чего-нибудь покрепче.

--- Ученика тебе надо, старик.
Ученик тебя будет гонять по всем окрестностям, пока жрецом станет.

--- Да какой из меня учитель.
Загублю только детскую душу...
Так, а вот, кажется, и тот самый поворот.
Сиртху-лехэ, следы есть?

--- Хаяй, давненько я этим не занимался, --- весело проскрипел Сиртху-лехэ, присев на корточки.
--- Отобью старое мяско...
Да, следы повозок и ног.
Мы пришли.

Сиртху-лехэ вытащил из мочажины немного грязи, смешал с содержимым мешочка.
Затем несколькими широкими мазками нанёс смесь на ступни, ладони, подмышки, шею, лицо.
Протянув мешочек Трукхвалу и Ситрису, знаком велел сделать то же самое.
Когда все натёрлись, Сиртху опустился на корточки.

--- Как от снега белого ни духу ни запаха, как от льда горного ни следа ни знака, так и от меня и от спутников моих ни духа ни запаха, ни следа ни знака, от порога и до порога.
Услышь, Лю, узри, Сат, ощути, Сан, унюхай, Митр, Сиртху в дорогу собрался.

Ситрис бросил взгляд на Трукхвала.
Тот с интересом прислушивался к охотничьему заговору.

--- Гуляй чаще, Трукхвал, и не такое увидишь, --- весело шепнул воин.

\section{Следопыт}

--- Я тоже вижу, --- кивнул Ситрис.
--- Повозка лежала вот здесь.

--- А ещё здесь было восемь человек, --- Сиртху поводил руками по земле, нежно раздвигая пальцами молодую поросль.
--- Двое погибли, их тела оттащили в сторону... хэ, один погибший потерял шнурок со штанов.

--- Давай мне, я попробую определить поселение, в котором были пошиты штаны, --- сказал Трукхвал.

--- Тебе не понадобится шнурок, жрец.
Вон там, в двадцати шагах, свежая могила.
Думаю, в ней лежат целые штаны вместе с их владельцем.
Пошли, поможете мне раскопать.

--- Как ты это делаешь, лехэ? --- восхитился Ситрис.
--- Я бы ни за что не увидел здесь могилу.
Научи-ка.

--- Ситрис, давай потом, --- буркнул Трукхвал.
--- Времени у нас в обрез.
Бери мотыгу и копай.

--- Только осторожно, --- предупредил Сиртху.
--- Старайтесь не повредить тело, оно многое может рассказать...

\section{Сообщники}

Ситрис разложил на траве найденное в карманах.
Трукхвал молниеносно зарисовывал на пергаменте детали одежды, записывал приметы полуразложившихся трупов.
Это были две женщины.
Сиртху задумчиво разглядывал их.

--- Кто-то аккуратно положил их в обнимку, связав руки пеньковой верёвкой, --- сказал он.
--- Так хоронят любовников в Ближнереках.
Кстати, длина верёвки может говорить о том, сколько они были вместе.
Хаяй...
Больше десяти пядей.
Парочка со стажем.

--- Ты уверен? --- перо Трукхвала зашевелилось с новой силой.

--- Они точно любовницы, --- поддержал Ситрис.
--- Они сражались спиной к спине.
Посмотрите на эти обломки стрел.
Они уже запачкались, но до сих пор видно, что обломки совпадают.
Арбалетную стрелу пустили с дерева.
Той, что пониже, пробило горло, а той, что повыше --- сердце.

--- Похоже на то, --- согласился Сиртху, пощупав пальцем стрелы.
--- Сколько живу, такое вижу впервые.
Глаза у тебя хорошие, воин.

--- Записал, --- ответил Трукхвал.
--- Можете ещё что-то сказать?

--- Больше ничего, --- сказал Сиртху.
--- Татуировок, приметных знаков на коже не вижу.
Украшения и одежда, на мой взгляд, совершенно обычные для тхитронцев.

--- Сражались спиной к спине и были убиты с дерева --- то есть это не нападавшие, --- Трукхвал начал загибать пальцы.
--- А вот похоронил их один из нападавших, потому что караванщики разбежались.
Он родом из Ихслантхара и хорошо знал убитых.
Я ничего не забыл?

--- Всё на месте.
Ещё могу сказать, что он очень сильный --- я бы не смог затянуть толстую пеньковую верёвку в такой тугой узел.

--- Нападавшим помогал кто-то из городских, --- заключил Трукхвал, смотал пергамент в трубочку и засунул в тубус.

--- Всё-таки Кхотлам? --- грустно спросил Ситрис.

--- Если и так, то действовала она не одна.
Это явно что-то новенькое для расследования.
Выйдем на этих людей --- узнаем больше.
Сиртху-лехэ, прикопай этих бедняжек и поставь метку на дереве, потом попробуем найти близких.
А ты, Ситрис, забери их вещи, только не потеряй, я сделал опись.
И давайте ещё пошаримся по окрестностям, хотя бы до полудня.
Чем больше деталей --- тем лучше.

\section{Лесорубная}

--- Здравствуй, --- приветливо начал Трукхвал.
--- У меня есть к тебе пара вопросов насчёт...

Ситрис едва успел прикрыть Трукхвала собой.
Топор женщины вонзился ему в наруч чуть пониже локтя, промяв дерево, толстую кожу и металлическое кольцо.
Хлынула кровь.
Нападавшая тем временем, петляя как заяц, бросилась в сторону реки.

--- Оставь меня, Трукхвал, я не умираю, --- прохрипел Ситрис, кривясь от боли.
\ml{$0$}
{--- Беги, она нужна нам живой!}
{``Chase her, we need her alive!}
\ml{$0$}
{Да подожди, куда побежал, оружие возьми!}
{Hold on, not that fast, take the weapon!}
Гений тактики, идолы тебя дери...

Трукхвал подхватил фалангу Ситриса и бросился следом, подгоняемый отборными воинскими фразеологизмами.

--- Так, эта дорога ведёт на улицу Горьких Конфет, туда она не побежит, --- бормотал Трукхвал, скидывая на бегу неудобную робу и жреческие сумки.
--- Надо перехватить её в районе Безымянного переулка...
Лесные духи, зачем я вообще в это ввязался.
Истина для него главное, видите ли.
Сидел бы себе в библиотеке и сидел...

Едва выбежав на Безымянный переулок, Трукхвал нос к носу столкнулся с нападавшей.
Только сейчас он понял, как выдохся от непривычной нагрузки.
Жрец не имел ни малейшего понятия, как сражаться в таком состоянии.
Акула Ситриса к тому же оказалась чересчур тяжела для его рук;
Трукхвал торопливо вытряхнул на землю балансир, что не особо улучшило ситуацию.

--- Бросай оружие! --- крикнул он, вскинув двумя руками Акулу.
Из окон высовывались головы;
местные жители во все глаза наблюдали за разворачивающимся действом.
Трукхвал уже пожалел, что снял жреческое одеяние.
Как жреца его бы немедленно защитили, но библиотекаря на такой глухой окраине Тхитрона мало кто знал в лицо.

Нападавшая без предисловий махнула топором.
Трукхвал, едва успев парировать удар, отлетел и перекувырнулся два раза.
Промелькнула мысль, что будь на месте Акулы любая другая фаланга --- удар бы просто её сломал.
Кое-как встав на ноги, Трукхвал снова принял боевую позицию.

--- Повоять не буу! --- выкрикнул жрец, едва шевеля разбитыми губами.

На него обрушилась ещё серия ударов.
Но на этот раз библиотекарь немного успокоился --- открылось второе дыхание, тело само вспомнило давно забытые движения.
Нападавшая не отличалась разнообразием приёмов, полагаясь на скорость и силу, а Трукхвал, при всей его неуклюжести, неплохо умел фехтовать и отклонять удары --- это в какой-то степени уравняло шансы.
Наблюдающие за сражением крестьяне, судя по всему, начали делать ставки и болеть за своих фаворитов --- старшина, улыбаясь, подходил к окнам и целыми горстями собирал конфеты, гвозди, проволоку, засахаренных пчёл...
Пару раз Трукхвал выкрикнул <<Кхар>> --- и получил в ответ лишь смех и ободряющие возгласы.

<<Ну и деревоголовый народ живёт в Кустодралах, --- удивлённо думал Трукхвал.
--- Нижний этаж же всех учит с самого детства --- сигналы зазря не подаём и не игнорируем.
Эти же просто не воспринимают сражение всерьёз.
Сообразят только когда кто-то из нас получит серьёзную травму.
А как раз этого и нельзя допустить>>.

Скольжение, лёд, змея, водяная мельница, уход кроликом от удара напролом...
Трукхвал уже не видел и не слышал ничего, кроме противницы, полностью сосредоточившись на поединке.
Акула вдруг стала лёгкой, как пёрышко, и Трукхвал по привычке встал боком, убрав левую руку за спину --- классическая школа Плюща для поединков.
<<Главное --- держать дистанцию, --- думал он.
--- В плотном контакте она меня сомнёт.
Пока между нами два шага, моих умений хватает для защиты.
Уход только перпендикулярно, никаких острых углов.
Против солнца не стоять.
Отступаем, но не позволяем бежать, уходим, но перекрываем дорогу.
Она шаг --- и я шаг.
И ещё раз...>>

Вдруг нападавшая сначала зарычала, а потом, упав на колени, взревела от боли;
из каждого её бедра торчало по стреле.
Сиртху-лехэ, наложив на тетиву третью стрелу, осторожно приблизился к женщине и ударом ноги повалил её в дорожную пыль.

--- Помогите её связать, любезные, что смотрите, --- обратился он к зрителям.
--- Спектакль окончен.

\section{Танец}

Трукхвал сидел в госпитале и бинтовал Ситриса на старой бочке.
Рядом лежала хмурая нападавшая с перевязанными бёдрами, скрестив руки на груди.
Её наскоро прикрутили цепями к скобе для факела.

--- Придётся наложить гипс, --- сказал Трукхвал Ситрису.
--- У тебя трещина локтевой кости и вывих в суставе.
Сухожилия не повреждены.

Сиртху-лехэ держал руку Ситриса, наблюдая за Трукхвалом.
В его глазах росло уважение.

--- Красиво танцуешь, Трукхвал, --- проскрипел он.
--- Я аж засмотрелся.
Даже прерывать не хотелось...

--- Он умеет, --- хмыкнул Ситрис.
--- Ещё бы тренировался почаще, а то я его по нескольку дождей на тренировочной площадке не вижу...

--- Горожане проигнорировали мой сигнал два раза, --- грустно сообщил Трукхвал.

\ml{$0$}
{--- И опять Кустодралы, --- кивнул Ситрис.}
{``Bushscrapes again,'' \Sitris\ nodded.}
--- Народ там непонятливый, хоть похлёбку в ухо лей, хоть глаз топором чеши.
Кхотлам, к слову, с ними как-то договаривалась, даже без криков и угроз...

--- А почему ты сам эту громилу не подрезал, Трукхвал? --- спросил Сиртху, кивнув на женщину.
--- Ты мог, это без сомнений.

--- Опасался сокращать дистанцию, --- признался Трукхвал.
--- Не было уверенности в успехе.
Ситрис прав --- практики у меня маловато.

--- В этот раз он всё правильно сделал, --- поддержал Ситрис.
\ml{$0$}
{--- В меру осторожно, в меру эффективно.}
{``Enough of caution, enough of efficiency.}
\ml{$0$}
{Претензий нет.}
{I wouldn't reproach him.''}

\ml{$0$}
{--- А у меня есть.}
{``But I would.''}

Сиртху и Трукхвал вздрогнули, и Ситрис зашипел от боли.
В дверях госпиталя стоял Первый жрец.

\ml{$0$}
{--- Кто-нибудь, ммм, собирается объяснить мне, что за морковка тут происходит?}
{``Is anybody, hmm, gonna tell me, what the carrot is happening here?''}

\section{Разнос}

--- Вы понимаете, что вы натворили?

Жрецы стояли, смотря в пол.
Первый жрец не повышал голос, не обвинял --- но в воздухе повисло тягостное чувство.

--- Даже если бы вы дали ей сбежать, это была бы не настолько сложная ситуация.
Закон есть закон, но в глубине души все солидарны: кутрап сбежал --- туда и дорога, руки марать не надо.

--- Мы ей предлагали, но она отказалась! --- развёл руками Ликхсар.
\ml{$0$}
{--- Кхотлам выжила из ума!}
{``\Kchotlam\ is out of her damn mind!''}

--- Я не буду приносить её в жертву, --- напрямик сказал Хонхо.
\ml{$0$}
{--- Двор Люм за пять дождей сделал то, чего не удавалось другим за сотню.}
{``The House of \Loem\ have done in five rains more than other wouldn't have in a hundred.}
\ml{$0$}
{Я не верю в виновность этого человека, и не поверю.}
{I do not believe that person is guilty, and I will not.}
Если надо --- готов понести наказание и выйти из Храма.

--- Да куда ты пойдёшь, старик, ты еле ходишь, --- поморщился Ситрис.
--- Да и школу пока что оставить не на кого.

--- Хонхо-лехэ, ты не понял.
\ml{$0$}
{Мы её и в жертву теперь принести не можем --- благодаря Трукхвалу!}
{We can't even sacrifice her anymore, thanks to \Trukchual!''}

\ml{$0$}
{--- Что это значит?}
{``What does that mean?''}

\ml{$0$}
{--- Она действительно невиновна --- вот что это значит!}
{``She's really innocent, that's what that means!}
\ml{$0$}
{Есть лук со звенящей тетивой, трубач, свидетели, улики, даже обвиняемые уже ждут в подвале.}
{We got a bow with a ringing string, shellblower, witnesses, evidence, even the people to accuse are waiting in the cellar.}
Но если мы назначим дополнительное слушание --- все ваши делишки вскроются, и тогда не только тебе, но и Храму в полном составе могут указать на ворота города!

--- Что вы сделали с трубачом? --- спросил Хонхо.

--- Я её отпустил без метки в обмен на информацию, --- сказал Трукхвал.
--- Она сообщила достаточно, чтобы мы распутали дело.
Ну и, хай, купил ей оленя на храмовое золото, потому что на ногах она стоять пока не может.

--- Да и к свиньям это золото, --- сказал Первый.
--- Но надо было всё-таки поставить метку.
Она виновна в разбое и вдобавок напала на жреца.
За меньшее сразу на алтарь отправляют.

--- Я сделал то, что считал нужным, --- неожиданно твёрдо заявил Трукхвал.
\ml{$0$}
{--- Мне нужна была истина, и я предложил за неё самую высокую цену, какую мог предложить.}
{``I wanted the truth, and I offered the highest price I could.''}

\ml{$0$}
{--- Очень по-жречески, хоть и недальновидно, --- признал Первый.}
{``Very priestly, although unwise,'' Head Priest admitted.}
\ml{$0$}
{--- Ладно.}
{``Well.}
\ml{$0$}
{Выкладывай Совету Храма, что накопал и чем нас закопают...}
{Tell the Temple Court what you've dug out and with what we'll be dug in ....''}

\section{Слушание}

--- Свидетельство бывшего бандита и человека, который обязан Кхотлам жизнью.

--- Во-первых, я требую извинений перед Нижним этажом.
На Нижнем этаже нет бывших бандитов, есть только воины.

--- Приношу извинения Ситрису ар'Эр, --- процедила сквозь зубы женщина.

--- Во-вторых, Ситрис и Сиртху --- не единственные свидетели.
Прочие в зале.
Поднимите руки.

Несколько рук поднялись.

--- И последнее --- Ситрис и Сиртху были всего лишь исполнителями и помощниками при расследовании.
Курировал расследование достойный доверия человек, не имеющий личной заинтересованности.

--- Кто?

--- Трукхвал ар'Хэ.

--- Жрец?
Что вся эта морковка значит?
Вначале вы тянете с жертвоприношением три декады, а потом жрец проводит дополнительное расследование!

--- У вас есть причины предполагать заинтересованность Трукхвала ар'Хэ? --- ледяным тоном спросил Первый жрец, пропустив выпад мимо ушей.

--- Насколько нам известно, он библиотекарь, --- заметил старшина камнерезов.
--- Мы крайне редко видим его в городе.
Какой резон библиотекарю проводить дополнительное расследование?
Давайте послушаем его.

--- Я... хай, --- смутился Трукхвал.
\ml{$0$}
{--- Мне просто показалось, что не вяжется что-то в этой истории...}
{``I just felt something strange about that story ...''}

\ml{$0$}
{--- И всё?}
{``And?''}

\ml{$0$}
{--- Всё.}
{``That's it.''}

--- Кто посоветовал вам начать расследование? --- требовательно спросила старшина Сада.
--- Кхотлам?
Первый жрец?
Другие люди?

--- Хитрам ар'Кхир, насколько я знаю, он мужчина Кхотлам.

--- Этот ненормальный ко всем приставал с этим вопросом, --- кивнула старшина.
\ml{$0$}
{--- В общем, понятно.}
{``That much is clear.}
Что сказали Первый жрец и сама Кхотлам?

\ml{$0$}
{--- Кхотлам отказалась отвечать на мои вопросы, в довольно грубой форме.}
{``\Kchotlam\ refused to answer my questions, in a quite rude manner.}
Предупреждать Первого жреца я не стал.
Моя ошибка, готов понести наказание.

--- То есть Первый жрец узнал о твоём расследовании пост-фактум?

--- Боюсь, что так.

Делегаты зашептались.

--- Мы позволяем Храму определённые вольности, но это ни в какие ворота!
Полный беспорядок!

--- Знал ли Трукхвал, что Кхотлам содержится в Келье Без Окон уже три декады?

--- Я узнал об этом в день начала расследования, --- объяснил Трукхвал.
--- Она сама мне об этом сказала, пригрозив совершить самоубийство, если вопрос не будет решён.

--- Я прошу прощения, --- поднялась в зале рука.
--- Насчёт самоубийства.
Можешь процитировать?

\ml{$0$}
{--- <<Если меня не принесут в жертву в ближайшее время, я перережу себе жилы>>.}
{`` `If I don't be sacrificed soon, I shall cut my strands.'}
\ml{$0$}
{Это дословно.}
{Verbatim.''}

--- Как получилось, что ты не знал о пребывании Кхотлам в Келье Без Окон?

\ml{$0$}
{--- Обычно я не имею никакого отношения к Келье Без Окон, --- сказал Трукхвал.}
{``Usually I have nothing to do with Windowless Cell,'' \Trukchual\ said.}
--- Но в тот день подошла моя очередь идти на кухню и, соответственно, разносить еду.
Чаще всего меня просто пропускают в очереди, потому что я сплю или переписываю.

В толпе раздались смешки.

\ml{$0$}
{--- Типичный библиотекарь...}
{``A classical librarian ....''}

\ml{$0$}
{--- Я обратил внимание, что она отказалась от еды, и судя по её состоянию, уже не впервые, --- продолжил жрец.}
{``I noticed that she refused to eat, and, given her physical state, not the first time,'' the priest continued.}
\ml{$0$}
{--- Поэтому решил поговорить.}
{``So I decided to talk to her.''}

\ml{$0$}
{--- Так, значит, Первый жрец, ты признаёшь, что держал кутрапа живым три декады? --- сузив глаза, прошипела старшина Сада.}
{``So you confess, Head Priest, that you have been keeping a \kutraph\ alive for three decades?'' Garden Head hissed, narrowing her eyes.}

\ml{$0$}
{--- Признаю, --- Первый демонстративно снял перьевую корону и положил на кафедру.}
{``I do.'' Head Priest defiantly took a feather crown off his head and put it on the pulpit.}
\ml{$0$}
{--- Это был мой приказ, и ответственность лежит на мне.}
{``It was my order, and I take full responsibility.}
\ml{$0$}
{Выборы нового Первого жреца уже назначены на послезавтра, все необходимые для передачи полномочий приготовления мной сделаны и формальности соблюдены.}
{The election of a new Head Priest is scheduled for the sunrise after tomorrow, I made all necessary preparations for the transfer of power, all formalities are handled.}
\ml{$0$}
{Сейчас же, с вашего позволения, я хочу дать слово Трукхвалу, чтобы он осветил подробности своего расследования.}
{Right now, with your permission, I'd like to give the floor to \Trukchual, so he could highlight details of his investigation.''}

Первый кивнул присутствующим, набросил на плечи дорожный плащ и вышел из зала.
В наступившей ошеломлённой тишине прозвучал лишь шелест пергамента Трукхвала и весёлый топот оленьих копыт, удаляющийся в сторону южных ворот.

\section{На выход}

--- Кхотлам, на выход.

Кхотлам вздохнула и, расправив штаны, поднялась на ноги.
Ну вот и всё.
Отбегалась, отплавалась, отсмеялась, отлюбила.
Пора зажигать фонарик.

Кхотлам уже не волновала мысль о том, что сейчас ей будет ужасно больно.
Это был просто факт.
Она переступила порог комнаты, и её ослепил солнечный свет, бьющий в окна храма.
<<Какое счастье, солнце пришло со мной попрощаться>>, --- подумала она.

Каждый шаг давался с трудом.
Она едва заставляла себя переставлять ноги.
Жрец шел рядом.
Но едва ее нога встала на ступеньку, ведущую на крышу, как сопровождающий мягко перехватил ее за локоть.

--- Тебе туда, --- сказал он, показав на лестницу в зал.
--- Совет тебя оправдал, ты свободна.

Кхотлам села там же, где стояла.
Её ногти до крови впились в плечи, из пересохшего горла вырвался глухой стон.
Жрец, покачав головой, поднял женщину на руки и понёс в зал.

\section{Бывший Первый}

--- Нам сегодня снова разбили окно.
Я отправил Эрхэ за стекольщиком.

--- Ожидаемо, --- кивнула Кхотлам.
--- Обычное дело, когда в город возвращается Разрушитель или Насильник.
Учитывая, что вернулась я не откуда-нибудь, а из Кельи Без Окон, я бы ожидала и худшего.
Последний раз такое случалось больше ста пятидесяти дождей назад.

--- Кто тот человек?
Что с ним случилось?

--- Я знаю лишь то, что он был обвинён в Насилии над ребёнком.
После оправдания он всё равно вынужден был уехать, потому что ему угрожали устроить самосуд.
В этот раз Храм доходчиво объяснил горожанам, что по закону я невиновна и всё ещё являюсь дипломатом Тхитрона, а это значит, что я пользуюсь храмовой неприкосновенностью и любой, кто поднимет на меня руку, сразу отправится туда, где я пробыла три декады.
Только уже без возможности выйти.
Если добавить, что мою невиновность определили Советы, выдвинуть мне вотум недоверия как дипломату тоже никто не рискнёт.
По крайней мере в ближайшее время.

Кхотлам потёрла висок.

--- Я сегодня опять кричала во мне?

--- Нет, --- сказал Хитрам.
--- Чуть-чуть повизгивала, но я тебя почти сразу успокоил.
Я приспособился.
Беру тебя вот так, --- он положил руку ей на загривок, --- и ты затихаешь.

Кхотлам прикрыла глаза и потёрлась головой о его руку.

--- У тебя снова появились щёчки, --- Хитрам нежно погладил щеку любимой.

--- К сожалению, внешние щёчки появляются гораздо быстрее внутренних, --- грустно сказала Кхотлам.
--- Внутри я всё ещё чувствую себя обглоданным скелетом.

--- Я буду гладить и греть тебя, пока скелет не обрастёт плотью.
У нас впереди целая жизнь на это.

--- Хрупкая она, эта жизнь.
Взмах ресниц --- и от тебя осталась только надпись в Келье Без Окон.

--- Меня это только вдохновляет.

--- Как и любого, кто был только по одну сторону той двери.

Хитрам смутился.

--- Прости, жизнь моя.
Я был недостаточно чуток.

--- Я знаю тебя, Хитрам.
Знаю твои глаза, твои губы, твои руки, твои мысли.
Ты чуток настолько, насколько можешь быть.
Просто то, что чувствую я, за пределами твоего понимания.

Кхотлам покачала головой и помяла в руке пергамент.

--- Получила письмо от Первого.

--- И где он сейчас?

--- Среди Людей Золотой Пчелы.
Ни один город, разумеется, его не принял бы с такой репутацией.
А Бродячий Храм взял сразу.
Они любят тех, кто верен клятве, а не букве закона.

--- Вы оба на своём месте, --- уверенно сказал Хитрам.
--- Он знал, где твоё и где его.
Потому и спас тебя...

--- Нет, не поэтому, Хитрам.
Я здесь вообще ни при чём.

--- Я не понимаю тебя, --- сказал Хитрам.
--- Но я буду стараться.
А пока пойду приготовлю поесть.

--- Ты смотри.
А то мои щёки станут ещё круглее.
Ты хочешь, чтобы дипломат стал мячиком?

--- Жду не дождусь, --- Хитрам поцеловал Кхотлам.
--- Твои любимые жареные мидии?

--- Что угодно, только не мидии.

\chapter{Розовые орхидеи}

\section{Отголоски прошлого}

Кхотлам открыла окно и вдохнула пряный ночной воздух.
Её всё ещё била дрожь.
Последнюю декаду Келья Без Окон снилась женщине каждую ночь.
Чаще всего там сидел связанный Хмурый.

--- Хватит, кормилица, --- просил он.
--- Отпусти меня на крышу.
Я больше не могу здесь сидеть...

Кхотлам плакала и цеплялась за его рубаху.
Она заколачивала дверь изнутри неизвестно откуда взявшимися досками, проверяла узлы на путах, которые стягивали руки и ноги Хмурого...

--- Кхотлам, прекрати, --- доносился из-за двери приглушённый голос Первого жреца.
--- Открой дверь.
Пожалуйста, Кхотлам.
Знак кирпича уже был, он наш единственный пленник!

Первый стучал, плакал, умолял...
Затем за дверью раздались жуткие крики.
Во рту появился горький привкус, мир замерцал всеми цветами радуги.
Отголоски совсем другой истории, из давно забытого детства...

Сон сменился, словно открылся и закрылся птичий глаз.
Хмурый лежал в луже крови на полу Кельи.
В руках Кхотлам была окровавленная стрела.
В этот момент бурлящее сознание наконец прорвало оковы спящего тела --- Кхотлам закричала.
И тут же проснулась от собственного крика.

Кхотлам набила трубку смесью табака и конопли.
Изящная деревянная трубка хака, украшенная орнаментом, раскрашенная в яркие цвета, ароматная и тёплая на ощупь --- последний подарок Си-Жака.

--- Возьми, --- грубо сказал вождь, протянув ей трубку.
Он не утруждал себя дипломатической вежливостью, субординацией между старшими и младшими, дисциплиной, жёсткими церемониями дарения.
Они были в комнате совсем одни, и он просто отдавал дорогой его сердцу предмет человеку, к которому испытывал глубокую признательность.
Когда Кхотлам взяла трубку, Си-Жак подошёл к дипломату и обнял её, как хака обнимают боевых товарищей --- левая рука на шее, правая на плече, лоб ко лбу.
Немыслимый поступок по отношению к иноплеменнику и тем более женщине.
Затем он ушёл, на прощание задев щёку Кхотлам торчащими из ушей синими перьями.

Кхотлам выпустила последнее кольцо и загасила трубку.
Конопля пригасила эмоции, голова отяжелела, захотелось спать.
<<Можно было бы пойти к Лисёнку, --- размышляла она, --- но не хочу, чтобы он дышал моим перегаром всю ночь.
К себе не лягу --- эта пустая лежанка рядом меня угнетает.
Чтоб тебя идолы загрызли, Пловец.
Устроюсь возле колодца.
Может, меня укусит змея и мои мучения наконец окончатся>>.

Несколько ночей Кхотлам спала с Ликхмасом.
Она дожидалась, пока он уснёт, тихо ложилась на пол рядом с ним, брала в свою узловатую руку маленькую детскую ручку и засыпала.
Уходила она утром, ещё до того, как ребёнок просыпался.
Кхотлам нравилась эта маленькая игра.
Она не догадывалась, что играет она не одна --- каждую ночь Ликхмас притворялся спящим, ждал Кхотлам и, когда она засыпала, укрывал её одеялом, спрятанным в углу.
Кхотлам списывала появление одеяла на Эрхэ.
Эрхэ ничего ни на кого не списывала --- она спала как убитая и ей было не до игр.

\asterism

--- Кхотлам.
Кхотлам.
Кхотлам!

Кхотлам разлепила глаза и застонала от боли в затёкшей шее.

--- Ну ты придумала тоже --- на улице спать, --- заворчала Эрхэ.
--- Вставай.
Ликхэ есть приготовила.
Хотя нет, не вставай.
Не шевелись.

--- Что такое? --- насторожилась Кхотлам.

--- Уже ничего, --- Эрхэ достала из рубахи Кхотлам трёх змей и выкинула их в кусты.
--- Пригрелись, рыбины.
Не ядовитые, но воняют дай духи...
Всё, иди сполоснись, а от тебя змеями теперь тоже пахнет.
Хэ, нет, опять стой.

Эрхэ сняла рубаху, обмотала ею руку и, поморщившись, потянулась к волосам Кхотлам.
На этот раз навстречу приключениям в кусты полетел волосатый тарантул, отважно растопырив ножки.

--- Рубаху в стирку, --- констатировала служанка.
--- Не хочу чесаться весь день.

--- Всё хорошо? --- спросила зеленоглазая Ликхэ, выглянув из дверей.

--- Да-да, --- Эрхэ смущённо засуетилась, --- мы идём, львёнок...

Львёнок показал острые зубки на массивных челюстях, распушил гриву и исчез в дверном проёме.

--- ...Я тоже хочу уйти, --- Эрхэ смотрела в пол и нервно перебирала шнурки рубахи.

--- Эрхэ, ты сейчас моя единственная посыльная, --- Кхотлам закрыла лицо руками.
--- И потом, твой кормилец взял с меня обещание...

--- Я знаю.
Но у меня тут такое дело... я женщину нашла.
Крестьянка.
У неё сестрёнка больная, и...

--- Ну так приводи их сюда!
Я-то думала, у тебя что-то совсем непоправимое.

--- Ты серьёзно? --- опешила Эрхэ.

--- Эрхэ, во Дворе двенадцать комнат, из них заняты три!
Если ей далеко ходить до поля --- можем что-нибудь придумать.

--- Нет-нет, ей недалеко, --- Эрхэ расцветала на глазах.
--- Она вкусно готовит.

--- Тем более приводи.
Я тебя отпускать не хочу.

<<Это я, конечно, крупный выигрыш сорвала, --- думала Кхотлам, уплетая ,,гнездо трасакха в сезон дождей``.
--- Ей бы в постоялом дворе работать.
Всё-таки протоколы протоколами, но Книга-кормилица в правильных руках --- это что-то.
\ml{$0$}
{От такой стряпни жить хочется>>.}
{Cooking like that is a reason to live.''}

Ликхмас сидел рядом и кормил сестрёнку Ликхэ.
У девочки тряслись руки, и она не могла донести черпачок до рта.

\ml{$0$}
{--- Я держу, --- говорил он, --- я держу...}
{``I hold you,'' he said, ''I hold you ...''}

Кхотлам с нежностью смотрела на них.
В груди нарастала щемящая обида.
Она против воли бросила взгляд на другую сторону стола, где всё так же лежала красная с синим циновка и стояла пустая миска.
Накрывая на стол, Эрхэ каждый день сковыривала корочку с душевной раны Кхотлам, но Кхотлам ничего ей не говорила.

<<Чтоб тебя идолы загрызли, --- в сердцах думала она.
--- Как ты мог его оставить?
Как ты мог оставить меня?>>

\section{Красные ленты}

--- Кормилица, а где Синяя Лента?

--- Она ушла, дитя.

--- Не лги мне.
Она умерла, да?

--- Нет, не умерла.
Просто ушла.
Если бы умерла, я бы так тебе и сказала.

--- Ликхэ такая грустная, словно она умерла, --- шёпотом сообщил ребёнок.

--- Я не грустная, Лисичка, --- Ликхэ вдруг подняла голову и громко рыгнула.
--- Пьяная я, на самом деле.
Синюю Ленту унесли красные ленты.
Такая вот праздничная одежда получилась.
Только танцевать в ней мне одной...

--- Эрхэ, уведи её спать, --- тихо сказала Кхотлам.
Служанка кивнула.

--- Жилище большое, поле до горизонта, а из домашних я одна осталась, --- пожаловалась Ликхэ.
--- На родных могилах больше цветов, чем у родных дверей. 

--- Ликхэ, пойдём, --- прошептала Эрхэ, поднимая любовницу на руки.
--- Я точно никуда не уйду.
Будут и наши с тобой праздники, и танцы, и цветы...

У Кхотлам резко заболел живот.
Она опустилась на колени у очага и начала греть похолодевшие руки.

<<Чтоб тебя идолы загрызли, Пловец>>.

\section{Дождь}

Кхотлам открыла дверь, и мир вокруг завертелся.
Она снова вернулась на Могильный берег.
Рядом снова шумела вода, а в носу и горле нестерпимо жгло, но на этот раз масло из Санта-Виктории было ни при чём.
Она втащила насквозь мокрого мужчину в дом и изо всех сил отвесила ему пощёчину.

--- Не смей так больше со мной поступать, --- в ярости прохрипела она.
--- Не смей так больше со мной поступать!

Хитрам шагнул к ней.
Из его руки выпал измятый, мокрый букет розовых орхидей.
За мгновение до того, как цветы коснулись малахитового пола, губы Кхотлам и Хитрама встретились.

\chapter{Маска и тушь}

--- Я тебя поздравляю, я снова беременна, --- проворчала Кхотлам, ощупывая полосатую шею.
--- Вот объясни, зачем вообще придумали влечение к мужчинам?
Почему я не цветущая, как Эрхэ?
Скольких проблем я могла бы избежать!

--- На данный момент --- не менее двух, --- флегматично отозвался Хитрам.
--- Кстати, что ты беременна, я тебе сказал ещё три дня назад.
Полоски были уже хорошо различимы.
Но в этот раз ты как-то очень медленно принимала этот факт.

--- Я уже в том возрасте, когда тигриную шею можно спутать со старческими морщинами!

--- Перестань, Пёрышко.
Ты ещё десяток детей выносить можешь помимо всех прочих дел.

--- Типун тебе на язык.
Не надо мне такого счастья.

--- Ты сохранишь ребёнка?

--- Я в настроении попробовать.

--- Если ты не хочешь...

--- Если я передумаю --- ты узнаешь.
Пока я скорее заинтригована.

--- Я тоже, кстати.
В этот раз пятна какие-то другие.
Они как будто ярче.
Может, двойня?

--- Если двойня --- второго ребёнка выкармливать будешь ты!
У меня грудь не бездонная!

--- Никаких возражений.
Опыт уже есть.

Кхотлам пригладила волосы и вздохнула.

--- Рыбина ты гнилая, Пловец.
Ты сам не знаешь, какая ты рыбина.

--- И не узнаю.
Всё, что могу предложить --- свою вернувшуюся тушку.

Кхотлам смягчилась.

--- Извини.

--- Не будешь больше меня шпынять?

--- Не буду, любовь моя.
Клянусь орхидеями, которые ты мне припёр.

--- Выкинь их уже, они засохли давно.

--- Я их в книгу вклеила.
Кстати, они очень красивые.
Впервые вижу такие яркие и пышные розовые орхидеи.
Где ты их сорвал?

Хитрам кашлянул и покраснел.
Кхотлам поняла, что история происхождения орхидей не особо романтичная, и сменила тему.

--- Я всё-таки не понимаю, почему ты вернулся.

--- Я совсем забыл, как находить женщин, --- пожал плечами Хитрам.
--- По привычке пошёл к Могильному берегу и стал ждать, когда море выбросит новую.
Подождал три декады, но море так и не смилостивилось.

Кхотлам хихикнула.
Впрочем, её лицо быстро стало серьёзным.

--- Раз уж ты здесь, уладишь для меня кое-какое дело?
Как бы смешно это ни звучало, нужна мужская рука.

--- Для того и вернулся, --- ухмыльнулся Хитрам.
--- Небось с хуторскими с востока нелады?

--- Да, и серьёзные, --- Кхотлам протянула Хитраму свиток пергамента.
--- Сатхемитр.
Половина населения --- хака, в основном пахари из северных племён, мнение купца-женщины они ценят примерно так же, как и мнение тхитронского Храма.

\ml{$0$}
{--- Небо высоко, Храм далеко.}
{``They sky be high, they Fane be far.''}

\ml{$0$}
{--- Вот именно такие.}
{``Exactly their type.}
\ml{$0$}
{Те, что сели, под стать --- ревнители старины, грубые и подозрительные к чужакам.}
{\Seli\ blood there is a good match---people for tradition, rude and suspicious of strangers.}
\ml{$0$}
{У них своя погода и свои ветра.}
{They have weathers and winds of their own.}
\ml{$0$}
{Слава духам, конфликт не этнический, а торговый.}
{Praise Spirits, it's a trade conflict, not ethnic.}
С этническим в такой глухомани я бы вообще не знала, что делать...

--- Хаяй, --- протянул Хитрам, просмотрев свиток.
--- Вчерашнее?
Тогда давай-ка я прямо сейчас съезжу, перед едой, ситуация мне вообще не нравится.
Набросай в общих чертах стратегию переговоров, почитаю по пути.
И передай мне заколку и печать на случай, если они ерепениться начнут.

--- Воинов тебе дать? --- Кхотлам выдернула чистый листок пергамента и зашуршала пером.

--- Одному идти не резон, но нет, воины их только разозлят.
Дай мне лучше парочку квартальных старшин поавторитетнее, желательно таких же, хуторских, только пусть они рот зазря не открывают, пропиши это чётко в письме...

\chapter{Взрывной характер}

\section{Изгнание}

--- У меня для вас сюрприз.

Кхотлам, сияя, передала вождю официальное письмо, скреплённое печатью Большого Дома.

--- Двадцать дождей?! --- воскликнула вождь, пробежавшись по письму.
--- Они решили изгнать нас на двадцать дождей?!

--- Кошмар! --- возмутился Ситрис.

--- Кхотлам, --- осторожно начал Эрликх, --- нельзя ли нам как-нибудь...

--- Нельзя! --- рассерженно рявкнула Кхотлам.
--- Они бы изгнали нас навечно, если бы не я!
Либо вечное изгнание, либо голова Кхохо!
Вы серьёзно думаете, что можно просто выкрасть осквернителя святилища и вам ничего за это не будет?
Идите к свиньям отсюда.
Пошли отсюда.
БРЫСЬ!

Кхотлам вытолкала воинов за дверь и закрыла её на три замка.

--- Как же я ненавижу северян, --- с ненавистью шептала она.
--- В последний раз.
Это в последний раз.
Ещё одна выходка --- и я отсюда уезжаю!
В бездну Север!
В бездну северян!

--- Кормилица, а я северянин? --- тихо спросил Ликхмас.
--- Ты меня тоже ненавидишь?

Кхотлам осеклась, поняв, что последние слова она выкрикнула во всё горло.

--- Нет, золотце, --- зашептала Кхотлам, прижав его к себе.
--- Конечно, нет.
Они меня обидели, и я рассердилась.
Мне не следовало этого говорить.
Я люблю тебя, мой маленький северянин, и всегда буду любить.
Пойдём поиграем, пока малыши спят...

\section{Няня}

--- Ты чересчур в это вкладываешься, --- сказал Хитрам, подливая ей вина.
--- Уж на что поле благодарное, и то не всегда --- то неурожай, то болезнь на кукурузу нападёт.
А ты о людях, которые тебе даже не друзья, и о пустоголовой воительнице, которая тебя ненавидит.

--- Она бы погибла! --- жаловалась Кхотлам.
\ml{$0$}
{--- Не могла же я её оставить там!}
{``I couldn't leave her there!''}

\ml{$0$}
{--- Да в том-то и дело, что могла! --- возразил Хитрам.}
{``Yes, you could, that's the point!'' said \Chitram.}
\ml{$0$}
{--- Могла.}
{``You could.}
\ml{$0$}
{Должна была.}
{You should've.}
\ml{$0$}
{Так бы сделал любой дипломат на твоём месте.}
{Every negotiator would've done that if they were you.}
\ml{$0$}
{Закон есть закон.}
{Law is law.}
То, что сделала ты, неслыханно по наглости и по виртуозности.
Ты оказала Храму, и Кхохо в частности, большую любезность.

--- И они вот так меня отблагодарили, --- вяло закончила Кхотлам, прихлёбывая вино.

--- Я думаю, не стоило ждать от них благодарности.

--- Я не ждала благодарности.
Но они могли хотя бы заткнуться и промолчать!

--- Я поговорю с ними, --- решительно заявил Хитрам и, накинув плащ, ушёл в дождь.

Он вернулся через десять михнет в сопровождении всего Храма.
Храмовники были насквозь промокшие и серьёзные.

--- Кхотлам, --- начал Первый жрец, выжав воду из волос прямо на пол.
\ml{$0$}
{--- Я приношу тебе глубочайшие извинения за, кхм, реакцию моего Храма.}
{``I offer you deepest apologies about, ahem, the reaction of my Temple.}
Мы все тебе очень благодарны за твою работу.
\ml{$0$}
{Особенно Кхохо.}
{Especially \Kchoho.''}

Кхохо прожгла его взглядом и промолчала.

\ml{$0$}
{--- Вы уже наложили на неё наказание? --- осведомилась Кхотлам.}
{``Have you already punished her?'' \Kchotlam\ coldly asked.}

\ml{$0$}
{--- Да, мы отстранили её от работы и...}
{``Yes, she's suspended from her work and---''}

\ml{$0$}
{--- ...и она бездельничает, --- раздражённо закончила Кхотлам.}
{``---and messing around,'' \Kchotlam\ angrily finished.}
\ml{$0$}
{--- Отличное наказание, отличная плата за все наши труды и беспокойства.}
{``What a good punishment, what a good payment for our work and our troubles.''}

--- Если ты хочешь что-то предложить, мы с удовольствием тебя выслушаем.

--- Раз уж основная часть трудов и беспокойств была на мне, я сама наложу на неё наказание.

--- Купец накладывает наказание на храмовника? --- захихикала Кхохо.
--- Очень смешно, да?..

\ml{$0$}
{--- Идёт, --- быстро сказала вождь.}
{``Deal,'' the warchief quickly said.}

\ml{$0$}
{--- Что?!}
{``What!''}

\ml{$0$}
{--- Договорились, --- поддержал Первый жрец, с неодобрением глядя на воительницу.}
{``Agreed,'' Head Priest supported, giving the warrior a grim look.}

--- Ы! --- Кхохо захлебнулась от гнева.
Ситрис хихикнул.

--- Вот и отлично, --- с облегчением сказала вождь и похлопала Первого по макушке.
--- Всё, Кхитлас, народ, пошли отсюда.
Они сами разберутся...
\ml{$0$}
{Прости ещё раз, Кхотлам.}
{One more time---I'm sorry, \Kchotlam.}
\ml{$0$}
{Мы сглупили.}
{We were clumsy.''}

Дверь закрылась, унеся с собой шаги храмовников, и наступила тишина.
В тишине отчётливо проступало свирепое дыхание Кхохо.

--- Надеюсь, ты не собираешься бить \emph{меня}? --- осведомилась Кхотлам.

Кхохо промолчала.

--- Не волнуйся, наказание для тебя простое, --- сказала Кхотлам.
--- Как видишь, я недавно разродилась двойней.
И в отличие от вас, храмовников, я несу ответственность за своё потомство.
Кормить, поить, мыть, одевать, пестовать двоих детей помимо работы --- это чересчур.
Даже для нас с Хитрамом.

--- Ну так думать надо не гениталиями, --- заявила воительница.

\ml{$0$}
{--- Безусловно, Кхохо.}
{``Of course, \Kchoho.}
\ml{$0$}
{Но твой совет, как видишь, немного запоздал.}
{But, as you can see, your advice is a little late.}
\ml{$0$}
{Поэтому вот тебе задание --- посмотри за моим старшеньким, за Лисёнком, пока я занимаюсь малышами.}
{Then, this is your task: look after my eldest---Baby Fox---while I'm stuck with the babies.''}

--- Нет! --- Кхохо непроизвольно сделала два шага назад и упёрлась спиной в стену.
На её висках пульсировали жилки.

\ml{$0$}
{--- Что такое?}
{``What's wrong?''}

\ml{$0$}
{--- Нет!}
{``Not this!}
\ml{$0$}
{Только не это!}
{Anything but this!''}

\ml{$0$}
{--- Он очень милый малыш.}
{``They are a cute child.}
Сейчас я вас познакомлю.
\ml{$0$}
{Лисёнок!}
{Baby Fox!}
Иди сюда.

Из-за тяжёлой занавеси робко выглянул зеленоглазый ребёнок с длинными волосами цвета меди.
На веснушчатой рожице замер испуг.

--- Лисёнок, вот твоя няня на эту декаду.
Её зовут Кхохо, и я ей доверяю.

Кхохо ещё сильнее вжалась в стену.
Её зубы стучали, грязные ногти царапали изящных малахитовых змей с чёрными глазками.
Налитые кровью глаза с ужасом наблюдали за маленьким страшилищем, которое с неловкой редкозубой улыбкой неотвратимо приближалось к ней.

--- Здравствуй, Кхохо! --- наконец промямлило страшилище, протянув к ней ручки.
--- Ты любишь куриный суп?

\chapter{Без имени}

\section{Любимая няня}

--- А Кхохо ещё придёт? --- допытывался Лисёнок.

--- Возможно, --- улыбнулась Кхотлам.
--- Она тебе нравится?

--- Она научила меня делать вот так! --- сказал Лисёнок и издал громкий неприличный звук.
--- А ещё я теперь знаю слово...

Ребёнок одним духом выговорил.
Эрхэ и Ликхэ закашлялись.

--- Отличное слово, --- похвалила Кхотлам, даже не изменившись в лице.
--- Только это слово, Лисёнок, оно не совсем обычное...
Оно острое.

--- Острое?

--- Да.
Некоторые слова острые как спица, некоторые --- как нож.
И если использовать его неосторожно, можно сделать человеку больно.
Поэтому будь осторожен, хорошо?

--- А почему Кхохо его постоянно говорит?

--- Кхохо --- воин, и она привыкла обращаться с острыми вещами.
А вот детям надо еще учиться.

\chapter{Лис и Оцелот}

\section{Беззащитный сон}

Чханэ была красива.
Кхотлам не могла на нее налюбоваться.
<<Конечно, я вообще не думала, что Лисенок кого-то полюбит, --- улыбнулась она про себя.
--- Не припомню, чтобы он кем-то увлекался.
Видимо, просто время не пришло>>.

Однако кроме красоты было ещё кое-что.
Большая девушка выглядела удивительно беззащитной во сне.
Девочки-близняшки тоже выглядели мило, но беззащитности в них не было.
Кхотлам почти видела тяжёлое детство, кормильцев, которым до ребёнка не было дела, многочисленные проблемы в Храме...

\asterism

Кхотлам аккуратно вынула фалангу из ножен.
Махнула раз, второй...
Тело автоматически встало в нужную позицию.
Ещё одним ударом Кхотлам снесла несколько веток, свесившихся со старой ивы.

Мимо проходил Ситрис, нагруженный амуницией.
Разумеется, он снова вызвался точить и править оружие.
Кхотлам точно знала, что три четверти времени он проводит не с клинками, а с симпатичным оружейником.

--- Кхотлам? --- удивился Ситрис.
--- Давно не видел, чтобы ты махала клинком.
Спарринг?

--- Спарринг, — кивнула Кхотлам.
--- Отобью старое мяско.

--- Не прибедняйся, не такое уж и старое, --- ухмыльнулся воин и, бросив на землю связку клинков, вытащил один.
--- В честь чего праздник-то?

--- Воительница приехала с запада, думаю ее устроить.
Это её фаланга.
Больно уж красивая.

--- Не возражаешь? --- Ситрис протянул руку.

--- Да, конечно, --- Кхотлам протянула фалангу собеседнику.

--- Новая, --- хмыкнул Ситрис, осмотрев оружие.
--- Следы использования есть, но не боевые.
Ею рубили ветки и, кажется, курицу.
Клеймо на пять дождей старше клинка.

Ситрис вернул клинок Кхотлам.
Его глаза смеялись.

--- Опять твои фокусы?

--- Не понимаю, о чём ты, --- непринуждённо ответила купец.
--- Просто хочу поупражняться, вспомнить былое.

--- Понимаю, --- подмигнул Ситрис. 
--- Вставай в позицию.
Сейчас понаделаем боевых следов использования.

--- Главное --- не сломай.
Чужая вещь.

--- Ты за кого меня принимаешь? --- возмутился Ситрис.
--- Я, по-твоему, Маликх, чтобы клинки ломать?

\end{document}
