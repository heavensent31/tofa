% \documentclass[a4paper,10pt,fleqn]{book}\usepackage{polyglossia}\setdefaultlanguage[babelshorthands=true]{russian}\setotherlanguage{english}\defaultfontfeatures{Ligatures=TeX,Mapping=tex-text}\usepackage{xcolor}\newcommand{\ml}[3]{#2}

\documentclass[a4paper,10pt,fleqn]{book}\usepackage{cooltooltips}\usepackage{polyglossia}\setdefaultlanguage{english}\setotherlanguage{russian}\defaultfontfeatures{Ligatures=TeX,Mapping=tex-text} \usepackage{xcolor}\definecolor{lightgray}{HTML}{bbbbbb}\color{lightgray}\newcommand{\ml}[3]{\textcolor{black}{#3}}

% ----------------------

\usepackage{amsmath,amssymb,amsfonts,xltxtra,microtype,graphicx,textcomp}
\usepackage{svg}

% ------ GEOMETRY ------

\usepackage[twoside,left=2.5cm,right=3cm,top=3cm,bottom=4cm,bindingoffset=0cm]{geometry}

% ------ FONT ------

\setmainfont{Linux Libertine}
\definecolor{darkblue}{HTML}{003153}

% ------ HYPERLINKS ------

\usepackage{hyperref}
\hypersetup{colorlinks=true, linkcolor=darkblue, citecolor=darkblue, filecolor=darkblue, urlcolor=darkblue}

% ------ EPIGRAPH ------

\usepackage{epigraph}
\renewcommand{\epigraphsize}{\footnotesize}
\epigraphrule=0pt
\epigraphwidth=8cm

\usepackage{etoolbox}
\AtBeginEnvironment{quote}{\itshape}
\makeatletter
\newlength\episourceskip
\pretocmd{\@episource}{\em}{}{}
\apptocmd{\@episource}{\em}{}{}
\patchcmd{\epigraph}{\@episource{#1}\\}{\@episource{#1}\\[\episourceskip]}{}{}
\makeatother

% ------ METADATA ------

\newcommand{\tofaauthor}{\ml{$0$}{Эмиль~Весна}{Emil~Viesn\'{a}}}
\newcommand{\tofatitle}{\ml{$0$}{ЗМЕИНАЯ~ЯМА}{Snake~Pit}}
\newcommand{\tofastarted}{11.11.2021}

% ------ FANCY PAGE STYLE ------

\usepackage{fancyhdr}
\pagestyle{fancy}
\fancyhead[LE,RO]{\thepage}
\fancyhead[LO]{{\small\textsc{\tofatitle}}}
\fancyhead[RE]{{\small\textsc{\tofaauthor}}}
\fancyfoot{}
\fancypagestyle{plain}
{\fancyhead{}
\renewcommand{\headrulewidth}{0mm}
\fancyfoot{}}

% ------ NEW COMMANDS ------

\newcommand{\asterism}{\vspace{1em}{\centering\Large\bfseries$\ast~\ast~\ast$\par}\vspace{1em}}
\newcommand{\textspace}{\vspace{1em}{\centering\Large\bfseries<...>\par}\vspace{1em}}
\newcommand{\FM}{\footnotemark}
\newcommand{\FL}[2]{\footnotetext{См. \textit{\hyperlink{#1}{#2}}.}}
\newcommand{\FA}[1]{\footnotetext{#1 \emph{\ml{$0$}{---~Прим.~авт.}{---~Author.}}}}

\newcommand{\theterm}[3]{\textbf{\hypertarget{#1}{#2}} --- #3}
\newcommand{\thesynonim}[3]{\textbf{#2} --- см. \textit{\hyperlink{#1}{#3}}.}
\newcommand{\theorigin}[3]{\textit{#1:} #2 --- #3}

% ------ DIFFICULT TO WRITE TERMS ------

\newcommand{\Aatris}{\"{A}\={a}tr\v{\i}s}
\newcommand{\Akchsar}{\`{A}kchs\r{a}r}
\newcommand{\Chhammitrai}{Chh\`{a}mm\={\i}tr\^{a}i}
\newcommand{\Chhanei}{Chh\r{a}n\^{e}i}
\newcommand{\Chitram}{Ch\"{\i}tr\'{a}m}
\newcommand{\Choe}{Cho\^{e}}
\newcommand{\choe}{cho\^{e}}
\newcommand{\Harrmatr}{H\r{a}rrm\`{a}tr}
\newcommand{\tHat}{H\={a}t}
\newcommand{\Hei}{H\r{e}i}
\newcommand{\hei}{h\r{e}i}
\newcommand{\Hoesitr}{Ho\`{e}s\={\i}tr}
\newcommand{\hoesitr}{ho\`{e}s\={\i}tr}
\newcommand{\Kchaagotr}{Kch\^{a}\={a}g\~{o}tr}
\newcommand{\kchaagotr}{kch\^{a}\={a}g\~{o}tr}
\newcommand{\Kcharas}{Kch\'{a}r\v{a}s}
\newcommand{\Kchatrim}{Kch\r{a}tr\"{\i}m}
\newcommand{\Kchenoel}{Kch\={e}no\^{e}}
\newcommand{\kchenoel}{kch\={e}no\^{e}}
\newcommand{\Kchenoet}{Kch\"{e}no\^{e}}
\newcommand{\kchenoet}{kch\"{e}no\^{e}}
\newcommand{\Kchoho}{Kch\`{o}h\^{o}}
\newcommand{\Kchotlam}{Kch\={o}tl\'{a}m}
\newcommand{\Kchotris}{Kch\={o}tr\v{\i}s}
\newcommand{\Kihotr}{K\^{\i}h\~{o}tr}
\newcommand{\kihotr}{k\^{\i}h\~{o}tr}
\newcommand{\Kukchuatr}{K\`{u}kchu\={a}tr}
\newcommand{\kukchuatr}{k\`{u}kchu\={a}tr}
\newcommand{\Laaka}{L\={a}\"{a}k\^{a}}
\newcommand{\laaka}{l\={a}\"{a}k\^{a}}
\newcommand{\Lechoe}{L\={e}cho\`{e}}
\newcommand{\lechoe}{l\={e}cho\`{e}}
\newcommand{\Likas}{L\^{\i}k\v{a}s}
\newcommand{\Likchmas}{L\={\i}kchm\r{a}s}
\newcommand{\Likchoe}{L\^{\i}kcho\^{e}}
\newcommand{\Loem}{Lo\~{e}m}
\newcommand{\Maaras}{M\"{a}\={a}r\v{a}s}
\newcommand{\Mitchoe}{M\={\i}tcho\^{e}}
\newcommand{\Mitlikch}{M\={\i}tl\={\i}kch}
\newcommand{\Mitris}{M\={\i}tr\={\i}s}
\newcommand{\Oerchoe}{O\r{e}rcho\^{e}}
\newcommand{\Oerlikch}{O\r{e}rl\'{\i}kch}
\newcommand{\Sat}{S\={a}t}
\newcommand{\Satchir}{S\={a}tch\"{\i}r}
\newcommand{\Satrakch}{S\={a}tr\`{a}kch}
\newcommand{\Seli}{S\r{e}l\={\i}}
\newcommand{\Sirtchu}{S\r{\i}rtch\'{u}}
\newcommand{\Sitris}{S\~{\i}tr\v{\i}s}
\newcommand{\Siusiu}{S\~{\i}u-s\~{\i}u}
\newcommand{\siusiu}{s\~{\i}u-s\~{\i}u}
\newcommand{\Sogcho}{S\"{o}gch\={o}}
\newcommand{\Sotron}{S\~{o}tr\`{o}n}
\newcommand{\Tchalas}{Tch\r{a}l\v{a}s}
\newcommand{\Tchammitr}{Tch\`{a}mm\={\i}tr}
\newcommand{\Tchanoe}{Tch\r{a}no\^{e}}
\newcommand{\Tchartchaahitr}{Tch\~{a}rtch\"{a}\={a}h\r{\i}tr}
\newcommand{\Tchartchu}{Tch\~{a}rtch\'{u}}
\newcommand{\Tchitron}{Tch\"{\i}tr\`{o}n}
\newcommand{\Tchu}{Tch\`{u}}
\newcommand{\tchu}{tch\`{u}}
\newcommand{\Technku}{T\`{e}chnk\r{u}}
\newcommand{\Tesarrokch}{Te's\'{a}rr\r{o}kch}
\newcommand{\Tesatron}{Te's\'{a}tr\v{o}n}
\newcommand{\Traa}{Tr\={a}\"{a}}
\newcommand{\traa}{tr\={a}\"{a}}
\newcommand{\Trai}{Tr\r{a}i}
\newcommand{\trai}{tr\r{a}i}
\newcommand{\TraRenkchal}{Tr\r{a}-R\={e}nkch\'{a}l}
\newcommand{\Trukchual}{Tr\`{u}kchu\r{a}l}


\begin{document}

% ------ TITLE PAGE ------

\begin{titlepage}
{\centering{~\par}\vspace{0.25\textheight}
{\LARGE\tofaauthor}\par
\vspace{1.0cm}\rule{17em}{1pt}\par\vspace{0.3cm}
{\Huge\textsc{\tofatitle}\par}
\vspace{0.3cm}\rule{17em}{2pt}\par\vspace{1.0cm}
{\Large\textit{\ml{$0$}{Фантастический~роман}{Science~fiction}}\par}
\vspace{0.5cm}\asterism\par\vspace{1.0cm}
{\textbf{\ml{$0$}{Начато:}{Started:}}~\tofastarted\par}\vfill
{\Large\ml{$0$}{Создано~в}{Created~by}~\XeLaTeX}\par}
\end{titlepage}

\tableofcontents

\part{Змеиная яма}

\section{Перцовка}

--- Живая, --- пробормотал Хитрам.
--- Точно живая.
Метвецы так не дерутся...

--- Думать надо, --- буркнула Кхотлам, скрестив руки на груди.

--- А я что, по-твоему, делал?
Ты несколько кхамит без дыхания лежала.
От глубокого морского сна только это и помогает.

Кхотлам промолчала.

--- Где я, для начала? --- наконец спросила она.

--- А ты как думаешь?

--- На Короне или на Ките?

--- Ааа, --- протянул Хитрам.
--- Всё настолько плохо?
Это Корона.
На юго-запад --- Валенсия.

--- Меня выбросило к Валенсии? --- ужаснулась Кхотлам.

--- Как видишь.

--- Сколько же я тогда без сознания-то была, --- пробормотала женщина.
--- А моя команда?

--- Больше никто не очнулся.
Сочувствую.

--- Я должна их осмотреть.

Кхотлам попыталась встать и тут же села, вскрикнув и схватившись за лодыжку.

--- У тебя обе ноги покалечены, куда ты собралась? --- Хитрам подскочил и поправил шину.
--- Похоронил я их уже.

--- Как --- уже? --- Кхотлам удивлённо оглядела берег и, увидев свежие могилы, зажала рот рукой.

Хитрам смутился.

--- Да я бы и тебя давно похоронил, только сомнения были.

--- Так зачем ты их закопал?
Может, и они живы!..

Кхотлам проползла на руках до ближайшей могилы и начала с остервенением раскапывать песок.

--- Хай, женщина, --- Хитрам попылался перехватить её руки.
--- Да подожди ты.
Женщина!
Да стой!
Они мертвы, хватит!

Кхотлам остановилась и обратила к Хитраму лицо в слезах.
Тот без сил опустился рядом.

--- Я нашёл вас ранним утром, --- объяснил он.
--- Сейчас уже к вечеру клонится.
На всех, кроме тебя и ещё двоих, были следы тления.
Перцовки я дал всем, очнулась только ты.
Те двое вон там лежат, как и лежали.

Кхотлам легла ничком в песок и зарыдала во весь голос.

\asterism

--- И долго мы тут сидеть будем? --- буркнула Кхотлам.
--- Если у тебя нет сил меня нести --- я поползу.
Только скажи, куда.

--- Я всегда возвращаюсь к вечеру, --- пояснил Хитрам.
--- Сёстры это знают.
Если я не вернусь --- меня пойдут искать.

--- Голова болит?

--- Если бы не болела --- я бы тебя понёс.

--- Какие мы сердитые.

--- От тебя одни проблемы, --- прямо сказал Хитрам.
--- Ноги заживут --- иди к свиньям.

--- А если не заживут?

--- Тогда ползи к свиньям.
Ладно, ладно, не плачь, это я зря сказал...

Вдали показались два бумажных фонаря.
Вскоре из темноты вынырнули и их обладательницы --- две женщины, по обычаю местных ноа закутанные в плащи на голое тело.
Лица женщин скрывались под круглыми травяными шляпами;
в руках они держали зонты-копья.

--- Хитрам, мне это надоело, --- без предисловий обратилась старшая к брату.
--- Это четвёртый корабль за дождь.
Все рыбаки приносят домой рыбу, и только ты приносишь домой трупы моряков.

--- В этот раз одна живая, Огонёк, --- весело откликнулся Хитрам.

--- И толку с неё?
В котелке не сваришь и на рынке не продашь.
Всё, со следующей декады идёшь на рынок, там от тебя больше пользы.

--- Братик, тебя ранили? --- вторая подбежала и стала ощупывать Хитраму голову.
--- Огонёк, у него голова разбита!

--- Была бы у него голова цела, нас бы здесь не было на ночь глядя.
Курочка, иди нарежь веток для носилок...

--- Для двоих?

--- Для одного, глупая!
Этот дурак пешком пойдёт, заслужил!

--- Ничего нового, --- поделился Хитрам с Кхотлам.
Та сконфуженно промолчала.

Впрочем, угроза оказалась пустой.
Когда носилки были готовы и Кхотлам устроилась на них, Огонёк без лишних слов взвалила брата на плечи и потащила домой.

\asterism

--- У тебя волосы пахнут морем, --- прошептал Хитрам.
--- А ещё в них песок.
И у меня на лежанке теперь тоже песок...

--- У тебя на лежанке и был песок! --- тихо возмутилась Кхотлам.

--- Нет, не было, пока ты не приползла!
Кто вообще разрешил тебе ползать?
У тебя ножки больные!

--- Я не спрашиваю чужого разрешения, если хочу поползать!

--- Кхотлам, мы не уснём тут вдвоём.
Лежанка чересчур маленькая.

--- Притащи мою и положи рядом, тогда всё будет в порядке!

Хитрам вздохнул и отправился за лежанкой Кхотлам.
Сёстры мужественно притворялись спящими.

--- Надо было тебя помыть, что ли, --- буркнул мужчина, разглаживая простыни.
--- Всё в песке...
Ай!
Это что, ракушка?
Кхотлам, лесные духи!
Ракушек мне только и не хватало в постели!

--- Спи уже.
Обними меня и спи.
Только ноги осторожно.

Хитрам молча обнял женщину и прижал к себе.

--- Ты тоже это чувствуешь, да? --- поинтересовалась Кхотлам.

--- Что мы как будто знаем друг друга с рождения? --- хмыкнул Хитрам.
--- Сразу почувствовал.
Как будто всё это время я высматривал твой корабль, а сейчас моё ожидание подошло к концу.
Если честно, думал, что все эти мысли из-за удара по голове...

--- Но ты-то меня по голове не бил!

--- Тебя море побило.

--- Разумное объяснение, --- признала Кхотлам и устроилась поудобнее, уткнувшись в Хитрама носом.

\section{И снова платки}

--- Слушай, Кхотлам, --- промямлил Ситрис, --- я тут это...

Кхотлам оторвалась от записей.

--- Что такое, Ситрис?

--- Когда ты, хай, сказала <<тысячу платков>>, я всё-таки думал, что это фигура речи...

--- Нет, это не была фигура речи.
Давай-давай, вышивай, не отвлекайся.
У тебя получается гораздо лучше, чем вчера.

--- Я иглу в руках держу второй раз в жизни!

--- А завтра будет третий.
Вышивай.

Ситрис угрюмо ковырнул иглой ткань и вдруг бросил хитрый взгляд на хозяйку жилища.

--- Кстати, ты обещала, что введёшь меня в Храм.

\ml{$0$}
{--- Я тебе ничего не обещала, Ситрис.}
{``I promised nothing, \Sitris.}
\ml{$0$}
{Тебе, наверное, приснилось.}
{Maybe you dreamt it.''}

--- Обещала, точно обещала.

--- Я сказала <<Посмотрим>>.
Посмотрим --- это не обещание.

--- Ты сказала <<Посмотрим, и может быть>>...

--- Ситрис, --- Кхотлам отодвинула в сторону чернильницу немного резче, чем следовало, --- тебе мало того, что я наложила вето на решение Советов и сделала Тхитрон единственным местом, где ты можешь жить без перспективы быть принесённым в жертву?

--- Я тебе очень-очень благодарен, --- быстро и испуганно протараторил Ситрис.

Увидев отголоски ужаса в чёрных как уголь глазах, Кхотлам смягчилась.

--- Ты здесь не раб, --- сказала она.
--- Не хочешь вышивать --- не вышивай.

--- <<Но тогда никакого Храма>>, --- вкрадчиво закончил за неё Ситрис.

\ml{$0$}
{--- Я этого не говорила.}
{``I've never told that.''}

\ml{$0$}
{--- Но ты ведь имела это в виду, верно?}
{``But you meant that, didn't you?''}

\ml{$0$}
{--- Хватит говорить мне, что я имела в виду.}
{``Stop telling me what I meant.''}

\ml{$0$}
{--- Разве это не очевидно?}
{``Isn't that obvious?}
\ml{$0$}
{Ты --- мне, я --- тебе.}
{It's give and take.''}

Кхотлам вздохнула и встала, торжественно сложив руки.

--- Я клянусь, что порекомендую тебя Храму вне зависимости от того, вышьешь ты платки или нет.
Также я освобождаю тебя от любых моральных обязательств, в свете которых вышивка платков может считаться возвращением морального долга.

Кхотлам села на циновку, задумалась и обмакнула перо в чернила.
Ситрис с несчастным видом посмотрел на стопку хлопковых лоскутков.

--- Ну и рыбина же ты, Кхотлам, --- буркнул он.
--- Помру я с этими платками.

--- Весьма неплохая смерть, --- ухмыльнулась женщина.

Ситрис вздохнул и снова принялся за работу.

\section{Беззащитный сон}

Чханэ была красива.
Кхотлам не могла на нее налюбоваться.
<<Конечно, я вообще не думала, что Лисенок кого-то полюбит, --- улыбнулась она про себя.
--- Не припомню, чтобы он кем-то увлекался.
Видимо, просто время не пришло>>.

Однако кроме красоты было ещё кое-что.
Большая девушка выглядела удивительно беззащитной во сне.
Девочки-близняшки тоже выглядели мило, но беззащитности в них не было.
Кхотлам почти видела тяжёлое детство, кормильцев, которым до ребёнка не было дела, многочисленные проблемы в Храме...

\asterism

Кхотлам аккуратно вынула фалангу из ножен.
Махнула раз, второй...
Тело автоматически встало в нужную позицию.
Ещё одним ударом Кхотлам снесла несколько веток, свесившихся со старой ивы.

Мимо проходил Ситрис, нагруженный амуницией.
Разумеется, он снова вызвался точить и править оружие.
Кхотлам точно знала, что три четверти времени он проводит не с клинками, а с симпатичным оружейником.

--- Кхотлам? --- удивился Ситрис.
--- Давно не видел, чтобы ты махала клинком.
Спарринг?

--- Спарринг, — кивнула Кхотлам.
--- Отобью старое мяско.

--- Не прибедняйся, не такое уж и старое, --- ухмыльнулся воин и, бросив на землю связку клинков, вытащил один.
--- В честь чего праздник-то?

--- Воительница приехала с запада, думаю ее устроить.
Это её фаланга.
Больно уж красивая.

--- Не возражаешь? --- Ситрис протянул руку.

--- Да, конечно, --- Кхотлам протянула фалангу собеседнику.

--- Новая, --- хмыкнул Ситрис, осмотрев оружие.
--- Следы использования есть, но не боевые.
Ею рубили ветки и, кажется, курицу.
Клеймо на пять дождей старше клинка.

Ситрис вернул клинок Кхотлам.
Его глаза смеялись.

--- Опять твои фокусы?

--- Не понимаю, о чём ты, --- непринуждённо ответила купец.
--- Просто хочу поупражняться, вспомнить былое.

--- Понимаю, --- подмигнул Ситрис. 
--- Вставай в позицию.
Сейчас понаделаем боевых следов использования.

--- Главное --- не сломай.
Чужая вещь.

--- Ты за кого меня принимаешь? --- возмутился Ситрис.
--- Я, по-твоему, Маликх, чтобы клинки ломать?

\end{document}
