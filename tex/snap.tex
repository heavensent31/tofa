\documentclass[a4paper,10pt,fleqn]{book}\usepackage{polyglossia}\setdefaultlanguage[babelshorthands=true]{russian}\setotherlanguage{english}\defaultfontfeatures{Ligatures=TeX,Mapping=tex-text}\usepackage{xcolor}\newcommand{\ml}[3]{#2}

% \documentclass[a4paper,10pt,fleqn]{book}\usepackage{cooltooltips}\usepackage{polyglossia}\setdefaultlanguage{english}\setotherlanguage{russian}\defaultfontfeatures{Ligatures=TeX,Mapping=tex-text} \usepackage{xcolor}\definecolor{lightgray}{HTML}{bbbbbb}\color{lightgray}\newcommand{\ml}[3]{\textcolor{black}{#3}}

% ----------------------

\usepackage{amsmath,amssymb,amsfonts,xltxtra,microtype,graphicx,textcomp}
\usepackage{svg}

% ------ GEOMETRY ------

\usepackage[twoside,left=2.5cm,right=3cm,top=3cm,bottom=4cm,bindingoffset=0cm]{geometry}

% ------ FONT ------

\setmainfont{Linux Libertine}
\definecolor{darkblue}{HTML}{003153}

% ------ HYPERLINKS ------

\usepackage{hyperref}
\hypersetup{colorlinks=true, linkcolor=darkblue, citecolor=darkblue, filecolor=darkblue, urlcolor=darkblue}

% ------ EPIGRAPH ------

\usepackage{epigraph}
\renewcommand{\epigraphsize}{\footnotesize}
\epigraphrule=0pt
\epigraphwidth=8cm

\usepackage{etoolbox}
\AtBeginEnvironment{quote}{\itshape}
\makeatletter
\newlength\episourceskip
\pretocmd{\@episource}{\em}{}{}
\apptocmd{\@episource}{\em}{}{}
\patchcmd{\epigraph}{\@episource{#1}\\}{\@episource{#1}\\[\episourceskip]}{}{}
\makeatother

% ------ METADATA ------

\newcommand{\tofaauthor}{\ml{$0$}{Эмиль~Весна}{Emil~Viesn\'{a}}}
\newcommand{\tofatitle}{\ml{$0$}{ЗМЕИНАЯ~ЯМА}{Snake~Pit}}
\newcommand{\tofastarted}{11.11.2021}

% ------ FANCY PAGE STYLE ------

\usepackage{fancyhdr}
\pagestyle{fancy}
\fancyhead[LE,RO]{\thepage}
\fancyhead[LO]{{\small\textsc{\tofatitle}}}
\fancyhead[RE]{{\small\textsc{\tofaauthor}}}
\fancyfoot{}
\fancypagestyle{plain}
{\fancyhead{}
\renewcommand{\headrulewidth}{0mm}
\fancyfoot{}}

% ------ NEW COMMANDS ------

\newcommand{\asterism}{\vspace{1em}{\centering\Large\bfseries$\ast~\ast~\ast$\par}\vspace{1em}}
\newcommand{\textspace}{\vspace{1em}{\centering\Large\bfseries<...>\par}\vspace{1em}}
\newcommand{\FM}{\footnotemark}
\newcommand{\FL}[2]{\footnotetext{См. \textit{\hyperlink{#1}{#2}}.}}
\newcommand{\FA}[1]{\footnotetext{#1 \emph{\ml{$0$}{---~Прим.~авт.}{---~Author.}}}}

\newcommand{\theterm}[3]{\textbf{\hypertarget{#1}{#2}} --- #3}
\newcommand{\thesynonim}[3]{\textbf{#2} --- см. \textit{\hyperlink{#1}{#3}}.}
\newcommand{\theorigin}[3]{\textit{#1:} #2 --- #3}

% ------ DIFFICULT TO WRITE TERMS ------

\newcommand{\Aatris}{\"{A}\={a}tr\v{\i}s}
\newcommand{\Akchsar}{\`{A}kchs\r{a}r}
\newcommand{\Chhammitrai}{Chh\`{a}mm\={\i}tr\^{a}i}
\newcommand{\Chhanei}{Chh\r{a}n\^{e}i}
\newcommand{\Chitram}{Ch\"{\i}tr\'{a}m}
\newcommand{\Choe}{Cho\^{e}}
\newcommand{\choe}{cho\^{e}}
\newcommand{\Harrmatr}{H\r{a}rrm\`{a}tr}
\newcommand{\tHat}{H\={a}t}
\newcommand{\Hei}{H\r{e}i}
\newcommand{\hei}{h\r{e}i}
\newcommand{\Hoesitr}{Ho\`{e}s\={\i}tr}
\newcommand{\hoesitr}{ho\`{e}s\={\i}tr}
\newcommand{\Kchaagotr}{Kch\^{a}\={a}g\~{o}tr}
\newcommand{\kchaagotr}{kch\^{a}\={a}g\~{o}tr}
\newcommand{\Kcharas}{Kch\'{a}r\v{a}s}
\newcommand{\Kchatrim}{Kch\r{a}tr\"{\i}m}
\newcommand{\Kchenoel}{Kch\={e}no\^{e}}
\newcommand{\kchenoel}{kch\={e}no\^{e}}
\newcommand{\Kchenoet}{Kch\"{e}no\^{e}}
\newcommand{\kchenoet}{kch\"{e}no\^{e}}
\newcommand{\Kchoho}{Kch\`{o}h\^{o}}
\newcommand{\Kchotlam}{Kch\={o}tl\'{a}m}
\newcommand{\Kchotris}{Kch\={o}tr\v{\i}s}
\newcommand{\Kihotr}{K\^{\i}h\~{o}tr}
\newcommand{\kihotr}{k\^{\i}h\~{o}tr}
\newcommand{\Kukchuatr}{K\`{u}kchu\={a}tr}
\newcommand{\kukchuatr}{k\`{u}kchu\={a}tr}
\newcommand{\Laaka}{L\={a}\"{a}k\^{a}}
\newcommand{\laaka}{l\={a}\"{a}k\^{a}}
\newcommand{\Lechoe}{L\={e}cho\`{e}}
\newcommand{\lechoe}{l\={e}cho\`{e}}
\newcommand{\Likas}{L\^{\i}k\v{a}s}
\newcommand{\Likchmas}{L\={\i}kchm\r{a}s}
\newcommand{\Likchoe}{L\^{\i}kcho\^{e}}
\newcommand{\Loem}{Lo\~{e}m}
\newcommand{\Maaras}{M\"{a}\={a}r\v{a}s}
\newcommand{\Mitchoe}{M\={\i}tcho\^{e}}
\newcommand{\Mitlikch}{M\={\i}tl\={\i}kch}
\newcommand{\Mitris}{M\={\i}tr\={\i}s}
\newcommand{\Oerchoe}{O\r{e}rcho\^{e}}
\newcommand{\Oerlikch}{O\r{e}rl\'{\i}kch}
\newcommand{\Sat}{S\={a}t}
\newcommand{\Satchir}{S\={a}tch\"{\i}r}
\newcommand{\Satrakch}{S\={a}tr\`{a}kch}
\newcommand{\Seli}{S\r{e}l\={\i}}
\newcommand{\Sirtchu}{S\r{\i}rtch\'{u}}
\newcommand{\Sitris}{S\~{\i}tr\v{\i}s}
\newcommand{\Siusiu}{S\~{\i}u-s\~{\i}u}
\newcommand{\siusiu}{s\~{\i}u-s\~{\i}u}
\newcommand{\Sogcho}{S\"{o}gch\={o}}
\newcommand{\Sotron}{S\~{o}tr\`{o}n}
\newcommand{\Tchalas}{Tch\r{a}l\v{a}s}
\newcommand{\Tchammitr}{Tch\`{a}mm\={\i}tr}
\newcommand{\Tchanoe}{Tch\r{a}no\^{e}}
\newcommand{\Tchartchaahitr}{Tch\~{a}rtch\"{a}\={a}h\r{\i}tr}
\newcommand{\Tchartchu}{Tch\~{a}rtch\'{u}}
\newcommand{\Tchitron}{Tch\"{\i}tr\`{o}n}
\newcommand{\Tchu}{Tch\`{u}}
\newcommand{\tchu}{tch\`{u}}
\newcommand{\Technku}{T\`{e}chnk\r{u}}
\newcommand{\Tesarrokch}{Te's\'{a}rr\r{o}kch}
\newcommand{\Tesatron}{Te's\'{a}tr\v{o}n}
\newcommand{\Traa}{Tr\={a}\"{a}}
\newcommand{\traa}{tr\={a}\"{a}}
\newcommand{\Trai}{Tr\r{a}i}
\newcommand{\trai}{tr\r{a}i}
\newcommand{\TraRenkchal}{Tr\r{a}-R\={e}nkch\'{a}l}
\newcommand{\Trukchual}{Tr\`{u}kchu\r{a}l}


\begin{document}

% ------ TITLE PAGE ------

\begin{titlepage}
{\centering{~\par}\vspace{0.25\textheight}
{\LARGE\tofaauthor}\par
\vspace{1.0cm}\rule{17em}{1pt}\par\vspace{0.3cm}
{\Huge\textsc{\tofatitle}\par}
\vspace{0.3cm}\rule{17em}{2pt}\par\vspace{1.0cm}
{\Large\textit{\ml{$0$}{Фантастический~роман}{Science~fiction}}\par}
\vspace{0.5cm}\asterism\par\vspace{1.0cm}
{\textbf{\ml{$0$}{Начато:}{Started:}}~\tofastarted\par}\vfill
{\Large\ml{$0$}{Создано~в}{Created~by}~\XeLaTeX}\par}
\end{titlepage}

\tableofcontents

\part{Змеиная яма}

\chapter{Песок и ракушки}

\section{Перцовка}

--- Живая, --- пробормотал Хитрам.
--- Точно живая.
Метвецы так не дерутся...

--- Думать надо, --- буркнула Кхотлам, скрестив руки на груди.

--- А я что, по-твоему, делал?
Ты несколько кхамит без дыхания лежала.
От глубокого морского сна только это и помогает.

Кхотлам промолчала.

--- Где я, для начала? --- наконец спросила она.

--- А ты как думаешь?

--- На Короне или на Ките?

--- Ааа, --- протянул Хитрам.
--- Всё настолько плохо?
Это Корона.
На юго-запад --- Валенсия.

--- Меня выбросило к Валенсии? --- ужаснулась Кхотлам.

--- Как видишь.

--- Сколько же я тогда без сознания-то была, --- пробормотала женщина.
--- А моя команда?

--- Больше никто не очнулся.
Сочувствую.

--- Я должна их осмотреть.

Кхотлам попыталась встать и тут же села, вскрикнув и схватившись за лодыжку.

--- У тебя обе ноги покалечены, куда ты собралась? --- Хитрам подскочил и поправил шину.
--- Похоронил я их уже.

--- Как --- уже? --- Кхотлам удивлённо оглядела берег и, увидев свежие могилы, зажала рот рукой.

Хитрам смутился.

--- Да я бы и тебя давно похоронил, только сомнения были.

--- Так зачем ты их закопал?
Может, и они живы!..

Кхотлам проползла на руках до ближайшей могилы и начала с остервенением раскапывать песок.

--- Хай, женщина, --- Хитрам попылался перехватить её руки.
--- Да подожди ты.
Женщина!
Да стой!
Они мертвы, хватит!

Кхотлам остановилась и обратила к Хитраму лицо в слезах.
Тот без сил опустился рядом.

--- Я нашёл вас ранним утром, --- объяснил он.
--- Сейчас уже к вечеру клонится.
На всех, кроме тебя и ещё двоих, были следы тления.
Перцовки я дал всем, очнулась только ты.
Те двое вон там лежат, как и лежали.

Кхотлам легла ничком в песок и зарыдала во весь голос.

\asterism

--- И долго мы тут сидеть будем? --- буркнула Кхотлам.
--- Если у тебя нет сил меня нести --- я поползу.
Только скажи, куда.

--- Я всегда возвращаюсь к вечеру, --- пояснил Хитрам.
--- Сёстры это знают.
Если я не вернусь --- меня пойдут искать.

--- Голова болит?

--- Если бы не болела --- я бы тебя понёс.

--- Какие мы сердитые.

--- От тебя одни проблемы, --- прямо сказал Хитрам.
--- Ноги заживут --- иди к свиньям.

--- А если не заживут?

--- Тогда ползи к свиньям.
Ладно, ладно, не плачь, это я зря сказал...

Вдали показались два бумажных фонаря.
Вскоре из темноты вынырнули и их обладательницы --- две женщины, по обычаю местных ноа закутанные в плащи на голое тело.
Лица женщин скрывались под круглыми травяными шляпами;
в руках они держали зонты-копья.

--- Хитрам, мне это надоело, --- без предисловий обратилась старшая к брату.
--- Это четвёртый корабль за дождь.
Все рыбаки приносят домой рыбу, и только ты приносишь домой трупы моряков.

--- В этот раз одна живая, Огонёк, --- весело откликнулся Хитрам.

--- И толку с неё?
В котелке не сваришь и на рынке не продашь.
Всё, со следующей декады идёшь на рынок, там от тебя больше пользы.

--- Братик, тебя ранили? --- вторая подбежала и стала ощупывать Хитраму голову.
--- Огонёк, у него голова разбита!

--- Была бы у него голова цела, нас бы здесь не было на ночь глядя.
Курочка, иди нарежь веток для носилок...

--- Для двоих?

--- Для одного, глупая!
Этот дурак пешком пойдёт, заслужил!

--- Ничего нового, --- поделился Хитрам с Кхотлам.
Та сконфуженно промолчала.

Впрочем, угроза оказалась пустой.
Когда носилки были готовы и Кхотлам устроилась на них, Огонёк без лишних слов взвалила брата на плечи и потащила домой.

\asterism

--- У тебя волосы пахнут морем, --- прошептал Хитрам.
--- А ещё в них песок.
И у меня на лежанке теперь тоже песок...

--- У тебя на лежанке и был песок! --- тихо возмутилась Кхотлам.

--- Нет, не было, пока ты не приползла!
Кто вообще разрешил тебе ползать?
У тебя ножки больные!

--- Я не спрашиваю чужого разрешения, если хочу поползать!

--- Кхотлам, мы не уснём тут вдвоём.
Лежанка чересчур маленькая.

--- Притащи мою и положи рядом, тогда всё будет в порядке!

Хитрам вздохнул и отправился за лежанкой Кхотлам.
Сёстры мужественно притворялись спящими.

--- Надо было тебя помыть, что ли, --- буркнул мужчина, разглаживая простыни.
--- Всё в песке...
Ай!
Это что, ракушка?
Кхотлам, лесные духи!
Ракушек мне только и не хватало в постели!

--- Спи уже.
Обними меня и спи.
Только ноги осторожно.

Хитрам молча обнял женщину и прижал к себе.

--- Ты тоже это чувствуешь, да? --- поинтересовалась Кхотлам.

--- Что мы как будто знаем друг друга с рождения? --- хмыкнул Хитрам.
--- Сразу почувствовал.
Как будто всё это время я высматривал твой корабль, а сейчас моё ожидание подошло к концу.
Если честно, думал, что все эти мысли из-за удара по голове...

--- Но ты-то меня по голове не бил!

--- Тебя море побило.

--- Разумное объяснение, --- признала Кхотлам и устроилась поудобнее, уткнувшись в Хитрама носом.

\section{И снова платки}

--- Слушай, Кхотлам, --- промямлил Ситрис, --- я тут это...

Кхотлам оторвалась от записей.

--- Что такое, Ситрис?

--- Когда ты, хай, сказала <<тысячу платков>>, я всё-таки думал, что это фигура речи...

--- Нет, это не была фигура речи.
Давай-давай, вышивай, не отвлекайся.
У тебя получается гораздо лучше, чем вчера.

--- Я иглу в руках держу второй раз в жизни!

--- А завтра будет третий.
Вышивай.

Ситрис угрюмо ковырнул иглой ткань и вдруг бросил хитрый взгляд на хозяйку жилища.

--- Кстати, ты обещала, что введёшь меня в Храм.

\ml{$0$}
{--- Я тебе ничего не обещала, Ситрис.}
{``I promised nothing, \Sitris.}
\ml{$0$}
{Тебе, наверное, приснилось.}
{Maybe you dreamt it.''}

--- Обещала, точно обещала.

--- Я сказала <<Посмотрим>>.
Посмотрим --- это не обещание.

--- Ты сказала <<Посмотрим, и может быть>>...

--- Ситрис, --- Кхотлам отодвинула в сторону чернильницу немного резче, чем следовало, --- тебе мало того, что я наложила вето на решение Советов и сделала Тхитрон единственным местом, где ты можешь жить без перспективы быть принесённым в жертву?

--- Я тебе очень-очень благодарен, --- быстро и испуганно протараторил Ситрис.

Увидев отголоски ужаса в чёрных как уголь глазах, Кхотлам смягчилась.

--- Ты здесь не раб, --- сказала она.
--- Не хочешь вышивать --- не вышивай.

--- <<Но тогда никакого Храма>>, --- вкрадчиво закончил за неё Ситрис.

\ml{$0$}
{--- Я этого не говорила.}
{``I've never told that.''}

\ml{$0$}
{--- Но ты ведь имела это в виду, верно?}
{``But you meant that, didn't you?''}

\ml{$0$}
{--- Хватит говорить мне, что я имела в виду.}
{``Stop telling me what I meant.''}

\ml{$0$}
{--- Разве это не очевидно?}
{``Isn't that obvious?}
\ml{$0$}
{Ты --- мне, я --- тебе.}
{It's give and take.''}

Кхотлам вздохнула и встала, торжественно сложив руки.

--- Я клянусь, что порекомендую тебя Храму вне зависимости от того, вышьешь ты платки или нет.
Также я освобождаю тебя от любых моральных обязательств, в свете которых вышивка платков может считаться возвращением морального долга.

Кхотлам села на циновку, задумалась и обмакнула перо в чернила.
Ситрис с несчастным видом посмотрел на стопку хлопковых лоскутков.

--- Ну и рыбина же ты, Кхотлам, --- буркнул он.
--- Помру я с этими платками.

--- Весьма неплохая смерть, --- ухмыльнулась женщина.

Ситрис вздохнул и снова принялся за работу.

\chapter{Комната без окон}

\section{Последнее желание}

Увидев Хитрама, Кхотлам зарыдала впервые с момента, когда её завели в тёмную комнату без окон.
Мужчина бросился к ней.

--- Кхотлам, я клянусь --- я найду способ тебя вытащить.
Не падай духом.

--- Нет, --- всхлипнула Кхотлам.
--- Если с тобой что-то случится, Лисёнок и Хмурчик останутся совсем одни.
Хмурчик только-только пришёл в Храм, а Лисёнок совсем сосунок.
Поклянись, что ты не будешь меня спасать.

--- Ты с ума сошла?
Я не могу пообещать такое.

--- Хитрам! --- сказала купец.
--- Это моё последнее желание!
Ты откажешь мне?

Хитрам долго смотрел в полные слёз зелёные глаза.

--- Хорошо, --- наконец сказал он.
--- Я клянусь, что не буду тебя спасать.
Но если кто-то захочет тебя вытащить, я помогу чем смогу.

--- Если тебя поймают, я тебя не прощу никогда.
Ты слышишь?
А теперь иди отсюда.

Кхотлам оттолкнула Хитрама.

--- Пёрышко, жизнь моя...

--- Уйди к свиньям, Пловец!
Мне даже смотреть на тебя больно!
Уйди!

Хитрам едва успел увернуться от тяжёлой деревянной скамеечки.
Скамеечка глухо треснула о каменную стену.

--- Я тебя не оставлю, --- тихо сказал Хитрам и вышел из комнаты.

\section{Молоко}

За дверью была едва слышная перебранка.

--- Возьми это золото.
Тебе лишнее, что ли?

--- Я не возьму.
Плевать на обычай.
Если хотите, можете его выкинуть.
Всё, что мне нужно --- это побыть с ней.

--- Хитрам, хватит.
И возьми эту гребаную пирамидку.
Я не могу оставить её себе, не могу выкинуть, ты должен её взять!

--- Я просто посижу с ней, пожалуйста, --- убеждал Хитрам.
--- Какой это может причинить вред?

--- Хитрам, ты знаешь правила, --- устало говорил Первый.
--- Никаких посиделок.
Я вам свидание устроил просто из уважения к Кхотлам, хотя должен был отказать.

--- Я никуда не уйду.
Да не нужна мне эта пирамидка!

Что-то громко звякнуло о стену.
Похоже, Хитрам швырнул тяжёлый кусочек золота подальше.

--- Всё, я взял, доволен?
А теперь дай мне побыть с ней.
Если надо, я буду сидеть перед дверью.

--- Не будешь!
Если ты не прекратишь бесчинствовать, я велю Кхохо тебя вывести!
Она тебя жалеть не будет, ты её знаешь!

--- Пусть попробует.

--- Кхохо!
Иди-ка сюда.
Проводи городского на выход.
И скажи остальным, что на ближайший дождь он отлучён от здания храма, за исключением лазарета.
Если будет сопротивляться --- обеспечь ему лазарет.

--- Слушай, парень, не дури, --- проворчала Кхохо.
--- Я ведь тебе правда сейчас руки переломаю.
Давай, давай, не зли меня, у меня плохой день сегодня.

--- У тебя всю жизнь плохой день, --- буркнул в ответ Хитрам, но, судя по удаляющимся шагам, всё-таки послушался.
Кхотлам выдохнула.

\asterism

Сейчас, когда ей принесли новую лампу, Кхотлам разглядела каморку.
Над лежанкой на уровне головы были круглые пятна, протёртые сотнями затылков.
Пол и стены были исчерчены тысячами посланий.
Рисунки, имена любимых, собственные имена, молитвы, проклятия, исповеди.

<<Передайте Митхэ, что я её люблю>>.

Цветочек.
Бабочка.
Неуклюже расставивший руки человечек.

<<Будь ты проклят, предатель>>.

(неразборчивая надпись на абисе)

<<Невезение чистой воды>>.
Классический цатрон, тоновая поэтика, изящный саркастический слог, ровный каллиграфический почерк, выдающий образованного человека.

Какая-то молитва змеистым письмом на хесетроне --- скорее всего, идол.

<<Я невиновен>>.
При виде этого послания у Кхотлам ёкнуло сердце, и она прекратила разглядывать надписи.

Болела набухшая грудь.
Кхотлам тёрла её, но боль не проходила.
<<Молоко пропадёт, --- с грустью думала она.
--- Я ведь даже выкормить не успела ребёнка.
Что я скажу Митхэ, когда встречу её \emph{там}?
Наобещала и не выполнила>>.

\section{Три мучительных декады}

Кхотлам сидела, повернувшись лицом в угол и прислонившись лбом к стене.
Она сильно осунулась --- глаза запали, нос заострился, волосы поредели, в них прибавилось седины.
Трукхвал с грустью посмотрел на нетронутый поднос с едой.

--- Кхотлам, поешь, --- робко сказал он.
--- Мидии --- это то немногое, что я действительно умею готовить.
Хочешь, принесу свежие?

--- Когда меня принесут в жертву? --- глухо спросила женщина.

--- Полагаю, когда будет нужно.

--- Трукхвал, я уже три декады сижу здесь.

--- Подожди, --- нахмурился библиотекарь.
--- Как это?

--- А вот так.
Мимо меня прошло пятеро детей.
Каждого я утешала, укладывала спать и провожала на крышу.
Я больше так не могу.
С меня хватит.
Если вы не принесете меня в жертву в ближайшее время, я перережу себе вены.

--- Может, для тебя оно и к лучшему, --- тихо сказал Трукхвал.
--- Не так худо, как на крыше.

--- Вы серьёзно ждёте, пока я покончу с собой? --- Кхотлам обратила страшные, запавшие, с синяками глаза к жрецу.
Тот вздрогнул.
--- Это ваш план, как не марать руки в моей крови?

--- Кхотлам, я не совсем в курсе ситуации, но я тебя понял.
Обещаю, я поговорю со жрецами.
Если никто не вызовется, в следующий знак кирпича я сам поведу тебя на крышу.
Поешь хоть немного.

--- Убери поднос, пожалуйста.
Меня тошнит от запаха мидий.

Трукхвал поднял поднос и задумчиво слопал пару жареных моллюсков.

--- Ты же ведь невиновна, да?

--- Какая теперь разница?

--- Для твоего мужчины, видимо, есть разница.
Он уже ввязался в три драки из-за того, что кто-то назвал тебя Разрушителем.
Пока что отделался сломанным носом и десятком синяков.
Крестьяне судачат, что он ненормальный и что ты его приворожила.
Я видел его сегодня на поле.
Он кормил грудью твоего ребёнка в перерывах между пахотой.
Если честно, немного тебе завидую --- я так и не встретил человека, который любил бы меня так же.

--- Он с тобой говорил, да?

--- Хай, он спросил, не могу ли я тебе как-то помочь.
Похоже, он у всех это спрашивает.

Кхотлам промолчала.

--- Ситрис сказал, что тебя, скорее всего, подставили.
Твои домашние в один голос твердят, что ты никогда бы этого не сделала.

Молчание.

--- Кхотлам, я понимаю, что ты меня почти не знаешь, --- проникновенно сказал Трукхвал.
--- Я простой жрец.
<<Я не оставлю без внимания ни ложь, ни заблуждение, и скажу во всеуслышание то, что считаю истиной>>.
Я здесь не ради тебя, я здесь ради истины.
Просто посмотри мне в глаза и ответь --- ты действительно сделала то, за что тебя осудили?

--- Ищи истину в другом месте, жрец, --- ответила Кхотлам.

Трукхвал поклонился и вышел за дверь.

\section{Следопыт}

--- Ты чего такой радостный, Трукхвал?

--- Я это, хай, давно днём по лесу не гулял, --- смущённо сказал жрец.

--- Ты вообще из храма выходишь?

--- Вечерком, на храмовые земли.
Больше люблю на крыше сидеть с чашей отвара или чего-нибудь покрепче.

--- Ученика тебе надо, старик.
Ученик тебя будет гонять по всем окрестностям, пока жрецом станет.

--- Да какой из меня учитель.
Загублю только детскую душу...
Так, а вот, кажется, и тот самый поворот.
Сиртху-лехэ, следы есть?

--- Хаяй, давненько я этим не занимался, --- весело проскрипел Сиртху-лехэ, присев на корточки.
--- Отобью старое мяско...
Да, следы борьбы.
Мы пришли.

--- Я тоже вижу, --- кивнул Ситрис.
--- Повозка лежала вот здесь.

--- А ещё здесь было восемь человек, --- Сиртху поводил руками по земле, нежно раздвигая пальцами молодую поросль.
--- Двое погибли, их тела оттащили в сторону... хэ, один погибший потерял шнурок со штанов.

--- Давай мне, я попробую определить поселение, в котором были пошиты штаны, --- сказал Трукхвал.

--- Тебе не понадобится шнурок, жрец.
Вон там, в двадцати шагах, свежая могила.
Думаю, в ней лежат целые штаны вместе с их владельцем.
Пошли, поможете мне раскопать.

--- Как ты это делаешь, лехэ? --- восхитился Ситрис.
--- Я бы ни за что не увидел здесь могилу.
Научи-ка.

--- Ситрис, давай потом, --- буркнул Трукхвал.
--- Времени у нас в обрез.
Бери мотыгу и копай.

--- Только осторожно, --- предупредил Сиртху.
--- Старайтесь не повредить тело, оно многое может рассказать...

\section{Сообщники}

Ситрис разложил на траве найденное в карманах.
Трукхвал молниеносно зарисовывал на пергаменте детали одежды, записывал приметы полуразложившихся трупов.
Это были две женщины.
Сиртху задумчиво разглядывал их.

--- Кто-то аккуратно положил их в обнимку, связав руки пеньковой верёвкой, --- сказал он.
--- Так хоронят любовников в Ближнереках.
Кстати, длина верёвки может говорить о том, сколько они были вместе.
Хаяй...
Больше десяти пядей.
Парочка со стажем.

--- Ты уверен? --- перо Трукхвала зашевелилось с новой силой.

--- Они точно любовницы, --- поддержал Ситрис.
--- Они сражались спиной к спине.
Посмотрите на эти обломки стрел.
Они уже запачкались, но до сих пор видно, что обломки совпадают.
Арбалетную стрелу пустили с дерева.
Той, что пониже, пробило горло, а той, что повыше --- сердце.

--- Похоже на то, --- согласился Сиртху, пощупав пальцем стрелы.
--- Сколько живу, такое вижу впервые.
Глаза у тебя хорошие, воин.

--- Записал, --- ответил Трукхвал.
--- Можете ещё что-то сказать?

--- Больше ничего, --- сказал Сиртху.
--- Татуировок, приметных знаков на коже не вижу.
Украшения и одежда, на мой взгляд, совершенно обычные для тхитронцев.

--- Сражались спиной к спине и были убиты с дерева --- то есть это не нападавшие, --- Трукхвал начал загибать пальцы.
--- А вот похоронил их один из нападавших, потому что караванщики разбежались.
Он родом из Ихслантхара и хорошо знал убитых.
Я ничего не забыл?

--- Всё на месте.
Ещё могу сказать, что он очень сильный --- я бы не смог затянуть толстую пеньковую верёвку в такой тугой узел.

--- Нападавшим помогал кто-то из городских, --- заключил Трукхвал, смотал пергамент в трубочку и засунул в тубус.

--- Всё-таки Кхотлам? --- грустно спросил Ситрис.

--- Если и так, то действовала она не одна.
Это явно что-то новенькое для расследования.
Выйдем на этих людей --- узнаем больше.
Сиртху-лехэ, прикопай этих бедняжек и поставь метку на дереве, потом попробуем найти близких.
А ты, Ситрис, забери их вещи, только не потеряй, я сделал опись.
И давайте ещё пошаримся по окрестностям, хотя бы до полудня.
Чем больше деталей --- тем лучше.

\section{Лесорубная}

--- Здравствуй, --- приветливо начал Трукхвал.
--- У меня есть к тебе пара вопросов насчёт...

Ситрис едва успел прикрыть Трукхвала собой.
Топор женщины вонзился ему в наруч чуть пониже локтя, промяв дерево, толстую кожу и металлическое кольцо.
Хлынула кровь.
Нападавшая тем временем, петляя как заяц, бросилась в сторону реки.

--- Оставь меня, Трукхвал, я не умираю, --- прохрипел Ситрис, кривясь от боли.
--- Беги, она нужна нам живой!
Да подожди, куда побежал, оружие возьми!
Гений тактики, идолы тебя дери...

Трукхвал подхватил фалангу Ситриса и бросился следом, подгоняемый отборными воинскими фразеологизмами.

--- Так, эта дорога ведёт на улицу Горьких Конфет, туда она не побежит, --- бормотал Трукхвал, скидывая на бегу неудобную робу и жреческие сумки.
--- Надо перехватить её в районе Безымянного переулка...
Лесные духи, зачем я вообще в это ввязался.
Истина для него главное, видите ли.
Сидел бы себе в библиотеке и сидел...

Едва выбежав на Безымянный переулок, Трукхвал нос к носу столкнулся с нападавшей.
Только сейчас он понял, как выдохся от непривычной нагрузки.
Жрец не имел ни малейшего понятия, как сражаться в таком состоянии.
Акула Ситриса к тому же оказалась чересчур тяжела для его рук;
Трукхвал торопливо вытряхнул на землю балансир, что не особо улучшило ситуацию.

--- Бросай оружие! --- крикнул он, вскинув двумя руками Акулу.
Из окон высовывались головы;
местные жители во все глаза наблюдали за разворачивающимся действом.
Трукхвал уже пожалел, что снял жреческое одеяние.
Как жреца его бы немедленно защитили, но библиотекаря на такой глухой окраине Тхитрона мало кто знал в лицо.

Нападавшая без предисловий махнула топором.
Трукхвал, едва успев парировать удар, отлетел и перекувырнулся два раза.
Промелькнула мысль, что будь на месте Акулы любая другая фаланга --- удар бы просто её сломал.
Кое-как встав на ноги, Трукхвал снова принял боевую позицию.

--- Повоять не буу! --- выкрикнул жрец, едва шевеля разбитыми губами.

На него обрушилась ещё серия ударов.
Но на этот раз библиотекарь немного успокоился --- открылось второе дыхание, тело само вспомнило давно забытые движения.
Нападавшая не отличалась разнообразием приёмов, полагаясь на скорость и силу, а Трукхвал, при всей его неуклюжести, неплохо умел фехтовать и отклонять удары --- это в какой-то степени уравняло шансы.
Наблюдающие за сражением крестьяне, судя по всему, начали делать ставки и болеть за своих фаворитов.

Вдруг нападавшая сначала зарычала, а потом, упав на колени, взревела от боли;
из каждого её бедра торчало по стреле.
Сиртху-лехэ, наложив на тетиву третью стрелу, осторожно приблизился к женщине и ударом ноги повалил её в дорожную пыль.

--- Помогите её связать, любезные, что смотрите, --- обратился он к зрителям.
--- Спектакль окончен.

\section{Слушание}

--- Свидетельство бывшего бандита и человека, который обязан Кхотлам жизнью.

--- Во-первых, я требую извинений перед Нижним этажом.
На Нижнем этаже нет бывших бандитов, есть только воины.

--- Приношу извинения Ситрису ар'Эр, --- процедила сквозь зубы женщина.

--- Во-вторых, Ситрис и Сиртху были всего лишь исполнителями и помощниками при расследовании.
Курировал расследование достойный доверия человек, не имеющий личной заинтересованности.

--- Кто?

--- Трукхвал ар'Хэ.

--- Жрец?
Что вся эта морковка значит?
Вначале вы тянете с жертвоприношением три декады, а потом жрец проводит дополнительное расследование!

--- У вас есть причины предполагать заинтересованность Трукхвала ар'Хэ? --- ледяным тоном спросил Первый жрец, пропустив выпад мимо ушей.

--- Насколько нам известно, он библиотекарь, --- заметил старшина камнерезов.
--- Мы крайне редко видим его в городе.
Какой резон библиотекарю проводить дополнительное расследование?
Давайте послушаем его.

--- Я... хай, --- смутился Трукхвал.
--- Мне просто показалось, что не вяжется что-то в этой истории...

--- И всё?

--- Всё.

--- Кто посоветовал вам начать расследование? --- требовательно спросила старшина Сада.
--- Кхотлам?
Первый жрец?
Другие люди?

--- Хитрам ар'Кхир, насколько я знаю, он мужчина Кхотлам.

--- Этот ненормальный ко всем приставал с этим вопросом, --- кивнула старшина.
--- В общем, понятно.
Что сказали Первый жрец и сама Кхотлам?

--- Кхотлам отказалась отвечать на мои вопросы, в довольно грубой форме.
Предупреждать Первого жреца я не стал.
Моя ошибка, готов понести наказание.

--- То есть Первый жрец узнал о твоём расследовании пост-фактум?

--- Боюсь, что так.

Делегаты зашептались.

--- Мы позволяем Храму определённые вольности, но это ни в какие ворота!
Полный беспорядок!

--- Знал ли Трукхвал, что Кхотлам содержится в Комнате Без Окон уже три декады?

--- Я узнал об этом в день начала расследования, --- объяснил Трукхвал.
--- Она сама мне об этом сказала, пригрозив совершить самоубийство, если вопрос не будет решён.

--- Я прошу прощения, --- поднялась в зале рука.
--- Насчёт самоубийства.
Можешь процитировать?

--- <<Если меня не принесут в жертву в ближайшее время, я перережу себе вены>>.
Это дословно.

--- Как получилось, что ты не знал о пребывании Кхотлам в Комнате Без Окон?

--- Обычно я не имею никакого отношения к Комнате Без Окон, --- сказал Трукхвал.
--- Но в тот день подошла моя очередь идти на кухню и, соответственно, разносить еду.
Чаще всего меня просто пропускают в очереди, потому что я сплю или переписываю.

В толпе раздались смешки.

--- Типичный библиотекарь...

--- Я обратил внимание, что она отказалась от еды, и судя по её состоянию, уже не впервые, --- продолжил жрец.
--- Поэтому решил поговорить.

--- Так, значит, Первый жрец, ты признаёшь, что держал кутрапа живым три декады? --- сузив глаза, прошипела старшина Сада.

--- Признаю, --- Первый демонстративно снял перьевую корону и положил на кафедру.
--- Выборы нового Первого жреца уже назначены на послезавтра, все необходимые для передачи полномочий приготовления мной сделаны и формальности соблюдены.
Сейчас же, с вашего позволения, я хочу дать слово Трукхвалу, чтобы он осветил подробности своего расследования.

Первый кивнул присутствующим, набросил на плечи дорожный плащ и вышел из зала.
В наступившей ошеломлённой тишине прозвучал лишь шелест пергамента Трукхвала и весёлый топот оленьих копыт, удаляющийся в сторону южных ворот.

\section{На выход}

--- Кхотлам, на выход.

Кхотлам вздохнула и, расправив штаны, поднялась на ноги.
Ну вот и всё.
Отбегалась, отплавалась, отсмеялась, отлюбила.
Пора зажигать фонарик.

Кхотлам уже не волновала мысль о том, что сейчас ей будет ужасно больно.
Это был просто факт.
Она переступила порог комнаты, и её ослепил солнечный свет, бьющий в окна храма.
<<Какое счастье, солнце пришло со мной попрощаться>>, --- подумала она.

Каждый шаг давался с трудом.
Она едва заставляла себя переставлять ноги.
Жрец шел рядом.
Но едва ее нога встала на ступеньку, ведущую на крышу, как сопровождающий мягко перехватил ее за локоть.

--- Тебе туда, --- сказал он, показав на лестницу в зал.
--- Совет тебя оправдал, ты свободна.

Кхотлам села там же, где стояла.
Её ногти до крови впились в плечи, из пересохшего горла вырвался глухой стон.
Жрец, покачав головой, поднял женщину на руки и понёс в зал.

\section{Бывший Первый}

--- Получила письмо от Первого.

--- И где он сейчас?

--- Среди Людей Золотой Пчелы.
Ни один город, разумеется, его не принял бы с такой репутацией.
А Бродячий Храм взял сразу.
Они любят тех, кто верен клятве, а не букве закона.

\chapter{Лис и Оцелот}

\section{Беззащитный сон}

Чханэ была красива.
Кхотлам не могла на нее налюбоваться.
<<Конечно, я вообще не думала, что Лисенок кого-то полюбит, --- улыбнулась она про себя.
--- Не припомню, чтобы он кем-то увлекался.
Видимо, просто время не пришло>>.

Однако кроме красоты было ещё кое-что.
Большая девушка выглядела удивительно беззащитной во сне.
Девочки-близняшки тоже выглядели мило, но беззащитности в них не было.
Кхотлам почти видела тяжёлое детство, кормильцев, которым до ребёнка не было дела, многочисленные проблемы в Храме...

\asterism

Кхотлам аккуратно вынула фалангу из ножен.
Махнула раз, второй...
Тело автоматически встало в нужную позицию.
Ещё одним ударом Кхотлам снесла несколько веток, свесившихся со старой ивы.

Мимо проходил Ситрис, нагруженный амуницией.
Разумеется, он снова вызвался точить и править оружие.
Кхотлам точно знала, что три четверти времени он проводит не с клинками, а с симпатичным оружейником.

--- Кхотлам? --- удивился Ситрис.
--- Давно не видел, чтобы ты махала клинком.
Спарринг?

--- Спарринг, — кивнула Кхотлам.
--- Отобью старое мяско.

--- Не прибедняйся, не такое уж и старое, --- ухмыльнулся воин и, бросив на землю связку клинков, вытащил один.
--- В честь чего праздник-то?

--- Воительница приехала с запада, думаю ее устроить.
Это её фаланга.
Больно уж красивая.

--- Не возражаешь? --- Ситрис протянул руку.

--- Да, конечно, --- Кхотлам протянула фалангу собеседнику.

--- Новая, --- хмыкнул Ситрис, осмотрев оружие.
--- Следы использования есть, но не боевые.
Ею рубили ветки и, кажется, курицу.
Клеймо на пять дождей старше клинка.

Ситрис вернул клинок Кхотлам.
Его глаза смеялись.

--- Опять твои фокусы?

--- Не понимаю, о чём ты, --- непринуждённо ответила купец.
--- Просто хочу поупражняться, вспомнить былое.

--- Понимаю, --- подмигнул Ситрис. 
--- Вставай в позицию.
Сейчас понаделаем боевых следов использования.

--- Главное --- не сломай.
Чужая вещь.

--- Ты за кого меня принимаешь? --- возмутился Ситрис.
--- Я, по-твоему, Маликх, чтобы клинки ломать?

\end{document}
