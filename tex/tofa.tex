%\documentclass[a4paper,12pt,fleqn]{book}\usepackage{polyglossia}\setdefaultlanguage[babelshorthands=true]{russian}\setotherlanguage{english}\defaultfontfeatures{Ligatures=TeX,Mapping=tex-text}\usepackage{xcolor}\newcommand{\ml}[3]{#2}

\documentclass[a4paper,12pt,fleqn]{book}\usepackage{cooltooltips}\usepackage{polyglossia}\setdefaultlanguage[babelshorthands=true]{russian}\setotherlanguage{english}\defaultfontfeatures{Ligatures=TeX,Mapping=tex-text} \usepackage{xcolor}\definecolor{lightgray}{HTML}{bbbbbb}\color{lightgray}\newcommand{\ml}[3]{\textenglish{\textcolor{black}{#3}}}

% ----------------------

\setlength{\headheight}{15pt}

\usepackage{amsmath,amssymb,amsfonts,xltxtra,microtype,graphicx,textcomp}
\usepackage{svg}

% ------ GEOMETRY ------

\usepackage[twoside,left=2.5cm,right=3cm,top=3cm,bottom=4cm,bindingoffset=0cm]{geometry}

% ------ FONT ------

\usepackage{ebgaramond}
\definecolor{darkblue}{HTML}{003153}

% ------ HYPERLINKS ------

\usepackage{hyperref}
\hypersetup{colorlinks=true, linkcolor=darkblue, citecolor=darkblue, filecolor=darkblue, urlcolor=darkblue}

% ------ EPIGRAPH ------

\usepackage{epigraph}
\renewcommand{\epigraphsize}{\footnotesize}
\epigraphrule=0pt
\epigraphwidth=8cm

\usepackage{etoolbox}
\AtBeginEnvironment{quote}{\itshape}
\makeatletter
\newlength\episourceskip
\pretocmd{\@episource}{\em}{}{}
\apptocmd{\@episource}{\em}{}{}
\patchcmd{\epigraph}{\@episource{#1}\\}{\@episource{#1}\\[\episourceskip]}{}{}
\makeatother

% ------ METADATA ------

\newcommand{\tofaauthor}{\ml{$0$}{Эмиль~Весна}{Emil~Viesn\'{a}}}
\newcommand{\tofatitle}{\ml{$0$}{ДАЛЁКИЕ~РЕЧИ}{Talks~of~Faraway}}
\newcommand{\tofastarted}{13.08.2012}

% ------ FANCY PAGE STYLE ------

\usepackage{fancyhdr}
\pagestyle{fancy}
\fancyhead[LE,RO]{\thepage}
\fancyhead[LO]{{\small\textsc{\tofatitle}}}
\fancyhead[RE]{{\small\textsc{\tofaauthor}}}
\fancyfoot{}
\fancypagestyle{plain}
{\fancyhead{}
\renewcommand{\headrulewidth}{0mm}
\fancyfoot{}}

% ------ NEW COMMANDS ------

\newcommand{\asterism}{\vspace{1em}{\centering\Large\bfseries$\ast~\ast~\ast$\par}\vspace{1em}}
\newcommand{\textspace}{\vspace{1em}{\centering\Large\bfseries<...>\par}\vspace{1em}}
\newcommand{\FM}{\footnotemark}
\newcommand{\FL}[2]{\footnotetext{См. \textit{\hyperlink{#1}{#2}}.}}
\newcommand{\FA}[1]{\footnotetext{#1 \emph{\ml{$0$}{---~Прим.~авт.}{---~Author.}}}}

\newcommand{\theterm}[3]{\textbf{\hypertarget{#1}{#2}} --- #3}
\newcommand{\thesynonim}[3]{\textbf{#2} --- см. \textit{\hyperlink{#1}{#3}}.}
\newcommand{\theorigin}[3]{\textit{#1:} #2 --- #3}

% ------ DIFFICULT TO WRITE TERMS ------

\newcommand{\Aatris}{\"{A}\={a}tr\v{\i}s}
\newcommand{\Akchsar}{\`{A}kchs\r{a}r}
\newcommand{\Chhammitrai}{Chh\`{a}mm\={\i}tr\^{a}i}
\newcommand{\Chhanei}{Chh\r{a}n\^{e}i}
\newcommand{\Chitram}{Ch\"{\i}tr\'{a}m}
\newcommand{\Choe}{Cho\^{e}}
\newcommand{\choe}{cho\^{e}}
\newcommand{\Harrmatr}{H\r{a}rrm\`{a}tr}
\newcommand{\tHat}{H\={a}t}
\newcommand{\Hei}{H\r{e}i}
\newcommand{\hei}{h\r{e}i}
\newcommand{\Hoesitr}{Ho\`{e}s\={\i}tr}
\newcommand{\hoesitr}{ho\`{e}s\={\i}tr}
\newcommand{\Kchaagotr}{Kch\^{a}\={a}g\~{o}tr}
\newcommand{\kchaagotr}{kch\^{a}\={a}g\~{o}tr}
\newcommand{\Kcharas}{Kch\'{a}r\v{a}s}
\newcommand{\Kchatrim}{Kch\r{a}tr\"{\i}m}
\newcommand{\Kchenoel}{Kch\={e}no\^{e}}
\newcommand{\kchenoel}{kch\={e}no\^{e}}
\newcommand{\Kchenoet}{Kch\"{e}no\^{e}}
\newcommand{\kchenoet}{kch\"{e}no\^{e}}
\newcommand{\Kchoho}{Kch\`{o}h\^{o}}
\newcommand{\Kchotlam}{Kch\={o}tl\'{a}m}
\newcommand{\Kchotris}{Kch\={o}tr\v{\i}s}
\newcommand{\Kihotr}{K\^{\i}h\~{o}tr}
\newcommand{\kihotr}{k\^{\i}h\~{o}tr}
\newcommand{\Kukchuatr}{K\`{u}kchu\={a}tr}
\newcommand{\kukchuatr}{k\`{u}kchu\={a}tr}
\newcommand{\Laaka}{L\={a}\"{a}k\^{a}}
\newcommand{\laaka}{l\={a}\"{a}k\^{a}}
\newcommand{\Lechoe}{L\={e}cho\`{e}}
\newcommand{\lechoe}{l\={e}cho\`{e}}
\newcommand{\Likas}{L\^{\i}k\v{a}s}
\newcommand{\Likchmas}{L\={\i}kchm\r{a}s}
\newcommand{\Likchoe}{L\^{\i}kcho\^{e}}
\newcommand{\Loem}{Lo\~{e}m}
\newcommand{\Maaras}{M\"{a}\={a}r\v{a}s}
\newcommand{\Mitchoe}{M\={\i}tcho\^{e}}
\newcommand{\Mitlikch}{M\={\i}tl\={\i}kch}
\newcommand{\Mitris}{M\={\i}tr\={\i}s}
\newcommand{\Oerchoe}{O\r{e}rcho\^{e}}
\newcommand{\Oerlikch}{O\r{e}rl\'{\i}kch}
\newcommand{\Sat}{S\={a}t}
\newcommand{\Satchir}{S\={a}tch\"{\i}r}
\newcommand{\Satrakch}{S\={a}tr\`{a}kch}
\newcommand{\Seli}{S\r{e}l\={\i}}
\newcommand{\Sirtchu}{S\r{\i}rtch\'{u}}
\newcommand{\Sitris}{S\~{\i}tr\v{\i}s}
\newcommand{\Siusiu}{S\~{\i}u-s\~{\i}u}
\newcommand{\siusiu}{s\~{\i}u-s\~{\i}u}
\newcommand{\Sogcho}{S\"{o}gch\={o}}
\newcommand{\Sotron}{S\~{o}tr\`{o}n}
\newcommand{\Tchalas}{Tch\r{a}l\v{a}s}
\newcommand{\Tchammitr}{Tch\`{a}mm\={\i}tr}
\newcommand{\Tchanoe}{Tch\r{a}no\^{e}}
\newcommand{\Tchartchaahitr}{Tch\~{a}rtch\"{a}\={a}h\r{\i}tr}
\newcommand{\Tchartchu}{Tch\~{a}rtch\'{u}}
\newcommand{\Tchitron}{Tch\"{\i}tr\`{o}n}
\newcommand{\Tchu}{Tch\`{u}}
\newcommand{\tchu}{tch\`{u}}
\newcommand{\Technku}{T\`{e}chnk\r{u}}
\newcommand{\Tesarrokch}{Te's\'{a}rr\r{o}kch}
\newcommand{\Tesatron}{Te's\'{a}tr\v{o}n}
\newcommand{\Traa}{Tr\={a}\"{a}}
\newcommand{\traa}{tr\={a}\"{a}}
\newcommand{\Trai}{Tr\r{a}i}
\newcommand{\trai}{tr\r{a}i}
\newcommand{\TraRenkchal}{Tr\r{a}-R\={e}nkch\'{a}l}
\newcommand{\Trukchual}{Tr\`{u}kchu\r{a}l}


\begin{document}

% ------ TITLE PAGE ------

\begin{titlepage}
{\centering{~\par}\vspace{0.25\textheight}
{\LARGE\tofaauthor}\par
\vspace{1.0cm}\rule{17em}{1pt}\par\vspace{0.3cm}
{\Huge\textsc{\tofatitle}\par}
\vspace{0.3cm}\rule{17em}{2pt}\par\vspace{1.0cm}
{\Large\textit{\ml{$0$}{Фантастический~роман}{Science~fiction}}\par}
\vspace{0.5cm}\asterism\par\vspace{1.0cm}
{\textbf{\ml{$0$}{Начато:}{Started:}}~\tofastarted\par}\vfill
{\Large\ml{$0$}{Создано~в}{Created~by}~\XeLaTeX}\par}
\end{titlepage}

\tableofcontents

\part{Путь}

\part{Пристанище}

\chapter*{Интерлюдия XIV. Страна снов}
\addcontentsline{toc}{chapter}{Интерлюдия XIV. Страна снов}

Сон --- это благословение.
Сон --- это проклятие.
Сон --- это лекарь и убийца.
Сон --- это начало и конец.
Рождённые просыпаются, не заснув;
умершие засыпают и не просыпаются.

У снов есть страна, и эта страна --- сон земли.
Страна снов земель есть сон Вместилища;
сны Вместилища так велики, что бесконечно малы, и так длинны, что бесконечно коротки.
Потому любой может путешествовать по стране снов беспрепятственно.

В стране снов день и ночь зависят не от светил, а от земли;
куда ты встанешь, такое [время суток] и будет;
сделай шаг за ворота, сверни с тропы, выйди из леса на песчаный берег --- и день сменится ночью.

В стране снов у дорог нет перекрёстков, только развилки;
если ты идёшь по одной дороге, то не видишь другую;
когда же дорога делает петлю, то она пропадает у тебя за спиной.

В стране снов все [существа и предметы] не имеют привычной нам формы и не делятся на живое и неживое, говорящих и кричащих\FM, птиц и зверей;
\FA{Сапиентов и животных, не относящихся к сапиентам.}
каждый там может быть одновременно жуком и птицей-попрошайкой, деревянной игрушкой и человеком, мужчиной, женщиной и шаманом, одним говорящим и другим [говорящим];
но все знают, кто есть кто, и называют друг друга по именам.

В стране снов нет слов и иероглифов;
там каждое слово значит всё и не значит ничего;
всё, что написано, есть мудрость и бессмыслица одновременно.

В стране снов так же есть любовь, ненависть, радость и грусть, страх и покой;
те, кто там живут, могут болеть, плакать, испытывать боль;
но делают они это только для того, чтобы понимать друг друга, ведь они не могут общаться, используя голос и письмо.

В стране снов единственное оружие --- ум, единственная ценность --- красота, единственный инструмент для навигации --- чувства;
глупцы там быстро впадают в отчаяние и оцепеневают, подобно мышам, попавшим во власть змеи;
те же, кто не познал любви, ненависти, радости и грусти, вечно блуждают в паутине дорог, не встречая никого и не встречаясь никому.

\chapter{[-] Изгнание}

\section{[-] Рикошет}

С самого рождения сапиент связан особым неписанным договором.
Не со Старой Личинкой, как полагает Бродячий Народ, но со средой, в которой он родился.
Этот договор определяется многим --- биологическим видом, собственными генами, культурой, кормильцами.
И всю жизнь, с самых первых дней сапиент следует оговорённой стратегии.
Любое изменение стратегии имеет цену;
порой сапиент даже не замечает, как платит за это изменение.
Если же изменение чересчур сильное, то ответом будет нэшевский рикошет.

Нэшевский рикошет называют по-разному.
В моём родном мире его величали <<невезением>>.
Люди наивно думали, что их преследует чья-то злая воля.
На самом деле нэшевский рикошет приходит за любое отклонение --- может, ты прогнулся там, где не следовало этого делать, или не прогнулся там, где предписывал твой договор.
Чаще всего от рикошета страдают любители стабильности и порядка, забывая, что мир --- обитель хаоса и изменчивости.

Стратегия не имеет ничего общего с личным путём.
Путь идёт изнутри и всегда ведёт к процветанию личности.
Стратегия же приносится извне и с упорством быка ведёт туда, куда ведёт, механистично исполняя заложенные в неё инструкции.
Меня, Чханэ, Митхэ, Зонтика и Веточку наши стратегии вели к алтарю.
Однако мы выбрали борьбу, и помощью ли духов, случайностью ли Вселенной мы ехали в одной упряжке --- отверженные племенем, но живые, здоровые и свободные, как ветер.

Митхэ держала в объятиях спящих мальчишек и смотрела на выбегающую из-под повозки дорогу.
У её губ пролегла складка.

--- Я не была дома и не попрощалась с кормильцами, --- сказала она.

--- Я думаю, для них гораздо важнее то, что ты жива, --- утешила её Чханэ.

Подруга повернулась ко мне, её пальцы забарабанили по моей ноге:

<<Лис, объясни, почему нельзя было вернуть их домой>>.

<<Для Сада и Цеха действия Храма легитимны, пока не доказано обратное, --- жестами ответил я.
--- Меня беспокоит не то, что их заберут новые жрецы, а то, что кормильцы сами вернут детей на алтарь>>.

Чханэ зябко поёжилась и бросила взгляд на детей.
\ml{$0$}
{На свете нет истин, к которым не был бы готов юный человек;}
{There are no truths which a young human isn't ready for;}
\ml{$0$}
{но сообщить эту --- именно эту, именно сейчас --- ни у кого не повернётся язык.}
{but this one---exactly this, exactly now---no one has the heart to tell.}
Впрочем, зная ученицу, я догадывался --- раз вопрос не прозвучал, значит, ответ уже найден.
Складка у её губ была чересчур глубока.

--- И тотем Сомнения мы с тобой так и не доделали, --- добавила Митхэ.

--- Тотем Сомнения? --- удивилась Чханэ.
--- Это ещё что?

--- Я не уверен, но полагаю, что это секрет, --- сообщил я.

Митхэ прыснула.

--- Митхэ, мы постараемся закончить его вместе.

--- Возможно, что так, --- девочка хихикнула и утёрла набежавшие слёзы.
--- А может, и не постараемся.

\asterism

На привале Чханэ целый вечер пыталась узнать про тотем Сомнения.
Наконец, после очередного витиеватого вопроса, Митхэ предложила сделать для подруги тотем Любопытства.
Чханэ замерла, обдумывая эти слова, затем расхохоталась и сказала, что поняла.
Поняла ли она на самом деле, ни я, ни Митхэ точно сказать не могли.
Тотем продолжал работать.

\section{[-] Первая любовь}

\textspace

--- Стоять! --- раздался позади знакомый голос.
--- Движение --- и я всажу тебе стрелу в спину.

Чханэ выругалась.

--- Стреляй, --- отозвался я и обернулся.
--- Ну давай же.

Ликхэ стояла, натянув лук.
Острие стрелы смотрело прямо мне в глаз.

--- Нам велели взять вас живыми или мёртвыми, --- сказала Ликхэ.
Даже с расстояния в десять шагов я видел, как дрожит наконечник стрелы.

--- Живыми ты нас не возьмёшь.
Или стреляй, или мы пошли, --- откликнулся я и двинулся в кусты.
Чханэ последовала моему примеру.

--- Лисичка! --- в голосе Ликхэ я услышал отчаяние.
Я снова обернулся --- она опустила лук.

Ликхэ подбежала ко мне и обняла, уткнувшись лицом мне в грудь.
Потом отстранилась --- на её лице осталась лёгкая влажная дорожка.
Губы её дрожали, кривясь в горькой улыбке.

--- Лисичка, почему она? --- зашептала женщина, мотнув головой на Чханэ.
В её глазах блестели слёзы.
--- Я любила тебя с самого детства.
Я красивая, умная, хорошо готовлю, ласковая.
Ведь так?

--- Да, это так, --- согласился я.
Чханэ хмуро молчала.

--- Я бы стала твоей просто так, по одному твоему слову.
Почему из-за неё, бесплодной пьяницы-кутрапа, ты разрушил свою жизнь?

--- Она --- мой друг, --- пожал я плечами.

Ликхэ посмотрела на Чханэ и понимающе кивнула.

--- Она Плачущий Ягуар, хоть и не носит раскраску, --- пробормотала воительница.
--- Она бы встала на мою защиту, даже если бы любила тебя, как я.

--- Думаешь, я люблю его меньше? --- проворчала Чханэ, поправив походный мешок.

Ликхэ промолчала.
Потом, легонько чмокнув меня в губы, воительница с извиняющимся видом потрепала Чханэ по щеке и повернулась, чтобы снова уйти на пост.

--- Идите, --- бросила она через плечо.
--- Я никого не видела.

\section{[-] Ничто человеческое не чуждо}

К моему удивлению, бодрствовала не только Анкарьяль.
Тхартху сидела рядом с ней, закутавшись в одеяло.
Я подсел к девушке.

--- Почему ты не спишь?
Завтра будем идти весь день.

Тхартху зябко дёрнула плечами.

--- Не хочется.

--- Это из-за Грей... Секхара?

Тхартху промолчала.
Я обнял её и потёрся носом о её ухо.

--- Расскажи, что тебя тревожит.

Губы девушки задрожали.
Глаза наполнились слезами.
Она помолчала, кусая губу.

--- Знаешь, как мы познакомились с Секхаром?

Я покачал головой.

--- Однажды он приходил в наш хутор --- искал какие-то реактивы для протравки.
Он ковал оружие.
И увидел меня, рисующую углём на белой стене.
Он сказал, что я ему нужна --- придумывать узоры для клинков.
Так он взял меня на работу.
Платил хорошо.
Потом... потом я просто осталась у него дома.
Вначале хозяйничала --- варила еду, убиралась.
Потом всё случилось само собой --- я поселилась и в его постели.

Анкарьяль, прислушиваясь к разговору, незаметно подобралась ближе и тоже обняла Тхартху.

--- Всё изменилось, когда он сказал мне, кто он на самом деле.
Он не охладел ко мне, не стал грубее.
Но стал чужим.
Бормотал что-то на непонятном языке, делал из воздуха непонятные машины.
Но я осталась, потому что любила его.

--- Ты любишь его до сих пор? --- шёпотом спросила Анкарьяль.

Тхартху шмыгнула носом и утёрла влагу, капающую из глаз.

--- Не знаю, Вишенка.
Наверное, да.
Но я не чувствую себя нужной.
Когда мы сидели за верстаком и травили металл, всё было по-другому.
То оружие, которое он делает сейчас, вполне обходится без моих рисунков.

--- Он тебя любит, --- проникновенно прошептала Анкарьяль.
--- Я это...

--- Вишенка, --- перебила Тхартху, --- для существа старше земли, на которой я стою, прожить жизнь от младенчества до смерти --- всё равно, что вымыть ноги перед сном.

--- Почему же ты пошла с нами? --- удивился я.

Тхартху всхлипнула и улыбнулась.

--- Погулять с богами в образе людей, послушать, о чём они говорят --- это похоже на сказку.
А с вами --- ещё и на добрую сказку.
Я всегда мечтала оказаться в такой --- доброй сказке с хорошим концом.

--- Конец ещё не написан, --- возразил я.

--- А я уверена, что он будет хорошим.

Мы немного помолчали.
Тхартху угрелась под тёплым одеялом и немного успокоилась.

--- Простите меня, --- смущённо прошептала она.
--- Вы спасаете мир, а я тревожу вас своими пустяками.

--- Это пустяки, которые способны перевернуть мир, --- возразил я.

--- Наверное, я чересчур строга к Секхару.
У вас ведь тоже было много кого?

Мы с Анкарьяль переглянулись.

--- Это сложно объяснить, --- сказал я.

--- У нас всё чуть иначе, чем у людей.
Аркадиу обычно старается избегать личных отношений, --- сообщила Анкарьяль.
--- Хоть он и утверждает, что не корректирует гормональный фон своих тел, но факт очевиден даже для меня.

--- Ты так говоришь, словно это плохо, --- возмутился я.
--- Урождённый демон мог бы и понять, что это необходимо для конспирации.
Кстати, гормоны в норме, я корректирую деятельность лимбической системы --- внешняя привлекательность должна сохраняться, это внушает людям доверие.
Раньше у меня были отношения с людьми, и дружеские, и половые.

--- Ага, --- ухмыльнулась Анкарьяль.
--- Одна женщина, если не считать той, что дрыхнет в палатке.
Даже моё невинное предложение совокупиться в свободное время он проигнорировал.
А ведь столько не живут, сколько мы друг друга знаем.

--- Это было десять жизней назад.
Те наши тела давно умерли.
Хватит притворяться, что это тебя задело.

Я постарался дать понять, что разговор закончен, но, к моему удивлению, за тему зацепилась Тхартху.

--- Ликхмас, расскажи о своей первой жизни.

--- Мне бы тоже хотелось услышать, --- непринуждённо улыбнулась Анкарьяль.
--- А то ты постоянно аккуратно избегаешь этой страницы своего существования.

--- Ах ты змеиное семя, --- проворчал я.

--- Мы с Грейсом имеем право это знать.
Да и, --- Нар погладила ладонью Тхартху по плечу, --- девушке, любящей сказки, будет интересно.

--- Это плохая сказка, Птичка, поверь на слово.

--- Лисичка, расскажи, --- Тхартху ткнулась в мою щёку головой.

Да, друзья действительно имели право это знать.

... История моя началась на планете под названием Драконья Пустошь.
Холодный неуютный мир, освещённый голубым гигантом, находился тогда под властью Красного Картеля.
Демоны практиковали там множество форм деспотизма, но мне это казалось само собой разумеющимся --- других миров для меня не существовало.
Я ничего не знал о Картеле.
Я даже не подозревал о Развязке Десяти Звёзд и о том, что Драконья Пустошь оказалась в самом центре этого чудовищного поля битвы.

Всё, что знал неграмотный талианский парень --- однажды отец снял со стены пулевое ружьё, попрощался с матерью... и не вернулся.

Всё, что видел парень после --- это толпы голодных беженцев, рассказывающих о вторгшихся в королевство Талиа инопланетных чужаках.
Это дети, которые не понимали, почему мать ушла из дома и воюет против них, почему отец устроил диверсию в родном городе.

Может быть, другой бы, взяв ружьё, пошёл мстить непонятно кому и сложил бы голову на полях сражений вслед за отцом.
Но парень был умён от природы и понял --- нужно найти врага и сражаться с ним его же оружием.
О магии, которой владели некоторые церковники, в народе ходили легенды...

--- Ты хочешь божественную силу? --- смеялся надо мной демон Картеля, выряженный в смешную рясу.
--- Иди отсюда, мальчишка.

Мальчишка вместо ответа приставил к голове святого отца пулевое ружьё.

Разумеется, демон мог убить меня, не пошевелив пальцем.
Почему он не сделал этого, знает лишь он сам, Яйваф Солёная Борода из клана Дорге.
Его глаза расширились, а потом он захохотал.
Хрипло и гнусаво.

--- Скажи, Лис, --- осторожно вмешалась Тхартху, --- что ты чувствовал, когда тебя превращали... в демона?

--- Представь, что на крыше крохотной хижины в один миг появился огромный дворец.
Примерно это я почувствовал в тот момент.
Дворец должен был раздавить хижину, сравнять её с землёй, но я выстоял.

Яйваф вложил в меня все боевые и тактические навыки, которые имел, не тронув ядро личности.
Это было грубейшим нарушением правил Картеля.
Сказать по правде, демон ко мне привязался --- феномен, который тщательно замалчивала пропаганда Ада.
Привязанность --- нулевая эмоция, не имеющая полярности.
Минус-хоргеты тоже могут её испытывать без вреда для себя.

Мальчишка, ставший демоном, не был похож на идеологов-церковников.
Уставший от воя о небесной каре и обещаний райских кущ народ впитывал его слова, как губка.
Многие знали, что он потерял отца, что он жил впроголодь --- и за ним пошли.
Сначала сотни, потом тысячи, а затем и миллионы.
Демоны Картеля, редко прибегавшие к помощи людей, поняли, что мальчишка может принести им пользу, и оказали ему всестороннюю поддержку.

Поддержку мальчишке оказал и Валериу XII, который впоследствии стал его лучшим другом.
Многие удивлялись несоответствию характера короля и смутных времён, но разгадка была проста.
Природную честность, худшее из качеств политика, Валериу с лихвой компенсировал импульсивностью.
Те, кто думали, что смогут играть с простодушным правителем, дорого за это заплатили.
Простой деревенский парень Люпино, познавший благодаря демонам все тайны интриг и шпионских войн, оказался для короля подарком судьбы.

Мальчишка с детства ненавидел <<войну масок>> и сделал всё, чтобы вывести врага в поле.
И так случилось, что в 1348 году от Зимы Великанов в заснеженной степи Серпенциару, на тракте Виа Галоледика сошлись объединённые силы Ада под командованием молодого амбициозного демона и талиано-саманское ополчение, которое вёл двадцатилетний Аркадиу Люпино.

Ман Великолепный потерпел сокрушительное поражение.
Орден Преисподней не смог исправить положение, и спустя пять лет демоны Ада вынуждены были уйти с Драконьей Пустоши.

--- Ман не ожидал такого от новоделка, --- заметила Анкарьяль.

--- Да уж, --- согласился я.
--- Он потом рассказывал мне, что никто никогда его так не удивлял.
Жаль его.

--- А что с ним? --- спросила Тхартху.

--- Его нет, --- пояснила Анкарьяль.
--- Казнён Картелем на Запах Воды дождей тысячу как.

--- А вы можете умереть? --- удивилась Тхартху.

--- Увы, Птичка, это так, --- кивнула Анкарьяль.

В столицу Аркадиу Люпино вернулся победителем.
Король Валериу XII собственноручно надел на него генеральскую гривну.
Стоявший справа от трона Яйваф ухмылялся в солёную бороду.
А слева сидела маленькая девочка и молча, серьёзным детским взглядом смотрела на новоиспечённую опору престола.

Годы летели.
Я занимался государственными делами и не заметил, как девочка превратилась сначала в девушку, а потом и в женщину.
И каждый раз при встрече --- на дворянском совете, на ужине по случаю приёма иностранных послов --- будущая королева Скорпия смотрела на меня тем же серьёзным немигающим взглядом.

--- Королева? --- ахнула Тхартху.
--- Это как Первая жрица?

--- Почти, Тхартху, --- улыбнулся я.
--- Но власть её была куда шире.

После смерти короля Скорпия взяла страну в свои железные руки.
И не только страну.
Впрочем, генерал Люпино особо и не сопротивлялся.

Скорпия не могла официально выйти за меня замуж.
В качестве мужа она выбрала слабовольного герцога с гомосексуальными наклонностями, предоставив ему титул и возможность распоряжаться своей половой жизнью, как заблагорассудится.
Постель королевы и значительная часть власти в стране достались мне.

Наши сыновья выросли и стали мужчинами.
Старший, который впоследствии был коронован как Вериту IX Скорпид, догадывался, кто его настоящий отец.
До самой моей смерти у нас сохранились дружеские отношения.

Разумеется, победа над силами Ада давала лишь временную передышку.
Новая волна интервенции началась через пятнадцать лет.

--- Только на этот раз господином Люпино занялись мы, --- осклабилась Анкарьяль.

Ман Великолепный, хоть и был отличным тактиком, увы, и в подмётки не годился команде Айну Крыло Удачи.
Она была мастером <<войны масок>> и быстро использовала мою слабость против меня.
Вчерашние союзники обратились во вражеских агентов.

--- Я хорошо помню этот момент, --- засмеялась Анкарьяль.
--- Мы с Айну после недолгой драки связали Кара полем.
Мне даже напрягаться не пришлось.
Я говорю: <<Ты погляди, как он на нас смотрит, волчонок волчонком>>.
А Айну смеётся: <<Ты не поверишь.
Его родовое имя --- Люпино>>.

Далеко не каждого демона Картеля имело смысл уничтожать.
Данных обо мне было собрано достаточно, и команда Айну приняла решение договориться.
Впрочем, им не пришлось стараться --- верен я был людям Талиа и королеве Скорпии, а не малопонятному и далёкому Картелю.

Я лично отправился к Скорпии, чтобы убедить её признать власть Ада над Талиа.
Королева долго смотрела на меня, потом на придворных и, сказав: <<Taliani potenta capitulatin, Valeridi not>>\FM, удалилась в свои покои.
\FA{
Талианцы могут сдаться, Валериды --- нет (язык талино).
}

Той же ночью Скорпия обмотала шею на сон грядущий упругим жгутом.
Её тело обнаружила служанка.

Я остановился, чувствуя, что поток воспоминаний стал чересчур сильным.
Удивительно, но Тхартху не восприняла моё волнение --- отражение чувств давно умершего тела --- как что-то странное.
Для её мифологического сознания не существовало ни демонизации, ни коррекции личности, ни обновлений;
для Тхартху я так и остался парнем с далёкой Драконьей Пустоши, давно не существующим человеком.
Она украдкой погладила меня, совершенно другое тело, по шее.

--- Странно, --- заметила Анкарьяль.
--- В исторических хрониках значится другая фраза --- <<Honor supremitu>>\FM.
\FA{
Честь превыше всего (язык талино).
}

--- Я был свидетелем, --- сказал я.
--- А это --- выдумка историков Лодемра Сурового, который отправил последнего Валерида --- Оливиу II Скорпида --- в изгнание.
История должна учить верности трону, а не превозносить поверженных врагов.

--- Я не знала, что вы были настолько близки, --- сказала Анкарьяль.
--- Ты, наверное, ненавидел нас.

--- Знаешь, что сказала мне как-то кормилица?
Когда целишься во врага, думай о броненосце, которого приготовишь на ужин, --- ответил я.
--- И неважно, что сегодня ты вынужден был убить своего сына.
Другой ребёнок жив и хочет есть.

\section{[-] Общая мечта}

\textspace

--- Так значит, ты решила уйти? --- спросил Грейс. --- Ну что ж, пусть будет
так.

--- Я мечтала о семье и спокойной жизни, Секхар. И сейчас всё это у меня будет.
Прости.

--- Тебе нет нужды оправдываться, --- покачал головой Грейсвольд и обнял Тхартху.

Тхартху чмокнула Грейса в щёку и повернулась к нам:

--- Надеюсь, вы не считаете меня предательницей?

--- О чём ты говоришь? --- поморщилась Чханэ.
--- Предательство --- это если бы ты бросила нас в бою.
Сейчас мы в безопасности и ты можешь делать всё, что пожелаешь.

--- Тогда прощайте.
И спасибо за всё.

Тхартху кивнула нам и убежала в дом.
Мы задумчиво смотрели ей вслед.

--- Хай, вот бы... --- мечтательно протянула Чханэ.

--- Тоже, --- закончила Анкарьяль.

--- Ага, --- добавили мы с Грейсвольдом.

В завершение этого короткого, но ёмкого разговора мы вздохнули и побрели дальше, к видневшимся из-за крон деревьев башням Сотрона.

\section{[@] Костёр в кругосветку}

\epigraph
{Медицина и морское дело всегда идут рядом;
игра с морем сродни игре со смертью\FM.
Невозможно быть доктором, не будучи самую малость моряком.}
{Марке Скрипта}
\FA{
В языке талино эти слова созвучны --- morer и mortir.
}

\textspace

--- Небо, --- обратился ко мне Костёр, --- я понимаю, что я теперь главный врач на материке, но можно ли мне взять отпуск?

--- Сколько унесёшь, --- пошутил я.

--- Половину солнечного года, складную лодку и утяжелённый киль, --- серьёзно ответил Костёр.

Я поперхнулся.

--- Утяжелённый?
Ты в кругосветное собрался?

--- Да.
Хочу проплыть через Тихий океан, заглянуть на вулканические острова.
Мы как раз их не осмотрели --- дельфины зашли только в дельты нескольких рек.

Тихим океаном колонисты сразу, не сговариваясь, окрестили огромное водное пространство между Китом, Кристаллом и западным побережьем Короны.
Я вначале запротестовал --- когда мы вернёмся на Тси, где уже есть Тихий океан, это может создать географическую путаницу.
Но на этот раз меня не стали слушать даже друзья.
Так океан и остался Тихим\FM.
\FA{
Помимо Тра-Ренкхаля и Тси, семантически похожее название самого большого водного резервуара встречается на 34\% обитаемых планет.
Наследие Древней Земли.
}

--- Хорошо, --- подумав, сказал я.
--- Бери лодку, киль я попрошу отлить.

Костёр широко заулыбался и убежал.

\section{[*] Митхэ и тигры}

\textspace

--- О, кажется, у меня есть одна хорошая новость, --- улыбнулась Эрхэ.
--- Кто-то собрался ловить тигров.

Воительница подошла к Митхэ и развязала шнурки нашейника.

--- Вот Атрис обрадуется, когда мы его найдём, да?

Митхэ потрясённо ощупала шею, на которой проступили бледные полосы.
Улыбка Эрхэ увяла, сменившись выражением ужаса.
Она тоже вспомнила.

--- Золото, это же Хат, да?..
Золотце?..
Скажи, что это Хат!..

\section{[*] Ненужный ребёнок}

\textspace

Митхэ пустым взглядом смотрела на растущий живот и теребила рукоять ножа.

--- Мне не нужен ребёнок от насильника.

--- Тогда отдай его мне, --- в глазах Эрхэ застыли слёзы.
--- Я воспитаю его или найду для него кормилицу.
Для меня важно, что это твой хранитель, и неважно, кто дал ему плоть.
А если он от Хата, Золото?
Что, если он от твоего самого любимого человека?

--- Мне плевать.
Я не хочу сомнений.
Я окончу его мучения, когда он сделает первый вдох.

--- И кем ты будешь после этого? --- буркнул Ситрис.

Акхсар гневно открыл рот, но Митхэ остановила его.

--- Пусть он говорит.

--- Ребёнок тебя не насиловал, --- объяснил разбойник.
--- Он вообще пока ни в чём не виноват, он просто появился и растёт.

--- Это потомство человека, который...

--- Это точно \emph{твоё} потомство, безо всяких сомнений, --- перебил Ситрис.
--- Я вообще не знаю мою дарительницу.
Но судя по тому, что меня оставили на дороге умирать, словно родившегося в междугодье идолёныша\FM, --- вряд ли она была хорошим человеком.
\FA{
У идолов Живодёра существует обряд проверки детей, которые родились в пятидневный период между двумя календарями.
Их относят на дорогу и оставляют там.
По легенде, именно в этот период должен родиться великий вождь, который <<удавит ядовитую змею и криком испугает ягуара>>.
Насколько известно, ни один ребёнок не пережил этот ритуал.
}
По крайней мере, у пиратов, которые меня подобрали, было больше человечности.

--- Тебя подобрали пираты? --- удивилась Эрхэ.
--- Ты говорил, что рос в деревне, как и...

--- Меня подобрали пираты, которые хотели совершить налет, --- буркнул Ситрис.
--- У ноа есть примета: если дорогу отряду перешел ребёнок --- битва будет проиграна.
Один из пиратов подобрал меня и отнёс в деревню...

--- И правильно сделал.
Дети, которых оставили умирать в лесу, перерождаются в зверей-людоедов и мстят...

--- Хватит этих глупых суеверий! --- раздражённо рявкнул Акхсар.
--- В конце концов, это её тело и решать должна только Митхэ!

--- Тогда пусть решает это сейчас, на ранних сроках, а не когда ребёнок отделится от её тела!
Ламповый хвощ и горькая лиана растут везде.
Берёшь по три грамматических горсти\FM, делаешь отвар на две чаши воды и чашу залпом, выкидыш будет ночью.
\FA{Грамматическая, или жреческая, горсть --- мера веса у ноа.}
Я могу всё сделать, не впервой.

Митхэ промолчала.

--- Не хочешь --- как хочешь.
Эрхэ дело говорит.
Воспитывать тебя никто не заставляет, отдай ребёнка ей.
Кем бы ни были его дарители, он хочет жить.

--- Я не хочу его даже видеть.

--- Скажи слово --- и я принесу тебе травы.
Но если ты его убьёшь после первого вдоха или оставишь умирать на дороге --- больше на меня не рассчитывай.

Ситрис схватил спальный мешок и ушёл поодаль от лагеря.

--- Велика потеря, --- буркнул Акхсар.

Эрхэ вздохнула и потрепала Акхсара по пегим спутанным волосам:

--- Пойдём спать.
Она сама разберётся.

\section{[*] Место Аурвелия}

\textspace

--- Капита Миция, я обвещать идти с тобвой, да, но...

Старик замялся.

--- Ты хочешь остаться здесь? --- вдруг поняла Митхэ.

--- Почти, --- кивнул старик.
--- Пвойдём, я поквазать.

Аурвелий повёл воинов на одну из самых высоких скал.
Старик с неожиданной сноровкой поднялся на самую вершину и подождал, пока его усталые спутники окажутся рядом.

Открывшееся воинам зрелище захватывало дух.
%{Breathtaking landscape lied upon the warriors.}
Эти земли не зря называли Сикх'амисаэкикх --- <<Страна леса, целующего камни>>.
Тысячи острых, словно копья, скал, звенящие ручьи, высокие деревья и цепкие заросли кустов, издалека кажущиеся мхом.
%{Thousands spear-sharp rocks, ringing rivers, tall trees, and bushes}
Стелющиеся облака смешивались с прядями низинного тумана, рвались на тысячи клочьев и исчезали под иссечёнными лучами вечернего солнца.

--- Этот река идёт к мворю, капита Миция, --- Аурвелий ткнул указателем\FM\ в тонкую вьющуюся ленточку реки, обвивавшую острые скалы и лоскутки сельвы.
\FA{
Древние эллатиняне указателем называли второй палец, т.к. он использовался в указательном жесте.
У людей-тси указательный жест представляет разогнутые первый и пятый палец при согнутых остальных, поэтому в ноа-лингва словом index обозначается именно пятый палец.
} 
--- Я ввидеть рыба горквона, много-много рыб, она точно прихводить из мворя, чтобы икра.
В дом есть инструмент, всё, что нужно.
Я взять инструмент, сделать крепкий лодка и вниз.
Там я буду с лодка и инструмент, я построю дом!

--- Я отпускаю тебя, сенвиор Амвросий, --- улыбнулась Митхэ.
--- Очень рада, что ты нашёл место для себя.

Старый ноа просиял.

Весь вечер старик на смеси языков взахлёб расписывал, как он построит лодку.

--- Я сильный, Эрхея, Аксарий, я мочь! --- твердил он.
--- Boni mani\FM\ --- смола много, seko!\FM\ --- доска с паром полюбиться и выгнуться в дуга!
\FA{Умелые руки (ноа-лингва).}
\FA{То, что надо (ноа-лингва).}

Эрхэ весело смеялась шуткам Аурвелия и хлопала в ладоши.
Акхсар клевал носом.
Ситрис сидел в углу и грустно думал о чём-то своём.

Митхэ спала плохо.
Ей снилась сырая пещера, где жрец приносил богов в жертву безумному человеку.
Ей снились чёрные силуэты людей, не плоские и даже не объёмные, а что-то хуже.
Дороги, по которым ступала Митхэ, проходили по одному месту несколько раз, не перекрещиваясь.
Вокруг летали фонари, которые пытались её убить.
Митхэ яростно рубила эти фонари саблей;
оружие проходило сквозь цветную бумагу, не причиняя ей вреда.
Митхэ отчаянно звала Атриса, но слышала в ответ только далёкую заунывную трель флейты.

\ml{$0$}
{Пробуждение пришло резко, словно воительнице гаркнули в ухо.}
{Awakening came as suddenly as if somebody had yelled in the warrior's ear.}
Занимался рассвет.
Акхсар и Эрхэ возились возле входа с готовкой.
Ситрис лежал рядом и безучастно смотрел в дырявый потолок домика.

--- Аурвелий умер, --- вполголоса сообщил разбойник и кивнул на лежащего в углу старика.

Митхэ закрыла глаза.

\ml{$0$}
{--- Из-за чего?}
{``What caused death?''}

\ml{$0$}
{--- Нет признаков ни удушения, ни отравления, ни кровотечения, ничего.}
{``No sign of asphyxiation, or poisoning, or bleeding, or anything else.}
\ml{$0$}
{Он лежит в той же позе, в которой заснул.}
{The position hasn't changed since he's asleep.}
\ml{$0$}
{Думаю, умер от старости.}
{I guess he died of old age.''}

\section{[-] Принятие}

Я сидел на циновке перед косоватым, чисто выскобленным столом в полторы пяди высотой.
Четыре чаши стояли пустые, в моей ещё дымилась остывающая густая похлёбка, которую я мимоходом зачерпнул из чьего-то котла.
Я испытывал чувство стыда перед неизвестной женщиной с улицы Стриженого Кактуса, но больше восполнить силы нам с детьми было нечем --- собранный походный мешок, в котором лежали пища и бутылки с водой, дожидался в тайнике возле дома.

Дети ели жадно.
Вначале я думал, что их просто не кормили новые адепты Храма, но вскоре Митхэ обмолвилась, что её кормилица --- Разрушитель.
Всё встало на свои места.

--- А я ел банан один раз, --- сказал Веточка.
--- Кормилец принёс золото, которое боги заплатили за моего братика.

Зонтик ничего не говорил ни во время еды, ни после.
Чханэ гладила его по жёстким чёрным волосам, избегая взгляда мальчишки.
Я осознал, что такой взгляд действительно трудно выдержать.

\ml{$0$}
{<<Мне надоело напоминать людям, чтобы они были людьми>>, --- вспомнил я слова Кхотлам.}
{``I'm sick of reminding humans to be humans,'' I remembered the words \Kchotlam\ once told me.}
\ml{$0$}
{С тех пор ничего не изменилось.}
{Nothing has changed since.}

Полуразрушенная хибарка, когда-то ставшая нам с Чханэ любовным гнёздышком, а теперь и убежищем, обветшала ещё больше, и я постарался привести её в порядок.
Лучи утреннего солнца едва проникали в дом через мутную плёнку бычьего пузыря, освещая заползшие в хибарку корни соседнего дерева.
Где-то на крыше деловито жужжали вокруг гнезда большие полосатые осы.
Почти так же, как и тогда.

В комнату вошла Чханэ и, вытащив из пыльного угла ещё одну циновку, села напротив, поплотнее закуталась в плащ.

--- Я уложила детей, --- сказала она, устало взлохматив рукой кофейные с оранжевой искоркой волосы.
--- Пусть отдохнут.

Я кивнул.

--- Так кто ты?
Бог?
Один из лесных духов в человеческом обличьи?
Древний герой, восставший из камня?
Откуда в тебе такие силы?

Я промолчал.
План, подразумевавший действие в строжайшей секретности, в самом начале начал трещать по швам.
Я знал, чем это обычно заканчивалось.

Чханэ вздохнула.

--- Ты Лис?

--- Да, я Лис.

--- Уже хорошо, --- криво усмехнулась Чханэ.

Я вдруг заметил, насколько она повзрослела с нашей первой встречи.
Кихотр при её рождении бросала рука куда более тяжёлая, чем рука Безумного --- девушке выпала доля гораздо суровее, чем многие могли бы представить.
И среди всех этих трудностей она сохранила человечность и благородство.
План планом, но оставить её сейчас на обочине пути было бы предательством и расточительством ценного материала.
Придётся взять её с собой, а значит --- дать кое-какие разъяснения.

--- Я пришёл, чтобы прогнать Безумного, --- начал я.

--- И зовут тебя Ликхмас.
Какое совпадение.
Только не нужно держать меня за дуру.

Я мысленно проклял это действительно невероятное совпадение.

--- Но это правда, Чханэ.

Чханэ несколько секунд пристально всматривалась в мои глаза, словно в глаза чужого человека.
Наконец она кивнула.

--- Я тебе верю.
А что ты есть?

--- Я --- подобие Безумного, заключённое в человеческое тело.

--- Хочешь прогнать Безумного и сам получать жертвы?

Я поперхнулся.

--- Нет.
Жертвы мне не нужны.

--- Понятно.

--- Что тебе понятно?

Чханэ смотрела в одну точку.

--- Вместо Лиса в его теле древний бог, Атркха'тху Люм'инэ, который пришёл, чтобы прогнать Безумных.

--- Почти так.
Только я и есть Лис.

--- А вот этому я не верю.

Наступил момент предельной откровенности.

--- Чханэ, --- сказал я.
--- Тебе страшно и одиноко.
Это нормально, когда против тебя восстаёт весь привычный мир.
Сейчас моя человеческая часть, которую ты знаешь как Ликхмаса ар’Люм, --- я, Чханэ, я чувствую то же самое.

--- Это неправда.
Ты всесильный.
Ты излечил меня.

--- Я не всесильный.

--- Хорошо.
Я тебе нужна? --- прямо спросила Чханэ.

--- Да, --- ответил я и, наклонившись, взял девушку за руки.
--- Иначе я бы ничего не стал тебе объяснять.

Чханэ выпростала свои руки из моих.
Наступило молчание, долгое, как ночь в далёких заснеженных землях за горами.
Наконец девушка оперлась о столешницу руками, мотнула головой, с отсутствующим видом пожевала попавшую в рот <<рыбку>>.

--- Да что же мне, всю жизнь по этой пустыне скитаться? --- пробормотала она.

--- Иди поспи.

--- Не хочу.

Я снова взял её за руки, и на этот раз она их не отдёрнула.

--- Если не желаешь идти со мной, если не веришь, что я --- это я, могу оставить тебя в любом месте, в каком пожелаешь, --- тихо сказал я.
--- Ты хотела на Кристалл --- я отвезу тебя туда.
Если надо, выстрою жилище.
Найдёшь себе мужчину, будешь жить так, как хочешь.

--- Нет...

Она замолчала, борясь с собой.
Потом добавила, уже куда уверенней:

--- Лис, не надо другого мужчины.
Мне ты нужен.

--- Я же страшный и непонятный, --- я позволил себе улыбнуться.
Чханэ взглянула мне прямо в глаза.

--- Пообещай одну вещь, --- серьёзно проговорила она.
--- Когда всё закончится... если мы останемся в живых... мы проживём жизнь так, как следует.
Мы больше никогда не переступим порог храма.
У нас будет жилище и дети.
Неважно, кто их выносит --- я хочу вырастить детей.
А до того --- я с тобой,

\begin{verse}
Пока нож не сотрётся в пыль,\\
Пока мышцы не станут струнами,\\
Пока не вырастут цветы из темени,\\
Пока посмертие не станет небытием\FM.
\end{verse}
\FA{
Клятва Маликха Ликхмасу, <<Легенда об обретении>>.
}

Я вспрыгнул на стол и обнял её, такую сильную, такую нежную...
Чханэ обмякла в моих руках, и я отнёс её к лежанке.
Сон --- вот что нужно подруге сейчас.

Чханэ всегда расстилала для нас постель одинаково --- одно одеяло в изголовье, сложенное гармошкой, два других уголком.
Удивительно, но даже в час самых тёмных сомнений насчёт меня она сделала всё точно так же.
Я уложил Чханэ слева, рядом с мирно посапывающими ребятишками, провёл рукой по волосам, погладил дуги бровей.

--- Спи, моя женщина.
Нам предстоит многое сделать.

--- Хай, ну да.
Свергнуть богов, --- пробормотала она, засыпая.

\section{[*] Лодка}

Завтрак прошёл в молчании.
Все рассеянно глотали пресную кашу с яйцом, думая о своём.

Наконец Митхэ нарушила тишину:

--- Как ноа хоронят своих?

--- Давай мы поедим и используем мою чашу как хэситр, --- устало ответил Акхсар.

--- Я спросила про обычаи ноа, а не про наши, Снежок.
Ситрис?

--- Я не припомню какой-то особой разницы, --- пожал плечами разбойник.
--- На суше хоронили в песке, а в море, понятное дело, отправляли в плавание.
У них считалось счастливым знамением, если первой тела достигала чайка, а не акула.
Поэтому самым уважаемым ещё и лодка доставалась.
Чтобы уже наверняка.

--- Лодка, --- повторила Митхэ.
--- Он мечтал о лодке.
Кто-нибудь знает, где можно достать лодку?

--- Золото, ты с ума сошла, --- буркнул Акхсар.
--- Мы в глуши.

--- Хорошо, тогда давайте сделаем сами.
Кто-нибудь знает, как делать лодки?

Акхсар и Ситрис переглянулись и отрицательно помотали головами.

--- Я не знаю точно, как делать лодки, --- вдруг подала голос Эрхэ, --- но мои кормилицы были плотницами, и я знаю один рецепт.
Дайте мне кхамит времени и выберите доски поровнее.

Эрхэ проглотила остаток каши и проворно бросилась в кусты.

Вскоре воительница вернулась с пучком растений и тут же занялась форменным колдовством.
От костра потянуло донельзя противной вонью.
Акхсар и Ситрис, наморщив носы, оттащили верстак едва ли не на четверть кхене и принялись обстругивать доски.

Когда доски были готовы, Эрхэ облила одну пахучим отваром и на глазах изумлённых мужчин лёгким усилием согнула её в полукольцо.

--- Обычно лодки собирают на стапеле, --- объяснила Эрхэ.
--- Однако я в судостроении профан, поэтому загоняйте доски один в один между этих двух камней и сушите так, мы сведём их на клин и стянем по типу бумажного фонарика.
Надо бы, конечно, как-то внахлёст, но ладно, в бездну.
Смолу я уже грею.

--- Прости, что куда стянем? --- переспросил Акхсар.

--- Я сделаю всё сама, Снежок.
И нечего нос воротить от <<луговой воды>>, --- бросила воительница Ситрису.
--- Из-за вас, нежные носики, верфи и лесопилки строят в такой глуши, что на ходьбу времени уходит больше, чем на работу...

К вечеру лодка была готова.
Она оказалась вовсе не такой крепкой, как хотелось бы Аурвелию, но очень красивой.
Акхсар разрисовал кромку борта замысловатым орнаментом, Эрхэ обложила лодку цветами и обвязала яркими ленточками, которые нашлись в дорожном мешке.

--- Жалко, что мы не найдём зелёный парус с вытканным солнцем, --- тихо пробормотала Эрхэ.

--- Что? --- не поняла Митхэ.

--- Да так, ничего.

\ml{$0$}
{--- Помоги мне, пожалуйста.}
{``Help me, please.''}

Эрхэ и Митхэ усадили старика, зафиксировали его верёвками и привязали сухощавую руку к рулю.

--- Не нужно было всего этого, Золото, --- хмуро сказал Акхсар.
\ml{$0$}
{--- О мёртвых заботятся лесные духи.}
{``The Silva Spirits take care of the dead.}
Давай я совершу над ним обряд?

--- Аурвелий вполне справится сам, --- отрезала Митхэ и оттолкнула лодку от берега.
\ml{$0$}
{--- Главное, что он понял, чего хочет.}
{``All that matters---he realized what he wants.''}

Управляемое мертвецом судно весело понеслось по извилистой речке и исчезло за поворотом.
\ml{$0$}
{Путь продолжался.}
{The road kept going.}

\chapter*{Интерлюдия I. Сат-скиталец}

\addcontentsline{toc}{chapter}{Интерлюдия I. Сат-скиталец}

Однажды в племени Бвожай-кхве появился мальчик.
У него был тихий голос, ещё он был косноязычен.
Племя презирало его, потому что без громкого голоса, как верили они, нельзя стать великим воином.

Однако мальчик вырос, стал юношей, и оказалось, что его дух чересчур силён для того, чтобы его презирали.
Однажды юноша принёс обет молчания, и с тех пор никто не услышал от него ни звука.

Вскоре молодого хака начали замечать в городах сели.
Несмотря на то, что он носил местную одежду, грубые черты выдавали в нём пришельца с востока.
Так как никто не знал его имени, ему дали новое --- Сатракх;
портной, которому приглянулся немой юноша, вышил новое имя на его одежде.
Так Сатракха стали узнавать все.

Легенда гласит, что он понимал все языки известных земель.
Считается, что он сплёл первые чётки, пока ходил по дорогам Короны;
говорят, что по протоптанным им тропинкам впоследствии проложили Восточный тракт.
Именно Сат ввёл обычай путников носить с собой зажжёный бумажный фонарь --- признак мирного паломника, далёкого от войн и интриг.
Легенда также гласит, что он прожил двести пятьдесят дождей, однако это лишь слухи;
многие путники стали носить бумажные фонари, и никто не обратил внимания, в какой из дождей одним фонарём стало меньше.

\chapter{[-] Бумажный фонарь}

\section{[-] Посохи}

Путники с бумажными фонарями --- особое общество.
Кто-то из них знает, что посвящён Сату-скитальцу, кто-то интуитивно повторяет привычки знакомых.
Сели, хака, ноа, ркхве-хор, зизоце, трами, травники, совсем редко --- дикие идолы с татуированными лицами.
Посохи ркхве-хор похожи на срубленные и слегка ошкуренные деревья.
Когда великаны кланяются, их посохи ощутимо скрипят.
Посохи травников едва годятся для того, чтобы ворошить костёр;
этот народ приветливо щёлкает челюстями и поднимает четырёхпалые руки горе.
Но все они --- одно: мы не мёртвы, не живы, мы в пути.

Вот навстречу попалась женщина-хака.
Молодое лицо с неглубокими морщинками, серые глаза, длинные дреды, сплетённые из её собственных волос и волос случайных попутчиков.
Штаны сели, красивая пылеройская накидка, чересчур широкая ей в плечах.
На родине женщину признали бы негодной и подвергли остракизму.
Но здесь, на дороге, властвует бумажный фонарь.
Мы встретились взглядами... как вдруг усталое лицо расцвело искренней щербатой улыбкой, и я неожиданно для себя улыбнулся в ответ.

Едем дальше.
Ни имени, ни привета, да и лицо прохожей уже расплывается в памяти, но вечернее небо на миг стало более голубым.

\section{[-] Совпадение}

\textspace

Дорогу вальяжно перешла недавно опоросившаяся самка дикобраза.
Выводок семенил вслед за ней.
Увидев повозку, самка повернулась к ней задом, с костяным треском встряхнула иглами и затопала, словно строй маленьких копейщиков.

Чханэ придержала оленей и подождала, пока дикобразовое семейство не уйдёт восвояси.

--- Плохая примета, --- пробурчала она и, подхватив чётки Сата, начала сосредоточенно искать узлы.
--- Лис, давай-ка просёлками поедем, поворот в паре кхене отсюда.
Как бы на засаду не нарваться, ни к чему нам сейчас это...

Я кивнул.

Извилистая тропинка, несмотря на дождь, оказалась в удивительно хорошем состоянии.
Мы оставили за плечами хутор, затем ещё один...
Судя по лежащему на обочине, почти заросшему землёй верстовому столбу, до дороги на Травинхал оставался один кхене.

Тропинка сделала крутой поворот, обнимая серую с чёрными крапинками скалу.
Я дёрнул поводья, и олени встали как вкопанные.

--- Не может быть... --- ахнула Чханэ.

Я не ответил.
Такое везение действительно бывает раз в жизни.

Прямо перед нами развернулся маленький палаточный лагерь.
Горели костерки, в походных котлах булькало что-то съестное.
Неподалёку паслись стреноженные олени.
А в двадцати шагах от нас, на воткнутом в землю копье трепетало на ветру перештопанное знамя.

Жёлтая пчела на тёмно-зелёном фоне.

\section{[-] Золотая Пчела}

\epigraph
{Даже над старой тропой светит сегодняшнее солнце.}
{Пословица сели}

Первый жрец назначил мне встречу глубокой ночью.
Чханэ и мальчики уже давно спали в палатке.
Митхэ оживлённо болтала со старыми жрецами, которых мучила бессонница;
кажется, они играли в какую-то игру.
Костры медленно, с тихим шипением умирали под моросящим дождём.

--- Когда-нибудь я стану жрицей! --- донёсся до меня гордый возглас девочки.
Старые жрецы восхищённо заохали.

<<Может, оставить её с ними? --- подумал я.
--- Бродячему Храму необязательно приносить жертвы...>>

Первый жрец застегнул клапан шатра и неторопливо, одну за другой разжёг четыре масляных лампы.

--- Твой ребёнок очень серьёзно настроен, --- сказал он.
--- Ты хорошо её воспитал.

--- Митхэ --- не мой ребёнок, --- сказал я.
--- Нас свёл случай.

--- А разве иначе бывает? --- усмехнулся жрец.
--- Даже если один человек вылез из другого, их всё равно свёл случай.
Все знакомства равноценны в самом начале и лишь потом обретают индивидуальность.
Так или иначе, мне девочка сказала, что ты --- её кормилец.

Я смутился.

--- У меня к вам есть необычная просьба.

--- Мы её не возьмём, --- предупредил вопрос Первый.
\ml{$0$}
{--- У нас, в отличие от городских, нет никаких предубеждений по поводу женщин, это вопрос милосердия.}
{``We're not prejudiced against women, unlike city priests---it's a matter of charity.}
\ml{$0$}
{Её бремя и так достаточно велико, чтобы взваливать на ребёнка ещё и тяготы жизни бродячего жреца.}
{The child's own burden is heavy enough to be carried with a burden of wandering priest on top of that.''}

--- Благодарю за гостеприимство, --- сказал я.
--- Мы несказанно рады, что обнаружили вас здесь.
Мне сказали, что вы отправились за море, в земли ноа.

--- Да, мы действительно собирались идти в земли ноа, --- кивнул жрец.
--- К сожалению, Кипящее море сейчас штормит, пришлось немного изменить маршрут.
Зашли в Хатрикас, затем по землям хака прошли на Север.

--- Хака вас пропустили? --- удивился я.

--- У них было много больных, они не возражали против нашей помощи.

Я промолчал.
Первый жрец стоял и прислушивался к шуму дождя;
он явно не торопился начинать разговор.
Наконец, спусля несколько михнет, мой собеседник нарушил молчание.

--- Прошу прощения, --- сказал он.
--- Я должен был убедиться, что ты не привёл за собой хвост.
Мне только что сообщили, что всё чисто.
Мы можем говорить свободно.

--- Я так понял, вы уже обсудили письмо.

--- Мы его прочитали, --- кивнул Первый.
\ml{$0$}
{--- Оно многое объясняет.}
{``Many things were explained.''}

\ml{$0$}
{--- Многое?}
{``Many things?''}

--- Аспекты сегодняшней общественной жизни в землях сели.
\ml{$0$}
{Перечислять долго и незачем.}
{There's no time nor reason to enumerate.''}

--- Вы опросите людей, чьи имена указаны в письме?

--- Среди людей Золотой Пчелы есть бывший Первый жрец Тхитрона.
Он знал Трукхвала ар'Хэ лично и подтвердил его почерк.

\ml{$0$}
{--- Бывший Первый жрец Тхитрона?}
{``Former Head priest of \Tchitron?}
\ml{$0$}
{Можно узнать, кто именно?}
{May I ask, who exactly?''}

\ml{$0$}
{--- Я.}
{``Me.''}

--- Как так вышло?

--- Долгая и запутанная история, как и у тебя.
Некоторые из моих жрецов до последнего времени вели с Трукхвалом переписку, и твоё имя там упоминалось неоднократно.
Мы слышали и о тхитронской диверсии.
Кроме того, мы осведомлены обо многих странных событиях, в том числе и в указанных двух Храмах.
\ml{$0$}
{На первое время этого достаточно.}
{That's enough for a while.}
Что ж, Ликхмас.
Скажи, сколько сейчас жрецов в Тхитроне?

--- Сейчас --- не знаю.
Перед уходом мы с Трукхвалом и моей женщиной перебили всех чужих.

--- Всех? --- поднял брови Первый жрец.

--- Я понимаю, что мы оставили город без защиты.
Но произошло это во время жертвоприношения, и жертва была принесена по всем правилам.
Мы оставили Храму время для действий.

\ml{$0$}
{--- Я прошу прощения.}
{``I beg your pardon.}
Просто Трукхвал, насколько я знаю, тихий миролюбивый хромец, да и вы с Тханэ ар'Катхар не производите впечатление людей, решающих проблемы силой.
Видимо, вам и правда пришлось туго.
\ml{$0$}
{Я так понимаю, что ты вне закона?}
{If I understand right, you are outlaw now?''}

\ml{$0$}
{--- Я не знаю точно, но скорее всего да.}
{``I don't know for sure, but more than likely yes.}
\ml{$0$}
{Воительница из моего Храма сообщила, что был отдан приказ брать нас живыми или мёртвыми.}
{A warrior from my Temple told me that she were ordered to take us dead or alive.''}

--- То есть ты связывался с Нижним Этажом?

--- Старые воины помогли нам бежать.
Среди них есть те, кто поддержал нового Первого жреца, но я не думаю, что они сделали это из корыстных побуждений.
Новые жрецы отлично умели запугивать и убеждать.

\ml{$0$}
{--- Охотно верю.}
{``I do believe.}
Мне рассказывали, что их дар убеждения граничит с колдовством, и бесхитростные люди очень часто попадают от них в зависимость.
\ml{$0$}
{Трукхвал погиб в бою?}
{\Trukchual, was he slain in fight?''}

\ml{$0$}
{--- Я забрал это письмо с его тела.}
{``I took this letter from his body.''}

\ml{$0$}
{--- Сочувствую.}
{``I'm sorry.}
Кстати, из какого ты дома?

--- Я хранитель Митхэ ар'Кахр, воспитывался в доме купца Кхотлам ар'Люм.

--- Кхотлам жива?

--- Жива.
Она до сих пор купец Тхитрона.
Вы её хорошо знаете?

--- Чересчур хорошо, --- ухмыльнулся Первый.
--- Это сейчас она остепенилась, ведёт жизнь примерного городского купца, но в молодости она натворила делов.
Скажем так, она ценила справедливость чуть больше, чем мир --- неудачное качество даже для очень умелого дипломата.
На этой почве она, например, хорошо подружилась с отрядом чести Митхэ ар'Кахр, и те не раз нам помогали почти бесплатно.
Один из её пассажей расхлёбывать пришлось мне.
К моему вящему неудовольствию, взвесив все за и против, я понял, что для Тхитрона она гораздо полезнее меня.
Первого жреца найти --- раз плюнуть, а вот таких дипломатов в канаве не найдёшь.

--- Ты справился?

--- Если она до сих пор на посту --- полагаю, что да, она сделала выводы и изменила стратегию.
Ещё одно доказательство, что я в ней не ошибся.
В целом вопросов у меня больше нет.
Бери пергамент и пиши.

Жрец дождался, пока я разложу письменные принадлежности на столике, и начал диктовать:

--- <<Трёхэтажному Храму от людей Золотой Пчелы>>.
Дату поставь послезавтрашнюю.

\begin{quote}
<<Мы знаем, что в землях сели действует группа жрецов, которая совершает перевороты в городах с целью захвата власти.
Возможно, что эта группа в настоящий момент контролирует абсолютное большинство Храмов.
Пусть всё остаётся так, как есть.
Мы не хотим войны и волнений.
Тем не менее, ввиду некоторых обстоятельств, мы вынуждены требовать, чтобы рекомая группа уступила людям Золотой Пчелы Храм Тхитрона.
В противном случае мы обнародуем сведения о деятельности рекомой группы со всеми вытекающими последствиями>>.
\end{quote}

Я поднял голову.

--- Это будет означать гражданскую войну.

\ml{$0$}
{--- Именно.}
{``Exactly.}
\ml{$0$}
{Они на это не пойдут.}
{They wouldn't allow that.''}

--- А что, если им нужна гражданская война?

--- Война войне рознь, Ликхмас.
Они могут стравливать между собой города и сословия, что они и делают сейчас.
Но поверь, \emph{такая} гражданская война им ни к чему.

--- Написал.

--- Отлично.
А теперь перепиши это письмо ещё раз.
Одна копия отправится во Двор Люм вместе с оригиналом письма Трукхвала.

Я вытащил второй листок пергамента, мимоходом восхитившись опытом Первого жреца.

--- Тебе был нужен мой почерк.

--- Безусловно, женщина, которая воспитала ребёнка, узнает его почерк.
Думаю, она будет рада узнать, что ты с нами.

--- Вам будет грозить опасность.

--- <<В жизни и смерти я буду щитом>>.
Это для вас, храмовников, опасность --- что-то из ряда вон выходящее.
Для людей Золотой Пчелы это обыденность --- договариваться, обходить чужие ловушки.
Я бы предложил тебе присоединиться к нам, потому что такой опыт на дороге не валяется.
Но времена, увы, не располагают.
Насчёт Храма не волнуйся --- кто бы это ни был, они уверены в победе.
Тхитрон нам уступят, считая, что это временная мера.
А маленькие бескровные победы всегда тянут за собой большую.

--- С Тхартхаахитром всё понятно.
Но примут ли тхитронцы тебя обратно, учитывая твою биографию?

--- Сейчас у них нет выбора.
Ворчать будут однозначно.
Я уверен, особо горячие головы даже разобьют пару окон в здании храма, чтобы напомнить мне, кто я есть.
Но на этом всё и закончится.

--- Что потом?

--- Потом --- чистая дипломатия.
Мы намекнём Тхартхаахитру, что просто хотели урвать свой кусок пирога из общей миски.
Людям, которые мыслят одинаково, проще договориться.
Поэтому нужно уметь думать по-другому.
Ты идейный человек, потому тебя и изгнали.
Они поняли, что с тобой нельзя договориться способами, которые им понятны --- подкупом, обманом или запугиванием.
Для них, жаждущих власти и благ, твой образ мышления --- чуждый, непонятный, а значит --- опасный.

--- Интересно, что их агентов нет среди вас.

--- Ты в этом уверен?

--- Уверен.
Я уже научился их отличать.

Первый жрец почесал щёку.

--- Нам с тобой ещё надо будет пообщаться, Ликхмас, но не сейчас, конечно же.
А насчёт агентов...
Эти люди хотят власти.
А какая власть у нас, у Бродячего Храма?

Я посыпал пергамент песком, поднялся из-за стола и протянул листки собеседнику.

--- Благодарю тебя.
Я рад, что оставляю Тхитрон в надёжных руках.

--- Рано благодарить.
Поблагодаришь, когда вернёшься в родной Храм.
\ml{$0$}
{Тебя будут ждать.}
{You will be welcome.''}

\ml{$0$}
{--- Я не могу вернуться.}
{``I can't return.}
\ml{$0$}
{Есть дела.}
{I have things to do.''}

К моему удивлению, он не стал задавать вопросов.

--- У вас провианта на день, --- подумав, сказал жрец.
--- Мы выдадим вам, сколько сможем, потому что в дождливые сезоны Север не склонен быть гостеприимным.
Сегодня ночуем вместе, а завтра на рассвете люди Золотой Пчелы идут в Тхитрон.
\ml{$0$}
{Ну и на всякий случай.}
{And, just in case.}
\ml{$0$}
{Я не знаю, что ты задумал, но можешь на нас рассчитывать.}
{I've no idea what you're up to, but you can count on us.''}

\asterism

Вскоре обстановка разрядилась.
Первый жрец заварил травы и достал откуда-то мешочек с конфетами, пряными и обжигающе-холодными на вкус.
Мы пили отвар и рассказывали друг другу истории из жизни.

--- Ты не поверишь, как приятно общаться с молодыми жрецами, --- улыбался Первый.
--- У нас молодёжи не было уже много дождей.
Наш Храм стареет.

--- Я думал, что Бродячие Храмы остались лишь в легендах, --- признался я.

Первый хохотнул.

--- И не ты один.
Ты же ведь знаешь, почему Бродячие Храмы в итоге исчезли?

--- Нет, не знаю.

--- Немудрено.
В поселениях об этом стараются не упоминать.
Но про гибель отряда чести Митхэ ар'Кахр ты наверняка знаешь.

--- Я встречался с ней в землях хака.
Митхэ обвинили в убийстве жреца.

--- Хай, так она жива? --- удивился Первый.
--- Сильная женщина.
У нас где-то в походной библиотеке хранится книга с её портретом на форзаце, старая-старая гравюра по коже...

Первый поднялся на ноги и зазвенел ключами.
Найдя нужный, он открыл походную библиотеку и вытащил на свет книгу.
Слабый тёплый свет масляной лампы выхватил потускневший профиль на покрытом пятнами форзаце.

--- Наверное, ты видел её не в лучшие дни её жизни.
А вот такой она была.
Маленькая, красивая и гордая, искреннее и смелое сердце.
Она происходила из южан, но северяне в ней души не чаяли.
Кстати, повернись-ка...
да, ты на неё очень сильно похож.

\ml{$0$}
{--- Я знаю.}
{``I know.}
\ml{$0$}
{Такой же мелкий и проблемный.}
{The same small trouble-maker.''}

Первый усмехнулся и передал книгу мне.

--- Да, она многим успела насолить, сама того не желая.
Дело было спланировано заранее, и принимал в нём участие не только Трёхэтажный Храм.
Издревле Бродячие Храмы были чем-то наподобие кузницы, в которой ковалось жречество и воинство, эдакой сокровищницей на крайний случай.
Если в каком-то из городов Храм переставал выполнять свою роль, заменяли его именно люди из Бродячих Храмов.

Я кивнул.
Мозаика сложилась полностью.

--- Борьба за власть?

--- Именно.
Началась она не сегодня и не вчера.
В прежние века мы приносили праздник в города, собирали целые толпы народа, давая возможность городским храмовникам наконец-то выспаться, выпить по чаше чего-нибудь и съездить в отпуск.
Городские жрецы лечились у нас, воины тренировались с приезжими, хотя чаще просто оставляли на них город.
Ведь это же здорово --- довериться знающему человеку, а не думать самому, что да как.
Но со временем, как видно, мы стали для городских не сменой и подмогой, а напоминанием, что они --- легкозаменимые детали механизма.

Жрец снова усмехнулся.

--- Вначале города стали различными способами ограничивать деятельность Бродячих Храмов.
Нам для лагеря начали выделять самые глухие места за городом.
Порой люди даже не знали, что мы рядом.
Затем начались проволочки с перемещениями --- воины учиняли допросы, задерживали и изменяли наш маршрут под разными предлогами.
Например, мой предшественник --- да хранят духи его покой! --- рассказывал с горьким смехом, как его не пропустили в деревню, охваченную эпидемией.
По какой причине, спросишь ты?
Потому что карантин!
Жреца не пустили из-за карантина!
Если бы это мне рассказал кто-то другой, я бы не поверил.

Первый закрыл лицо руками и покачал головой.

--- Из-за отсутствия заработка Бродячие Храмы начали редеть.
Некоторые осели, были и те, кто занялся торговлей.
На моём веку один Бродячий Храм стал бродячим театром --- и это ещё не самая неприятная метаморфоза, потому что именно как результат такой политики тогда начался всплеск разбоя и наёмничества.

--- В моём детстве о разбойниках говорили постоянно, --- припомнил я.
--- Так это?..

--- Бывшие воины, Ликхмас.
Воины, которых фактически оставили без средств к существованию.
Кто ещё обладал достаточным количеством опыта, чтобы обводить вокруг пальца храмовников?
В Бродячих Храмах остались в итоге лишь идейные люди, готовые к трудностям.
Да, были те, кто выжил, кто научился обходить ловушки и решать проблемы, которые устраивали городские Храмы.
Как будто своих проблем не хватало...
Когда отряд чести Митхэ перебили, это было сигналом именно для них --- оставшихся.
Людям Золотой Пчелы пришлось в итоге заключить с Тхартхаахитром негласный договор --- мы не вмешиваемся в дела городов, а нам дают немного зарабатывать.
Вот так, Ликхмас.
Это перештопанное знамя знавало лучшие времена...

--- Ты ведь давно ждал этот момент, не так ли?
Момент, чтобы вмешаться?

Жрец улыбнулся и промолчал. 

\section{[-] Встреча с Грейсом}

Ихслантхар встретил нас неприветливо.
Дождь, грязь по щиколотку.
Едва найдя постоялый двор, мы поняли, что лучше остановиться у кого-нибудь из горожан.
Двое или трое вежливо отказали, сославшись на тесноту, ещё десять были гораздо менее вежливы, но подавляющее большинство даже не откликнулось на <<гостевой>> стук.
Пришлось вернуться на постоялый двор.

Хозяин, в противоположность горожанам, оказался на удивление вежливым и отзывчивым человеком.
Однако его вежливость и отзывчивость стоили дорого;
не встреть мы Людей Золотой Пчелы, которые пополнили наш провиант, пришлось бы совсем туго.

--- У нас не хватит золота надолго здесь остаться, --- резюмировала Чханэ.
--- Сколько времени тебе потребуется, чтобы найти друга?

Я промолчал.
Поиск товарищей всегда был проблемой;
пробуждённые, разумеется, узнавали друг друга сразу, но спящего демона засечь практически невозможно.
Тем не менее способ был.
Демон даже в спящем состоянии заставляет тело совершать некие стереотипные действия --- так называемые МПС\FM, которые могут выдать его союзникам.
\FL{mbs}{Маркерные поведенческие стереотипы}
Я с сожалением посмотрел на стену дождя за окном.
Разумеется, все жители сидят по домам.

Наступила ночь.

Первой ушла спать Чханэ, уведя мальчиков.
Митхэ сидела со мной.
Вскоре её разморило, и девочка заснула у меня на коленях, разлив по полу отвар.
Зал постепенно опустел, но хозяин стоял у стойки, словно кого-то ожидая.

Вдруг колокольчик у дверей зазвенел, и в зал вошла грустная, немного сутулая молодая женщина.

--- Тебе как обычно? --- вскинулся хозяин.

--- Добавь ещё перца, --- попросила женщина.

Женщина не глядя опрокинула в себя три чаши крепкого пива с перцем, закусила орехами и заплакала.

--- Опять? --- сочувственно сказал хозяин.

--- Он меня вообще не слушает, --- тихо прорыдала женщина.
--- Бормочет ночами, а сам...
Как будто я не существую.

--- Тхартху, позови всё-таки жрецов, --- посоветовал хозяин.
--- Это уже не дело.
Секхар всегда был чудным, но сейчас с ним явно что-то не в порядке.

--- Прошу прощения, --- поднял я руку.
В пустом зале слова отозвались эхом, и собеседники вздрогнули.

--- Вам что-то ещё? --- спросил хозяин.

--- Нет-нет, --- покачал я головой.
--- Просто так вышло, что я жрец.
Не храмовый, путешествую.
Если нужна моя помощь, я помогу.

--- Я не смогу оплатить твои услуги, --- сказала женщина.
--- Храмовники сделают всё бесплатно, и...

--- И по какой-то причине ты до сих пор к ним не идёшь, --- подхватил я, --- а изливаешь печаль трактирщику.

--- Это мой брат, --- с укором сказала женщина.
--- Как ты смеешь!

--- А пиво --- твоя сестра.

--- Что?

--- Тебя выдают жесты, вышивка-оберег подмышкой и чётные слоги, растянутые и извилистые, как истоки Рек-Близнецов.
А у твоего собеседника городские манеры и чистый столичный выговор, более присущий пожилым, нежели людям его возраста.
Если не хочешь, чтобы я лез в твои дела --- так и скажи.
Только не надо врать.

Женщина сжалась и бросила взгляд на хозяина.
Тот отвернулся и сделал вид, что протирает посуду.

--- Приюти меня, мою женщину и троих детей на полдекады, --- сказал я.
--- Как видишь, мы тоже переживаем не лучшие времена.

Тхартху снова бросила взгляд на хозяина, но тот, видимо, решил больше не вмешиваться.
Наконец женщина кивнула.

--- Пойдём со мной.

--- Я присмотрю за девочкой, --- добавил хозяин и кивнул мне.
Я аккуратно положил головку спящей Митхэ на скамью и укрыл потеплее плащом.

Улица, поворот, улица, поворот, улица, поворот...
Я уже начал подозревать, что Тхартху наивно пытается запутать меня.
Но вот по левую руку появилась кузня.
Прошлёпав в грязи, мы поднялись к красивой узорчатой двери.

Всего одна лампа освещала жилище.
Это было не пламя;
в обычном фонаре весело жужжал настоящий маленький светодиод.
У верстака сидел и клевал носом высокий мужчина с небольшим брюшком.
Длинные спутанные <<рыбки>> с металлическими кольцами подмели столешницу, когда он обернулся к вошедшим.

--- Наконец-то, --- проворчал низкий голос знакомыми интонациями.
--- Я тебя заждался.

Я без слов подошёл и заключил его в крепкие объятия.

\asterism

--- Еда ещё осталась.
Вот суп сварил, --- бормотал Грейсвольд.
--- Птичка, кушай, малышка.
Прости, что я так с тобой обращался последнее время, но по-другому я просто не мог послать Аркадиу сигнал.
Долго объяснять.
Ты как сам-то, дружище?
На улице жуткая погода.

Тхартху с круглыми глазами, почти не слушая, хлебала томатный суп.

--- Грейс, что случилось? --- спросил я.
--- Ты получил мой сигнал о пробуждении?
К чему эти игры в домики-дороги?

Грейсвольд покряхтел и сел за стол.

--- Получить-то получил, но я сам напортачил с пробуждением, --- сказал он.
Говорил друг на языке тси --- видимо, хотел, чтобы Тхарту тоже поняла часть разговора.
--- Чувствуешь?
Весь дом в технике.
Всё, что можно, против отслеживания.
Можешь осмотреть, только осторожно.

Я <<осмотрел>> жилище.
Наноустройства были везде, в каждом камне пола, в каждой доске стены.
Чтобы сделать такое количество, нужно...

--- В общем, ты понял, да, --- вздохнул технолог.
--- Я почти без энергии.

--- Как так вышло?

--- У меня во время пробуждения начался акбас.
Самый настоящий, со спецэффектами.

--- Ты потерял над собой контроль? --- удивился я.

--- Да, --- кивнул технолог.
--- У тебя подобного не было?

--- Нет, --- покачал я головой.
--- Хотя постой...

Я вспомнил Чханэ.
В голове промелькнули алтарь, пробуждение и странная борьба тела и демона, которой никогда раньше не бывало.

--- У нас большие проблемы, Аркадиу, --- глухо резюмировал технолог, словно читая мои мысли.
--- Что-то происходит, не только вокруг, но и внутри нас с тобой.

Технолог выразительно коснулся пальцем виска.

--- Ну ты же знаешь, что обычно происходит при акбасе.
Всё концентрируется на одной мысли --- надо срочно обезопасить себя, любой ценой, любыми доступными средствами.
В общем, результат был прямо противоположным --- засветился я по полной.
Агенты Картеля встали на уши.

--- Блеск, --- буркнул я.

--- Я не мог послать вам сообщение раньше, --- развёл руками Грейсвольд.
--- Пришлось надеяться на твою смекалку.

Грейсвольд кивнул на Тхартху.

--- Она меня очень любит.

--- И поэтому ты вынудил её ходить в постоялый двор за выпивкой?
А сказать нельзя было?

Грейсвольд выглядел ошеломлённым.

--- Моя женщина знает, --- добавил я.

--- План не на той стадии, когда можно доверять всем подряд.

--- <<Всем подряд>>? --- процедила Тхартху.

Грейсвольд бросил виноватый взгляд на подругу.
Та сидела, скрестив руки на груди и глядя в стену.

--- Пять дождей, Секхар, --- голос Тхартху дрожал.
--- Пять дождей моей любви, но я по-прежнему <<все подряд>>.

Мы промолчали.
Тхартху грустно усмехнулась и встала.

--- Очень рада, что ты <<выздоровел>>.
Похоже, для меня пришло время показать, что хотя бы в Ближнереках люди верны своему слову.

Она повернулась ко мне.

--- Я приготовлю места и завтрак тебе, твоей женщине и детям.
Повтори только, как тебя зовут, я не разобрала.

--- Зови меня Ликхмас ар'Люм э'Тхитрон.
Или Играющая-Камнем-Лиса.

--- Женщине?
Детям? --- удивился Грейс.
--- Ты сюда выводок притащил?

--- Так вышло, --- развёл я руками.
--- Мы оба хороши.

--- Нет, вообще очень умно.
Агенты обращают меньше внимания на семейных, демоны чаще одиночки.

--- Всё, забыли, --- прервал я его.
--- Главное --- все живы, остальное приложится.
Пойду приведу своих.
Тхартху старалась меня запутать, но не учла, что постоялый двор у вас видно из окна.

Женщина густо покраснела.

--- Ликхмас, сейчас ночь, --- осторожно заметила она.
--- Я думаю, не стоит будить детей ночью.

--- Если переезжаешь в хороший дом, подойдёт любое время, --- ответил ей Грейсвольд.
--- Ложись-ка ты тоже спать, у тебя ещё пиво в глазах плещется.
Я сам приготовлю лежанки.
Как тебя зовут, Аркадиу?
Ликхмас, точно, только что прозвучало.
Сколько детей?

Я показал ему большой и указательный палец, затем направился к двери.

\ml{$0$}
{--- Десять тысяч дождей, а мозгов нет, --- резюмировал технолог мне вслед.}
{``Ten thousand rains, has got no brains,'' the technologist summed up to my back.}
--- И когда только успел троих настрогать!
Две пяди в обхвате, от макушки до пяток --- кукурузный початок, а стручок уж дорогу в амбар знает...

\section{[-] Живите кто хотите}

Перед уходом Грейсвольд долго стоял и гладил резную колонну жилища.
Жилище было совершенно не похоже на прочие, стоявшие на улице.
Я был уверен, что подобного ему не нашлось бы и во всём Ихслантхаре.

--- Это хорошее жилище, --- сказал Грейс.
--- Мы с Тхартху строили его вместе.
Резьба и рисунки полностью принадлежат её рукам.

--- Оно выглядит точь-в-точь как в моих детских мечтах, --- улыбнулась женщина.
--- Ликхэ-лехэ делал конфетные дворцы, и я ребёнком часто забегала к нему в лавку, смотрела на них и думала, что я хочу такой домик.

Грейсвольд бросил на неё смущённый взгляд:

--- Может, всё-таки останешься?
Тебе ничего не грозит.

--- Без тебя мне здесь делать нечего, --- отрезала Тхартху.
--- Так что хватит об этом.

--- Как скажешь.

--- Мы уходим навсегда.
Надо отдать дом сельве, чтобы путь был безмятежным и удачным.

--- Надо, --- признал Грейс, --- но у меня рука не поднимется.

--- У меня тоже, --- призналась Тхартху.
--- Что ж...

Женщина вытащила из-под навеса чистую коротенькую доску и положила на скамью.
В её крепких руках заплясал кукхватровый резец, разбрасывая деревянные кудряшки и коротенькие щепочки во все стороны.

--- Хорошая идея, --- похвалил Грейс, оглядев работу.
--- Поставь её на рога наличника, вон туда.
Там она будет держаться крепко и её увидят все.

--- Отдавать людям даже лучше, чем отдавать сельве, --- улыбнулась Чханэ.
--- Это не будет забыто.
Пусть жилище принесёт радость новым хозяевам.

Мы аккуратно прикрыли калитку и, бросив последний прощальный взгляд на жилище, отправились по размытой дороге на юг.

Фонарь у двери остался гореть, освещая прихотливую резную надпись:

<<Дом свободный, живите кто хотите>>.

\section{[*] Одинокий Столб}

Одинокий Столб можно было назвать городом с большой натяжкой --- он едва насчитывал полторы сотни строений, включая храм, гостиницы и исторические здания.
Тем не менее он представлял из себя настоящую крепость, ограждённую стеной.
Это была не символическая стена, подобная той, что в Тхитроне;
качество этой стены может продемонстрировать один простой факт --- она не пригодилась ни разу.
Даже Молчащие идолы, издалека осмотрев укрепления крохотного города, взвесили риски, оценили возможную выгоду --- и благоразумно обошли святилище стороной.

В темноте при взгляде с дорог или реки Одинокий Столб казался леденцовым дворцом --- ажурные венцы башен и кромка стены сияли от света оранжевых хрустальных тыкв, внутри которых пылал настоящий природный газ.
Хитроумное древнее приспособление отводило газ от окружающих святилище болот;
за день его собиралось достаточно, чтобы тыковки пылали всю ночь.

Название святилищу дал, как можно догадаться, Столб --- один из четырёх талисманов города.
По поводу возраста Столба шли ожесточённые споры, но ни у кого не вызывало сомнений, что он был возведён самими Древними.
Столб был сделан из неизвестного материала, похожего на белоснежно-чистый фарфор, оплетённый кукхватровыми кольцами.
Как и стена, Столб имел хрустальные окошки, и они также светились по ночам, но как и зачем --- не знал никто\FM.
\FA{
Главной реликвией Одинокого Столба является надземный модуль (маячок) золотодобывающей установки тси.
Как показало исследование устройства, оно прекратило работу за пять тысяч лет до описываемых событий, перейдя в режим ожидания.
В настоящее время Столб признан культурно-исторической ценностью планеты Тра-Ренкхаль и подлежит охране.
}

Прочими тремя талисманами города были тысячелетний молчащий кедр, Запретная роща --- роща благородного баньяна, а также развалины Су'макхэтхвал --- пятнадцать древних землянок, первое поселение возле Столба.
Каждая реликвия находилась в своём квартале: Столб --- в Верхнем Приозерье, Кедр --- в Квартале Кедра, Су'макхэтхвал относился к Кварталу Снежных Бород, а Запретная роща была в ведении Большого и Малого Домов.
При этом любые две реликвии разделяло не более ста шагов.

\section{[*] Никто}

Воинов в святилищах нет.
Есть только стражники.
Стражники приносят немного другую клятву.

--- Ты фыпрала хороший тень, штопы пройти шерес эти форота, Митхэ ар'Кахр, --- улыбнулся стражник.
Несмотря на строгое соблюдение правил северного диалекта, слова он выговаривал нежно, приглушённо и протяжно, сливая некоторые похожие звуки в один;
этот акцент --- <<шёпот любовника>>, или <<голубиное воркование>> --- был верным признаком коренного жителя святилища, но многие сели по незнанию могли счесть его носителя иноплеменником.

--- Здравствуй, Трасакх, --- коротко ответила на приветствие Митхэ.

--- Эрхэ ар'Люм, Акхсар ар'Лотр, --- Трасакх прищурился, разглядывая Ситриса, --- и Ситрис ар'Эр.

Ситрис вздрогнул.

--- Ты меня знаешь?

--- Ни ф коем слушае, --- строго сказал стражник.
--- Орушие --- полнорасмерное, киншалы, стрелы, лесфия --- в корсину. 
Саколки, пишутерию и фстроенное ф отешту мошно остафить, --- добавил он, услышав тяжкий вздох Ситриса.
--- Каштый пусть сам кормит сфои страхи.

--- А лук?
Ружьё? --- удивился Акхсар.
--- Раньше забирали стрелковое.

--- Хай, мы потумали и решили, што тостатошно сапирать поеприпасы, --- развёл руками стражник.
--- Ресультат тот ше, а на склатах теперь мноко места.
Но если ты путешь пить кофо-то луком, я ефо тоше саперу.

Спутники Митхэ сложили оружие в большую плетёную корзину.
Митхэ, поколебавшись едва ли мгновение, вынула кинжал, затем положила в корзину саблю.

--- Хай, у тепя нофая сапля? --- удивился стражник, затем вытащил её из корзины и осмотрел.
--- Красифая.
Слушайно не рапота Секхара ар'Митр э'Сотрон?

--- Его, --- улыбнулась Митхэ.

--- Стиль ошень уснафаемый, хотя послетнее фремя, по слухам, крафирофкой санимаются ефо маленькие тети.
Я перетам её на хранение ф Польшой Том.

--- Почему туда? --- насторожилась воительница.

--- Так нато, Митхэ ар'Кахр, --- строго ответил стражник.
И полушёпотом добавил:
--- Тела и саконы сели нас касаются ф той ше мере, што и фсекта.
А фот форы к нам сашастили.
Орушие шресфышайно сенное, и я настоятельно рекоментую сокласиться с этой мерой претосторошности.

--- Хорошо.

--- Неуютные нынше фремена, --- вздохнул стражник.
--- Усомнилась ли пы ты фо мне тоштей тфатсать насат?

--- Прости, --- поклонилась Митхэ.
--- На мою долю за последнее время выпало слишком много.

--- Мы слышали, --- кивнул мужчина.
--- Кстати, не шти слишком мнокофо от Ссепленных Рук.
Сейшас у них ф колофах поттершание не мира мешту наротами, а фитимости сопстфеннофо нейтралитета.

Тон стражника сквозил презрением;
на памяти Митхэ Трасакх никогда не позволял себе так отзываться о собственном Храме.
Митхэ и Эрхэ переглянулись.

--- Что ты имеешь в виду? --- спросил Акхсар.

--- Ты не найтёшь кофо-то, кто помошет тепе с Атрисом, Акхсар ар'Лотр, --- без обиняков пояснил стражник и открыл ворота.
--- Поэтому, если хошешь помоши, проси её у никофо.
Никто, как ни утифительно, фсекта ф курсе сопытий.
В катакомпах мноко этих никофо, и периотически ис-са них мы шефо-то нетосшитыфаемся.
Например, хороших сапель.
Храни фас Сит.

\asterism

То, что стражник назвал <<катакомбами>>, на самом деле было целым городом, но уже подземным.
Вернее, аж двумя городами --- хйяр Тар и хйяр Катх\FM, общая площадь которых превышала площадь самого Одинокого столба в полтора раза.
\FA{
Хйяр Стариков и Паучий Хйяр.
}
Концентрические пещеры\FM, центром которых был всё тот же Столб, за прошедшие тысячелетия поросли кальцитовыми натёками, и когда-то давно в них решили сделать жилища.
\FA{Результат работы золотодобывающей установки.
Устройство пропускало через породу продольные волны, облегчавшие процесс добычи.
}
Пока не выяснилась неприятная деталь --- при наличии хотя бы одного слабого места с болот в систему пещер поступал тот самый газ, делая нижние этажи подземных городов непригодными для жизни.
Вначале с этим пытались бороться, но природа взяла своё.
На хйяр Тар и хйяр Катх махнули рукой, и они теперь лишь изредка поражают красотой приезжих, достаточно смелых, чтобы спуститься в разлом.

Жили в подземельях и воры.
Это было крайне редкое занятие в землях сели --- честным трудом можно было заработать проще и быстрее.
Поэтому ворами становились либо совсем юные беспризорники, либо идейные люди, которые не могли представить жизнь без риска, либо кутрапы, которых ждала смерть в любых уголках Короны.
Причём беспризорники, повзрослев и получив метку за Насилие, чаще всего просто уезжали подальше и начинали новую жизнь, благо такая возможность представлялась ещё много раз --- пока человек не совершал преступление в пределах нового города, он был для него чист и честен.
Формально.
Люди же, несмотря на все увещевания дипломатов, заранее настороженно относились к человеку с метками трёх-четырёх городов.

В Одиноком Столбе воры жили очень тихо, воровали в основном пищу и одежду, аккуратно и помалу, так как знали --- если дать повод стражникам заняться катакомбами, то катакомбы опустеют надолго.
Заблудившимся в подземельях приезжим воры часто устраивали экскурсии за золото или по крайней мере показывали дорогу наружу.
Именно поэтому их не трогали и на воровство смотрели сквозь пальцы.

<<Под землёй растут лишь грибы, а из грибов лепёшки плохие>>, --- говорили жители святилища, если речь заходила о разломах.
Очень часто хйяр Тар и хйяр Катх были единственным местом, где мог жить кутрап.
Кое-кто из жителей специально оставлял пищу и одежду на видном месте.
А раз в декаду местные жрецы отправлялись ночью <<погулять>> с полным набором медицинских инструментов.
Так и жили.

\asterism

<<На это золото я могла бы купить Атрису свободу, --- думала Митхэ, ощупывая реликвии пустым взглядом.
--- Мёртвый груз, безделушки, бесполезная материализация бесполезных слов>>.

Задвижка на западном окне была сломана.
Язычок выдвигался не более чем на фасолину.
Если подбить задвижку, она слетит.

<<Во мне остаётся всё меньше человеческого.
Я впервые в жизни задумалась о воровстве всерьёз>>.


\chapter{[-] Кипящий котёл}

\section{[-] Встреча с Анкарьяль}

--- Вон она, --- ухмыльнулся Грейс и указал на сторожевую башенку.

Возле башенки стояла женщина --- высокая и поджарая, словно дикий олень.
Природную красоту несколько оттеняло хищное выражение лица, которое не пропадало даже во время улыбки.
У её бедра покоилась длинная и широкая, словно доска, кукхватровая сабля.
Короткая стрижка, татуировки на плечах и рёбрах выдавали храмового воина.

--- Точно она, --- убеждал меня Грейсвольд.
--- Ты посмотри на этот шедевр.
Бревно, а не сабля.
А ещё на два цана длиннее слабо?
И на пядь шире.
Если она поднимет саблю в ветреный день, её утащит к свиньям.

--- На татуировку посмотри, --- ответил я.
--- Та, что начинается под левой грудью и уходит подмышку.
Вот, смотри, руку подняла...

--- Она, --- Грейсвольд кивнул, шумно выдохнул и потянул меня вперёд.

Едва мы подошли, как воительница повернула к нам голову на великолепной длинной шее:

--- Нужна помощь?

--- О да, --- сказал Грейсвольд.
--- Почеши мне спинку.

--- Выпивка и секс вон там, --- воительница кивнула в сторону постоялого двора.
\ml{$0$}
{--- Если хочешь меня, то придётся подождать.}
{``If you want me, you have to wait.}
\ml{$0$}
{В лучшем случае до конца моей смены, в худшем --- до конца моей жизни.}
{At best, 'til the end of my shift; at worst, 'til the end of my life.''}

--- Да какой секс, --- буркнул технолог.
--- Не узнаёшь, что ли?

--- Скажем так --- я тебя не знаю, --- пожала жилистыми плечами женщина.
--- Не узнаю тоже, но это следствие.

Мы переглянулись.

<<Она не пробудилась>>.

<<О светлая твердь, опять...>>

Спустя секхар мы уже волокли сопротивляющуюся воительницу в кусты.

--- Кто я? --- строго спросил Грейсвольд после того, как я отвесил женщине хорошую оплеуху.

Она не ответила.
Я сделал несколько весьма болезненных ударов по корпусу, и связанная воительница согнулась в три погибели.

--- Кто я? --- снова спросил Грейсвольд.

Молчание.
И снова сеанс экзекуции.
После четвёртого вопроса у горла воительницы появился клинок.

--- Кто я? --- почти ласково спросил Грейсвольд.

<<Урод>>, --- жестами показала воительница.
Я вытащил кляп, и она закашлялась.

--- Мне это уже надоело! --- заявила она.
--- У меня ещё неделю всё болеть будет!

\asterism

--- Уроды, --- бормотала Анкарьяль, разглядывая себя в бронзовом зеркале.
--- Оба.
Вот как я теперь с этими синяками в храм пойду?
Аркадиу, это ты мне фингал поставил?
Я же сказала --- бить только по закрытым одеждой участкам.
И сказала я это в двадцать шестой раз!

--- Я случайно, честное слово, --- развёл я руками.
--- Ты очень бодро блокировала мой первый захват, и мне пришлось тебя огорчить.
Храму скажешь, что мы просто чуть переусердствовали с объятиями --- давно не виделись.

Грейсвольд с нежностью смотрел на подругу, рассеянно ковыряя пальцем опухшую губу и новорождённую прореху в зубах.

--- Красивое у тебя тело.
Даже с синяками.
Прям чувствуется стиль.

Анкарьяль расплылась в белозубой улыбке.

--- А ты какой-то не толстый.
И даже очень привлекательный, а не как обычно.

--- Среди сели крайне мало толстяков, --- развёл руками технолог.
--- Взял, что было.
И вообще, это было обидно.

--- Чего на правду-то обижаться?
Харизма в тебе всегда была, а вот на изящество ты обычно плевал.

--- Может быть.
Вот кто совсем на себя не похож, так это Аркадиу.
Я его едва узнал.

--- Он мелкий и худой, но какой-то очень сильный для своей комплекции, --- Анкарьяль критически меня осмотрела.
--- Надо потом посмотреть, что у него по генам.
Я впервые вижу, чтобы от пальцев оставались такие синяки.
Хотя подожди... ты же жрец, да?
Библиотекарь, наверное?

--- В библиотеке много времени провёл, --- хихикнул я.
--- А что?

--- Тогда всё ясно.
Наш библиотекарь пальцами персиковые косточки ломает в труху.

Женщина снова посмотрелась в зеркало и пятернёй начесала чёлку на заплывший глаз.

--- Пошли, --- кивнула она нам.
--- Вроде не заметно.
До конца смены ещё кхамит, но я обычно удирала чуть пораньше --- скука стоять на южном конце, ничего интересного...

\section{[-] Доверие Короля}

\ml{$0$}
{Ощущали ли вы себя когда-нибудь в центре бури?}
{Did you ever feel standing in the heart of the storm?}
\ml{$0$}
{Взгляните на знакомых.}
{Look at the people around you.}
Среди них обязательно найдётся тот, чьи глаза неестественно спокойны.
Мысль в глубине их зрачков мощна и извилиста, но эта мысль витает в разреженном облаке чувств.
\ml{$0$}
{Радость их слаба, немощна их печаль, а любовь мимолётна, как осеннее золото Крайнего Севера.}
{Their joy is weak, their sorrow is faint, and their love is fickle like golden fall of the Far North.}
Однако если вы вглядитесь, то на самом краю радужки можно заметить величественное вращение ока бури, которую эти люди поднимают одним своим существованием.
\ml{$0$}
{Такое невозможно забыть; я знаю --- в вашей памяти всплыло хотя бы одно лицо.}
{It can not be forgotten; I know, at least one face has surfaced from your memory now.}

\ml{$0$}
{Родись Митрис ар'Люм на другой планете, он мог бы стать инженером или цветочником, и, надо признать, неплохо бы справился со своим делом.}
{If \Mitris\ ar'\Loem\ was born on another planet, he would be an engineer or a florist---and admittedly the good one.}
Однако в обществе сели его незримая роль --- роль ока бури --- подкреплялась вполне реальной властью.
Король-жрец был единственным сели, имеющим право говорить от имени народа.

Королей-жрецов избирали на неопределённый срок --- до момента, пока советы не решали выбрать нового.
\ml{$0$}
{Подробностей отбора кандидатов никто не знал, но всех объединяло одно --- их лица узнавали во всех уголках земель сели, а имена звучали далеко за пределами.}
{Nobody knew all the details how candidates were selected, but they all had one thing in common: their faces were recognized in all the \Seli\ lands, and their names were told far beyond.}

Я оглядел зал.
Удивительно, но здесь не было той кричащей роскоши, которую я мог бы ожидать от средоточия власти.
Небольшой уступ, сделанное из малахита кресло со слегка потрёпанной тканой накидкой, каменный стол и два ряда деревянных кресел.
От залов тхитронского храма этот отличался только величиной, в остальном всё было то же самое, даже <<ласточкины ниши>>, в которых изредка попискивали птенцы и деловито крутили длинными хвостиками пичуги.

Король-жрец --- высокий зеленоглазый мужчина с орлиным носом и гордыми тонкими губами --- встретил меня стоя рядом с креслом, как и полагалось встречать гонца.
Восемь воинов --- по-видимому, его лучшие люди --- располагались чётко по уставу: по четверо с каждой стороны стола, двое с щитами --- ближе, полукругом, двое с духовыми ружьями --- чуть поодаль.
Где-то в тенях возле стены прятались убийцы.
Анкарьяль знаком показала, что можно действовать согласно вариации номер четыре --- стоящие в зале опасности не представляли.

<<Кто люди вокруг?>>

<<Его лучший друг, Тхиситр, бывший воин, ныне семейный --- первый по левую руку, вне подозрения, в бою чрезвычайно опасен, несмотря на пузо, поэт, заядлый игрок в Метритхис и курильщик.
Ещё трое слева, две женщины справа и убийца у северной стены --- неброские, но исполнительные адепты Малых Храмов, вне подозрения.
Убийца у западной стены... Траклам ар'Со, тоже Малый Храм, известный стрелок и боец, даже не пытайся с ним состязаться в скорости.
Очень красиво лепит из глины и рисует, дельфины у входа --- его работа.
Ещё одна с краю, ноа с двумя <<грифами>> у пояса --- Корнела Костелла, севия феррари с Острова Невезучих, циркачка, актриса, невольница с рудников, виноделка, контрабандистка, держательница убежища для брошенных детей, охотница на работорговцев, ныне бродячая учительница фехтования.
Личность тёмная, но вряд ли демон, любит Короля-жреца как кошка, по одному его слову и убьёт, и умрёт, дойдёт до боя --- держись от неё как можно дальше, если хочешь жить.
Прочих я не знаю --- вполне возможно, это наёмники какого-то отряда чести>>.

<<Он не доверяет воинам Трёх Этажей>>.

<<И его сложно в этом винить.
Но причина ещё и в другом --- ключевые фигуры ему недоступны, и рассчитывать приходится на преданный ему лично разношёрстный сброд.
Король-жрец --- не хозяин даже в собственной пирамиде>>.

--- Итак.
Ликхмас ар’Люм э’Тхитрон? --- спокойно осведомился Король-жрец.

Я коротко поклонился.

--- Он самый, Король-жрец.
Мои друзья...

Король-жрец едва заметно улыбнулся на этом слове.

---  ... Хатлам ар’Мар э’Тхартхаахитр, Секхар ар’Сатр э’Ихслантхар, Чханэ... Тханэ ар’Катхар э’Тхаммитр.

Друзья по очереди отвешивали короткий поклон.
Король-жрец нахмурился.

--- Вам требуется помощь врача?

Грейсвольд тут же смущённо спрятал щербатую улыбку и вытер с губ кровь.
Анкарьяль же справедливо решила, что увиденное нет смысла прятать, и откинула волосы назад, обнажив всё многоцветье синяков и ссадин.
Охрана Короля-жреца неуверенно потянулась к поясам;
подруга украдкой сделала им какой-то знак.

--- Благодарю, --- кивнул я.
--- Но позже, дело не терпит отлагательств.

--- Хатлам ар’Мар, --- обратился Король-жрец к Анкарьяль, --- я наслышан о твоей доблести, но не припомню, чтобы ты предупреждала Первого жреца или меня об уходе из Храма или о взятой на себя роли эмиссара.

Разумеется, ироничный вопрос был вполне ожидаемым.
По словам Анкарьяль, Король-жрец был когда-то её любовником.
Недолго.

--- Король-жрец, --- заговорила Анкарьяль, --- я ещё раз обращаю внимание на то, что дело срочное.
Разговор обо мне может подождать.

Король-жрец кивнул и, разгладив робу, сел на малахитовое кресло.

--- Говорите.

--- Я пришёл для того, чтобы поднять вопрос о целесообразности культа Безумного, --- максимально чётко отчеканил я.

Надо отдать должное Королю-жрецу --- он не повёл и бровью.
Охранявшие его воины выпучили глаза, а некоторые даже в нарушение устава переглянулись между собой.
И только ближайшие к малахитовому креслу как будто невзначай повернули щиты.

Король-жрец улыбнулся одними губами.

--- Если не ошибаюсь, Тхитрон обладает определённой автономией.
Если жители этого славного города приняли решение свести счёты с жизнью --- это их дело.
При чём тут я?

--- При том, Король-жрец, что народ сели в самое ближайшее время рискует получить тысячи нахлебников вместо одного, --- непринуждённо подал голос Грейс.
--- Можно ли назвать жизнью то, что последует за этим --- сложно сказать.

Доверенные Короля-жреца напряглись и зароптали.

--- Тихо, --- бросил Король-жрец.
--- Если дело серьёзное, следует закрывать глаза даже на очевидное богохульство.
Я понял ваши слова, пришельцы.
И я так понимаю, что вы бы не явились просить у меня закрытой аудиенции, не принеся с собой весомые доказательства.

--- Мы --- одни из пришельцев, --- сказал я.

Король-жрец усмехнулся.

--- Что-то ещё?

Я вытянул руку и создал яркий источник света над ладонью.
Ничтожная трата энергии, вполне допустимая, чтобы не выдать себя.
По залу прокатился сдавленный вздох.

--- Дешёвый трюк, --- буркнул Король-жрец.
--- Даже не знаю, почему я всё ещё трачу на вас время.

--- Хорошо, --- внезапно весело крякнул Грейс.
--- Король-жрец, давай начистоту.
Что тебя убедит?

--- Если вы сможете победить моих людей в сражении --- так и быть, я выслушаю вас до конца.
В наше время военная сила --- неоспоримый аргумент.

Король-жрец произнёс последние слова с едва заметной иронией.
Грейсвольд тихо вздохнул --- он тоже понял, что убедить сейчас нужно не Короля-жреца, а его людей.
Хозяин Трёх Этажей уже всё осознал и принял решение.
Ему жизненно необходимо с нами поговорить.

--- Хатлам ар’Мар способна победить твоих людей в одиночку, --- поклонился я.
--- Если ты позволишь, она это с радостью продемонстрирует.

Я схватил за руки Чханэ и Грейса и оттащил их ко входу.
Анкарьяль лениво сняла с пояса саблю и бросила её в угол.

Воины бросились на Анкарьяль одновременно, без предупреждения, строем <<клешня>>.
Просвистели в воздухе яркие оперённые стрелки, пролетел чей-то томагавк.
Однако Анкарьяль двигалась с немыслимыми даже для сели скоростью и точностью.
Её путь, как говорили в народе, был прочерчен сохой небесного пахаря, того, кто сеет звёзды и собирает урожай облаков\FM.
\FA{
Имеется в виду комета или метеор.
}
Не успел я досчитать до трёх, как все десять воинов, включая прятавшихся в занавесях убийц, уже лежали на каменном полу зала без движения.
Анкарьяль, дыша чуть глубже, чем обычно, так же лениво прошла в угол и подобрала саблю.

--- Они невредимы, Митрис, --- пояснила она и, подняв одного из воинов, аккуратно посадила безвольное тело на кресло.
--- Придут в себя ещё до заката.

Король-жрец окинул взглядом заваленный бесчувственными телами зал, медленно встал и указал рукой на скрытую в тени неприметную дверь за мраморным креслом.

--- Прошу.

\section{[-] Покои Короля}

\epigraph
{Если упадок перестаёт притворяться процветанием --- его время сочтено.}
{Постулат Элект}

--- Итак, Ликхмас ар’Люм.
К слову, это твоё настоящее имя?

--- Моя дарительница любила <<Легенду об обретении>>, --- усмехнулся я.

--- Кихотр, --- признал Король-жрец.
--- Не самое счастливое имя.
Говори, что знаешь и что тебе от меня нужно.

В маленьком уютной кабинете Короля-жреца царила темнота.
Единственными источниками света были узенькое высокое окошко в каменной стене, через которое проникал золотистый пучок вечерних солнечных лучей, и жёлтая шарообразная лампа, левитирующая в полупяди от поверхности каменного стола.
Лампа заливала поверхности на расстоянии трёх локтей абсолютно ровным, почти не дающим теней, приятным глазу светом.

Раритет.
Этот ночник остался ещё от предков-тси.
Грейс тут же с восхищённым бормотанием побежал к лампе --- исследовать устройство.
Мы с Королём-жрецом проводили его взглядами.

--- Это лампа с Тхидэ.
Не требует ни масла, ни огня.
Иногда я думаю о свете, иногда просто чувствую надобность, и лампа загорается.
Не сразу ярко, а постепенно, так, что глаза успевают привыкнуть.
Прекрасное чудо.

Грейс по-особому нажал на корпус.
Раздался мелодичный звон, и лампа открылась, словно цветок, показав механизм.
Король-жрец поднял бровь.

--- С лампой всё в порядке, Грейс --- профессионал, --- успокоил я его.

--- Когда что-то ломается, оно издаёт менее приятный звук, --- резонно ответил Король-жрец.
Анкарьяль ухмыльнулась.

<<Технология у них в крови>>.

В следующий миг нас окатило лёгкой вспышкой гамма-излучения.
Мы дёрнулись как ужаленные.
Грейсвольд тут же выключил лампу и тихо выругался.

--- Извините, --- пробормотал технолог.
--- Сейчас отрегулирую обратно.
Ничего себе ночничок.
Эти тси --- ненормальные...
Надеюсь, здесь генератора экрана не предусмотрено?

Король-жрец покачал головой и повернулся к нам.

--- Итак, я слушаю.
Правильно ли я понял, что вы одни из... богов?

--- Мы такие же, как они, но не на их стороне, --- сказала Анкарьяль.

Король-жрец вздохнул, впервые обнаружив признаки слабости.

--- Вишенка, ты ли это? --- едва слышно пробормотал он.

--- <<Свист лягушки, рассекающий ночь, будет нашей с тобой колыбельной>>\FM, --- так же тихо произнесла Анкарьяль.
\FA{
Цитата из стихотворения Эрхэ Колокольчик.
}
Король-жрец вздохнул ещё глубже и повернулся ко мне.

--- Вы тоже хотите жертв?

--- Как раз наоборот --- мы пришли помочь, --- ответил я.
\ml{$0$}
{--- И нам очень нужно знать, что здесь происходит.}
{``And we really need to know what's going on.''}

\ml{$0$}
{--- Откуда мне знать, что вы хотите нам помочь?}
{``How could I know you want to help us?''}

\ml{$0$}
{--- Я не могу этого доказать.}
{``I can't prove it.}
\ml{$0$}
{Тебе придётся поверить мне на слово.}
{You have to trust my word.''}

Король-жрец помолчал.

\ml{$0$}
{--- Уже около двадцати дождей ходят странные слухи.}
{``Odd rumors have been spreading across, about twenty rains.}
\ml{$0$}
{Я никогда не считал существующий порядок должным, но сейчас рушится даже он.}
{I never took the existing order for granted, but it falls apart like everything else.''}

\ml{$0$}
{--- Рушится? --- переспросил я.}
{``Falls apart?'' I asked.}

\ml{$0$}
{--- Именно, Ликхмас ар'Люм.}
{``Exactly, \Likchmas\ ar'\Loem.}
Это не перестройка, как бывает на Перекрёстке, это окончательное, бесповоротное и, главное, всеобщее разрушение.
%{It's a destruction, complete, total, and there's no going back.}
Жрецы стали лечить и приносить жертвы не по канону.
Они пренебрегают масками, плащами и правилами Отбора, их забрызганные кровью робы стали символом ужаса.
Воины перестали служить родным храмам, подаваясь в наёмники и разбойники.
Даже храмовые воители чаще занимаются террором, а не обучением молодых.
Города стали требовать полной независимости от Тхартхаахитра --- и это во времена, когда Живодёр опять показал клыки!
Недавно пришли вести, что люди нарушили нейтралитет в Омуте Духов и осквернили святилище, убив шамана идолов.
Это позор для нашего народа...

--- Это мы знаем, --- прервал Короля-жреца Грейс.
\ml{$0$}
{--- Нам нужны данные о людях, причастных к переворотам.}
{``We need data concerning takeovers and the implicated staff.''}

\ml{$0$}
{--- Так вы и об этом знаете, --- проговорил Король-жрец.}
{``So it's known as well,'' Priest-king said.}
\ml{$0$}
{--- Тогда вы должны понимать, что сейчас не все города отчитываются передо мной.}
{``Therefore, you must know that several cities are not answerable to me now.}
Все отчёты о смене жречества, которые у меня есть, в этой книге.

Король-жрец открыл резной деревянный шкаф и вытащил переплетённый в кожу том.
Я аккуратно взял книгу из изящных рук и открыл на последней странице.
Грейс схватил раритетную лампу за металлический шнурок и подошёл, чтобы подсветить текст.

--- Вот, последние два года, --- Король-жрец ткнул в страницу узловатым пальцем с тусклым кукхватровым перстнем.
--- Хатрикас.
Смерть трёх жрецов, в том числе Первого.
Смена.
Травинхал.
Смерть десяти жрецов, без подробностей.
Смена.
И главное --- сравните.
Вот, письмо сорокалетней давности.
Тогда инциденты в храме были событием чрезвычайным, и отношение...
Вот, расследование, аж на три страницы.
Смена --- подробные биографии, прежние должности.
Организация работы в условиях недостатка кадров, обнаруженные уязвимости, предложения по их устранению.
О, вы посмотрите, после того случая они даже форум организовали, вот тезисы и результаты.
Вот, кстати, тоже хороший отчёт --- три дождя назад из Тхитрона, некоего Трукхвала ар'Хэ.
Всё на месте, несмотря на то, что старик остался там чуть ли не в одиночестве.
А теперь взгляните на вот это.
Это отписки, по-другому назвать нельзя.

--- А какие причины? --- поинтересовалась Анкарьяль.

--- Причины разнообразные --- диверсии идолов, пылерои, несчастные случаи.
Почему-то очень часто --- взрывы.
И такие сообщения из восьми городов.
Ещё два замолчали безо всяких сообщений, при том, что караваны оттуда ходят до сих пор...
Безумные меня озари...

Король-жрец выхватил у Грейса лампу, подошёл к висящей на стене карте и обвёл пальцами несколько городов.

--- Землю сели рвут на куски, Митрис, --- подтвердила Анкарьяль.
--- И чем больше мы медлим, тем ближе гражданская война.

Король-жрец потрясённо посмотрел на неё.

--- <<Разделяй и властвуй>>?
Города спорили насчёт земель, были даже военные столкновения!
У меня и мысли не возникало, что это спланированная кампания!

--- Король-жрец, --- Чханэ впервые обратилась к нему, --- эти трое знают, что делать.
Я видела, на что они способны.
Лис... Ликхмас ар’Люм спас меня после расчленения на алтаре.

--- Чханэ, --- я жестом остановил девушку.

Король-жрец сделал круг по комнате и остановился у узенького окна.
Заходящее солнце сверкнуло в его длинных каштановых волосах.

--- Я знаю, \emph{Чханэ ар’Качхар}, на что способны боги.
Я видел радужное безумие и расплавленные огнём стены.
Вы говорите, что теперь их тысячи.
\ml{$0$}
{Но если вы пришли ко мне, значит, есть средство, чтобы их остановить?}
{But, if we are talking, there is a way to stop them, isn't there?''}

\ml{$0$}
{--- У нас есть \emph{методы}, --- веско сказал Грейс.}
{``We have \emph{methods},'' Grejs firmly said.}
\ml{$0$}
{--- Но нам нужно множество тех, кто пойдёт за нами.}
{``But we need lots of followers.}
\ml{$0$}
{Мы не знаем, насколько сильны пришлые --- возможно, нам нужен будет перевес в численности не меньше чем сто к одному.}
{We do not know how strong these outlanders are; perhaps they should be outnumbered more than a hundred to one.''}

--- Сто к одному? --- резко повернулся к нему Король-жрец.
--- Если это правда, то мы не можем победить на поле боя при любом перевесе.

--- Вы не можете, и мы не можем, --- сказал Грейс.
\ml{$0$}
{--- Поэтому мы нужны друг другу.}
{``That's why we and you need each other.}
Мы хотим позвать не только сели, но и ноа, и возможно даже пыле...

--- Почему именно ноа?

--- Порт Коралловой бухты --- важный стратегический пункт, --- ответил я.
--- Кроме того, ноа --- ближайшие родичи сели, а это значит...

--- Ничего это не значит, --- проворчал Король-жрец.
--- Самые страшные распри --- всегда между родичами.
И не нужно водить меня за нос Коралловой бухтой.
Я прекрасно помню, что возле Яуляля есть нечто более древнее и ценное.
По-видимому, вы знаете, что это такое, и намереваетесь это использовать как оружие.

Мы с Грейсом переглянулись.
Анкарьяль опустила голову, улыбаясь чему-то.

--- Скорее как щит, --- ответил Грейс.
\ml{$0$}
{--- В каком-то смысле это и был щит против таких, как мы.}
{``It was, in some way, a shield against beings like us.}
\ml{$0$}
{Ваши предки, тси, не особо жаловали наше племя.}
{Your roots, Qi, did not welcome our kind.''}

\ml{$0$}
{--- Думаю, у них была причина.}
{``I suppose they had a reason.''}

\ml{$0$}
{--- Была, --- согласился я.}
{``They had,'' I agreed.}
--- Но поверь --- лучше уж мы, чем они.
\emph{Эти} не станут с вами даже разговаривать.

--- Они уже с нами не особо разговаривают.

Король-жрец сделал ещё один круг по комнате.

--- Вы хотите, чтобы в походе участвовало как можно больше, но, как видите, я теряю власть.

--- Ну что ж, мы рады, что Король-жрец понимает суть задачи, --- подытожил Грейс.
--- Нам нужен народ сели.
Единый и готовый сражаться за свободу.

--- Означает ли это, что мне следует совершать диверсии во враждебных Храмах?

--- Храмы для нас потеряны.
По крайней мере пока, --- сказала Анкарьяль.
--- Диверсионную войну нам не выиграть --- один на один демон разделается с любым воином.
Более того, я бы причислила к врагам все Храмы без исключения, так как Безумный сейчас играет на стороне пришельцев, а для большинства жрецов защита народа важнее твоих приказов.
Тебе нужно заручиться поддержкой тех, кто живёт вне храма --- крестьян, ремесленников и купцов.

--- Почему заговорщикам просто не совершить переворот здесь?
Они могли бы получить страну, не наделав шума.

--- Гражданская война выгоднее, --- объяснил я.
--- Они питаются страданиями, как и Безумный.
Кроме того, после гражданской войны человек теряет веру в дружбу и родство.
Такими людьми проще управлять.

--- А как \emph{мне} отличить врагов от друзей? --- с едва заметной иронией спросил Король-жрец.

--- Начни с меня, Башенка, --- сказала Анкарьяль и положила руку ему на плечо.
--- Друг я или враг?

Король-жрец молча смотрел на женщину.
В его глазах блеснуло давнее, уже подзабытое чувство.

--- Со столицей я разберусь, --- пообещала Анкарьяль.
--- А в дальнейшем полагаюсь на твой ум.

\section{[-] Лампа и щит}

\textspace

<<Так что это за лампа, Грейс?>> --- знаками спросил я.

<<Это не лампа, --- оскалился технолог.
--- Оптический процессорный блок широкого диапазона, переделанный в лампу.
Скорее всего, по причине неустранимой поломки контроллера, вернее, одного из его модулей.
Возможно, часть корабля тси.
А я ещё думаю, какой-то свет подозрительно ровный...
Давай подробности позже?
Я сутки буду на пальцах объяснять>>.

<<Интересное в памяти нашёл?>>

<<Накопитель зашифрован, я его оцифровал, но, нюхом чую, ничего важного там нет.
Сейчас, увы, это просто ночник Короля-жреца>>.

Когда мы вышли из кабинета, почти все воины уже пришли в себя.
Они поприветствовали поклоном Короля-жреца и чуть более глубоким --- Анкарьяль.
Оглянувшись на подругу, я увидел на её лице слабую ироничную улыбку.
Люди не меняются нигде и никогда.

--- Итак, жду вестей, --- заключил Король-жрец.
Анкарьяль кивнула и решительным шагом направилась к двери, перед этим незаметно зацепив его робу кончиками пальцев.

\ml{$0$}
{--- Король-жрец, почему ты нам поверил? --- тихо спросил я.}
{``Priest-king, why do you trust us?'' I asked quietly.}

\ml{$0$}
{--- Ты назвал пришедших с тобой <<друзьями>>, --- так же тихо ответил Король-жрец.}
{`` `Friends' is what you called people who came with you,'' Priest-king just as quietly answered.}
\ml{$0$}
{--- Это отступление от правил --- на аудиенции должно говорить <<спутники>>.}
{``It's a derogation, the rules prescribe to say `companions'.}
Ты, как я понимаю, не читал книгу <<Средоточие>>?

Я помотал головой.

--- Это нормально.
Если коротко, <<Средоточие>> --- это протокол по переговорам, составленный с учётом культурных особенностей всех известных народов.
Когда ведёшь переговоры, важно никого не обидеть и выразить свои мысли максимально понятно, а это сложно даже со знанием языка.
В <<Средоточии>> описаны слова, которые категорически нельзя использовать в переговорах, так как они могут быть поняты или переведены неправильно.
Очень часто результаты переговоров с одним народом передаются соседям, которые даже при использовании правильных слов могут истолковать их совершенно иначе --- вот и попробуй в таких условиях обойти все острые углы.

--- Кормилица мне рассказывала, --- кивнул я.

--- <<Средоточие>> --- это уникальная в своём роде книга, потому что у неё нет автора, она регулярно дополняется и сверяется с копиями из других городов.
Обычно книгу более-менее знает только библиотекарь, который делает выписки для купца, ну и Король-жрец, разумеется.
Прочим жрецам она нужна разве что для общего развития.
Согласно книге, называть пришедших с тобой следует <<мои спутники>>, не иначе.
Это подчёркивает, что ты --- глас народа, а сопровождающие связаны с тобой исключительно деловыми отношениями и не влияют на ход переговоров.
Воины часто становятся смертельными врагами тех, против кого они сражались, и такие осложнения тебе ни к чему.
\ml{$0$}
{Кстати, тот, кто владеет городом сейчас, употребил термин <<мои люди>>.}
{By the way, the one who rules this city now, he used the term `my men'.}
\ml{$0$}
{Когда приходит беда, она выдаёт себя речами.}
{When bad things come, their words turn them in.''}

--- Всего лишь слова.

--- Ты можешь насмехаться надо мной, называть меня легковерным дураком, да и сам я не прочь порой упрекнуть себя, однако мне позарез были нужны такие союзники, как вы.
Помнишь слова клятвы?

--- Какие именно?

--- <<Я не наврежу человеку ни моими чувствами...

--- ... ни моим невежеством>>, --- закончил я.

Король-жрец улыбнулся и, повернувшись, удалился в свой кабинет.

\section{[-] Правосудие}

\textspace

Люди вокруг одного из воинов повернулись к нему лицом и встали в круг.
Мы с Анкарьяль аккуратно отступили в заросли папоротника и затаились, наблюдая за происходящим.

Люди молчали.
Воин, оглянувшись по сторонам, сделал попытку взяться за фалангу --- в ответ окружившие его показали спрятанные в рукавах ножи.
Один из них --- хмурый, неулыбчивый мужчина, низкорослый и широкоплечий --- вышел вперёд.
Анкарьяль шёпотом сообщила мне, что это крестьянин с улицы Летающего Арбуза.

--- Нетрукх ар’Сар, --- сказал он сухим монотонным голосом.
--- Мы --- сели Тхартхаахитра --- обвиняем тебя в Насилии и Разрушении.
Сад и Цех независимо друг от друга вынесли тебе приговор --- изгнание.
Однако Храм не желает считаться с решениями Советов, и было принято решение свершить правосудие там, где это представится возможным, и теми, кому представится такая возможность.

Воин окинул взглядом окруживших его, выискивая слабое место.

--- Что я совершил и кто предоставил доказательства моей вины? --- сурово бросил он.

--- Ты мучил моего ребёнка, а потом убил его, --- хрипло сказала стоявшая справа женщина.
--- Трое моих соседей были свидетелями того, как ты повёл мальчика в джунгли.
Изуродованное тело нашёл мой мужчина.

--- Мы знаем, что это ты.
Отпираться бесполезно, --- сказал мужчина, говоривший первым, и протянул воину хэситр.
--- Возьми.

Воин молниеносно схватился за фалангу и попытался ударить крестьянина шипом гарды.
Движение было отточенным до идеала --- три взмаха, и воин выбрался бы из окружения.
Но смерть, минуя ворота, вошла через калитку;
из рукава женщины вылетел мясницкий топорик, и голова воина с глухим стуком упала на дорожный камень.
Кормилица не собиралась давать шанс тому, кто замучил её дитя.

Собравшиеся, как один, придержали тело и сели на пятки.
Крестьянин поднял голову с ещё не померкшими глазами и, аккуратно взяв её за нижнюю челюсть, вылил ей в рот хэситр.
Из перерубленной глотки потекла смешанная с водой кровь, желваки дрогнули.

--- Ты был болен, --- сказал крестьянин.
--- Болезнь твоя была страшна, и ты страдал непомерно, причиняя боль другим.
Мы были бессильны излечить тебя и могли думать только о жизни для нашего народа.
Лесные духи могут унять твою боль --- Сан-сновидец погрузит тебя в сон, Обнимающий Сит подарит тебе любовь, а Печальный Митр споёт песню, которая тебя исцелит.
Прости же своих убийц и встреть их у ворот пристанища, когда их время придёт.
Мы помним про кровь на наших руках.

--- Мы помним, --- хором отозвались окружающие.
Сидящие вблизи обмакнули рукава рубах в смешанную с дорожной пылью кровь.
Затем все, как по команде, встали на ноги.

--- Что теперь, Митрам? --- спросил кто-то у крестьянина.

--- Наш долг перед спящим выполнен, --- сказал крестьянин.
--- Похороните его.
Малыш, Чайка, Стебелёк --- за реку.
Кусочек, Рыбка, Аромат --- к себе в квартал.
Остальные за мной.
Кажется, пора напомнить Храму, кем и для чего он был построен.

Люди разошлись бесшумно, как тени.
Кормилица убитого мальчика и, по-видимому, её мужчина подняли тело воина, погрузили на носилки и куда-то понесли, негромко напевая плачущую песню.

--- Минус один активный агент Картеля, --- шёпотом констатировала Анкарьяль.
Я кивнул.

\ml{$0$}
{--- Впервые так гладко.}
{``First time it goes smoothly.}
\ml{$0$}
{Может, нам даже вмешиваться не придётся.}
{Maybe we don't have to interfere.''}

--- Придётся, --- заверила Анкарьяль.
\ml{$0$}
{--- В храме демоны рангом повыше и их больше, они могут справиться и с большой толпой.}
{``The daemons inside the temple are higher in rank and number, they can handle even a big crowd.}
Идём на площадь, подождём, пока котелок закипит.

\section{[-] Обычай}

\epigraph
{<<Мысль материальна>> --- эту идею старательно внушают вам те, кто боится ваших действий.
Мысли и молитвы ничего не значат --- ни ваши, ни мои!
Можно хоть тысячу лет говорить о прекрасном городе, но пока хоть один человек не возьмёт в руки мастерок, на месте города будут расти девственные леса.}
{Анатолиу Тиу.
Речь перед лиманскими повстанцами}

Когда-то давно, в детстве, я видел, что такое Дело Перекрёстка.

Глубокой ночью раздался чёткий и громкий тройной стук в дверь.
Кормильцы проснулись и тут же начали одеваться.
По-военному.
Делали они всё без излишней торопливости, но и без промедлений.

Минуту спустя Кхотлам, в последний раз поправив <<разбойничьи крылья>> и пояс, собрала на затылке пучок и закрепила его заколкой хэма --- знаком дипломата и беспристрастного арбитра.
Хитрам протянул мне маленький кинжал, а кормилица ласково сказала:

--- Лис, малыш, иди в погреб и ложись где-нибудь в укромном месте.

Я взял кинжальчик и спустился на первый этаж.
Меня поприветствовали слуги --- старый Сиртху и милая Эрхэ, оба в доспехах и при оружии.
Сиртху отвёл меня за руку в погреб и аккуратно закрыл за мной дверь.

Тогда всё закончилось благополучно, но я так и не сомкнул глаз.
Дело было не в духоте и не в том, что спать пришлось на тюках, укрывшись ковром.
Погреб в нашем доме не был чем-то отделённым от внешнего мира --- там слышался шум ветра, шаги людей, разговоры и смех.
В ту ночь же царила невероятная, потрясающая тишина.
По этой тишине всегда можно отличить Дело Перекрёстка от вторжения.
Проникновение врагов в город --- это крики, топот и звон оружия.
Люди непрерывно сообщают друг другу, где опасно и какими тропами идёт враг.

Дело Перекрёстка всегда проходит в молчании.
Единственными звуками до того, как всё начнётся, были и остаются они --- три зловещих стука в дверь.

Так было и сегодня.
Люди просыпались, надевали доспехи и присоединялись к товарищам;
некоторые обходили дома соседей и стучали в их двери.
Однако напряжение было из ряда вон.
На город надвигалась гроза с востока;
тяжёлый воздух давил на грудь, по коже пробегали электрические ящерицы.
Вскоре из квартала ювелиров послышался звон оружия, прервавшийся ужасным, позорным для сели предсмертным криком.
Мы с Анкарьяль переглянулись.
Разумеется, это было сигналом для тех, что засели в храме.

Толпа стеклась на площадь бесшумно, как потоки чёрной воды.
Факелов и фонарей не было --- их зажигали только в начале церемонии.
Никто даже не переговаривался --- это не запрещалось, но осуждалось.
До начала никто из людей не знал, зачем их позвали.

Все терпеливо ждали появления виновника торжества.
При всей серьёзности обычая бывали случаи, когда Дело Перекрёстка начинал какой-нибудь горячий юнец, желавший показать себя.
Однако если собравшиеся решали, что вопрос не стоит двух часов сна, зачинщика ждал сезон самой тяжёлой работы.
Такое же наказание грозило тем, кто после третьего удара в дверь остался в постели --- пренебрежение к судьбе народа сели не прощали.

Наконец в дальнем конце площади появился робкий огонь факела, и все с облегчением зажгли свои.
Многие зашептались, едва перекрывая далёкий вой предгрозового ветра, слабое шуршание горящей травы и треск перегретого дерева.

Несколько десятков человек протащили что-то через толпу.
Люди на мгновение разошлись, образовав зазор, и я смог разглядеть страшную ношу --- это были человеческие тела в доспехах.

--- Пытались нас остановить, --- пояснил толпе высокий мускулистый мужчина с толстой, как дерево, шеей и стал аккуратно раскладывать безвольные тела у подножия храма.
Несколько человек отделились от толпы --- кто-то узнал родственников и друзей, кто-то просто решил помочь с посмертным обрядом и захоронением.

Зачинщик Дела Перекрёстка --- эту роль под молчаливое согласие прочих взял на себя крестьянин Митрам --- встал на вторую ступень пирамиды и, вынув из ножен фалангу, поднял её над головой.

Шёпот моментально стих.

\section{[-] Дело Перекрёстка}

\epigraph
{И сказал Талим воинам: <<Hе подумайте, будто Я пришел, чтобы упразднить Закон или разрушить Порядок;
не пришел Я, чтобы упразднить, но чтобы восполнить недостающее>>.
И сложили воины оружие, и сняли воины панцири, и приветствовали Талима как старого друга.}
{Хакем-Аят, 5:17--18}

--- Дело Перекрёстка --- решаем, куда идти, --- начал Митрам ритуальной фразой, и слова его низковатого хриплого голоса разнеслись над затихшей толпой.
--- Храм перестал выполнять свой долг.
Храм даёт приют и защиту Разрушителям и Насильникам.
Сегодня мы свершили правосудие над Нетрукхом ар’Сар.
Его вина была доказана на советах Цеха и Сада, но Храм отказался изгнать его.

В воздух взметнулись несколько фаланг в ножнах.
Этим люди показывали, что желают взять слово.

--- Я прошу собравшихся вне очереди дать слово Тхаласу, чтобы он объяснил смерть этих людей, --- продолжил Митрам.
Высокий мускулистый мужчина кивнул и тоже поднял зачехлённый топор.
--- Есть ли те, кто против?

Ответом было молчание.
Митрам вложил оружие в ножны и повёл затёкшей рукой.
Тхалас снял с топора чехол и поднял его над головой.

--- Люди, которых мы убили --- воины из храма и купец, который прибыл позавчера, --- Тхалас приятным, немного скрипучим басом поочерёдно назвал их имена.
--- Они пытались силой и уговорами посеять смуту, забыв, что Перекрёсток --- не место для привала.
Манис, слово тебе.

Следующей обнажила фалангу худенькая, как тростинка, болезненного вида женщина.

--- Мой ребёнок Кхотси недавно заболел.
Он уже шёл на поправку, когда пришёл Кхирас ар’Люм и забрал его для алтаря.
Он не исполнил надлежащего ритуала и не спросил согласия ребёнка.
Ягодка, тебе слово.

--- Что здесь происходит? --- раздался строгий окрик с вершины храма.
Люди запрокинули головы, расматривая закутанную в робу фигуру.

--- Здесь Дело Перекрёстка, Марас, --- громко сказал Митрам.
--- Если тебе есть что сказать --- спускайся.
И остальных позови.

<<Готовность номер один, --- я ласково подёргал Анкарьяль за прядку волос на Эй-F14.
--- Марас похож на демона.
Общая модуляция\FM, воздействует на центр страха>>.
\FA{
Общая модуляция --- модуляция голоса, которая вызывает у произвольной группы сапиентов определённого вида максимальную реакцию в виде определённой эмоции.
}

<<Общая модуляция?
Самонадеянно, не находишь?
Это всё-таки потомки тси, а не дикари с периферических планет>>, --- Анкарьяль потеребила меня за штанину на том же языке.

Разумеется, демон знал о присутствии Ада и собирался давить на толпу максимально грубо.
Он вычислил, что перевес в силе будет на его стороне, пока толпа не приняла решение.
Вывод был абсолютно верным.
Если бы нас обнаружили, то дело решилось бы не убеждением людей, а простой дуэлью хоргетов.
А то и экраном --- мы не знали, какое оборудование Картель здесь успел собрать и установить.

Мы с демоницей аккуратно скользнули между людей и заняли место в десяти шагах от подножия храма.
Марас и ещё пятеро жрецов величественно, намеренно не спеша спустились по лестнице.
Из боковых дверей бесшумно вытекли воины.
Четверо смело спрыгнули в толпу и встали вместе со всеми, ещё шесть человек остались стоять у средней стены.

<<Вот они все, конфетки, --- с улыбкой повертела лучистыми глазами Анкарьяль.
--- Презрение к сапиентам играет с ними злую шутку>>.

--- В чём нас обвиняют? --- сурово спросил Марас, оглядев толпу.
Тон был подобран великолепно, и кое-кто смущённо потупился.

--- \emph{Тебя} пока ни в чём не обвиняют, Марас, --- громко сказала Анкарьяль.
--- Но если ты хочешь повиниться, то дождись своей очереди.

В толпе раздались смешки.
Психологическое нападение было блестяще отбито.
Ближайшие ко мне люди посмотрели на воительницу с одобрением, некоторые окинули настороженным взором меня.

--- Вы даже не представляете, чью кару вы сейчас навлекли на весь город! --- неприятно каркнул похожий на тощую ворону жрец.

--- Не испытывай наше терпение, Эрликх, --- вдруг заговорил Митрам, и его слова камнем легли на мои плечи.
Крестьянин модулировал речь не хуже демона.
--- Жрецы --- исполнители обрядов, владыки над словом и силами природы, но Перекрёсток вне вашей власти.
Сейчас мы не имеем ни лиц, ни золота, ни домов, ни профессий, ни родного племени, сейчас мы все --- слушающие, смотрящие и говорящие тени.
Поэтому закрой рот и жди своей очереди!

Последние слова прозвучали одновременно с далёким громовым раскатом и повисли в воздухе, словно едкий дым пожарища.
Похожий на ворону жрец промолчал и, нервно переступая сапогами, посмотрел на Мараса.
Тот застыл, подобно статуе.

--- Ягодка, слово тебе, --- уже спокойнее добавил Митрам.

Ягодкой оказался молодой парень-плотник.
Он вытащил из ножен фалангу и молча поднялся на уступ.

--- На моих глазах убили человека.

Убийство --- событие не из редких;
но было в тоне парня что-то, что заставило всех притихнуть.

--- Дело было возле хутора Солнечная Поляна.
Я пилил ветку чёрного дерева, когда с тракта прилетел всадник на взмыленном олене.
Вероятно, он принёс какие-то вести с востока.
Но с дерева спрыгнул убийца и заколол всадника.
Вскоре подоспела погоня --- два воина.
Убийца переговорил с ними, и вместе они оттащили тело к обочине.

Стояла оглушительная тишина.
На время умолк даже ветер.

--- Я принял происходящее за казнь разрушителя.
Но кое-что смутило меня.
Убийца говорил на неизвестном мне языке, и те два воина тоже.
Они действовали так, словно не желали свидетелей.
Они не совершили посмертного обряда, как полагается людям сели поступать с мертвецами, никто даже не вылил в его рот полагающийся напиток.
А главное --- на его руке не было никаких знаков.
Рука лежала в мою сторону, и в этом я ручаюсь.
В убийце я узнал Митракх ар’Кхир.

Толпа, как один человек, посмотрела на среднюю стену, на стоявшую поодаль низкорослую коренастую воительницу.
Внезапно руку с жреческим кинжалом поднял старик со взглядом и повадками хищной птицы.

--- Я всё сказал.
Хитрам-лехэ, тебе слово, --- парень кивнул жрецу и опустил оружие, поморщившись от боли в уставшем плече.

--- Может ли кто-то подтвердить слова Хонхо ар’Лотр? --- осведомился Хитрам.

Молчание.

Жрец усмехнулся:

--- Так я и знал.
Его слова так же нельзя подтвердить, как и слухи о насилии над детьми в Храме, как и многое из того, что я услышал здесь.
Я уважаю Митрама --- я лечил его детей ранее, одного из них мы с соблюдением всех обрядов увели в храм и принесли в жертву.
И мне кажется, что у него нет повода усомниться во мне.
Да, Митрам.

--- Я не сомневаюсь в тебе, Хитрам-лехэ, --- заговорил крестьянин, едва успев вытащить оружие.
--- Но ты не можешь говорить от имени всех.
Хай?
Да, человек, говори.
Только назовись для начала, я тебя не знаю.

Я аккуратно поднял обнажённую саблю.

--- Ликхмас ар’Люм э’Тхитрон, --- я рискнул назвать настоящее имя, вспомнив, что в книге Митриса его нет.
--- Я не уроженец и не житель Тхартхаахитра, но я сели.
Есть ли те, кто оспорит моё право на речь?

Толпа молчала.
Я продолжил:

--- Я понимаю желание почтенного Хитрама защитить подчинённых ему людей.
Но я также служу Храму. И каждый должен помнить, что он не может быть лицом Храма.
У Храма нет и не будет лица.

Откуда-то справа донесся одобрительный гул.

--- Народ сели всегда давал Храму самостоятельность в обмен на исполнение Храмом его роли.
Все знают, что люди в храме гибнут не только на алтаре, но это никого в городе не касается --- до того самого момента, пока жрецы и воины не начинают скверно исполнять свои обязанности... или покрывать Насилие и Разрушение.

Одобрительный гул сменился криками гнева.

--- Пустая болтовня! --- рявкнул Хитрам-лехэ.
--- Как будто мы... --- Старик осёкся под тяжёлым взглядом Митрама.

Марас вскинул руку с кинжалом.

--- Да, почтенный Марас, --- сказал я.

--- Сегодняшняя ночь похожа на низкопробный спектакль, --- выплюнул жрец, и я снова ощутил мощное интонационное давление.
--- Мы принимаем решение, важное для столицы, а слово берёт уроженец Тхитрона.
Кто ты такой, чужак?
Друг ли ты?
Может быть, ты и вовсе, --- Марас сделал многозначительную паузу, --- \emph{кутрап}?

На меня оглянулись все, кто стоял рядом.

<<Спокойно, --- тут же дёрнула меня за руку Анкарьяль.
--- Если бы он знал наверняка --- мы были бы уже мертвы.
Стой на месте>>.

--- Кто \emph{знает} об убийствах в храмах? --- продолжал Марас.
--- Покажи мне тех, кто \emph{знает}!

Окружающие сверлили меня отнюдь не дружелюбными взглядами.

--- Далее.
Мы подвергаем воинов суду дикарей\FM, забывая о том, что закон предписывает сначала выслушать рассказ о случившемся из их уст.
\FA{
То же самое, что суд Линча.
}
Не говоря уже о том, что по тому же забытому вами закону за первое доказанное Разрушение полагается клеймо и изгнание, а не смерть.
Да-да, Митрам, за суд дикарей над Нетрукхом я мог бы обвинить в Разрушении тебя!

Кто-то громко выругался.
Я подпрыгнул и успел увидеть, как люди, взявшись за руки, образовали <<круг спокойствия>> вокруг матери убитого мальчика.
Самообладание изменило ей.

--- А меня, Марас?!
Это я убила Нетрукха!
У тебя хватит наглости обвинить в Разрушении меня?! --- кричала она, потрясая мясницким топориком.

Марас продолжил, не дожидаясь, пока она успокоится:

--- Далее.
Мы обвиняем убийцу в Разрушении на основании совершенно бессмысленных слов одного человека.
На каком таком незнакомом языке может говорить Митракх, если любой сели говорит на всех известных языках или хотя бы знает, как они звучат?
Как можно сделать вывод о преступлениях человека лишь по отсутствию знаков на его руке?
Я такого бреда в жизни не слышал.

На этот раз толпа забеспокоилась сильнее.
Парень-плотник открыл рот и попытался что-то крикнуть, но стоящие рядом властно положили руки ему на плечи, принуждая соблюдать правила.
Митрам хмурился, скрестив руки на груди.

--- Почему, вместо того чтобы вызвать на совет Митракх ар’Кхир и спросить у неё обстоятельства этого убийства, люди занимались пересудами и распусканием слухов?

<<А ты хорош, ящерица раздавленная>>, --- одними губами выругалась Анкарьяль.

--- И самая вкусная часть манго, напоследок, хотя именно с неё уважаемый Эрликх хотел начать.
Да, Митрам, любящий закрывать чужие рты, это я тебе сейчас! --- Марас ткнул в крестьянина пальцем.
--- Почему Дело Перекрёстка устроили в тот момент, когда мы ожидали знака кирпича?
В результате я вынужден был спуститься и защищать свою честь вместо того, чтобы защищать город и его...

--- А меня ты уже за жреца не считаешь, Марас? --- раздался с вершины пирамиды величественный рокот.

Король-жрец спускался вниз, на ходу стягивая с себя <<кровавый плащ>> и обтирая им испачканные руки.

--- Прошу прощения за неподобающий вид.
Ещё прошу слова вне очереди.

Толпа согласно зашумела.
Марас замер, с плохо скрываемой ненавистью глядя на Короля-жреца.

--- Может быть, кто-то это уже сказал, но я повторю: жрец отличается от крестьянина тем, что не может бросить пост, даже если угрожают его чести и жизни.
Марас приносил другую клятву, не такую, как я?
Безумные нарисовали знак кирпича своей дланью, но приносить жертву пришлось единственному жрецу, который ещё не забыл о долге.
Я как мог просил ребёнка потерпеть ради народа --- иначе мне одному не удалось бы уложить его на алтарь.

Толпа дружно ахнула.
Анкарьяль прикрыла лицо рукой.

<<Земля и небо, знак кирпича.
Они были готовы всю столицу!..>>

<<Спокойно.
Бой ещё не закончен, --- одёрнул я подругу.
--- Будь начеку>>.

--- Слухи, которые ходят в народе --- отнюдь не слухи, --- продолжал Король-жрец.
--- Хотите знать, кто их распускает?
А давно ли вы видели в городе Согхо?
А Ликхлама?
Я надеюсь, что лучшим воинам моего Храма удалось уйти живыми от Митракх.

Анкарьяль шлёпнула меня по руке и стала проталкиваться к Королю-жрецу.
Я осознал всю опасность ситуации --- он играл по-крупному.

--- Я даже скажу вам больше, --- сказал Король-жрец и, схватив под руки старика Хитрама и крепыша Митрама, начал спускаться вниз.
--- Первая ступень храма (шаг) --- это не просто его основание (шаг).
Это то (шаг), что сейчас...

Король-жрец отпустил руки спутников и повернулся лицом к замершим на ступенях жрецам.

--- ... отделяет людей от нелюдей.

Я ожидал криков, но толпа ошеломлённо молчала.

--- Я обвиняю в Насилии и Разрушении тех десятерых, что стоят на ступенях.

--- А доказательства, Король-жрец? --- осведомился Марас.
--- Или ты думаешь, что твой пост освобождает от необходимости доказывать?

--- Что за губки каждый из вас засунул в правую ноздрю, прежде чем вы вышли сюда? --- ответил вопросом Король-жрец.
--- Они и сейчас ещё не растворились, и их можно заметить --- носы у тебя и стоящих с тобой вздуты справа.
Именно так я определил, что Хитрам-лехэ не с вами --- у него нос чистый.

Толпа зашлась в тревожном шёпоте.

--- Бред, --- рявкнул Марас.

--- У Митракх и Эрликха губки потекли, оставив на губах опалесцирующие полосы, --- Король-жрец указал на воительницу и худого жреца.
--- Я никогда раньше не видел подобное вещество.
Неужели это действительно лекарство, которое помогает избежать радужного безумия, или это мне просто послышалось?

Митракх неожиданно ягуаром прыгнула на Короля-жреца, но между ними тенью возникла Анкарьяль.
Коренастая воительница всхлипнула и обмякла, Анкарьяль отбросила тело в сторону и вскинула окровавленный нож.

\ml{$0$}
{--- Даже не думай, --- прорычала она Марасу.}
{``Don't even think about it,'' she growled at \Maaras.}
\ml{$0$}
{Её демон предупредительно вспыхнул, обнаружив таким образом себя.}
{Her daemon showed itself by warning `flash'.}

Толпа как по команде ощерилась в сторону демонов натянутыми луками и духовыми ружьями.

Картель умел признавать поражение.
Они не стали дожидаться конца церемонии.
Марас бросил пару слов стоявшим на ступенях демонам, и те пошли через расступившихся сели прочь, в сторону городских ворот.
Перед тем, как уйти, Марас просверлил взглядом Анкарьяль и криво, судорожно улыбнулся:

--- Ещё увидимся, Ангара Краснобуря.

\section{[-] Военный совет}

Закончилась церемония Перекрёстка мирно.
Гроза прошла стороной, на площадь упало едва ли десять капель.
Местный купец объявил голосование, по которому из народа были временно выбраны люди для координации военных сил и отправления обрядов.

Король-жрец вынужден был сам ходить по больным, оставив роль учителя старику Хитраму.
Казалось бы, какая учёба в такие суровые дни?
Однако большая часть горожан восприняла переворот как нечто само собой разумеющееся.
Более того, Хитрам-лехэ целых пять дней посвятил законам сели, пока впечатления детей от Дела Перекрёстка были ещё свежи.

Сбежавшие жрецы вскоре приехали обратно и занялись обычными делами.
На исходе декады вернулись воин-цикада Ликхлам и молча вручили Королю-жрецу свёрток с девятью разномастными прядями волос.
Пряди служители Храма торжественно вплели в убранство маленького, незаметного тотема, который появился у задних ворот храма вскоре после Дела Перекрёстка.

На сегодняшнее совещание Король-жрец опаздывал.
Кресло Анкарьяль тоже пустовало.
Заскучавшие люди катали на каменном столе мячик.

--- Да что ж такое, --- не выдержал Хитрам.
--- Сколько их ждать?

--- Хитрам-лехэ, ты уже старый, тебе не понять, --- игриво бросила Чханэ и ткнула меня под рёбра.
Остальные присутствующие рассмеялись и продолжили перекидывать тугой кожаный шарик.
Хитрам сверкнул на воительницу выпученными глазами, но промолчал.

--- Мне тоже не понять, --- заметили Ликхлам.
--- Но если надо --- значит, надо.

--- В Кругу все равны, --- пожала плечами Чханэ.

--- Я участвую в Кругу как бард, --- сообщили Ликхлам.
--- Я ни с кем не делили ложе и не собираюсь.

Неразлучная парочка появилась вскоре после разговора, принеся в каменный зал запах прелой травы и густой аромат феромонов.
Анкарьяль непринуждённо села в кресло и углубилась в бумаги.
Король-жрец всеми силами пытался стереть с лица широкую довольную улыбку, не зная, что следы зубов на шее и сухие травинки в волосах и так выдают его с головой.

<<Что? --- одними губами спросила Анкарьяль у выразительно глядящего на неё Грейсвольда.
--- Он нам план спас.
Должна же я его отблагодарить>>.

<<Ты его с самого Дела Перекрёстка благодаришь по десять раз на дню>>, --- напомнил Грейс.
Анкарьяль, сложив большой и указательный пальцы кольцом, показала через него язык и подмигнула мне.

Воины с любопытством смотрели на непонятную жестовую беседу.
Что удивительно --- адепты Трёхэтажного Храма восприняли новость о демонах очень спокойно.
Никто не просил нас показать чудеса, как бывало на других планетах, через пару дней перестали пялиться, а сегодня меня и вовсе подняли как дежурного по кухне.
Подумаешь, храм захватили демоны-враги, а потом их победили демоны-союзники, не нарушать же из-за этого режим.
Пришлось готовить.

Король-жрец словно прочитал мои мысли.

--- Ликхмас, сегодня на завтрак мне подали изумительный ягодно-мясной суп, кажется, из пятой главы\FM.
\FA{
Имеется в виду пятый раздел <<Десяти тысяч блюд>>.
}
А на Втором этаже поинтересовались, не собираешься ли ты остаться насовсем.
Есть подозрение, что эти факты связаны.

--- Просто я вычисляю точное количество ингредиентов по таблице, а не сыплю на глаз.

--- Мне интересно, в этом Храме хоть кто-то, за исключением библиотекаря, заглядывал в таблицы Книги-кормилицы?
Те самые, которые в конце? --- вздохнул Король-жрец.

Присутствующие, включая Анкарьяль, вдруг нахохлились.

--- Впрочем, ладно.
Перейдём к делу...

\section{[-] Ликхлам}

У входа я едва успел перепрыгнуть через кого-то, лежащего поперёк прохода.

--- Хэ!

--- Всё нормально, это Ликхлам, --- успокоил меня Король-жрец и перепрыгнул через лежащее тело.
--- Они любят путаться под ногами.

--- Вообще-то я на посту, --- сообщило тело женственным контральто.
--- Так, на волосы не наступайте, только расчесали.

--- Ты хочешь отрастить волосы, как у Маликха? --- усмехнулся я, окинув взглядом вход в храм.
Осветлённые каким-то веществом пряди лежали везде, словно их обладатель не стригся с рождения.

--- Я тебе скажу больше, Ликхмас: я уже отрастили волосы лучше, чем у Маликха, --- кокетливо сказали Ликхлам.
--- Легендарная коса в два кхене и реальная в человеческий рост --- это немного разные вещи.

--- Не в человеческий, а в пылеройский.
В тебе два цана роста, забыли? --- усмехнулся Король-жрец.

Мы прошли чуть дальше.

--- На самом деле они очень опытные, --- шёпотом сообщил мне Король-жрец.
--- Их привычка лежать, а не стоять у входа уже спасала нас несколько раз.
Лазутчик поднимается, смотрит --- в дверях никого нет.
Идёт к окнам, спокойно залезает в здание --- и ему на плечо ложится лучший клинок Тхартхаахитра: <<Добрый вечер!
Чем могу помочь?>>.
Они откуда-то с Водораздела родом, поэтому не знаю, где они этого нахватались --- тактика изумительная.
К советам Ликхлам лучше прислушиваться.

--- Я уже понял, --- кивнул я.
--- Они единственные смогли убежать от демонов.

--- После того, как попытались отравить их за обедом.
Никому из нас ничего не сказали, сами провели расследование, сами спланировали покушение, когда поняли, что баланс власти нарушен.
Не удалось, конечно --- я так понял, что яд вы умеете распознавать.

--- Не только распознавать, но и составлять противоядие, когда яд уже в крови, --- усмехнулся я.

--- Хай, об этом Ликхлам точно знать не могли.
Но одного демона они всё-таки зарубили, пока убегали --- случайно встретили.
Я же говорю --- одни из лучших здесь.

--- Интересно, какой совет они дали бы мне.

--- Мне они однажды дали совет, который я запомнил на всю жизнь.
<<Играй любые вариации, не выходя из аккорда>>.
Очень помогает при переговорах.

--- Придерживайся основной линии, лавируя в деталях?

--- Именно.
Музыку всегда делают маленькие отдельные ноты.

\section{[-] Песня Ликхлама}

Из зала доносился богатый голос Ликхлама под тяжёлый бронзовый аккомпанемент цитры:

\begin{verse}
\ml{$0$}
{Яркие краски,}
{Brightness of colors,}\\
\ml{$0$}
{Тёплая кожа,}
{Skin heated by arteries,}\\
\ml{$0$}
{Мерные мысли и новый момент.}
{Race of new moments, and rhythmical minds. }\\
\ml{$0$}
{Вечное счастье,}
{Edgeless happiness,}\\
\ml{$0$}
{Дни непохожие,}
{Different afternoons,}\\
\ml{$0$}
{В солнечных линиях новый рассвет,}
{Covered by sun rays renewed sunrise,}\\
\ml{$0$}
{Новый рассвет...}
{Renewed sunrise ...}

Долгими кхене, тропами резвыми,\\
Надо ли, надо ли\\
Идти нам в туманы ли?\\
Бежать от обмана\\
Дорогами главными,\\
Длинными ночами\\
Под буйными травами?
\end{verse}

--- Красивая песня, --- мечтательно проговорила Чханэ.
--- Люблю её.

--- Так, как Ликхлам, её никто не поёт, --- заметил проходивший мимо Трукхвал.
--- Даже те сотронцы, которые её считают своей, сказали, что песня как будто для его голоса сочинена.

\section{[-] Выговор}

С улицы зашёл жрец и тут же сбросил с себя промокшую насквозь робу.

--- Я не уверен, но, по-моему, тотем Сомнения буря снесла к свиньям, --- сообщил он.

Двое жрецов побежали и выглянули наружу.

--- Спорное утверждение, --- заявил один.

--- Я бы даже сказал, дискуссионное.

--- Подождите, --- засмеялся я.
--- Здесь есть тотем Сомнения?

\ml{$0$}
{--- Вряд ли кто-то может сказать наверняка, --- сказал Король-жрец.}
{``I wonder if anyone can tell it for sure,'' Priest-king said.}
\ml{$0$}
{--- Но есть предположение, что он стоит...}
{``Nevertheless, there is speculation that it stands---''}

\ml{$0$}
{--- Стоял, --- хором поправили его жрецы.}
{``Stood,'' priests corrected him in unison.}

\ml{$0$}
{--- Да, благодарю.}
{``Yes, thank you all.}
\ml{$0$}
{Есть предположение, что он стоял у западного входа в храм.}
{There is speculation that it stood at the western entrance to the temple.''}

\ml{$0$}
{--- Я не уверен, что это предположение доказуемо, --- заявил Тхалас.}
{``I'm not sure the speculation is provable,'' \Tchalas\ declared.}

--- Возможно, имеют место иллюзии восприятия, --- ввернул Трукхвал.

--- Может, хватит? --- рявкнул Хитрам-лехэ.

Жрецы переглянулись и расхохотались.

--- Извини, Ликхмас, --- утирая слёзы, проговорил Король-жрец.
--- Но это такая весёлая вещь.
Традиция пришла, как я понимаю, от Людей Золотой Пчелы, которые сейчас обосновались в Тхитроне.
Ты ведь тоже из них?

--- Я провёл с ними какое-то время.

--- Поэтому ты в курсе.
А вот наш плотник был не в курсе.
Когда мы пытались объяснить ему, что нужно, он решил, что Храм в полном составе спятил.
Я даже не знал, что у Бродячих Храмов есть такие забавные традиции.

--- Я не поддерживаю идею подобных бессмыслиц, но плотнику надо сделать выговор за халтуру, --- заявил Хитрам-лехэ.
--- Где это видано, чтобы новые столбы валились от одной бури?

--- Не надо выговоров, --- поднял руку Король-жрец.
--- Просто попросите его починить.
Для ремесленника это достаточно понятный намёк.

\section{[@] Лампа}

Я думаю, что пришло время рассказать о буднях.

Впрочем, что рассказывать?
В общем и целом наша жизнь мало отличается от той, что мы вели на Тси-Ди.
Я понял это сегодня под вечер, когда побережье Коралловой Бухты огласила прекрасная песня, лившаяся непонятно откуда.

<<Ты это слышал?>> --- позвонила мне Листик.

<<Это не Безымянный?>> --- осведомилась Пирожок.

<<Это не я, --- тут же сообщил бог, едва я соединил его в конференцию.
--- Но звучит очень красиво.
Источник звука --- обшивка Стального Дракона>>.

Я отправился к кораблю.
Ошибка Безымянного была вполне простительной --- действительно, обшивка резонировала так, что песню было слышно за два километра.
Но источником звука оказался Баночка, который лежал в техническом канале.

\begin{verse}
И это лучшее на свете колдовство,\\
Ликует солнце на лезвии гребня,\\
И это всё, и больше нету ничего,\\
Есть только небо, вечное небо...
\end{verse}

--- О, привет, --- сказал Баночка, высунув голову наружу.
--- А я как раз про тебя пою.

--- Про меня? --- засмеялся я.
--- Я не вечный.
Ты в курсе, что тебя слышит половина планеты?

--- Ой, --- спохватился плант и добавил по общему каналу:
--- <<Извините, это Закрытая-Колба-с-Жизнью в техническом канале номер пятнадцать Стального Дракона.
Я не думал, что будет настолько громко>>.

<<Мне понравилось, --- сказала Пирожок, --- но давай ты вначале вылезешь наружу.
У меня есть система, которая настроена на музыкальные команды из техканалов, и если ты её случайно активируешь, я рассержусь.
Конец связи>>.

--- Откуда такая красивая песня?

--- Из романа <<Свесив ноги с края Вселенной>>, Свет-Мерцающего-Осколка, --- ответил Баночка.
--- Я подумал, раз уж мы спаслись на литературном персонаже, неплохо было бы просмотреть первоисточник.
И не пожалел --- полночи пролетели, как гамма-квант.
Послушай, помоги откалибровать уровень.
Кодить лень, и так уже час провозился.

--- Давай я, --- предложил я и достал компьютер.
--- Как раз проветрюсь, на травке посижу.

--- Хорошо, когда рядом есть друг, --- улыбнулся Баночка.
--- Здесь столько всего нового...
Сегодня выяснилось, что некоторые зашитые в ядро системы константы не годятся для этой планеты.
И это на космическом корабле, а не на личном планетном транспорте, могли бы догадаться и вынести их в настройки.
Разберёшься?
Документация плохонькая где-то была...

--- Я уже копался в хранилище Дракона, найду, --- успокоил я его.

Плант переслал мне номера моделей и снова занялся подгонкой деталей.
Вскоре программа была написана, уровень откалиброван, и я блаженно развалился в тени Дракона.
Труд --- это радость, но безделье --- счастье.

Мимо меня пролетела деталь вместе с чьим-то испуганным ругательством.
Я рефлекторно прыгнул на неё и попытался прижать к земле, но, неожиданно почувствовав невесомость, оттолкнул деталь обратно.

--- Ай!

--- Всё нормально, это трансдуктор, --- невозмутимо сообщил Баночка.
--- Его заклинило на невесомости и слегка покорёжило во время посадки, лапки спружинили, он сам выпрыгнул из креплений.
\ml{$0$}
{Ква-ква.}
{Croak, croak.}
Хомяк, ты лучше бы оптикой занялся, чем обращать внимание, что тут где летает!

--- А я чем занят, по-твоему? --- осведомился откуда-то Хомяк.
--- Кстати, вот тебе подарок.
В блоке сгорел контроллер.
Я заменил на новый, сейчас поставлю прошивку.

Сверху прилетел оптический процессорный блок, едва не разбив мне голову.

--- Небо, если хочешь ещё раз испытать радость материнства, встань под пластину, --- философским тоном посоветовал Баночка.

Я повертел блок и трансдуктор в руках.
В голову пришла идея.
Я включил мультитул и принялся за работу.

Спустя десять минут всё было готово.
Правда, трансдуктор я впаял не идеально, на шве торчали несколько острых кристаллов, но получился симпатичный антигравитационный мячик.
Я бросил его Баночке.

--- Не чересчур высокотехнологично для мячика-то? --- засмеялся Баночка.
--- Давай лучше лампу сделаю.
Ученье --- свет.

Баночка подключился к устройству.
Оно оскорблённо зажужжало.

--- А, накопитель он вытащил, вот досада.
У тебя не найдётся какого-нибудь дохлого?

Я вытряхнул из карманов хлам.
Разумеется, там нашлись три старых накопителя, парящая видеокамера и опреснитель морской воды --- эта мелочь теряется в первую очередь.
Минуту спустя Баночка поставил на процессорный блок прошивку от ночника и переписал драйвер полуживого контроллера.
Устройство засветилось в темноте мягким жёлтым светом, и мы невольно залюбовались получившейся безделушкой.

--- И чем мы занимаемся? --- протянул наконец Баночка.

--- Может, подарим царрокх? --- предложил я.
--- Толку от него всё равно нет, а выглядит красиво.
Уничтожить его у них технологий не хватит, пользоваться этой лампой сможет тысяча поколений.
Безвредная приятная вещь, идеальный подарок.

\ml{$0$}
{--- Обойдутся, --- сказал Баночка.}
{``They'll survive without it,'' Flask said.}
\ml{$0$}
{--- Я лучше Кошке подарю, она оценит этот чёрный юмор.}
{``I'd love to give it to Cat, she'll certainly appreciate this kind of black humour.}
\ml{$0$}
{Кстати, ты Кошку не видел?}
{By the way, have you seen Cat?}
\ml{$0$}
{Я о ней думал всё утро.}
{I've spent the morning thinking of her.}
\ml{$0$}
{Может, это любовь, Небо?}
{Could it be love, Sky?}
\ml{$0$}
{Ты когда-нибудь влюблялся в людей?}
{Have you ever felt in love with a human?''}

Хомяк не выдержал и высунул из технического хода всклокоченную голову:

--- Да чем вы там заняты?

\section{[@] Мечты создателей}

\textspace

--- Странно это, --- сказал я.
--- Инженеры проектировали корабль, рабочие строили его --- и всё ради того, чтобы горстка существ неизвестной им цивилизации могла спастись.

--- Для них это уже не имеет смысла, --- с сожалением сказал Баночка.

--- Это имеет смысл для нас, --- ободрительно сказал Фонтанчик.
--- Ради этого и стоит жить и работать.
Ведь однажды, хоть через тысячи, хоть через сотни тысяч лет твоя работа может спасти кому-то жизнь.

--- Жалко, что мы не знаем их имён и даже того, были ли у них имена, --- сказала Заяц.

--- Зато я точно знаю, о чём они думали, --- осклабился Фонтанчик.

Все удивлённо повернулись к нему.

--- <<Интересно, смогу ли я заставить этот кусок металла летать?>>

--- О, и я знаю, --- весело подхватила Заяц.
--- <<Да когда же начнёшь летать, как надо?>>

--- <<Мальструктура в девяносто пять нанометров.
Завтра я лично поеду в цех и буду бить их по рукам, пока эти бездельники не научатся калибровать приборы>>, --- мрачно закончил Баночка.

--- Никакой романтики в вас нет, --- обиделся я.
--- Может, они думали о звёздах?

--- Ага, только о них и думали, --- саркастически рыкнул Фонтанчик, и все засмеялись.

\section{[-] Ценная книга}

\textspace

К сожалению, книга обрывалась на самом интересном месте.
Я ещё раз с некоторым сожалением перечитал последние строчки.

--- Прости? --- раздался рядом звучный голос Короля-жреца.
Он как раз проходил мимо, и я, по-видимому, в забытьи произнёс последнюю фразу вслух.

Тут у меня родилась идея.

--- Митрис, ты не занят?

--- Если тебе что-то нужно, Ликхмас, время найду.

Я жестом подозвал Короля-жреца поближе и показал ему книгу.

--- Страниц не хватает, а мне очень хотелось бы её дочитать.
Может быть, у вас в библиотеке есть копия?

--- Хай, --- задумался Король-жрец.
--- Впервые слышу про такую книгу.
Но попробовать можно.
Идём за мной.

Я поднялся со скамьи и пошёл вслед за высокой стройной фигурой Короля-жреца.

Вскоре мы очутились возле резной деревянной двери с изображением Удивлённого Лю.
Впрочем, приглядевшись, я увидел, что его брови подправлены глубокими разрезами от ножа, которые кто-то безуспешно пытался замазать глиной.

--- Испуганный Лю на двери библиотеки появился аккурат в день первого исчезновения, --- тихо пояснил Король-жрец.
--- Это была наша лучшая воительница.
Она была убита где-то здесь, в храме, и, судя по всему, дралась до последнего.
За ней был закреплён весь Северо-Восточный квартал, её знали и уважали торговцы со всего света.
О ней многие справлялись после.
А я не знал, что ответить...

Король-жрец аккуратно постучал и приоткрыл дверь, встретив удивлённый взгляд Тхаласа --- одного из недавно вернувшихся жрецов.

--- Здравствуй, Митрис.
Нечастый ты у нас гость.
А ты?..

--- Ликхмас ар’Люм э’Тхитрон, --- сказал я и слегка поклонился.

--- Тхалас ар’Сатр э’Тхартхаахитр.

--- Это не ты случайно ломаешь?..

--- Персиковые косточки пальцами?
Да, это я, --- Тхалас возвёл очи горе.
--- Я всю жизнь переписываю и читаю эти книги, я могу проконсультировать тебя по любой из них.
Но в городе обо мне знают только то, что я ломаю пальцами персиковые косточки.

--- Надо признать, люди говорят об этом с восхищением, --- рассудительно заметил Король-жрец.

--- В бездну такое восхищение, --- буркнул Тхалас.
\ml{$0$}
{--- Чем могу быть полезен?}
{``So, how can I help?}
\ml{$0$}
{Надеюсь, вы не за косточками сюда пришли?}
{I believe it has nothing to do with peach pits.''}

Я протянул Тхаласу огрызок своей книги.
Глаза Тхаласа вылезли на лоб.
Помолчав, он вернул её мне.

--- Я так понимаю, ты знаешь, что это за книга.
Мне она очень нужна, --- сказал я.

Тхалас бросил взгляд на Митриса --- тот сделал библиотекарю какой-то знак.
Тхалас кивнул.

--- Дело в том, что за книгой уже приходили, --- сказал Тхалас.
--- Хитрам-лехэ.
Сказал, что ему очень нужно это сочинение, при этом название сказал с ошибкой, как будто только что услышал его от другого человека.
Он был... сам не свой.
Такое ощущение, что его или били, или сильно испугали.
Я сказал, что такой книги у нас нет.
Пока меня не было, кто-то рылся в книгах, и весьма непочтительно...
Это копия из Тхитрона?
Насколько я знаю, третий экземпляр хранился именно там.

--- Да, --- сказал я.
--- А где хранится ещё один?

--- Если с ним всё в порядке, то в Яуляле, в землях ноа, --- пожал плечами Тхалас и, нахмурившись, погладил самодельный корешок.
--- Почему книга в таком состоянии?

Я вкратце рассказал, как дневник попал ко мне.
Тхалас кивнул.

--- Пойдём.
Только давайте быстро и тихо.
Правда, не понимаю, чем она вам поможет --- никто так и не понял, что там написано.

Мы поднялись в его кабинет.
Тхалас, быстро оглянувшись, нажал на какой-то выступ в стене и с ловкостью иллюзиониста вытащил из ниоткуда книгу.

--- Спрятал на всякий случай, --- пояснил он и передал мне фолиант.
--- Вот.
<<Повесть о Существует-Хорошее-Небо, вожде сели, наезднике Железного Змея>>.

--- <<Командире Стального Дракона>>, --- машинально поправил я.

--- Что? --- хором спросили Король-жрец и Тхалас, услышав полупонятные, произнесённые с необычным акцентом слова.

--- Долгая история, --- ответил я.

Король-жрец понимающе кивнул, Тхалас потупился --- по-видимому, понял, что пришелец не так прост, как кажется.

Я пролистал книгу и вздохнул.
Увы, этот вариант тоже не был полным --- книга заканчивалась на середине записи номер сто сорок три.
Следов вырванных страниц не было --- видимо, они были потеряны ещё много дождей назад.

--- Я возьму её переписать? --- спросил я у библиотекаря.
Тот исподлобья бросил на меня взгляд.

--- Это очень ценная книга, --- веско ответил он.

--- Какой бы ценной она ни была для тебя, библиотекарь, вряд ли ты осознаёшь её истинную ценность, --- столь же веско ответил я.
--- Ручаюсь за неё жизнью.

--- Можешь приходить и переписывать сколько угодно, Ликхмас ар’Люм.
Из библиотеки книга выйдет только через мой труп, --- упрямо ответил жрец.

Я кивнул и аккуратно передал книгу честному библиотекарю.

--- Благодарю тебя, Тхалас.
Приду завтра с бумагой и пером.

\section{[-] Язык Древних}

Вскоре я уже читал следующую главу, одновременно делая перевод на Эй и отмечая совершенно непонятные технические термины, чтобы потом проконсультироваться у Грейсвольда.
Тхалас вначале сторонился меня, но неизменно пытался заглянуть в мои записи, проходя мимо.
Наконец, на третий день моего пребывания в библиотеке, душа жреца не выдержала --- он сел рядом и без обиняков попросил меня прочитать ему несколько страниц.

--- Здесь много вещей, которые ты не сможешь понять, --- предупредил я.

--- Если ты постараешься объяснить, я постараюсь понять, --- ответил библиотекарь.

Я усмехнулся и начал медленно читать вслух.
Комнату наполнили звуки древнего языка, многогранного, математически точного и невообразимо прекрасного по звучанию.
Они шелестели в тёмных углах, катались мягкими комочками пыли по книжным полкам, звенели фарфором чаши и стеклом чернильницы.
Тхалас слушал, открыв рот.
Наконец, прочитав несколько страниц, я спросил его:

--- Ты всё понял?
Ты не задаёшь вопросов.

Тхалас перевёл дух и улыбнулся, опустив голову.

--- Я не понял ни слова в описаниях машин и вряд ли когда-нибудь пойму.
Но чувства, которые испытывал Существует-Хорошее-Небо, не угасли со временем.
Книга пылает, --- жрец неожиданно закутался в плащ и поднялся, чтобы уйти, --- и это... это прекрасно.

Уже подойдя к двери, он обернулся:

--- Ты исполнил мою давнюю мечту, Ликхмас ар’Люм.
Я услышал своими ушами язык Древних.
Теперь я могу двигаться дальше.

\section{[@] Красный стафилококк (ЖС)}

\textspace

--- О, биологи явились, --- заметила Кошка.
--- Давно вас не было видно.
Поешьте с нами, что ли.
Чем вы там занимались?

--- Микрофлорой, --- хриплым басом ответил Цветущий-Мак-Под-Кустами.
--- Теперь местная живность для нас безопасна.
Почти.
Осталось только новые гены вставить и...

--- Тринадцать тысяч штаммов микроорганизмов и восемь тысяч видов многоклеточных, --- гордо пропищала Листик.
--- И да, мы измотаны.

Все находящиеся в комнате зааплодировали.
Трудяг тут же потащили к столу --- кормить.

--- Как же я хочу спать, --- пожаловалась Лист за кружкой чая.
--- Я уже ощущаю себя транспортной РНК, а в ландшафте мне мерещатся кластеры аминокислот...

Все дружно рассмеялись и начали наперебой советовать, как лучше отдохнуть.

Мак сидел тихо.
Я заметил, что он почти не притронулся к еде.

--- Тебя что-то беспокоит? --- тихо спросил я, наклонившись к нему.

--- Да так, ерунда, --- пробормотал он.

--- А всё-таки?

Мак помолчал и пожевал массивной челюстью.

--- Некоторые штаммы вызывают у меня беспокойство.
И ещё... Небо, я не хочу говорить при всех.

Мы встали и незаметно вышли за порог.
Дверь с мягким скрежетом закрылась, и весёлый гул товарищей остался где-то вдали.

--- Небо, всё нужно делать предельно быстро, --- без предисловий начал Мак, повернувшись ко мне.
--- У нас нет времени.
Инфраструктура чересчур хрупка и ненадёжна.
Адаптацию тси нужно провести в ближайшее время.

--- У тебя есть какие-то подозрения?

Мак помялся и потёр волосатой ручищей бритый затылок.

--- Даже обычные исследования дались нам с трудом.
Не хватало привычных мелочей --- оборудования, реактивов.
Кое-что собрали на коленке, кое-что заменили аналогами попроще, но всё же.
Дальше будет только сложнее.
Если мы в результате какой-то неожиданности потеряем часть инфраструктуры, тси будут подвержены смертельным болезням.
Чем больше мы медлим с необходимыми манипуляциями, тем выше риск.

Я кивнул.

--- Надолго хватит иммунитета?

--- Поколений на триста, не больше, --- угрюмо сказал Мак.
--- Дальше микроорганизмы с вероятностью восемьдесят процентов найдут способы обхода, и надежда только на уже имеющиеся у тси защитные механизмы и адаптивную изменчивость.

--- Мак, ты молодчина, --- сказал я и, обняв биолога, уткнулся ему в живот.
--- Я надеюсь, этого хватит.
Успокойся.
Вы с Листик проделали гигантскую работу.
Через триста поколений здесь уже будет подобие Тси-Ди.

Мак слабо улыбнулся, почесал огромной кистью мою голову и погладил мне брюшко.

--- Как себя чувствует молодёжь?

--- Молодёжь на подходе, --- улыбнулся я.

--- Я могу на тебя рассчитывать?

--- Я скажу врачам, чтобы они поторопились со всеобщей генотерапией, --- сказал я.
--- А ты иди спи.

--- Боюсь, что мне потребуется медикаментозная кома, --- грустно заметил Мак и направился к каютам.
--- Мой мозг на пределе.
Благодарю тебя, Небо.

\section{[-] Красный стафилококк (БВ)}

Я читал, а в моей голове вертелся образ книги, которую я брал из тхитронской библиотеки в далёкой юности.
Цветущий-Мак-Под-Кустами...

Я бросился в библиотеку.

--- Ликхмас, --- вздрогнул Тхалас, едва не опрокинув чернильницу.
--- Что у вас, демонов, за привычка врываться без стука?

--- Извини, --- сказал я.
--- Мне снова нужна твоя помощь.

--- Книга?

--- Да.
Цветущий-Мак-...

--- ...-Под-Кустами, --- закончил за меня библиотекарь.
--- <<Машина жизни>>, <<Солнце и вода>>?

--- <<Мир в капле>>.

--- А я надеялся, что нет.
Кихотр, только просушил и убрал.
Книга всё-таки должна немного полежать после, хай, ты понял...
Подожди, сейчас достану.

Библиотекарь вскоре вернулся с толстенным томом в руках.
Я открыл ароматную, пахнущую свежим пергаментом и чернилами книгу --- со страниц посыпалась тонкая силикагелевая пыль.

Да, это было то, о чём я думал.
Отчёт о микрофлоре Тра-Ренкхаля, наиболее консервативных антигенах, возможных путях мутации и способах противостояния.
Схемы из квадратиков --- последовательности аминокислот и нуклеотидов.

Единственной опасной инфекцией сели была болотная лихорадка.
Неужели Мак не знал о возбудителе?

Страница пятьдесят один развеяла мои сомнения.

<<Красный стафилококк, штамм 32.
Потенциальный патоген, атипичная изменчивость>>.

После описания четверть страницы занимал постскриптум, который из поколения в поколение добросовестно воспроизводили переписчики.
Длинный, полный злости и неприличных в обществе тси слов.
К сожалению, я не смогу их перевести --- смысл будет безнадёжно потерян.
Подставьте вместо них самые неприличные слова, которые употребляются в вашем обществе.

\begin{quote}
<<Как же я устал, (ругательство).
А самое, (ругательство), замечательное --- программа FADA-1101, которая могла бы это просчитать, осталась дома, (ругательство).
Да чтоб эту Машину (ругательство), и стафилококк этот (ругательство), и компьютер этот (ругательство), и умников, которые вовремя компьютер не перепрошили, сто раз (ругательство).
И да, Лист, Морковка и Гайка, мне абсолютно (ругательство), это мой (ругательство) отчёт и я пишу такую (ругательство), какую считаю нужным>>.
\end{quote}

--- Хай, --- вдруг оживился Тхалас, --- помню эту страницу.
Иногда переписчики дурачатся, оставляют друг другу записочки на полях, но чтобы Древние...
Учитель находил это забавным, а вот мне почему-то не смешно.
Я бы не стал так проклинать даже смертельного врага.
Интересно, кто этот Мелкий-Красный-Виноград?
И что он сделал?

Я улыбнулся. В моей груди росло восхищение великим человеком древности.
Как ни крути, а обещание защитить триста поколений Мак выполнил.
Сражение с микрофлорой чужой планеты учёные-тси выиграли с большим отрывом.

--- Ликхмас? --- напомнил о своём существовании Тхалас.

Я закрыл книгу и приготовился объяснять.

\chapter{[-] Король-жрец}

\section{[@] Дети Неба}

К сожалению, мы не захватили с Тси-Ди детских капсул.
Ни одной.
Пришлось по старинке заворачивать малышей в провощённые одеяла;
я натёр два одеяла выделениями восковых зеркалец, а затем немного пожевал их во рту.
Об этом давно ушедшем методе всем апидам рассказывали во время обучения, но у меня даже мысли не было, что когда-нибудь он мне пригодится.

Вскоре в каюту завалились друзья --- с поздравлениями.
Я только покормил малышей и ещё не успел завернуть их.

--- Небо, поздравляю! --- Заяц сияла, как солнышко.
--- Как личиночки?
Ути-пути!
Хорошие, толстенькие.
Первые апиды-тси, родившиеся здесь!..
Фонтанчик, ты чего?..

Канин стоял и смотрел на детей со странным выражением лица.

--- Фонтанчик?

--- Я это... первый раз вижу...

--- Серьёзно? --- удивился Шмель.
--- А разве у Неба...

--- Были! --- воскликнул Фонтанчик.
--- Но вы их кладёте в капсулу, и они лежат себе тихо в углу.
Небо только каждый час на кормление отлучался.
А что там в капсуле, я ни разу не видел!

--- Смотри, --- засмеялся я.
--- Можешь даже потрогать.
Только одним пальчиком и осторожно, хорошо?

Фонтанчик осторожно протянул огромную лапу к малышу и тут же отдёрнул.

--- Он похож на желе.

--- Косточек у них ещё нет, --- объяснил Шмель.
--- И мозга тоже почти нет.
Они только едят и ползают.

Фонтанчик выглядел так, словно ему стало дурно.

--- Ну как?
Нравятся? --- лукаво улыбнулась Заяц.

Канин насупился и бросил на меня смущённый взгляд.

--- Нет.

--- Зато честно, --- отметил я, пока другие хохотали.

--- Я серьёзно, это совсем не смешно, --- сказал Фонтанчик.
--- Щенята рождаются слепыми, но по ним сразу видно, что да, это твои дети!
А тут просто большие толстые червяки!

--- Поэтому мы их и прячем, дружочек, --- пошутил я.
--- Вот случится что-нибудь со мной --- ты их ещё и кормить будешь.
И гладить, чтобы кровь не застаивалась.
И следить, чтобы они окуклились как надо, а не как им захотелось.
А вот потом начнутся ути-пути.

--- Не, ну если нужно будет, то я конечно... --- смутился канин.
--- Только пусть кормит, гладит и окукливает Заяц, а я на себя ути-пути возьму!

--- Учти, детишки-апиды очень юркие, --- предупредила Заяц.
--- Меня как-то Комар попросил присмотреть за только вылетевшими ребятишками.
Я столько шишек набила, пока пыталась их поймать, а им хоть бы что!

--- Они даже по стенам и потолку носятся, если есть за что зацепиться, --- сказал я.
--- Всё-таки щенята себе такого не позволяют.

--- Апиды интереснее других детишек, они умненькие, язык схватывают на лету, --- продолжала делиться впечатлениями Заяц.

--- И в этот момент рядом обязательно должен быть взрослый, --- добавил Шмель.
--- Иначе дети изобретут собственный язык для общения, и отучить их от него очень сложно.
Первые сорок дней после вылета дети не спят.
Вообще.
Поэтому на период выкукливания мы образуем группы --- пока одни спят, другие гуляют, играют и болтают с этой оравой молодняка.

--- А правда, что вы после выкукливания не можете своих от чужих отличить? --- спросил Фонтанчик.

--- А зачем их отличать? --- удивился Шмель.
--- Все свои же, одна популяция.

--- А ещё у них есть привычка --- старт-прыжок, --- Заяц увлечённо схватила Фонтанчика под руку.
--- У предков апид были крылья, а потомкам от полётов остался только ювенильный рефлекс взлёта.
Выглядит очень забавно.

--- Уговорили, --- проворчал Фонтанчик.
--- Ради таких впечатлений я их ещё и окукливать согласен!

--- Соглашайся, --- ухмыльнулся Мак.
--- С апидами всё же проще, чем с дельфинятами.

--- А, слышал эту историю десятилетней давности, --- припомнил Фонтанчик.
--- На дельфинят забыли надеть маячки, и группа просто уплыла в океан.
Они погибли?

--- У дельфинов в океане Ди просто не осталось естественных врагов, --- покачал головой Мак.
--- Крупные акулы вымерли во время Тараканьей войны, и их, насколько я знаю, даже не стали завозить --- как-то обошлись.
Так что, скорее всего, дельфинята одичали и начали питаться рыбой --- много ума для этого не нужно.
Кстати, подобное не раз и не два уже случалось.
Вначале диких дельфинов пытались искать, возвращать в цивилизованное общество.
Потом махнули рукой.
Они живут себе в океане, никого не трогают.
У них своя культура, свои языки...

Я вдруг вспомнил, что в молодости смотрел художественный фильм про океанолога, полюбившего дикую дельфиниху.
Кажется, потом он уплыл с её стаей.
А может быть, и нет.
Я смотрел фильм с апидом, который очень меня любил.
Он гладил мои ноги и пытался поцеловать.
К середине фильма ему это удалось, и я так и не узнал, чем закончилась история.
Молодость --- чудесная пора.

Фильмы, старые сводки новостей, воспоминания о доме.
Кусочки прежней жизни, которой больше нет и не будет...

--- Кошмар какой --- один биологический вид говорит на десяти разных языках, --- задумался Фонтанчик.
--- Это как минимум нерационально.

--- Считается, что дивергенция языка характерна для технологически неразвитого общества и является предпосылкой для дивергенции сапиентного вида, --- ввернул Баночка.
--- Кажется, ещё в позапрошлом тысячелетии проводили наблюдения за дикими лисами.
Между коэффициентом различий в сигнальной системе и частотой скрещиваний есть отрицательная корреляция.

--- Данные неоднозначны, --- заметил Мак.
--- Я бы поспорил с корректностью постановки эксперимента.
Тогда не могли учесть фактор Паука, а точнее, его частный случай --- <<провал>>...

--- Так или иначе, эти данные впоследствии использовались при построении нейронных сетей, --- заупрямился Баночка.
--- Метод <<лисья любовь>> использовали даже ещё мы, несмотря на разработку более быстрого К-контроллера...

--- Разработчики <<лисьей любви>> обратили фактор Паука в фичу! --- засмеялась Заяц.
--- Нейронные сети на ней работали идеально, но пришлось написать кучу костылей, чтобы пристыковать их к более старым стандартным структурам!
Я ещё в школе один лично оптимизировала, думала, с ума сойду!
Но зато сколько гордости потом было --- кусок моего кода проверила комиссия, вставили в код Машины и...

Заяц запнулась.
В воздухе повисло неловкое молчание.

Я знал, о чём думали друзья.
<<Лисья любовь>>, несмотря на архаичность и плохую совместимость, не содержала в себе трёх тонких уязвимостей, характерных для К-контроллера.
В том, что смертельная ошибка пошла с него, не сомневался никто.
Но в одном я разошёлся с прочими инженерами: все говорили о случайности, а я интерпретировал некоторые обстоятельства как следы диверсии.
Комар по секрету сообщил, что несколько лет назад была пресечена попытка Картеля внести достаточно странные изменения в генофонд тси.
Задумка удалась бы, если бы не случайность.
Возможно ли, что настолько продуманный план был отвлекающим манёвром?

И что за демон мог настолько быстро разобраться в архитектонике?
Неужели это был изменённый тси, которого пропустили врачи?

<<Этого просто не может быть>>.

--- Лисы --- это хорошо, --- выручил всех Фонтанчик.
--- У меня одна мышковала рядом с домом, очень ласковая, я с ней общался.
У ног тёрлась, но гладить не позволяла.
Один раз стащила рубашку, которую сшила Заяц, разорвала её в клочья и вывалялась в них --- ей понравился запах...

--- Свежо предание, --- поморщилась Заяц, явно обрадовавшись перемене темы.
--- Скажи уже честно, что ту рубашку ты сам порвал, неряха!

--- Я же тебе показал обрывки, на них шерсть рыжая!

--- Порвал и лисой натёр!

Заяц кинулась другу на шею, и он вдруг по-щенячьи взвизгнул.

--- Ай!
Моё ухо!

--- Будешь знать, как врать, --- буркнула Заяц, выплюнув попавшие в рот шерстинки.

Канин демонстративно отвернулся от женщины и вдруг, не дожидаясь разрешения, погладил ребёнка.
Личинка вытянула крошечную голову и клацнула челюстями.
Фонтанчик заулыбался.

--- Они смешные.
А чем ты их кормишь, Небо?
Пчелиным молоком?

Я напряг челюсти и выплюнул на ладонь желтовато-зелёную каплю.
Фонтанчик наклонился, осторожно понюхал и лизнул.

--- Похоже на медовый крем, --- резюмировал он.
--- Слишком сладко, но для торта сгодится.

--- Ой, а дай мне тоже попробовать? --- попросила Заяц, просунув голову у друга под мышкой.

--- Эй, он всё-таки не вас выкармливать должен! --- засмеялся Шмель.
--- Хватит детей объедать!

\section{[-] Смерть Короля}

\epigraph
{Выбор собственной смерти --- высшая привилегия мыслящего существа.
Это похоже на то, как философ, закончив книгу, заверяет её своей собственной подписью.
Порой историки пытаются переписать книгу чьей-то жизни, но покажите мне того, у кого это на самом деле получилось!
Мёртвых нельзя прославить и опозорить более, чем они прославили и опозорили сами себя.
Вдвойне верно сие для тех, кто выбрал смерть самостоятельно.}
{Анатолиу Тиу}

\textspace

Король-жрец посмотрел по сторонам.
Пути для отступления не было.

--- Сам спрыгнешь или тебе помочь? --- участливо спросил Марас.

--- Я не смогу спрыгнуть сам, Марас, --- с достоинством отозвался Король-жрец.
Враги оскалились.

--- Как же ты жалок, --- рыкнул жрец.
--- От тси осталась только кучка бесполезных генов.

--- Я не знаю, что такое <<гены>>, --- смиренно признал Митрис.
--- Но предки оставили мне ещё кое-что.
Мудрость, заключённую в древних изречениях.
Я не могу оценить всю их глубину, но ведь вы, всеведущие пришельцы, знаете язык тси в совершенстве?

Агенты захохотали.
Митрис достал из кармана робы маленькую книгу и открыл её на последней странице.

--- <<И погаснут звёзды без мысли.
И бродячие земли остынут без желания.
И Вместилище рассыплется в прах, не осознав своего рождения.
Но просвещённый не познает такой судьбы --- летописцем чувств его будет пламя, летописцем желаний его станет вода, а последнюю мысль впишут в свиток Великого Ветра.
\emph{Одиннадцать, сорок девять, девяносто два}>>.

Глаза Короля-жреца остекленели.
В следующую секхар два врага осели, захлёбываясь кровью.
Третьим на горячий камень упал слабо улыбающийся Митрис.
Из руки Короля-жреца выпал обсидиановый скальпель и, звеня, укатился в редкую сухую траву.

--- Kuna\FM, --- констатировал Марас, убирая духовое ружьё.
\FA{
Мразь, выкидыш, ублюдок, мусор (сектум-лингва).
}
Его помощник вытащил из шеи мертвеца стрелку и столкнул безвольное тело со скалы.

\section{[-] Боль}

\textspace

Анкарьяль сидела на каменной скамье, обняв завёрнутого в пелёнку малыша Листика.
Её глаза смотрели вдаль, рот был приоткрыт --- лицо человека, перенёсшего психологическую травму.
Я видел такие десятки тысяч раз.

Мы с Грейсвольдом подошли к ней.
Толстяк крякнул и сел рядом, я опустился на корточки перед женщиной.
Она крепче прижала к себе ребёнка и отвела взгляд.

--- Как ты себя чувствуешь? --- спросил Грейс.

--- Я его не оставлю, --- бросила Анкарьяль.

--- Кто это решил?
Анкарьяль Кровавый Шторм или человеческая женщина по имени Хатлам ар’Мар?

--- Обе.

Грейсвольд многозначительно посмотрел на меня.

--- Хватит переглядываться за моей спиной, --- абсолютно ровным, напоённым смертью тоном произнесла Анкарьяль.

--- Нар, почему мы здесь? --- спросил Грейс.

--- Я не знаю, зачем вы ко мне пришли.

--- Ты помнишь о задании, которое...

--- Мне не нужно напоминать.

--- И?

--- Я его не оставлю.
И да, если чей-то демон хоть шевельнётся в мою сторону, то распадётся на составляющие прежде, чем успеет реализовать пару инструкций программы.

Тяжесть угрозы мы с Грейсвольдом осознали далеко не сразу.
Технолог спустя пару секунд инстинктивно отодвинулся от подруги и закутался в плащ --- он как никто знал, на что способна Анкарьяль Кровавый Шторм.

--- Аркадиу, пойдём поговорим...

--- И о чём вы собрались говорить? --- поинтересовалась Анкарьяль.

--- Нар...

--- Грейс. Пошёл. Вон.

Тон Анкарьяль говорил, что она на пределе.
Мы с Грейсвольдом поняли, что недооценили серьёзность ситуации.
Тяжело вздохнув, технолог поднялся и ушёл, оставив молча сходящую с ума женщину на меня.

Разумеется, я слышал о таких случаях.
Именно для этого обучали полевых врачей.
Глубокая интеграция демона и сапиентного мозга не могла не сказаться на обоих, и порой демоны начинали действовать нелогично из-за древних инстинктов.

Я сел рядом с Анкарьяль и обнял её, уткнувшись носом ей в шею.
Мы сидели очень долго.
Наконец она чуть повернула ко мне голову.

--- Аркадиу, ты урождённый человек.
Расскажи, что с этим делать.

Я знал, что сейчас я должен быть предельно честным.
Объединённый разум Хатлам-Анкарьяль распознает любую ложь, и последствия могут быть катастрофическими.

--- Я представляю, что ты чувствуешь.
Когда-то я потерял важного для меня человека.
Всё прошло --- разумеется, не бесследно.

--- Почему именно я? --- этот совершенно человеческий вопрос принадлежал Хатлам.

--- Мы не выбираем то, что получаем при рождении, Вишенка, --- сказал я.
--- Я знаю, что демон заставляет тебя чувствовать странные, порой очень неприятные вещи.
Но подумай, как много всего он тебе открыл.
Дали, о которых ты даже не подозревала.

Анкарьяль слабо улыбнулась.

--- Я не уверена, что хотела бы знать об этих далях.
Моё существование...
Какое же оно жалкое.

Это была самая опасная тенденция --- вся память демона пропускалась через призму эмоций сапиента.
\emph{Жалкое}.
Так характеризовали наше существование люди, кани, планты и прочие --- именно этим древним словом.

--- Сколько тел я сменила?
Сколько жизней прожила?
Служба, работа, исследования.
И мои тела умирали, так и не узнав, что такое счастье.

--- А что, по-твоему, счастье, Нар?

Анкарьяль посмотрела на меня и улыбнулась:

--- Мир.

Ребёнок на руках Анкарьяль зашевелился, сонно загулил и зачмокал.
Анкарьяль нежно, на грани слуха зашептала и легонько покачала малыша, пока он не заснул покрепче.

--- Мне так больно, Аркадиу.

--- Ты только что потеряла любимого человека, Нар.
Скажи слово --- и я это остановлю.
Существуют мощные, сверхселективные корректоры настроения...

--- По-твоему, это решение? --- пробормотала Анкарьяль.
--- А что будет, если я потеряю тебя?
Грейса?
Кто тогда облегчит мою боль?
Что тогда от меня останется?..

Анкарьяль захлебнулась.
Я промолчал и покрепче обнял подругу.

--- Ты когда-то терял друзей.
Скажи, что ты сейчас испытываешь к ним?

--- Они живут во мне.
И для большинства это единственная жизнь, которая у них есть сейчас.
Не думаю, что они были бы против такой.

--- Он очень умный, --- сказала Анкарьяль.
--- Как ты, когда я тебя впервые встретила, только он ещё и красивый.

Я улыбнулся.
Анкарьяль поняла, что сморозила нечто обидное, и тоже виновато улыбнулась.

--- Я бы так хотела, чтобы он пошёл со мной.
Представь, как было бы здорово, если бы Башенка был с нами.
Жизнь за жизнью.
Каким бы демоном он мог стать со своим умом...

--- А ещё он всегда был бы с тобой.

--- Да, --- выдохнула Анкарьяль.
--- Да.
Рядом было бы существо, которое всегда со мной.
Которое никогда не предаст и не ударит в спину.
И вот что от него осталось, --- Анкарьяль со слезами на глазах поправила пелёнку Листика.
--- Теперь понимаешь?

--- Это не он, Нар.
Митрис умер.
И этот ребёнок тоже умрёт.
Но у него будут свои дети.
А у тех детей будут ещё дети.
Это жизнь Ветвей Земли --- одни передают огонь другим, угасая сами.
Слепая эволюция не даровала им бессмертия в их понимании.

Анкарьяль молчала.

--- А теперь возьми себя в руки и подумай, что лучше для тех, кто ещё жив.
И помни --- я всегда с тобой, какое бы решение ты ни приняла.
И Грейс тоже.

Я поднялся, нежно поцеловал женщину в солёные от слёз губы и пошёл в дом.

\section{[-] Тяжёлое решение}

Солнце уже клонилось к закату.
Мы с Грейсвольдом сидели за столом, глотая приготовленную технологом перцовую похлёбку.
Разговор не клеился.

Анкарьяль зашла уже под конец трапезы.
На её опухшем лице ещё не просохли слёзы, но глаза уже горели прежним, стальным блеском.

--- Аркадиу, я оставлю ребёнка в семье.
Но только до конца войны.
Если мы выиграем --- я вернусь и заберу его.

Я кивнул.

--- Ты не будешь изменять мою память, восприятие и эмоциональный фон.
То, что есть, я переживу сама, без постороннего вмешательства.

Я кивнул ещё раз.
Анкарьяль повернулась к Грейсвольду:

--- А ты бессердечная рыба.

--- А я тебя всё равно люблю, --- просто ответил Грейсвольд.
Анкарьяль несколько мгновений смотрела в спокойные глаза технолога, излучавшие глубокий, древний свет бесчисленных жизней.
Потом слабо улыбнулась и кивнула ему:

--- Я тебя тоже, старый друг.

\section{[-] Дом для ребёнка}

--- Ты, должно быть, Хатлам ар’Мар?
Заходи.

Дверь нам открыла пожилая, но ещё красивая купчиха.
В волосах её уже протянулись тонкие серебряные нити, кожа на шее слегка потеряла эластичность --- на вид ей было дождей сто --- сто двадцать.

--- Заходи, заходи, и не обращай внимания на бардак, --- сказала хозяйка.
--- Ты вовремя, я послезавтра собиралась в Тхитрон вместе с семейством.
Всегда недолюбливала ненужное барахло, а в результате накопила его столько, что пройти негде.

Анкарьяль вошла и оглядела комнату, заваленную походными мешками и ящиками.
Я остался у порога.

--- А ты чего стесняешься?
Ну-ка заходи, --- без обиняков бросила мне купчиха.
--- В моём доме гости у дверей не жмутся.
Вы, наверное, голодны?
Дорога с Тхартхаахитра неспокойная.

--- Да, я бы не отказалась поесть, --- согласилась Анкарьяль.

--- Сатракх! --- крикнула купчиха.
--- Поесть принеси гостям в зал!
Садитесь, --- добавила она нам, указывая на ящики у очага.
--- Мой мужчина сейчас принесёт мясо и похлёбку.

Анкарьяль кивнула и села на ящик.
Листик на её руках зашевелился и загулил.

--- Ты её мужчина? --- тихо спросила у меня купчиха.

--- Нет, друг.

--- Хаяй.
Ещё одна женщина, потерявшая мужчину.
Что за время такое звериное?

--- Как ты узнала? --- удивился я.

--- Да у неё на лице всё нарисовано, я ж не первый день живу.
Просто для уверенности спросила.
Садись, садись, ящики не золотые.
Здесь статистика по распределению товаров за нулевые годы Церемонии.
Я бы выкинула к свиньям это топливо, только Первый жрец устроил истерику --- вдруг понадобится...
Сатракх, ты там умер?
Надеюсь, что да, потому что если ты не принесёшь еду сейчас же, я исправлю ситуацию предельно жестоко!

Вскоре подоспел и Сатракх --- коренастый мужчина дождей на двадцать младше хозяйки дома, едва ли не ниже меня.
Холодные глаза его смотрели спокойно, губы под разбитым носом кривились в жутковатой, но гостеприимной улыбке --- рваную рану на лице зашивали второпях.
В покрытых шрамами загорелых руках дымились две плошки с чем-то очень ароматным, кои он и поставил на столик возле очага.

--- Кушайте.
Сатракх готовит так, что вам на всей Короне не сыскать лучшей похлёбки с мясом.
Книга-кормилица говорит одно, а у Сатракха своё мнение, и едва ли не лучше.
Ему бы стать хозяином постоялого двора, но он всю жизнь мочил сапоги в походах и радовал своими блюдами воинов.
Да где же Манис-лехэ?
Время уже!
А, вот он, вижу, --- хозяйка, увидев в окне сгорбленную фигуру, бросилась к дверям.

Манис-лехэ, одетый в протёртую синюю робу, поклонился всем присутствующим.
Это был совсем древний старичок, дождей ста восьмидесяти.
Ходил он с трудом --- на колене стоял примитивный фиксирующий аппарат, не позволяющий сгибать ногу.
Хозяйка, бормоча под нос что-то своё, выудила для него удобный табурет и, не обращая внимания на слабые протесты старика, укутала больную ногу шерстяным одеялом.
Жрец дрожащими пальцами достал бумагу и перо.

--- Имена дарителей? --- обратился он к Анкарьяль.

--- Хатлам ар’Мар э’Тхартхаахитр, М... Митрис ар’Люм э’Ихслантхар.

--- Король-жрец? --- ошеломлённо шепнула купчиха.
--- Который недавно...

--- Да, --- ответил я.
--- Не задавай вопросов.

--- Поняла, --- купчиха схватила меня за руку, потом обратилась к старику Манису:
--- Манис-лехэ, это ребёнок убитого Короля-жреца.
Что будем делать?

Старик задумался.

--- Ээээ...
Вот как.
Хай.
Я запишу вас под другими именами, а настоящие имена помещу в мою тайную книгу.
О ней знает только мой ученик, Сорас ар’Хэ.
Если я умру до того, как...

Мы с Анкарьяль кивнули.
Я уже собирался просить о чём-то подобном, но эти люди поняли меня без слов --- ещё одна таинственная способность потомков тси.

Когда всё было записано, купчиха жестом позвала Анкарьяль следовать за ней.
Я остался сидеть у очага.

С улицы прибежали двое детей-погодок.
Сатракх оскалил кривой рот и со звериной ловкостью вспрыгнул на четвереньки:

--- Арр, дети.
Ягуар приготовил вам кушать.

Дети радостно закричали и побежали вслед за кормильцем на кухню.
Из дальних комнат пришла купчиха и, подобрав полы платья и погладив по плечу уплетавшего похлёбку старого жреца, села рядом со мной.

--- Пусть побудет немного с дитём.

--- Ты, я слышал, многих приютила? --- поинтересовался я.

--- Видел детей?

--- Да, --- кивнул я.

--- Ни одного из них не выносила, --- рассмеялась купчиха.
--- Но все мои.
Уже четырнадцатый выводок будет.
Я и с Сатракхом так познакомилась.
Приходит ко мне ночью это замызганное чудовище в шрамах с дитём.
А мне самой недавно воительница оставила Улыбашку.
Я ребёнка беру, делаю записи, всё по правилам.
А он ходит и ходит, да никак не отстанет.
Выставлю за дверь --- акх! --- он в окно пролез.
Так и остался ночевать, рыбина обкусанная.
Видать, сам Хри-соблазнитель нас свёл.

Я кивнул.

--- А насчёт подруги не беспокойся, --- вдруг добавила она.
--- Это она сейчас в помутнении, чушь всякую городит.
Улыбашка у меня не первый, до него ещё пятерых воительницы приносили.
Все клялись, что вернутся и заберут.

--- И?

--- Поверишь, ни одна даже не навестила, --- рассмеялась купчиха.
--- Себя не хвалят, но быть кормильцем --- это талант.

\section{[-] Птичье гнездо}

Той ночью мы легли <<птичьим гнездом>> вокруг Анкарьяль.
Сели часто использовали этот приём, чтобы облегчить чью-то душевную боль.

Мы с Грейсом имели достаточно общие представления о <<гнезде>>.
Чханэ проявила куда больше сноровки.
Посмотрев на наши телодвижения, она сделала несколько уничижительных замечаний сексуального характера и взялась за дело сама.
Анкарьяль, следя со слабым интересом за её действиями, послушно выполняла все распоряжения.

Первым делом Чханэ подложила под голову Анкарьяль свёрнутое одеяло, а не обычную подушку.
<<Чтобы дышать было легче>>, --- объяснила девушка.
Затем она жестом подозвала Грейса и уложила его справа, чуть выше, чтобы голова Анкарьяль устроилась у него на груди.
Сама Чханэ легла слева, чуть ниже, и уткнулась носом в шею Анкарьяль.
Последним, слева от Чханэ, лёг я --- так, чтобы Анкарьяль могла видеть моё лицо.

--- Запоминайте: самый сильный и надёжный закрывает сзади, самый маленький и ласковый ложится спереди и кладёт голову на грудь... --- объясняла Чханэ.

--- Это ты тут у нас самая маленькая? --- улыбнулся Грейсвольд.

--- По возрасту --- да! --- нашлась Чханэ.
--- Не перебивай.
Самый близкий человек ложится так, чтобы было видно его лицо, и беседует с ней.

--- Да о чём нам с Каром говорить, --- пробормотала Анкарьяль.

--- Давай об облачках, --- предложил я.

--- Она не хочет, Лис, поэтому заткнись, --- посоветовала Чханэ.
--- Просто слушайте дыхание друг друга.
Оно должно синхронизироваться.
Ничего страшного, если кто-то начнёт плакать --- в <<гнезде>> печали делятся на четверых.

<<Птичье гнездо>> всем понравилось.
Чханэ уложила нас так виртуозно, что мы были и достаточно близко к Анкарьяль, и в то же время наши объятия не были слишком навязчивыми.

Не прошло и нескольких михнет, как все заснули мёртвым сном.

\section{[@] Мальструктура}

\epigraph
{Нет ничего страшного в ошибке, невнимании или промедлении.
Но когда один ошибся, второй недосмотрел, а третий бездействует --- случаются большие несчастья.}
{Пословица сели}

На Стальном Драконе произошла авария.
Причина банальная --- мальструктура отводящей пластины реактора\FM.
\FA{
Термин переведён буквально, технический смысл неясен.
}
Её ширина --- девяносто пять нанометров.
Баночка как в воду глядел.

Мы потеряли кольцевую теплицу.
Тор-отсек просветило, и Хомяк-И-Четыре-Орешка велел наглухо закрыть его.
От спасательной операции тси отказались.
Мак, взбешённый оскорбительными замечаниями техников, отправился вместе с Листик за биологическим оборудованием.
Вдвоём они размонтировали и вытащили большую часть.
Через полчаса работы у Мака пошла носом кровь, но он сделал попытку проникнуть ещё и в тор-отсек.
Листик и ещё несколько тси силой утащили его из сектора четыре и положили в лазарет с лучевой болезнью и в состоянии крайнего психического истощения.
Костёр сказал --- жить будет.
После техники зашли к Маку и попросили прощения за сказанное.

Специалисты приняли решение вывести Стальной Дракон в космос.
Я пытался их убедить, что не всё потеряно, что Безымянный может залатать пробоину.
Но многие этому решительно воспротивились, и я, к своему стыду, дал волю злости и покинул собрание.

Впрочем, уже через час я пожалел. что ушёл.
Почти все колебавшиеся тси, коих была почти четверть, выступили за вывод корабля.
Мои сторонникам не хватило двенадцати процентов голосов.

Когда Фонтанчик рассказал об этом, меня впервые в жизни посетила мысль о самоубийстве.
Совершенно спокойная и оттого ещё более страшная.

Как же тяжело быть командиром.

\emph{Существует-Хорошее-Небо,}

\emph{уставший инженер компьютерных систем Тси-Ди.}

\section{[@] Умолчание}

\textspace

--- Почему ты не сказал про неё? --- заорал Фонтанчик.

--- Я сам узнал только позавчера, когда проводил плановый осмотр перед пуском, --- оправдывался Хомяк, опасливо пятясь.
--- Это есть в отчёте, можете посмотреть.
Но энергия нужна срочно, да и корабль перенёс полёт, и потому...

--- Слюни можешь вытереть своим отчётом! --- Фонтанчик, похоже, едва сдерживался, чтобы не ударить техника.
--- Это нестандартная нагрузка, тупая ты голова!
Вектор...

--- Хватит, --- остановила друга Заяц.
--- Что сделано, то уже не повернёшь вспять.
Хомяк, позови всех, будем решать проблему.

--- Я предлагаю отстранить его от работы! --- рявкнул канин.

--- И что это даст? --- развёл я руками.
--- Как ни крути, Хомяк лучше всех осведомлён о происходящем.
Это и наша ошибка тоже --- мы взвалили чересчур много обязанностей на нескольких техников.
Почему отчёт Хомяка прошёл мимо вас?
Вы не знали о пуске?
Почему, в конце концов, техники комплекса не проанализировали ситуацию самостоятельно, а решили положиться на техников систем Дракона?

Фонтанчик яростно зыркнул, плюнул и ушёл.

\section{[@] Отчёт Хомяка}

Техникам удалось медленно снизить мощность реактора и на время остановить распространение повреждения.
Взрыв реактора --- вещь непредсказуемая.
Остатки топлива могли выжечь наше поселение, а могли и спровоцировать тектонический сдвиг, грозящий катаклизмами всей планете.
На Тси-Ди однажды взорвался реактор --- записи об этом событии с подробным разбором ошибок нам давали чуть ли не в детстве...

Я вдруг вспомнил, что в школе нам рассказывали и про один из самых известных случаев применения <<Живой стали>> --- техника Слово-Рубинового-Лазера.
Тот техник, поняв, что взрыв неизбежен и он остался заблокированным в помещении, применил <<Живую сталь>> и даже после денатурации мозга ещё пять секунд отдавал команды системе, пытаясь локализовать поражение.
Это спасло, согласно заключениям нескольких комиссий, не менее двух миллионов жизней и впоследствии стало главным аргументом в пользу сохранения механизма <<Живой стали>>.
\ml{$0$}
{Технику установили памятник с эпитафией, ставшей крылатым выражением: <<Растворившись, живу>>.}
{A memorial in honor of him had been made, there's an epitaph that's become a catchphrase: `Dissolved, I live'.}

А ещё учитель сказал, что около тысячи лет после взрыва многие техники замечали присутствие в Сети ещё одного, не ассоциированного ни с каким сапиентом виртуала.
Он значился в логах как Слово-Рубинового-Лазера, выполнял действия обычного техника, время от времени писал отчёты, но источник сигнала установить не удавалось, равно как и отключить неведомую подпрограмму.
Но это, разумеется, или чья-то неумная шутка, или городская легенда, очередная история для походов и пижамных вечеринок.
Я не говорю, что это невозможно технически;
но едва ли тси с такими высокими моральными качествами стал бы тратить бесценное время на интеграцию себя любимого в программную оболочку, принося невинных в жертву слепой стихии.
Я не представляю, что чувствует плант, от которого остаётся только сверхпроводниковая тень, но какой бы ужас не таился в такой форме существования, техник выдержал пять секунд.

Я помотал головой.
Зачем я вообще про него вспомнил?..

Вскоре все специалисты собрались у берега моря.
Хомяк бодро перечислил известные на текущий момент данные.

--- ... Двадцать четыре процента.
Сектор четыре просветило...

--- Ну что, сволочь? --- рыкнул Фонтанчик.
--- Кольцевую теплицу насмерть засветил и радуешься?

--- Да, не <<теплицу просветило>>, а всего лишь <<сектор четыре>>, --- поддержала Заяц.
--- Научись уже нести ответственность за свои ошибки, хотя бы на словах.

--- Тихо, --- призвал я к порядку.
--- Кто-нибудь поискал жизнеспособные клетки в тор-отсеке?

--- Нет, --- ответил за Хомяка Фонтанчик.
--- Этот тупица велел наглухо замуровать тор-отсек.
Я ведь прав?

--- А ты бы сам туда полез? --- тяжёлым тоном осведомился Мак.

--- Согласно протоколу, если радиация в секторе превышает... --- уже менее бодро начал Хомяк.

--- Ясное дело.
Трусливый раствор аминокислот.
Лучше бы ты там оказался, от тебя жареного больше пользы.

--- Хрустально-Чистый-Фонтан, держи себя в руках, --- устало сказала Листик.
--- Проблем хватает.
Хомяк, протоколы протоколами, но теплица у нас была одна, и перед закрытием тор-отсека следовало поискать добровольцев.
Продолжай, что там ещё.

Меня на мгновение потряс спокойный тон Листик, когда она говорила о смерти существа, которому была ближе всех тси.
Но тут я заметил сильную замедленность моргания и понял --- женщина-плант приняла приличную дозу селективного нормотимика.

Вскоре отчёт закончился, и группа погрузилась в молчание.
Всем было дано время на обдумывание проблемы.

\section{[@] Дебаты}

Дебаты, как и ожидалось, с самого начала пошли жарче некуда.

--- Придётся вывести корабль, для нас он уже бесполезен...

--- Что значит <<бесполезен>>?
Нужно спасти оборудование, оставшееся в четвёртом секторе, --- сказала Листик.
--- Анализатор белков, система для постройки ферментов, реактивы для...

--- Ну вот сама и иди за ними! --- ощерилась Пирожок.
--- Какая от этой узкоспециализированной техники польза?
Сейчас вы --- придаток лазарета.
Вам медицинское оборудование вынесли?
Вынесли.
Вот и делайте свою работу!

Листик безучастно, не моргая, смотрела на молодую кани.
Нормотимик гасил <<отвлекающие>> эмоции, и биологу просто нечего было ответить на несправедливые слова.

--- Мы тебя, гнилые руки, от всей заразы Трёх Материков привили! --- вспылил Мак.
--- Пошли, Листик.
Сами вытащим и дальше будем исследовать, а этой гнили пусть её слова поперёк горла встанут.

--- Утрудился, герой ты наш, посмотрите на него, --- бросил кто-то с краю.
--- Всё кресло просидел.
Иди и теплицу заодно вытащи, раз такой смелый.

Мак побагровел, и я понял: если его не остановить, он действительно сейчас кинется в радиоактивный склеп.

Я встал и попросил тишины.

--- Я прекрасно понимаю, что все устали.
Давайте не будем забывать --- все тси без исключения работают на общее благо.
Мак, выслушай меня и можешь идти хоть на северный полюс.

Мак сел и хмуро скрестил руки на груди.

--- У нас есть один вариант.
Надо попросить Безымянного, --- сказал я.
--- Он один сможет подобраться к эпицентру, произвести точную диагностику и залатать пробоину.
В крайнем случае он достанет клетки...

Договорить мне не дали.

--- Доверять клетки теплицы хоргету?
Вы все в одночасье с ума посходили? --- снова разъярился Мак.
--- Да я лучше сам в тор-отсек полезу!

--- А если он напортачит, Небо? --- сказал Фонтанчик.
--- Поселение сгорит, как бумажка!

--- Кроме того, можем ли мы ему доверять реактор? --- подал голос Хомяк.
Остальные закивали, выражая согласие.

--- Он уже пытается нами манипулировать, --- мрачно сказал Баночка.
--- Эта странная музыка...

Кошка молчала, обхватив руками голову.
Я напрасно смотрел на неё, ожидая поддержки.

--- Нужны ли мы ему? --- добавил кто-то.
--- Все узлы планетарной защиты, кроме последнего, готовы!
Воспроизвести технологию не составит...

--- Даже если он честен с нами, в чём я лично уверен, Небо, --- веско сказал Фонтанчик, --- он ничего не знает о технологии, с которой будет иметь дело.
То есть декодировать то, что он там увидит, придётся нам, в реальном времени.
Мы, в свою очередь, ничего не знаем о восприятии Безымянного, все адаптационные обновления он осваивал сам, без нашего участия.

--- Кроме того, для него это существенные траты, --- добавила Заяц.

--- Мы возместим убытки позже, --- сказал я.
--- Главное --- действовать быстро.

Окружающие возмущенно зароптали.

---  <<Возместим>>? --- развела руками Пирожок.
--- Ты сказал <<возместим>>?
Может, мы ему сразу поклоняться начнём, как те осьминоги?

--- Я предлагаю вам единственный шанс сохранить корабль! --- потерял я терпение.
--- У меня ощущение, что вы уже похоронили Дракона!
Как сделать золото без корабля?
Где вы будете производить ядерный синтез --- на примусе?
Или глиняную печь соберёте?

--- Небо, что ещё за <<вы>>? --- удивилась Заяц.
--- Ты один из нас, забыл?

--- Похоже, что да, --- тихо констатировал Мак.

--- В бездну корабль! --- завопил Хомяк.
--- Это всего лишь техника!

--- И систему тоже в бездну, ты это хотел сказать? --- крикнул я.
--- Это тоже <<всего лишь техника>>?
Я хочу напомнить, что эта техника в том числе делает отличие между нами и племенами по ту сторону пролива!
Кому-нибудь надо напомнить, как они живут?

--- Они хотя бы живут!
И сейчас стоит вопрос о нашем выживании!

--- Да!
И когда он стоял ранее, Безымянный позволил тси жить!

--- Разумеется! --- буркнул Мак.
--- Ему нужны наши технологии!

--- В любом случае мы должны сохранить корабль, --- начал я чуть спокойнее.
--- Это <<спичечная>> технология, поколение Ночи, он может работать даже на сухой траве.
Мы могли бы удалить реактор...

--- И куда мы его денем? --- закричал Хомяк.
--- На планете реактор утилизировать нельзя.
Даже если вы выбросите его в глубоком космосе, без двигателя корабль останется там же!

--- Я бы послушала мнение хорошего врача, насколько фонящий реактор опасен для здоровья, --- саркастически фыркнула Пирожок.
--- Где Костёр?

--- В кругосветном, --- откликнулся кто-то.

--- Блестяще, --- зло затявкала Пирожок.
--- А как насчёт меня, Небо?
Мне отпуск можно?
Прямо сейчас, от вас, шайки идиотов?

--- Пирожок, --- вдруг взметнулась Кошка, --- я думаю, Костёр получил отпуск потому, что он не использует дружбу для достижения своих низких целей.

Кошка едва успела наклониться --- Пирожок явно вознамерилась пятернёй снести женщине голову.
На физика тут же навалились трое тси.

--- Грязная обезьяна, червяга сопливая, --- яростно булькала и брызгала слюной Пирожок, небезуспешно пытаясь вырваться из мощных объятий Фонтанчика.

Баночка с ненавистью смотрел на Пирожок и тискал свой меч.
Его глаза --- я даже не подозревал, что такое возможно --- из тёмно-зелёных превратились в вишнёво-красные.
Я вдруг осознал, что никто из присутствующих не задумывается о простой вещи --- если маленький плант сейчас выйдет из себя, берег превратится в заваленную трупами арену.
Никто из безоружных тси просто не сможет его остановить.
Кошка, похоже, тоже это поняла и испуганно бросилась ему на шею.

--- Тихо, тихо, дружочек... --- зашептала она.
--- Милый мой, успокойся...
Она просто не в себе...

Перепуганная Кошка плакала и закрывала Баночке глаза и уши ладошками, словно это могло оградить его от потока сквернословия.

Все были чересчур заняты рычащей, ругающейся и дерущейся Пирожок, чтобы обращать внимание на эту тихую борьбу.

\section{[@] Победа усталости}

К счастью, нежность Кошки и сила Фонтанчика победили.
Пирожок тихо извинялась перед братом, которому она едва не сломала руку, Баночка успокаивал Кошку --- женщину немилосердно трясло от пережитого испуга.
Тси задумчиво притихли.

--- Что ж, --- заговорила Заяц, пытаясь сгладить неловкую тишину.
--- Проблема освещена достаточно хорошо.
Предлагаю голосование.

--- Его результаты и так ясны, --- пробормотал я.

--- Возвращайся к нам, Небо.
Ты уже один раз принял решение в одиночку, пока все лежали в анабиозе, --- заметил Мак.
--- По-моему, это вскружило тебе голову.

--- И оставило вас в живых, --- заметил я и направился к лагерю.

--- Опять <<вас>>, --- буркнул Мак.

--- Небо, --- дрожащим голосом позвала меня Кошка.
Я не откликнулся.

--- Друг хоргетов, --- прошипел кто-то в тишине.

--- Так слово <<друг>> --- это теперь оскорбление? --- горько спросил я.
--- Мне нечего здесь делать, морально цивилизация уже мертва.
Известите, когда я должен буду своими руками закопать её последнюю надежду.

--- Хочешь снова сыграть в героя? --- фыркнула Пирожок.

--- Если кто-то ещё горит желанием вывести плавящийся космолёт на орбиту, я с радостью уступлю.

Ответом было молчание.

\section{[*] Схватки}

\textspace

\ml{$0$}
{--- Мы не сможем проникнуть в Тхитрон незамеченными, --- сказала Эрхэ.}
{``We can't get into the city unnoticed,'' \Oerchoe\ said.}
--- Стража у врат узнает Митхэ.
И меня тоже, и Акхсара.

\ml{$0$}
{--- Так, на что ты намекаешь, я не понял, --- буркнул Ситрис.}
{``So what are you implying, I didn't get it,'' \Sitris\ grumbled.}
\ml{$0$}
{--- Не пойду я через ворота, есть и более приятные способы самоубийства.}
{``I will not enter through the gate, there are nicer ways to commit suicide.}
\ml{$0$}
{Я чересчур заметный!}
{I'm too recognizable!''}

--- Я уже тебе сказала, что мы...

\ml{$0$}
{--- Я вообще не понимаю, к чему такой риск.}
{``I can't understand why take such a risk.}
\ml{$0$}
{Роды можно принять прямо на обочине, а ребёнка передать в город с людьми.}
{We can deliver a baby on the road shoulder, then somebody can deliver it to the city.''}

\ml{$0$}
{--- Ты принимал роды?}
{``Can you deliver a baby?''}

\ml{$0$}
{--- Слушай, мне уже давно не двадцать пять дождей!}
{``Come on, I'm over twenty-five rains old!''}

\ml{$0$}
{--- А что ты будешь делать, если она закровит или подхватит болотную лихорадку?}
{``And what will you do if she starts bleeding, or catch swamp fever?}
Митхэ уже не слишком молода, а болота от Тхитрона в тринадцати кхене!

Ситрис угрюмо промолчал.

\ml{$0$}
{--- Ладно, --- буркнул он наконец.}
{``You win,'' he finally mumbled.}
--- В таком случае лучший способ избежать излишнего внимания --- суета.
Я предлагаю разыграть схватки и ехать через ворота.
Пропустят быстро и не посмотрят.

\ml{$0$}
{--- Отлично.}
{``Perfect.}
\ml{$0$}
{Вот ты её и повезёшь, --- заключила Эрхэ.}
{You will carry her,'' \Oerchoe\ ended an argument.}
--- Митхэ, сможешь сыграть роженицу?

--- Никаких игр, --- отрезала Митхэ.
--- Хватит их с меня.

--- Но...

--- Дождёмся схваток и поедем в родах.
Шейка уже созревает, я только утром проверила.

--- Опасно ехать в родах верхом!

--- Это куда менее опасно, чем то, что нас могут узнать.

--- Хорошо.
Ситрис, держи своего оленя осёдланным на всякий случай.

--- Кажется, я уже сказал, что не собираюсь везти верхом роженицу через городскую стражу.

Наступило молчание.

--- Снежок, --- начала Митхэ, --- он прав, в суете нас сложнее...

--- Подожди, --- взмахом руки остановил её Акхсар.
Его лицо окостенело от гнева.
\ml{$0$}
{--- Это вопрос принципа.}
{``This is a matter of principle.}
\ml{$0$}
{Ситрис, ты понимаешь, что подвергаешь товарищей опасности?}
{\Sitris, do you realize that your comrades risk because of you?}
\ml{$0$}
{Эрхэ, Митхэ, меня, наконец?}
{\Oerchoe, \Mitchoe, me, at last?''}

\ml{$0$}
{--- Я подвергаю ненужной смертельной опасности \emph{себя}.}
{``It's \emph{my} risk, deadly and unnecessary.}
\ml{$0$}
{И это всё, что мне нужно понимать.}
{That's all I need to realize.}
\ml{$0$}
{Одно дело --- встретиться с храмовниками на дороге, и другое --- в городе.}
{It's one thing to face templers on the road, and quite another to do it in the city.}
\ml{$0$}
{В своём городе храмовник умеет проходить сквозь стены.}
{In their city templer can walk through walls.''}

\ml{$0$}
{--- Ты сдохнешь трусом, с клинком в заднице.}
{``You will rot like a coward, with blade in your arse.''}

Наступила ещё более глубокая тишина.
Ситрис повернул к нему голову.

\ml{$0$}
{--- И?}
{``So what?''}

Акхсар сжимал и расжимал кулаки.

\ml{$0$}
{--- Ну сдохну, ну трусом, ну клинок где-то там, --- Ситрис встал и подошёл к кипящему от ярости воину.}
{``Well, I will rot, well, like a coward, well, with blade somewhere in me,'' \Sitris\ stood up, then came close to the warrior which was full of fury.}
\ml{$0$}
{--- Что ты этим хочешь сказать?}
{``What do you mean by that?}
\ml{$0$}
{Что твоя овеянная славой жизнь и славная гибель чем-то лучше?}
{Maybe your glorious life and honorable death are any better than mine?''}

Эрхэ попыталась встать между ними, но мужчины мягко оттолкнули её.

\ml{$0$}
{--- Подожди, Эрхэ, --- сказал Ситрис.}
{``Wait, \Oerchoe,'' \Sitris\ said.}
\ml{$0$}
{--- Это, как выразился наш герой, <<вопрос принципа>>.}
{``This is, in the words of our hero, `a matter of principle'.}
Какого принципа только, он и сам не понимает.
Ему с детства вдолбили в голову, что он должен следовать принципам, чтить клятву.
Заставили эти принципы заучить и жить по ним.
И вот перед нами лоб восьмидесяти дождей --- тело, принадлежащее отряду, мысли, принадлежащие женщине, заплесневелые умения, принципы и чувства, рождённые в давно сгнивших черепах.
\ml{$0$}
{Ничего своего.}
{Nothing of his own.}
\ml{$0$}
{Ты не мужчина, ты склад чужого имущества.}
{You're a warehouse for other people's property, not a man.''}

--- Ты всё сказал? --- задыхаясь от гнева, пропыхтел Акхсар.
--- Всё, что мог, дырявый горшок?
В тебя налили кровь и выставили на улицу, вот ты и бродишь, запихивая в себя помои, конопляную жижу, мужское семя и женскую влагу, пытаясь обрести своё простое глиняное счастье --- счастье помойного горшка быть полным!

--- Ну, я тебя выслушал, предположим, --- поморщился разбойник.
\ml{$0$}
{--- Что дальше-то?}
{``Then what?}
\ml{$0$}
{Как ты собрался решать <<вопрос принципа>>?}
{How you're going to solve `a matter of principle'?}
\ml{$0$}
{Я не буду с тобой драться.}
{I shall not fight you.}
\ml{$0$}
{Я не желаю следовать вашему <<плану>>.}
{I refuse to follow this `plan'.}
\ml{$0$}
{И я не обязан ничего доказывать, тем более тебе.}
{And I have no things to prove, nor reasons to prove anything to you.''}

\ml{$0$}
{--- Потому что ты брехло и трус.}
{``Because you're a liar and coward.''}

\ml{$0$}
{--- Благодарю, я уже это уяснил.}
{``I got it, thanks.}
\ml{$0$}
{Что-то ещё?}
{Something else?''}

\ml{$0$}
{--- Мы взяли тебя, потому что...}
{``We took you with us because---''}

\ml{$0$}
{--- <<Мы>>? --- перебил его Ситрис.}
{``\emph{We}?'' \Sitris\ interrupted him.}
\ml{$0$}
{--- <<Взяли>>?}
{``\emph{Took}?''}

\ml{$0$}
{--- Не смей. Меня. Перебивать! --- рявкнул Акхсар.}
{``You---dare not---interrupt---me!'' \Akchsar\ roared.}
Носы мужчин разделяла уже какая-то пядь.

--- Акхсар, перестань, --- сказала Эрхэ, снова попытавшись влезть между ними.
Руки всех троих заплясали в непередаваемом танце, на который способны только воины в шаге от применения силы.
--- Ситрис, не обижайся, мы все нервничаем, это понятно...

--- Да какие обиды? --- фыркнул разбойник.
--- Я не держу обид на побеждённых.

Акхсар рыкнул и, схватив Ситриса за пучок волос, дёрнул изо всех сил.
Ситрис отшвырнул в сторону Эрхэ, все трое покатились по земле, но тут же оказались на ногах.
Фаланги мужчин вылетели из ножен и нацелились в открытые шеи;
сабля Эрхэ запоздало брякнулась сверху.

Ситрис оскорблённо-яростно дышал;
трубчатые заколки остались в руках Акхсара, и казавшиеся прямыми волосы свернулись в тугие чёрные кудри, давно не знавшие мыла и горячей банной воды.
Две пары глаз --- холодная северная трава и огнистый южный уголь --- скрестились, словно копья.

--- Перестаньте, --- простонала Митхэ, закрыв лицо руками.

--- Хорошо, --- неожиданно легко согласился Ситрис и убрал фалангу.
В его глазах сверкнули слёзы.
\ml{$0$}
{--- Знаете, я всё понял.}
{``You know, I've got it.}
\ml{$0$}
{Вы все думаете точно так же.}
{All of you think the same.}
\ml{$0$}
{Звон и сияние!}
{Ring and shine!}
\ml{$0$}
{Хорошо.}
{Well.}
\ml{$0$}
{Я с самого начала был лишним на этом празднике героев, и ничего, собственно, не изменилось.}
{All along I was but a passer-by on this hero festival, and nothing actually changed.}
\ml{$0$}
{Благодарю за приятную прогулку, благослови вас Сат-скиталец.}
{Thanks for the nice walk, may Stranger \Sat\ bless you.''}

Разбойник подхватил дорожный мешок, вскочил на оленя и скрылся в лесной полумгле, только дробно простучали копытца.
Акхсар сплюнул.

--- Ты пень пустоголовый, --- бросила растрёпанная Эрхэ, вытряхивая из складок рубахи сучки и листья.
--- Ну зачем, зачем тебе понадобилось брать его на принцип?

\ml{$0$}
{--- На войне следует исполнять приказы!}
{``Commands must be obeyed at war!''}

\ml{$0$}
{--- Любимый, мы не на войне.}
{``We're not at war, my love.}
\ml{$0$}
{И клятвой мы больше не связаны.}
{We have no oath to honor.}
Отряда чести больше нет, ты понимаешь?! --- Эрхэ швырнула саблю на землю и перешла на крик, больше похожий на плач.
\ml{$0$}
{--- Мы просто хотим помочь другу!}
{``We just want to help our friend!''}

Митхэ вдруг тоненько вскрикнула и спиной сползла с мшистого бревна.
Эрхэ в один миг оказалась возле командира.
Этот завывающий крик нельзя было спутать ни с чем другим.

--- Акхсар, схватки!
Садись на Серебряного...
До Тхитрона около десяти кхене, надеюсь, что мост ещё на месте...
На Серебряного, дурень, он быстрее, я тебе её подам...

\section{[-] Крепкая спина}

\textbf{<речь Ликхмаса, после которой часть народа решила уйти на север и договориться с Безумным>}

--- Ликхмас-тари, --- сказал мужчина, --- я не знаю, было ли правдой то, что написано в <<Легенде об обретении>>.
Я не знаю, предвидел ли Карлик будущее.
Но я знаю одно --- пока я и моя семья живы, никто не ударит тебя в спину.
Мы будем твоей спиной.

Затем он обернулся к стоящим вокруг людям:

--- Я верю этому человеку, как верю и в то, что скоро <<Легенда>> обретёт счастливый конец.
Если же нет --- то больше она никогда не будет звучать на земле, ибо некому будет её рассказать.

\section{[-] Эволюционный процесс}

\textspace

--- Выживут одни трусы и наплодят трусов, --- процедила сквозь зубы Анкарьяль.

--- Ты плохо понимаешь эволюционный процесс, --- возразил я.
--- Это один народ, одна популяция, один котёл генов.
Да, храбрецы и альтруисты гибнут чаще, но они делают жизнь своих соплеменников лучше.
Соплеменники, которые несут другие сочетания тех же самых генов, получат шанс на размножение и породят новых героев.
Именно поэтому, что бы ни говорили старики, никогда не исчезнут бескорыстные помощники, никогда не переведутся храбрецы.
В этих людях не просто их собственная сила.
В них сила их рода, их вида.

\section{[*] Мстительная тень}

--- Ситрис ар'Эр э'Тхинат!

Разбойник обернулся, как ужаленный.
В пяти шагах стояла крепкая молодая воительница с растрёпанными волосами.
Ситрис отскочил ещё на два шага, но под полубезумным взглядом даже это расстояние не показалось достаточным.

--- А я ведь поклялась, что найду тебя.

--- Здравствуй, Кхохо, --- справившись с собой, проговорил Ситрис.
--- Давно не виделись.
Ты теперь в Храме Тхитрона?

--- Не оттягивай мой язык в сторону! --- пропела воительница.
--- Помнишь платок?
Помнишь?

У разбойника задрожали колени.
Конечно, он помнил.
Неумело вышитый шёлковый платочек возвращался во снах снова и снова последние три дождя.
Однако...

--- Какой платок? --- осведомился он.
--- О чём ты?

--- Помнишь это? --- Кхохо отбросила спутанные волосы с лица, натянув пальцами шрам-улыбку.
Безумный зелёный взгляд вытягивал силы, жёг нестерпимым пламенем...

Мир завертелся вокруг Ситриса.
Из вечерних сумерек вдруг полезли дикие образы, отголоски беспризорного детства и разгульной юности.
Они тянули к Ситрису холодные липкие руки.
Безумные зелёные глаза светились, словно кошачьи, мертвенно-зелёный свет пробивался сквозь шрам-улыбку...

Колени уже не дрожали, а ходили ходуном в священном ужасе.
Что происходит?
Этого же не может быть...

--- Помнишь, Ситрис?
Помнишь ту милую женщину, которой дочь подарила платок?
Помнишь Хонхо ар'Мар?
Я грела её в своих объятьях, но её руки дрожали, словно листья мимозы.
Помнишь этот шрам? --- Кхохо оскалилась, натянув шрам ещё сильнее.
--- Помнишь?
Я сделала себе точно такой же, Ситрис.
Помнишь его?

Ситрис вытер со лба холодный пот и обругал себя последними словами.
Образы пропали так же быстро, как и появились.

--- Хороший трюк, Кхохо.
А я на миг решил, что за мной действительно пришла мстительная тень.

--- Она пришла, --- нежно прошептала Кхохо.
--- Я поклялась убить тебя, гниль, и теперь тебе не скрыться во мраке джунглей.

Уже на слове <<скрыться>> воительница летела в атаку.
Ситрис едва успел схватиться за эфес фаланги.

\section{[*] Ничья}

--- А неплохо, --- тяжело дыша, признал Ситрис.
Влажное дерево приятно холодило разгорячённую спину под рубахой.
Кхохо кивнула и, вытащив мехи, впилась в горлышко.

--- Дай воды глотнуть, рыбина лохматая.

--- Я тебе её в глотку засуну, --- посулилась воительница.
--- Вместе с мехами.

Однако мехи всё-таки протянула.
Ситрис сделал пару глотков и едва успел перехватить нож противницы.

--- Ладно, ладно, --- хмуро пропыхтела она, высвобождая руку.
--- Перестань.

--- Сама перестань!
После драки кулаками не машут.

--- Это не кулак! --- парировала Кхохо.

Воительница и разбойник посидели ещё немного.

--- Ты как здесь оказался-то? --- проворчала Кхохо.
--- Я слышала, ты с Митхэ ар'Кахр ушёл на запад и воевал в отряде чести.

--- Было дело, --- кивнул Ситрис.

--- Она здесь?

--- С чего бы ей быть здесь?
На Кристалл ушла, конечно.

--- А ты?
И от неё сбежал, трус?

--- Я не самоубийца, чтобы...

Кхохо вдруг захихикала.

--- Что смешного?

--- Да видели мы её.
На белом олене, с Акхсаром ар'Лотр, через южные ворота.
Три улицы дружно притворились, что ничего не происходит.
\ml{$0$}
{Здесь её в обиду не дадут, не бойся.}
{Don't worry, there's no human to let her be harmed.}
\ml{$0$}
{Север помнит.}
{The North remembers.''}

Ситрис промолчал.
Кхохо оценивающе посмотрела на него.

--- Я бы ожидала, что ты сбежишь ещё в Тхартхаахитре.
Или сдашь Митхэ с потрохами.

--- Я, может, и трус, но не предатель, --- буркнул разбойник.
--- А ты почему уехала из Кахрахана?

Кхохо смутилась.

--- Проблемы.

--- Ясно.
Я так понял, что в Тхитроне на кутрапов смотрят сквозь пальцы.

--- Я не кутрап! --- взорвалась Кхохо и вскочила на ноги.

--- Какие ещё проблемы могли заставить тебя уехать на Ближний Север?

\ml{$0$}
{--- Морду разбила вождю!}
{``I broke the warchief's face!''}

\ml{$0$}
{--- За что?}
{``Why?''}

\ml{$0$}
{--- Он порезал струны на моей цитре!}
{``He cut the strings of my zither!''}

\ml{$0$}
{--- Ты ещё и играешь, рыбина? --- хмыкнул Ситрис.}
{``Do you play the zither, you fish?'' \Sitris\ snickered.}
\ml{$0$}
{--- Если музыка соответствует твоему характеру, то я солидарен с вождём.}
{``If your music suits your character, I agree with the warchief.''}

Кхохо швырнула нож, и Ситрис едва успел его поймать.

\ml{$0$}
{--- Ты действительно до сих пор на меня сердита?}
{``Are you really still mad at me?''}

--- Нет, конечно, --- буркнула Кхохо и снова плюхнулась рядом с Ситрисом.
\ml{$0$}
{--- Если честно, мне до тебя дела нет.}
{``Honestly, I don't care about you.}
Просто в памяти всплыл хасетрасем и я немного занервничала.
\ml{$0$}
{Старые воспоминания...}
{Memories of yore ...''}

--- Можешь представить меня здешнему Храму? --- спросил Ситрис.

--- С чего бы? --- фыркнула Кхохо.
--- Иди к купцу, её зовут Кхотлам ар'Люм, и наплети ей что-нибудь.
Эта дура пристраивает даже ублюдков вроде тебя.

Кхохо поднялась, подхватила саблю и, бросив почти пустые мехи Ситрису, пошла в город.

\ml{$0$}
{--- Нож не потеряй, я за него полгода пахала.}
{``Don't lose the knife, I worked for half a year to buy it.''}

Ситрис посмотрел на нож.
Маленький изогнутый клинок с кольцом на рукояти, украшенный гравировкой в виде широкого улыбающегося рта.
Гриф-клинок, как называют такие ножи тенку.

\ml{$0$}
{--- Ты про этот рыболовный крючок? --- засмеялся разбойник.}
{``This kind of fishhook?'' the outlaw laughed.}
\ml{$0$}
{--- Цена завышена.}
{``It's overpriced.}
\ml{$0$}
{За полгода я храпом заработаю больше, чем он стоит.}
{If I slept for gold for half a year, I would earn more than it costs.''}

Кхохо обернулась.
Её глаза сузились, но злости в них не было.

\ml{$0$}
{--- Я буду причиной твоей смерти, --- спокойно констатировала воительница.}
{``I'll be a cause of your death,'' the warrior calmly told.}
\ml{$0$}
{--- Это единственная неизменная истина в изменяющемся мире.}
{``It's the only changeless thing in the permanently changing world.''}

Ситрис улыбнулся.
Он чувствовал своим обострённым нюхом, что Кхохо говорит правду, но почему-то её слова не испугали, а успокоили.
Смерть больше не таилась в закоулках города и сумраке леса, она приняла человеческое обличье и обрела голос.

\ml{$0$}
{--- Надеюсь, ты не воспользуешься этим, --- разбойник повертел нож Кхохо в руках.}
{``You won't use that, I certainly hope so,'' the outlaw fiddled with \Kchoho's knife.}
\ml{$0$}
{--- Такого позора не переживу даже я.}
{``Even I can't take such a dishonor.''}

Кхохо поспешила отвернуться и быстро уйти.
Ей очень хотелось захохотать, но казалось неуместным делать это при заклятом враге.

Ситрис уже собирался встать, когда на поляну вышел Цапка.
В зубах кота красовалась жирная капюшонная крыса.

--- А ты-то что удрал, дружище усатый? --- вздохнул Ситрис.
--- Думаешь, в Тхитроне мыши толще?

Кот с некоторой иронией взглянул на разбойника, и тот почувствовал себя круглым дураком.
Да, действительно, при чём тут вообще толщина мышей?

\section{[*] Мать}

\epigraph
{Я --- словно дом, двумя окнами в сад,\\
Сквозь глаза листья падают в пол.\\
Входят в меня и уходят друзья,\\
Остывают следы их шагов.\\
Я жду тех, кто не приходит,\\
Открыта настежь дверь.\\
Ветер сквозь комнаты гонит\\
Звон пустоты, сметая всё на пути...}
{Эрхэ Колокольчик}

\textspace

--- Я могу стоять!
Я могу идти! --- кричала Митхэ.
--- Я прошу вас только помочь мне одеться!

--- Ты слаба, --- сквозь слёзы сказала Кхотлам.
--- Золото, одумайся.
Ты не сможешь сражаться.

--- Смогу, --- прорычала Митхэ, безуспешно пытаясь затянуть ремни нагрудника.
Придя в ярость, она дёрнула ремешок и оторвала его.
--- Безумный-кровопийца!
Помоги мне, Акхсар!

--- Золото, тебе нужно отдохнуть, --- Акхсар схватил воительницу под руки.
--- Золото, послушай меня.
Мы найдём Хата, я обещаю тебе.
Только останься.

Митхэ остановилась и уткнулась лбом в холодный малахит стены.

--- Я не могу.
Не могу, ребята.

Наступило молчание.
Задумчиво потрескивали головёшки в очаге, за окном шумел дождь.
Воительница высвободилась из объятий Акхсара, медленно, сбивающейся походкой подошла к тростниковой колыбельке, в которой посапывал ребёнок.
Митхэ устало улыбнулась и провела пальцем по маленькой, сжатой в кулачок ручке.
Потом оглянулась на замерших у стены друзей.

--- Вы думаете, что я сошла с ума, да?

Акхсар и Кхотлам переглянулись, не зная, что ответить.
Наконец Кхотлам неуверенно подошла к воительнице и дрожащими руками принялась завязывать ремешки.

--- Я буду молиться лесным духам, чтобы ты вернулась к ребёнку.

--- Не знаю, вернусь ли я, Пёрышко, --- проговорила Митхэ.
--- Но если я не попытаюсь найти Атриса, то буду жалеть об этом всю жизнь.
Если он мёртв --- лучше будет и мне умереть.
Нет ничего хуже, когда ребёнка воспитывает несчастная женщина. Пообещай мне, что будешь для Лисёнка счастливой кормилицей...
если я не вернусь.

--- Обещаю.

--- Снежок, --- голос Митхэ приобрёл мягкий оттенок, --- ты мой друг, и твоя верность уже давно перешагнула границы обычной дружбы.
Поэтому, если хочешь, уходи, но будь другом ещё раз --- переседлай Серебряного.
У меня самой не хватит сил.

Акхсар опустил глаза, кивнул и вышел за дверь.
Кхотлам проводила его долгим печальным взглядом и снова занялась доспехами.

--- Ремешки не сходятся, --- виновато прошептала Кхотлам.

--- Тяни, Пёрышко, тяни изо всех сил, не бойся.
Это всего лишь тело.

Ремни затрещали, и грубая кожа сдавила Митхэ живот и грудь, впилась в нежную, размякшую плоть.
\ml{$0$}
{Воительница не шевельнула ни одним мускулом, но у неё на лбу выступили крупные капли пота.}
{The warrior moved no one muscle, but sweat was beaded her forehead.}

Митхэ, шатаясь и хрипло дыша, подошла к стойке и привычным движением подпоясалась, перебросив саблю за бедро.
Бросила полный боли прощальный взгляд на тростниковую колыбельку и открыла дверь, запустив в зал рокочущий шум ливня.
Ветер ворвался в дом, разбросал по полу сухие листья и освежил лицо воительницы.
Её взгляд стал более осмысленным.

--- Скажи, я плохая? --- грустно улыбнулась она напоследок подруге.

--- Ты лучшая из тех, кого я знаю, --- заверила её Кхотлам.
Митхэ обняла женщину и, приподнявшись на носках, поцеловала её в бледные, мокрые от слёз губы.

--- Храни тебя лесные духи, Пёрышко.

Кхотлам кивнула.
Воительница нырнула в дождь, как в толпу сражающихся.
Купец долго смотрела ей вслед и, плотно закрыв дверь, подошла к колыбельке.

--- Ой, а кто у нас тут такой маленький Лисёнок?
Кто у нас такой золотой?.. --- ласково зашептала она, поправляя одеяльце.

Тоскливо тянулись михнет.
Вдруг слабый сквозняк заставил женщину повернуть голову к двери.
Там стоял мужчина.
Кхотлам могла бы поклясться, что он вошёл через скрипучую закрытую дверь без единого звука.
С его чёрных сальных кудрей капал дождь;
обсидиановые глаза не отрываясь глядели на колыбель.

\ml{$0$}
{--- Это ты Кхотлам ар'Люм? --- хрипло спросил мужчина.}
{``\Kchotlam\ ar'\Loem\ is your name?'' the man asked hoarsely.}

\ml{$0$}
{--- Ты ещё успеешь её догнать, --- тихо ответила ему Кхотлам.}
{``It isn't too late to go after her,'' \Kchotlam\ quietly answered.}

\ml{$0$}
{--- Я знаю.}
{``I know.''}

Кхотлам улыбнулась, и Ситрис понял, что пришёл куда надо.
В улыбке купца не было ни жалости, ни презрения.
На миг разбойник подумал, что пришёл в дом давно забытой любовницы, которая сегодняшним утром вдруг вспомнила, что любит и ждёт, --- в Кхотлам была нежность, не омрачённая мыслями, не иссушённая временем.

Женщина подошла к гостю и погладила его по мокрой впалой щеке:

\ml{$0$}
{--- Сними одежду и садись у огня.}
{``Take your clothes off and sit by the fire.}
\ml{$0$}
{Рубаху тоже снимай, не стесняйся.}
{Take off the shirt as well, feel free.}
\ml{$0$}
{Хватит с тебя испытаний.}
{You've had enough of trials.''}

Головешки трещали в очаге.
Где-то вдали раскатился гром.
Ребёнок в тростниковой колыбельке открыл глаза и испуганно заплакал.
Кхотлам нежно и тихо запела, мерно качая колыбельку, как когда-то волны Могильного пролива качали её корабль;
под тихое пение ребёнок и завернувшийся в старое одеяло Ситрис заснули --- почти одновременно.

\chapter{[-] Осаждённые}

\section{[-] Встреча с мамой}

\textspace

--- Кормилица! --- закричал я.

--- Лисёнок! --- Кхотлам бросилась мне в объятия.

--- Я думал, ты дома!
Ведь из Тхитрона приходят письма с пометкой <<Двор Люм>> и...

--- Это не я, --- засмеялась кормилица.
--- Оставила твоих сестрёнок.

--- Манэ и Лимнэ?
Так им же...

--- Им уже под сорок дождей, дитя, --- сказала Кхотлам.
--- Они совсем взрослые.
В одном только не изменились --- всё у них общее: носят одну одежду, спят на одной лежанке, даже мужчину нашли одного на двоих...
Давно ты не был дома.

--- Зря ты оставила на них город в такие времена, --- упрекнул я её.

--- Они способные, у них всё получится, Тхитрон обязательно выстоит, --- заверила меня кормилица.
--- Я бы не сделала так, если бы сомневалась.
В крайнем случае, если всё будет совсем плохо, люди Тхитрона уйдут на север, к Ледяной Рыбе.
Идолы за нами не пойдут, они недолюбливают холод, а с хака и местными людьми мы как-нибудь разберёмся.
Еды хватит на несколько дождей, я велела наделать сушенины\FM, в этом году поля просто взбесились...
\FA{
Смесь из высушенных овощей и вяленого мяса, пересыпанная солью с углём.
}
Да что мы тут в проходе стоим, людям мешаем, пойдём внутрь...

\section{Языки кормилицы}

--- Лисёнок, --- обратилась ко мне Кхотлам, --- я понимаю, что сейчас не лучшее время для тебя, но мне нужно выучить язык демонов, против которых мы сражаемся.

--- Многие из них переговариваются неслышно для вас или используют код, --- заметил я.

--- И всё-таки.
Ты упомянул сектум-лингва.
Не мог бы ты о нём рассказать?..

\section{[-] Беременность}

\textspace

Я посмотрел на спящую Чханэ.
Она дышала ровно, с лёгким присвистом.
Интересно, в чём причина её бесплодия?

Я закрыл глаза, и мой демон приступил к подробному анализу её организма.

Да, причина состояла в эндокринном сбое.
Давняя травма мозга сдвинула гормональный баланс, и мужчина превратился в женщину.
Но не до конца --- механизм, запускающий фертильность женского организма, остался спать.
Длительные депрессии, стрессы, употребление алкоголя только усугубили её состояние.

Однако детородные органы Чханэ были сформированы правильно, никаких пороков развития их я не нашёл.
А значит, проблема решаема.

Я начал тонкое преобразование клеточных структур.
Люди-тси сильно отличаются от большинства других видов.
Однако принципы у них общие, а я всегда любил экспериментировать.

... Здесь изменить концентрацию одного медиатора, там другого...

... Добавить киназ, увеличить экспрессию гена рецептора...

... Тихо, фермент, мой хороший, куда так разогнался?
Вот тебе ингибирующий лиганд, остынь...

Постепенно, михнет за михнет, цепная реакция охватывала все глубинные структуры гипоталамуса.
В тонком аромате женского пота появились другие, игривые нотки.
Чханэ задышала медленнее, на лице повилась улыбка --- изменились даже её сны.
Вскоре она чуть-чуть приоткрыла глаза:

--- Лис, ты уже проснулся?
Который час?

Я вместо ответа нежно поцеловал подругу.
Она, тёплая и мягкая, сонно откликнулась на мои ласки.

Через три дня на шее Чханэ появились стигмы беременности.

\section{[-] Ловля котов}

\textspace

Кормильцы разнесли всем плошки с едой.

--- Хай, девочка моя, ты котов ловить собралась? --- заулыбалась Кхотлам и ласково погладила Чханэ по полосатой шее.
Затем незаметно для всех отвесила мне самую тяжёлую затрещину в жизни, добавив одними губами: <<Пень пустоголовый>>.

Чханэ засмущалась и начала есть.
Прикончив миску, она засмущалась ещё больше и попросила добавки.
Следующую миску Кхотлам принесла ей уже сама, не дожидаясь просьбы.
Чханэ, урча, перемалывала зубами куски мяса.
Подруга напоминала голодного оцелота, как никогда в жизни.

\textspace

\section{[-] Тайна старика}

\epigraph
{Во времена, когда мы уже видим предел технологического и биологического прогресса, правильная настройка важнее аппаратных характеристик.
Это применимо как к компьютерным системам, так и к сапиентным существам.}
{Длинный-Мокрый-Хвост, мыслитель, химик-технолог Тси-Ди}

\textspace

--- Почему именно я?

--- Критериев много.
Насчёт тебя я знаю только два.
Ты каким-то образом обратил на себя внимание Храма в детском возрасте.
Кроме того, ты --- хранитель Митхэ ар'Кахр.
Её потомство, наряду с потомством действующего Короля-жреца и ещё нескольких значимых личностей, имеет приоритет.

--- То есть вы выбираете не только по личным качествам, но и по качествам дарителей?

--- Способности могут наследоваться, это очевидно.
Но само по себе происхождение не является ключевым фактором, оно всегда идёт в дополнение к личным качествам.

--- Я так понимаю, что после отбора претендент воспитывается в особых условиях.

--- Условие только одно --- будущий Король-жрец не должен ни в чём нуждаться.
Его должны воспитывать лучшие, любящие кормильцы.
У него должен быть доступ ко всем знаниям сели, его должны тренировать лучшие воины по его надобностям.
Он должен быть обучен искусству обольщения и не знать недостатка в женщинах и мужчинах, если он испытывает в них потребность.
Ясное дело, голодать и испытывать унижение он тоже не должен.

--- И поэтому...

--- И поэтому претендентов много.
Сложно уследить за всем, особенно в лихие времена.
Тебя мы отбраковали еще пять дождей назад.

--- Благодарю за честность, --- ухмыльнулся я.
Старик не ответил на улыбку.

--- Да, тебя отбраковали именно поэтому.
Ты не наигрался в детстве и превращаешь в игру даже серьезное дело.
Трукхвал ар'Хэ лично написал, сразу после тхитронской диверсии.

--- Что именно он написал?

--- Раз уж ты спросил, --- ядовито ответил Хитрам-лехэ и, достав из кармана робы свиток, протянул мне.

\begin{quote}
<<Трёхэтажному Храму от Трукхвала ар'Хэ, жреца Тхитрона.\\
--- дождя 12011, Год Церемонии 5.\\
~\\
С сожалением сообщаю, что Ветер Короны должен быть отменён.
Ликхмас ар'Люм проявляет опасные качества --- он импульсивен, чрезмерно подвержен чувствам и потому часто подвергает себя опасности.
Кроме того, он склонен недооценивать опасность ситуации и играет с ней, как ребёнок с оружием --- безусловная ошибка Кхотлам.
Кхотлам со мной не согласна, она всё списывает на возраст и считает, что ему нужно дать ещё один шанс.
Подозреваю, её родительские чувства к Ликхмасу значительно сильнее чувства долга.
Я верю в то, что Ликхмас будет прекрасным жрецом, из тех жрецов, что готовы стать щитом для народа и положить свою жизнь за жизни других.
Я предполагаю, что со временем он станет уважаемым человеком, лидером мнений.
Но я категорически против того, чтобы он когда-либо претендовал на пост Короля-жреца>>.
\end{quote}

--- Так кто же?..

--- Это тайна, --- отрезал Хитрам-лехэ.
--- Может быть, когда-нибудь тебя в неё посвятят.
Надеюсь, я этого не увижу.

--- Мне очень нужно знать, кто это!
Я должен передать ему...

--- Я не могу сказать, --- раздражённо ответил старик.
--- Всё, Король-жрец, разговор окончен.
Помни своё место.

Старик вырвал у меня из рук пергамент и потряс им перед моим лицом.

--- Я должен был прочитать это перед советом Храма, --- заявил он.
--- Не заставляй меня пожалеть о том, что я этого не сделал.

Хитрам-лехэ ушёл в свою келью, хлопнув дверью.
Чтоб этих сели с их традициями!

Разумеется, старик ни в чём не виноват.
Здесь чувствовалась рука куда древнее, чем его.
Тси, именно они.
Они были очень открытыми, но настоящие тайны их учили хранить с детства.
<<Доверить тайну тси --- всё равно, что не доверять её никому>>, --- сказал как-то давно мёртвый знакомый.
В его тоне слышалось многое --- презрение, бессильная злость... и тоска по чему-то очень важному, чего мы, демоны, были лишены.

\section{[-] Жрец в тени}

Я долго раздумывал, стоит ли ещё раз попробовать убедить старика.
Новому Королю-жрецу следовало передать жизненно важные инструкции.
Но вдруг судьба преподнесла мне сюрприз.

В другом конце коридора показалась высокая фигура.
Трукхвал ар'Со.
Я приметил его, как только вошёл в храм.
Чёрные как смоль волосы и огромные глаза им в тон.
Величественная осанка, походка кобры и спокойный взгляд человека, видевшего в этой жизни многое.
Да, возможно, это его готовили.
Скорее всего, именно он был главным претендентом на роль Короля-жреца, пока события не пошли так, как пошли.

--- Трукхвал, --- окликнул я его.

--- Король-жрец, чем могу служить? --- коротко поклонился мужчина.
Чёрные как ночь глаза смотрели ожидающе, без малейшего намёка на вызов или подобострастие.
Мои сомнения рассеялись.

<<Да, это тот, кто мне нужен>>.

--- Ты уйдёшь на север с беженцами, --- без обиняков сказал я.
Мужчина нахмурился.

--- Я хотел бы отправиться на юг, Ликхмас, --- мягко промолвил он.
--- Я плохо владею оружием, но вам будут нужны лекари, счетоводы и люди для отправления обрядов.
С беженцами уйдёт достаточно белых плащей.

--- Это приказ, --- сказал я не терпящим возражений тоном.
--- Если нас разобьют, ты возьмёшь командование на себя и поможешь ушедшим сели закрепиться на севере.

--- Командующий силами беженцев уже назначен, --- заметил Трукхвал.
--- Решать не тебе и не мне.

--- Ты прекрасно знаешь, что значит <<командование>>.

В глазах Трукхвала мелькнуло понимание.
Он кивнул.

--- Король-жрец, если вас разобьют... --- блестящие брови едва заметно шевельнулись, и я вдруг осознал, что этот человек сумел вложить в мимику ещё не высказанный вопрос.

--- Покорись, Трукхвал, --- опередил я.
--- Дай Безумным гарантии лояльности народа сели.
Свобода и честь --- ничто, если о них некому будет вспомнить.

На тонких губах Трукхвала медленно проступила улыбка.
Он тоже пытался читать непонятные книги.

--- Безумный --- не Безымянный, Ликхмас, --- мягко сказал мужчина.
--- Но я попробую.

--- Тебя назовут трусом, --- предупредил я.

--- Моё имя предпочтут забыть, --- кивнул жрец.
--- <<В жизни и смерти>>.

--- <<Жизнь и процветание>>, --- невпопад брякнул я.

Лицо Трукхвала просветлело, словно больше всего на свете он ожидал услышать именно это.
Жрец поклонился, и спустя несколько мгновений его величественно развевающаяся роба исчезла за углом.

Обернувшись, я увидел, что Хитрам-лехэ вышел из своей комнаты.
Старик тяжёлым взглядом смотрел на меня.

--- Ты ему рассказал?

--- Это твоя тайна, --- ответил я.

--- Тогда что ты ему сказал?

--- А это моя тайна, --- улыбнулся я и направился в зал.

Хитрам преградил мне путь, его костлявая рука больно впилась мне в плечо.

--- Место Короля-жреца --- не твоё место, Ликхмас ар’Люм, --- без обиняков сказал старый жрец.
Он шумно дышал мне в лицо, и это дыхание не предвещало ничего хорошего.
--- Я знаю, как обстоят дела, и потому проголосовал за тебя.
Но истина остаётся неизменной --- это не твоё место.
Помни, пока живёшь.
Хотя всё решится само --- Безумный от тебя даже косточек не оставит.

Я кивнул.
Хитрам помолчал и вдруг смущённо закашлял.
Костлявая рука исчезла в рукаве робы.

--- Я надеюсь, что ты приведёшь сели домой.
Тогда старик сможет попросить у тебя прощения за сказанное.

\section{[@] Прощание со стригами}

\textspace

Вскоре мы остановились.

--- Вот тот самый воздушный поток, --- Облачко указала в небо.
--- Очень удачно, что мы заметили его.
Мы сможем лететь очень долго и обязательно найдём подходящее для жизни место.

Всю дорогу товарищи молчали.
Но сейчас, в последний миг перед прощанием, они не выдержали.

--- Вы нужны нам! --- страстно заговорил Мак.
--- Исследования, постройка системы, да мало ли!..

--- Вас чересчур мало! --- со слезами на глазах перебила его Заяц.
--- А у нас пока есть медицинское оборудование и...

--- Медицинских данных о стригах у вас почти нет, --- спокойно сказала Облачко.
--- Ваши врачи могут принести больше вреда, чем пользы, Костёр об этом сказал не один раз.
И пожалуйста, --- Облачко вперила немигающий взгляд в Мака, --- не считайте это предательством.
Просто примите как факт, что миграция необходима для выживания нашего вида.

Мак вздохнул и обхватил бритую голову руками.
Заяц плакала.

--- Это пожелание всех глазастиков? --- спросил я.

Совы кивнули почти одновременно.

--- Может быть, вам нужно что-то ещё?

--- Мы не унесём много вещей, --- сказал Редуктор-Оси-Планеты.
--- Всё необходимое с нами.

--- Жаль, что мы не успели узнать вас получше, --- сказал я.
--- Вы достойные сапиенты.
Удачи вам.

Стриги как по команде подтянулись и уставились круглыми глазами в небо, где медленно тёк тёплый воздушный поток.

--- Пусть ваши крылья не знают переломов, --- сказала Облачко, окинув взглядом провожающих.

Крылья --- это то, что отличало этих больших сильных птиц от меня, маленькой наземной пчелы.
Я знал, что мы с глазастиками уже никогда не будем единым народом --- наши пути разойдутся насовсем.
Но последнее пожелание Облачка засело в моей памяти.
Это было пожелание равным.
Это было пожелание тси.

Я обнял молодую женщину, уткнувшись лицом в пушистое перо.
Она в ответ ласково цапнула клювом мой усик.

\section{[-] Последний корабль}

Последний корабль на север отошёл утром.
На причале стояло тяжёлое молчание.
Люди обнимали друг друга теплее, чем обычно.

У одного из вёсел сидел Трукхвал.
Жрец сменил свою робу на одежду моряка и собрал чёрные волосы лентой.

Лусафейру всегда говорил: <<При составлении самого простого плана нужно продумывать сразу три пути отступления.
А если об этом правиле знают враги, то пять>>.
Я не стал давать задание Трукхвала никому другому.
Да, он вполне мог погибнуть, статистика неумолима.
Но опыт подсказывал мне, что на мотивированных статистика работает плохо, пока они не сделают своё дело.

Вырвавшись из размышлений, я заметил, что Трукхвал всё это время смотрел на меня.
Я помахал ему.
Жрец едва заметно улыбнулся и кивнул.

--- Глубина! --- рявкнул капитан.
Бодро застучал барабан, и моряки, ухая, навалились на вёсла.

\section{[-] Письмо Люм}

\textspace

Гонец подбежал ко мне и почти рухнул в поклоне.
Воины подхватили его под руки и усадили за стол.
Я набрал воды в чашу, и человек впился в неё.

--- Почему ты так бежал? --- поинтересовалась воительница слева.
--- Так даже оленя загнать можно.

\ml{$0$}
{--- Сообщение от Двора Люм, --- пробормотал человек, едва отлипнув от чаши.}
{``Message, the House of \Loem,'' the man muttered as soon as drained his goblet.}
\ml{$0$}
{--- <<Беженцы дали бой объединённым силам врага и отбились малой кровью, но удерживать полуразрушенный Тхитрон более нельзя.}
{\emph{``Refugeers fought back united enemy forces, losses are low, but ruined \Tchitron\ can't be held.}}
\ml{$0$}
{Город отдан сельве.}
{\emph{We left the city to the Silva.}}
\ml{$0$}
{Отныне каждый сам за себя.}
{\emph{Everyone for himself now.}}
\ml{$0$}
{Больше не пишите>>.}
{\emph{No more letters.''}}

\section{[*] Отказ}

\textspace

\ml{$0$}
{--- Тебе нечего предложить взамен.}
{``You have nothing to offer.''}

Митхэ аккуратно выдвинула из ножен саблю.
Легенда Серого Рассвета весело сверкнула всем великолепием гравировки.

Си-Абву ухмыльнулся.

\ml{$0$}
{--- И чем ты будешь защищаться, если мои воины захотят узнать, что у тебя между ног?}
{``And what will you defend yourself by, if my warriors try to find out what you've got between your legs?}
\ml{$0$}
{Листом юкки?}
{A yucca leaf?''}

Митхэ, не моргая, спокойно смотрела на вождя Си-Абву.
Молниеносный удар --- и он будет лежать у её ног, захлёбываясь собственной слюной, а Акхсар и Эрхэ так же быстро расправятся с пятью его воинами.
Но что это будет значить?
Новая бесконечная война между сели и хака, новые жертвы?

Си-Абву, казалось, видел её мысли насквозь.

\ml{$0$}
{--- Разговор окончен, женщина.}
{``This negotiation is over, woman.}
\ml{$0$}
{Пленник мой, и я буду делать с ним всё, что пожелаю.}
{The slave is mine, and I will do with it whatever I desire.}
\ml{$0$}
{А я желаю, чтобы его сегодня на закате принесли в жертву Великому.}
{So I desire it to be sacrificed to the Great, at nightfall.''}

--- Сегодня Такани-Жой, чужеземцы, --- добавил старейшина.
--- После жертвы будет большое празднество.
\ml{$0$}
{Оставайтесь и почтите богов вместе с нами.}
{You can stay and honor the gods with us.''}

Митхэ, не сказав ни слова, поднялась и махнула друзьям.
Вместе они покинули душный шатёр вождя.

\section{[*] Середина Дождя}

В тот день хака праздновали Такани-Жой, Середину Дождя.
Несмотря на грозу, в поселении царило оживление: люди в праздничных одеждах сновали под навесами, весело переговаривались, пели песни.
В жилищах играла музыка и залихватски ухали танцоры, отбивая голыми пятками по половым доскам.
Но там, где проходила мрачная Митхэ с друзьями, веселье слегка стихало.
Девушки пугливо прятались под покрывалами, когда колючий взгляд Акхсара скользил по ним;
мужчины презрительно смотрели на коротковолосых, по-военному одетых Митхэ и Эрхэ, и обменивались короткими сексуально-уничижительными замечаниями.

--- Зачем они кутаются?
Такая духота, --- буркнул Акхсар, кивнув на девушек.

\ml{$0$}
{--- Они уверены, что одежда скрывает их тела, --- пояснила Эрхэ.}
{``They think their bodies can't be seen through clothes,'' \Oerchoe\ explained.}
--- Помнишь тенку?
Пленников достаточно было завесить покрывалом, как попугайчиков, и они ничего не видели.

--- Вон те мужчины только что сказали, что ты годишься лишь для секса.
\ml{$0$}
{Я прочитал это по губам.}
{I read their lips.}
\ml{$0$}
{Интересно, как они это поняли, если они даже не видят тебя сквозь одежду?}
{How could they know if they can't even see through your clothes?''}

\ml{$0$}
{--- Я не уверена, что они вообще что-то в этой жизни понимают, --- призналась Эрхэ.}
{``I'm not sure they know anything in this life,'' \Oerchoe\ admitted.}
\ml{$0$}
{--- Что будем делать, Золото?}
{``What will we do, Gold?}
\ml{$0$}
{Атриса нам не отдадут.}
{They will not give \Aatris\ back willingly.''}

\ml{$0$}
{--- Попробуем отбить, --- ответил за подругу Акхсар.}
{``We'll try to steal,'' \Akchsar\ answered for the friend.}

\ml{$0$}
{--- И что дальше?}
{``What then?}
\ml{$0$}
{Новая война?}
{A new war?''}

--- Проблемы в любом случае будут, --- сказала Митхэ, впервые открыв рот.
--- Но вам, думаю, куда проще будет объяснить Советам оглушённого шамана, чем убитого вождя, верно?

Эрхэ кивнула.

--- Узнать бы дорогу к их святилищу.

--- А чего её узнавать, --- сказал Акхсар и взглядом указал на раскрашенные столбы, за которыми вилась безлюдная, уходившая в гору тропинка.
--- Давайте только пойдём по обочине и потише, а то мы тут как кетцаль, поющий в ясный полдень.

\section{[*] Дети}

Вскоре воины оказались на небольшой площадке перед естественной пещерой, в которой хака устроили капище.
Раскрашенные столбы стояли здесь, словно высокие, коренастые стражники в праздничных одеждах.

<<А где стража?>> --- поинтересовалась Эрхэ птицей-лирой.

<<Может, празднуют>>, --- предположил Акхсар.

<<Ага, --- саркастически ответила Митхэ.
--- В селении находятся три воина сели, желающие отбить пленника, а вся стража празднует.
Я скорее поверю, что мы просто опередили перехватчиков на минуту-другую.
Ладно, пошли внутрь>>.

Митхэ бесшумно, приникнув к стене, вошла в преддверие пещеры.
Печально капал с потолочных сталактитов конденсат, тихо потрескивали угли почти погасшего костра, нервно поскрипывала решётка приоткрытой темницы.

Митхэ осторожно заглянула в темницу.
Никого.

<<Они уже начали жертвоприношение. Быстро за мной>>, --- знаками показала Митхэ и бросилась в глубь пещеры.

--- Не спеши, --- вдруг раздался детский голос, и в темноте проступил силуэт мальчика хака дождей пятнадцати.
В руках мальчик держал игрушечную деревянную пику.

Митхэ замерла, разглядывая неожиданного противника.

--- Ты не пройдёшь мимо меня, Митхэ ар’Кахр, --- серьёзным тоном сказал мальчик на необычайно чистом цатроне.

Воины завороженно рассматривали мальчика.
Наконец Акхсар поудобнее перехватил нож и двинулся вперёд.

<<Стой>>, --- зашипела Митхэ.

<<Золото, нам нужно спешить>>, --- напомнил ей Акхсар.

<<Это же ребёнок>>, --- укоризненно сказала Эрхэ.

--- Почему ты не хочешь нас пропустить? --- спросила Митхэ.

--- Потому что этого человека желают боги, Митхэ ар’Кахр, --- всё с той же забавной серьёзностью ответил мальчик.
--- Ты хочешь отнять пленника у богов.
Я не могу этого допустить.

\ml{$0$}
{--- Я пройду к святилищу, хочешь ты этого или нет, --- сказала Митхэ.}
{``I shall come into the sanctuary, whether you want it or not,'' \Mitchoe\ said.}

--- Тогда тебе придётся убить меня, --- сказал мальчик и, поджав широкие губы, неуклюже вскинул пику.
Акхсар вдруг напрягся --- от кончика шёл сильный запах лакового сока.

\ml{$0$}
{<<Золото, копьё отравлено.}
{\emph{``Gold, the spear is poisoned.}}
\ml{$0$}
{Будь осторожна>>.}
{\emph{Be careful.''}}

--- Мне не нужно тебя убивать, --- ласково сказала Митхэ, успокаивающе махнув Акхсару.
\ml{$0$}
{--- Я просто заберу у тебя пику, и никто не пострадает.}
{``I'll just take your pike, and nobody gets hurt.''}

\ml{$0$}
{--- Заберёшь, --- согласился мальчик.}
{``You will,'' the boy agreed.}
\ml{$0$}
{--- Но я здесь не один.}
{``But I'm not alone here.''}

Из темноты показались ещё двое детей с таким же простым оружием.

Митхэ замерла, кусая губы. Атриса с минуты на минуту должны принести в жертву, ей оставалось лишь... но детей?..

Акхсар и Эрхэ наполовину вытащили из ножен клинки.

<<Стоять>>, --- снова прошипела Митхэ.

--- Послушай, малыш, --- ласково сказала она, присев на корточки.
--- У меня дома остался ребёнок.
Такой же, как ты. И тот пленник --- его \textit{papa}. У тебя есть \textit{papa}?

--- Есть, --- раздался за её спиной звучный голос Си-Абву.
Акхсар звучно, витиевато выругался.

Митхэ обернулась.
Вождь стоял в парадном облачении, без оружия, и смотрел на воинов-сели.

--- Молодец, сын.
Ты настоящий защитник святилища.
Однажды ты станешь прекрасным вождём.

Мальчик осклабился и крепче сжал своё нехитрое оружие.

--- Как тебе моя стража, Митхэ ар’Кахр?
Я знал, кого поставить.

\ml{$0$}
{--- Я всё равно пройду, --- бросила Митхэ.}
{``I shall pass anyway,'' \Mitchoe\ said.}

\ml{$0$}
{--- Не пройдёшь, --- ответил вождь.}
{``You shall not,'' the chieftain answered.}
--- Ты отнимешь копья у троих детей, но не у тридцати.

Из коридоров выбежала целая армия ребятишек --- неумело раскрашенных и босоногих.
Каждый сжимал отравленную пику, и конец каждой пики смотрел прямо на воинов-сели.

--- Дети, --- деловито сказал вождь, --- сегодня вы будете защищать наши святыни.
Если эти чужестранцы хотя бы попытаются сделать шаг в сторону Ахам-Бвесы, убейте их.

--- Будет сделано, --- хором ответили дети.

Митхэ, не говоря ни слова, села на землю и закрыла лицо руками.
Акхсар и Эрхэ с ужасом смотрели на командира.

Акхсар бросил ледяной взгляд на вождя и медленно взялся за фалангу.

\ml{$0$}
{--- Ты заплатишь, --- сказал он.}
{``You will pay,'' he told.}

--- Ну-ну, Акхсар ар’Лотр, --- примирительно поднял руки Си-Абву.
\ml{$0$}
{--- Тебе не победить.}
{``You can not win.}
\ml{$0$}
{Перестань.}
{Stop it.''}

Эрхэ села рядом с Митхэ и обняла её.

--- Золотце, --- ласково сказала она.

--- Я не смогла, Обжорка, --- сквозь тихие рыдания прошептала Митхэ.
--- Не детей.

--- Я знаю, --- заверила её Эрхэ.
--- Ты была самым честным воином, защитой и голосом слабых.
Ты не хотела украшать своё тело, не стремилась к власти, твои желания всегда были просты и человечны.
Именно поэтому я когда-то бросила <<Стервятников>> и пошла за тобой.
Живущие по звериным обычаям всегда будут побеждать честных и принципиальных.
Но это не значит, что нужно прекращать бороться.

--- Я не хочу больше бороться, --- прошептала Митхэ.
--- Я хочу быть с Атрисом.

--- Тогда встань и сражайся до конца.

Митхэ зарыдала ещё сильнее и замотала головой.

--- Хватит, Обжорка, --- проворчал Акхсар.
--- Мы проиграли, это ясно как солнечное утро.
Золото, я знаю, что тебе хочется... много чего.
Но сейчас, ради всех лет, что мы провели вместе, ради всех кхене, что мы прошли --- встань и прими поражение достойно.
Старина Хат вряд ли хотел бы увидеть тебя такой.

Митхэ кивнула и, тяжело дыша, встала на ноги.
Дети настороженно приподняли копья, а вождь, слабо улыбаясь чему-то, прислонился к холодной каменной стене и скрестил руки на груди.

Вскоре со стороны святилища раздался слабый, приглушённый толщей камня стон.
Атрис словно знал, что его женщина стоит за стеной --- прочие под ножами шаманов кричали во весь голос, вселяя в сердца слушающих ужас перед волей Безумного.
Атрис стонал, а Митхэ отстранённо вспоминала те далёкие счастливые минуты, которые она проводила со своим мужчиной.
Встречи, расставания, тихие беседы под одеялом, горячий травяной отвар, шум дождя...
Атрис видел в дожде нечто большее, чем просто воду с неба, и для Митхэ дождь тоже стал родным и близким.
Когда они встретились, шёл дождь.
Когда они гуляли и целовались, тоже почему-то всегда шёл дождь.
Где-то снаружи и сейчас гремела гроза и хлестал ливень, но капли его бились о твёрдый холодный камень, тщетно пытаясь охладить пылающую агонию Атриса и тяжёлую, бессильную печаль Митхэ.

<<Что, если мы родились, чтобы Это сделать?>>

Спустя двадцать михнет, которые показались Митхэ вечностью, стон затих навсегда.

\section{[@] Собачьи нежности}

Занимался вечер.
Сухие травяные стебельки шелестели под лёгким степным ветром, зрелые колоски и метёлки звенели своим богатством --- твёрдыми семенами.
Я сидел в траве, особенно остро ощущая её сухость, прохладу западного горного ветра и дружеские чувства к Фонтанчику и Заяц.
Друзья сидели, обнявшись.
Заяц угрелась на груди Фонтанчика и дремала.

Я посмотрел на них.
Сколько я их знаю?
Фонтанчик был моим другом с ранней юности.
Мы вместе играли, учились и работали.
Позже к нам перевелась Заяц и как-то очень легко влилась в нашу компанию.

Я не знаю, когда Заяц и Фонтанчик стали любовниками.
Они скрывали это достаточно долго --- увы, но межвидовые связи многими не приветствовались.
Что могло сблизить человека и кани настолько, что они стали спать вместе?
Трудно сказать, но иногдая видел удивительную вещь --- они сидели рядом и переговаривались без слов.
Так было и сейчас --- между друзьями шёл неслышный для меня диалог.

Около двадцати лет Заяц и Фонтанчик даже жили в одном доме --- так называемое <<гнездование>>, пережиток времён, когда у тси ещё существовало культурное явление семьи.

--- Знаешь, Небо, --- вдруг заговорил Фонтанчик, --- насчёт глазастиков...
Многие спорили на тот счёт, гуманны ли эти эксперименты по отношению к живым существам.
Но ведь когда-то и наши с тобой предки были другими.

--- Да, --- согласился я.
--- Ты бы бегал по лесам в поисках мелких животных, а я опылял бы цветы.
И мы, если и встретились, то не смогли бы пообщаться.

Заяц открыла глаза и погладила Фонтанчику живот.

--- Когда-то собаки были домашними питомцами, а теперь мы равны.
Я бы умерла, если бы тебя не было.

Мы промолчали.

В траве лежала и тяжело дышала умирающая пчела.
Её брюшко судорожно сокращалось, пытаясь прогнать драгоценный воздух через трахеи; крылья подрагивали в последней инстинктивной попытке взлететь.
Едва ли пчела осознавала, что неспособность взлететь равносильна гибели.
За неё уже всё решила странная игра случайности и естественного отбора.

А где-то неподалёку жила своей жизнью её родная уютная борть.
Пчела всё отмеренное ей время отдала колонии, вылетела в очередной рейс --- и не вернулась.
Как и я.
Как и Заяц с Фонтанчиком.
Как и сотни миллиардов тси до нас --- безымянные и безликие.

Я аккуратно тронул росший рядом колокольчик, вытряхнув на пальцы немного коричневатой пыльцы, и опылил соседнее растение.

--- Ну вот, моё пчелиное предназначение можно считать выполненным, --- сказал я.

Фонтанчик задумался.

--- А мне, чтобы выполнить собачье, нужно поймать мелкое животное?
Хмм...

Фонтанчик со звериной ловкостью нырнул рукой в траву и вытащил за хвостик пищащую, перепуганную до смерти серую мышь.
%{squeaking gray mouse, half-dead of fear.}
Поднёс её к глазам и критически рассмотрел.
Мышь перестала биться и в страхе замерла, разглядывая выпученными глазами собачье лицо.

\ml{$0$}
{--- Отпусти её, --- укоризненно сказала Заяц.}
{``Let it be,'' Hare reproachfully said.}
\ml{$0$}
{--- У неё сейчас сердечный приступ случится.}
{``It's almost got a heart attack.''}

Фонтанчик опустил мышь в траву, и она едва слышно скользнула в заросли.

--- Наверное, нужно было её съесть, --- философски заметил Фонтанчик.
--- У меня как-то плоховато с инстинктами.
Хотя ладно.
Мы уже стали ближе к природе.
Посмотрите вокруг.

--- Мы и есть природа, --- поправил я друга.
Фонтанчик кивнул.

--- А моё предназначение? --- спросила Заяц.

--- А ты сиди, обезьянка, --- Фонтанчик крепче обнял женщину.
--- Если верить историкам, ты сама эволюционировала, к тебе претензий нет.

Мы рассмеялись.

Закат окончательно догорел.
Где-то вдали, с кажущихся чёрными пятнами скал, поднялись в небо пятнадцать огромных птичьих силуэтов, сделали величественный круг и унеслись на запад.
Мы с Фонтанчиком проводили их взглядом.

--- Надеюсь, у них всё будет хорошо, --- сказал я.

Фонтанчик промолчал, а затем весело рявкнул и, схватив Заяц в охапку, начал облизывать ей уши.
Заяц пискнула.

--- Ты что делаешь?
Эй, хватит!
Фонтанчик!
У тебя язык шершавый!

Фонтанчик поднялся на ноги и посадил Заяц на плечо.

--- Я, конечно, ни на кого не намекаю, но наш командир прохлаждается в ночной степи с неизвестными личностями и любуется закатом, когда у него полно дел, --- Фонтанчик схватил меня за шкирку и усадил на другое плечо.
От неожиданности я щёлкнул зубами.
--- Пойдём-ка, Существует-Хорошее-Небо.
Тебя ждёт твой народ.

--- Ну вот, я вся мокрая, --- засмеялась Заяц.
--- Собачьи нежности...

\section{[*] Гарант}

Вскоре ливень прошёл, и ненадолго выглянуло заходящее солнце --- первый знак, что боги благословили Середину Дождя.
Дети разбежались по домам, а Митхэ сидела у входа в святилище.
Акхсар и Эрхэ сидели с двух сторон, нежно обняв воительницу.

Митхэ тупо глядела в пространство и думала о том, что Ситрис не попался бы в такую простую ловушку.
Как обычно, он не послушал бы её приказа;
для разбойника существовал только один командир --- собственное чувство опасности.
Он перешагнул бы и через тридцать, и через сто детских трупов, и ни один из маленьких воинов даже не успел бы выговорить слово <<mama>>.

<<Митхэ, Митхэ, --- вспомнила воительница его полный сарказма невесёлый смех, --- какая разница, кто перед тобой, если он собирается выпустить погулять твои кишки?>>

Митхэ вдруг осознала, что сейчас убила бы детей в любых количествах, лишь бы вернуть любимого.
Её личное счастье зависело от случайно встреченного трусоватого разбойника.
И она позволила ему уйти...

Сидевший рядом Акхсар тихо, но отчётливо ругался в чей-то адрес.
Его мысли были о том же.

<<Как же тяжело быть командиром>>.

Тяжело ступая по мокрой утоптанной глине, к воинам шёл вождь Си-Абву.
За ним плёлся сын.
На лице мальчика замерло выражение обиды.

--- Отец, это было нечестно! --- вполголоса сказал он.
--- Это недостойно вождя!
Она просто хотела...

--- Ещё раз услышу от тебя подобное --- и ты вылетишь из моего дома быстрее ласточки, --- суровым шёпотом рявкнул Си-Абву.
--- Я уже всё тебе объяснил.
Пошёл вон.

Сын бросил яростный взгляд на отца и молча ушёл вниз к поселению.
Вождь встал перед воинами сели и, приосанившись, закутался в плащ.

\ml{$0$}
{--- Желаешь ли ты увидеть его тело?}
{``Would you see the body?''}

\ml{$0$}
{--- Нет, --- сказала Митхэ.}
{``No,'' \Mitchoe\ said.}
\ml{$0$}
{--- Похороните его так, как считаете нужным.}
{``Bury him as your tradition demands.''}

\ml{$0$}
{--- Мы съедаем тела посвящённых богам.}
{``Bodies of the sacrificed are to be eaten.''}

--- Да будет так, --- ответила Митхэ, опередив возмущение друзей.
Скрип зубов Эрхэ, казалось, был слышен в десяти шагах.

Вождь кивнул.

--- До нас дошли слухи, что в своих землях ты теперь вне закона, Митхэ ар’Кахр.
Моя \textit{w\o{}izh} умерла несколько дождей назад от болезни, и мне нужна новая.

--- Что значит это слово? --- равнодушно спросила Митхэ.

Вождь впервые смутился.

--- Это значит, что женщина клянётся в верности мужчине, они живут вместе...

--- Ты хочешь, чтобы мой клинок служил тебе?

--- Да, я хочу этого.
А ещё я хочу, чтобы ты была в постели только со мной и рожала детей только от меня.

Митхэ спокойно рассматривала смуглое лицо вождя.

--- Золото, мне его зарубить? --- будничным тоном поинтересовался Акхсар.

--- Лучше дай мне, я из его стержня сделаю свистульку и зашью в щеку его ублюдку, --- в тон ему добавила Эрхэ.
--- Он будет висеть на дереве на папашиных кишках и свистеть, пока не околеет от голода.

Вождь не повёл и бровью, услышав эту витиеватую угрозу.

--- Я жду твоего решения.

--- Я согласна, --- чётко проговорила Митхэ.
Акхсар зажмурился, Эрхэ прикрыла лицо рукой.
--- Ты первый мужчина, одержавший надо мной верх.
Я умею признавать поражение.
Я буду твоей... \textit{w\o{}izh}.

Вождь поклонился и пошёл вниз по тропе, к поселению, в котором уже разгорались праздничные костры и гулял народ.

--- Митхэ...

--- Больше кровопролития не будет, Снежок, --- прервала Акхсара воительница.
--- Мы своей кровью добыли мир... и, клянусь лесными духами, пока я жива, этот мир будет под моей защитой.

--- Это унижение! --- воскликнул Акхсар.
--- Он убил твоего мужчину, а теперь хочет, чтобы...

--- Да, это унижение, --- ответила Митхэ.
--- И вождь за него заплатит сполна, я клянусь.
Но народ хака здесь ни при чём.
Те женщины, которых мы видели, хотят быть со своими мужчинами.
Те мужчины, которых мы видели, хотят, чтобы по возвращении домой их встречали дети.
Всем до единого знакомы чувство дружбы и любви.

--- И всё же...

--- Если ты не хочешь идти со мной, друг, я тебя отпускаю.
Моя жизнь в моих руках, и сейчас мне совершенно плевать, как она закончится.

Акхсар обиженно крякнул и тут же попытался скрыть это за смущённым кашлем.

--- Твоя жизнь закончится достойно, Золото, --- горячо прошептала Эрхэ.
--- Ты будешь счастливее нас с Акхсаром.
Я верю в это.
Если же нет, значит, и справедливость --- лишь красивая сказка.

Митхэ не ответила.

--- Так ты уже всё решила? --- наконец осведомился Акхсар.

--- Да, Снежок, --- кивнула Митхэ.
--- Атрис будет гордиться мной в пристанище духов.

--- Тебе придётся спать с ним, --- тихо заметил воин.

--- Я потерплю.

--- У хака и сели не может быть детей.

--- Похоже, что они об этом не знают, как и о том, что женщины-сели способны рожать без мужского семени.
В крайнем случае у нас есть мужчина, --- Эрхэ лукаво подмигнула Акхсару.

--- Ну, Снежок, как тебе план? --- спросила Митхэ.
--- По-моему, сели и хака давно пора завязать отношения попрочнее.

Акхсар угрюмо посмотрел на женщин из-под седеющих бровей... и вдруг тепло улыбнулся тонкими губами.

\section{[-] Полководец}

\textspace

Я осмотрел молчаливые ряды сели.
Как убедить их следовать за мной?
Ветер трепал перья моей короны.
В памяти разом всплыли картины давних битв и походов --- полководцы на устрашающих машинах или лучших ездовых животных, красочные плакаты, кричащие знамёна, полные праведного гнева речи...
Но все пафосные слова улетучились, едва я взглянул на этих простых бесхитростных людей.
Кто-то смотрел на меня насмешливо, кто-то с тенью жалости --- я выглядел чересчур молодым для своей роли.
И только Чханэ, верная Чханэ, стоящая в первом ряду с нашим ребёнком на руках, ободряюще мне кивнула.

Вдруг моё внимание привлёк стоявший рядом с ней мужчина.
У него было тело строителя, привыкшего таскать тяжёлые камни и брёвна, покрытые мозолями руки.
Но, несмотря на внушительную физическую силу, его ноша была чересчур тяжела --- помимо шести мешков он нёс на руках двух детей.

Я сделал первое, что пришло мне в голову.
Я спрыгнул с оленя, подошёл к мужчине, снял с его плеч три тяжёлых, словно набитых камнями мешка и взвалил их себе на спину.
Сейчас я понимаю, как смешно выглядел в тот момент: роба моя перекосилась, её пола задралась выше пояса.
Налобный ремешок одного из мешков смял перьевую корону, а затем сорвал её с головы, и лёгкий головной убор, подхваченный ветром, улетел куда-то в кусты.
Но я старался об этом не думать.
Схватив оленя под уздцы, я молча пошёл по дороге.

И сели двинулись за мной.

\chapter{[-] Исход}

\section{[-] Союз с хака}

\textspace

Я оглянулся.
Передо мной стоял Имжу.
Я узнал его с трудом --- за прошедшие годы юнец с превратился в сильного, уверенного в себе мужчину, и глаза его, прежде неукротимые и яростные, светились холодным светом недюжинного ума.
Но самая главная перемена --- он перестал чернить кожу и красить ресницы, маскируясь под хака.
Ярко-зелёные глаза принадлежали моему брату --- теперь в этом уже никто не мог сомневаться.

Рядом, робко глядя на меня исподлобья, стояла Митхэ ар’Кахр.

--- Мои приветствия, Ликхмас ар’Люм, --- Имжу, сохраняя вежливую полуулыбку, заговорил на чистейшем цатроне.
--- Рад увидеть тебя снова.

Я вместо ответа подошёл и по очереди обнял гостей.
Имжу дёрнулся, но ответил на объятия со всей искренностью, которую позволяла ситуация.
Митхэ обняла меня крепко-крепко и тут же отстранилась.

Я пригласил Имжу и Митхэ за стол и предложил им еды.

--- Мы не голодны, Ликхмас, --- ответил Имжу.
--- К тому же вряд ли у нас есть время на застолье.

--- Я вижу, что ты стал вождём, --- сказал я, отметив число и цвет перьев в волосах.
Три красных, голубое и пучок фиолетовых перьев согхо --- вождь был избран во время изнурительного похода, когда его предшественника нашла гибель.
Осанка и уверенный, повелительный тон мужчины только подтвердили этот внушительный знак.

--- Меня выбрали боевым вождём племени Inh\o s-laka\FM.
\FA{
Спокойное озеро (хака).
}

--- Итак, я тебя слушаю.

Имжу начал рассказ.
По его словам, хака давно готовили новый поход на соседей.
Об этом редко говорили вслух, но очевидное скрыть сложно.
Однако время похода, его масштабы и, главное, направление удивили всех.

--- Что именно тебя смутило?

--- Старейшины единогласно приняли решение о походе на внеплановом совете, --- ответил Имжу.
--- Между тем более половины моих соплеменников были против войны с сели.
Именно поэтому я здесь.
Объясни мне, брат, почему против сели одновременно выступили все окрестные народы?
Мы слышали, что и пылерои пустынь, и идолы Живодёра пошли против вас.
Даже тенку вдруг вспомнили старые обиды --- слышны вести, что они всерьёз намерены отомстить и вернуть себе земли, потерянные во время Второй Приречной войны.
Все эти племена не могли договориться между собой.
Притязания тенку даже выглядят смешными --- Приречная война закончилась шестьсот дождей назад, а Водораздел формально принадлежит святилищу и подчиняется законам Весёлого Волока.

--- Ты не считаешь это случайностью?

Имжу хмыкнул, и на секунду в его спокойном облике вновь проступило юношеское высокомерие.

--- Только невежда посчитает это случайностью.

--- Ты пришёл за ответом?

--- Да, --- сказал Имжу.
--- И ради этого ответа я поставил под удар свою репутацию, приостановив диверсии хака на ваши отступающие силы.

--- Сели выступили против культа Безумного, --- сказал я.

Имжу присвистнул.

--- Странная реакция, --- заметил я.
--- Похоже, что ты не удивлён.

--- Разумеется, --- ответил молодой вождь.
--- На совете хака звучала речь об этом.
Старейшины сказали, что Безумный покарает сели, и завоевать их земли будет намного проще.
Но почему сели пошли на это самоубийство?

--- Потому что сели не считают это самоубийством, --- подал голос Грейсвольд.

Имжу холодно воззрился на технолога.

--- Объясни свои слова, человек.

--- Мы можем уничтожить Безумного, --- ответил я.
--- Так что, если хака вспомнит, сколько его родичей умерло на алтарях и сколько было убито сели, то может сделать правильный выбор.

Митхэ словно воды в рот набрала.

--- Это невозможно, --- проворчал Имжу.
--- Я вам не верю.

--- В таком случае мы проиграем, --- просто ответил Грейсвольд.
--- Хака достанутся восточные земли сели, и твой народ будет поклоняться Безумному, как и десять тысяч дождей до этого времени.
У вас будут охотничьи угодья и плодородные сады, ваши земли будут граничить с воинственными тенку и несговорчивыми идолами Живодёра, а сели будут уничтожены и канут в забвение.

--- Не самая приятная перспектива, --- заметил Имжу.
--- Многие хака мечтают завоевать земли сели, но мало кто задумывается о том, как оборонять эти земли от пятнадцати пылеройских кланов и идолов Живодёра.

--- Дружить всегда выгоднее, --- заключил Грейсвольд.

Имжу сидел и смотрел на меня проницательным взглядом.

--- Вы сами верите в свои слова, --- наконец сказал он.
Это был не вопрос, а утверждение.

--- Ты можешь нам помочь?

--- Я могу встать между вами и союзом племён, --- ответил Имжу.
--- Но это не продлится долго.
Мои соплеменники признают меня миролюбцем и лишат звания.
Мне очень повезёт, если я останусь в живых.

--- Но если останешься, то о тебе будут слагать легенды ещё десять тысяч дождей, --- пообещал Грейсвольд.
--- Ни один вождь хака пока не осмелился бросить вызов Безумному.

Имжу опустил голову.
Последние слова явно задели его честное, но честолюбивое сердце.

--- Есть ли недовольные среди старейшин? --- поинтересовался я.

--- Хватает, --- уклончиво ответил Имжу.
--- Решения исходят от небольшого замкнутого круга, члены которого хорошо мне известны.

Интеллект молодого вождя поражал.
Даже в племени дикарей гены великих учёных-тси расцвели в полную силу.
Я испытующе посмотрел на Митхэ --- по словам Кхотлам и Акхсара, она была сильна в бою, но отнюдь не в интригах.

--- Есть ли возможность их устранить?

--- Очень маленькая, --- Имжу ответил не раздумывая.
Похоже, он сам много над этим размышлял.
--- Даже если затея удастся, это восстановит народ против нас.
Место убитых займут новые, кто пока скрывается в тени.

Мы с Грейсвольдом переглянулись.
Разумеется, это был Картель --- этот стиль сложно не узнать.
<<Разделяй и властвуй>>.
<<Держи ресурсы в мусорной куче>>.

--- Возможно, мне стоит отправиться с Имжу и решить дело на месте? --- предложил Грейсвольд.

--- Можем ли мы тебе доверять? --- впервые подала голос Митхэ.

Грейсвольд усмехнулся своей крупнозубой лучистой улыбкой.
Я удивлялся тому, что технолог тратил много времени на получение последних обновлений по психологии улыбки, но после понял, насколько мощным оружием она была.
Отточенная до мельчайших деталей, максимально подогнанная под тип лица и при этом непринуждённая, улыбка Грейсвольда обезоруживала людей, и даже самые яростные противники технолога не могли ей противиться.

--- Твои предки ещё не покинули Тхидэ, когда я одерживал первые победы, Митхэ ар’Кахр.

--- Ты не ответил на её вопрос, --- заметил Имжу.

--- Верно, --- согласился Грейсвольд и вышел.

\section{[-] Опыт и знания}

Вскоре старейшины были устранены.
Грейсвольд применил весьма точный и изящный тактический приём.
Я не буду приводить его в книге, так как он до сих пор представляет нашу с Грейсвольдом тайну --- все причастные демоны Картеля уничтожены.

Грейсвольд приобрёл большой вес.
Я уже не был для Имжу неоспоримым авторитетом.
Стоило технологу взять слово --- и молодой вождь вместе со своими людьми сидели и слушали, открыв рты.
Мы с Анкарьяль только посмеивались.

--- Я скоро начну ревновать, --- пошутил я.

--- Эх, если бы все ценили опыт и знания, как эти дикари, --- со вздохом заметила Анкарьяль.
Я мысленно с ней согласился.

\textspace

\section{[-] Серебряный амулет}

--- Мой олень, Серебряный.
Он всегда знал верную дорогу.
Пусть его дух поможет тебе.

Митхэ протянула мне амулет, грубо вырезанный из лобной кости оленя.

\section{[-] Последний фрагмент книги}

--- Вот книга, котору ты просил, --- сказал ноа и протянул мне том.

Я пролистал книгу и усмехнулся.
Смешного в этом мало, но ноа тоже хранили фрагмент --- начало, немного из середины и самые последние записи.
Впрочем, с этой книгой картина произошедшего почти сложилась.
Почти.

Я поблагодарил человека и вернул ему книгу, пообещав зайти позже с бумагой и пером.

\section{[-] Разведка дипломата}

--- А ещё я обнаружила пятерых агентов Красного Картеля, --- доложила кормилица.

--- С чего ты решила, что это агенты? --- высокомерно скривилась Анкарьяль.

--- Они говорили на сектум-лингва.
И да, --- опережая вопрос, продолжила Кхотлам, --- я теперь умею отличать классический сектум-лингва от языка ноа.

Анкарьяль со всё возрастающим уважением смотрела на кормилицу.

--- Скажи их хасетрасем, и я лично рекомендую тебя к оцифровке, Кхотлам ар'Люм.

--- Не стоит, Анкарьяль Кровавый Шторм, --- улыбнулась купец.
--- Однако мне есть что сказать помимо хасетрасем...

\asterism

Кормилица действительно сообщила несколько важных деталей.
Презрение к сапиентам в очередной раз сыграло с Картелем злую шутку --- они говорили на сектум-лингва открыто, проверив местность лишь на присутствие демонов.
Некоторые термины были сказаны на языке Эй, таблица B44, но Кхотлам повторила их с механической точностью.

Во-первых, Картель оказался в курсе нашей численности --- <<их здесь трое или пятеро, не больше>>.
Во-вторых, враги знали о присутствии <<Падальщика>> (меня) --- постарался один из визоров, собравший полную картину буквально из пылинок информации.
В-третьих, мы получили имена некоторых присутствующих --- Гимел (биолог Гимел Кадмиевый Зелёный), Малингве (стратег Малингве Пустая Карта) и Монстр (старый позывной интерфектора Мост Ликующая Плазма, с высокой вероятностью намекающий на личность говорящего --- он состоял в Манипуле Смеха, одном из самых жестоких подразделений армии Картеля).

--- Кроме этого, они сказали вот это, --- закончила Кхотлам и проворковала фразу на неизвестном мне языке.
Грейсвольд испуганно вздрогнул, Анкарьяль прищурилась.

--- <<Это солнце чересчур яркое для меня, любовь моя.
Я хочу домой>>.
Язык шакуната времён Союза Воронёной Стали, планета Чёрная Скала.

Кхотлам наморщила лоб.

--- Это были мужчина и женщина.
Вокруг них постоянно царит странная, очень неприятная атмосфера --- они внушают сильный страх и желание спрятаться, уйти подальше.
Это чувствовала не только я, многие окружающие вели себя соответственно.
Даже от их голосов мороз по коже --- словно ветер в скалах или хрип умирающего.

--- Они выглядели как родственники? --- спросила Анкарьяль.

--- Очень близкие, --- кивнула Кхотлам.
--- Я бы даже решила, что это брат и сестра, ставшие любовниками.

--- Кто это? --- поинтересовался я.

--- Те, с кем лучше не встречаться, --- пробормотал Грейсвольд.
--- Демоны Чёрной Скалы были стержнем армии Союза.
У них особая, не имеющая аналогов философия, которая, к сожалению, оказала на Картель большое влияние.
При этом даже союзникам из числа Картеля с ними очень сложно общаться.
Там, где звучал шакуната, царил ужас.

--- В живых их осталось всего двадцать два, --- добавила Анкарьяль.
--- Двойняшки среди них самые известные, они тоже сейчас в Манипуле Смеха.

Так мы узнали о присутствии легатов Фуси Абрикосовый Посох и Нуива Пустая Тыква.

Анкарьяль, узнав о присутствии на захолустном Тра-Ренкхале как минимум троих интерфекторов Манипулы Смеха, извинилась и покинула палатку.
Грейсвольда начало жестоко рвать и лихорадить уже ночью, я же заранее принял песочник ползучий и отделался небольшим головокружением.
Реакция Стлока плюс сбои в работе демонов --- я вообще сомневаюсь, что кому-то из агентов Ада за последний телльн приходилось переживать такое запредельное напряжение.

\section{[-] Страусы}

Невдалеке показалось гнездовье шипастых страусов.
Увидев меня, самец зашипел и распустил смертоносные перья.
Однако вскоре, заметив в облаке пыли бесшумно идущую армию сели, страус предпочёл быстро ретироваться вместе с выводком.

\section{[-] Триумфальное шествие}

Исход был похож на триумфальное шествие.
Люди пели песни и смеялись, несмотря на вспыхивающие то тут, то там стычки с диверсантами.
Мёртвых никто не хоронил и не оплакивал.

Вот какой-то воин властно поднял женщину, севшую оплакать сестру:

--- Пойдём!
Ты скоро сама с ней встретишься.

--- Если я её не оплачу, сестра не достигнет пристанища! --- запротестовала женщина.

--- Путь к пристанищу всегда был прост и прям --- смерть, --- веско сказал воин.
--- Она уже там, и ей нужен лишь покой.
А мне нужны голоса для дорожной песни.

Воин запел.
Его собеседница, выхватив окровавленную фалангу, подхватила хриплым яростным сопрано, от которого меня по коже продрал мороз.
Они двинулись по дороге, держась за руки.

В этот момент я отстранённо понял, что я им уже не нужен.
Им не нужен Король-жрец.
Они сами прекрасно знают, что делать, и делают это.

\section{[-] Свобода}

\textspace

--- И что теперь получается? --- скривившись, произнёс Митракх.
--- Мы одного хозяина --- Безумного, да охранят нас лесные духи от гнева Его, --- поменяем на других?
Вас?
А как же свобода?
Ликхмас-тари, ты говорил, что мы будем свободны!

Я хотел ответить, но меня опередил старик Саритр:

--- Митракх, --- прохрипел он и закашлялся, сложившись вдвое.
Пара молодых воительниц поспешили к нему и подхватили пожилого крестьянина под руки.

--- Храни вас лесные духи, дочки. --- Саритр выплюнул на раскалённый камень кровавый комок и утёр рот рукавом, глядя на подбоченившегося молодого кузнеца.
--- Проклятая пустынная пыль...
Да, да, я стою, обереги вас.
Митракх, дитя, над тобой всю жизнь властвовал Совет.
Сейчас нас в бой ведёт вождь.
А ещё тобой безраздельно правит твоя женщина, --- старик прочистил горло и кивнул на стоявшую неподалёку крестьянку.
Та зарделась.

Митракх стушевался, глядя то на старика, то на смущённо теребящую эфес фаланги женщину, но тут же обрёл прежний надменный вид.

--- Ты всю жизнь кому-то служил, дитя, как и я.
Какой свободы ты хочешь?

--- Вождя выбрали на совете, --- вызывающе произнёс Митракх.
--- Выбрали самого достойного, того, кто думал о благе племени больше, чем о еде на завтрашний день.
А женщину я выбрал сам --- я знал, что она меня любит.

--- Правильно, парень.
Почему мы служили Безумному?
Скольких детей он забрал у тебя?
Пятерых?

Митракх стоял, опустив голову.
Его женщина прикусила губу.

--- Будь забыто Его имя, --- хрипло произнесла она.

--- Посмотри на него, Митракх, посмотри, --- старик кивнул на меня.
--- Он не такой, как Безумный.
Это видно сразу.
Да, идолы меня разгрызи, если бы даже молодой Ликхмас попросил одного ребёнка из тех восьми, кого я лично за руку отвёл... --- Саритр опять зашёлся надрывным кашлем.
На этот раз его подхватили мы с Митракхом.

--- Ох, обереги вас духи...
Недолго мне осталось, дети.
Совсем недолго, --- просипел старый крестьянин, улыбаясь измученной улыбкой.
--- Хотелось бы погулять с вами на празднике после битвы, но я её не переживу.
Надеюсь, хоть кого-то из врагов с собой утащу...

Митракх не ответил.
Он застывшим взглядом смотрел на обвисшего у нас на руках Саритра.
В его глазах медленно зрело понимание...

--- Не волнуйся, Ликхмас ар’Люм, --- наконец пробормотал он, старательно избегая моего взгляда.
--- Отпусти его, он стоит.
Я отведу дедушку к воде, ему станет легче.
Идём, Саритр-лехэ, идём...

\section{[-] Подмога}

\textspace

<<Неужели мне придётся сражаться с ними в одиночку?>> --- мелькнуло у меня в голове.

Однако спустя две михнет со стороны моря раздался протяжный свист.

--- Корабли! Корабли! --- закричали сели.

Я поскакал к морю.
Могильный пролив пестрел парусами --- зелёными, красными, синими, белыми.
Родовые знаки ноа, клановые гербы трами.
Вода, рассекаемая штевнями, вёслами и натянутыми фалинями, кипела от бьющихся хвостов.

--- Они запрягли дельфинов в корабли? --- вдруг недоумённо спросила женщина рядом со мной.

Она оказалась права.
Дул бейдевинд, не позволявший развить большую скорость.
Дельфины притащили огромный флот Кита к Могильному берегу меньше чем за кхамит --- аккурат к сражению.

Анкарьяль соскочила с первой же лодки и бросилась ко мне через толпу ноа, которые с радостными криками бросились обнимать соплеменников.

--- Соскучился? --- ухмыльнулась она.
--- Пойдём на совет.
Грейс сейчас будет, приведёт старейшин.

--- Но как?

--- А, ты про дельфинов.
Они сами предложили помощь, представляешь?
В Коралловую бухту приплыло более шестнадцати тысяч --- чуть ли не две трети племён Кипящего моря.
По расчётам Картеля флот должен был прибыть после того, как основные силы сели будут разгромлены.
Мы им поломали очередной план.
Пойдём, пойдём.
У нас есть десять михнет до того, как нужно будет послать парламентёров.

--- Ты думаешь, я смогу с ними договориться?

--- Вероятность мала, --- пожала плечами подруга, --- но переговоры --- это лишние михнет для доработки боевого строя.
Хотя... вдруг тебе повезёт.
Я уже даже не удивлюсь.

\section{[-] Причина промедления}

\textspace

--- Почему он не атакует нас сейчас? --- задумался я.
--- Что его останавливает?

--- Я знаю Эйраки, --- ответил Грейс.
--- Он самоуверен и скуп до ужаса.
Он хочет покончить со всеми --- с нами, сели и прочими восставшими тси --- одним ударом.
Раньше я сомневался, действительно ли это он Безумный, но сейчас сомнений больше нет.

--- Хорошо, --- согласился я, --- тогда что останавливает Картель?
Они-то не дураки?

--- С этим сложнее, --- признал Грейсвольд.
--- Во-первых, неясно, почему Картель терпит присутствие Эйраки.
Они надеются использовать его как источник энергии?
Возможно, но для источника энергии он чересчур сильно вмешивается в их стратегию.
Картель славится привычкой растаскивать кости по одной прежде, чем их сгребут в кучу.

--- А диверсии?

--- Будь воля Картеля, Аркадиу, диверсий было бы столько, что от Кахрахана мы шли бы в гордом одиночестве.
Эйраки их сдерживает, это безусловно.

--- Они надеются использовать его как ударную силу? --- предположил я.

--- Вполне правдоподобно, если не знать, что агентов Ада здесь трое, и у двоих нет энергии на обратную дорогу.
Об этом Картель уже осведомлён.
Встаёт вопрос: против кого они собираются использовать Эйраки?

--- С чего ты взял, что его хотят использовать против кого-то конкретного, а не как бесплатную охрану? --- скептически осведомилась Анкарьяль.
--- Если мы всё-таки смогли бы послать сообщение своим...

--- ...то никто бы не пришёл, --- закончил Грейсвольд.
--- Стратеги Ада не станут посылать силы на верное самоубийство непонятно ради чего, и Картель это знает.

Мы переглянулись.

--- История тёмная, ребята, --- пробормотал Грейсвольд, --- однако у меня стойкое ощущение: Картель чувствует угрозу, и угроза исходит не от нас.
Именно поэтому Эйраки ещё жив.
Разумеется, Картель будет использовать демиурга и против нас, и против наших сапиентов, которые благодаря их размолвкам превратились в серьёзную силу...

--- Серьёзную? --- хохотнула Анкарьяль.

--- Нар! --- внушительно буркнул Грейсвольд.
--- Взгляни правде в глаза --- у нас в руках сейчас сила, которая способна попортить кровь даже отожравшемуся до ка'нетовской константы демиургу, не говоря уже о местном ковене Картеля.
Если против нас Эйраки мог применить грубый поток масс-энергии, то сели на него плевали, они даже ничего не почувствуют.
Потребуются глубокие знания в технологиях массового поражения и траты масс-энергии, чтобы хоть что-то с ними сделать!

--- А МПДЛ? --- спросил я.

--- Химическое оружие против полумиллиона сапиентов, рассеянных по всему Могильному берегу?
Извините, не так --- против четырёхсот тысяч тси, которые осведомлены о природе <<божественной кары>> и в любой момент могут реализовать протокол <<Кристалл>>?
Это даже не смешно, он потратит на одну атаку половину нажранной масс-энергии.
Тем более МПДЛ --- скорее деморализующее средство.

--- Отравленный дождь?

--- Пусть Картель для начала попробует вызвать обычный дождь в этих широтах.
Потом поговорим об отравленном.

--- Антарида?

--- Мы в пустыне, на берегу моря, --- сказала Анкарьяль.
--- Антарида очень быстро сдохнет от ветра, голода и низкой температуры, не принеся сколь-нибудь значительного ущерба.

--- Ещё биология?
Вирусы?

--- Силы рассредоточены, Аркадиу.
Химия и биология здесь совершенно бессильны.
Кроме того, биология тси до недавнего времени не изучалась вообще никем.

--- Тектонические сдвиги?

--- Судя по непрекращающимся землетрясениям, для Эйраки что-то раскачать так же сложно, как и что-то успокоить.
Для демиурга он плоховато знает свою планету.

--- Оружие массового поражения наверняка есть у Картеля, они никогда...

--- Сомневаюсь, что у них есть что-то серьёзнее P-гранат, --- перебил меня Грейсвольд.
--- Для того, чтобы держать в повиновении дикарей, этого достаточно.

--- Ядерное оружие.

--- Чтобы выпустить терракотового волка без подготовки, нужны технологии Картеля и серьёзные затраты масс-энергии со стороны Эйраки.
То есть опять всё упирается в их кооперацию.
Наши шансы высоки, пока голова, руки и палка принадлежат разным людям.
Кстати, а что насчёт тех осцилляций, которые...

--- Я уточнила наши шансы, --- перебила его Анкарьяль.
--- Один к двумстам тридцати.
При том, что, по твоим словам, всё сложилось для нас невероятно удачно.

--- Ну, двести тридцать --- это уже не десять тысяч, согласись, --- заметил технолог.

--- Чтоб тебе с таким шансом ложка в рот попадала, --- съязвила Анкарьяль.
Она всегда становилась очень раздражительной перед серьёзным боем --- между мозгом и хоргетом шёл интенсивный обмен информацией, каждый навык проверялся и перепроверялся, и мозг на вмешательство в его дела реагировал испорченным настроением.

--- Подожди, Нар, --- махнул я рукой.
--- Грейс, ты что-то начал говорить про осцилляции.

--- Да? --- удивился технолог и задумался.
--- Лесные духи, уже забыл.
Наверное, что-то несущественное.

\section{[-] Стрелохвосты}

У кромки берега мы с Чханэ заметили чьё-то верещащее тельце.

--- Ребёнок, --- ахнула Чханэ и бросилась на помощь.

Подруга оказалась почти права.
Стрелохвост-подросток, из любопытства подплывший чересчур близко к берегу, оказался зажатым в скалистой расщелине.
Увидев нас, он испуганно заверещал и забился ещё сильнее.

--- Нянечка-нянечка, --- ласково заговорила Чханэ.
--- Не шевелись, я тебя вытащу.

Мы аккуратно повернули дельфина и освободили зажатый хвостовой плавник.
Стрелохвост бросился наутёк, но шагах в двадцати вдруг повернулся к нам и сделал великолепный тройной пируэт.

--- Это он благодарен, --- объяснил я.

Чханэ фыркнула.

--- И так ясно.

--- Ничего не ясно, --- возразил я.
--- Может быть, он чего-то испугался в глубине.

--- Он бы подпрыгнул и ударил хвостом по воде, --- тоном учителя объяснила Чханэ.

--- Ты раньше общалась со стрелохвостами?

--- Я их впервые вижу, но это очевидно.
Всё, Лис, не докучай, нам нужно догнать нашу колонну.

Вот и ещё один любопытный факт из психофизиологии тси, требовавший объяснения.
Я и раньше замечал, что человеческие дети безошибочно угадывали невербальные сигналы апид, достаточно далёкой в эволюционном плане Ветви.
А Чханэ знала, как выражают эмоции стрелохвосты, не видев их ни разу в жизни.

Через полдня пути у кромки берега начали появляться другие стрелохвосты.
Вначале десяток, потом два. Они перекликались, и в их голосах звучало удивление.
Сели, которым редко случалось увидеть стрелохвостов, сбежались на берег и начали размахивать руками.
Вскоре некоторые, в особенности знающие языки купцы, начали различать в гомоне стрелохвостов знакомые слова.

Купчиха Ситхэ, идущая рядом с нами, долго вслушивалась, а потом вдруг ахнула:

--- Я начала понимать их язык, надо поговорить!

Ситхэ забежала в воду и махнула остальным.
Сели умолкли.
Дельфины, как ни странно, поняли, что с ними хотят поговорить, и подплыли поближе.

--- Хай! --- закричала Ситхэ.
--- Мы знаем, как победить богов!
Безумного!
Нам нужна ваша помощь!

--- \emph{Помощь? Помощь?} --- загомонили дельфины, разобрав знакомое слово.
Вперёд выплыл белый как снег крупный самец.

--- \emph{Помощь почему?} --- вдруг спросил он.

--- \emph{Помощь почему? Помощь почему?} --- заверещали дельфины.

--- Безумный бог! --- крикнула Ситхэ, изо всех сил пытаясь подражать непривычному говору.

--- \emph{Помощь почему?} --- снова спросил белый стрелохвост.

--- Убивать --- мучить --- нельзя видеть! --- крикнула Ситхэ и для наглядности указала пальцем в небо.

Стрелохвосты вдруг замолчали, и все сели почувствовали --- они поняли и осознали последнюю фразу.
Дельфины всегда были гораздо более свободолюбивыми, нежели сухопутные сапиенты.
Прежде они никогда не знали рабства, единственной непреодолимой границей для них был берег.
Они не приняли ультиматум Безумных --- и боги в ответ подвергли морских обитателей геноциду.

--- \emph{Это нельзя,} --- вдруг сказал белый стрелохвост.

--- \emph{Нельзя,} --- как-то неуверенно согласились остальные дельфины.

--- Мы можем! Нужна помощь! --- крикнул кто-то из рядов сели, сообразив, какие слова стрелохвосты легко понимают.
Остальные подхватили его слова:

--- Мы можем!

--- Нужна помощь!

Белый дельфин отвернулся и поплыл к ожидающим его соплеменникам.
Вскоре они молча ушли в сияющие под вечерним солнцем воды.

--- Струсили, --- разочарованно крякнул какой-то старик.

--- Знаешь, Хонхо-лехэ, я их не виню, --- многозначительно ответила женщина со стигмами беременности и небольшим животиком.

Сели, удручённо переговариваясь, отправились дальше.

Наутро нас ждал сюрприз.
Прибрежные воды кипели от бьющих хвостов.
Дельфины плыли параллельно идущим по берегу сели и одобрительно свистели.
Я насчитал как минимум полторы тысячи особей.

Сели приободрились.
Отступление медленно превращалось в победный марш.
Моё сердце запело.
Оставалось взять последний аккорд --- вдали показались жёлтые башни крепостей ноа.
Feci quod potui.

\section{[-] Тайна имён}

\textspace

--- Это правда? --- резко спросил Сайрулай.
--- Вы говорили с дельфинами?

--- Да, --- раздался женский голос, и вперёд вышла купчиха Ситхэ.
--- Я говорила с океаническим народом.

--- Моряки ноа умеют общаться с дельфинами, но, по словам очевидцев, сели говорили с дельфинами на их языке, --- продолжал Сайрулай.

--- Наши языки очень сильно похожи, --- поклонилась Ситхэ.

По залу прокатился вздох потрясения.

--- Сели --- родичи дельфинов? --- выкрикнул кто-то.
--- Вы вышли из моря?

Я поднял руку.

--- Прошу слова, сенвиор Сайрулай.

--- Говори, --- кивнул ноа.

--- То, что видели твои люди --- правда.
Это имеет под собой причину, о которой почти забыли сели и давно забыли вы.

Я помолчал. Сайрулай прищурился.

--- Все мы --- сели, ноа, дельфины, идолы, травники, народ трами и ркхве-хор --- пришли с планеты Тхидэ.
Когда-то мы были единым народом, ели одну пищу и говорили на одном языке.
Я могу рассказать причину, по которой языки сели и ноа отличаются больше, чем языки сели и дельфинов.
Более того, я могу назвать имена тех, кто придумал ваши обычаи и ритуалы.

Я поднял над головой книгу Существует-Хорошее-Небо и вкратце рассказал о плане тси.
В зале воцарилась мёртвая тишина.

--- Если вы мне не верите, задумайтесь о том, почему сели и ноа могут иметь детей, а браки ноа с тенку, хака и зизоце обречены на бесплодие, --- закончил я.
--- Если вы мне не верите, посмотрите на лорику на груди ваших воинов.
Ваша кожа принимает кожу сели и идолов, как родную, но отторгает кожу прочих народов.
Если вы мне не верите, послушайте свои тела --- они помнят давнее родство.

Зал по-прежнему молчал, и я продолжил:

--- Когда-то давно наши предки, тси, говорили на равных с богами этого мира.
Они не обладали божественной силой, вы будете даже покрепче и половчее их.
Но силы, подобные Безумному, трепетали, едва заслышав их имена.
Трёхсложные имена, которые до сих пор носят сели;
<<тайные прозвища>> идолов Живодёра, состоящие из трёх слов;
ваше, ноа, <<имя-ключ>>, который получает каждый ребёнок и которое хранит в тайне всю жизнь, состоит из тех же трёх слов.

Я подошёл к Сайрулаю.

--- Как тебя зовут, тси?

Сайрулай в смятении смотрел на меня.
Он понял, что я от него хочу.
Он потоптался на месте и оглянулся на сидящих в зале.

--- Сайрулай, не надо!
Не делай этого! --- раздался из толпы женский голос.
--- Они заберут воду из твоих глаз и воздух из твоих лёгких!

Сайрулай судорожно выдохнул и вдруг в упор посмотрел на меня.
На смуглом лице расцвела улыбка.

--- Лёд-Сковавший-Траву, --- полушёпотом сказал он.

--- Играющая-Камнем-Лиса, --- громко представился я и протянул ему руку.
--- Жизнь и процветание, друг.

\section{[-] Вместе}

\textspace

В первый день сели и ноа сторонились друг друга и упрямо не хотели идти вместе.
На второй день они уже ели из одних котлов, но по-прежнему не желали беседовать.
На третий день отличить в толпе сели и ноа уже было невозможно --- мне попадались развесёлые отряды людей непонятной этнической принадлежности, выряженных как попало и гомонящих на смеси сели, ноа-линга и отвратительного цатрона.

---  Ну что ж, культурную реинтеграцию можно считать состоявшейся, --- невозмутимо резюмировал Грейсвольд.

---  Как-то она чересчур быстро произошла, --- недоверчиво буркнула Анкарьяль.
--- Так не бывает.

--- Возможно, создатели культур предусмотрели на этот счёт какие-то лазейки, --- предположил я.
--- Даже не возможно, а скорее всего.
Языки друг друга они знали прекрасно, но говорить считали ниже своего достоинства.

--- Одну из лазеек мы и использовали, --- заметил Грейсвольд.
--- Имена.

--- Тогда почему они не подружились раньше? --- скептически спросила Анкарьяль.

--- Не было повода, --- буркнул Грейсвольд.

Анкарьяль покачала головой и ухмыльнулась.

--- Дьявол.
Хотела бы я познакомиться с этими Баночкой и Кошкой.
Заставить два народа вести тысячелетнюю вражду, чтобы при надобности они безо всякого труда объединились --- прямо перед носом ошеломлённого врага.

\section{[-] Поющие дюны}

Солнце ещё не поднялось высоко, и мы продолжали идти.
Недалеко рычал грот, в котором вдребезги разбивались огромные, как горы, волны.
Вдруг пустынный берег огласил утробный гул, похожий на рёв пылеройской трубы.

Моя армия тут же легла, зарывшись в песок.
Арбалетчики не спеша зарядили свои арбалеты и взяли направление звука.
Спустя пять секхар раздался громкий хохот.
Ноа, которые даже не подумали двинуться с места, всё это время с недоумением наблюдали за действиями сели.

--- Король-жрец, --- ко мне подошёл смуглый, как обсидиан, воин.
Его зубы сияли, как жемчуг.
--- Вели своим людям подняться, опасности нет.

--- Что это? --- спросил я.

--- Дюны поют.

Я всё ещё колебался.

--- Труби, Король-жрец.
Нет опасности.
Могильный берег --- певчая пташка.

Я протрубил <<отбой>>, и мой сигнал повторили ещё несколько раковин.
Ноа, смеясь, хлопали сели по плечам и говорили, что всё хорошо.

Утробный гул повторился снова.
И вдруг ноа запели ему в унисон.
По всему берегу раздался грохот знаменитых боевых барабанов пустыни.

<<Чересчур шумно>>, --- подумал я и тут же одёрнул себя.
Эти люди знали лучше меня, как поднять свой дух перед боем.
Да и вряд ли бы ноа послушали мой совет касательно шума.

\section{[-] Встреча с Птичкой}

\textspace

--- Секхар! --- вдруг пискнул кто-то, и на шее у Грейса повисла счастливая Тхартху.

--- Птичка! --- изумлённо буркнул он.
--- Почему ты не ушла на Север?

--- Меня мучила совесть, что я бросила вас тогда, --- смутилась женщина.
--- Поэтому я здесь.

--- Разумеется, ты не собираешься сражаться?

--- Разумеется, собираюсь, --- обиделась Тхартху.
--- У меня есть дети и мужчина, я не хочу их опозорить.
В случае поражения нас и так всех уничтожат.
Да, я уже слышала, какие лепёшки вы напекли в своих походах.
Об этом только и говорят.

--- Тхартху! --- сурово крикнул сзади низкий женский голос.
Тхартху снова пискнула и бросилась обнимать Анкарьяль.

--- Ну-ка брысь отсюда! --- без предисловий бросила воительница, оторвав от себя Тхартху за шкирку, как котёнка.
--- Тут сейчас кровавый шторм будет!

--- Нар, а я все эти годы тренировалась, рисовала! --- Тхартху чисто по-женски проигнорировала последние слова подруги.
--- Хочешь, тебя нарисую?
Ты красиво смотришься в одежде ноа.

--- Она рисует, как сам Митр не умеет, --- поддержал её неожиданно подошедший мужчина.
--- Благодаря ей мы живём в хорошем достатке.
Она расписывает храмовые книги и свои делает, для детей...

--- Хай, хватит, --- вдруг смутилась Тхартху.
--- Не надо про мои книги.

--- Вот, --- мужчина, не обращая на неё внимания, гордо вытащил из-за пазухи тоненькую книжечку сказок.

<<Приключения Секхара, звёздного странника>>.

--- Эээ... --- только и смог протянуть Грейсвольд.

--- Дети их до дыр затирают, --- радостно сообщил мужчина.
--- Фантазии ей не занимать...

--- Митрис! --- многозначительно сказала Тхартху.
--- Подожди меня, я должна с ними поговорить.

Мужчина довольно осклабился и ретировался с книжкой в руках.

Мы помолчали.
Наконец Тхартху по очереди нас обняла.

--- Рада вас видеть.
Я буду молиться лесным духам, чтобы завтра они, вопреки своей природе, сражались с нами бок о бок.

--- Птичка, держись подальше от фронта, --- попросил Грейсвольд.

Тхартху улыбнулась технологу и лёгкой гибкой тенью исчезла в толпе.

\section{[-] Собачья песня}

\textspace

Анкарьяль вместо ответа запела песню на языке Хргада, развитой цивилизации кани на одной далёкой планете:

\begin{verse}
И дочери в шаг [один миг] лишились отцов,\\
И сыновья в шаг лишились матерей,\\
Щенята [дети] плакали и пели,\\
И кровь была им горька, как полынь-трава,\\
И плоть была им подарком перед [временем] голода и смерти\FM.\\
\end{verse}
\FA{
У кани Хргада была традиция поедать мёртвых после битвы.
Обычно в пищу употреблялись враги, в песне же говорится о поражении --- щенятам пришлось есть собственных родителей, чтобы не умереть с голоду.
}

Звуки песни, напоминавшие поскуливание и вой, заставили вздрогнуть стоявших вблизи воинов.
Когда-то, ещё на заре цивилизации, люди пытались обучить кани музыке.
Собаки прекрасно усвоили эту неотъемлемую часть человеческой культуры, но для себя стали сочинять нечто особенное, непривычное человеческому уху.
Впрочем, для тех, кто знал язык, песни кани были вполне понятными --- про любовь и славу, радость и скорбь.
Да и чувства юмора собаки не были лишены.

\section{[-] Мысли о броненосце}

\textspace

--- Итак, --- начал я.
--- Первое, что вы должны запомнить.
Для демонов не являются препятствием ни ваши доспехи, ни ваше оружие, ни ваши тела.
Если он захочет убить именно вас, он сделает это мгновенно или почти мгновенно, и ничто не сможет ему помешать.
Демон в живом теле имеет слабое место --- он интимно связан со своим носителем, тело составляет его важную, но не обязательную часть.
Вы можете убить носителя и на несколько секхар и даже михнет вывести демона из строя --- он будет дезориентирован и не сможет сражаться.
Многие демоны вообще не запрограммированы на серьёзную деятельность без тела --- то есть, убив тело, вы обезвредите демона надолго.
Но саму сущность не под силу убить никому из вас.

--- Даже одну? --- задал кто-то вопрос.

--- Даже одну.

Окружающие переглянулись.

--- Тогда как мы можем их победить? --- возмущённо пробубнил седой как лунь старик с выпученными глазами.

--- Если говорить предельно просто --- вашими чувствами.
Любовью, счастьем, нежностью, ощущением здоровья, благополучия.
Яростью, упоением.
Любыми светлыми сильными чувствами, которые вы можете испытывать.

--- Как чувства помогут их одолеть? --- подняла руку молодая воительница слева.

--- Хоргеты... эээ... как бы состоят из чувств.
Это их пища, их суть.
Те, против которых мы сражаемся, состоят из ненависти, страха и бессилия.
Чувства, которые испытывает каждый из вас, чересчур слабы, чтобы уничтожить даже одного из них.
Однако, согласно нашим данным, у них недостаточно визоров, и наше воинство способно сильно помешать демонам и даже полностью дезориентировать их.
Они будут словно летучие мыши в ярко освещённой комнате --- биться о стены и ломать себе крылья.
Тогда они станут уязвимы, и мы с Анкарьяль... Хатлам ар’Мар сможем их уничтожить.
Мы знаем, как это делать.

--- Мне не нравится, что целый народ должен полагаться на трёх якобы демонов, --- пробормотал сидевший в углу мужчина.
Остальные зашептались, выражая согласие.

--- К сожалению, плана лучше я предложить не могу, --- развёл я руками.
--- Безумный даже без Картеля чересчур силён, а у нас недостаточно энергии, чтобы связаться с союзниками и позвать подмогу.
Да, мы разные, но наши судьбы связаны: вы --- наша единственная надежда на выживание, а мы --- ваша единственная возможность победить.
Разумеется, кто хочет, тот может уйти перед битвой.
Каждый ушедший снизит наш шанс на победу.

Неожиданно для меня последняя фраза оказала магическое действие --- шёпот прекратился.

--- Лисёнок, ты так и не сказал, что мы должны делать, --- подала голос Кхотлам.

--- Как бы это ни звучало --- нужно сражаться, испытывая любовь, сострадание и нежность.

--- Малыш, я тебя не понимаю.

--- Ну как же, кормилица, --- укоризненно сказал я.
--- Мне ли рассказывать тебе, как идти в бой с любовью в голове?

Кхотлам смотрела на меня.
Остальные смотрели на Кхотлам.

--- Думай о вкусном броненосце, --- скопировав её интонации, произнёс я.

Кормилица с нежностью посмотрела на меня, потом повернулась к застывшим в непонимании людям.

--- Сейчас объясню.

\section{[-] Переговоры}

\textspace

\ml{$0$}
{--- Лусафейру, --- ухмыльнулся Эйраки.}
{``Lusafejru,'' Ejraci smirked.}
\ml{$0$}
{--- Непутёвый сын и здесь не может оставить старого отца в покое.}
{``The prodigal son refuses to leave his old man in peace, even here.''}

\ml{$0$}
{--- Безусловно, история твоей отцовской любви весьма трогательна, --- поморщился я, --- но я здесь не за этим.}
{``Certainly, the parable of your paternal love is very touching,'' I've made a wry face, ``but I'm not here for that.}
Я предлагаю Эйраки Морозу, бывшему максиму прима Ордена Преисподней, вернуться в созданную им фракцию.
Солдатам Картеля я предлагаю отступить в соответствии с...

Мои слова прервал громкий хохот.
Даже Эйраки не смог удержаться от тонкой, похожей на обезьяний оскал улыбки.

--- В моей новой фракции, --- Эйраки изящно оттенил эти слова движением бровей, --- есть такая... хмм... поговорка.
<<Хозяин не предлагает сдаваться, хозяин принимает капитуляцию>>.

--- Автор этого изречения не мудр, --- заключил я.

--- Сколько вас, предатель Люпино? --- осведомился один из командиров Картеля.
--- Двое?
Четверо?
Во сколько раз уменьшится войско Ада, если я тебя сейчас убью?

--- Вы устроили великолепный спектакль, Аркадиу, но не более того, --- сказал Эйраки.
--- Передайте этим дикарям: пусть расходятся по домам, если не желают их потерять.
Мне нужна биомасса.
Как после этого будете спасать свои жизни вы --- не моё дело, но это однозначно ваш шанс.

Демоны снова расхохотались.

Вдруг сзади застучали копыта и закричал олень.
Я обернулся.
Кормилица скакала к нам с непередаваемой грацией --- я даже не знал, что она умелый наездник.

Кхотлам подъехала ближе и осадила оленя.
Оценивающий взгляд пробежался по врагам, тут же выделив главного --- Эйраки.

--- Это ещё кто? --- осведомился Эйраки.

--- Эйраки Марос, называемый также Безумным, --- слова сектум-лингва Кхотлам произнесла настолько чисто, с характерной презрительной интонацией, что я вздрогнул.
--- Я, Испачканное-Мёдом-Перо, принесла тебе сообщение.
Ты терроризировал моих людей на протяжении десяти тысяч дождей, и мы ничем не могли тебе ответить.
Но рано или поздно любому террору приходит конец.
Именем сели... нет, \emph{именем тси} я обвиняю тебя в Разрушении.

На лицах Эйраки и агентов Картеля застыло изумление.
Изумление и страх.
Название древнего народа, который на равных говорил с Адом и Картелем, ещё жило в памяти демонов.

--- Согласно законам нашего народа, за подобное преступление казнят.
Но я хочу дать тебе шанс и последовать путям великих предков, о которых поведал мне сын.
<<Ушедший неподсуден>>\FM.
\FA{
Eksita not ivrisdikta --- общий для многих миров закон, являющийся следствием 1 закона Socia definicia --- <<право не принадлежать к обществу стоит выше законов общества>>.
Общий смысл: вышедший из общества не несёт перед обществом обязательств.
В период разработки универсального морального кодекса использовался народом тси;
однако, в отличие от прочих народов, тси распространили закон и на семью, что фактически привело к уничтожению семьи как института.
}
Если ты добровольно уйдёшь навсегда --- никто не станет тебе мстить.

--- А если...

--- Смерть, --- бросила Кхотлам.
Слово рассекло воздух ударом сабли.
--- Прочее --- смерть.

Эйраки рассмеялся.
На его лице медленно, но неотвратимо проступала звериная ярость.

--- Передай своим, Испачканное-Мёдом-Перо, --- прорычал Эйраки наконец.
--- Я сотру в пыль ваши города.
Я превращу в пепел ваши книги.
Я уничтожу саму память о планете Тси-Ди и микробах, которые возомнили себя её хозяевами.

Кормилица спокойно хмыкнула и развернула оленя.

--- Пойдём, Лисёнок, --- бросила она мне.
--- Больше нам говорить не о чем.

\section{[-] Закрытая дверь}

\textbf{<Люди поют колыбельную, идя на бой>}

Даже их религия была направлена на то, чтобы успокоить человека, а не напугать его.
Я вспомнил веру, в которой когда-то давно на Драконьей Пустоши воспитывала меня мать: страх перед Великим и Бесконечным, строгое исполнение предписаний и безусловное раболепие...

<<Кем бы ты ни был, лесные духи встретят тебя, успокоят твою боль и проводят в последнее пристанище.
И любящие воссоединятся, больные будут исцелены, помощники, защитники и трудяги смогут незримо служить живым, а уставших ждёт сладкий сон под вечные песни Печального Митра>>.

Я оглядел стоящую передо мной армию.
Пылерои, люди и идолы, попадались даже травники.
Мужчины, женщины с детьми, старики, даже подростки, способные держать оружие.
На их лицах было написано одно --- они знают, против кого сражаются, и пришли не за победой.
В моей груди немедленно выросло сильное и нежное чувство, с ароматом то ли материнских, то ли дружеских, то ли любовных объятий, и страх перед небытием, который сидел во мне все эти жизни, ушёл.

Ушёл навсегда, почтительно поклонившись и закрыв за собой дверь.

\chapter{[-] Могильный берег}

\section{[@] Последний полёт дракона}

Взлёт прошёл без особых происшествий.
Фонтанчик аккуратно повёл корабль, стараясь не перегружать обшивку и двигатель.
Мы видели, как товарищи очертили круг повышенной радиации и выпустили сборщик радионуклидной пыли.

Полёт проходил в молчании.
Фонтанчик и я прислушивались к системам корабля, готовые при любом подозрительном параметре выжать полную скорость.
Время от времени Фонтанчик многозначительно смотрел на меня.
Да, мы понимали, что значит для выживших тси потеря вначале кольцевой теплицы, а теперь и Стального Дракона.
Половина инфраструктуры была связана с ним, а без топлива мы можем распрощаться с надеждой развернуть последний узел планетарной защиты.
Без замыкания пространства узлы превратятся в красивые, но бесполезные золотые здания.

--- Это не конец, Небо, --- подал голос Фонтанчик.
--- Ничего ещё не потеряно.

Я кивнул, понимая, что друг сам не верит в истинность своих слов.

--- Может, мы ещё успеем попросить Безымянного? --- пробормотал я.

Фонтанчик с горькой улыбкой ткнул в мониторы.

--- Здесь всё.
Обменная система тоже.
На дне колокола показатели нормальные, но нюхом чую --- трещина заползла уже и туда.
Поздно, Небо.
Сейчас я бы принял помощь даже максима Картеля, но время мы упустили.

Пятьдесят тысяч, сто тысяч километров...
Планета Трёх Материков казалась нам небольшим голубовато-бело-жёлтым кругом.
Радиационный фон на нижних уровнях достиг критической точки.

--- Небо, запускай робота.

--- А если он...

--- Мы успеем, --- успокоил меня Фонтанчик.
--- Всего-то до шлюпки добежать.

Я тронул имплант, и мы вместе прыгнули в гравитационный лифт.
Почти сразу корабль ощутимо тряхнуло.

Я посмотрел в иллюминатор, медленно осознавая случившееся.
Да, роботы были перепрограммированы Машиной.
Включенное мной устройство просканировало системы корабля и, поняв, что он находится в аварийном состоянии и готов взорваться, приняло решение --- вернуться к начальной точке полёта.
К нашему поселению.

Одновременно с этим взвыла сирена --- машина взяла на себя управление реактором.
Цепная реакция медленно выходила из-под контроля.

Фонтанчик тоже понял.

--- Мою бабушку, --- выдохнул он.

Я едва успел навести несколько внутрисетевых экранов и обесточить два отсека, но было поздно.
Почти сразу включились десять перепрограммированных роботов, в том числе самый опасный --- робот-техник.

--- Небо, быстро в шлюпку.

--- Я...

--- Небо, --- Фонтанчик повернулся ко мне, и я увидел в его глазах тот неповторимый блеск, который появляется у существа, полностью осознавшего свой выбор.
--- Не спорь.
Да и дуэлянт я куда лучший, чем ты.

Дуэлянт.
Забавное слово.
Его истинное, древнее значение было забыто, и <<дуэлями>> назывались соревнования техников.
Двое техников по команде начинали попытки взломать имплантаты друг друга и проникнуть в кору, вызывая определённое ощущение --- обычно это была вспышка света или светящийся кружок.
Таким образом отрабатывались навыки быстрого взаимодействия техников друг с другом и с аппаратурой.

Здесь же дуэль была нешуточной --- проигравшего ждало отключение.

Фонтанчик пошёл по направлению к лифту с непередаваемой грацией --- красивый, статный, высокий пёс.
Одновременно с этим каналы связи едва ли не раскалились, как золотая проволока.
Мозг сапиента вступил в смертельную виртуальную схватку с пятнадцатью механизмами, которые все вместе превосходили его по мощности в три раза.

Корабль тем временем, неовратимо набирая скорость, приближался к оставленной нами планете.

Вдруг корабль снова тряхнуло, и меня ослепила вспышка гамма-излучения.
В горле пересохло от страха, перед глазами промелькнула страшная картина из Палат Войны --- существа, умирающие от лучевой болезни.
Я бросился к Фонтанчику:

--- Уходим!

Бесполезно.
Глаза друга остекленели, его разум полностью слился с системой.
Из уголка рта капала слюна, зрачки то увеличивались, то уменьшались, рефлекторно пытаясь помочь фехтовальщику, мечущемуся в бешеном виртуальном танце со смертью.
В другом конце коридора заскрежетал металл --- маленький робот-техник медленно, крадучись, приближался к живой добыче.

Мозг Фонтанчика тебе не по зубам, и ты решил нанести удар по беспомощному телу?
Ну уж нет.

...Как-то раз Баночка поднял меня на рассвете.

--- Идём, --- без предисловий бросил он.
--- Я разработал стиль боя для апид, <<Танцующий кузнечик>>.
Мне нужно его испытать.

--- Я не хочу драться! --- запротестовал я.

--- Один приём, Небо.
Один приём, и больше я тебя не побеспокою.

<<Прыжок надежды>>.
Баночка заставил меня повторить этот приём триста восемнадцать раз --- пока не убедился, что я натренировал его до автоматизма.

\ml{$0$}
{--- В прыжке боец грозен, но беспомощен.}
{``Fighter in a jump is fearsome, but helpless.}
\ml{$0$}
{Его полёт быстр, но его траектория предсказуема.}
{They fly fast, but predictable.}
Опытные бойцы царрокх знают это --- они называют прыгающего <<брошенным камнем>>.
\ml{$0$}
{Но ты --- не камень и не будешь лететь, как камень, ведь, в отличие от царрокх, ты знаешь законы физики.}
{But you're not a stone, and you won't move like a stone, for---unlike Te's\'{a}rr\r{o}kch---you know physical laws.}
<<Прыжок надежды>> --- это то, что заставляет расслабиться, то, что обманывает противника кажущейся предсказуемостью...

Я привёл все инструменты мультитула в готовность.
Сейчас всё решит один-единственный удар --- мой или робота.
Я попытался абстрагироваться от ужасающего гамма-излучения и сосредоточиться на других чувствах.

Шесть лапок.
Они тихо ступали по металлической поверхности пола.
Робот знал о резонанс-взрывателе и плазменном резаке в моём мультитуле.
Он тоже следил за мной всеми датчиками, которые у него были.
У него шесть лапок.
Как и у меня.

Гулкий, упругий удар.
И ещё раз.
И ещё.
Я слышал своё сердце так ясно, как никогда в жизни.
Я ещё живой.
Я ещё могу сражаться.

--- Жизнь и процветание! --- боевой клич Тараканов вырвался непроизвольно.
Теперь я вспомнил всё до мельчайших деталей.
Именно эту фразу я слышал, не имея ушей, и кричал, не имея рта.
Эти, именно эти слова выкрикивали мои сородичи, полуголые, грязные и голодные.
Им вторили голоса таких же истощённых кани, плантов и людей.
Крик разносился во выжженным радиоактивным пустошам, некогда буйствовавшим великолепием низких вечнозелёных кустарников.
Словно все эти тысячелетия он, записанный в самых древних моих генах, только и ждал подходящего момента.

Сейчас планета Ди жива и цветёт.

Мы с роботом бросились в атаку одновременно.
Взрыв --- и шесть металлических лапок разлетелись по коридору.
Следом за этим меня накрыла ослепляющая головная боль.

--- Небо, какого ты ещё здесь? --- слышал я в полузабытьи голос друга.
--- Поганый дым, ничего не вижу...
Тебя чем так оглушило?
Техник отключился, наверное, какая-то неисправность...
Давай, приходи в себя, с этой мелочью я разберусь!

Я почувствовал, как сильные руки друга нежно подняли меня и положили в мягкое кресло.

... Пришёл в себя я уже далеко от Стального Дракона.
Ужасное, непередаваемое ощущение гамма-излучения медленно угасало.
Задыхаясь от боли, я ощупал голову.
Усика и кожи вокруг него не было --- луч резака оставил лишь островок запёкшейся крови.

Тоскливо тянулись минуты.
Но вот я получил запрос на связь.

\ml{$0$}
{--- Небо, это я.}
{``Sky, it's me.}
\ml{$0$}
{Я победил.}
{I won.''}

У меня отлегло от сердца.

\ml{$0$}
{--- Молодец.}
{``Well done.}
\ml{$0$}
{Вылетай немедленно.}
{Leave immediately.''}

\ml{$0$}
{--- Поздно.}
{``Too late.''}

Фонтанчик включил видеосвязь, и я увидел друга.
Из его носа и дёсен сочилась кровь, его улыбка напоминала оскал боли.
Высохшие глаза растрескались и подёрнулись мутной слепой плёнкой.

--- Триста грэй, Небо.
\ml{$0$}
{Я буду вести корабль, пока смогу.}
{I'll drive while I can.}
\ml{$0$}
{Ты --- лучший командир, и я счастлив, что всю жизнь дружу с таким выдающимся тси.}
{You're the best cap, and I'm happy being friends with such an outstanding Qi.''}

Я кивнул, чувствуя, как сжимается сердце.

--- Ты самый лучший из тех, кого я знаю, --- сказал я.
\ml{$0$}
{--- Я люблю тебя.}
{``I love you.''}

\ml{$0$}
{--- Я тебя тоже, Существует-Хорошее-Небо.}
{``I love you too, Existing-Good-Sky.}
\ml{$0$}
{Попрощайся со всеми за меня.}
{Give my fond farewell to everybody.}
\ml{$0$}
{А Заяц передай: <<Как бы ни горело моё тело, любовь к тебе всегда будет на один градус жарче>>.}
{Also tell Hare this: \emph{`However much my body burns, my love to you is always one degree hotter'}.''}

Фонтанчик светло улыбнулся в пространство, вытер левой рукой кровь, невзначай активировав <<Живую сталь>>.
Затем изображение исчезло --- друг не хотел, чтобы я видел его лицо, искажённое препаратами в пустую, маскообразную гримасу.

Вскоре недалеко от меня вспыхнула новая звезда с моим другом в самом её центре.

\section{[-] Кровь и песок}

\textspace

Р-гранаты.
Картель перешёл к высокотехнологичному оружию.
Почти бесшумная вспышка --- и несколько десятков бойцов просто испарились.
Грейсвольд едва слышно ахнул и начал отлавливать и обезвреживать боеприпасы.

Недалеко от нас неистовствовала чья-то флейта.
Неизвестный менестрель наигрывал бодрый, согревающий кровь марш.
Бойцы вокруг него <<светились>> мощно и ровно, мешая вражеским демонам распознать музыканта.
Время от времени кто-то из агентов Картеля делал рывок в его сторону, расчищая себе дорогу фалангой, но защитники тут же наваливались на него вдесятером.
Некоторые безропотно, не прекращая <<светиться>>, принимали своим телом первый тяжёлый удар.

Вот я почувствовал на языке страшную горечь МПДЛ --- последняя отчаянная попытка Эйраки вернуть восставший скот под привычный хлыст.
Митхэ потратила весь воздух в лёгких на сигнал <<химическая атака>> и бросилась ничком на землю.
Тси тут же сбились в ватаги, прикрылись щитами, начали бледнеть и заваливаться.
Многие успели оглушить или обездвижить находящихся рядом людей Тра-Ренкхаля, мешая тем бесноваться в радужном безумии.
Агенты Картеля, которые заранее приняли антидот, удесятерили атаки, но всё было бесполезно.
Спустя половину михнет моя армия вырвалась из оков <<Кристалла>> и снова бросилась в бой, потеряв едва ли тысячу.

<<Ребята, готовьтесь! --- крикнула Анкарьяль.
--- Эйраки понял, что дело дрянь, и разрешил применить нуклеарное оружие!
База... Тридцать четыре секунды!.. Лишь бы... Дьявол! Ааа...>>

Остаток сообщения потонул в шуме ПКВ.
Я почувствовал, как Анкарьяль приняла на себя один, второй, третий удар... и тут всё стихло.
Над восточным горизонтом показался лучистый <<хвост>> ракеты-стрелы.
Она летела совершенно бесшумно, опережая звук в два раза.
Вокруг неё искрились расщепляемые в полёте атомы азота.

--- Вот чудо, --- ахнула прямо рядом со мной старушка, воткнув окровавленную фалангу в песок.

--- Это Удивлённый Лю! --- заорал кто-то.
--- Удивлённый Лю вернулся!
Удивлённый Лю летит к нам на помощь!

--- Мы обретём крылья!

--- Мы скоро обретём крылья!

Битва приостановилась, все как зачарованные смотрели на летящую к ним смерть.
Они лучились восхищением и надеждой.
На агентов Картеля, которые замерли или попадали на землю, никто не обращал внимания.

Вдруг неведомая флейта замолкла.
Ракета, сделав странный, ломаный вираж, улетела в зенит, и надо мной показался огромный источник <<света>>.
Моё тело инстинктивно успело поднять руки в жалкой попытке защититься.

<<Эйраки Мороз, ты заплатишь.
За каждое моё существо, за каждое мгновение боли>>, --- с этими словами на меня обрушился обжигающий поток.

Эйраки взревел и испустил волну <<антисвета>>.
Это не было дуэлью хоргетов, это было состязание грубой силы --- два невероятно мощных хоргета били друг друга чистым неструктурированным потоком масс-энергии.
Попавшие под горячую руку агенты Картеля отчаянно кричали, распадаясь на составляющие.
Кое-кто из сапиентов вздрогнул и посмотрел в сторону битвы, которую не мог видеть;
прочие как зачарованные продолжали смотреть на фиолетовый хвост ракеты...

<<Аркадиу! --- донёсся до меня слабый крик Грейсвольда.
--- Прикрывай Нар, прикрывай!
Я уже, я почти!..>>

Я из последних сил наводил защитные барьеры над подругой.
Моя энергия таяла, словно снежинка под лучами солнца.
Семьдесят процентов... сорок... десять...

<<Отлетался.
Прости-прощай, жестокий и прекрасный мир>>.

\section{[@] Одичание неизбежно}

Шлюпка вошла в плотные слои атмосферы и выпустила тонкие крылья.
Крылья управлялись аналоговым механизмом --- разработка предназначалась для самых разных ситуаций, в том числе для трудностей с энергоносителями.
Подо мной расстилался океан, где-то вдалеке виднелся узкой полоской пролив Скарр.
Я аккуратно вёл шлюпку, стараясь направить в сторону нашего поселения.
Но расчёты показали, что до суши я не дотяну.

Как-то странно закружилась голова.
На мгновение я потерял представление не только о верхе и низе, но и о направлении стрелы времени.
Стены начали ощутимо вибрировать и нагреваться.
Я убрал крылья на треть длины, опасаясь сломать их, затем увеличил тангаж, едва не завалил судно и тем самым потерял ещё несколько драгоценных километров.

Океан.
Да, похоже, что моя жизнь закончится именно здесь.
Здесь, в тёплой экваториальной эпипелагиали, водились хищные рыбы, против которых я не имел ничего, кроме четырёх слабых рук.

Четырёх?
Кажется, что-то не так.
Впрочем, неважно...

Боль утихла, и мои мысли обратились к чему-то далёкому, безбрежно-спокойному, обманчиво спокойному, как этот океан.
Жизнь.
Как же она удивительна.
Сколько поколений сменилось, пока появился я --- и вот пришло время уйти и мне, чтобы дать дорогу другим.

В такие моменты становишься честен с собой.
Нет, мы не успеем построить систему планетарной защиты.
Там, на Стальном Драконе, только что превратился в пар один из главных её архитекторов вместе с базами данных, половиной аппаратуры и энергоносителем для ядерного синтеза.
Скоро умру и я.
Кто знает, сколько отмерено остальным?
Каждый из нас --- носитель бесценной информации и опыта.
Увы, носитель непростительно хрупкий.

И вместе с этими мыслями пришла ещё одна, которую я до поры до времени отбрасывал.
Одичание тси неизбежно.
Нас слишком мало, и наши знания ничего не значат без инфраструктуры.
Сколько нужно времени, чтобы её восстановить?
Десять, пятнадцать поколений?
За это время большая часть знаний будет безнадёжно утеряна.

Значит, нужно обеспечить тси одичание, вдруг сказал я себе.
Эта мысль, кощунственная по своей сути, даже не вызвала у меня отторжения.
Да, одичание неизбежно.
И значит, что его нужно подготовить максимально качественно.

Вслед за этими словами по моей спине пробежал холодок, а затем сердце сжалось так, что я чуть не застонал.
Эту мысль необходимо передать остальным, жизненно необходимо.
Я не могу умереть, пока не буду убеждён в том, что другие найдут моё сообщение.

Привычным движением я вскинул мультитул.
Но устройство отозвалось молчанием.
Микросхемы приняли на себя всю мощь косого испаряющего луча, сохранив мне голову и жизнь.

Голова снова закружилась со скоростью электрона.
Нет, это не голова, это кружится неприветливый мир, выбивая у меня из-под рук панель...
Я снова и снова отчаянно пытался поцарапать краем мультитула стабитаниум и только спустя минуту осознал всю бесполезность этой затеи.
У меня промелькнула мысль записать сообщение на собственной коже или костях, но потом я вспомнил про хищных рыб.
Вдруг взгляд упал на привинченную к креслу табличку из мягкого алюминия.
Она находилась прямо на стыке, и увидеть её можно было, лишь приведя кресло в рабочее положение.

<<Кому пришло в голову сделать алюминиевую табличку?>> --- удивился я и, на секунду забыв о проблемах, пригляделся к написанному.

\begin{quote}
\ml{$0$}
{<<Искренне надеемся, что это никогда не прочитают, но желаем тебе удачи и мягкой посадки.}
{``We sincerely hope it will never be read, but we wish you good luck and soft landing.}
\ml{$0$}
{Держи штурвал крепко, лови попутный атмосферный поток и держись его.}
{Keep a good grip on the wheel, catch a fair atmospheric current and follow it.}
\ml{$0$}
{Если же ты ранен или ослаб --- расслабься и займись собой, мы сделали всё, чтобы ты пережил посадку.}
{If you're wounded or weak, just relax and take care to yourself, we've done our best to make you survive the landing.}
\ml{$0$}
{Под сиденьем сокровища, чтобы ты не грустил.}
{Under your seat there are some treasures to cheer you up.}
\ml{$0$}
{Руки-крюки из Четырнадцатого>>.}
{Butterfingers from the Fourteen.''}
\end{quote}

Я заглянул под сиденье, и на душе потеплело.
<<Восток-запад>>, или, как изысканно выразилась один раз Кошка, <<пасхальное яйцо>>.
Крохотный пенал, заполненный мелкими вещицами.
Какая-то труха, похожая на давно рассыпавшиеся цветы, серёжка c титановым анодированием, окаменевшая конфета в выцветшем пластиковом фантике, прядь красивых чёрных волос в жаропрочной смоле, <<вечное>> перо для граффити...

Снова закружилась голова.
Конфетка.
Как же давно я их не ел.
Я развернул фантик и отправил конфету в рот.
Прошла целая вечность, прежде чем я выплюнул на ладонь безвкусный кусок жаропрочной смолы.
Нетронутая конфета так и лежала рядом...

Перо!

Кристалла осталось совсем чуть-чуть, но оно ещё действовало.
Я вывел на панели пять огромных иероглифов:

\begin{quote}
<<Одичание неизбежно.
Нуждается в подготовке.
Небо>>
\end{quote}

Я приоткрыл верх капсулы и вдохнул свежий морской воздух.
Какая чудесная погода.
Я и забыл, что на этой части планеты сейчас весна.
Мимо проплыло низкое облако, громыхая и ворча небольшими электрическими разрядами, словно обижаясь на белых кучерявых красавцев выше.
Прямо на облаке сиял радужный <<глаз>> с тенью моей шлюпки в зрачке.
Я немного отстранённо думал, сколько мне осталось ещё лететь.
Голова, нет, только не сейчас...

Рулевой механизм повело в сторону, тонкое крыло взвыло от боли, испуганно запищали приборы.
Шлюпка нырнула вниз, словно подбитая чайка.
Но я этого уже не видел.

\section{[-] Цена свободы}

<<Аркадиу!
Загружайся!>>

Я ощутил пространство и время.
Следующим пришло осознание произошедшего.
Тело откликнулось последним.
Я открыл глаза.
Надо мной сидел улыбающийся во все зубы Грейс.

--- Доброе утро, человеческое отродье.

Я промолчал.
Время было похоже на дорогу с колдобинами --- иногда я проваливался в ямы, иногда взлетал на кочках.
Болел затылок, которым я ударился о присыпанный песком валун.
К горлу подкатывала тошнота.
И тут я вспомнил...

--- Нар!

--- Лежи тихо, она жива.

--- Грейс, что произошло?

Технолог продолжал скалиться.

--- Если я тебе расскажу, то ты признаешь меня вселенским гением.
Хотя так оно и есть.
Пока вы с Анкарьяль занимались Эйраки, я запустил Золотой город.
Это всего лишь один узел планетарной защиты тси, но мощности в нём скрывались приличные.
Я на несколько микросекунд создал большого, жирного хоргета.
Он был очень некрасивым, но вот вычислительные мощности его превосходили мощности Эйраки как минимум в миллион раз.
Простенькая программа --- защита Эйраки кракнула, как ореховая скорлупа, и старик превратился в младенца.
Я обрадовался, да только преждевременно.
Нар упала, ты её прикрываешь, а массивная минус-сингулярность начала трещать по швам --- видимо, я задел модуль стабилизации.
Всё, думаю, отлетался.
За микросекунды пришлось вспомнить почти всё, что я когда-либо знал...
Я поменял полярность сингулярности, флип-флоп.
Преобразование Шмидта, одно из четырёх фундаментальных омега-преобразований, незаслуженно забытое.
В итоге агенты Картеля, которые открылись и уже приготовились пополнить энергию, приняли расслабляющий плюс-душик и радостно аннигилировали.
Планета наша.

--- Так значит, всё хорошо?

Грейс нахмурился.

--- Аркадиу, у тебя мозг повреждён?
Я об этом и рассказываю!
Если для тебя важно, то Чханэ и Плачущий Ягуар --- дарительница твоего тела --- тоже выжили.
Непонятно как.

И тут до меня дошло.
<<Взгляд>> хоргета почувствовал исходящее от Грейса яркое сияние.
А ещё...

--- Ты тоже ничего так светишься, --- ухмыльнулся Грейсвольд.
--- Твой демон в фоновом режиме напитался.

--- А Анкарьяль?

--- Ну... нет, --- смутился Грейс и, взяв мою руку, положил её на чьё-то тело, лежащее за его спиной.
--- Она чересчур потратилась.
Думала, видишь ли, что погибнет, и сражалась как в последний раз.
Демон отключён от тела, амнезия небольшая будет... хаяй... я её подпитываю пока что, восстанавливается.

Я обратил <<взгляд>> на лежащую рядом Анкарьяль.
Её демон светился слабо, словно крохотная свеча.
Время от времени исходящий от неё свет трепыхался, словно птенец, в тяжёлом сне вынужденной экономии энергии.

--- Она молодец, --- одобрительно проворчал Грейс.
--- Ты почувствовал, как она ударила в первый раз?
Эйраки не на шутку растерялся.
Вряд ли он когда-либо встречался с таким противником.
А кое-кто из Картеля смылся сразу, не дожидаясь продолжения.

--- Они доложат о случившемся, --- сказал я.

--- Я тоже доложил, --- успокоил меня Грейс.
--- Уже успел слетать до Капитула и обратно.
Околопланетное пространство заполнено нашими агентами, на поверхности уже разворачивают микростанции.
У нас с тобой небольшая передышка.

Я подполз поближе к подруге.
Глаза Анкарьяль блуждали, как у младенца, на лице замерло отсутствующее выражение.
Я погладил подругу по испачканной мокрым песком голове и поцеловал её в лоб, надеясь, что она это почувствует.
Демон-птенец, жадно ловя исходящие от моего тела слабые эманации любви и сострадания, трепыхнулся и снова свернулся в маленький тёплый комочек.

--- Эйраки напоследок сообщение оставил, --- отстранённо, словно не мне, пробормотал Грейс.

--- Какое?

--- <<So’ana fre, ‘et>>\FM.
\FA{
Познай мой страх, предатель (сохтид).
}

Я откинулся на спину.

--- Кто это был?
Кто отвёл атомную бомбу и атаковал Эйраки?

--- Безымянный, кто ж ещё, --- грустно сказал технолог.
--- Видимо, он погиб --- больше плюс-искажения и осцилляций ПКВ я не чувствую.

--- Сколько сапиентов последовало за своим богом? --- спросил я, ощущая собственное нежелание слышать ответ.

--- Больше половины --- почти двести тысяч, --- откликнулся Грейс.
--- Радужное безумие, затем Картель устроил резню.
И ещё...

--- Что?

--- Убитые --- в основном народ сели и те, высшие пылерои, которые не боялись и сражались, пока могли держать оружие.
Они сбились вокруг нас троих, закрывая собой и запутывая интерфекторов.
Я даже не подозревал, что сапиенты с палками в руках способны настолько эффективно противостоять демонам.
Падали почти сразу --- на их место вставали другие.
Армия этих странных коротышек, трами, полегла полностью, поголовно, некому даже вернуться домой и сообщить о том, что произошло.
Я еле-еле вытащил вас из-под горы тел.
Если бы не они, мы бы с тобой не разговаривали...
А вот хака и тенку воины Картеля просто испугали и оставили в живых.

Где-то невдалеке плескалось море.
Солнце медленно клонилось к закату.
Из-за низкой песчаной дюны послышалась плачущая песня сели.
Я долго молчал, осознавая простой жестокий факт: здесь, на берегу Могильного пролива, был затоптан последний цвет моего рода, поднявшийся в едином порыве непонятно ради чего.

\section{[-] Воинская доля}

Я брёл через усеянный телами берег.
Голова гудела, но я не имел ни малейшего желания исправить это.
На меня то и дело оглядывались сидящие на корточках люди, кани, планты.
Их глаза были чисты, как небо.
Сели говорили: <<Раз тебе идти по жилам джунглей в одиночестве, то мы соберёмся вместе на твоих поминках>>.
Однако сегодня всё было иначе: мёртвые шли толпами, живые сидели по одному.

--- Мы победили? --- недоверчиво спросил кто-то в темноте.

--- Да, --- ответили ему.

--- Это не похоже на победу! --- выкрикнул человек.
--- Посмотри вокруг!
Какая победа?
С чего ты взял, что мы победили?

--- Мы ещё живы, дружище, --- ответили ему сразу несколько товарищей.
--- Мы ещё живы!

--- Ликхмас! --- из темноты выскочила Митхэ ар'Кахр, бросилась мне на шею и тут же отстранилась.
Её глаза горели спокойным огнём воина.
--- Король-жрец, у нас мало пищи и пресной воды.
Какие будут распоряжения?

--- Поговори с хака и ноа, --- сказал я.
--- Потом возьми добровольцев и отправляйся за провизией и водой как можно быстрее.

--- Ты как? --- тихо спросила родильница.
--- Кстати, вон твоя женщина сидит.
Я видела её в отражении ножа...

Мы с Чханэ посмотрели друг на друга одновременно.
Подруга улыбнулась и показала мне спящего ребёнка.
Большего не требовалось.

--- Чханэ? --- из темноты показался пожилой хромающий мужчина.
Женщина застыла.

--- Ты? --- услышал я её сдавленный шёпот.

--- Дитя! --- заплакал мужчина и, оступившись, упал прямо на Чханэ.
Звуки борьбы и недовольное гуление малыша сменились поцелуями.
--- Ты живая, слава духам!
Я всех духов закормил, только чтобы...

Мужчина, всхлипывая, что-то затараторил с сочным западным акцентом.

--- Почему у тебя воинская стрижка?
Я же попросил тебя держаться подальше от Храма!
Ты живая, ты правда живая?
Хай...
А это чей ребёнок?
Змейка, ты откуда малыша взяла?..

Чханэ, смеясь, рассказала кормильцу, откуда появляются дети.
Дослушивать я не стал и отправился дальше.
Митхэ, словно тень, шла рядом.

Вот Кхохо.
Она бросила в пылу боя свою серебряную саблю и била врагов двумя обломками корабельной реи, наспех заострёнными ножом.
Чей-то клинок ударил её прямо в загривок, почти отделив голову от тела.
На лице воительницы замерло выражение брезгливого удивления: <<Это ещё что за дрянь?>>.
Рядом с ней никто не стоял.

В двадцати шагах лежал Ситрис.
Родильница подбежала к нему.

--- Привет, Митхэ, --- вяло помахал ей воин.
--- Рад, что ты живая.

--- Здравствуй, дружище, --- улыбнулась Митхэ, деловито осмотрела раны и показала мне скрещённые пальцы.

--- Да знаю я, --- проворчал Ситрис.
--- Как будто первый раз в боях.
Знаешь, Митхэ, мне ужасно хочется жить.
Даже больше, чем тогда, когда мы с тобой бежали по старому тракту через Сикх'амисаэкикх.
Жизнь --- это самое прекрасное, что со мной случилось.
Но что-то этот мир стал для меня чересчур сложным...

Ситрис обвёл ладонью вяло кровоточащие раны.
Я потянулся за чашей.

--- Не-не, Ликхмас, забудь.
Ни секунды жизни не хочу тратить на такую ерунду.
Как Кхохо, не видел её?
Хе-хе, ты слушай.
Сабля расплавилась у неё в руках, честное слово.
Кхохо так ругалась, что вокруг неё враги сами дохли, она ж её любит больше жизни.
Потом подобрала какие-то палки и давай их посреди боя строгать.
\ml{$0$}
{Я её пытаюсь прикрыть и ору, ты дура, приоритеты хоть перед смертью научись расставлять.}
{I was trying to cover her and shouted: you fool, set your priorities straight, at least once before we die.}
А она палки заострила и пошла, я тоже какую-то дубину схватил во вторую руку.
Мы так хорошо двигались <<танцем согхо>>, как по книжке, плечо к плечу.
А потом она пропала, и меня искололи всего.
Неужто старушку-веселушку всё-таки нашёл клинок?

--- Умерла быстро, --- кивнул я.
--- Зарубили сзади.

--- Сзади?

Ситрис хрипло рассмеялся, забрызгав кровью песок.

--- Ну и дура, --- тихо сказал он и умолк.

Родильница, вздохнув, достала хэситр и мех с водой.

--- Рада, что ты с ним успел познакомиться, --- сказала она.
--- Он был разбойником, его объявили вне закона во всех городах Юга.
Когда мы проходили Кахрахан, он бросил свою шайку и пришёл ко мне на службу, несмотря на угрозу казни.
Очень добросердечный человек, не чета многим законопослушным лицемерам.

--- Он как-то спас мне жизнь, --- сказал я.

--- Правда?
А я ругала его за то, что он пообещал искать Атриса, но сбежал в Тхитроне, --- покачала головой Митхэ.
--- Всё-таки помощь возвращается, хоть и в неожиданном обличье.
Иди, Король-жрец.
Я почту его память и отправлюсь выполнять приказ.

\section{[-] Цвет Земли}

Кхотлам сидела и отстранённо чертила щепкой фигуры на песке.
Услышав мои шаги, она посмотрела на меня.
Сквозь меня.

--- Лисёнок...
Дитя...
Мы победили?

--- Да, --- только и смог выдавить я.

--- Хорошо, --- улыбнулась она, бросила щепку и погладила молодого, мёртвого как камень Хитрама по чёрным волосам.
--- Видишь, Пловец, не зря ты отплавался.
Только лучше уж я бы с тобой.

Я сел перед ней и взял её руки в свои.

--- Ты же знаешь, как мы познакомились?

--- Ты потерпела кораблекрушение, а Хитрам вытащил тебя на берег.

--- На самом деле история была ещё красивее, --- улыбнулась Кхотлам.
--- Я примотала себя к корабельному рулю, наглоталась воды и окаменела.
Хитрам в одиночку вытащил меня на берег вместе с десятью матросами.
Все они были мертвы, и меня он посчитал мёртвой.
Хитрам уже начал копать нам могилы в песке, но долго не решался положить меня в могилу.
Он сказал: <<Такая красота просто не может принадлежать мертвецу>>.
И влил мне в нос масло пустынного перца из Санта-Виктории!..

Кхотлам всхлипнула.

--- Первое, что я сделала, когда смогла шевелиться --- это со всей силы ударила беднягу по лицу какой-то палкой, до того мне было обидно за горящий нос.
В общем, носы после этого горели у нас обоих.
В нашу первую ночь мы поклялись, что никому этого не расскажем, пока один из нас не умрёт...

Откуда-то из ткани сумерек возникла мрачная однорукая фигура.

--- Кхотлам.

--- И тебе не хворать, Акхсар ар’Лотр, --- не оглядываясь, ответила кормилица.

--- Когда-то ты хотела быть со мной, но я был чересчур влюблён в дорожную пыль.
Сейчас я один, без очага и крыши, --- без обиняков сказал старый воин и положил здоровую руку матери на плечо.
--- И знаешь, все эти годы я почему-то вспоминал, как уютно у тебя в жилище.
Примешь ли ты меня?
Могу ли я стать твоим мужчиной и дожить эту чересчур долгую жизнь с тобой?

--- Хитрам очень тебя любит, --- сказал я кормилице.
--- Он вернул тебя к жизни тогда и хотел бы, чтобы ты жила дальше сейчас.

Кхотлам молчала.
И вдруг её глаза на секхар снова засияли.
Она прикрыла их, словно стесняясь этого блеска, и ткнулась головой в шею Акхсара:

--- Помоги мне его похоронить.
Если ему не хватит сил достигнуть пристанища --- мне не видать ни покоя, ни жизни.

Акхсар сел, достал чашу и пузырёк с краской и, зажав чашу между коленей, стал аккуратно вырисовывать знаки.
Кхотлам встала и пошла к ближайшему костру, чтобы принести свет.

Я пошёл дальше.
Но что-то было не так, как прежде.
Люди по-прежнему бродили среди тел, выискивая живых, или сидели над умершими.
Но было и то, чего я не заметил.

Вот плакал осиротевший ребёнок.
К нему подошла женщина и, оглянувшись, подхватила его на руки.

--- Кормилицааа...

--- Тихо, тихо.
Я кормилица.
Я пришла.

Молодая женщина утешала рыдавшего над телами парня:

--- Тише, тише, только не кашляй, сердце не выдержит.
Расслабься.
Я буду твоей женщиной до самой смерти.
Слышишь, Манис?
Я всегда буду с тобой.
У нас будет много детей.

Я понаблюдал за ними.
Вскоре девушка подняла парня, и они подошли к потерянно сидящему на песке старику, который, опустив голову, ковырял тростью песок:

--- Лехэ, кормильцем нам будешь?

Это происходило везде.
На время забыв о мёртвых, люди и пылерои, идолы и травники заботились о живых.
Вот старик-человек робко положил руку на мускулистое плечо понуро сидящего пылероя.
Тот дёрнулся, но не отодвинулся.
Вот травник вел под руку раненную женщину.
Всюду начали разгораться костры.
Идолы и люди, сели и хака, прежде заклятые, непримиримые враги, сидели рука в руке, не говоря ни слова.
Кто-то тихо напевал.

Тогда я и начал понимать, почему тси называли цветом Ветвей Земли.
То, что я увидел в тот день, я не видел больше нигде во Вселенной.

\section{[@] Медицина бессильна}

\textspace

Костёр покрутил в руке курительное приспособление, подаренное ему вождём царрокх.

--- Извини, Небо, я бессилен, --- сказал наконец врач.
--- С кольцевой теплицей мы бы выходили тебя, но увы.

--- И сколько мне осталось?

Костёр пролистал мои данные и нахмурился.

--- Тридцать часов.
У апид первичная реакция гораздо короче, она пройдёт через два-три часа, как раз можешь поспать.
Ещё около двадцати часов будет стадия мнимого благополучия, дальше умирание.
<<Фотон-8>>\FM\ в лучшем случае растянет это время на сутки-двое.
\FA{
<<Фотон>> --- поддерживающий комплекс химических агентов и клеточных культур при острой лучевой болезни.
Восьмой номер --- для апид-тси.
Состав средства неизвестен.
}
У меня осталось две ампулы.

--- В таком случае не трать их, --- сказал я.
--- Не хочу менять чьё-то здоровье на двое суток.

--- Небо, может, ты хотя бы перед смертью подумаешь о себе?

--- Я бы и рад, но о себе думать поздновато, --- буркнул я.
--- Выключай.

Костёр кивнул, выключил инфузию и вышел.
Спустя несколько секунд меня охватил жар.

\section{[@] Раскол}

Врач вернулся спустя десять минут.

--- Я всё-таки поставлю тебе некоторые препараты, --- сказал Костёр извиняющимся тоном.
--- Не дефицитные, не бойся.
Обычные жаропонижающие и вегетотоники, их как песка на пляже.

Я отмахнулся.

--- Небо, я хочу беседовать с тобой как можно дольше.
Ты не поверишь, как я этого хочу.

Это были слова друга, а не врача.
Я не мог сопротивляться их искренности и подчинился.
Жар спал.
У меня на лбу выступил обильный холодный пот.

Костёр подумал и, поколебавшись, зарядил в плечевой имплант ещё одну капсулу --- с мутной, очень тёмной жидкостью.

--- Сок одного из местных растений, --- объяснил он.
--- Когда стадия мнимого благополучия закончится... если почувствуешь, что совсем невыносимо, переключи имплант на четвёртый канал.
Протокол <<Тайфун>> может отказать из-за разрушения клеточных структур, а это --- быстрая и совершенно безболезненная смерть, гораздо лучше нашего препарата для эвтаназии.
По ощущениям будет похоже на засыпание.
Знаешь, как будто ночник гасишь в голове и ложишься в мягкую постельку.

--- Хорошая вещь, --- улыбнулся я.
--- Биологи нашли?

--- Сами удивились, --- кивнул Костёр и ухмыльнулся в воротник.
--- Не всё на этой планете жестоко к нам.
Так, кто там ломится?

--- Небо пришёл в себя? Срочно нужно поговорить! --- раздался приглушённый голос Баночки.

--- Ему нужен покой! --- громко сказал Костёр.

--- Нет-нет, --- я попытался сказать погромче, но голос сорвался.
--- Пусть войдут, Костёр.
Если что-то важное...

--- Сказать им?..

--- Не сейчас.
Вначале дело, прочее потом.

Костёр махнул рукой. Баночка и Кошка осторожно подошли.

--- Как ты?

--- Лучше всех, --- соврал я и неуклюже изобразил танец.
Культурологи заулыбались, Костёр отвернулся и сделал вид, что копается в ларце.
--- Что у вас?

--- Мы только с обсуждения твоего послания.

--- Какого послания? --- удивился я.

--- <<Одичание неизбежно и нуждается в подготовке>>, --- процитировал Баночка.

Я вдруг осознал, что не помню ничего подобного.

--- У тебя амнезия, --- сказал Костёр.
--- Большая удача, что ты успел записать свою мысль.
Хотя можно было воспользоваться дневником в импланте, тебя всё равно никто не съел.

--- Кто мог меня съесть? --- удивился я.

--- Акулы, --- объяснил Костёр.
--- Я просмотрел твой дневник.
Послё отлёта у тебя было сужение сознания и бред.
Повторяется одно и то же слово --- голова, голова, голова...
Не знаю, почему ты решил, что тебя съедят --- шлюпку можно разбить вдребезги разве что на первой космической скорости, и уж точно не о воду.

Одичание неизбежно.
Нужна подготовка.
Да, это до ужаса логичная мысль, и она вполне могла прийти мне в голову.

--- Общество раскололось, --- без обиняков сказала Кошка.
--- Многие до сих пор не хотят верить, что мы проиграли.

--- Чего они хотят? --- осведомился я.

--- Уйти на экваториальный щит и попробовать пробиться через кору планеты, --- сказала Кошка.
--- На поверхности катастрофически мало металла, но в мантии он должен быть.
Да, Небо, мы понимаем, что это значит.
Мы говорили о возможных последствиях, но они ничего не захотели слушать.

--- Нас предостерегали от ухода цивилизации под землю! --- выдохнул я.
--- Неужели тси забыли это предостережение?
<<Титановые небеса --- предвестник рабства>>...

--- В отчаянии и горе нормально стать забывчивым, --- рассудительно заметил Костёр.

--- Мы с Кошкой считаем, что твой план неплох, --- перешёл к делу Баночка.
--- Он осмотрителен и выполним, хоть и не даст сиюминутных результатов.

Плант вытащил из сумки компьютер и включил его.

--- Тут у меня... эээ... кое-какие мысли появились, --- смущённо сказал он.
--- Кошка тоже помогла.
Мы обдумали устройство общества царрокх и решили выкроить по этой мерке новую культуру для выживших тси.
Искусственную культуру, в которой можно воспитывать молодых тси в отдалённых поселениях.

--- Вы собрались загнать тси в Древний мир? --- с ужасом пробормотал я.
--- Неужели я вложил именно такой смысл во фразу <<подготовка к одичанию>>?

--- А у нас есть выбор, Небо? --- грустно усмехнулась Кошка.
--- Посмотри вокруг.
Мы на чужой дикой планете.
Вокруг одни голые камни, которых не касалась даже нога, не говоря уже об инструменте.
Инфраструктура нулевая.
Корабль был для нас всем, но мы лишились его.

--- Раньше постройка планетарной защиты казалась само собой разумеющимся, --- добавил Баночка.
--- А теперь это выглядит, словно мы собрались возвести дворец с фонтанами в безводной пустыне.

Я промолчал.

--- Так вот, --- Баночка деланно бодро продолжил рассказ.
--- Культура царрокх --- это сплав мифологического мышления с примитивными технологиями.
Мы могли бы адаптировать её для тси.
С одной лишь разницей --- мифология и технологии тси будут не остатками былого великолепия, брошенными на произвол судьбы, а продуманной системой, которая обеспечит возможно скорый технологический подъём в будущем...

--- У нас есть информация об изменчивости окружающей среды, --- перебила планта Кошка.
--- Биологи, экологи и геологи проделали гигантскую работу.
На основе этого можно предсказать с неплохой точностью всё, включая изменения языка, пути миграции примитивных народов и особенности их культуры...

--- Мы постараемся записать большую часть информации на долговечных и легко воспроизводимых носителях, чтобы, когда технологический и научный прогресс достигали определённой точки, новые открытия происходили проще и быстрее, --- увлечённо затараторил Баночка.
--- Даже песни и стихи, которые мы запустим в новую культуру, будут психологически подготавливать тси к прогрессу!..

--- Вам хватит знаний?

--- Мы используем всё, что в нашем распоряжении, --- заверила Кошка.
--- Работа, разумеется, на годы и десятилетия.
Но, если честно, это первое по-настоящему серьёзное дело для меня.

--- Как и для меня, --- поддержал подругу Баночка.
--- Я вообще не припомню, чтобы кто-то из знакомых брался за такой масштабный и важный научный проект, при этом больше похожий на художественное произведение.

--- А если сюда придёт Ад или Картель? --- спросил я.

Баночка и Кошка как воды в рот набрали.
Но им на помощь неожиданно пришёл Костёр.
Он хмыкнул и опустил голову.

--- Я никогда не верил в точный расчёт, --- признался он, повернувшись к культурологам.
--- В медицине власть случайности гораздо ощутимее, чем в других областях --- она может и спасти, а может и унести чью-то жизнь.
Вряд ли когда-то было иначе.
Многие из знакомых врачей уходили в подобие мистицизма --- носили счастливые пробирки на цепочке, читали стишки перед сложными процедурами.
На эту тему даже исследования были.
Я, разумеется, так не делал, но тоже старался верить в лучшее.

Культурологи заулыбались.

--- Ну, больше нам ничего и не остаётся, --- развёл руками Баночка.
--- Бессмертных среди нас нет.
Всесильных тоже.

--- И всё же я внесу некоторые коррективы на случай вторжения хоргетов, --- едва слышно пробормотала Кошка.
--- Вряд ли это поможет, конечно...

Вдруг глаза Кошки загорелись.

--- Вот я дура, --- женщина хлопнула себя по лбу.
--- У меня под рукой всё это время был хоргет-девиант, а я строила дома из змей\FM...
\FA{
Строить дома из змей --- производить логические выкладки, основываясь на непроверенных данных.
}
Безымянный, я буду твоей верной жрицей до самой смерти, только дай тебя изучить!

С этими словами она схватила под мышку громко протестующего Баночку и выбежала из палатки.

--- С ума посходили, --- покачал головой Костёр.
--- Но звучит интересно.

Врач неожиданно нежно погладил меня по голове и, бормоча что-то обыденно-успокоительное, занялся моей инфузионной системой.
На секунду я вдруг очутился дома, вне тревог, опасностей и забот.
Немедленно потянуло в сон.

Где-то вдали тихо запел Безымянный.
Интересно, откуда он знает <<Песню рассвета>>?..
Мак подхватил.
Как чудесно сочетаются их голоса.
Фонтанчик и Комар спорят, как лучше приготовить лилового краба.
Вереск сказала, что его обязательно нужно запечь в глине.
Что ж, у тси ещё остались кулинары со вкусом.
Мы точно не пропадём.

--- Ведь правда же, кормилица?..

--- Ясное дело.
Чистая правда.
Спи, малыш, --- Костёр снова погладил меня по голове.
В его глазах тихо позванивали оплавленные осколки корабельного хрусталя.

\section{[-] Обычаи погребения}

\textspace

Воздух огласили далёкие воинственные крики.
Ноги перешли на бег прежде, чем я успел осмыслить, что происходит.

--- Ни за что! --- кричал кто-то.
--- Не для того мы...

--- Тихо! --- крикнул я, вбежав в толпу людей.
Толпа замолчала.
Пара парней, смекнув, тут же подхватили меня и усадили на плечи.

--- Это Король-жрец! --- крикнул один из них.
--- Пусть он говорит!

--- Что происходит? --- осведомился я.
Люди понурились.

--- Пылерои, --- один из мужчин кивнул в сторону.
--- Они хотят наших мертвецов.

Я похлопал парней по головам, и они пронесли меня через толпу к мрачно стоящей группе пылероев.

--- Что случилось? --- спросил я на цатроне.
--- Мои люди говорят, что вам нужны мертвецы.

--- Меня зовут Рычащая-Звёздна-Бездна, --- сказала седая волчица, статная и гордая.
--- Я старейшина клана Лёгкого Пера.
У нас кончилась провизия, и нам неоткуда её взять.
Наши кланы были изгнаны, а стада забиты.
Если мы вернёмся в пустыню, то найдём смерть, а не пищу.

--- Что с вашими мертвецами?

--- Мы уже заготовили все тела, которые смогли, --- ответила Бездна.
--- Но этого мало.
Я знаю, что у вас тоже нет провизии.
Я предлагала разделить мясо, но люди не пожелали обсуждать это.

Люди молчали.
Все как один смотрели на меня.

--- Как надлежит людям сели поступать со своими мертвецами, родичами и друзьями? --- спросил я.

--- Мы хороним их, а не делаем из них жаркое! --- рявкнул молодой жрец.
Толпа согласно зашумела.
Я поднял руку, призывая к молчанию, и обратился к волчице.

--- Как поступают пылерои со своими мертвецами, родичами и друзьями? --- спросил я на цатроне.

--- Мы хороним их, --- сказала волчица.
--- Съедаем мы лишь самых смелых и сильных врагов.
Но сейчас другие времена --- мы хотим лишь выжить и набраться сил.

--- Считаете ли вы врагами людей сели?

--- Мы сражались с ними бок о бок, делили горечь утрат и радость победы, --- сказала волчица.

--- Окажете ли вы уважение тем, кого съедите?

--- Мы почтим их, как своих родичей и друзей, --- сказала волчица.

Я повернулся к сели и перевёл её слова.
Толпа неуверенно зашевелилась.

--- Нужны ли нашим мертвецам их тела? --- сказал я напоследок.
--- Духи уже давно ушли искать пристанище, и многие из вас лично указали им путь.
Что ещё нужно тем, кто пал в битве?

--- Но они надругаются над телами! --- крикнул кто-то.

--- Им окажут уважение! --- сказал я.
--- Вождь пылероев сказала, что съедают они лишь самых сильных и храбрых.
Кто сомневается в том, что наши павшие были сильны и храбры?

Толпа забормотала.

--- Может ли быть для павшего большая честь, нежели быть благословлённым старыми врагами?
Неужели среди сели не найдутся те, кто захочет помочь в голодный час товарищам по оружию и друзьям, пришедшим в час нужды?

--- Я попрошу на коленях ради моего клана, --- сказала волчица.

--- Ещё чего! --- махнул я рукой.
--- Сели не унижают так даже врагов.

--- Тогда чего они хотят?

--- Подожди, --- ответил я и громко добавил на сели:
--- Предлагаю голосование.

--- Не нужно, Король-жрец, --- сказал подросток, стоявший у самого края.
Он вышел из толпы и подошёл к Рычащей-Звёздной-Бездне.

--- Возьми моя кормилица, вождь, --- сказал он на ломаном цатроне.
--- Я есть очень близко человек к ней.
Она кормить меня, и я мочь сделать она тебе, чтобы ты тоже есть.

Волчица поклонилась.
Подросток протянул ей руку, Бездна аккуратно взяла её, и вместе они ушли к телам.

Вскоре из толпы вышли ещё десять человек, знаками поманили пылероев и тоже ушли.

Я повернулся к мрачно ожидающим людям.

--- Пылерои не возьмут ни одного тела, которое вы не захотите им отдать.
Решайте сами --- павшие уже дома, а в ваших руках чья-то жизнь и смерть.

К вечеру пылероям отдали достаточно тел, чтобы они смогли пополнить провизию.
Несколько самых уважаемых жрецов пошли к пылероям, чтобы помочь с заготовкой мяса и проследить за исполнением обрядов.
Вернувшись, они сказали, что мёртвым не на что жаловаться --- из их костей пылерои сделали несколько тотемов, украсив их лентами, снятыми с собственного тела украшениями и амулетами.
Многие унесли с собой черепа, пообещав установить их в своих святых местах.

Клан Лёгкого Пера выразил желание уйти с людьми к Кахрахану и жить на землях сели.
На собрании было принято решение отдать пылероям на кормление и охрану Вялую Степь, которая вполне годилась для скотоводства.

\section{[@] Два общества, два пути}

\textspace

Кошка и Баночка спорили до хрипоты.

--- ... и ещё здесь очень хорошее место для центра торговли! --- говорил Баночка.

--- Да кому нужна твоя торговля, если нечего есть? --- возражала Кошка.
--- Бассейн Ху'тресоааса\FM\ --- идеальное место.
\FA{
<<Река, просящая пить>> (цатрон).
}
И пища, и пути сообщения...

Я слушал.
Какие же у меня умные друзья.
Впрочем, моё восхищение быстро растаяло, когда они перешли на повышенные тона.

--- Что значит <<избавиться от языка тси>>? --- возмутилась Кошка.
--- Это наша история, наше богатство, наше...

--- Язык тси будет изолирующим фактором! --- пытался убедить её Баночка.
--- Создавать двуязычное общество <<цатрон-тси>> бесполезно, народ всё равно перейдёт на что-то одно, и скорее всего --- на биологически родное.
Я считаю, что следует использовать один из <<общих>> языков.
Например, сектум-лингва.

--- Язык Картеля? --- взвыла Кошка.
--- Как тебе такое в голову могло прийти?

--- Да какая разница, чей это язык? --- не выдержал плант.
--- Главное, что он простой и...
Небо, перестань так на нас смотреть!
Мне уже стыдно от твоего взгляда!
Скажи лучше, кто из нас прав.

--- Я думаю, что вы оба правы, --- сказал я.
Кошка и Баночка разом умолкли.

Я приподнялся на локтях в капсуле.

--- Пусть Кошка делает своё общество на Короне, в бассейне Ху'тресоааса.
А ты, Баночка, сделаешь своё на берегах пролива Скарр.
Обсуждать детали будете вместе, но в своём проекте хозяин имеет приоритет в принятии решений.

--- Это напрасное распыление сил, --- покачал головой Баночка.

Кошка нахмурилась.

--- А ведь мысль-то здравая, Баночка.
У нескольких непохожих культур, даже если они не идеальны, шансов выжить больше, чем у одной идеальной.
Тем более что эта одна культура тоже не будет идеальной, просто не может быть такой.
Как думаешь?

Кошка вспорхнула и грациозно встала посреди палатки.
Мы с Баночкой с интересом наблюдали за ней.

--- Если сапиентный вид находится на ранней стадии развития, мультикультурализм --- это спасение для него.
Вот взять хотя бы нас.
А если точнее --- нашу одежду.

Женщина изящно потрепала себя за рукав рубашки.

--- У каждого из нас дома был станок, печатающий одежду.
Любого покроя, с любой структурой ткани.
Собственно, покрой был важен разве что для удобства и красоты, защитные качества одежды обеспечивала наноткань.
В этой одежде можно спокойно гулять как в пустыне, так и в тундре.
Впрочем, даже если она и оказывалась не по погоде, то ничего страшного --- у нас всегда были в пределах досягаемости транспорт и уютные жилища.
Собственно, именно благодаря нашим технологиям мы и могли себе позволить такую роскошь, как единый всепланетный язык и единую культуру.

--- Я что-то не понял, --- буркнул Баночка.
--- Ты назвала наш язык и культуру <<роскошью>>?
Ты же сама только что...

--- Ты не дослушал, --- развела руками Кошка.
--- А теперь взять примитивные племена, лишённые развитой инфраструктуры и технологий.
Каждый аспект их жизни в той или иной степени подчинён выживанию, включая язык и изготовление одежды.
Они вынуждены изготавливать одежду из того, что есть, добиваясь эффективности не наноструктурой материала, а покроем, толщиной пласта и прочими, как ты изволил недавно выразиться, <<ухищрениями>>.
Всё это стало традициями.
Более того, эти традиции писаны кровью --- ведь если одежда оказывалась неэффективной и человек заболевал, без должных технологий он вполне мог умереть.

Баночка кивнул и что-то пометил в своих записях.

--- То есть, как мы и говорили.
Культурный дрейф имеет место быть, но доля эволюционного процесса всё-таки больше.

--- Верно, дорогой мой! --- подтвердила Кошка.
--- И потому я возьму все культурные наработки местных --- именно местных! --- аборигенов.
То есть тех царрокх, которые живут по берегам Ху'тресоааса.
Всё, что мне нужно --- понять предназначение отдельных деталей, слегка оптимизировать их культуру и вплести в неё нужные для развития тси идеи.

--- А мне, значит...

--- А тебе придётся попотеть, потому что на берегу Скарра никто не живёт.

--- Да я не против, --- ухмыльнулся Баночка.
--- Обожаю сочинять легенды!

--- Я тебе помогу, и прямо сейчас.
Не знаю, какие силы дёрнули тебя упомянуть сектум-лингва, но идея была блестящей.

--- Почему? --- насторожился Баночка.

Кошка с сияющим лицом открыла на компьютере статью и ткнула её под нос планту.

--- <<Сектум-лингва --- упрощённая форма языка эллатино...
Эллатиняне создали огромную империю на берегу тёплого тропического моря>>, --- процитировала Кошка, не дожидаясь, пока собеседник прочитает.
--- Это язык людей, живших на берегу тёплого тропического моря.
Он как будто специально создан для твоего прибрежного государства.
Я думаю, что тебе стоит взять именно его, а не местные языки.
Это будет связь с другими планетами.
Хоть, по дошедшим до нас данным, поддержка сектум-лингва и прекратилась, на потомковых формах языка говорит половина обитаемой Вселенной.

--- Хорошо, предположим, что я выбрал правильно и этот язык действительно приспособлен для приморских условий среды, --- сказал Баночка.
--- Но тогда получается, что тебе придётся отказаться от языка тси.
Мы не знаем точно, к каким условиям приспособлен он.

--- А вот здесь я хочу устроить небольшое соревнование, --- улыбнулась Кошка.
--- Моё общество будет, как я и хотела, двуязычным.
Язык народа тси против языка малоизвестного, но обкатанного жизнью в джунглях народа царрокх.
Пусть потомки тси сами выберут сильнейшего.

--- А ещё наши виды будут врагами, --- глухо сказал Баночка.
Улыбка на лице Кошки застыла.

--- Что?
Что ты?..

--- Да ладно тебе, Кошка, --- угрюмо поднял руки плант.
--- Если в примитивном обществе даже разные народы становятся рабами собственной парадигмы выживания, что говорить о разных биологических видах?
Один вид станет агрессивным --- прочие станут такими же или погибнут.
Это ты не скорректируешь никак.

--- И всё же попробуйте, --- сказал я.
--- Придумайте мирную тактику, которая эффективно противостояла бы агрессии.
Я очень надеюсь, что ко времени нового расцвета наши виды друг друга не перебьют.

Друзья промолчали.
Кошка сидела, чертила таблицы, набирала тексты.
Время от времени она смотрела на нас с Баночкой и тайком смахивала набегающие слёзы.

\chapter{[-] Дорога домой}

\section{[-] Цитра павшего}

\textspace

--- Ликхмас, сыграй, пожалуйста.
Ты ведь умеешь?..

Я провёл пальцами по струнам цитры.
Хороший инструмент.
И где только Кхохо нашла это произведение искусства?
Цитра стоила не меньше трёх кукхватровых клинков.

--- Сыграй, Ликхмас, --- вмешался Эрликх.
--- Только красиво, а не то Кхохо восстанет из мёртвых и надаёт тебе по рукам.

Я аккуратно покрутил колки, проверил подушечками пальцев натяжение, дёрнул пару струн для проверки.
Помусолив палец, немного смазал струны над ладами.
Инструмент сложный, но составить программу для двигательного аппарата моего тела не составляло труда.
Анкарьяль усмехнулась, глядя на меня, заулыбался и Грейс:

--- Старая песня...

--- Хлебом не корми, дай побренчать...

Я обратился к родильнице:

--- Что сыграть?

--- Что хочешь.

Конечно же, она хотела бы ещё раз услышать то, что играл Атрис, но прятала это за внешним равнодушием.
Что же он мог играть?

Мои руки будто перестали принадлежать мне.
Подушечки перчатки вначале туго, а потом всё легче и легче стали скользить по тонкому металлу.
Моя слюна разошлась по струнам, смочила их, подушечки заскользили ещё легче, звук цитры стал выразительнее и ярче.
Я покачал ногой, с лёгким жужжанием разогнался маховик, приводящий в движение смычок.
Ногти правой руки едва заметно касались тонкой бронзы, и цитра отзывалась на ласку, впускала меня по шажку в свои покои.

По мере игры знакомые фигуры и схемы вставали на место, словно кусочки мозаики.
Классическая музыка Тси-Ди, отголоски той музыки, которая звучала на Древней Земле.
Понимание приходило постепенно --- Атрис был хоргетом.
Об этом кричало всё --- описанные родильницей случаи, его мастерство, которое явно выходило за рамки возможностей смертных.
Атрис был не из тех хоргетов, кто сгребает человеческие эманации, словно горы золотого песка, он умел довольствоваться малым.
Светлые воспоминания людей, проходивших мимо него на площади, --- этого ему было вполне достаточно.

Рядом запела деревянная флейта-лоза --- сначала робко, потом всё смелее.
Я приглушил соло цитры и дал ей дорогу, вскоре невдалеке отозвались окарина и варган.
Классическая музыка Тси-Ди переплеталась с дикими мотивами народа сели удивительно гармонично.
Я дождался, пока они уступят мне место, и повёл мелодию по новой дорожке.
Флейта вступила через такт --- владел ею настоящий мастер.

Я взглянул на Митхэ --- она плакала, закрыв лицо ладонями.
По лицу Чханэ тоже текли слёзы, но она старательно подпевала моей песне тихим, почти не заметным контральто.
Многоголосье переливалось, словно горный ручей по камням, вплетая в себя новые и новые голоса, играя светом полночных звёзд и огней далёких домов.
Привычные уху голоса людей чередовались с хриплыми басами пылероев и нежными детскими голосками идолов.
Эта песня не имела слов.
Она не нуждалась в словах.
Каждый голос, каждый инструмент рассказывали свою маленькую историю --- о рождениях и смертях, радостях и печалях, дорогах и приютах, друзьях и врагах.

Грейсвольд откинулся на свод палатки и, прикрыв глаза, откровенно наслаждался музыкой.
Анкарьяль с выражением лёгкой ностальгии выдёргивала ниточки из разорванного рукава.

Аккорды шли друг за другом, цепляясь, сплетаясь и переходя друг в друга.
В моём сознании крутился квинтовый круг, на первый взгляд бессистемно, но каждый аккорд гармонично вставал на своё место, словно очередная цифра трансцендентной иррациональной математической константы.

Голоса один за другим умолкали, очарованные общим ассонансом.
Цитра и флейта нежно и устало добирали последние затухающие вариации.
Четыре аккорда, глубокие, как жизнь, шли друг за другом.

Уверенность. Сомнение. Печаль. Надежда.

Уверенность. Сомнение. Печаль. Надежда.

Уверенность.

Цитра тихо умолкла под надрывную, стонущую, зовущую трель флейты.

Над нашими головами пронеслись первые солнечные лучи.

\section{[-] Тхартху}

\textspace

--- Ты удивительно похожа на одну женщину, которую я знал, --- сказал старик.

--- Что за женщина? --- спросила Чханэ и погладила старика по плечу.

--- Купчиха, которой я служил, с редким родовым именем Катхар.

Чханэ перестала улыбаться.
Разумеется, это могла быть только одна ар’Катхар, единственная женщина, которая носила это имя до Чханэ.
Старик качался, словно одурманенный.

--- Если бы ты знала, как я перед ней виноват, девочка.
Если бы ты только знала...

--- Расскажи, лехэ, --- ласково прошептала Чханэ, --- в чём твоя вина?

--- Расскажу.
Всё расскажу, --- добавил он громко, заметив, что окружающие навострили уши.
--- Пусть все знают, какой я трус...

... Старик когда-то был её слугой --- выполнял поручения за небольшую плату.
По его словам, Тхартху ар’Катхар всё всегда делала сама, но отказывать парню, который хотел заработать лишние крупицы золота, не стала.
Постепенно он взял на себя и приготовление пищи, и уборку, и рассылку писем.
Не привыкшая к такой роскоши женщина вначале сопротивлялась, но потом уступила --- помощь Кхарама освобождала ей время для других важных дел.

Когда купчиха жила ещё в столице, она оказалась втянута в жреческий заговор.
Её бы это, может быть, и не коснулось, но под удар попал молодой жрец Сатракх ар’Сит, который незадолго до этого был сражён её белозубой улыбкой и лучистыми изумрудами глаз.

Однажды он не встретил её у входа в храм.
Тхартху знала, что он не мог забыть о встрече.
Быстрая разведка показала --- в храме что-то произошло, и любимого человека держали в темнице, готовя к алтарю.

У любви могучие руки, но слабые глаза.
Тхартху не стала выяснять, кто прав, а кто виноват.
Весь её многолетний опыт дипломатии сконцентрировался в одном простом желании --- вернуть любимого.

--- Она просто написала шесть писем и поручила мне их разнести некоторым людям в храме, --- рассказывал старик.
--- Причём некоторым нужно было отдать в руки, другим --- подкинуть в келью, а мимо некоторых людей нужно было просто пройти с письмом на виду, они сами отбирали его у меня...

В его глазах светилось давнее, неугасающее восхищение.

Что было в письмах, неизвестно.
Известен результат --- полный зал трупов: восемь воинов и двенадцать жрецов.
Письма Тхартху спровоцировали словесную перепалку на Верхнем этаже, затем снизу пришли возмущённые воины, и вскоре собравшиеся --- беспрецедентный случай в истории --- похватались за оружие.
Король-жрец едва избежал смерти --- по счастливой случайности он незадолго до столкновения отошёл в уборную.

Тхартху тем временем вошла в храм и забрала у мёртвого жреца ключи от темницы.
Они с Сатракхом безо всяких препятствий вышли на площадь, где влюблённых ждала оленья упряжка.

--- Мы ехали без остановок, целыми днями, --- вспоминал старик.
--- Хозяйка Тхартху и Сатракх сидели то в больших чёрных сундуках, то на козлах.
Они переодевались и меняли направление несколько раз, чтобы ввести в заблуждение встречных.
Последний раз она надела шёлковое зелёное платье, которое стоило ей целое состояние.
Тхартху часто носила платья, а не штаны, как большинство.
Она очень стеснялась своих ног, говорила, что они толстые и некрасивые.
Подумать только, ведь если кто-то видел её глаза, на ноги уже никто и не смотрел!..

Путь занял много дней.
Тхартху знала, что за ними едут, и твёрдо решила не уступать ни пяди дороги.
Оленей гнали насмерть, заменяя при первой возможности.

--- Мы уже сошли с Западного тракта и покатили по дороге на Предгорье, когда нас настиг убийца, --- горько пробормотал старик.

Поверенный Тхартху неожиданно получил в плечо дозу кураре и обмяк на козлах.
Затем прыгнувший с дерева воин придержал оленей, привязал безвольное тело к креслу и направился к сундукам чёрного дерева.
Первым ему попался сундук Сатракха.
Убийца откинул крышку и без предисловий ткнул жреца копьём, затем вытащил тело из повозки, оголил руку и начал срезать приметную татуировку --- лик Удивлённого Лю с каким-то изречением.
Тхартху, слыша снаружи возню, выпрыгнула из сундука с радостным: <<Я здесь!>>

--- Я никогда не забуду её лица, --- прорычал старик.
--- Убийца срезал татуировку, а она смотрела.
Просто смотрела своими лучистыми глазами-изумрудами, которые померкли в один миг.

Убийца рывком содрал лоскуток кожи, вытер окровавленные руки о штаны и фалангой обрубил постромки одного из оленей.

--- А меня ты не убьёшь? --- тихим, беззлобным голосом спросила Тхартху.

Убийца без выражения посмотрел на неё.

--- Ты живи, --- воин вскочил на освобождённого оленя и умчался обратно на север.

Когда слуга смог двигаться, Тхартху сидела и гладила блестящие волосы молодого, мёртвого как камень жреца.
На её лице не было слёз.
Только любовь и бесконечная благодарность.

--- Больше всего я боялся, что она сойдёт с ума, --- качал головой старик.
--- Напрасно я говорил ей, что нам нужно спешить...

--- Мне спешить уже некуда, --- просто отвечала Тхартху.

Наконец, после бесплодных попыток слуги её уговорить, Тхартху попросила оставить её одну.

--- Пружинка, --- ласково сказала она, потрепав парня по плечу.
--- Я освобождаю тебя от службы.
Возьми всё, что есть в повозке, оно твоё.
Мне отдай только стилет, тот, новенький, кошелёк с камнями, да хэситр налить не забудь.

--- Я любил её и не мог отговорить, --- со слезами на глазах бормотал старик.
Чханэ обняла его за плечи, окружающие как-то незаметно сели в круг, лицом к старику.
--- Повозку я привёл в Предгорье и подарил какому-то бедному крестьянину.
Остальное раздал детям.
Совесть не позволила мне взять ничего из вещей хозяйки Тхартху.

--- Где ты её оставил? --- спросила Чханэ.

Старик хлюпнул носом и задумался.

--- На горизонте уже виднелась пустыня.
Кажется, мы были где-то недалеко от Тхаммитра.

Мы с девушкой с улыбкой переглянулись.
Чханэ крепче обняла старика и, перед тем как сбросить с его шеи застарелый камень, мы оба успели представить --- каждый по-своему, конечно, --- как Тхартху ар’Катхар, в последний раз поцеловав любимого человека, вылила ему в полураскрытый рот хэситр, схватила кошель, стилет и двинулась сквозь пыльный горячий ветер к стоящему на краю пустыни городку.
Она ещё не знала, что в ней теплилась жизнь единственного ребёнка, которому суждено было дать потомство.

Зелёные глаза Тхартху сияли.

\section{[-] Возвращение}

\textspace

Я подошёл к мужчине.
Чересчур худой, нескладный, с длинными неухоженными волосами.
Он сидел, нежно улыбаясь, и гладил флейту.
Его глаза смотрели куда-то вдаль, не фокусируясь ни на чём.
На нём была потёртая, явно чужая одежда --- рубаха чересчур широка в плечах, на ногах вместо штанов --- мужская юбка ноа.

Я присел перед ним на корточки.

--- Как тебя зовут?

Он поднял на меня чистые, как утреннее небо, глаза.
Его лицо словно засветилось изнутри.

--- У меня нет имени.

--- Имя Атрис тебе о чём-нибудь говорит?

--- Это всего лишь имя, --- неопределённо ответил менестрель.
--- Оно красиво, как и все имена.

--- Ты знаешь, что я играл?

--- Эта музыка мне знакома.

--- Откуда?

--- Не знаю.

--- Кто ты?

--- Я хожу из города в город и играю на площадях.

--- Кто твои дарители?

--- Я их не помню.

Митхэ, заметив менестреля, подошла ближе.
Я поднялся.

--- Знакомься.
Это Атрис.

--- Что? --- не поняла родильница.

--- Вот этот менестрель --- твой погибший мужчина.

Митхэ смотрела на меня, как на сумасшедшего.

--- Я видела смерть Атриса своими глазами.
Да и потом, юноша совсем не похож на него и чересчур молод!
Он твой ровесник!

--- Знаю.
Но это и есть Атрис.

Митхэ присела на корточки перед странным парнем и потрепала его по щеке.
Тот посмотрел на неё лучистыми глазами, улыбнулся и отвёл взгляд.
Воительница вздохнула.

--- Я поняла.
Он меняет тела, как ты, Ликхмас?

--- Да.
Он хоргет.

Митхэ смотрела на юношу, наморщив лоб, пыталась и пыталась найти в нём знакомые черты...
Увы, их не было --- менестрель был нескладен и некрасив.
Наконец глухо спросила:

--- Ты знаешь Митхэ ар’Кахр?

Улыбка мужчины угасла.
Он долго смотрел на женщину, потом опустил голову.

--- Скорее всего, он не помнит тебя.
Бывают очень простые хоргеты, имеющие лишь программу заякоривания в теле и получения эманаций.
Они не помнят своих жизней, своих прежних имён.
Этот использует для получения эманаций музыку.
Возможно, всё, чем наполнена память парня --- музыкальные гармоники и паттерны.
Можно сказать, уникум среди хоргетов --- я о таких даже не слышал.

--- У тебя красивая цитра, --- неожиданно вмешался юноша.
--- Можно?

Я осторожно вложил инструмент в тонкие длиннопалые руки.
Юноша уверенно, не глядя покрутил колки, пробежался по струнам.
Потом приложил ухо к деке цитры и насладился отзвуком.

--- Хорошая.
Береги её.

--- Можешь взять.
Я дарю цитру тебе.

--- Это очень дорогой подарок, --- возразил юноша.
--- Я не могу его принять.

--- Она принадлежала погибшему воину из моего Храма.
Я думаю, хозяйка хотела бы, чтобы цитра попала в достойные руки.

--- Вот уж точно нет, --- поморщился Эрликх.
--- Она как-то сказала, что отдаст цитру только Печальному Митру, и то ему придётся попотеть, чтобы её впечатлить.

--- Её звали Кхохо ар’Хетр, --- заметил менестрель.
--- Я запомню это имя.

--- Откуда ты знаешь? --- поразился я.

--- Тут монограмма, --- голосом взрослого, объясняющего ребёнку самоочевидные вещи, пояснил юноша и, перевернув цитру, показал мне сложный полустёртый иероглиф на деке.
Его я не заметил.

Цитра была именной.
Я почти не сомневался, что её смастерили специально для Кхохо.
Кто сделал, за какие заслуги?
Воительница унесла эту тайну в пристанище.

--- У егво нетт имвени, --- вмешался один из воинов ноа, до этого наблюдавший за нами.
--- Мы савём егво Дели-хвон.

--- Да, они зовут меня Ласточкой, --- подтвердил юноша и засмеялся.
--- Хотя летаю я неважно.

--- Егво музыква силы первед бвоем, зажигваетт любвовь и заставляетт старый челвовек плаквать, --- продолжал ноа.

--- Это он, --- глухо пробормотала родильница.
--- Это точно он.

--- Я тебе сказал, что это он.

--- Послушай, Ласточка, --- прошептала Митхэ и опустилась перед менестрелем на колени.
--- У тебя есть родные?

--- Я один в этом мире, --- сказал парень.

--- Хочешь пойти со мной?

Менестрель долго всматривался в лицо женщины, её приоткрытый рот с зубами, похожими на обмытый рекой кварц, её чёрные губы, морщинки под раскосыми зелёными глазами...
И прошептал:

--- Я пройду с тобой до смерти и ещё...

--- ... и ещё пару шагов, --- всхлипнула мать и обняла его крепко-крепко.
Выпавшая из худых рук цитра грустно и нежно звякнула.

\section{[-] Лесные духи}

\textspace

Ко мне подошли три человека и поклонились.

--- Храни тебя духи, Ликхмас-тари, за избавление от Безумного.

--- Благодарите Хатлам ар’Мар, --- сказал я и кивнул на спящую Анкарьяль.
--- Она уничтожила Безумного, чтобы защитить друга, и едва не лишилась при этом жизни и бессмертия...

Пришельцы, не говоря ни слова, сели вокруг Анкарьяль.
Затем так же молча встали и ушли.

Вскоре люди стали приходить чаще.
Кто-то приносил мне еду, кто-то драгоценности.
На стене палатки ночью появился странный знак, похожий на охранительный знак лесного духа.
Но этот дух был мне не знаком.
Я спросил старика Хитрама о новоявленном знаке.

--- Это Самоотверженный Хат, --- уклончиво ответил Хитрам и быстро ретировался.

То же самое отвечали мне и остальные сели.
Чутьё подсказало --- новым лесным духом при жизни стала Анкарьяль Кровавый Шторм, живущая в теле Хатлам ар’Мар.

Я решил поговорить на эту тему с одним из старых жрецов.

--- Лесные духи на самом деле люди, которые посвятили жизнь служению, --- пояснил мне жрец после долгого молчания.
--- Мало кто знает, например, что Обнимающий Сит и девушка по имени Ситхэ, которой люди приносят в дар золото --- одна и та же личность...

Я рассказал о неожиданном открытии Грейсвольду, Чханэ, Митхэ и Атрису.
Грейсвольд по-доброму усмехнулся на свой обычный манер, Чханэ ахнула, а Митхэ вздохнула:

--- Да, так оно и есть.
Атрис как-то сказал, что он и есть тот самый Печальный Митр.
А я подумала, что он шутит... --- Митхэ с нежностью посмотрела на менестреля.
Тот улыбнулся и отвёл лучистый взгляд в сторону.
--- Печальный Митр учил меня играть на флейте, Печального Митра я любила всю свою жизнь.
Ирония...

\section{[@] Килограмм хлорофилла}

--- Я придумал даже их название, --- сказал Баночка.

Кошка поморщилась:

--- Может, они сами решат, как себя называть?

--- Если им не понравится моё --- так и будет, --- улыбнулся плант.
--- Но что-то мне подсказывает, что это название они примут.
Novai.

--- Красиво, --- признала Кошка.
--- А что значит?

--- <<Новые>>.

--- А их язык --- старейший язык Вселенной --- будет, значит, <<nova lingva>>? --- захохотала Кошка.
--- У тебя удивительное чувство юмора.

--- Вообще-то d'noveio lingva, но твой вариант благозвучнее, --- плант с серьёзным видом сделал пометку в своих записях.
--- А ты могла бы влюбиться в мужчину с хорошим чувством юмора?

Кошка на секунду отвлеклась от записей.

--- В этом случае мне пришлось бы вывести ужасное умозаключение.

--- Какое?

--- Что килограмм хлорофилла --- это не самое страшное, что может быть в мужчине.

--- Я красивый, --- обиженно заметил Баночка.
--- Да, зелёный, да, немножко низковат для вашего вида.
Но красивый.

--- Очень красивый, --- согласилась Кошка.
--- Поэтому давай ты будешь опылять девочек своего вымирающего вида, а не тратить силы на бесплодные сношения со мной.

--- Можно подумать, ты прям занята возрождением \emph{своего} вымирающего вида, --- грустно буркнул Баночка и с головой ушёл в работу.

\section{[@] Вот это поворот}

--- Я уже придумал первых персонажей для легенд, --- сказал Баночка.

--- И кто они?

--- Ты и я.
Если мне не суждено быть с тобой в реальности, может, мне удастся взять реванш в сознании потомков, --- улыбнулся плант.

Кошка покраснела.

--- Для тебя важно, чтобы мы были вместе?

--- Для меня нет ничего неважного, если это связано с тобой.

Кошка помолчала.

--- Завтра вечером я буду свободна и полна сил.

Баночка кивнул.

--- Кстати, как их будут звать?
Персонажей.

--- Имя твоего --- Ликх'аамас, <<бегущая по лесу>>.
Меня будут звать Муал'ликх, <<лежащий на цветах>>.
Через тысячу лет, конечно же, имена видоизменятся.
Но я надеюсь, что никто не забудет, что ты --- прекрасная большая белокожая женщина-человек, а я маленький, зелёный и лысый мужчина-плант.

--- Такое сложно забыть.

\section{[@] Меч Баночки}

--- А где твой меч? --- спросил я.

--- Я его спрятал на берегу, --- смущённо сообщил Баночка.
--- Знаешь, у царрокх есть истории про героев, которые находили чудесное оружие и боролись с врагами.
Я вырубил для меча крипту в одной скале и положил его туда.
Возможно, мой меч найдёт великий герой будущего.

--- Почему? --- удивился я.

--- Потому что я не герой, --- просто ответил плант.
--- Для меня ещё не пришло время владеть оружием.
Я вполне могу защитить себя, но ещё не знаю, где граница.

--- Какая граница?

--- Граница между защитой и нападением, --- сказал Баночка.
Он был серьёзен, как никогда.

\section{[@] Страх Машины}

\epigraph
{Мировоззрение не обязано соответствовать реальности на сто процентов.
Во Вселенной нет места нашим желаниям.
Оставьте для них угол хотя бы в её отражении.}
{Людвиг Вейерманн}

\textspace

Я понял, что Машина не была бездушной убийцей.
Она просто пыталась защититься от неведомой опасности всеми доступными ей средствами.
Это было похоже на страх ребёнка перед жуком, и по трагическому стечению обстоятельств жуком оказалась целая цивилизация.

Моя цивилизация.

--- Послушай, Безымянный, --- сказал я богу, --- я не могу развеять твои страхи.
В конце концов, я всего лишь тси, не меньше, но и не больше.
Я осознаю свою конечность и быстротечность и принял их как данность.

Безымянный слушал меня молча.
Остальные тоже притихли.

--- Я знаю, что в масштабах времени жизни Вселенной не только мои, но даже твои деяния ничтожны.
Пока горит любая из звёзд, цивилизация, подобная моей, может появиться и угаснуть десять тысяч раз, с моим участием или без него.
Всё, что мы можем --- это обеспечить себе комфортное существование, придумав смысл для того, что не может иметь смысла.

--- Придумай смысл для меня, Существует-Хорошее-Небо.
Я не знаю даже, где его искать, --- ответил Безымянный.

--- Я попрошу тебя об одной вещи, и она может стать для тебя смыслом.

--- Слушаю.

--- Я умираю, --- сказал я, чувствуя странную лёгкость после этих слов.
Да, я умираю, и что здесь такого?
Должно же это было когда-нибудь случиться.
--- Мне осталось от силы сутки.
Поэтому я прошу тебя --- не от имени тси, а от своего имени --- позаботься о моём народе и о тех людях, которые прибыли первыми.
У тебя достаточно силы, чтобы защищать их и помогать в трудные времена.
Они будут совершать ошибки, они не примут тебя сразу --- это неизбежно.
Но, если ты чувствуешь ту связь, которая между мной и тобой, исполни моё желание.

Безымянный молчал.
Тси, замерев, ожидали его ответа.

--- Я понимаю твои мотивы, --- наконец произнёс бог.
--- Ты принял за аксиому необходимость благоденствия твоего народа.
Эта аксиома мешает вам взглянуть на ситуацию математически, но...
Скоро ты исчезнешь, Существует-Хорошее-Небо, и без тебя мне будет... --- Безымянный помедлил, подбирая нужное слово, что означало колоссальную умственную работу, --- ... холодно.

Заяц тихо ахнула.

--- Я приму эту аксиому.
Это всё, что останется мне от тебя.

--- Благодарю тебя, Безымянный.

--- Я ещё ничего не сделал, --- ответил бог.
--- Для начала я побуду с тобой, пока ты не умрёшь.
Друзья ведь должны поступать именно так?

\section{[-] Добрый бог}

\epigraph
{Если бы ваш бог существовал и обладал качествами, которыми вы его наделили, он был бы ужасно одинок и несчастен.}
{Мартин Охсенкнехт}

\textspace

Атрис сидел у окна, обняв цитру, и смотрел на заходящее солнце.
Кто-то --- наверное, Митхэ --- вымыл его, дал ему новую чистую одежду, подстриг ногти.
Вычесанные волосы Атриса сияли.
Я подсел к нему.

--- Среди людей ходят слухи, что вернулся Безымянный бог.
Землетрясения утихли, пустыни впервые за много дождей расцвели...

Атрис улыбнулся и положил цитру на подоконник.

--- Много слухов ходит по улицам Яуляля.

--- Так как тебя зовут?

--- У меня нет имени.

--- Не может быть, чтобы у тебя не было имени.

--- На свете есть множество вещей, у которых нет имени.
Неужели они от этого страдают?

--- Тра-Ренкхаль будет занят демонами Ада, Безымянный, --- перешёл я к делу.
--- Ты это знаешь?

Атрис взглянул на меня.
Внезапно в чистых, как небо, глазах мелькнула древность.
Седая, поросшая мхом древность.

--- Да, знаю.
Он уже вами занят.

--- Есть ли тебе что сказать представителю Ада?

Атрис засмеялся.

--- Ты смешной.

--- А всё-таки?
Нам бы не хотелось занимать планету против желания демиурга.

Атрис разгладил складки на штанах.

--- Но вы это уже сделали, верно?
И я не могу вам помешать.

--- Не можешь, --- подтвердил я.

Атрис задумался.

--- Просто дай мне честный ответ.
Мой народ будет счастлив?

--- Ты имеешь в виду нгвсо?

Атрис помолчал, погладил цитру.

--- Первые люди пришли сюда с намерением завоевать Тра-Ренкхаль.
Подчинить его.
Выкорчевать деревья, посаженные не ими.
Я вернул их к природе, о которой они забыли.

--- Люди живут здесь испокон веков, --- сказал я.

--- \emph{Мой народ} живёт здесь испокон веков.

--- А люди, планты, кани и прочие, по-твоему, не имеют прав на Тра-Ренкхаль?
Ты видел прибытие их предков, но для ныне живущих это единственный дом!

Атрис внезапно снова засмеялся.

--- Для меня честь назвать тебя потомком, Аркадиу Люпино.

--- Ты уклоняешься от ответа, --- обвинил я его.

Атрис вздохнул.
Улыбка его потускнела.

--- Когда-то, уже после падения первых людей, сюда прилетел корабль.
На нём была жалкая горстка живых существ.
Они бежали, спасаясь от верной гибели, но не вели себя, как завоеватели, стремящиеся выжить любой ценой.
Хоть им и пришлось несладко, они отнеслись к планете с уважением.
Они считали нгвсо за равных, они спросили моего благословения.
Они просили меня позаботиться и о потомках первых людей, хоть те, на мой взгляд, и не заслужили такого обращения.
Они --- тоже мой народ.

Атрис снова улыбнулся, заметив в моих глазах понимание.

--- Так ответь мне, Аркадиу Люпино, --- продолжил Атрис.
--- Мой народ будет счастлив?

--- Да, --- потвердил я.
--- Нгвсо, тси и первые люди.
Неважно, сколько у них глаз, два, три или шесть --- Ад позаботится обо всех.

--- Тогда мне неважно, в чьи цвета будут раскрашены знамёна Тра-Ренкхаля, --- заключил Атрис.
--- И ты, демон, останешься для меня ребёнком от любимой женщины.
Какие бы тела мы оба ни носили.

Я вздохнул и обнял молодого менестреля.
Его слова пришлись мне по душе.

--- Скажи, почему ты назвался Безымянным? --- спросил я.

--- У меня действительно нет имени, --- пожал плечами Атрис.
--- Когда я только создал нгвсо, меня поразило то, что они давали друг другу имена, но отказывались давать имя мне.
Сейчас я понимаю, что таким образом нгвсо пытались отделить меня от их вида, показать мою уникальность, но тогда это стало причиной кризиса моей личности.
И вдруг появились тси, которые дали мне множество имён, словно я был одним из них.
Митрис, Атрис, Ласточка, Котелок...
Я запомнил все.

--- Как называть тебя мне?

--- Как хочешь, --- улыбнулся демиург.
--- Мне кажется, ты уже сделал выбор в пользу одного из них.

--- Ты прятался всё это время здесь?

--- Да, --- смутился Атрис.
--- Выбор был непростым, но я его сделал.
Я не мог победить, выступив открыто против Эйраки.
Я старался успокоить тектонические плиты, но землетрясения всё равно были чересчур частыми.
К тому же много энергии уходило на маскировочное устройство.

--- Сейчас всё позади, --- сказал я.
--- Тебе больше не будет надобности держать плиты --- я вызвал специалистов-геологов, они приведут планету в порядок.

--- Благодарю тебя, Аркадиу, --- тихо сказал Атрис.

Мы долго сидели в обнимку, вглядываясь в догорающий закат.
Под окнами мы заметили Митхэ.
Она посмотрела на нас, слабо улыбнулась чёрными губами и тут же поспешила сделать вид, что просто идёт по своим делам.
Мы проводили её взглядами до поворота улицы.

--- Скажи мне, Безымянный.
Чем тебе так важна простая женщина из людей? --- поинтересовался я.

--- Тысячи почитали творца, миллионы поклонялись богу, --- сказал Атрис и взглянул на меня чистыми, счастливыми глазами.
--- Но только одна человеческая женщина полюбила изгнанника.

\section{[-] Племя великанов}

\textspace

Рядом с Чханэ стоял её кормилец, высокий и крепкий оцелот с широкими чёрными очками вокруг глаз.
Увидев меня, мужчина приблизился и хмуро осмотрел меня со всех сторон.

--- Сойдёт, --- буркнул он и снова подошёл к дочери.
--- Зря ты выбрала Короля-жреца.
У них жизни нет, одни заботы.

--- Когда я его выбрала, он был мальчишкой, а не Королём-жрецом, --- развела руками Чханэ.
--- И вообще, спина моя, перестань.
Он хороший.

--- Плохих на Третьем этаже и не держат.
Мелкий только, --- ухмыльнулся кормилец.
--- Я уж думал, ты найдёшь себе высокого мужчину и нарожаешь племя великанов.

\textspace

\section{[-] Дом и флейта}

\textspace

Акхсар невидящим взглядом смотрел в гаснущий костёр.

--- Послушай, Золото, --- вдруг заговорил он.
--- Сколько у нас с тобой детей?

Присутствующие явно не ожидали такого вопроса.
Все, не исключая Грейсвольда, испытующим взором смотрели на воительницу.

Митхэ задумалась.

--- Имжу, Корешок, Капелька...
Остальных не помню.
Кажется, пятеро.
А почему ты вспомнил?

--- Забавно.
Сколько детей мы оставили на пути, сколько всего прошли бок о бок, а по-настоящему любим тех, кто не был с нами и пяти дождей.

--- Может, потому и любим, --- пожала плечами Митхэ и украдкой бросила взгляд на Атриса.

Акхсар кивнул, потянулся и вгляделся в тёмные джунгли.

--- В двух кхене поворот на Тхитрон.
Нужно выйти пораньше, чтобы успеть привести дом в порядок до заката.

--- Тогда прощайте, --- Митхэ встала, чтобы обнять друзей.
Кхотлам непонимающим взглядом уставилась на неё.

--- А ты... не с нами?

Митхэ смутилась и переглянулась с Атрисом.

--- Золото, --- сказал Акхсар, --- для тебя рядом со мной всегда будет место.
Кхотлам только за.
Кажется, вы с Хатом мечтали о доме?

Митхэ смутилась ещё более и попыталась скрыть это за возмущённым фырканьем.

--- Пожалуйста, не напоминай мне о нём.

--- Да ладно притворяться.
Ты светишься с самого Могильного берега.
Все и так уже знают, что Ласточка --- это Атрис.
Кстати, Хат, --- Акхсар хлопнул менестреля по плечу, --- с возвращением.
Не знаю, что ты такое, но я очень рад видеть тебя снова.

--- Я тебя тоже рад видеть, дружище, --- улыбнулся менестрель.
--- Прости, что заставил волноваться, когда меня приносили в жертву.
Мне следовало рассказать вам.

--- Он Печальный Митр в человеческом обличье, --- сообщила Митхэ.

--- Это правда? --- строго спросила Кхотлам у менестреля.
Тот улыбнулся.

--- А ведь я догадывался, --- Акхсар щербато оскалился и захохотал.
--- Кто ещё умеет так играть на цитре?

Я отметил, насколько старый воин помолодел со вчерашнего дня.

Вскоре веселье утихло.
Над джунглями пронеслись первые солнечные лучи.
Мы зачарованно уставились на светлеющее лиловое небо.
Первым опомнился Акхсар.

--- Хай, мы идём или яйца высиживаем? --- осведомился он.
--- Дом уже наверняка лианами зарос.
Кхотлам?
Атрис?
Митхэ?

--- Идём.
Дом не должен долго пустовать, --- весело сказал Атрис и начал собирать лежащую на земле амуницию.
Кормилица сладко потянулась и полезла в палатку --- будить Эрхэ.

--- А мне вот интересно, как ты её нашёл, --- полувопросительно сказала вдруг Чханэ.
--- Митхэ ар’Кахр.
Я понимаю, что вы боги, всё можете, но...

--- А ведь действительно, --- вскинулась Митхэ.
--- Как ты меня нашёл, Атрис?

Менестрель вместо ответа снял с шеи Митхэ кожаный мешочек, развязал его и вытряхнул на ладонь какую-то труху.

Обломки расписанной тростниковой флейты.

--- Я вмонтировал в неё микроскопические устройства, чтобы следить за твоим местоположением.

--- Ты носила с собой сухую травку с микроскопическими устройствами? --- поморщилась Чханэ.
--- Ты странная.
Я бы скурила ненароком.

Все рассмеялись.
Митхэ сквозь хохот объяснила Чханэ, что собой представляет <<травка>>.

--- Хай.
Там и от флейты-то ничего не осталось, --- тихо пробурчала Чханэ.

Но Митхэ её не услышала.
Она смотрела на Атриса и улыбалась, как и долгие годы назад.
По её лицу рассыпались счастливые морщинки.

--- Я же говорила, что я её сломаю.
Она хрупкая.

--- Я знал, что ты её обязательно сломаешь, --- засмеялся Атрис.
--- Поэтому вмонтировал не одно устройство, а шестьдесят четыре.

\section{[@] Последнее желание}

\epigraph
{Nvna dia de vita, nvna verba d'orata\FM!}
{Анатолиу Тиу.
Последнее слово на суде}
\FA{
Ни дня из жизни, ни слова из речей! (талино).
}

\textspace

После всех ужасов, которые я видел на Тси-Ди, эта планета кажется мне вполне сносной.
У нас есть надежда.
У нас есть шанс вернуться.

Потомок, который читает это.
Я не знаю, через что пришлось пройти тебе.
Я даже не знаю, поймёшь ли ты то, что я написал, но помни: недалеко от северного полюса небосвода есть семь ярких звёзд, таких, как я нарисовал.
Самая нижняя из них --- это она.
Помни это и никогда, никогда не теряй её из виду.
Расскажи о ней детям, расскажи о ней внукам, расскажи всем, кого ты знаешь, но только не теряй её из виду.
Помни о ней всегда.
Я знаю, что ты называешь домом совсем другую планету, но я тешу себя надеждой, что однажды тси вернутся домой.

Я на грани смерти.
Лучевая болезнь.
Пятьсот грэй не выдерживает ни одно млекопитающее.
Лишь у большой пчелы с изменённым метаболизмом есть шанс протянуть чуть дольше, и я благодарен предкам за этот шанс.
Ухожу я с лёгкой головой --- feci quod potui.
И если ты, потомок, найдёшь ту звезду на небе, если в тебе хоть на секунду вспыхнет мечта, которой я дышу --- значит, моя жизнь прожита не зря.

Золотой город.
Ущелье Мёртвого Ребра.
Я не помню, что хотел сказать.
Где я?
Ах да.
Кажется, начинается.
Дневник пора заканчивать.

Возьми мою руку, Костёр.
Я знаю, что ты меня слышишь.
Четыре к пяти, как тогда.

Я надеюсь ещё раз увидеть рассвет --- он здесь красив.
И Безымянного.
Он любит дождь.
И музыку.
Я надеюсь, что был для него... впрочем, это чересчур личное.
Пусть оно умрёт со мной.

\emph{Существует-Хорошее-Небо, до самой смерти своей инженер компьютерных систем Тси-Ди.}

\emph{Планета Трёх Материков, 12.003.28137.3.}

\section{[-] Давай дружить}

\textspace

--- А ты знаешь что-нибудь о мече Баночки? --- спросил я Атриса.

--- А, --- оживился он и спрыгнул с подоконника.
--- Идём покажу.

Мы вышли из Яуляля и шли около двух часов, пока не попали на узкую незаметную тропинку меж береговых скал пролива.
Эта тропинка и вывела нас к той самой крипте, которая описана в книге.

Атрис создал источник света и внёс его в крипту.
Красивые паутинчатые нервюры на потолке, тонкие колонны, резные арки --- у Баночки явно был талант архитектора.
Всё великолепно сохранилось --- я ощутил внутри камня тончайшую стабитаниумовую арматуру, снаружи же камень был покрыт вязким полимером.
В дальней нише стоял хрустальный ларец с мечом, под ним находилось что-то вроде гробницы.

Я подошёл ближе.
Да, классическая сабля народа сели, отличающаяся разве что материалом, внутренним устройством и искусностью гравировки.
Форму придумал Баночка.

--- И его так никто и не нашёл?

--- Его должен взять герой, --- сказал Атрис.
--- Бери ты.

--- Не-не, --- замахал я руками.
--- Здесь он смотрится гораздо лучше.

--- Митхэ тоже отказалась, --- тихо посетовал Атрис.

Мы рассмеялись.

--- А кто в гробнице? --- поинтересовался я и погладил резной камень крышки.
--- <<Лучшему другу, советчику и чесальщику спины>>.
Это та ворона, что ли?

--- Нейросеть, --- кивнул Атрис.
--- На самом деле это не гробница, а памятник, часовенка.
Прах Баночка развеял над морем, чтобы его подруга была свободной и летала вечно.
Очень умная для птицы.
Баночка по ней сильно скучал.

--- Как он умер? --- спросил я Атриса.

--- За работой, как и большинство тси, --- ответил менестрель.
--- Моя планета помнит их руки.

--- А Существует-Хорошее-Небо? --- продолжал допытываться я.
--- Он увидел рассвет?

--- Да.
Он увидел рассвет.
Последние полчаса, когда его мозг постепенно отключался, он повторял одну и ту же фразу.
Никто не понимал, кому она адресована.

--- Что за фраза?

--- <<Я был тебе хорошим другом?>>

\chapter*{Интерлюдия X. Смерть Чхалоса}
\addcontentsline{toc}{chapter}{Интерлюдия X. Смерть Чхалоса}

\textspace

В глазах окружающих, которые смотрели на него кто искоса, украдкой, кто прямо, не мигая, застыло одно --- смерть.
Обострённым чувством, которое не раз спасало его, Чхалас понял --- в вине яд.
Чуткий нос даже уловил едва заметный горьковатый запах лакового сока.

<<Хороша была жизнь, --- подумал Чхалас.
--- Негоже испортить её недостойной смертью>>.

Купец улыбнулся своей очаровательной хитрой улыбкой, которая смущала даже его заклятых врагов.

<<Скажи-ка, друг, --- обратился он к трактирщику, --- достойно ли подавать человеку хэситр, не освятив его ликами духов?
Мертвецы с такого пойла ещё неделю животом страдают!>>

И, с наслаждением наблюдая за вытянувшимися лицами окружающих, весёлый купец осушил чашу.
Его бездыханное тело опустилось на пол мгновение спустя.

Так умер Чхалас, посмеявшись над своими убийцами.
Пусть станут пищей ягуара те, кто рассказывает иначе!

\chapter{[-] Новый мир}

\section{[-] Тот, кто грёб}

\textspace

Зал застыл.
Все в недоумении смотрели на меня.

--- Я --- Король-жрец одной битвы. --- сказал я.
--- И эту битву я проиграл.
Не врагу, но природе, сотворившей людей, и трудам предков, сделавших природу этих людей ближе к совершенству.

Окружающие молчали.

--- Поэтому придётся провести выборы нового Короля-жреца, --- сказал я.
--- И я сразу предлагаю первую кандидатуру --- Трукхвал ар'Со э'Тхартхаахитр.

По залу пробежала волна шёпота.

--- Я отказываюсь, --- сказал Трукхвал.
--- Я прошёл путь до Ледяной Рыбы, и этот путь будет являться мне в кошмарах даже под пение Печального Митра.

--- Велика заслуга --- сбежать на север! --- крикнул кто-то.
--- Пусть нами руководит тот, кто понюхал пыль Могильного берега!

--- Продолжать жить порой тяжелее, чем идти на смерть, --- рассудительно заметил Хитрам-лехэ.
Мне захотелось подойти к старику и погладить его по голове.
Все уже забыли, что когда-то он поддерживал тиранию Картеля в Тхартхаахитре.
Все, кроме него.

--- В мирное время рука, привыкшая к перу, ценнее руки, державшей саблю, --- поддержал я, вызвав новую волну возмущённого шёпота.

Но вдруг слова попросила старуха, ведавшая складом.
Я кивнул.

--- Скажу, как умею, --- рявкнула она на весь зал, заставив шёпот утихнуть.
--- Трукхвал шёл с нами, разделяя все горечи и невзгоды.
Он сидел на вёслах вместе с моряками и не волынил, а грёб.
Он нёс поклажу и готовил пищу.
Он договорился с тремя кланами Живодёра, чтобы они пропустили нас к горам, и при нём не было даже кинжала, чтобы себя защитить.
Он встал на поле боя между северными дикарями и сели.
В него попал камень из пращи, но он дружелюбно говорил с дикарями на их языке, и дикари согласились дать сели место возле Хвоста, продали нам шкуры для плащей и шапок.
Я не понимаю ничего в ваших жреческих искусствах.
Вы можете выбрать другого, более достойного на ваш вкус, но, чтоб вас всех, не дать Трукхвалу право быть выбранным --- поступок, достойный глухого и слепого идиота!

--- Бабушка Вода, с меня хватит такой жизни!.. --- заговорил Трукхвал.

--- А тебя никто и не спрашивает, жрец! --- парировала старуха.
--- Ты клятву приносил?
Приносил, и я тому свидетель была!

--- Трукхвал, кошмары позади, --- сказал я.
--- Ты согласился принести жертву куда большую, чем та, которая ждёт нового Короля-жреца.

Мужчина опустил голову.

\section{[-] Книжный человек}

Голосование назначили на следующий день.
Я подивился тому, насколько по-разному отреагировали на Трукхвала простые люди и жрецы.
Вернувшиеся беженцы знали это имя очень хорошо.
Когда я назвал его имя, площадь взорвалась ликующими криками.
Его выбрали большинством голосов.

--- Моя речь будет краткой, --- сказал Король-жрец, подняв руку.
--- Как вам известно, Король-жрец, подобно купцу, может раз в четыре года наложить вето на решение Советов.
Я воспользуюсь этим правом сейчас --- и, клянусь, это будет последний раз, когда я сделаю это на своём посту.

Собравшиеся недоумённо переглянулись и забормотали.

--- Тридцать восемь дождей назад Митхэ ар'Кахр и воины её отряда чести были обвинены в Разрушении.
Многие из них стали кутрапами.
Я отменяю это решение для всех, мёртвых и живых.
Отныне они чисты перед законом.

Трукхвал вдруг опустился на колени.
Толпа замолчала.

--- Я прошу у них прощения за все беды, которые были причинены этим вынесенным второпях вердиктом.
Предки были мудры, придумывая законы, но они не могли знать всё.
Наша задача --- признать и исправить ошибки предков, чтобы дать пример потомкам, которым надлежит признать и исправить наши собственные ошибки.

--- Ему не дадут долго править, --- тихо сказала мне Анкарьяль.
--- Ад поставит на ключевые места своих людей.

--- Разумеется, --- сказал я.
--- Но кто лучше него отстроит страну после войны?
А когда его решат сместить... я постараюсь, чтобы Трукхвалу не причинили вреда.

--- Он сам будет рад уйти, --- заметил Грейсвольд.
--- Книжный человек.
Его стезя --- учить других.

--- И на посту Короля-жреца он будет заниматься именно этим, --- тихо добавил я.
--- Учить людей мирной жизни.

Друзья понимающе улыбнулись.

\section{[-] Непройденный путь}

\textspace

--- Аркадиу, что с тобой? --- Грейс, крякнув, подсел ко мне на камешек.

Я помолчал, собираясь с мыслями.

--- Народ Тра-Ренкхаля... чудесен.
Понимаешь ли ты, Грейс, что сейчас мы видим последних истинных сапиентов Ветвей Земли?
Тех самых, великих исследователей Вселенной.
Тех, кто не боялся лететь через тысячи парсак к неведомым мирам, кто познавал мир и отдавал жизнь за крупицу познания.
Это последние.
Остальных мы, демоны, давно превратили в марионетки селекцией.

--- Аркадиу.
Во-первых, ты не совсем...

--- Подожди.
Я читал некоторые отчёты о Тси-Ди.
Ад предлагал им свой протекторат, и не раз.
Они знали, что будут жить хорошо, что мы в этом заинтересованы.
Но они отвергли это.
Они выбрали свободу.
И они были единственными материальными существами, которые говорили с нами --- богами этого мира --- на равных.
И знаешь, Грейс... больно видеть, что это --- последние в своём роде.
Скоро их не станет.

--- Тра-Ренкхаль занят демонами Ада, --- возразил Грейс.
--- Они будут счастливы.
Разве нет?

--- А как же свобода? --- с горечью проворчал я.

Грейс вздохнул и хлопнул меня по плечу.

--- Они сами сделали свой выбор, Аркадиу.
У них был шанс стать равными нам.
Но они предпочли остаться людьми, лишь формально улучшив себя.
Что такое совершенствование регенерации, разгрузка генома, расширение диапазона воспринимаемых волн?
Идиоадаптации, жалкая попытка идти в ногу со временем.
Они создали нас --- существ более совершенных, и их единственным разумным выбором была демонизация.
Они её отвергли.
Поэтому мы победили.

--- Этого можно было избежать, --- пробормотал я.

--- Нельзя, --- отрезал Грейс.
--- Эволюцию не остановить.
Если кто-то твёрдо решил остаться прежним, он станет жертвой.

Технолог поднялся с явным намерением уйти.

--- И тебя устраивает роль хищника?
Паразита? --- бросил я ему в спину.

Грейсвольд посмотрел на меня с жалостью, и я впервые за всё это время ощутил, какая между нами пропасть.

--- Человеческий детёныш.
Что ты, что Тахиро, одни и те же вопросы.
И Анкарьяль от вас нахваталась.
Повторю специально для тебя --- всё это время я жду момента, когда пути сапиентов и демонов разойдутся.
Я воевал ради этого --- ради простора для учёных, ради того, чтобы оставить этот лоскуток Мира Фотона чуть лучше, чем он был при нас.
Я мечтаю о моменте, когда смогу пойти своей дорогой.
А вот ты, выношенный женщиной оцифрованный мозг, сможешь последовать за мной?
Сомневаюсь.

\section{[-] Два шага после смерти}

\textspace

Атрис с родильницей давно умерли.
Но Безымянный, разумеется, не бросил свой народ и после смерти бренного тела.
Только сейчас...

--- Истину тебе говорю! --- горячился старый жрец в питейном доме.
--- Ходит по джунглям парень с цитрой, и цветы от его игры распускаются!
А с ним женщина-северянка постарше чуть, в боевой раскраске чёрной, на Митхэ на нашу похожая!

--- Хватит заливать, старик! --- смеялись завсегдатаи.

--- Светятся, как жуки-фонарики!
И ходят, и говорят о своём о чём-то, и смеются, как серебряные серьги звенят!
А я как Согхо кликнул --- нет их, будто и не было! --- кричал старик.

--- Хлебнул, лехэ, лишка!
Иди проспись!

--- Кормилица!
Милая!
Я Безымянного встретил и Митхэ-воительницу! --- ребёнок десяти дождей дёргал кормилицу за платье.
--- Она меня поцеловала в щёку!
У неё цветы в волосах сами растут!

--- Ну что ты такое говоришь, Ликси, --- испуганно забормотала женщина.
--- Нельзя про богов выдумывать ничего.
Иди играй.

Слухи, конечно.
Один сказал, второй повторил.
Не станет Безымянный устраивать эти игры с иллюзией ради того, чтобы его увидел один старик или мальчик.
Да и превратить Митхэ в хоргета он не мог.
Аппаратура нужна, энергия.

Был, конечно, один вариант.
Безымянный мог интегрировать её личность в свою программу --- бог стал бы двуединым.
Две личности, соединённые воедино программой плюс-сингулярности --- неразлучные и вечно молодые.
Но это означает сбои, проблемы совместимости.
Не мог разумный хоргет пойти на такой жуткий риск.

Или мог?

\section{[-] А дом стоял}

\epigraph{
\ml{$0$}
{И прошел ливень, и вздулись реки, и подул ветер, и обрушились на дом тот, --- а он не рухнул, ибо основание его было на скале.}
{The rain came down, the streams rose, and the winds blew and beat against that house; yet it did not fall, because it had its foundation on the rock.}
}{Сапфировая Книга, Прядь Матеуша, 7:25}

\textspace

Мы с Чханэ шли, держась за руки.
Вокруг царил вечерний сумрак, в котором отчётливо виднелась громада храма --- точь-в-точь как во время нашей первой прогулки.
Вдруг я остановился.
Мой взгляд упал на купеческий двор.

Мой дом.

--- Я подожду тебя в храме, --- шепнула Чханэ.
Она легонько сжала мою руку и отпустила её.

Я сделал несколько шагов, не веря в происходящее.
Я ожидал, что Ликхмас ар'Люм воспримет это как возвращение домой;
но то, что Двор назовёт домом Аркадиу Валериану Люпино, стало для меня полной неожиданностью.

Секхар --- и я, словно мальчишка, бросился домой.
У меня из глаз катились слёзы.

Я погладил резную колонну, украшенную змеями и ликами духов, и ещё раз восхитился мастерством зодчего.
Война началась и окончилась, а дом по-прежнему стоял, как будто ничего не произошло.

Дверь подалась удивительно легко, гораздо легче, чем всегда.
Резные перила в виде разинувших рот змей, ступеньки --- одна, две, три... тринадцать.

Очаг пылал.
Манэ и Лимнэ одновременно повернули головы ко мне.
Плетение, петли которого они секхар назад молниеносно перекидывали друг другу на пальцы, замерло.

--- А я тебе говорила, что он придёт сегодня, --- наконец сказала Лимнэ сестре.

\section{[-] Пирог}

\textspace

--- А, это наш мужчина пришёл, --- спохватилась Лимнэ.
--- Согхо.
Несмотря на имя, он немного неразговорчив, а сегодня ещё и устал.
Не тревожь его беседой, хорошо, Лис?

В зал зашёл мужчина чуть постарше меня, сел за стол и зачерпнул себе каши из горшка.
Я кивнул ему.

--- Король-жрец, --- поклонился мужчина и невозмутимо принялся за еду.

\ml{$0$}
{--- Уже нет, --- улыбнулся я.}
{``I was,'' I smiled.}

\ml{$0$}
{--- Это навсегда, --- протянул мужчина.}
{``It can't be undone,'' the man drawled.}
Это были последние слова, которые я услышал от него за вечер.
Закончив еду, Согхо по очереди поцеловал Манэ и Лимнэ, затем ушёл спать в их комнату.

--- Он хороший, --- заверила меня Манэ.
--- Кхотлам говорила, что каждой из нас лучше завести своего мужчину, но найти двух неразлучных мужчин сложно.

--- Поэтому мы нашли Согхо, который любит нас так, словно мы --- одна женщина, --- подхватила Лимнэ.
--- Это тоже большая редкость.

В конце трапезы Манэ поставила на стол красивый пирог, украшенный плетёнкой.

--- Мы ведь пообещали...

--- Что обещали? --- удивился я.

--- Сегодня первый день Тростника, --- укоризненно сказала Манэ и указала на пирог.
--- Неужели ты забыл, братик?

\chapter*{Интерлюдия последняя. Ярость Маликха}
\addcontentsline{toc}{chapter}{Интерлюдия последняя. Ярость Маликха}

<<Почему?>> --- закричал Маликх.

<<Зачем будут нужны воины, если не от кого будет защищаться?>> --- ответили ему нападавшие.

\textspace

Маликх метался, словно пламя пожара.
Его клинки и вражеские стрелы встречались остриё к острию, нежились, словно целующиеся карпы, а затем стрелы возвращались к пустившим их.
Вокруг Маликха была стена из двух клинков, но она не была подобна стене из камня.
Лунное молоко ослабило его, а врагов было слишком много, и иногда кто-то оказывался достаточно умел, чтобы порезать Маликха.
Один порез, два, три...
Вскоре по телу воина текли тринадцать крохотных, но упрямых ручейков, отнимавших его силы и скорость.

Понял Маликх, что не защитить ему Ликхмаса мастерством.
Схватился он за эфес Клинка Осколков, что висел у него на поясе, и взвыли окружившие Маликха враги.
Раз за разом разлетался Клинок вдребезги и отрастал снова, словно соляной кристалл на солнце;
каждый осколок остротой превосходил кукхватр и обсидиан, и летел он, не разбирая дороги.
Кровь лилась реками, вой и плач стоял такой, что притих в ужасе весь город.

Вздохнул Маликх;
полегче пошла работа.

<<Ликхмас! --- позвал он друга.
--- Собирайся, нужно уходить!>>

Но Ликхмас ему не ответил.

\textspace

<<Почему?>> --- прошептал Ликхмас.

<<Зачем будут нужны жрецы, если некому приносить жертвы?>> --- ответил ему Кхарас.

\textspace

Дым выжигал глаза, огонь лизал волосы Маликха, оставляя обгоревшие чёрные клочья и лёгкий пепел, но воин не двигался с места.
На его руках, как спящий ребёнок, лежал Ликхмас.
Маликх легонько потряс друга за плечи, словно надеялся, что тот проснётся...

Вдруг взгляд Маликха упал на злополучный камень, лежащий в отрубленной женской руке.
Маликх схватил его и сжал до хруста в ладонях.
Из глаз воина брызнули слёзы, и он прошептал:

<<Ты не достанешься никому.
Никто и никогда тебя не найдёт!
Никто и никогда!>>

Маликх бросил кихотр.
Ярость и боль воина пылали жарче дома, в котором он остался.
Усиленные божественной мощью кихотра, они внезапно соединились с земным пламенем.
Огонь, окрасившись в ужасный синий цвет, как живой, накинулся на соседние дома.
Закричали заживо горящие в собственных жилищах люди.
Треск дерева и скрежет раскалённого камня сливались в жуткие повторяющиеся слова: <<Никто и никогда!
Никто!
И никогда!..>>

Вскоре от города не осталось ничего, кроме голой седой земли с застывшими реками расплавленного камня.
Немногие выжившие, сбившись в угрюмые ватаги, сразу же отправились в другие поселения --- на пепелище не осталось ни одного мертвеца, и не нашлось у людей хэситра, который можно было бы вылить в несуществующие рты.

\chapter*{Историческая справка}
\addcontentsline{toc}{chapter}{Историческая справка}

\textbf{Безопасная секция архива Ордена Преисподней\\
Выписка \#103:4AD-0000C}

~

Аркадиу Шакал Чрева через год безукоризненной работы в отделе культуры самовольно улетел с планеты.
Пятьдесят лет спустя он был обнаружен на планете Тра-Ренкхаль, где попытался возродить независимую сапиентную цивилизацию и создать союз урождённых сапиентов --- Скорбящих.
Его планы были раскрыты агентами Ада.
Аркадиу был схвачен и осуждён.
Его сообщникам, среди которых был демиург планеты Митрис Безымянный, а также некоторым из Скорбящих удалось скрыться.

23.0004.34198 Аркадиу Шакал Чрева был казнён.

Исполнитель --- Анкарьяль Кровавый Шторм.

Наблюдатели --- Грейсвольд Каменный Молот, Стигма Чёрная Звезда.

Казнь Шакала послужила прецедентом к масштабному исследованию, в результате которого оцифровывание низших форм жизни было запрещено законодательством Ада, а все уже зачисленные в штат урождённые сапиенты подвергнуты кардинальной коррекции личности.
Отделом 100 была проведена усиленная чистка.

В настоящее время достоверно выявлены следующие связи персонажей с реальными личностями:

Тханэ ар'Катхар --- Таниа Янтарь, интерфектор Скорбящих (ликвидирована).

Тхартху ар'Хэ --- Тхартху Танцующая Тень, визор Скорбящих (ликвидирована).

Митхэ ар'Кахр --- женская сущность бога Митриса Безымянного, мелиоратора Скорбящих (ликвидирована).

Атрис --- мужская сущность бога Митриса Безымянного, мелиоратора Скорбящих (ликвидирован).

Митхэ ар'Тро --- Микиа Седая, один из ведущих стратегов Скорбящих (в розыске).

Манэ ар'Люм --- Маниа Нарисованная, диктиолог Скорбящих (в розыске).

Лимнэ ар'Люм --- Лимниа Грустный Хвост, диктиолог Скорбящих (в розыске).

Записал: Кес Бледный Глаз

Проверили: Хара-лита Плачущий Клинок, Лимн Кард из клана Тахиро.

\part{Скорбящие}

\chapter*{Предисловие}
\addcontentsline{toc}{chapter}{Предисловие}

Это отвратительная книга.
И дело даже не в отсутствии у меня таланта к писательству.
Отвратительны по сути описываемые в ней события --- так же отвратительны, как вскрытие гнойника, забой животного или прочистка засорившейся канализации.
Едва ли кто-то усомнится, что иногда это необходимо;
но искать в сделанном честь и тем более красоту бесполезно.
Происшедшее не спасают и не оправдывают даже те крупицы нежности и тепла, которые я постарался вложить в книгу.
И если кто-то другой, описывая то же самое, заикнётся о героизме --- можете смело назвать его идиотом или лжецом.

\textspace

Здесь нужно дать кое-какие объяснения.
Демоны не сразу научились <<видеть>> сапиентов.
Долгое время связь с телом была единственным способом почувствовать Мир Фотона.
Чувствительность первых демонов оставляла желать лучшего --- мы не могли засечь спящих, чрезвычайно сложной задачей было выслеживание животных.
Первые люди не могли нас уничтожить;
однако благодаря нашему несовершенству они дали нам достойный отпор.
Не так-то легко оказалось превратить в рабов тех, чьи предки были свободны на протяжении двадцати пяти тысячелетий.

\textspace

Возможно, кому-то история покажется неправдоподобной, но всё описанное здесь --- чистая правда.
Это не легенда и не миф, всё происходило у меня на глазах.
К сожалению, моего друга, от имени которого ведётся повествование, уже нет в живых.
Но я вам скажу --- то, что он дошёл до того финала, до которого дошёл, само по себе стечение обстоятельств с вероятностью, стремящейся к нулю.

Пусть вас не огорчает смерть Аркадиу.
Когда-то он сказал мне, что множество существ, встретившихся на его пути, живут в нём, и для многих это единственная жизнь, которую они имеют.
Аркадиу продолжает жить во мне.
А у меня, надо признаться, странное ощущение: я увижу смерть моей Вселенной.
Я, не самый достойный из её детей, буду наблюдать, как гаснут звёзды и галактики, как испаряются чёрные дыры, и мне даже доведётся сказать милой старушке последнее <<прощай>> --- за миг до того, как тишина сменится молчанием.

Я не знаю, откуда это ощущение.
Но оно жило во мне с того самого момента, как я осознал свою конечность.

Радует одно --- если я и доживу до столь почтенного возраста, то только по своей инициативе и никак иначе.
Читателям же, которые вынуждены уходить из жизни, едва переступив её порог, я скажу --- относитесь к смерти спокойнее.
Это граница жизни, но границы есть у всего, что имеет название.
Наверное, именно поэтому я принял решение дописать творение Аркадиу --- он не закончил историю, оставил своего героя на перепутье, пытающимся найти ответ на важные вопросы своей эпохи.

Я не питаю надежды, что кто-то поймёт мои желания и стремления.
Открывая книгу, раб видит то, что хочет, свободный --- то, к чему готов.
Истину же не дано увидеть никому, даже автору --- книга часто говорит больше, чем он хотел бы сказать, и даже больше того, что он осознаёт.
Я лишь надеюсь, что эти записи будут для мыслящего существа маяком в озерце его короткой, но важной жизни.

\emph{Искренне ваш, Г.}

\chapter{[:] Закат эпохи}

\section{[:] Закат}

\ml{$0$}
{--- Красивый закат.}
{``A beautiful sunset.''}

\ml{$0$}
{--- Ага.}
{``Yeah.''}

Огонь жадно пожирал дымящийся концентрат.
Где-то вдалеке вторила треском обсидиановая глыба, обрамлённая полуостывшей лавой.
Поднималась Четвёртая Луна, похожая на непрожаренную каменную рыбу.
Деловито попыхивал молодой вулкан Арашияма, окутывая рыбу мягкими клубами дыма.
С сейсмологической станции уже сообщили, что скоро Арашияма взорвётся, принеся на значительную часть Преисподней незапланированную пепельную зиму.
Месяц пепла, сыплющегося с тёмных небес, затем ещё три месяца пахнущего серой снега.
И горе тому, кто попытается утолить жажду талой водой...

Вулканическая пустошь навевала тоску.
Жить там было не то чтобы невозможно, а скорее неспокойно.
Старые вулканы на берегах Разлома обильно удобряли и без того плодородную почву;
люди утаптывали ил и водорослевый перегной в любую расщелину, способную их удержать.
Молодые же устраивали неприятные сюрпризы вроде пирокластических потоков, вынуждавших аборигенов заселять только определённые возвышенности, а ещё регулярно затягивали небо тучами, проливавшими пепельный снег, неурожай и голод.
Возле Арашияма люди не селились.
По крайней мере, так полагали до недавнего времени.

Фигура, закутанная в чёрный плащ, нетерпеливо шевелила посохом дымящиеся блоки полимера.
Вторая, грузная и слегка неловкая, лениво ковыряла ножом землю.
Со стороны казалось, что разговор не клеится, но вряд ли это соответствовало истине.

\ml{$0$}
{--- Лу, что по времени?}
{``Lu, timeline?''}

\ml{$0$}
{--- Время есть.}
{``There is time.''}

\ml{$0$}
{И снова молчание.}
{Silence again.}
\ml{$0$}
{Звезда медленно коснулась горизонта.}
{The Star slowly touched the horizon.}

Фигура в чёрном вынула из складок плаща странное приспособление, похожее на металлический веер с раструбом.
Веер заплясал в дыму и зажужжал, то втягивая в себя дым, то выдувая его плотными длинными клубами.
Одно движение, два, три --- и в темнеющие небеса отправилась вязь призрачных знаков.

Звезда, осветив дымовую надпись, села за горизонт, и в небе засияли фиолетово-красные змеи.
Они извивались, трепетали, как флаги на ветру.
Восхищённо крякнул толстяк, бросивший на время свой нож;
его товарищ откинул капюшон, обнаружив тугие белые кудри и гордое молодое лицо.
Оба, не отрываясь, вглядывались в сапфировые небеса, где разворачивался поистине божественный спектакль.
Но длилось это зрелище недолго --- махнув хвостом, самый большой аспид рассыпался в воздухе жемчужным пеплом вслед за последним лучом Звезды.
Наступила ночь.

Улыбающиеся мужчины посмотрели друг на друга.

\ml{$0$}
{--- Никак не могу к этому привыкнуть, Грисвольд, --- смущённо признался молодой.}
{``Can't get used to that, Griswold,'' the young shyly said.}

\ml{$0$}
{--- Да, --- вздохнул толстяк.}
{``Yeah,'' the fat man sighed.}
\ml{$0$}
{--- Такая красота --- и в таком месте.}
{``Such a beauty in such a place.}
\ml{$0$}
{Подумать только.}
{Just imagine.}
\ml{$0$}
{В моём родном мире было всё --- изумрудные леса, сапфировые моря.}
{My home world has anything--- emerald forests, sapphire seas.}
\ml{$0$}
{Но такого не было.}
{But nothing like that.''}

\ml{$0$}
{--- А ты сам откуда?}
{``Where are you from?''}

--- Третья планета системы IF-517, <<Фомальгаут>> по-старому, --- Грис махнул рукой куда-то в сторону горизонта.
--- Меня туда послали ещё первые люди.

\ml{$0$}
{--- Первые люди? --- ахнул Лу.}
{``First humans?'' Lu gasped.}
\ml{$0$}
{--- Сколько ж тебе тогда?..}
{``How old are you, then ... ?''}

Грисвольд горько усмехнулся и вместо ответа снова махнул рукой.

\ml{$0$}
{--- Так ты бог.}
{``So, you're a god.''}

\ml{$0$}
{--- Творец и владыка, --- пошутил толстяк.}
{``Creator and Lord,'' the fat man joked.}
\ml{$0$}
{--- Ладно, хватит болтовни.}
{``Well, I'm done talking.}
\ml{$0$}
{Отдых закончен.}
{Playtime's over.}
\ml{$0$}
{Ждём ответа, твой квадрант --- номер четыре.}
{Wait for an answer, quadrant number four is yours.''}

\ml{$0$}
{--- Грис, а как ты сюда попал?}
{``Gris, tell me, how did you get here?}
\ml{$0$}
{И что у тебя за странное имя?}
{And what kind of name is `Griswold'?}
\ml{$0$}
{Я не узнаю язык.}
{I can't place the language.''}

--- Потом, --- в голосе Грисвольда прозвучала сталь.

Крякнув, Лу сел на камень и принялся посохом ворошить непрогоревший полимер.

\ml{$0$}
{--- Напомни, Грис, зачем мы торчим в этой неуютной обгорелой пустоши?}
{``Remind me, Gris, why are we stuck here, in that unpleasant burned wasteland?''}

\ml{$0$}
{--- Мы ждём сигнала, --- недовольно буркнул толстяк.}
{``We're waiting for a signal,'' the fat man grunted.}
\ml{$0$}
{--- Если повезёт, люди ответят и мы сможем узнать местоположение следующей явки.}
{``If we're lucky, humans will answer with the coordinates of the next safe house.}
\ml{$0$}
{Твой квадрант --- номер четыре.}
{Quadrant number four is yours.''}

\ml{$0$}
{--- И это приблизит нас к обнаружению бункера, --- голос Лу был предельно насыщен сарказмом.}
{``And it gets us one step closer to the bunker,'' Lu's voice was saturated with sarcasm.}

\ml{$0$}
{--- Сайгон, учитывая обстоятельства, я не вполне понимаю, почему ты вызвался идти со мной, --- устало сказал Грис.}
{``Saigon, given the circumstances, I don't quite understand why you volunteered to go with me,'' Gris wearily said.}
\ml{$0$}
{--- Тем не менее, у меня есть приказ, который я намерен выполнить, чтобы сохранить свой ранг.}
{``Nevertheless, I've got an order I intend to carry out, which allows me to keep my rank.}
\ml{$0$}
{Это приказ твоего отца.}
{Your father gave me that order.}
\ml{$0$}
{Арракис сказал --- использовать все имеющиеся способы, чтобы отыскать бункер, и это один из них.}
{Arrakis said: `use every possible way to find that bunker', and what we're doing here is one of those ways.}
\ml{$0$}
{Твой квадрант --- номер четыре.}
{Quadrant number four is yours.''}

Лу недовольно замолчал.

Дррррр... --- странный крик, похожий на птичий, разнёсся по выжженной каменной пустыне.
Друзья даже не изменили позы --- только левые руки почти синхронно вынули пулевые пистолеты.

--- Не двигайся, --- шёпотом предупредил Грисвольд напарника.

Минута прошла в томительном ожидании.

--- Проверим? --- шёпотом предложил Лу.
--- Сидим, как чайки на столбах.
Давно бы сняли, если б захотели.

--- Я посмотрю.
Подержи меня.

Лу аккуратно сел рядом с толстяком и обхватил его за пояс.
Грисвольд на секунду закатил глаза и обмяк.
Парень крякнул под весом навалившейся на него туши.

--- Извини, --- пробормотал толстяк и отряхнулся.
--- Тело Мистраль пустое.

--- Кто это сделал?

--- Не нашёл.

--- Что?
Как не нашёл? --- забывшись, парень заговорил во весь голос.
--- Так это не?..

--- Люцифер, тихо, волки тебя дери!

--- Извини, --- прошептал Лу.
--- Она жива?

--- Не знаю.
Уходим, сейчас же.
Про костёр забудь.

\section{[:] Мефистофель}

Дорога до лагеря лежала по склону Арашияма.
Извержение отгремело только вчера, и от тускло светящейся лавы за пять ярдов разило нестерпимым жаром.
Люцифер тихо, сквозь зубы ругался --- обычный на вид камешек на дороге в мгновение ока проплавил ботинок, выставив наружу, на съедение волнам огня голые пальцы.
Толстяк молча прыгал с одного валуна на другой, выбирая самые холодные места.

\ml{$0$}
{--- Стоять.}
{``Freeze.''}

Голос прозвучал как будто из ниоткуда.
Два путника подобрались, но тут же расслабились.
Свои.

\ml{$0$}
{--- \emph{Плесень и мрак.}}
{``\emph{Darkness and mold.}}
Грисвольд-тринадцатый, Люцифер из клана Мороза, --- прошелестел Грис.

Из почти не заметной с дороги пещерки вышел некто, закутанный в чёрный плащ.
В руке тускло блеснул пистолет.

\ml{$0$}
{--- \emph{Соль и табак.}}
{``\emph{Tobacco and salt.}}
Мефистофель из клана Мороза.
Как проходит эксперимент?

\ml{$0$}
{--- Мистраль исчезла без оповещения.}
{``Mistral just vanished without warning.}
\ml{$0$}
{Мы вынуждены были...}
{``We had to---''}

\ml{$0$}
{--- Мистраль жива, готова к отчёту и к поcледствиям своих действий.}
{``Mistral is alive, ready to report and face the consequences.}
Ваш транспорт я отогнал в безопасное место, заберёте его позже.

\ml{$0$}
{--- Арракис велел тебе встретить нас? --- удивился Грисвольд.}
{``Arrakis told you to meet us, then?'' Griswold wondered.}

\ml{$0$}
{--- Его, --- лаконично ответил Мефистофель, кивнув на Лу.}
{``Him,'' Mephistopheles succinctly answered, nodded at Lu.}

Мгновение спустя из пещеры выкатился маленький экспедиционный мотоцикл.
Мефистофель жестом велел друзьям садиться.
Худенький Лу поморщился, когда его зажало между каменной спиной Мефистофеля и животом Грисвольда, но промолчал.
Мефистофель, первая генерация клана Мороза, был напрочь лишён чувства эмпатии, и просьба потесниться в лучшем случае вызвала бы лишь холодный вопрос <<Зачем?>>.

Минуту спустя мотоцикл нырнул в густую завесу пара, и Арашияма скрылся из виду.

\ml{$0$}
{--- Так что там? --- поинтересовался Грисвольд.}
{``So what is this all about?'' Griswold asked.}

\ml{$0$}
{--- Вторжение минусов, --- бросил Мефистофель.}
{``\textit{Mina} invasion,'' Mephistopheles replied.}

Люцифер вздохнул.
Больше информации не будет --- иначе Мефистофель бы всё рассказал сразу.
Здесь начиналась его работа --- работа стратега.

\ml{$0$}
{--- Мало нам было этих подземных крыс, --- выругался парень.}
{``It was bad enough with those ground rats,'' he said.}
\ml{$0$}
{--- Хорошо ещё, что их технологии не позволяют убить демона...}
{``We should be thankful their technologies are unable to kill a daemon ...''}

\ml{$0$}
{--- Их технологий вполне хватает, чтобы портить нам кровь последние десять лет, --- возразил Грисвольд.}
{``Their technologies are good enough to play on our nerves for the last ten years,'' Grisvold answered.}
\ml{$0$}
{--- Где бункер?}
{``Where is the bunker?}
\ml{$0$}
{Мы так ничего и не узнали.}
{So far we've got nothing.''}

\ml{$0$}
{--- Я сообщу Арракису, что эксперимент был прерван по веской причине и в санкциях нет нужды, --- сказал Мефистофель.}
{``I will tell Arrakis that the trial was interrupted for a reason and there is no need in penalties,'' Mephistopheles said.}

\ml{$0$}
{--- Тысяча благодарностей, ты просто эталон доброты, --- саркастически проворчал Грисвольд.}
{``A thousand thanks, you're absolutely a standard of kindness,'' Griswold grumped sarcastically.}
--- Что нового насчёт диверсантов?

--- Есть данные, что они начали ходить по селениям и обучать дзайку-мару\FM\ распознавать инкарнатов, --- сообщил Мефистофель.
\FL{seijmar}{Сейхмар}
\ml{$0$}
{--- Команда Гало поймала одного агента.}
{``Halo's squad captured an agent.}
Точнее, раздобыла его труп --- он самоуничтожился ещё до начала допроса.

Мефистофель вывел мотоцикл на утоптанную дорогу и выжал полную скорость.
Друзьям пришлось замолчать.

\section{[:] Доверие}

--- И ещё одна неудача, --- Грисвольд тяжело опустился на скамью и забросил на спину полотенце.
%{``One more failure,''}
\ml{$0$}
{--- Третья по счёту.}
{``A third one.''}

\ml{$0$}
{--- Мы с самого начала занимались ерундой. --- буркнул Лу.}
{``We have been screwing around from the beginning,'' Lu said.}

\ml{$0$}
{--- Я тебе уже сказал: я не собираюсь обсуждать приказы твоего отца.}
{``I  told you, I'm not the one to question your father's orders.''}

\ml{$0$}
{--- Ты можешь меня выслушать?}
{``Can you just hear me out?''}

Грисвольд скрестил руки на груди.
<<Почему он так настойчиво пытается вывести меня на откровенность?>>

\ml{$0$}
{--- Люцифер, я не стратег.}
{``Lucifer, I'm not a strategist.}
\ml{$0$}
{Я простой технолог и делаю то, что мне говорят.}
{I'm a humble technologist and do what I'm told.''}

\ml{$0$}
{--- Я ещё раз тебя спрашиваю: ты собираешься выслушать то, что я говорю, или нет?}
{``I ask you again: can you just listen to what I'm about to say, or not?''}

Грисвольд несколько секунд смотрел в голубые глаза Лу и вздохнул.

\ml{$0$}
{--- Валяй.}
{``Go ahead.}
\ml{$0$}
{Что мы должны делать?}
{What should we do?''}

\ml{$0$}
{--- Мы должны подключить этих тама\FM.}
{``We should mobilize that tama\FM\ folks.}
\FL{tama}{Тама}
\ml{$0$}
{Они неплохо показали себя при поиске руды.}
{They proved themselves well in search of ores.''}

\ml{$0$}
{--- Арракис сказал --- людям доверять нельзя, --- ответил Грис.}
{``Arrakis said: humans can't be trusted,'' Gris answered.}
\ml{$0$}
{--- И в этот раз я с ним согласен.}
{``And this time I have to agree.''}

\ml{$0$}
{--- При чём здесь доверие?}
{``How trust is relevant?}
\ml{$0$}
{Нам с Гало нужны любые доступные данные.}
{Me and Halo need all data we can get.}
\ml{$0$}
{Верифицируем их мы, а не люди.}
{Verification is our job, not a humans'.''}

--- Тем, в бункере, тоже нужны данные.
И если все получат то, что хотят, то ситуация никуда не сдвинется.

Лу явно хотел возразить, однако, сделав выразительное движение бровями, промолчал.
\ml{$0$}
{<<Это ошибочное суждение, но и у тебя, и у отца недостаточно знаний, чтобы понять его ошибочность>>.}
{\textit{It's an error of judgement, but not father, nor you have enough knowledge to understand that.}}

\ml{$0$}
{--- Отец не хочет даже попробовать.}
{``Father refused even to try.}
Я предлагал ему разбить задачу на части, не вызывающие подозрения, и дать людям только их.

--- Лу, ты у нас глава Ордена?

--- Отец сделал меня и Гало стратегами.
Зачем мы ему нужны, если в итоге решение принимает он, и далеко не всегда лучшее?

\ml{$0$}
{--- Это и есть руководство.}
{``That's the essence of leadership.}
\ml{$0$}
{Учесть мнение каждой стороны и принять решение, обязательное для исполнения.}
{Take every opinion into account, then make a decision that everyone is to abide by.''}

\ml{$0$}
{--- Тогда он плохой руководитель.}
{``He's a weak leader, then.''}

\ml{$0$}
{--- Не вздумай ляпнуть такое при легионе.}
{``Wouldn't be clever to say that in front of legion.''}

\ml{$0$}
{--- Ты меня за идиота считаешь?}
{``You take me for an idiot?''}

--- Сайгон, нас очень мало, --- тихо сказал Грис, оглянувшись по сторонам.
--- Куда бы ни вёл нас Арракис, он ведёт нас всех.
Всех, понимаешь?
Делить силы --- худший расклад из всех.

\ml{$0$}
{--- Не надо пересказывать мне эти пропагандистские формулы.}
{``Stop repeating all that propaganda formulae to me.}
\ml{$0$}
{Я мог бы курировать людей сам, не подключая ни одной единицы из сил Ордена!}
{I could oversee that humans by myself, without mobilization of one single Order unit!''}

\ml{$0$}
{--- Тогда что тебе мешает? --- раздражённо бросил Грисвольд и отвернулся.}
{``So what's the hold-up?'' Griswold said with irritation, then turned his back.}

Лу изумлённо смотрел в спину толстяка.

\ml{$0$}
{--- Хватит прожигать дыру в моей спине, --- пробурчал технолог.}
{``Stop burning holes in my back,'' the technologist muttered.}
\ml{$0$}
{--- Хочешь --- делай.}
{``If you want it, do it.}
\ml{$0$}
{Я молчок.}
{I'll be quiet.''}

\section{[:] Переговоры с наблюдателями}

\ml{$0$}
{--- И тем не менее Тысяча Башен не совсем понимает, в чём сложность борьбы с восставшими людьми.}
{``...Nevertheless, Thousand Towers doesn't quite understand what's the problem with bringing human rebels into line.''}

\ml{$0$}
{--- Основная проблема в том, что этот бункер специально разрабатывался для борьбы с демонами, --- объяснил Лу.}
{``The main problem is that bunker was designed specifically to fight against daemons,'' Lu explained.}
--- Как вы знаете, Преисподняя не всегда была выжженной вулканической планетой, когда-то здесь было спокойно и даже зелено.
Когда глава нашего клана, Арракис, прибыл с Мороза вместе с Дорге, здесь шла война между колонистами с Земли и демиургом.
Люди использовали всю мощь имеющихся технологий, чтобы лишить демиурга масс-энергии, и заставили его уйти.

\ml{$0$}
{--- Мы слышали о том, что Преисподняя была выжжена демиургом в отместку, --- полувопросительно сказал наблюдатель.}
{``We heard rumors that Nether was scorched by its demiurge as a revenge,'' the observer said.}

\ml{$0$}
{--- Едва ли, --- подал голос Грисвольд.}
{``Hardly,'' Griswold said.}
\ml{$0$}
{--- Для демиурга причинить вред планете --- всё равно что отрубить себе руку.}
{``For a demiurge, harming their planet is like severing their own hand.''}

Наблюдатель непонимающе уставился на технолога.

\ml{$0$}
{--- В общем, для большинства это неприемлемо, --- поправился Грисвольд, оглядев бионические конечности собеседника.}
{``I mean, it's unacceptable for most of them,'' Griswold corrected himself, noticed bionicle limbs of the observer.}
--- Наши учёные считают, что это сделали люди.
Но лично я, как имевший опыт управления планетой, склоняюсь к тому, что биолитосфера не была стабилизирована и без контроля демиурга развалилась сама.

--- И вот через тысячу лет, когда люди уже растеряли большую часть технологий, а мы взяли власть, бункер был снова кем-то найден, --- сообщил Самаэл.
--- Видимо, там же были найдены старые записи, которые и объяснили людям, с кем они имеют дело.

--- Мы даже не уверены, что бункер в окрестностях Арашияма один-единственный, --- продолжил Гало.
--- Как не можем быть уверены, что таких сооружений нет где-либо ещё на планете.
Методология первых людей была отточена до совершенства, и современные дикари следуют ей неукоснительно, словно священным догмам.
Именно поэтому нам нужны ваши демоны.

\ml{$0$}
{--- Разумеется, --- улыбнулся наблюдатель.}
{``Of course,'' the observer smiled.}
--- Мы прекрасно понимаем, что эти сооружения должны быть найдены и изучены со всей возможной тщательностью, ведь это --- залог победы над Союзом.
Орден Тысячи Башен был бы чрезвычайно рад помочь вам и получить данные касательно технологий первых людей.

Айну коротко поклонилась.

--- Однако, к сожалению, у нас нет свободных демонов для выполнения этой задачи, --- закончил наблюдатель.
--- Тот резерв, который был подготовлен к работе с вами, пришлось срочно задействовать на Тысяче Башен.

--- То есть вы отказываетесь нам помогать? --- осведомился Люцифер.

--- Юноша, --- начал наблюдатель, --- вы превратно истолковали наши слова.

--- Во-первых, не <<юноша>>, а <<сайгон Люцифер>>, --- нахмурился Гало.

--- Прошу прощения, сайгон Люцифер.
Я только хотел заострить внимание на том, что <<отказываемся помогать>> и <<не имеем свободных демонов>> --- разные вещи.

\ml{$0$}
{--- Да что вы говорите, --- буркнул Грисвольд.}
{``How could we confuse one and the other,'' Griswold said.}

--- По вашим указаниям мы приготовили вам всё, включая тела, а вы говорите, что не имеете свободных демонов? --- вспылила Айну.
--- Чем это занят Орден Тысячи Башен, что свободных демонов нет?

--- Вы можете получить любую информацию о наших занятиях по официальному запросу, --- сухо ответил наблюдатель.
--- Я лишь сообщаю решение нашего штаба.

<<Ага, --- хмыкнул Гало.
--- Смиренный служитель, как же.
Мы как-то протестировали троих на полномочия.
Его полномочия соответствуют как минимум нашим с Лу, а молчун с краю, чьё имя я даже не припомню, вполне может быть вторым лицом в Ордене>>.

<<И сейчас, когда тайное убежище стало полем битвы, эти корольки решили аккуратно сбежать, провоцируя нас на высылку наблюдателей, --- подтвердил Лу.
--- Действуйте как угодно жёстко по своему усмотрению, но ни в коем случае не вздумайте их отпускать>>.

Айну и Самаэл кивнули.

--- Мы обязательно сделаем официальный запрос, --- сухо сказала Айну.
--- Я прошу прощения за последние слова, вы --- отличные исполнители.
И тем не менее, как я полагаю, у вас должны быть готовые объяснения отдельных аспектов.

--- Задавайте вопросы, и я отвечу, --- неприятно улыбнулся наблюдатель.

--- Были ли нападения со стороны Союза на Тысяче Башен? --- начал Самаэл.

--- Они и сейчас есть.
Не настолько серьёзные, как здесь, но тем не менее оттягивающие значительную долю наших сил.
Вот, пожалуйте.
Полный расклад...

<<Ложь, --- ухмыльнулся Лу, просмотрев данные.
--- Нападения они пресекли в корне, так как имели сношения с Союзом втайне от нас и получали разведданные от завербованных минусов>>.

<<Выкладка точна?>> --- спросила Айну.

<<Подтверждаю.
Статистика по потерям, --- кивнул Гало.
--- Они их завысили, но тем не менее иначе как грамотно и, главное, вовремя собранными разведданными такое не объяснить>>.

<<В общий котёл отправьте информацию по руководителям их разведки, --- добавил Арракис.
--- Имена, биографии.
Мне совсем не нравится, что они на шаг впереди>>.

<<А также по тому, насколько свободны они в своих действиях>>, --- вставил Лу.

Арракис кивнул.
Если он и понял намёк сына, то предпочёл этого не показывать.

--- Были ли попытки установить диалог с Союзом?

--- Были.
Берен, предоставь записи.
Как видите, мы даже попытались заключить мир, но уже во время подписания пакта последовала серия диверсий.

<< ... на Преисподней, --- закончил Гало.
--- Настолько виртуозная ложь, что неотличима от правды.
Они каким-то образом откупились от Союза, и я бы многое отдал, чтобы узнать, каким>>.

<<И оба раза они рассчитывали на то, что ложь будет раскрыта, --- добавил Лу.
--- Цель та же --- высылка наблюдателей>>.

<<О как, --- хмыкнула Айну.
--- То есть у нас ещё один возможный враг, который так и нарывается на кинжал>>.

<<Не совсем>>, --- возразил Гало.

<<Скорее это переход от сотрудничества к вооружённому нейтралитету, --- пояснил Лу.
--- Возможно, это и было предметом договора с Союзом: Орден Тысячи Башен не будет вмешиваться в конфликт и примкнёт к победителю>>.

<<Дьявол, мы не можем даже отозвать наших демонов с Тысячи Башен, --- посетовала Айну.
--- Потому что рычаг давления на Орден ну никак не лишний>>.

<<И они об этом знают, --- закончил Гало.
--- Вполне возможно, что они даже рассчитывают на остатки адских сил после нашего поражения.
Больше-то легионерам некуда будет идти>>.

<<Лично мне всё ясно, --- буркнул Арракис.
--- Самаэл, задай им для приличия ещё пять-десять умных вопросов, сохраняя дружественно-требовательный тон.
Люцифер, создавай конференцию прямо здесь --- нам срочно нужен план.
И ещё --- пусти параллельно слабозашифрованную меченую легенду.
Посмотрим, где у нас протечка>>.

\section{[:] Условия и ответ}

\textspace

Доклады поступали один за другим --- дезертировал третий отряд, затем восьмой.
У Ордена осталось меньше трети легиона, и, судя по настроениям, скоро не будет даже её.
Дзайку-мару, словно зная о происходящем, отказывались подчиняться представителям.
Вскоре поступило сообщение, что забастовку объявили носильщики.

Айну, как обычно, сорвалась с места в карьер.

--- Грисвольд, Люцифер, Мистраль, Гало.
Вы идёте со мной.
База в Такэсако разрушена.
Нужно найти и унести уцелевшее оборудование.

--- Прости, что значит <<унести>>? --- поинтересовался Гало.

\ml{$0$}
{--- Это значит, что ты возьмёшь всё барахло руками, взвалишь себе на плечи и ножками дойдёшь сюда, в лагерь.}
{``It means: all the garbage we can find you'll pick up by your hands, then carry on your back, then return here by your feet.''}

--- То есть узел дозаправки транспорта в Акияма также разрушен? --- уточнил Лу.

Айну промолчала.

--- Я стратег, а не носильщик, --- тихо сказал Люцифер.

\ml{$0$}
{--- Если не будешь носильщиком добровольно --- станешь трупом принудительно, --- посулилась Айну.}
{``If you're not a carrier by choice, you'll be a cadaver by force,'' Ain\"{u} promised.}
\ml{$0$}
{--- Союз Воронёной Стали об этом позаботится.}
{``Blued Steel Union gonna take care of that.''}

\ml{$0$}
{--- Я смотрю, дела у Ордена Преисподней совсем плохи, --- масляно улыбнулся один из наблюдателей.}
{``Things are going bad for Order of Nether, as I can see,'' an observer said with a smile of predator.}

\ml{$0$}
{--- Поблагодари меня, что пока ещё можешь смотреть, --- сказала Айну.}
{``Thank me, you still can see by now,'' Ain\"{u} answered.}

\ml{$0$}
{--- Вы же понимаете, что у ваших слов и поступков могут быть последствия? --- осведомился наблюдатель.}
{``I guess you understand that your words and your actions would have consequences?'' the observer asked coldly.}

--- Если ты сейчас же не закроешь рот, это поймут все и сразу.
До мельчайших подробностей, --- демон Айну предупредительно вспыхнул, приведя в готовность боевые модули.
Наблюдатель побледнел и вытянулся в струнку.

\ml{$0$}
{--- Если Орден Преисподней переживёт эту схватку, Тысяче Башен придётся понять и принять очень многое.}
{``If the Order of Nether outlives this battle, Thousand Towers will have many things to understand and accept.}
\ml{$0$}
{Или ответить, --- пояснил Люцифер, поняв, куда дует ветер.}
{Or payback,'' Lucifer explained---as soon as he's seen the way the wind blows.}

На этот раз побледнели все наблюдатели.
Айну одобрительно посмотрела на юношу.

\ml{$0$}
{<<Впервые слышу от тебя угрозу.}
{\emph{``The first threat of you I've ever heard.}}
\ml{$0$}
{Впечатляет>>.}
{\emph{Very impressive.''}}

\ml{$0$}
{<<Ты просто никогда не брала без спроса его косметику, --- ухмыльнулся Гало.}
{\emph{``You never borrowed his makeup without permission,''} Halo smirked.}
\ml{$0$}
{--- Я чуть не поседел раньше времени>>.}
{\emph{``My hair almost grayed before their time.''}}

--- Я правильно понял, что мы ваши пленники? --- вызывающе спросил наблюдатель, говоривший первым.

--- Вы --- куски мяса, пока ваша ценность не подтверждена Орденом Тысячи Башен, --- без обиняков ответила Айну.
--- Если подтверждения не последует, мы поступим с вами как с мясом.

Присутствующие легионеры как по команде взяли наблюдателей на прицел.
Айну кивнула Люциферу.

\ml{$0$}
{--- Малышня, я вас жду через минуту.}
{``Midgets, I want you ready in one minute.}
\ml{$0$}
{Из оружия возьмите только пистолеты.}
{Take only handguns, and no weapon more.}
\ml{$0$}
{Если нас там встретят --- оно в любом случае не поможет.}
{If we are met there, weapon will be useless anyway.''}

\section{[:] Тела на обочине}

\textspace

Люцифер поморщился, Айну подняла бровь, а Грисвольд почувствовал, что у него задрожали ноги.
Столбы вдоль дороги были увешаны человеческими телами без кожи.

--- Ну и вонь, --- сказал Люцифер и, достав респиратор, манерно прикрыл им нос и рот.
\ml{$0$}
{--- Кто это сделал?}
{``Who have done that?''}

\ml{$0$}
{--- Точно не мы, --- констатировал Гало.}
{``We've not,'' Halo stated.}
\ml{$0$}
{--- И не Союз.}
{``Union neither.}
Демоны не стали бы тратить время на такие глупости.
\ml{$0$}
{Кожу сняли уже после смерти.}
{They were skinned after death.''}

\ml{$0$}
{--- Проведи анализ, брат, --- попросил Люцифер.}
{``Analyze it, brother,'' Lucifer asked.}

Гало подошёл к трупу и внимательно осмотрел его со всех сторон, затем обнюхал подсушенную горячим ветром плоть.
Демон стратега издал слабое свечение.

--- Точность 0.904, --- сказал Гало.
\ml{$0$}
{--- Это сделали дзайку-мару.}
{``Dzaiku-maru have done it.}
\ml{$0$}
{Убивали и снимали кожу методично, без торопливости.}
{They were killing and skinning methodically and slowly.}
\ml{$0$}
{Многие убиты во сне.}
{Many of victims were killed in their sleep.}
Я думаю, что это вендетта, хотя впервые вижу, чтобы с трупами врагов так поступали.

--- Что могло спровоцировать вендетту? --- удивилась Айну.

--- Союз использовал сапиентные тела, --- немного невнятно предположил Люцифер из-за респиратора.
--- Маловероятно, что они следовали всем местным обычаям и традициям.
Дзайку-мару не видят демона, они видят тело и мстят телу.

--- Я согласен с Лу, --- вмешался Грисвольд.
--- Такие казусы регулярно случались на моей планете, если туда забредал очередной нейтрал-неумеха.
Чаще всего это заканчивалось смертью его тела, но обычно он успевал прихватить десяток-другой нападавших.

Друзья переглянулись.

<<Это твои тама?>> --- иронически спросил Грис.

<<Своё дело они сделали, --- ответил Лу.
--- Данные у меня, планирую подсунуть их в общее поле как случайную находку>>.

<<Главное, что ты ни о чём не жалеешь>>.
Лу почувствовал в словах технолога холодок.

--- Кстати, --- добавил Гало веселее, чуть обычно.
Все навострили уши.
--- Что вы думаете насчёт погоды?
Я чувствую слабый привкус цветочной влаги.

<<Есть вероятность, что за нами идёт лазутчик --- неуточнённое сапиентное тело.
Местоположение по эманациям определить не могу, но чувствую иные неуточнённые признаки присутствия>>.

--- Ты оптимист, Гало, --- рассмеялась Айну.
--- Погода не изменится.

<<Вести себя как обычно, следовать тем же курсом>>.

Айну кивком подтвердила приказ, и демоны, в последний раз посмотрев на страшную картину, двинулись дальше.
Скрипели и ворчали под ветром сохнущие верёвки.
Трупы махали вслед демонам мёртвыми руками, как махали всем прочим путникам, имевшим несчастье пройти мимо.

\section{[:] Мальчик с мечом}

\epigraph
{Счастье --- это когда из кошмара можно проснуться.}
{Пословица сели}

Через десять миль начало сказываться недосыпание.
Демоны изо всех сил пытались заставить свои тела идти, как обычно, но вскоре перестали помогать даже психостимуляторы.
Грисвольд вздыхал, Гало злобно пинал мелкие камни, Айну мрачно молчала.
Люцифер, который передал свой груз технологу, выглядел бодрее всех.

--- Чтоб эти тела, --- бормотал Гало.
--- Почему дзайку-мару не избавились от такой бесполезной вещи, как сон?

--- Во время сна мозг восстанавливает метаболический потенциал, --- заметил Люцифер.
--- Сон --- это песочница, в которой человек без вреда для себя может подготовиться к травмирующим ситуациям.
А ещё сон --- это дар видеть желанный мир.
Или хотя бы не видеть нежеланный.

--- Ещё немного, и я брошу эту тушу на дороге, --- пообещал Гало.
--- Я ей не пастух.

Лазутчик не показывался.
Айну пару раз наудачу просмотрела местность, но неведомый сапиент каким-то образом прятался от <<взгляда>> интерфектора.
Грисвольд предположил, что он искусственно ввёл себя в ступорозное состояние.

Тропинка круто свернула к северу, огибая скалистый холм.
Вскоре Гало, поколебавшись, попросил Грисвольда понести и часть его груза.
Технолог тяжело вздохнул и согласился.
Айну хотела сделать братьям строгое внушение, но Грисвольд покачал головой.
<<Не до этого>>.

Ветер переменился, и смешанный с дымом пар унесло в сторону.
В надвигающихся сумерках резко проступили очертания холма.

--- Смотрите туда, --- Люцифер, не оглядываясь, махнул на скалу.
--- Вон он, наш лазутчик.

Сонная дымка мгновенно улетучилась.
Демоны, остановившись, как будто невзначай встали в боевой порядок.

На скале стоял мальчишка.
Он опирался на дайту-соро\FM{} в ножнах, словно на посох.
\FA{
Дайту-соро --- ритуальное оружие Древней Земли, длинный изогнутый меч из стабитаниума.
Народ нихон, отправляя своих людей в космические путешествия, обязательно давал им дайту-соро как талисман.
}
Рваные лохмотья, едва прикрывавшие тело, развевались на сухом горячем ветру.
Мелкие острые камешки то и дело секли плоское смуглое личико, но мальчишка стоял и продолжал смотреть.
Ему было всего шесть-семь стандартных лет, но он держался, словно завоеватель на пьедестале.

--- Шпион, как пить дать, --- констатировала Айну, подтягивая ремешки куртки.
--- Гало, подпусти его поближе и потяни время, я устала и не хочу бегать по этим скалам.

--- Эй! --- закричал Гало, и эхо многократно повторило его крик.
--- Мальчик!
Подойди!

Мальчишка, поколебавшись, перехватил поудобнее оружие и начал спускаться.

--- Дзайку-мару, --- доложила Айну.
--- Как подойдёт на пять шагов, стреляйте в голову, кому там сподручнее, и пошли дальше.

Мальчишка спрыгнул с небольшого уступа и замер в шести шагах.
Айну достала пистолет, Люцифер и Гало замерли с руками на кобурах.

--- И что это такое? --- осведомилась Айну, заметив смятение товарищей.

--- Ты ж сказала --- в пяти шагах, --- буркнул Гало.

--- Я сказала --- кому сподручнее, --- заплетающимся языком поправила Айну.
--- Вы идиоты.
Может, вам расстояние сообщать в виде диапазона допустимых значений?
С точностью до микрометра?

--- Пристрели мальчишку, чтоб тебя, --- устало бросил Грисвольд.
--- Все хотят спать, а мы уже две минуты здесь торчим.

Пока демоны разговаривали, маленький оборванец подбежал к Айну и обнял её ноги.
Айну приставила пистолет к его виску.
Мальчишка поднял улыбающееся лицо, его огромные карие глаза взглянули на демоницу.

--- Ты красивая, ситу-ну-онна\FM, --- сказал мальчик.
\FA{
Ситу-ну-онна --- на Преисподней: женщина, владеющая оружием и боевыми искусствами.
}
--- Когда я вырасту, ты будешь моей женой.

--- Хорошо, что не сейчас.
Хоть отдохну ночью, --- усмехнулась Айну и как ни в чём не бывало спрятала пистолет.
--- Что ты здесь делаешь, ребёнок?
Шпионишь за нами?

--- Да, --- признался мальчик.
--- Вы похожи на великих воинов из сказок.
Я следил за вами целых три часа.

--- Айну, --- буркнул Гало и вытащил оружие.
Айну жестом остановила его, демон нехотя подчинился.

<<Если забрызгаешь мне комбинезон, стирать заставлю вручную>>, --- добавила демоница.

--- Как ты скрылся от нас? --- поинтересовался Грисвольд.

Мальчишка бросил перед демонами автоматический шприц с остатками препарата.
Глаза Гало расширились, он уронил рюкзаки и подхватил шприц.

<<Нейролептик, --- передал он остальным.
--- Скорее всего, ксидатин низкой степени очистки>>.

Только сейчас Айну заметила в улыбке мальчика что-то искусственное.
Глаза оставались бесстрастными и отстранёнными.
Когда мальчишка вытянул руку, чересчур короткий рукав изорванной куртки соскользнул, и демоны увидели корявые детские иероглифы, вырезанные ножом на коже запястья:

\ml{$0$}
{<<ИДИ ЗА НИМИ>>.}
{\textsc{Follow them.}}

Грейсвольд вздрогнул.
Ксидатин напрочь подавлял мотивацию.
Людям требовались годы, чтобы научиться действовать в ступоре, в условиях минимальной эмоциональной стимуляции.
Для этого нужно было чёткое понимание, что и зачем ты делаешь.
Но ребёнок?..
Грейсвольд мысленно подивился силе его воли.
Он знал, что людям труднее всего идти против собственных эмоций.
Или против их отсутствия.

--- Ты не желаешь нам зла, теперь я вижу, --- сказала Айну, заворожённо разглядывая кровавую надпись на детской руке.
--- Мы не можем долго с тобой беседовать.
Возвращайся домой.

Ножны на поясе Айну расстегнулись, клинок выдвинулся на дюйм.

--- Я пойду с вами, --- без раздумий выпалил мальчишка.

--- Какой смелый.
В таком случае --- вперёд, --- скомандовала Айну, безо всяких фокусов убрав клинок и поправив заплечный мешок.

<<Ты с ума сошла?>> --- удивился Грисвольд.

<<Да хватит уже, --- отозвалась демоница.
--- Убивать надо сразу, а мы театр устроили>>.

\ml{$0$}
{--- Зачем ты носишь с собой эту бесполезную палку? --- спросила Айну.}
{``Why do you carry this useless stick?'' Ain\"{u} asked.}
\ml{$0$}
{--- Брось.}
{``Drop it.''}

\ml{$0$}
{--- Нет, --- просто ответил мальчик.}
{``Nope,'' the boy answered simply.}
\ml{$0$}
{--- Это фамильная ценность, и он старше вас всех вместе взятых.}
{``It's a family heirloom, and in addition it's older than all of you combined.}
\ml{$0$}
{Проявите уважение.}
{Show respect.''}

\ml{$0$}
{--- Дело твоё, --- так же просто согласилась Айну, отметив своё полное нежелание ему отказывать.}
{``Suit yourself,'' Ain\"{u} agreed in the same manner, thinking of her own reluctance to deny.}
\ml{$0$}
{--- Лу, ты за него отвечаешь.}
{``Lu, you're in charge of him.''}

Люцифер хмуро посмотрел на демоницу, потом на неожиданного подопечного.
Тот, недолго думая, перебросил ремешок дайту-соро за плечи, подошёл к демону и бесцеремонно запрыгнул ему на закорки.
Хрупкий Лу охнул и едва не оступился.

--- Вези меня, --- оскалился мальчишка.

--- Когда придём, я высплюсь, а утром лично тебя прирежу, --- полузадушенным голосом посулился Люцифер.
\ml{$0$}
{--- Следи за своей уважаемой железкой, она меня по заднице бьёт.}
{``Watch your respectable piece of iron, it's been hitting my ass.''}

\ml{$0$}
{Впрочем, угроза была пустой.}
{This threat, however, was empty.}
Мальчик успел заинтересовать Люцифера так же, как и очаровать Айну.

Отряд двинулся дальше.
Айну шла молча.
Люцифер, угрюмо везущий на плечах мальчишку, тоже молчал.
Грисвольд уже не дышал, а пыхтел под тяжестью рюкзаков.

Мальчик начал выходить из ступора.
Грисвольд несколько раз бросал взгляд на его лицо --- в нём оставалось всё меньше искусственности.
Бродяжка начал проявлять неподдельный интерес к спутникам и грузу, который они несли.

--- Зачем ты носишь ножницы в ножнах? --- спросил он Лу.

--- Потому что они почти то же самое, что нож, но ими сложнее порезаться, --- буркнул Лу.

--- Это выглядит смешно!

--- Смешно выглядит твоя палка.
Ею даже людей сложно резать, хотя разрабатывалась она именно для этого!

Вскоре инициативу в разговоре взял Гало.

\ml{$0$}
{--- Где ты научился так притворяться?}
{``Where did you learn to pretend like that?''}

\ml{$0$}
{--- Это называется <<быть вежливым>>, --- просветил демона мальчик.}
{``That's called `to be polite','' the boy enlightened the daemon.}
\ml{$0$}
{--- В нашем доме бывали разные люди, но мать учила, что хозяин должен быть вежливым со всеми гостями.}
{``Our home was visited by all sorts of people, but mother taught me that a master should be polite to all guests.''}

\ml{$0$}
{--- Ты считаешь себя хозяином?}
{``You call yourself \emph{a master}?''}

\ml{$0$}
{--- Вы определённо прибыли издалека, --- улыбнулся бродяга.}
{``You've certainly come from far away,'' the tramp smiled.}
\ml{$0$}
{--- А я здесь живу с рождения.}
{``Unlike me, who was born here.''}

--- Ты знаешь правила знакомства, дзайку-мару?

Мальчишка знал, что словом <<дзайку-мару>> бродячие торговцы называли всякую мелочь для ремёсел --- заклёпки, полоски ткани, нитки, --- но не повёл бровью.

--- Знаю, \emph{хорохито}, --- сказал он.
--- При первой встрече люди, если они не враги, кланяются, называют свой род, родовые девизы и имена.

--- И почему ты не следуешь этим правилам? --- ядовито осведомился Гало.
--- Скажи своё имя.

\ml{$0$}
{--- Поклонись, --- парировал мальчишка.}
{``Take a bow,'' boy retorted.}
Его голос был непринуждённым, но чувствительный к нюансам интонации слух демонов заставил всех изумлённо посмотреть на маленького спутника.
Гало зашипел.
Он тоже услышал оскорбительный ультиматум.

--- Если бы ты не висел на шее моего брата, я бы нафаршировал тебя пулями.

Айну расхохоталась.
Грисвольд хихикнул и тут же закашлялся.

\ml{$0$}
{--- Это что-то, --- тяжело дыша, признал технолог.}
{``Amazing,'' Griswold declared gasping for breath.}
\ml{$0$}
{--- В такую глупую историю я ещё не попадал.}
{``I’ve never got into such a stupid situation.}
\ml{$0$}
{Хорошо, доиграем сценку до конца.}
{So, let’s act this play to the end.''}

Грисвольд приблизился к Люциферу и поклонился, насколько позволяли быстрый шаг и тяжёлый груз.

\ml{$0$}
{--- Я Грисвольд из рода... эээ... Грисвольда.}
{``I’m Griswold, belong to ... hmm ... Griswold house.}
\ml{$0$}
{<<Добрый Грис --- лентяя приз>>.}
{\emph{We're a prize for lazy guys.}''}

\ml{$0$}
{Мальчик церемонно кивнул толстяку.}
{The boy ceremoniously nodded his head to the fat man.}

\ml{$0$}
{--- Я из рода Ханаяма.}
{``I belong to Hanayama house,'' the boy told.}
\ml{$0$}
{<<Благородная кожа --- лучшее одеяние>>.}
{\emph{``Noble skin is the best suit.''}}

\ml{$0$}
{--- Случаем, не твоих родичей освежевали во время недавней вендетты? --- осклабился Гало.}
{``Quite by chance, aren't that your family who have been skinned at the recent vendetta?'' Halo grinned.}
\ml{$0$}
{--- Кто бы это ни сделал, они не лишены чувства юмора.}
{``Whoever have done it, they are not deprived of sense of humour.''}

--- Я отвечаю только за свою кожу, --- заметил мальчик.
--- И она пока на мне.
Грисвольд мог бы передать часть груза мальчику.
Я понесу один рюкзак, а длинноволосый понесёт меня.

\ml{$0$}
{--- Иди ты в дупло с такими идеями, --- пробормотал Люцифер.}
{``Go to hollow with such ideas,'' Lucifer muttered.}

Айну обогнала Люцифера и тоже поклонилась мальчику.
Бродяга поднял огромные живые глаза на женщину и улыбнулся белозубой улыбкой.

--- Тахиро.

Айну открыла рот, чтобы назвать своё имя, но Люцифер не выдержал:

--- Может, вы прекратите переговоры на моём горбу?

\section{[:] Сладкое безделье}

\epigraph
{Сотрудничество с сильным соперником --- хороший способ отступления.}
{Пословица Преисподней}

Люцифер сладко потянулся в своей постели.
Вчерашний план увенчался полным успехом.
<<Разбежавшиеся>> войска Ордена застали врага врасплох.
На указанный путь, по которому прошли два стратега, технолог и интерфектор, слетелись аж одиннадцать диверсионных групп.
Все они попали в засаду и были уничтожены.

<<Как же я люблю ничего не делать, --- лениво думал Люцифер, смакуя утреннюю негу.
--- А вот быть приманкой --- не очень>>.

Люцифер обладал потрясающим умением наслаждаться отдыхом, зная, что нежиться ему оставалось считанные минуты.
График у демонов по-прежнему был плотный.

Лу перевернулся на другой бок и закрыл глаза.
Вскоре его вывел из забытья лёгкий спазм в мышцах --- забавный сигнал, говорящий о готовности мозга погрузиться в сон.
Лу уже даже засунул под прохладную подушку разгорячённую руку и настроился на продолжение отдыха, как зажужжал браслет на левой руке.

<<Как обычно>>, --- философски заключил Люцифер и, сев на кровати, начал с наслаждением тереть глаза, зевать и потягиваться.
Со стороны это выглядело вопиющим бездельем --- если не знать, что стратег специально выделял для потягиваний, зевания и утренней мастурбации восемь минут пятьдесят секунд драгоценного времени.
На личную гигиену всем выделялась двадцать одна минута, но специально для Лу Арракис сделал исключение;
он мог заниматься своим телом аж тридцать пять минут и двадцать секунд --- с учётом душа, антибактериальной обработки тела, нанесения ухаживающих средств и небольшого количества макияжа.
Исключительное право Лу выбивал потом и кровью.
Ему пришлось собрать внушительную статистику, чтобы доказать --- при ухоженном теле его демон работает гораздо эффективнее.
И даже это бы не помогло, если бы на сторону парня совершенно неожиданно не встала Айну.

--- Идите вы в дупло со своей дисциплиной, --- заявила она Гало и Арракису.
--- Вам же ясно показали, что результат лучше.

--- Сегодня Лу, а завтра --- все легионеры начнут минуты требовать? --- возмутился Гало.

--- Я им всем лично надушу подмышки и накрашу глаза, если это повысит эффективность, --- рявкнула Айну, выйдя из себя.
--- В общем, или Лу получает свои минуты, или ищите другого императора\FM.
\FA{
Император --- ранг и должность в раннем Ордене Преиподней, соответствующие максиму терция и командующему планетарными вооружёнными силами соответственно.
}
Мне нет резона работать с демонами, ставящими методы превыше результата.

Ещё одна хорошая новость поступила за завтраком.
Наблюдателей, сливших информацию Союзу, вывернули наизнанку.
Тысяча Башен безоговорочно приняла условия Преисподней --- на вулканическую планету прибыли две трети их высшего командования и почти все учёные, аккурат в приготовленные для интерфекторов тела.
Фактически это означало объединение двух организаций в одну.
Наступило кратковременное локальное равенство сил.
Следовало заняться восставшими дзайку-мару.

Люцифер уже начал просматривать полученные данные и прикидывать, как ставить новые эксперименты, когда его остановил Гало:

--- Ты чего, брат?

--- Нужно провести ещё серию экспериментов, чтобы...

--- Зачем? --- Гало ухмыльнулся и показал брату автоматический шприц Тахиро.
--- Спускайся в подвалы, Айну сейчас приведёт твоего сопляка.
Он знает, где бункер.

\section{[:] Провели}

--- Я вам ничего не скажу, --- сразу заявил Тахиро.
--- Ничего.

Мальчик выглядел не лучше, чем в день, когда отряд Айну нашёл его.
На руках и ногах пестрели ссадины, в разбитом носу засохла кровь.
Руки и ноги были наскоро обездвижены кабельной стяжкой, на шее красовался ортопедический воротник, из углов рта торчал кляп-гантель, не позволявший крепко сжать зубы.
Левый рукав рубахи был мокрым от слюны;
видимо, мальчик лежал связанным несколько часов.

--- Зачем ты его связала? --- спросил Лу и попытался снять гантель.
Айну остановила его руку на полпути.

--- Это не я, --- ответила Айну.
--- По словам стражи, он попытался причинить себе вред.

--- Некоторые ломают себе кости и осколками режут крупные сосуды, --- сообщил дежурный легионер и, бесцеремонно засунув пальцы Тахиро в рот, обнажил зубы.
--- Те немногие, кому повезло остаться с функциональным набором зубов, используют их --- например, чтобы откусить язык.
Некоторые агенты специально затачивают один резец перед делом --- именно на этот случай, вполне хватает, чтобы вскрыть вены.
У этого пацанёнка, как видите, зубы в порядке --- видать, из хорошей семьи, неплохо питается и...

--- И ты свободен, легионер, --- чуть резче, чем обычно, сказал Грисвольд.

--- Как скажете, сама, --- дежурный коротко поклонился и вышел.

--- Я ничего не скажу, --- повторил Тахиро.

--- Если я буду тебя пытать, ты расскажешь всё, --- без обиняков сказала Айну.
--- И совершить самоубийство у тебя не получится, даже не надейся.
Думаю, после твоей жалкой попытки втереться в доверие ты уже догадался, что мы умнее твоих соплеменников.

Мальчик понурился.
Гало грубо вздёрнул его в сидячее положение и тут же, посмотрев на мокрые от слюны руки, начал вытирать их об одежду.
На его лице застыло отвращение.

--- Видишь те кандалы? --- Айну показала на стену.

Тахиро кивнул.

--- Знаешь, что там недавно произошло?

Тахиро помотал головой.
Вернее, пошевелил, насколько позволял воротник.

--- Я повесила туда девушку чуть постарше тебя, --- тихо сказала Айну.
--- Ты хотел взять меня в жёны, верно?
Значит, ты должен знать, что происходит между мужчиной и женщиной, когда они остаются одни.

Тахиро исподлобья смотрел на демоницу.

--- Прекрасный акт творения, --- нараспев начала Айну.
--- Экстаз и начало новой жизни в одной чашке.
Если, конечно, по взаимной любви и согласию.
Однако представь, что мужское начало дарует не жизнь, но боль и смерть.

Айну показала Тахиро ацетиленовую горелку с длинным цилиндрическим жалом.
Тахиро вздрогнул.

--- После первого щелчка, --- Айну щёлкнула горелкой, --- она начала издавать звуки.
После второго (щелчок) начала рассказывать то, что знала и не знала.
И когда она закончила... --- Айну направила горелку на кандалы, --- ...я включила струю на полную мощность и ушла на обед.

Тахиро смотрел на Айну.
Грисвольд вдруг поймал себя на мысли, что выражение лица мальчика совершенно не подходило к ситуации.
Так палачи смотрят на тех, чья участь уже предрешена.

--- Был ли у неё выбор? --- ласково закончила Айну.
--- Был.
До первого щелчка.
Но девочки всегда плохо соображают перед этим.
Особенно если... --- Айну щёлкнула горелкой, --- ...это происходит в первый раз.

<<Айну, хватит>>, --- взмолился Грисвольд.
Тахиро продолжал смотреть на Айну тем же бесстрастным немигающим взглядом.

--- Айну, --- начал Люцифер.
--- Пытки --- это грубый способ.
Всегда можно найти...

Айну взмахом руки остановила юношу.

<<Мы не договаривались, что ты будешь играть хорошего палача>>.

--- Да при чём здесь это! --- воскликнул Лу вслух.
--- Пытки --- это непрофессионально и расточительно!
Если ты будешь ломать людей направо и налево, нам будет некем править!
Механизм нужно изучать в действии аккуратно, а не...

--- Заткнись, --- отрезала Айну и обратилась к мальчику:
--- Если ты всё понял, то я тебя слушаю.

\ml{$0$}
{Тахиро ответил не сразу;}
{Tahiro didn't speak at once;}
\ml{$0$}
{он смотрел на Лу расширенными в изумлении глазами.}
{he stared at Lu in wonder.}
\ml{$0$}
{Этот взгляд ещё долго являлся потом стратегу во сне.}
{The strategist has dreamt of that look for a long time since. }
\ml{$0$}
{Раскосые чёрные глаза не просили поддержки, не жаловались на судьбу, в них читался растерянный вопрос: <<Что происходит?>>.}
{Upturned black eyes did not look for support, nor complain, they bewilderedly asked: ``What's happening here?''}
\ml{$0$}
{Стратег, поморщившись, отвернулся --- словно боялся, что вопрос заразит и его разум.}
{The strategist made a wry face and turned his head away---as if he's afraid the question is contagious to his mind.}

--- Я тебя слушаю, Тахиро, --- повторила Айну.

\ml{$0$}
{--- Я требую обмен, --- мальчик продолжал смотреть на Лу.}
{``I demand exchange.'' The boy continued looking at Lu.}

\ml{$0$}
{--- Ты торгуешься с нами, дзайку-мару? --- ухмыльнулся Гало.}
{``Want to establish some trade relations, dzaiku-maru?'' Halo smirked.}

\ml{$0$}
{--- Да, --- просто ответил мальчик.}
{``Yes,'' the boy answered simply.}
\ml{$0$}
{--- Вы в состоянии войны с другими хорохито, это я знаю.}
{``The state of war exists between you and other horohitos, I know that.}
\ml{$0$}
{Каждый час пытки --- потеря ресурсов для вас.}
{Every hour of torture is waste of resourses for you.}
\ml{$0$}
{Я продаю вам несколько часов драгоценного времени.}
{I offer you several precious hours.''}

Айну подняла бровь.
Она явно не ожидала от Тахиро такого взрослого подхода к делу.

--- Вы дадите мне послать в бункер сообщение за пять минут до того, как начнётся штурм.

\ml{$0$}
{--- За минуту.}
{``One minute.''}

\ml{$0$}
{--- За четыре.}
{``Four.''}

\ml{$0$}
{--- За минуту, --- повторила Айну.}
{``One single minute,'' Ain\"{u} repeated.}
\ml{$0$}
{--- Или мы расторгаем сделку.}
{``Or we cancel the contract.''}

\ml{$0$}
{--- Значит, расторгаем.}
{``It's cancelled, then.''}

У Тахиро дрожали от страха губы, пока он произносил эту фразу.
Грисвольд ещё раз подивился его мужеству.

\ml{$0$}
{--- Тахиро, --- ласковый голос Айну был напитан смертельным ядом, --- я хочу прояснить один момент.}
{``Tahiro,'' gentle voice of Ain\"{u} was soaked with deadly poison, ``I'd like to clear one thing up.}
\ml{$0$}
{Речь не идёт о \emph{нескольких} часах драгоценного времени.}
{This is not a question of \emph{several} precious hours.}
Ты, конечно, не женщина, но я найду подходящее отверстие для горелки.
\ml{$0$}
{И когда я найду, мне не понадобятся часы, чтобы вытащить орешек из скорлупы.}
{And when I find it, I'll need less than an hour to shell a nut.''}

\ml{$0$}
{В раскосых глазах Тахиро блеснул азарт:}
{Gambler's excitement twinkled in Tahiro's eyes:}

\ml{$0$}
{--- Спорим?}
{``You bet?''}

Айну задумалась.
Несколько долгих мгновений она, прищурившись, разглядывала маленького пленника.

\ml{$0$}
{--- Три минуты, --- буркнула она наконец.}
{``Three minutes,'' she finally grunted.}

\ml{$0$}
{--- По рукам, --- без раздумий бросил Тахиро.}
{``Deal,'' Tahiro agreed without thinking.}

\ml{$0$}
{--- Что за сообщение ты хочешь послать?}
{``What message do you want to send?''}

\ml{$0$}
{--- <<Всё пропало, спасайся кто может>>.}
{\emph{``It's over, run for your lives.''}}

\ml{$0$}
{--- Как оригинально, --- скривилась Айну.}
{``How original,'' Ain\"{u} grimaced.}
\ml{$0$}
{--- Тебе нужно что-то ещё?}
{``Do you need something else?''}

\ml{$0$}
{--- Я знаю, что вы добыли дымный веер.}
{``I know you've got a fume fan.}
\ml{$0$}
{Дайте его мне.}
{Give it to me.}

--- Ты знаешь сигналы дымного веера? --- поднял брови Гало.

--- Я знаю всё, в том числе и код.
На ваши сигналы никто не отвечал, потому что они изменяются в зависимости от направления ветра и фазы Второй Луны.

\ml{$0$}
{--- Какое кострище будешь использовать?}
{``Which fire pit will be used?''}

\ml{$0$}
{--- Никакое.}
{``None.}
\ml{$0$}
{Найдите фумаролу в виде жабы на южном склоне Арашияма.}
{Find a toad-shaped fumarole on southern side of Arashiyama.''}

\ml{$0$}
{--- Фумаролу? --- выпучил глаза Гало.}
{``A fumarole?'' Halo goggled.}
\ml{$0$}
{--- Так вы использовали дым фумарол, а не...}
{``So you used to use fumaroles, not---''}

\ml{$0$}
{--- Да, конечно, дубина, --- выпалил Тахиро, --- люди бункера носили кремень просто для отвода глаз!}
{``Sure, you dumb piece of wood,'' Tahiro exploded, ``bunker people used to carry a flint as a distraction!''}
\ml{$0$}
{И система кострищ была лишь отвлекающим манёвром.}
{And fire pit system was a decoy as well.}
Её использовали только в случае, если посланец чувствовал слежку.
\ml{$0$}
{Только хорохито будет разводить костры, если есть фумарола.}
{Only a horohito would make a fire if there is a fumarole.''}

Гало и Люцифер переглянулись... и разразились диким хохотом.

В этот раз люди их провели.

\asterism

--- Я посылаю сигнал, мы ждём три минуты и я сообщаю вам координаты входа в бункер.

--- Могу ли я ручаться, что твои данные окажутся истинными? --- осведомилась Айну.

\ml{$0$}
{--- Я клянусь тебе честью моего дома.}
{``I swear with honor of my house.''}

\ml{$0$}
{--- Есть ли честь в том, что ты предаёшь соплеменников? --- ухмыльнулся Гало.}
{``Treachery of your own kind---what kind of honor is it?'' Halo smirked.}

\ml{$0$}
{--- В том, чтобы умирать под пытками, точно нет чести, --- заявил мальчик.}
{``Death under torture is definitely none,'' the boy contended.}
\ml{$0$}
{--- Я признаю лишь смерть в бою.}
{``Death in battle is the only death with honor.}
\ml{$0$}
{Считай это способом выжить и дождаться боя.}
{You may think I try to survive and wait for the battle.''}

\ml{$0$}
{<<Он опасен, Айну, --- сообщил Гало.}
{\textit{``It's dangerous, Ain\"{u},''} Halo said.}
\ml{$0$}
{--- Этого фенрира\FM{} следует уничтожить>>.}
{\textit{``This fenrir\FM{} should be destroyed.''}}
\FA{
Фенрир --- шпион или диверсант, пользующийся внешней привлекательностью и не вызывающий подозрений, обычно ребёнок или человек, внешне похожий на ребёнка (терминология Ордена Тысяча Башен).
}

<<Я против, --- заявил Люцифер.
--- Мальчик крайне интересен как объект исследования.
Я бы попробовал интегрировать его в наше общество>>.

Айну почесала подбородок и оглядела связанного Тахиро.

\ml{$0$}
{<<Хорошо, Лу, я дарю этого поросёнка тебе.}
{\textit{``Well, Lu, I give this piglet to you.}}
\ml{$0$}
{Можешь убить его, когда сочтёшь нужным>>.}
{\textit{Kill it whenever you want.''}}

Айну махнула рукой и вышла из подвала, Грисвольд и Гало последовали за ней.
Лу остался наедине с Тахиро.

Мальчик молчал.
Лу следовал его примеру.
Разумеется, робкие ростки доверия между ними сметены безжалостным вихрем реальности.
Эти двое не могут быть товарищами --- они навсегда останутся хищником и жертвой.
Мальчик понимал это не хуже демона.
Он видел перед собой лишь коварного врага, готового на всё ради победы.

Стратег до смерти не любил этот этап в любом деле --- когда в закрытой системе всё понятно, всё доступно взгляду, но процесс уже запущен в нежелательном направлении.
И будь ты хоть семи пядей во лбу, тебе придётся просто стоять и ждать подходящего момента, чтобы исправить ситуацию.

Последние шаги давно затихли.
Наконец стратег нарушил молчание:

--- Слушай, ребёнок...

--- Благодарю тебя, что вступился.

--- Как ты узнал? --- удивился Люцифер.

--- У нас есть легенда, что некоторым людям однажды преграждают дорогу два брата --- Футаго Рэй и Футаго Ицу.
В глазах одного брата --- жизнь, в глазах другого --- смерть.
И человек пытается задобрить, подкупить или побороть братьев, не зная, что шанс что-то изменить уже упущен и он может лишь наблюдать, как братья спорят.

--- Занятная легенда.
Мне нужно твоё мнение как... представителя другой формы жизни.

\ml{$0$}
{--- Я слушаю.}
{``I'm listening.''}

\ml{$0$}
{--- Как думаешь, благодаря чему существа приходят к сотрудничеству?}
{``Why do you think beings start to collaborate with each other?''}

\ml{$0$}
{--- Благодаря пониманию, --- пожал плечами мальчик;}
{``They understand each other,'' the boy shrugged;}
его отвратительная сбруя не позволяла большего.

\ml{$0$}
{--- А чтобы кто-то понял, достаточно ли просто объяснить?}
{``Then, if you want someone to understand, is it enough to explain?''}

\ml{$0$}
{--- Я думаю, вполне.}
{``I think, yes, it is.}
\ml{$0$}
{Доходчиво и искренне.}
{Clearly and sincerely.}
\ml{$0$}
{Истина всегда недалеко от искренности.}
{Verity is always close to sincerity.}
\ml{$0$}
{Очень важно показать ход мысли.}
{The way you think is important to be shown.}
\ml{$0$}
{Желательно говорить на языке собеседника --- так учила меня мать.}
{It's advisable to speak in language of interlocutor, as my mother taught me.''}

\ml{$0$}
{--- Но что, если собеседник не поверит?}
{``But what if interlocutor will not believe?''}

\ml{$0$}
{--- Это его право.}
{``It's his right.}
\ml{$0$}
{Считаешь собеседника за равного --- уважай его права.}
{To treat someone as equal means to respect his rights.''}

Лу помолчал.

--- На самом деле мы уже были близки к обнаружению бункера, --- наконец сказал он.
--- Это я послал твоих родичей на дело, из-за которого их линчевали.
Вся информация о бункере, которую они, сами того не зная, собрали в пустошах, уже у меня.

\ml{$0$}
{--- Ты просишь прощения?}
{``Do you need my forgiveness?''}

\ml{$0$}
{--- Я не ставил целью их уничтожить.}
{``Destroying them was not on purpose.}
\ml{$0$}
{Расценивай это как хочешь.}
{It's yours to interpret.''}

\ml{$0$}
{Тахиро кивнул:}
{Tahiro nodded:}

\ml{$0$}
{--- Это было доходчиво.}
{``You've made it clear.''}

\ml{$0$}
{--- Я могу снять путы, если ты пообещаешь, что не попытаешься сбежать или убить себя.}
{``I can untie you if you promise not to flee and not to kill yourself.''}

\ml{$0$}
{--- Я уже поклялся всем, что у меня есть.}
{``I have nothing left to swear with.''}

Лу молча вытащил ножницы, разрезал стяжки и вышел, оставив дверь приоткрытой.

\section{[:] Тушение}

\textspace

--- Только не говорите, что этот сброд будет драться, --- поморщился Арракис.

--- Приготовиться к залпу, --- бросил Гало.

--- Легион, отбой, --- тут же подняла руку Айну.
--- Ждём моего приказа.

Один из восставших вышел перед собратьями, собираясь обратиться к ним с пламенной речью.

--- Потуши его, --- лениво бросила Айну стоящему рядом легионеру.

Легионер вскинул винтовку --- и оратор упал с простреленной головой, едва успев выкрикнуть <<Люди!>>.
Толпа дрогнула.

--- На этой планете речи произносятся или мной, или с моего разрешения, --- резюмировал Арракис.
--- Самаэл, я надеюсь, ты сумеешь объяснить это дзайку-мару.

Самаэл кивнул и направился к восставшим, замершим в растерянности.

--- Учись, сайгон, --- ухмыльнулась Айну, бросив лукавый взгляд на Гало.
\ml{$0$}
{--- ,,Один патрон решает любую проблему``\footnotemark.}
{``\textit{One bullet solves any problem}\footnotemark.''}
\FA{Девиз Союза Воронёной Стали, употреблённый в ироничном ключе.}

--- Или можно было просто узнать, чего они хотят, --- вставил Люцифер.
--- А потом насыпать рыбкам мотыля или сменить наконец воду, чтобы они не плавали в собственном дерьме.

--- А твой брат --- теоретик-идеалист, --- закончила Айну со вздохом.

--- Из вас всех я единственный, кто ухаживает за аквариумом и террариумом в штабе, --- парировал Лу.
--- Не будь меня, вся животина давно бы уже сдохла.

--- Ты животину завёл, тебе и ухаживать, --- флегматично ответила Айну.

--- А я чем занимаюсь сейчас, по-твоему? --- раздражённо бросил Лу.
--- Ладно.
Время покажет, кто из нас теоретик.

--- Безусловно, --- кивнул Арракис.
--- И время покажет это в самое ближайшее время.
Штурмовой отряд, вы переходите под командование Люцифера.
Остальной легион --- на указанные точки.
\ml{$0$}
{У нас есть дела.}
{We've got much work to do.''}

Гало удивлённо посмотрел на отца:

\ml{$0$}
{--- Отец, может быть, мне?}
{``Father, maybe I?''}

\ml{$0$}
{--- Нет, Гало, ты нужен в штабе.}
{``No, Halo, I need you in the HQ.}
\ml{$0$}
{Самаэл уже разогнал толпу.}
{Samael's just dispersed the crowd.}
\ml{$0$}
{С моими легионерами бункер сможет взять даже бродячий торговец.}
{With my legionnaires, even a peddler can storm that bunker.''}

Люцифер кивнул и скомандовал построение под пристальным взглядом Айну.

<<Она за меня беспокоится, как за члена семьи, --- подумал стратег.
--- Война ещё не совсем отравила её сущность>>.

\section{[:] Склад}

Люцифер шёл по коридору и в который раз удивлялся, сколько мертвецов осталось в бункере.
Его легионеры убили лишь треть;
прочие умерли ещё до прихода демонов.
Это могла быть людоедская тактика, заставляющая врага поверить, что никто и не думал бежать.
С той же вероятностью это могло быть совершенно бессмысленное самоубийство --- люди Преисподней не особо раздумывали перед тем, как убить или умереть.
Подсчёт пищи и инструментов не смог дать чёткого ответа на то, сколько людей было в бункере час назад --- слишком широкие доверительные интервалы, слишком много помех.
Даже состав и объём воздуха подкачали во всех смыслах --- комнаты были маленькие, но вентиляция работала так, что набитая трубка раскуривалась в руках сама.
Облава по селениям могла бы прояснить этот вопрос, но ресурсов для облавы было недостаточно.

\ml{$0$}
{<<Мой интеллект пересиливает, --- думал Лу.}
{\textit{My intelligence is overpowering the situation,} Lu thought.}
\ml{$0$}
{--- Но недостаточно быстро.}
{\textit{But not fast enough.}}
\ml{$0$}
{Чересчур мало данных.}
{\textit{Too few data.}}
\ml{$0$}
{Чересчур много тех, кто мне противостоит.}
{\textit{Too many people who resist me.}}
Отец вставляет палки в колёса, Гало помешался на своих амбициях, Айну чересчур ограничена, Самаэл развёл бюрократию, Мистраль --- трусиха, а Грисвольд ведёт себя так, словно происходящее его не касается.
\ml{$0$}
{Чтоб вы все сгорели дотла!}
{\textit{May all of you burn to ashes!}}
\ml{$0$}
{Не могу же я биться на три фронта в одиночку!>>}
{\textit{I can't fight on three fronts at once alone!}}

Лу мысленно поставил заметку --- следует поговорить с Грисвольдом о Гало.
Дальше этот спектакль продолжаться не может.

Разумеется, за взятием бункера последует волна восстаний.
Люцифер достаточно хорошо изучил психологию людей, чтобы быть к этому готовым.
Люди Преисподней не любили чувствовать себя проигравшими;
после каждой большой победы Ордена они бросались в бессмысленную, неорганизованную контратаку.

<<И в этот раз, чёрт возьми, мы будем действовать уговорами.
Я лично на совещании заткну всех вояк, включая Гало.
\ml{$0$}
{Бункер мы вынуждены были зачистить, это была неоспоримая военная необходимость.}
{We had to clear the bunker, that was an undeniable military necessity.}
\ml{$0$}
{Но после каждой нашей атаки остаются целые толпы сирот и вдов, жаждущих крови.}
{But each our attack leaves behind masses of orphans and widows, clamoring for blood.}
\ml{$0$}
{Нет уж.}
{Fuck it.}
\ml{$0$}
{Пусть оставшиеся в живых будут несчастны как-нибудь по-другому --- не за счёт Ордена и меня лично>>.}
{Let survivors be unhappy some other way, not through the Order or my bloody self.''}

Носильщики тащили добычу к выходу.
Лу с некоторым удовольствием наблюдал за ними.

<<Как же хорошо, когда кто-то таскает тяжести за тебя>>, --- думал он.

Вот командный пункт.
\ml{$0$}
{Три тела со следами сэппуку.}
{Three bodies with signs of seppuku.}

Вот шифровальная.
Бежавшие не удовлетворились простым убийством шифровальщика;
ему раздробили голову до кашеобразного состояния.
Излишняя, но далеко не глупая предосторожность.

На столе ещё дымилась разворошённая стопка листового пластика.
Люцифер поднял один лист.
Перфокарта.
Фрагмент публичного ключа шифра Курибутону --- одного из первых квантово-устойчивых асимметричных шифров, использовавшихся переселенцами с Древней Земли. % Kuributonu
Взломан чуть ли не пять тысяч лет назад.

<<Мда.
Ну и музей.
Разумеется, они полагались не только на шифр, иначе мы бы давно раскрыли канал связи.
Кстати, а вот и стеганограф...
О небо, сборник стихов о море, эпоха как его там...
Это же моя настольная книга в детстве была, год три тысячи восемьдесят пятый, она тогда стотысячным тиражом вышла.
,,Лилия о пяти лепестках``, засечки на чертах иероглифов, количество чешуек на играющих рыбках...
Уж кто-кто, а я мог бы и догадаться>>.

--- Киноу, --- подозвал Лу посыльную.

--- Сайгон, --- поклонилась женщина.

--- Знакомая книжка?

--- <<Волны Ниигаты>>, юбилейный выпуск на столетие Накагавы Юкико, --- отчеканила легат.

--- Я тебя обожаю, ты в курсе?
Аккуратно, без давления, конфискуем весь юбилейный тираж и меняем его на тот же сборник с иллюстрациями от Окинотори.
Загрузи типографию и выдели для конфиската хранилище с надлежащими условиями.

--- Сайгон, --- Киноу козырнула и ушла.

<<Отлично, --- удовлетворённо подумал Лу.
--- Народ сильно возмущаться не будет, сборник с Окинотори красивый, а нам этого хватит, чтобы разобщить шпионские каналы связи на полгодика.
Пусть лучше наша типография печатает Накагаву, чем эту пропагандистскую чушь>>.

Вот склад с пищевым порошком.
Его количество уже подсчитали, пробы для выявления места производства взяли.
Возле апробированной коробки лежали два трупа --- самоубийца и убитый защитник.

Лу осмотрел склад, просмотрел коробки с помощью тепловизора.
Да, это пищевой порошок.
Мужской у входа, женский --- чуть подальше.
\ml{$0$}
{Больше ничего.}
{Nothing more.}

Но что-то не давало покоя стратегу.
\ml{$0$}
{Коробки.}
{Boxes.}
\ml{$0$}
{Они все были совершенно одинаковые.}
{They were identical.}
\ml{$0$}
{При этом каждый последующий ряд отстоял от предыдущего на какое-то расстояние --- совершенно произвольное.}
{But every row was shifted slightly relative to the previous, and a distance was entirely random.}
\ml{$0$}
{Впрочем, не совсем произвольное...}
{However, not entirely ...}
\ml{$0$}
{Совсем не произвольное!}
{Entirely not!}

--- Разобрать коробки, --- распорядился Лу уже в спины уходящим легионерам.

Вскоре под коробками обнаружилась пустота.
Там сидели двое мужчин, пять женщин и тринадцать детей.
У всех на лицах замерло туповатое выражение;
моргали они редко и медленно, словно под водой.
Люцифер разглядывал их так, словно сам был не рад своей находке.

--- Нейролептический сопор, --- хмыкнул один из легионеров.
--- Нам доставить их в допросную?

\ml{$0$}
{--- Убить всех.}
{``Kill 'em all.''}

--- Сайгон\FM, не лучше ли взять некоторых женщин с выводком?
\FA{
Сайгон --- ранг в раннем Ордене Преисподней, соответствующий легату прима.
}
Если к ним правильно подступиться, они живо заговорят.

\ml{$0$}
{--- Возможно, нам стоит их отпустить.}
{``Maybe we should release them.}
\ml{$0$}
{Что скажешь, Фумиэ из клана Дорге?}
{What would you say, Fumie of the Clan Dourgue?''}

Глаза солдата забегали, словно у провинившегося.
Прочие замерли и вытянулись в струнку, прислушиваясь к разговору.

--- Нам нужна информация, сайгон, --- осторожно начал Фумиэ, --- но неповиновение дзайку-мару ни в коем случае не должно оставаться безнаказанным.

\ml{$0$}
{--- То есть после допроса их следует уничтожить?}
{``It means they must be destroyed after examination, mustn't they?''}

\ml{$0$}
{--- Безусловно.}
{``Unquestionably.''}

\ml{$0$}
{--- Ты не считаешь это расточительностью, напрасной тратой ресурсов?}
{``Don't you consider it waste of resourses?''}

\ml{$0$}
{--- Мы используем штыки вместо патронов, когда это возможно.}
{``We use bayonets instead of bullets if possible.''}

--- Я про дзайку-мару.
Они тоже являются ресурсом.

--- Прошу прощения, сайгон, --- стушевался легионер.
--- Дзайку-мару --- опасный ресурс, который может легко выйти из-под контроля.

\ml{$0$}
{--- И твоё мнение разделяет весь легион?}
{``And does the legion share your view?''}

Фумиэ бросил взгляд на стоящих рядом товарищей, но они старательно делали вид, что разговор их не касается.

\ml{$0$}
{--- Я не могу говорить за легион, но я разделяю мнение командования.}
{``I mustn't speak for the whole legion, but I share the view of the command.''}

\ml{$0$}
{--- В таком случае, легионер, оставь привычку обсуждать мои приказы, --- поморщился Лу.}
{``In that case, legionnaire, break the habit of questioning my orders,'' Lu made a wry face.}
Лицо маленького Тахиро промелькнуло в его памяти чересчур ясно.
--- Бункер выжечь до каменных стен, убить всех обнаруженных дзайку-мару, стены проверить на пустоты и трещины.

Фумиэ с явным облегчением козырнул и резво отправился выполнять приказ.

Люцифер сел на разбитую коробку и, набив трубку адским вереском, раскурил её от щепки.
Он вдыхал и выдыхал терпкий дым, лишь иногда морщась, когда раздавался приглушённый звук втыкающегося в тёплое тело штыка.

\ml{$0$}
{<<Отец меня убьёт, --- с грустью думал стратег.}
{\textit{Father will kill me,} the strategist sadly thought.}
\ml{$0$}
{--- Но пока я за главного, пыток не будет.}
{\textit{But no tortures while I'm in charge.}}
\ml{$0$}
{Даже если это была последняя возможность проявить себя>>.}
{\textit{If even it was my last opportunity to lead.}}

\section{Смех Гало}

--- Люцифер, ещё раз.
Ты приказал обменять у дзайку-мару одну книгу с поэзией на другую книгу с той же самой поэзией.
И отправил заказ на книги с поэзией в наш отдел пропаганды, в типографию.

--- Именно так.

--- А что ты намерен сделать с ними?

--- Ничего.
Аккуратно собрать и положить в хранилище с соблюдением условий хранения.
Через год владельцы их заберут.

--- Я имел в виду владельцев книг, Люцифер.

--- Уговорить обменять книжки и заверить, что сборник им вернут.

--- И всё?

--- И всё.

Арракис повернулся ко второму сыну:

--- Гало, ты можешь как-то объяснить этот приказ?

--- Могу, но не вижу смысла, --- весело сказал Гало.
Его ситуация, похоже, забавляла гораздо больше, чем остальных.
--- Лу --- сайгон, и его приказ должен исполняться, даже если он кажется бессмысленным.

--- Нет-нет, я не против обратной связи, --- поднял ладони Лу.
--- Младшие должны контролировать старших.
Но вот тут налицо какая-то чушь.
Если бы я ещё приказал кланяться мне и подносить дары, я бы понял.
Но чем им книжки не угодили?
Я их даже не для себя печатаю!

--- Если бы ты велел себе кланяться, вопросов бы как раз не возникло, --- хихикнул Гало.

--- Сыновья, вы меня в гроб загоните, --- Арракис устало прикрыл лицо ладонью.
--- Гало, можно тебя попросить заверить его приказ своей подписью?
Ко мне уже в третий раз приходят центурионы с вопросом, не превышает ли юный Люцифер свои полномочия.

--- Почему я юный, а Гало нет? --- горестно простонал Лу.
--- Я на пятнадцать минут его старше!

Гало ухмылялся во весь рот.
Лу вдруг посетило желание вмазать брату по лицу.

\asterism

--- И что это было? --- буркнул Люцифер.
--- Тебя рассмешил мой приказ?

--- С приказом всё отлично.
Очень изящный, точный и экономически выгодный ход.
Я бы так не смог.

--- Тогда в чём дело?
Тебя рассмешил этот спектакль со мной в главной роли?

--- Прости, но это правда смешно.

--- А мне нет.
Ты же прекрасно понимаешь, что ни один центурион не побежал бы к отцу, не будь у него прямого приказа?
Ты вообще видел, чтобы кто-то в легионе капал на своего начальника через его голову, тем более максиму прима?

--- Подожди, Лу, --- Гало впервые за несколько минут перестал ухмыляться.
--- Ты серьёзно думаешь, что отец приказал за тобой следить?
Брось.

--- Это уже не первый инцидент, и я тебе об этом...

--- Лу, давай начистоту, --- Гало остановился и развернулся к брату.
--- Есть некие правила приличия, и ты им не следуешь.
Из-за этого, и только из-за этого, у тебя с легионом регулярное недопонимание.

--- Что ты имеешь в виду?

--- Сам знаешь, --- Гало демонстративно вытащил из уха Лу серёжку и сунул ему в карман куртки.
--- Это война, и детским играм здесь не место.
Начни взрослеть, и твои приказы перестанут подвергать сомнению.

--- Спасибо за поддержку, --- огрызнулся Лу.
--- Хрена с два я тебе теперь табак одолжу.

--- При чём здесь табак?

--- Детские игры.

\asterism

--- Может, погуляем?

--- Я не хочу с тобой гулять, --- Тахиро подбросил в руке плоский камень и отправил его прыгать по волнам.
--- От тебя смердит телами моих родичей.

--- Мне одиноко, --- вдруг признался Лу.

--- Мне тоже, --- кивнул мальчик и, вытерев руки о юкату, встал.
--- Ну тогда пошли гулять.

\asterism

Смех утих.
Лу вдруг стал серьёзен.

--- Слушай, это звучит глупо, но... оставайся у меня.
Я тебе не заменю родителей, но старшим братом побыть могу.

--- Да какой из тебя старший брат, --- поморщился Тахиро.
--- За тобой самим глаз да глаз нужен.

--- Значит, договорились.
Ну и да, для всех остальных ты моя игрушка.

--- Я не хочу быть игрушкой.
Даже невзаправду.

--- Ты будешь моей игрушкой для всех, кому я не доверяю.

--- И для твоего брата тоже.

--- Почему ты так думаешь?

--- Ты пришел ко мне и ищешь во мне брата, --- просто ответил Тахиро.
--- Очевидно, что ты потерял брата в нём, Русахиро.

--- Люцифер.

--- Лусафейро.

--- Лу.

--- Лу.

--- Сошлись.
Пошли домой.

<<Конечно, он согласился, чтобы продолжать шпионить.
Он всё ещё меня ненавидит.
Я чувствую волны ненависти из-под его вежливой улыбки.
Но это неважно.
Он остался.
Я хотел, чтобы он остался по своей воле, и для меня не так важны его мотивы>>.

\chapter{[U] Начало}

\section{[U] Встреча}

\textspace

Несмотря на то, что снаружи царила негостеприимная обстановка заброшенной орбитальной станции, в виртуальной <<комнате>> было довольно уютно.
Собравшиеся сидели в маленькой ажурной беседке из жемчужного дерева, освещённой нежным розовым светом бумажного фонарика.
Из окружающей беседку полутьмы то и дело вырывались огромные, размером с орех, индиго-светлячки и обиженно жужжали, ударяясь о фонарь.

Обстановку придумала Митхэ, женская сущность Митриса Безымянного.
По её словам, наводить домашний уют для неё всегда было истинным удовольствием.

Разумеется, никто из присутствующих не высказал удивления, увидев Лусафейру Лёгкая Рука, а вернее, так называемую базовую копию его личности.
Но вопрос повис в виртуальном воздухе, и первым его озвучил хмурый чернявый мужчина в углу --- Аркадиу Люпино.

--- Грейс, не мог бы ты нам объяснить, как главный оборонительный стратег и максим секунда Ордена Преисподней оказался на собрании адской <<пятой колонны>>?

--- Я подумал, что ему будет интересно, --- осклабился Грейсвольд.

Аркадиу и Анкарьяль кивнули --- им этого объяснения было вполне достаточно.
Но остальных ответ технолога не удовлетворил.
Атрис --- мужская сущность Безымянного --- улыбнулся с едва заметным скепсисом.
Таниа Янтарь громко, выразительно хмыкнула.

--- Итак, --- Аркадиу попытался сгладить напряжение собравшихся.
--- Сегодня мы должны обсудить вопрос эксплуатации сапиентов хоргетами.

--- В чём смысл этого вопроса? --- подал голос Лусафейру.
Говорил он очень мягко, в его устах слова словно теряли острые углы.
--- Эволюция сделала своё дело.
На всякого хищника найдётся хищник сильнее.
Я думал, что вы хотите обсудить нечто иное.

Чханэ гневно уставилась на стратега.
Тот встретил её взгляд холодно и спокойно.

--- Это Разрушение и Насилие в одном горшке, --- резко сказала воительница.

--- Вы имеете в виду законы примитивного человеческого племени планеты Тра-Ренкхаль? --- осведомился Лусафейру.

--- Эти законы были придуманы культурологами цивилизации Тхидэ, --- бросила Чханэ.
--- Они гуманны...

--- ... и совершенно бесполезны для выживания, --- добавил Лусафейру.
--- Что и показало сражение на берегу Могильного пролива, которое едва не унесло остатки твоего народа в \emph{пристанище духов}, --- последние слова он произнёс на идеально чистом сели с характерной интонацией.

Анкарьяль и Аркадиу переглянулись.
В силу большого совместного опыта они начали понимать, с какой целью Грейсвольд привёл Лусафейру.
Увы, опасности, повисшей над заговорщиками, это не убавляло.

Чханэ, похоже, тоже поняла, чего добивается стратег.
Она расслабилась и откинулась на кресле.

--- Почему вам, демонам, так сложно объяснить простую вещь вроде гуманности? --- мягко спросила она.

--- Гуманность --- это то, что связывает руки одним людям, пока другие строят против них козни, --- объяснил Лусафейру.
--- Ваше понимание --- понимание тси --- исключение из правил, результат культурного дрейфа.
Вы выросли в тепличных условиях, вне конкуренции.
Кажется, Аркадиу писал в отчёте, что по прибытии на Тра-Ренкхаль тси очень удивились даже оружию для убийства сапиентов...

--- А я всегда думала, что Ад гуманен по отношению к сапиентам, --- подала голос Митхэ.

--- О нет, флейта моей жизни, --- неожиданно рассмеялся Атрис, --- Ад блюдёт собственную выгоду и ничего сверх этого.

Анкарьяль удивлённо посмотрела на менестреля --- он никогда не казался ей особо сообразительным.

--- Верно, --- подтвердил Лусафейру.
--- На планетах, население которых составляют Ветви Звезды, мы вынуждены практиковать деспотизм, так как сапиенты Ветвей Звезды испускают плюс-эманации при угнетении.

Чханэ издала странный звук --- не то плюнула, не то выругалась.

--- Что такое деспотизм? --- вмешалась Анкарьяль.
--- Сапиентов угнетают под угрозой насилия, хоргеты находятся под тотальным информационным контролем.
Деспотизм везде.

--- Анкарьяль в силу юношеского максимализма преувеличивает, но доля истины в её словах есть, --- признал Грейсвольд.
--- В Ордене нам дают полную свободу, но не забывают <<предупреждать>> или уничтожать, если мы что-то делаем \emph{подозрительно}.
Да, нам дают объяснения любых событий, но когда за фабрикацию доказательств берётся мощнейший отдел аналитики, что можно этому противопоставить?
Вычислительные мощности, опыт и количество актуальной информации прямо пропорцинальны доле власти.
Да, Лу?

--- Отчасти, Грейс, --- мягко ответил Лусафейру.
--- Будучи главным оборонительным стратегом, я трачу треть ресурсов на вычисление вероятности, с которой какое-либо моё действие вернётся ножом в спину.

Митхэ хохотнула и тут же осеклась, поняв, что стратег не шутит.
Он даже не улыбнулся.

--- Ваше удивление понятно.
Большинство из вас выросли в достаточно свободных сообществах.
В Ордене Преисподней степеней свободы и незанятых ниш крайне мало.
Любое действие любой группы тщательно анализируется всеми остальными --- не окажутся ли затронуты собственные интересы.
Объясняется всё это извечной байкой об ограниченных ресурсах, но истина так же извечна --- каждый старается урвать кусок как можно больше, про запас, чтобы он не достался тем, у кого кусков и так много, и не нарушил хрупкое равновесие.

--- Ну так идём с нами, --- вдруг страстно сказала Чханэ, наклонившись к Лусафейру.
--- Мы никогда не ударим тебе в спину.
За себя я отвечаю.

Лусафейру, похоже, смутился.
Не эмоционально.
Он просто понял, что это правда.

--- Таниа Янтарь, какая жизнь ожидает существо, которое обладает таким количеством ценных знаний?
Я пленник Ада, при попытке сбежать я не успею сделать и шага.
Если мне и удастся каким-то невероятным способом покинуть Капитул, кто меня защитит и обеспечит эманациями?
Ты?

--- Да!
Во мне боевые навыки Анкарьяль Кровавый Шторм!

--- Я не всесильна.
Не нужно делать из меня микоргета, --- возмутилась Анкарьяль.

--- Ты погибнешь совершенно зря, --- с лёгкой грустью заключил Лусафейру.
--- Я могу понять твой энтузиазм --- поначалу оцифрованные опьянены ощущением всесилия.
И многие это поддерживают, превращая оцифрованных новобранцев в пушечное мясо.
К сожалению, большинство из них гибнет в первые сто лет после демонизации.
Оставшиеся в живых, как и до оцифровки, начинают больше доверять логике и статистике, нежели рассказам о героизме и собственной исключительности.

--- Давайте не будем отклоняться от темы, --- прервал спорящих Грейсвольд.
--- Лу, так что?

--- Я не буду вам помогать, если ты об этом, --- сказал Лусафейру.
--- Это решать не мне.
Я, если вы не заметили, всего лишь базовая копия.
Если вы хотите возродить Тси-Ди на Тра-Ренкхале --- действуйте, я же буду действовать так, как должен.
Грейс, я ведь стёр разговор после того, как ты вышел от меня?

--- Ты подал сигнал, что стёр.
Что-то передать?

--- Хм, --- задумался Лусафейру.
--- Скажи мне при встрече следующее: <<Поклонись>>.
Интонация и мимика имеют значение.

--- Понял, --- кивнул технолог.

--- Тогда самоуничтожаюсь, --- сказал Лусафейру и исчез.

--- Стой, куда! --- простонала Чханэ.

--- Чханэ, так нужно.
Тем более что это только неполная копия, --- успокоил её Грейс.

--- Копия твоего друга, которая думала и чувствовала, как он, --- невзначай заметил Атрис.
Грейсвольд смущённо замолчал.

--- Что насчёт остальных? --- спросил Аркадиу.
--- Кто-то хочет отказаться от плана?

Ответом было молчание.

--- Тогда перейдём к главному, --- сказал Аркадиу.
--- Я --- официально предатель Ордена.
Митрис не принадлежит к Ордену, но связан с ним договором типа <<Демиург --- Метрополия>>.
О существовании Таниа Орден, будем надеяться, не знает.
А вы, --- обратился он к Грейсвольду и Анкарьяль, --- верные служители с безупречной репутацией.
Если вас уличат в сомнительных действиях --- отдел 100 распутает клубок очень быстро, и Ад ждёт такая чистка, какой не было со времён его становления.

--- Именно поэтому я и приготовил \emph{это}, --- улыбнулся Грейс и помахал пустой рукой, словно сжимая в кулаке маленький круглый предмет.

--- Что это? --- поинтересовалась Чханэ.

--- Пакет ложных воспоминаний, --- сказал Грейвольд.
--- Его созданием занимались девятнадцать демонов из отдела аналитики.
Не волнуйтесь, --- поспешно добавил он, увидев, как Атрис вздохнул и прикрыл лицо рукой, --- не волнуйтесь.
Я разбил задачу на фрагменты, не вызывающие подозрений, и использовал самых лояльных демонов.
На всякий случай я скомпрометировал одного из них связью с Картелем, чтобы увести следствие в другую сторону.

--- Грис, при всем уважении, --- Анкарьяль скрестила руки на груди.
--- Ты один, а в отделе 139 находятся девятьсот шестьдесят демонов.
Ты не можешь быть умнее и дальновиднее тысячи специалистов.
Может, в этот раз твой слепой метод сработал, но это чистое везение и включать такие действия в план работы просто недопустимо.

--- Что в пакете? --- поинтересовался Аркадиу.

--- У нас с тобой возникли разногласия, и ты отбыл, не сказав куда.
Также в несколько ранних воспоминаний добавлены малозначимые эпизоды, которые обычно проходят мимо отчётов, но опытному аналитику намекают на твоё помешательство.
Пакетов всего... эээ... семь.
Да-да, Митхэ ар’Кахр, я подготовил для тебя отдельный.
Учтено всё --- структура личности, взаимоотношений, даже погрешности восприятия и анализа.
Если тебя, Аркадиу, или кого-то из твоих демонов поймают и вывернут наизнанку, о нашей с Анкарьяль причастности не узнает никто.
Обратное также верно.

Атрис нахмурился.

--- Тогда как мы будем держать связь?

--- Рисунки на песке, --- улыбнулась Анкарьяль.

Чханэ недоумевающе хмыкнула.

--- Помните, когда мы были детьми на Тра-Ренкхале, у нас была забава --- рисовать на песке? --- откликнулся Грейсвольд.
--- А ещё мы строили песчаные домики...

--- И когда набегала волна или поднималось солнце, домики рассыпались, --- на лице Митхэ появилась понимающая улыбка.
--- Пара волн --- и песок становился гладким, проступали водяные рёбрышки, словно к нему не прикасалась рука человека.

--- Я всё равно ничего не поняла, --- заявила Чханэ.

--- Тогда, милая, Вселенная преподнесёт тебе ещё один сюрприз, --- рассмеялась Анкарьяль.

Чханэ погрустнела.

--- Так значит, я теперь буду считать вас врагами?
Даже если вы умрёте?

Анкарьяль с нежностью посмотрела на подругу и кивнула.
Чханэ криво улыбнулась.

--- Вы лучшие, кого я знаю.

--- Удачи, --- тихо сказал Грейс и кивнул Анкарьяль.
Оба растворились в воздухе.
Оставшиеся дружно вздохнули и ушли в себя, ожидая, пока закончится интеграция пакетов воспоминаний в систему.

\section{[U] Проблема теплиц (переработать!)}

\textspace

Проблема создания кольцевой теплицы --- в организации генетического аппарата, оптимизации и стабилизации его работы.
Когда учёные Ада взялись за это дело, им пришлось работать с огромным количеством ферментов и каскадов.
Клетки, нагруженные таким количеством генов, быстро выходят из строя из-за случайных мутаций, даже в условиях стерильной среды.
А учёные-тси, согласно данным, решили эту проблему максимально изящно --- альтернативным сплайсингом.
Один ген этого создания, хоть и довольно длинный, мог дать такое количество вариаций, какое обычным живым существам и не снилось.
Скорость мутации генов кольцевой теплицы была ниже таковой в генах гистонов эукариот.
Кольцевая теплица также обладала уникальной системой слежения за собственным генотипом --- специальная ферментативная система фактически вычисляла хэш всех имевшихся хромосом и запускала апоптоз при любой мутации.
Подобное произведение генной инженерии не смог создать никто во Вселенной, даже хоргеты.

--- Интересно, как они выглядели, --- задумался я.

--- Понятия не имею, --- хмыкнул Грейс.
--- И вряд ли кто-то имеет.
По одним данным это растение, по другим --- среди тси были люди, которые с ними общались: ласкали их, разговаривали.
И теплицы им отвечали.
Вряд ли растение способно говорить.

\textspace

\section{[U] Нужна теплица}

\textspace

--- С помощью кольцевой теплицы можно в кратчайшие сроки оживить практически любую планету в обитаемой зоне, не пользуясь услугами хоргетов, --- сказал я.
--- А теперь представьте, что будет, если это знание станет общедоступным.
Всеобщая экспансия за пределы ка'нетовского радиуса.
Любой хоргет сможет двигаться сколь угодно далеко в любом направлении, найти подходящую планету и стать её полновластным демиургом.
Каждому хоргету --- по планете!
Ад и Картель находятся в истинном равновесии ровно до тех пор, пока у них есть физические границы.
Убери границы --- и старый мир, основанный на идее дефицита, рухнет, как когда-то рушились колониальные империи.

Я помолчал, собираясь с мыслями.

--- Второе направление работы --- планетарная система.
Нам необходимы её чертежи.
Демиурги беззащитны перед организациями демонов.
С доступностью системы планетарной защиты гегемонии Ада и Картеля придёт конец.
Более того, демиург вступит в мутуалистические отношения с сапиентной цивилизацией своей планеты, так как от её технического развития будет напрямую зависеть его собственная безопасность.

--- Мне не нравится то, что ты предлагаешь, Аркадиу, --- сказал Грейсвольд.
--- Жизнь демиурга для каждого демона --- это хорошо, но я хотел бы общаться с себе подобными, а не запираться в планете на веки вечные.
То, что ты предлагаешь, разрушит социум демонов --- как экспансия за пределы ка'нетовского радиуса, так и превращение демонов в богов.
И существует ещё одна проблема.
Для минус-демонов места в этой системе не будет.

--- Технологии тси помогут нам и здесь.
Вдруг существует возможность вывести нуль- или даже минус-сапиентов?

--- Вероятность крайне мала, --- рассудительно сказал технолог.
--- Боюсь, что если нам не удастся этого сделать, Картель будет драться до последнего.
А это значит, что они почти наверняка победят.
Поэтому, если ты действительно хочешь создать дивный новый мир, следует проработать момент с минус-демонами.

\textspace

\section{[U] Подруга}

Со дня битвы на Могильном берегу прошёл год.
Жизнь потихоньку возвращалась в привычное русло.
Вместе с добровольцами-хака и прибывшими с Запада мастерами тхитронцы восстановили большую часть города.
С крыши храма была видна лишь одна оставшаяся проплешина на месте квартала каменщиков;
квартал был одним из самых хорошо укреплённых районов восточного Тхитрона, после осады там не осталось камня на камне, и место поглотила густая молодая поросль.
Её было решено не трогать --- оборонять город больше было не от кого.

Анкарьяль стояла рядом со мной.
Её жёсткие прямые волосы, уже отросшие до плеч, трепал настойчивый юго-западный ветер.
Пока ещё незрячий, по-младенчески синий глаз, растущий в пустой глазнице, светился недовольством.

--- Мне это не нравится, --- заявила она.

--- Ты это уже говорила.

--- Повторяю ещё раз: мне это не нравится.

--- Что именно тебе не нравится?
Место встречи?
Выбранные люди?
Перспектива превращения этих людей в демонов?

--- Я не о них!

Я облокотился о колонну и вздохнул.

--- Тогда в чём дело?

--- Мы были прекрасной командой, Аркадиу, --- тихо сказала Анкарьяль.
--- Я, Грейс, ты.
Я знаю, что такое плохая команда, я знаю, о чём говорю.
Я не хочу тебя терять.

--- И мы можем остаться командой.

--- Нет, не можем.

--- Ну, если ты про прежние наши отношения --- стоять плечом к плечу, выполняя приказ --- то да, не можем.
Но есть другие возможности.

--- Ты вообще понимаешь, о чём ты меня просишь?

--- Я тебя ни о чём не прошу.

Анкарьяль потёрла висок и тоже облокотилась о колонну, вперив в меня взгляд разнокалиберных, разноцветных глаз.

--- Я помогла тебе устроить эту встречу, --- сказала она.
--- У тебя есть методика и люди.
Дальше --- сам.
Хорошо?

---  <<Сам>>?
<<Хорошо>>?

Анкарьяль вдруг сползла по колонне вниз, сжалась в комочек и спрятала лицо в ладони.
Она дрожала крупной дрожью.

Я погладил подругу по голове.

--- Пойду вниз, --- шепнул я. 
--- Время.

Ответом были сдавленные рыдания.

\section{[U] Вербовка}

\textspace

Собравшиеся недоумённо переглянулись.

--- Ты больше не будешь сражаться рядом со своими товарищами? --- спросил Акхсар.

--- Нет.

--- Почему? --- требовательно спросил воин.
--- Они показали себя, Ликхмас.
Если бы я мог, я бы встал с ними рядом без колебаний.

--- Я хочу собрать собственную команду.
Однако существуют определённые правила, и мои товарищи не могут присоединиться ко мне по регламенту.

--- Новая команда?
Ты хочешь набрать других демонов?

--- Да.
Вас.

\textspace

--- Анкарьяль как-то подметила, что технология у вас в крови.
Я верю, что это действительно так.

--- Мы --- не наши предки, Ликхмас, --- строго сказал Акхсар.
--- Мы не можем летать между звёзд, управлять вашими машинами...

--- Я видел, как ты обращаешься с моим мотоциклом.
Ты освоил механизм и управление за декаду, увидев его впервые в жизни.

--- Это же просто механический олень.
Я должен знать, как на нём ездить и как его чинить.

--- Чтобы стать демонами, вам придётся овладеть такой же технологией.
Просто мотоцикл будет немного сложнее.

Акхсар нахмурил кустистые брови и задумался.

--- А ты, Кхотлам, --- обратился я к кормилице, --- я отдал тебе лишний компьютер для твоей работы.
Сейчас ты уже свободно пишешь на нём программы.

--- Ещё один язык, --- пожала плечами Кхотлам.
--- И немножко математики.

--- Истина в том, что на это не способны взрослые в других племенах, --- громко сказал я.
--- Ни хака, ни зизоце, ни прочие подобные вам народы во Вселенной не способны вот так, походя, понять принцип работы машины или концепцию квантовой архитектуры.
Вы были рождены для работы с машинами.
Многие из вас считают демонов сверхъестественными существами, но в действительности вас --- именно вас, потомков тси --- от демонов отделяет один шаг.
И я могу помочь сделать этот шаг.

--- Ты сделаешь нас бессмертными, Ликхмас? --- спросил кто-то.

--- Вы получите долгое существование, но не бессмертие.
Если сапиентам уготована лишь одна первичная смерть, то вы будете умирать ею сотни и тысячи раз.

Многие опустили головы.

--- Лисёнок, --- сказала Кхотлам, --- значит ли это, что мы переживём всех родных?

--- Вы можете пережить даже свой собственный народ и землю, на которой стоите, если вам повезёт.
Если же вас настигнет вторичная смерть, то не будет для вас пристанища --- вы умрёте навсегда, и никогда уже не будете живыми.

Все дружно вздохнули.
Акхсар громко скрипнул зубами.

--- Что я буду видеть и слышать, если стану демоном? --- спросил воин.

--- Вы испытаете ощущения, которые даже не можете себе представить.
Будут среди них и неприятные, и поистине ужасные.
Вам придётся долго и упорно тренироваться, чтобы управлять своими новыми силами.

--- Зачем ты предлагаешь нам такую участь? --- недоумённо поинтересовалась женщина с краю.
--- Менять одну смерть на тысячи, бессмертие и покой в пристанище на борьбу и забвение...

Окружающие закивали, выражая согласие.

--- Мне нужна помощь, --- просто сказал я.
--- Я не требую от вас, не взываю к чести предков, не рассказываю о тех чудесах, которые вы познаете, хотя и чудес будет немало.
Я смиренно прошу о помощи, потому что это --- величайшая и, вполне возможно, напрасная жертва.

--- Почему напрасная? --- удивился Акхсар.

--- Вы помните владычество Эйраки.
Это был единственный угнетатель.
Десятки тысяч миров стонут под гнётом других, не менее могущественных демонов.
И многие из них не одиноки, существуют целые организации, обладающие огромными ресурсами и глубокими знаниями.
Мы можем проиграть, и от нас не останется даже воспоминаний.

--- Если мы все погибнем, что-то изменится? --- спросил Акхсар.

Я улыбнулся.
Тот же вопрос задавали мне многие перед битвой на Могильном берегу;
ответ на него ничуть не изменился.

--- Тем, кто пойдёт за нами, будет легче.

--- Чем мы можем тебе помочь, братик?
Мы --- простые люди, --- заметила Лимнэ.
--- Я владею клинком недостаточно даже для того, чтобы сразить умелого воина...

--- Я научу вас тому, что знаю.
Мне понадобятся не столько воины, сколько те, кто готов трудиться и искать --- знатоки живых существ, строители машин, книжные люди.
Даже те, кто хорошо умеет что-то одно, смогут помочь.

Сестрёнки кивнули в знак того, что поняли.

--- Даю на размышление пять дней.
Если кто-то решит, что готов --- жду вас здесь в Змею 1, на восходе солнца.
И помните --- обратного пути уже не будет.

\section{[U] Первые}

--- Никто не придёт, как пить дать, --- сквозь зубы процедила Анкарьяль.
За последние два дня я услышал эту фразу не менее ста раз.
--- Ты вообще узнавал, как проходит вербовка?
Сапиентам предоставляют полную информацию, а ты сделал всё, чтобы их отпугнуть.

--- Придут немногие, --- согласился я, --- но их ценность выше, чем у тех, кто зарится на долгую жизнь и чудеса.

--- Никто не придёт, --- повторила Анкарьяль.
--- Я почувствовала, как их напугали слова о посмертии.
Пристанища не будет!
Такое и в страшном сне привидеться не может.

--- Его не будет в любом случае, --- тихо напомнил я.
--- Просто сели, как ни крути, дикари, а вот демонам верить в такие сказки не к лицу.

Мы ждали долго в напряжённом молчании.
Тихо потрескивая, горела свеча.
И вот в назначенный час раздался стук в дверь.

--- Ты проиграла, --- бросил я, направившись к двери.
--- Один есть.

--- Много ты сделаешь с одним, --- горько прошептала Анкарьяль.

За дверью меня встретили девять человек.
Я молча пропустил их в дом.
Все, не дожидаясь приглашения, принялись наполнять свои чаши чаем из котелка.

--- Что сказала кормилица? --- обратился я к сестрёнкам.

--- Она не пойдёт.
Акхсар хотел, но сказал, что бросить Кхотлам будет для него бесчестьем, --- ответила Манэ.
--- Кормилица дала ему второй шанс.

--- А ваш мужчина?

--- Он не хочет лишаться пристанища, но сказал, что целиком нас поддерживает, --- объяснила Лимнэ.

--- Ты же говорила, что хочешь покоя и не желаешь бессмертия, --- улыбнулся я Чханэ.
Подруга не ответила на улыбку, её взгляд был предельно серьёзен.

--- Когда мы говорили, всё было хорошо, --- проворчала она.
--- Теперь я вижу, что не всё в порядке и моё присутствие необходимо.

--- А ты почему пошёл, Митрам? --- обратился я к молодому ювелиру.

\ml{$0$}
{--- Ты сказал про строителей машин, --- пожал тот плечами.}
{``You mentioned machine builders,'' he shrugged.}
\ml{$0$}
{--- Я люблю машины, увлечён ими с детства.}
{``I love machines since I was a child.}
\ml{$0$}
{Посмертие --- весьма разумная цена за знание.}
{Afterlife is a good price for knowledge.''}

<<Эти слова могли бы принадлежать древнему тси>>, --- невольно вырвалось у Анкарьяль.
Я мысленно согласился.
Произнёсший подобное --- уже не дикарь.

--- А вы --- жрецы, --- поклонился я двум мужчинам и женщине в синих робах.
--- Врач, счетовод и...

---  ... и Митхэ ар'Тра, --- закончил приятный контральто.
Женщина откинула капюшон робы, и слабый свет выхватил из темноты тяжёлые серебристые волосы и благородное лицо, в котором я вдруг обнаружил знакомые черты.
--- Масло-Взбитое-Взглядом.

--- Я чту традиции Севера, Митхэ, --- улыбнулся я.
--- Учитель должен знать всех учеников по именам.

--- А мы с братом просто пришли помочь, --- вмешался крестьянин Согхо, махнув на стоящего рядом незнакомого мужчину.
--- Я всю жизнь обрабатывал землю, знаю, как выращивать кукурузу, помидоры и цветы.
Мой брат --- змеелов.
Мы оба любим своё дело.
Можем ли мы пригодиться?

Я улыбнулся.
Два диктиолога, интерфектор, технолог, три когитора-информатика и два биолога.
Превосходное начало.

--- Вполне, --- кивнул я.
--- Только вы понятия пока не имеете, что вам придётся выращивать и каких змей ловить.

<<Ты же не надеешься, что они сразу станут профессионалами? --- скептически произнесла Анкарьяль.
--- Они дикари, Аркадиу.
Многие из них могут сойти с ума даже из-за процесса демонизации>>.

<<Лесные духи, --- махнул я рукой.
--- Уж ты-то помнишь, с чего начинал я>>.

<<Ты был ребёнком>>, --- напомнила Анкарьяль.

Тут она права.
Мысль набрать в свои ряды детей я уже рассматривал.
Но перед глазами тут же вставал Отбор, когда ничего не понимающему созданию предлагали сделать главный выбор в его жизни...

Анкарьяль понимающе кивнула.
Она, как обычно, читала мои мысли.

<<Что ж, тебе придётся работать с тем, что есть>>.

<<,,Тебе``?
А ты со мной, Нар?>>

Подруга вздохнула и отошла набрать чаю из котелка.

\chapter{[:] Агнец}

\section{[:] Желание демона}

<<Тахиро повзрослел>>.
Лу, откинувшись в кресле, любовался другом.
Длинные чёрные волосы, миндаль огромных глаз, острые скулы... и ни намёка на бороду.

--- Что-то ты давно не сливал информацию повстанцам, --- сказал стратег.
Тахиро недоумённо замолчал.

--- Эээ...

--- Да не прикидывайся.
Первые два года ты был информатором у повстанцев.
Но потом прекратил с ними общение.

--- Я хотел помочь своему народу, --- смутился Тахиро.

--- Не оправдывайся.
Это было забавно.

--- А мне нет, --- вдруг резко сказал Тахиро.

--- Прости, не хотел тебя обидеть.
Но всё-таки, почему ты перестал?

--- Привык к сытой и спокойной жизни в лагере.
Любая зверушка в конце концов приручается.

--- Я никогда не считал тебя зверушкой.

--- Поэтому ты и смог меня приручить.

Друзья расхохотались.

--- На самом деле я просто потерял мотивацию, --- признался Тахиро.
--- Чем больше я узнавал о вас, тем больше понимал три вещи.
Первое --- вы не настолько опасны, какими вас представляют в поселениях.
Второе --- вы не настолько глупы, какими вас представляют в поселениях.
Третье --- вы не настолько едины и однообразны, какими вас представляют в поселениях.
Таким образом, для меня встал вопрос --- как с вами бороться правильно и нужно ли вообще с вами бороться.

\ml{$0$}
{--- И что надумал?}
{``Got any ideas?''}

\ml{$0$}
{--- Я в процессе.}
{``I'm in the middle of thinking.''}

--- Ну ты это, держи в курсе, --- Лу захихикал над собственной шуткой и тут же отчаянно закашлялся, подавившись слюной.

\textspace

--- Подумай только, Тахиро, --- мечтательно сказал Лу, --- речь о таких расстояниях и такой гравитации, что даже свет не летит прямо, а течёт, словно вода между камней!

--- Не вижу в этом ничего особенно удивительного.

--- Правда? --- удивился Лу.

Тахиро вместо ответа достал из сумки линзу.

--- Тоже вариант, --- немного ошарашенно признал Лу.
--- Но заставить свет летать кругами ты не можешь.

--- А смысл? --- ухмыльнулся Тахиро.
--- Кто увидит летающий кругами свет?

--- Ты споришь не ради истины, а ради победы, --- поморщился Лу.

--- Так я прав?

--- Мне нечего на это ответить.

Тахиро помолчал.

--- Какая-то безвкусная победа, --- наконец признался он.
--- Ты позволил мне победить, а не проиграл.
Это разные вещи.

Лу ухмыльнулся.

Закатные аспиды угасли, и небо расчертили брошенные горстью метеоры.

--- У нас принято загадывать желание, когда сгорает метеор, --- сказал Тахиро.
--- Говорят, оно непременно сбудется.
Только нужно сказать его вслух.

Люцифер засмеялся.

--- Я думаю, что для камня, пролетевшего миллиарды километров, оскорбительно служить моим мелким желаниям.

--- Он не обидится, --- Тахиро похлопал Лу по плечу.
--- Без тебя его существование было бы совершенно бессмысленным.

--- Скажи, как у тебя так выходит?

--- Ты о чём? --- не понял Тахиро.

--- Вот взять меня.
Во мне масса информации --- факты, формулы и структуры, связанные между собой определённым образом.
Это большая часть того, что нам удалось собрать из наследия первых людей.
Можно сказать, я квинтэссенция их науки.
Но ответы на все вопросы есть только у тебя.
И я бы не сказал, что они лишены смысла.
Иногда мне даже кажется, что твои слова спали во мне всю мою жизнь и проснулись, едва ты заговорил.

--- Это называется <<философия>>, сын.

Арракис, как обычно, подошёл незаметно.
Друзья переглянулись.

--- Я уверен, что у тебя в памяти есть масса философских концепций, --- продолжал Арракис, положив холодные как лёд руки на плечи Лу.
--- Жизнь ещё научит тебя различать их в речах людей.
Философия помогает людям преодолеть акбас, помогает им защититься от бесконечности, хаоса и случайности, царящих во Вселенной.
Та же философия делает их ограниченными и невежественными.

--- Почему? --- поинтересовался Люцифер.

--- По причине, которую ты уже озвучил.
Люди уверены, что у них есть ответы на все вопросы.
А теперь оставь своего питомца наблюдать за звёздным небом --- ему это полезно.
Ты нужен мне в штабе.
В конце концов, ты создан для того, чтобы править, не так ли?

--- Подожди, Лу, --- Тахиро схватил друга за руку.
--- Один из метеоров точно был твоим.

--- Если я создан для того, чтобы править, --- Лу бросил взгляд на Арракиса, --- я бы стал лучшим из правителей.
А ты, Тахиро?

--- Я бы сделал всё, чтобы лучший из правителей дожил до конца войны, --- сказал Тахиро.

--- Увидимся за ужином, --- Лу улыбнулся и растворился в ночной тишине.

\section{Ксидатин}

--- У меня нет уверенности, что мы решили проблему производства ксидатина, --- проговорил Лу, крутя в пальцах тонкий слиток циркония.

--- Лаборатория уничтожена семь лет назад, Люцифер.
По твоим же подсчётам, она могла обеспечивать почти всех, кто был в бункере.

--- А еще мы нашли шприцы с дозами у деревенских, не далее месяца назад.
Что это было --- старые запасы?
Вы брали у трупов разведчиков пробы на ксидатин?

--- У нас нет ресурсов исследовать каждый плевок, Люцифер.

--- Именно поэтому версию с несколькими лабораториями отметать не следует, --- заключил Лу.
--- Ксидатин легко синтезируется из аптекарских препаратов.
Единственной проблемой являются катализаторы, --- Лу помахал циркониевым слитком, --- но если их добыть --- использовать можно почти бесконечно.
А препараты мы по понятным причинам не можем изъять из оборота.

--- По понятным причинам? --- уточнил Гало.

Лу закатил глаза.

--- Ладно, ладно, я понял, --- поднял руки Гало.
--- Хватит делать \emph{это лицо}.

--- Я не понимаю, в чём проблема, --- сказал Арракис.

--- Для Лу изьятие лекарств --- табу, --- сообщил Гало.

<<Это какой-то сюр, --- устало подумал Люцифер.
--- Какое в бездну табу?
Я точно не сплю?>>

--- Надеюсь, вы не хотите устроить эпидемический бунт? --- ровным голосом сказал он.

--- Мы можем изъять ограниченное число лекарств, --- пожал плечами Арракис.

--- Изъятие любого лекарства --- увеличение заболеваемости, --- объяснил Гало.
--- Увеличение заболеваемости --- увеличение риска эпидемий.
Лу имел в виду это.
Но я думаю, что...

--- В случае эпидемии дзайку-мару обвинят нас, ограничивавших оборот лекарств, --- перебил Лу.
--- И будут правы, несмотря на то, что их аргументы будут другие и логически неправильные.
Я против.
И даже не вздумайте выносить это на рассуждение --- я сделаю всё, чтобы наложить вето.

--- Я тебя понял, --- кивнул Арракис.
--- Тогда я хочу тебя попросить заняться проблемой ксидатина самостоятельно.

Лу кивнул.
<<Ну ещё бы ты не попросил>>.

\section{[:] Трубки}

\textspace

--- Что ты делаешь? --- удивился Люцифер.

--- Хочу проработать некоторые моменты стратегии, --- уклончиво ответил Гало.

--- У нас сейчас свободное время, --- напомнил Тахиро.

--- У тебя вся жизнь --- свободное время, дзайку-мару.

--- У тебя тоже, --- непринуждённо парировал Тахиро.
--- Ты --- вольная птица.

--- Я ценен как работник, а не как подопытное животное.

--- А в чём разница?

--- В уровне интеллекта, --- ядовито ответил Гало.

--- Это точно.
Ни один грызун не додумается крутить своё колёсико во время отдыха.

--- Курить хочется, --- лениво проворковал Люцифер.
--- Гало, кинь какую-нибудь трубку.

Гало запустил в тазик с трубками руку и бросил Люциферу первую попавшуюся.

--- Нет, Гало, не эту, --- сказал Люцифер, осмотрев брошенное.
--- У неё чересчур короткий чубук.

--- Ты сам сказал <<какую-нибудь>>.

--- А ещё я сказал <<не эту>>.

--- Это же очевидно, Гало, --- усмехнулся Тахиро.
--- Если Лу сказал <<какую-нибудь>>, то ты должен взять трубку и спросить, эту ли хочет Лу.
Если Лу сказал <<любую>>, то у него очень хорошее настроение и можно не спрашивать, а просто кинуть первую попавшуюся.

--- Что за чушь ты городишь? --- буркнул Гало.

--- Отнюдь, --- возразил Лу.
--- Тахиро правильно понял, что первое слово означает определённое нежелание в сочетании с неопределённым желанием.
Второе обозначает неопределённое желание при полном отсутствии нежелания.

--- Брат, тебе сложно просто сказать, какую трубку ты хочешь?

--- Трубки все мои, и я не обязан делать чёткий выбор между ними, --- объяснил стратег.
--- В теории я могу закурить столько трубок за раз, на сколько хватит рта и лёгких.
Так что, пожалуйста, кинь мне какую-нибудь трубку, а эту коротышку положи обратно.

--- Подними задницу и возьми сам.

Гало раздражённо выключил терминал и вышел из комнаты, хлопнув дверью.

--- Как хорошо, когда есть близкий родственник, --- Лу откинулся на спинку кресла и вперил взгляд в скучный каменный потолок.
--- Желания курить как не бывало.

\section{[:] Мико}

\textspace

--- Как тебя зовут?

Девушка что-то пробормотала.

--- Прости? --- Тахиро наклонился, припав ухом почти к её рту.

--- Мико из дома Мацумори, --- пролепетала девушка.

--- Я Тахиро из дома Ханаяма, --- кивнул Тахиро и тут же добавил, увидев расширившиеся от страха глаза:
--- Что-то не так?

--- Ты Ханаяма?

--- Да.
<<Благородная кожа --- лучшее одеяние>>.

--- <<В сосновом лесу нечисть не водится>>, --- машинально ответила Мико и запнулась.
--- Ты... ты не будешь мстить?
Мой дом...

--- Я знаю, Мико, --- Тахиро обнял девушку за плечи.
--- Ты не виновата, ведь ты тогда ещё даже не родилась.
Всё в прошлом.

\ml{$0$}
{--- Благодарю тебя, Тахиро, --- прошептала Мико.}
{``I thank you, Tahiro,'' Miko whispered.}
\ml{$0$}
{--- Кто-нибудь ещё из твоего дома спасся?}
{``Is there any of your kinsmen who survived as well?''}

\ml{$0$}
{--- Только я.}
{``Only me.}
\ml{$0$}
{Вернее, нас было двое, но...}
{Actually, there were two of us, but ...}
\ml{$0$}
{Моя сестрёнка была совсем малышкой, недавно родилась.}
{My sister, she was just a baby, newborn.}
\ml{$0$}
{Я не знаю, что с ней произошло.}
{I don't know what happened to her.}
\ml{$0$}
{Она долго плакала, потом перестала плакать и окоченела.}
{She had been crying for a long time, then she stopped crying and got rigor.}
\ml{$0$}
{Мне пришлось оставить её в пустошах.}
{I had to leave her in the wastelands.''}

\ml{$0$}
{--- Сочувствую, --- пробормотала девушка.}
{``I'm sorry,'' the girl mumbled.}
\ml{$0$}
{--- Мне очень жаль.}
{``I'm so sorry.''}

Тахиро смотрел на девушку и никак не мог насмотреться.
Она была воплощением нежности и женственности.
Когда Тахиро обнял её, на веснушчатом тонкокожем лице расцвела милая белозубая улыбка.
Узкие глаза с такэсакским, <<шу>> раскроем, светились радостью и смущением;
девушка крепко прижимала маленькие ручки к груди, и этот жест окончательно свёл Тахиро с ума.

--- Главное --- не бойся, --- улыбнулся Тахиро.
--- Сейчас ты в безопасности, тебя никто не обидит.
Когда ты придёшь в себя, я возьму тебя в жёны.

Мико хихикнула.

--- Правда?

--- Правда.
У тебя будет безопасный дом.

\textspace

Тахиро вдруг понял, что мысль о мести даже не пришла ему в голову.
Он знал, что Ишида и Мацумори виновны в гибели его дома, убийстве его матери и отца, но...

<<С каких пор мне стало плевать на мой дом? --- удивился Тахиро.
--- Едва ли это просто чувство к женщине>>.
Затем он посмотрел на Люцифера и Грисвольда --- Лу показывал другу какую-то очень весёлую пантомиму, и толстяк нервно икал в перерывах между приступами смеха.
Этих демонов не связывали ни родственные узы, ни место рождения, ни жизненный опыт.
Среди демонов Ордена было много таких --- без роду, без племени, рождённых неизвестно где, переживших неизвестно что, прибившихся к первым встреченным ими представителям своего вида.

<<Я такой же, как они, --- подумал Тахиро.
--- Мой дом мёртв, и всё, что мне досталось от него --- имя и меч.
Мне некого защищать, кроме меня.
Мне не за кого отвечать, кроме меня.
Наверное, это действительно больше не имеет значения>>.

Его взгляд снова упал на Мико.
Девушка украдкой рассматривала его издалека.

<<Думаю, нужно сделать ей подарок>>.

\section{[:] Изнасилование}

Тахиро достаточно неплохо научился управляться с ЧПУ-станками.
Большая часть из них, разумеется, управлялась омега-консолью, но примитивный нейроинтерфейс --- двухступенчатый зрительно-слуховой --- тоже сделали.
Видимо, на случай нехватки масс-энергии.
Тахиро просидел у 3D-принтера целых три часа, но браслеты получились действительно восхитительные --- такой тонкой работы в людских поселениях было не найти.
Титановая филигрань перемежалась с глазурованной мозаикой, изображавшей аистов, маки и сосновые ветви.

\textspace

Девушка дрожала с головы до ног.

--- А что произошло потом? --- спросила Айну.

--- А потом они... меня...

--- Изнасиловали, --- сухо подсказала Айну.

--- Да, --- шёпотом подтвердила девушка.
--- Меня и раньше... ещё когда я служила семье Таками... но в этот раз всё было кошмарно... я не представляла, что так вообще возможно.

--- Было очень страшно, --- подсказала Айну.

--- Да.
Больно не было, а вот страшно...

--- Всё ясно, --- сказала Айну.
--- Стимуляция.
Как и мы когда-то, они думали, что это эффективный способ.

Демон Айну вспыхнул --- и что-то приглушённо хлопнуло.
Девушка медленно легла на стол с застывшими глазами.
Изо рта вытекла тонкая струйка крови.

--- Зачем? --- рявкнул Тахиро.

--- Тебя никто не спросил, дзайку-мару, --- отозвалась Айну.
\ml{$0$}
{--- Избавься от трупа.}
{``Get rid of the corpse.''}

Тахиро схватился за дайту-соро и тут же упал на землю, заорав от боли.
На его теле появилось множество мелких болезненных царапин, словно на парня в один миг набросилась стая озлобленных крыс.

--- Я сказала --- избавься от трупа, --- не повышая тона, повторила Айну.

\section{[:] Пищевая цепь}

Грисвольд нашёл Тахиро не сразу.

Парень сидел за складами, рядом со свежей могилой.
Он снял с себя всю одежду, от него разило острым запахом какого-то крема, рядом лежала пустая ампула из-под обезболивающего, но Тахиро всё равно морщился при каждом движении.

--- Я не понимаю, как она могла мне нравиться раньше, --- пробурчал парень, пытаясь вытереть злые слёзы краем кимоно.
\ml{$0$}
{--- Бесчеловечная дрянь.}
{``Inhumane scum.''}

--- Она не плохая, Тахиро, --- сказал Грисвольд.
--- Поверь, я видел плохих демонов.
Айну не такая.

\ml{$0$}
{--- Она относится к людям как к вещам! --- рявкнул Тахиро.}
{``She treats humans like implements!'' Tahiro roared.}
--- <<Избавься от трупа>>!
Глоточная бомба!
\ml{$0$}
{Она забила девушку как свинью!}
{She slaughtered the girl like a pig!}
А ещё она использовала шугокскую пытку\FM, словно я какой-то... какой-то...
\FA{
Шугокская пытка --- нанесение множества мелких ран на тело.
}

Тахиро захлёбывался от гнева.
Грисвольд вздохнул.

--- Когда я был ещё молодым демиургом, я тоже относился к людям, как к материалу, --- сказал он.
--- Что произошло?
Я просто начал вас изучать.
Изучал как материал, а в итоге понял, насколько мы с вами похожи.
Я никогда не относился к людям как к равным, но вот это знание, которое я когда-то получил, не позволило мне относиться к ним, как к вещам.

--- Она же тоже...

--- Айну --- демон, Тахиро.
Форма жизни, отличная от вас, несмотря на то, что демоны часто существуют в связке с человеческим телом.
Она живёт уже в двадцать раз больше, чем ты.

Тахиро помолчал.

\ml{$0$}
{--- Я обещал Мико, что возьму её в жёны.}
{``I promised to marry Miko.''}
Я думал, что делаю ей свадебный подарок, а сделал погребальное убранство.

Грисвольд удивлённо посмотрел на могилу.

\ml{$0$}
{--- Ты о?..}
{``You're talking about---''}

\ml{$0$}
{--- Да.}
{``Yes.''}

Грисвольд огляделся.

--- А где твой соро?

--- В могиле, с ней.

--- Это же твоя семейная реликвия!
Его создали ещё на Древней Земле!

\ml{$0$}
{--- Это просто железяка.}
{``It's just a piece of iron.}
\ml{$0$}
{Но чтобы отогнать сущности на чёрной тропе, нужно оружие.}
{But she needs a weapon to repel creatures of the Black Road.}
\ml{$0$}
{Надеюсь, хотя бы против демонов посмертия этот меч чего-то стоит.}
{I hope this soro is worth something, at least against afterlife demons.''}

Тахиро прижался к прохладной стене склада и с облегчением задышал.
По его лицу снова потекли слёзы.

--- Как же мне надоело бессилие человеческого тела.

\ml{$0$}
{--- Я тебя понимаю, --- кивнул Грис.}
{``I can understand,'' Gris nodded.}
\ml{$0$}
{--- Сотня царапин --- и обладатель самого смелого сердца может лишь сидеть и стонать от боли.}
{``A hundred scratches, and the bravest heart can only sit and groan in pain.''}

\ml{$0$}
{--- Я не стонал.}
{``I wasn't groaning.''}

\ml{$0$}
{--- Вы, люди, придумали какие-то глупые ограничения для нормальных реакций организма.}
{``You people invented plenty stupid restrictions for normal reactions of your bodies.}
\ml{$0$}
{Я не перестану считать тебя храбрецом из-за слёз и стонов.}
{I keep thinking of you as a brave man even if you cry and groan.}
\ml{$0$}
{Это вообще никак не связано.}
{Actually, these things have no connection.''}

\ml{$0$}
{--- Правда?}
{``Is it true?''}

\ml{$0$}
{--- Правда.}
{``It is.''}

\ml{$0$}
{--- Тогда можно я поплачу?}
{``May I cry a little, then?''}

\ml{$0$}
{--- Не нужно спрашивать разрешения.}
{``No need to ask permission.}
\ml{$0$}
{Можешь даже покричать.}
{You may even scream.''}

Тахиро испустил дикий протяжный рёв боли, многократно отозвавшийся эхом в молчащем лагере, и из его глаз хлынули слёзы.
Рыдания рвались из сдавленного горла, словно обезумевшие птицы из клетки.
Грисвольд почувствовал, что парню необходима ласка, протянул руку --- и тут же отдёрнул, поняв, что причинит прикосновением ещё больше страданий.

<<Надеюсь, сегодня будет дуть свежий прохладный ветер>>, --- подумал технолог, глядя в тёмное небо.
На своей планете он знал, как поднять бриз и свернуть воздух в торнадо;
на этой приходилось довольствоваться надеждой.

На шум пришёл легионер.
Увидев Грисвольда, он вытянулся и отвесил короткий поклон:

--- Что-то случилось, сама?

--- Всё в порядке, легионер, --- сухо ответил Грисвольд.
--- Возвращайтесь на пост.

--- Вас понял, сама.

Легионер ушёл за склады, и ветер закружил сухие листья на голом камне.
Несколько мгновений мимо Тахиро и Грисвольда летел крохотный смерч, вбирая в себя щепки и пыль;
собеседники проводили его взглядами.
Тахиро суеверно выставил вперёд ладонь и что-то прошептал в крепко сжатый кулак.

--- Грис, можешь сделать меня демоном?

Технолог смутился.

--- Методика ещё сыровата, --- уклончиво сказал он.
--- Пока что я с командой провожу опыты на... на других людях.
Пожалуйста, не смотри на меня так.
Если эти опыты не проведу я, их проведут другие.

--- По-твоему, это оправдание? --- осведомился Тахиро.

--- Не нужно ставить мне это в вину, --- холодно буркнул Грисвольд.
--- Когда-то вы проводили опыты на мышах.
Сейчас человечество в мышах не нуждается.
Чем вы, люди, отличаетесь от этих зверьков?

--- У нас есть...

--- Ничем, --- перебил парня Грисвольд и поднялся на ноги.
--- Когда мы добьёмся первых стабильных результатов по оцифровке, я превращу тебя в демона.
Тогда --- и только тогда --- твоё слово будет что-то значить против слова Айну.

--- Вы за это заплатите, --- пробурчал Тахиро сквозь слёзы.
--- Я клянусь порогом моего дома, вы заплатите!

--- А кто привлечёт нас к ответу? --- криво улыбнулся Грисвольд.
--- Ты?

Тахиро промолчал.

--- А пока подумай над одной вещью, Тахиро, --- Грисвольд многозначительно поднял палец.
--- Что, если бы мыши могли привлечь к ответу вас, м?
Все тысячелетние рассуждения о справедливости человечество вело, находясь на вершине пищевой цепи.
Хорошенько над этим подумай.

--- Тогда я верну людей на причитающееся им место, --- запальчиво бросил Тахиро.
--- Ваше племя вершины пищевой цепи недостойно.

Грисвольд хмыкнул и пошёл прочь, проворчав себе под нос что-то неразборчивое.

\section{[:] Извинение}

\textspace

--- Грис, прости, --- тихо сказал Тахиро.
--- Я срываюсь на тебя иногда, я знаю...

--- Я не обиделся.

--- Ты хорошо на меня влияешь, --- горячо заговорил Тахиро.
--- Я чувствую, как после наших бесед с меня спадает вся эта дикость, в которой я рос и продолжаю жить сейчас.
Жестокость, презрение к людям.
Странно, но во мне этого куда больше, чем в тебе, хоть ты и не человек.

Грисвольд кивнул и углубился в свои записи.

\ml{$0$}
{--- Давай сбежим, --- вдруг зашептал Тахиро.}
{``Let's go far away,'' Tahiro suddenly whispered.}
\ml{$0$}
{--- Плевать на Орден.}
{``Screw the Order.}
\ml{$0$}
{Оцифруй меня, мы найдём планету и будем там богами, будем разбивать сады и возводить дворцы.}
{Digitize me, and we'll find a planet to be her gods, we'll lay out gardens and build palaces.}
\ml{$0$}
{Ты, я и Лу.}
{You, me, and Lu too.}
\ml{$0$}
{Я предпочту умереть, защищая последний приют, чем раздирать этот несчастный клочок камня на ресурсы.}
{I'd rather die for my last home, instead of tearing that piece of stone apart for resources.''}

Грисвольд невольно вздрогнул.
Это не были слова дзайку-мару;
юноша говорил как хорохито.

\ml{$0$}
{--- Твоя мечта исполнится очень скоро, ты умрёшь, --- сухо ответил технолог.}
{``Your dream will come true soon, you'll die,'' the technologist coldly answered.}
--- Ты же слышал слова Арракиса на последнем заседании штаба.
Союз Воронёной Стали действует так же.
С нейтралами сейчас разговор короткий.
Их скоро не останется, все демоны будут втянуты в войну.
\ml{$0$}
{Без Ордена мы не выживем.}
{Without the Order we can't survive.''}

\ml{$0$}
{--- У нас нет выбора?}
{``We have no choice, then?''}

\ml{$0$}
{--- Нет.}
{``We haven't.}
\ml{$0$}
{По крайней мере пока, --- Грисвольд позволил себе лёгкую улыбку.}
{At least not yet.'' Griswold smiled.}
\ml{$0$}
{--- Но я подумаю над твоим предложением.}
{``I can think over your proposal, though.}
\ml{$0$}
{А тебе лучше поспать.}
{And you could use some sleep.''}

\ml{$0$}
{--- Едва ли я сегодня засну, --- буркнул Тахиро.}
{``I can hardly sleep tonight,'' Tahiro said.}

--- Возьми из лаборатории стерильные простыни и попроси у Лу морфин.

\ml{$0$}
{--- Это зло.}
{``It's evil.}
\ml{$0$}
{Морфин превращает людей в животных, и даже одной дозы достаточно...}
{Morphine turns human into animal, and one dose is enough to---''}

\ml{$0$}
{--- Избавь меня от этих глупых суеверий, --- перебил его Грисвольд.}
{``Spare me those silly superstitions,'' Grisvold interrupted him.}
--- На наркоту садятся лишь рабы и люди, глухие к собственным желаниям.
\ml{$0$}
{Взгляни на Лу --- он курит адский вереск раз в неделю, а иногда месяцами трубку в руки не берёт.}
{Look at Lu: he smokes hell heather one time a week, and sometimes he doesn't touch a pipe for months.}
\ml{$0$}
{Просто потому что ему некогда, или потому что не хочет.}
{Just because he has no time, or no desire.}
\ml{$0$}
{Скажешь, он такой же наркоман, как те, в поселении, у которых от трубки уже губы гниют?}
{You think he's a junkie like those, from the village, who have rotten lips because of pipe?''}

\ml{$0$}
{--- Отец говорил, что все, кто хоть раз вдохнул дым --- наркоманы.}
{``My father used to say: anyone who ever breathed the smoke is a junkie.}
\ml{$0$}
{Мой дед...}
{My grandfather---''}

\ml{$0$}
{--- Твой дед познал удовольствие.}
{``Your grandfather have known the pleasure.}
\ml{$0$}
{Твой отец осознал опасность.}
{Your father have known the danger.}
\ml{$0$}
{Тебе предстоит увидеть закономерность, взвесить пользу и вред.}
{And you're the one to understand the patterns, weigh the pros and cons.}
\ml{$0$}
{Именно так люди и переходили от веры к знанию.}
{That's how people go from faith to science.''}

\ml{$0$}
{--- Откуда ты знаешь, что это предстоит именно мне?}
{``How do you know I'm the one?''}

\ml{$0$}
{--- Я не знаю никого другого, кто на это способен.}
{``I know no one else who can do that.}
\ml{$0$}
{Возьми у Лу морфин и поставь капельницу.}
{Get some morphine at Lu's, and put in an IV.}
\ml{$0$}
{От одного раза ничего не будет.}
{Nothing's gonna happen because of one dose.''}

\chapter{[U] Побег}

\section{[U] Просьба о прощении}

\epigraph
{
\ml{$0$}
{Чему учит нас история Тси-Ди?}
{What the history of Qi-Di teaches us?}
\ml{$0$}
{Всегда оставляйте своим врагам шанс вести достойное существование.}
{Always give your enemies a chance to make a decent living.}
\ml{$0$}
{Любая идеология бессильна против отчаяния обречённых.}
{Every ideology is powerless against desperation of the doomed.}
\ml{$0$}
{Великая Матка сказала, что ни один апид-солдат не вернётся домой, пока Молокоеды и Тараканы не будут истреблены.}
{Huge Queen said: no Soldier of Apis will return home until Milkers and Cockroaches are wiped out.}
\ml{$0$}
{<<Да будет так>>, --- ответили тси и втоптали Ульи в песок.}
{``So be it,'' said Qi and stomped Hives in the dust.}
}{Кельса Пушистая}

\textspace

--- Трукхвала убили, --- сказал Грейсвольд.
--- Короля-жреца.

Я выругался.

--- Кто?

--- Наши, разумеется.

--- Я же сказал им, что с Трукхвалом проблем не будет!
Я проинструктировал его, чтобы он сложил с себя обязанности по первому требованию!

--- Боюсь, --- печально улыбнулся Грейсвольд, --- у него ничего и не требовали.

Я сделал несколько кругов по комнате.

--- Ты уверен, что это убийство?

--- У него на теле было несколько необычных кровоподтёков.
Ты знаешь, наши работают чисто, для сапиентов всё выглядело как обычный несчастный случай.
Однако кровоподтёки... он щипал себя перед самой смертью.
Ранить себя он не мог --- убийцы почуяли бы кровь и заинтересовались.

--- Это были просто щипки или...

--- Или символ, да.
Два знака письменности тси --- <<уголок>> и <<цепочка>>.

Модификатор пассивного залога и конец предложения.
<<Что-то, результат чего ты наблюдаешь, было сделано кем-то>>.

--- Он пытался оставить послание жрецам?

--- Неясно, кому предназначалось послание, но он хотел, чтобы об убийстве узнали.

--- От кого исходил приказ?

--- Штрой Кольцо Дыма, командующий силами Ада.
Был ли это её приказ или свыше, не знаю.

--- Давай, Минь, объясни мне ещё и это! --- в сердцах обратился я к архивариусу.
--- Этот человек поднял земли сели, идолов и хака из руин.
Он в жизни не причинил вреда даже птице, не говоря уже о сапиенте.
Однако Ад уничтожает его --- мягкого, покладистого лидера, который вёл свой народ к процветанию!

--- С этим не всё так просто, --- пробормотал Минь.
--- Да, демоны Ада длительное время поддерживают процветание сапиентов, чтобы получать плюс-эманации.
Но дело в том, что неограниченная эволюция невыгодна --- существует опасность восстания против Ада.
Тси-Ди ярко это показала --- тси не согласились на власть демонов.

--- И каким же образом Ад решает эту проблему? --- глухо проворчал я, зная ответ наперёд.

--- Апокалипсис.
Полное низвержение цивилизации, доведение сапиентов до первобытного состояния.
Затем цикл повторяется.

--- И даже это запланированное событие используется в целях пропаганды, --- добавил Грейсвольд.
--- Угадай, кого в нём обвинят.
Обычная история.

--- Так значит, именно поэтому Ад настороженно отнёсся к генофонду тси, --- проговорил я.
--- Тси улучшили свои тела, сделав их сверхприспособленными и быстрообучаемыми, и легко могут выйти из-под контроля.

--- Аду невыгодно существование потомков тси в их нынешней форме, --- резюмировал Грейсвольд.
--- Я не думаю, что тси будут уничтожать, но отрицательный отбор неизбежен.
Возможно, Трукхвал и не представлял непосредственную опасность как личность, но он определённо обладал опасными с точки зрения Ада генами.
Ты же помнишь про отбор Королей-жрецов?

Я встал и прошёлся по комнате, теребя пальцами браслет-терминал.
Выходит, все наши старания были зря.
Мы подвергали себя риску и отправляли сапиентов на верную смерть ради ещё одной рабской хижины размером с планету.

--- А я пообещал Митрису, что его народ будет счастлив.

--- Аркадиу, --- Грейсвольд, похоже, не на шутку испугался последней фразы, --- надеюсь, ты не собираешься сделать какую-нибудь глупость?

--- Разумеется, нет.

--- Тогда куда ты собрался?

--- Просить прощения, --- сказал я и сдавил браслет.
Бесшумно разъехались двери комнаты, и я вышел наружу, в переплетение стальных коридоров.

\asterism

Я знал, сколько времени требуется отделу 100, чтобы заняться мной вплотную.
Статистика по захватам агентов Картеля стоила мне больших затрат.
Тридцать секунд на обработку информации обо мне, столько же на принятие решения, минута на то, чтобы роботы контрразведки подобрались к моему телу.
Дальше ожидание и слежка, затем захват или уничтожение.

Минь уже должен был передать информацию.
Бедный архивариус вынужден был переступить через собственную гордость из-за чувства страха, не зная, что и это входило в мой план.

Вокруг стояла тишина.
Я бы даже сказал, что было чересчур тихо.
Скорее всего, контрразведка уже парила на расстоянии вытянутой руки, прикрытая всей мощью технологий Ада.

Я направился в архивы.
Существовал только один способ сбежать --- нужно было заинтриговать отдел 100 и убедить их, что мой побег принесёт для дела больше пользы, чем моя поимка.


\section{[U] Виа}

Лифт опустился на тридцать восьмой этаж.
Вокруг по-прежнему было подозрительно тихо, и я тихо вздохнул, словно озабоченный чем-то.
На самом деле это был вздох облегчения --- мои действия явно не укладывались в предсказанную аналитиками схему, и контрразведка медлила с поимкой.

Дверь отъехала в сторону, и я встретился глазами с молодой женщиной-плантом.
Архивариус Виа Серая Ласточка --- так её зовут.
Она молча улыбнулась и кивнула.
Разумеется, отдел 100 уже предупредил и проинструктировал её.

--- Здравствуй, Виа, --- поклонился я.
--- Мне нужна твоя помощь.

--- Секунду, --- сказала она и, приложив палец к виску, закрыла глаза.
Открыла.
--- Извини, Аркадиу.
Перенаправила срочный запрос.
Слушаю тебя.

--- Мне нужна как можно более точная временная лента последнего сантителльна существования цивилизации Вилет'марр.

--- Лотос? --- Виа смотрела на меня с интересом, но я обострёнными чувствами ощущал некоторую наигранность.
Логика понятна --- люди Тра-Ренкхаля были потомками вилет'марр, и аналитики контрразведки предупредили архивариуса о таком направлении беседы.

--- Да.
Дело в том, что некоторые аналитики...

--- Некоторые --- это кто?
Выражайся точнее, Аркадиу.
Я не могу тратить время на проверку источников.

Провокация.
Я был к ней готов, но знал, что мой ответ не идеален.
Если Виа или держащие с ней связь аналитики за ближайшие двадцать секунд найдут, к чему придраться --- пиши пропало.
Они не знали, что я осведомлён о подключении к делу контрразведки, и не могли затягивать беседу --- это выдало бы их с головой.

--- Гимел Кадмиевый Зелёный, диктиолог Картеля, действовавший...

--- Он погиб на Тра-Ренкхале во время твоего последнего задания? --- прикрыв глаза, спросила Виа.

--- Да, я его убил.

На самом деле Гимел погиб во время состязания грубой силы между Безымянным и Эйраки, но знал это только я.

--- Ты поддерживал с ним связь? --- поинтересовалась Виа.

--- Информация получена в момент его уничтожения.

--- Почему сказанное тобой расходится с отчётом? --- снова прикрыла глаза Виа.

--- А почему я пришёл к тебе сам, а не написал запрос по общему каналу? --- задал я риторический вопрос.

Виа растерялась и немного испугалась.
Она поняла, что дело действительно серьёзное --- если бы не предупреждение контрразведки, она бы и не поняла, что перед ней шпион.
Всё укладывалось в привычные рамки --- мои человеческие повадки, мои полномочия как легата терция не раскрывать некоторую информацию до окончания собственного расследования и даже промедление в пять лет, которое было призвано охладить интерес агентов Картеля к моему последнему заданию.
Плюс репутация --- Виа изучала досье большинства демонов и никак не могла ожидать от меня подвоха.

--- Хорошо.
Чем располагал Гимел?
Я тебя слушаю, --- Виа справилась с собой, и в её глазах загорелся спокойный огонь готового к работе профессионала.

<<Пока неплохо>>.

--- Гимел вычислил время прибытия людей Лотоса на Тра-Ренкхаль.
Использовались данные археологии, астрономии и генетики.
Точность весьма высока --- плюс-минус два года.

Археология у Картеля всегда была своеобразной --- образцы оцифровывались и тут же уничтожались, чтобы их не изучили враги.
Культурологи Ада, среди которых были и оцифрованные сапиенты, настояли на запрете подобного вандализма.
Разумеется, без крайней необходимости.
А биологи Ада, благослови лесные духи их расхлябанность, увлеклись изучением тси и девиантной фауны, отложив в долгий ящик отчёт о людях Тра-Ренкхаля.
Я был свидетель тому, как начальника отдела распекал прибывший чин за то, что биологи не работают над проблемами совместимости демонов с местными сапиентами.
Тот смущённо оправдывался, мол, на планете чересчур много интересного.
В итоге отдел настрочил под диктовку Безымянного отписку для Капитула, и на этом изучение людей Тра-Ренкхаля завершилось.

--- Поняла, --- сказала Виа.
--- Временная лента готова.
Ты хочешь ознакомиться с ней сам или я тебе ещё нужна?

Я улыбнулся и решил ещё раз погладить отдел 100 против шерсти:

--- Если ты не занята, то я хотел бы разобрать её с тобой.

Виа несколько растерянно махнула мне рукой, призывая следовать за ней.

\section{[U] Ветер свободы}

Вскоре мы сидели в пустом кабинете перед экраном компьютера.
Виа непроизвольно вжималась в кресло, по её лбу струился пот;
похоже, что здесь был агент контрразведки, прикрытый щитом невидимости, и женщина опасалась ненароком его задеть.
За подобную неосторожность её могли лишить должности.
Эти случаи не афишировались, но о них знали все --- почему-то решающим в вопросе профпригодности демона было слово не его коллег по отделу, а военных.

--- Скажи мне как аналитик, в какое время вилет'марр могли послать экспедицию на Тра-Ренкхаль? --- спросил я.

--- Не раньше 1190-го оборота, --- не задумываясь ответила Виа.
--- Ранее вилет'марр просто не располагали нужными технологиями.
Я бы даже отодвинула эту границу оборотов на сорок --- шестьдесят, с учётом психологии этих людей --- они тяжеловаты на подъём.

Виа осторожно потянулась к плечевому импланту.

--- Ты не против?

--- Нет, конечно, --- я улыбнулся и сам сделал ей инъекцию успокоительного.
--- Тяжёлый день?

--- Если бы только день, --- Виа жалобно улыбнулась.

--- Я тебя понимаю.

--- Ты один из немногих, кто может понять.

--- А сколько времени потребовалось бы кораблю, чтобы достичь цели?

Виа почесала кожистую зелёную головку, раскрыла и сложила венчик.

--- Между Тра-Ренкхалем и Лотосом лежит пылевой <<блин>>, напрямую лететь они не могли.
Дело в том, что первые люди отправляли экспедиции не на конкретную планету, а на так называемое направление.
Какие-то из преобразованных планет на направлении могли быть непригодны для жизни, какие-то могли стать непригодными за время полёта, всякое происходило.
\ml{$0$}
{Иногда, уже прилетев на вполне годную планету, люди наотрез отказывались высаживаться, и были в своём праве.}
{People have sometimes reached a good planet and then refused to land on, and it was their right to do so.''}

\ml{$0$}
{--- И такое было?! --- удивился я.}
{``Has this really happened!'' I got surprised.}

\ml{$0$}
{--- Разумеется!}
{``Of course!}
\ml{$0$}
{Планета --- это не придорожная гостиница, а дом для ближайших ста поколений потомков!}
{Planet is not a roadside hotel, but home for the next hundred generations of descendants!}
Не все нюансы можно рассмотреть с расстояния в сто парсак, даже с помощью демиурга.
Поэтому, если маршрут предполагал, например, пять преобразованных планет, топлива брали на шесть циклов.

--- Чтобы был шанс вернуться, --- кивнул я.

--- Хотя бы маленький, но чтобы был, да.
Это очень важный момент, Аркадиу.
Первая экспедиция на направление была без страхового цикла.
Они успешно добрались до цели, но пригодной оказалась лишь последняя планета на маршруте.

--- Я понимаю.
Психологическое напряжение было гигантским.

--- Мягко сказано.
На корабле начался массовый психоз, закончившийся бунтом.
Десятую часть команды пришлось изолировать до конца полёта, многие из них погибли.
Так что сам факт, что экспедиция оказалась так далеко от Лотоса, за пылевым облаком, говорит именно об этом --- Тра-Ренкхаль был не первой и даже не третьей планетой, которую они посетили.
Дай-ка компьютер, я попробую вычислить маршрут.

Я уступил Виа место.
Плант прищурилась, и в её красивых чёрных глазах заиграли голубые огни голографического дисплея.

--- В архивах точно нет информации о времени отлёта?

--- Нет, --- покачала головой Виа.
--- Я бы и сама была рада, если бы подобная информация сохранилась.
Вот, смотри.
Кратчайшее время полёта --- триста восемьдесят семь оборотов.
Это на максимальной скорости.
Если предположить, что по пути они останавливались для починки, дозаправки или ради научного интереса --- выйдет все шестьсот.

Я кивнул.
Виа посмотрела на меня.
Страха в её глазах больше не было --- только любопытство.

--- Я, конечно, понимаю, что это секрет, но всё же --- зачем тебе это?
Ты думаешь, что вилет'марр были где-то ещё?

--- Я \emph{знаю}, что они были где-то ещё, --- сказал я.
--- Если твои данные верны, то вилет'марр двигались со сверхсветовой скоростью.
Точнее, в две световых.
И это при условии, что они, как ты говорила, никуда не заворачивали.
В чём лично я сомневаюсь.

Пластик под пальцами Виа жалобно хрустнул.
Она начала с бешеной скоростью открывать какие-то документы и просматривать их.

--- Да не может этого быть, --- шептала она.
--- Вот это --- данные Ясмер.
А это я лично вычисляла, и команда Флавика сказала, что всё верно.
Все данные, чтоб мне сдохнуть, верифицированы, и не на один раз.

Я промолчал.
Виа перестала мучить компьютер и откинулась в кресле.

--- Ну не могли же они использовать омега-телепортацию! --- жалобно воскликнула она.
--- Перенос сознания при омега-телепортации невозможен, даже с синхронизацией мозг-демон-мозг.
На расстояние вытянутой руки через квантовую запутанность --- куда ни шло, но всё, что дальше, равносильно самоубийству.
Запрет на ОТ шёл чуть ли не со времён ранней Эпохи Богов!
Зря, что ли, первые люди преодолевали космос на субсветовых?

--- Успокойся для начала, --- бросил я.
--- Гимел тоже мог ошибиться.
Верификацию, насколько я понял, прошли только используемые им данные анализов.
Алгоритм, согласно которому он проводил вычисления, мне не известен.
Но меня беспокоят тси.
Случайно ли то, что некогда обладавщие высокими технологиями вилет'марр, а затем и тси в итоге оказались на одной планете?
Что, если эти две цивилизации поддерживали связь, и тси прилетели за помощью?

Виа смотрела на меня.
Я знал, что её демон уже нарисовал нужную мне картину.

--- Это значит...

--- Это значит, что существование мультипланетных союзов сапиентов возможно не только в Империи Плеяды, --- закончил я.
--- Разумеется, в наше время это больше любопытный факт, чем что-то полезное.
Сложно допустить мысль, что на территории фракций может возникнуть законспирированная сапиентная организация, верно?

--- Я не думаю, что контрразведка допустит подобное, --- как-то неуверенно согласилась Виа.

--- Так или иначе, история тёмная и требует анализа, --- заключил я.
--- Я отправлюсь на Лотос и попробую что-нибудь узнать об экспедиции.
Я почти уверен, что наши демоны просмотрели очень важные детали.

Виа неловко улыбнулась.

--- Желаю удачи.
Ты знаешь, что искать?

--- Да, --- ответил я и пошёл к выходу.
--- В памяти Гимела были ещё кое-какие зацепки.

Виа махнула мне рукой.
Уже у порога я услышал за спиной тихий-тихий вздох облегчения.

<<Всё-таки неплохо играть на пограничных значениях.
Сместил среднее, увеличил точность --- формально никаких противоречий, а картина кардинально поменялась.
Надеюсь, ты достигла пристанища духов, Айну.
Ты была замечательным учителем>>.

... Дверь лифта закрылась, но никто так и не попытался меня пленить.
Я почувствовал, как мои волосы нежно ласкает ветер свободы --- давно забытый и такой желанный.

\chapter{[:] Другая форма жизни}

\section{[:] План раскрыт}

--- Ты собираешься оцифровать Тахиро.

Грисвольд вздрогнул и выронил ключ.

--- Откуда ты это узнал?

--- Я стратег, Грис.
Моя задача --- распознавать схемы и структуры в чужих действиях.
И задача эта --- не только моя.

--- Кто ещё знает?

--- Благодаря Гало --- весь штаб Ордена.
Как ты знаешь, решение об оцифровке дзайку-мару ещё не было принято, и Гало, который всецело <<против>>, чрезвычайно заинтересовался совершенно отчётливыми приготовлениями к этому событию.
Я бы на твоём месте поспешил на заседание, которое проходит прямо сейчас.

\ml{$0$}
{--- Они начали его без меня?!}
{``Did they start without me!''}

--- Ты же занят сборкой транспорта, верно?

Грисвольд судорожно выдохнул.

\ml{$0$}
{--- Ты бы хоть меня предупредил.}
{``You might have warned me.}
\ml{$0$}
{Я едва успел убедить Гало, что ответственность за всё лежит на мне.}
{I barely had time to convince Halo that the responsibility is all mine.}
\ml{$0$}
{Наплёл ему красивую историю про то, как хотелось бы мне сделать демоном своего человеческого друга.}
{I told him a lovely story: how I wish I could turn my human friend into a daemon.''}

\ml{$0$}
{--- Ты подставил Тахиро?}
{``You framed Tahiro?''}

\ml{$0$}
{--- А что мне ещё оставалось сказать, Грис?}
{``So what could I say, Gris?}
\ml{$0$}
{Давай, придумай что-нибудь более правдоподобное, ты же у нас гений конспирации.}
{Come on, think of something more believable, you tradecraft genius.}
\ml{$0$}
{Меня временно отстранили от дел за самоуправство, Гало меня ненавидит, а Тахиро остаток жизни будет спать со мной в одной комнате и в одной кровати, потому что это единственное место, где у него есть шанс проснуться.}
{I was suspended because of excess of authority, Halo hates me, and Tahiro will sleep in my room and in my bed the rest of his life, because it's the only place where he has chance to wake up.''}

\ml{$0$}
{--- Ч-что?}
{``W-what?''}

\ml{$0$}
{--- Грис, я сохранил твою репутацию, но его я больше не смогу защищать.}
{``Gris, I saved your reputation, but I can't save him anymore.}
\ml{$0$}
{Свод законов Ордена никак не защищает людей.}
{The Order law does not protect humans.}
\ml{$0$}
{Люди --- это скот, чья участь --- дойка, случка и забой.}
{Humans are cattle destined for milking, breeding, and slaughter.}
\ml{$0$}
{Тахиро жив по той причине, что он --- моя собственность, не представляющая опасности.}
{Tahiro is alive because he's my property, presenting no danger.}
\ml{$0$}
{Став демоном, Тахиро из умной игрушки превратится в сильного игрока.}
{After daemonization Tahiro goes from smart toy to strong player.}
\ml{$0$}
{Его прирежут превентивно, лишь бы этого не случилось.}
{They will cut his throat preventively to avoid that.''}

\ml{$0$}
{--- Что мне делать?}
{``What must I do?''}

\ml{$0$}
{--- Ты должен признаться во всём и хотя бы попробовать отстоять свою точку зрения.}
{``You must confess all, then at least try to prove a point.}
\ml{$0$}
{Идею оцифровать именно Тахиро, равно как и карт-бланш на все действия, можешь повесить на меня.}
{You can say I'm the one who gave you the idea to digitize Tahiro, along with carte blanche to do all you did.}
\ml{$0$}
{Ваша с Тахиро дружба должна оставаться в тайне --- то, что позволительно чудаку Люциферу, тебе не простят.}
{Your friendship with Tahiro should remain a secret---they won't forgive you for things that are permissible for freak Lucifer.}
\ml{$0$}
{Жизнь мальчика в твоих руках.}
{The boy's life is in your hands now.}
\ml{$0$}
{Единственный способ его спасти --- сделать субъектом адского права.}
{Turning him into a subject of the Hell law is the only way to save him.''}

Грисвольд сорвался с места с неподобающей толстяку прытью и понёсся к выходу из ангара.

--- Лу, прошу тебя, спрячь хотя бы вон те два модуля!
Диапазон моих личных ячеек --- десять-четырнадцать!
Только роботов не используй, они логи хранят!

--- А как тогда?..

--- Тележка!
Гидравлическая тележка в углу!
Перекати механизмы на палету и подцепи тележкой!

Люцифер с грустью осмотрел два тяжеленных агрегата, затем бросил взгляд на свои холёные пальцы --- и испустил тяжкий вздох.

--- Вот какого дьявола я опять должен таскать тяжести? --- обратился он к геликоптеру.
Геликоптер сочувственно промолчал.

\section{[:] Совещание}

\textspace

Арракис потребовал, чтобы Грисвольд объяснил свои действия.
Фраза была, разумеется, формальной --- по сути Грисвольду предстоял завуалированный публичный допрос.
Согласно правилам отдела пропаганды, подобная формулировка должна была употребляться по отношению к высокоранговым демонам, чтобы не нарушать субординацию;
однако по сути <<объяснения>> могли потребовать у любого, чьи действия стояли на самой грани закона, точнее, интересов руководства.
<<Объяснение действий>> подразумевало вопросы произвольного демона из исследовательских отделов под неусыпным контролем военных стратегов.

По странному стечению обстоятельств, произвольным демоном оказался один из учёных бывшего Ордена Тысячи Башен, Север Солнечная Дева.
Он сразу обращал на себя внимание грамотным построением речи на сохтид, тихим робким голосом и удивительно естественной мимикой.
У него единственного из всех присутствующих были очки с простыми стёклами и аккуратная подстриженная борода.
Согласно досье, Север по праву считался одним из флагманов изучения людей Тысячи Башен и предпочитал, чтобы его называли <<человеческим>> именем --- Сальватор Сольмё.

Разумеется, Север был проинструктирован заранее, как следует вести беседу.
Но по какой-то причине --- скорее всего, от волнения --- Север перевёл разговор на более привычные для учёного технические подробности.

--- Грисвольд, объясните всем, пожалуйста, в чём сложность демонизации людей.

--- В демонизации как таковой сложности нет, это стандартный процесс снятия квантовой структуры объекта с последующей записью в омега-модуль.

--- Какой именно объект подвергается оцифровке?

--- Нервная система вплоть до чувствительных окончаний, нервно-мышечных синапсов и некоторых нейрогуморальных структур.

--- Почему необходимо снятие отпечатка всей нервной системы?

--- В модуляции сигналов участвуют все указанные звенья.
Головной мозг, даже вместе со спинным, не самодостаточны.
Разработанный нами интерфейс монтируется к цифровым аналогам нервно-мышечных синапсов и чувствительных окончаний.

--- Я прошу прощения, коллега, но известны случаи, когда люди управляли машинами с помощью нейроинтерфейса, подключённого напрямую к коре, --- Север вопросительно посмотрел на Грисвольда поверх очков.

--- Сравнение некорректное, Сальватор, --- поклонился Грисвольд.
--- Речь шла о подключении устройства к живому организму, а не о полной замене живого организма квантовой моделью.
В упомянутом вами нейроинтерфейсе сигнал с коры подвергается сложному процессу стабилизации и модуляции сигнала, как это происходит в нижних отделах нервной системы.

--- Благодарю за пояснение.
Каким образом обеспечивается функционирование квантовой модели?

--- Модель помещается в среду, поддерживающую виртуальный клеточный метаболизм.
В будущем можно преобразовать квантовую модель мозга в самодостаточную математическую модель.
Если мы получим некоторое количество полноценных оцифрованных людей, мы сможем постепенно оптимизировать части виртуального мозга.
Я не думаю, что возможна замена всех отделов, но даже оптимизация ствола мозга даст огромный прирост производительности.

--- Прошу прощения, вынужден прервать, --- поднял руку Самаэл.
--- Вы чересчур углубляетесь в технические подробности и...

--- Извините, это чрезвычайно интересный момент, --- перебил Север.
--- То есть вы хотите сказать, что к модели нервной системы можно пристыковать вычислительные модули, модули чувствительности и устойчивости?

--- Я переформулирую ваш вопрос и отвечу на него: да, квантовая модель мозга человека совершенно точно может взять на себя роль программного ядра омега-сингулярности, со всеми вытекающими последствиями.

Присутствующие зашептались.

<<Самаэл очень не хотел, чтобы разговор дошёл до этой фразы, --- ухмыльнулся Грисвольд.
--- Пусть они это проглотят>>.

--- Надеюсь, вы не думаете, что это ядро сравнимо по эффективности с ядрами демонов? --- скептически осведомился Гало.

--- На данном этапе сравнивать довольно сложно.
Безусловные преимущества оцифрованного мозга --- дешевизна, уникальность строения, сложность прямого взлома.
Недостатки --- сложность настройки и гораздо меньшая пластичность.
Проще говоря, это дешёвый одноразовый аналог демона --- с неплохим запасом устойчивости и интеллекта, но с низким потолком чувствительности и проигрывающий в быстродействии.

--- Проигрывающий на два порядка, --- присовокупил Гало.

--- Два порядка? --- удивился Север.
--- Это очень много, Грисвольд, особенно применительно к военной сфере.

--- Тем не менее некоторые наши специалисты по вооружению считают, что с таким запасом устойчивости это простительный недостаток.
Кроме того, совершенно необязательно делать из оцифрованных людей солдат.
\ml{$0$}
{Как вы знаете, Сальватор, есть множество профессий, которые не требуют скорости.}
{You know, Salvator, there are many professions that don't need speed.}
\ml{$0$}
{Я своё пузо отрастил не просто так.}
{My belly's grown too fat for a reason.''}

Шутка предназначалась Гало.
Но стратег даже не улыбнулся.
В его глазах светился гнев;
Грисвольд вдруг понял, насколько непривычно видеть эту эмоцию на копии кукольного личика Люцифера.
Зато Север, к удивлению Грисвольда, отреагировал на юмор вполне адекватно.

\ml{$0$}
{--- Я полагаю, у вас масса опыта, --- с улыбкой сказал он, поправляя очки.}
{``You've got tons of experience, I guess,'' he said with a smile and readjusted his glasses.}
--- А какова скорость естественного износа квантовой модели?

--- Примерно в пять раз медленнее таковой в живом человеке.
Согласно прогнозам, износ математической модели можно довести до шестидесятитысячекратного замедления.

--- То есть, Грисвольд, вы хотите сказать, что максимальное время жизни оцифрованного человека приблизится к времени жизни демона?

--- Именно, Сальватор.

--- Это абсурд! --- возмущённо выкрикнул кто-то из учёных.
--- Люди живут не более восьмидесяти оборотов, их интеллектуальный потенциал охватывает вчетверо меньший срок и является достаточным исключительно для того, чтобы особь выжила и оставила потомство.
А вы хотите дать им в руки вечность и сложнейшие модули, в равной степени недоступные для их понимания!

\ml{$0$}
{--- Единственное, что хочу я --- это использовать возможность и исследовать объект, насколько это возможно, --- с достоинством ответил Грисвольд.}
{``The only thing I want is to use the opportunity, as well as possible, and research the object, as well as possible,'' Griswold said with dignity.}
--- Я не являюсь учёным, но не думаю, что это можно ставить мне в вину.
Если кто-то сомневается в наших данных и нашей методологии, вы всегда можете провести соответствующие исследования самостоятельно.

--- Полагаю, я буду первым, кто этим займётся, --- кивнул Север.
--- Какие задачи стоят перед вами сейчас?

--- На данный момент единственная большая задача --- отработка методики.
Многое зависит от особенностей конкретного человека, его способностей и психологических параметров.
Из некоторых получаются живые, но абсолютно неработоспособные демоны с узким каналом ввода-вывода.
У других канал сохранен, но они надламываются психологически или сходят с ума.

--- Можно ли решить это тренировкой?

--- Разница статистически незначимая.
У некоторых исследователей создалось впечатление, что значение более имеет мотивация, нежели тренировка.
Наши подопытные, разумеется, шли на процесс под угрозами, шантажом и обманом.

--- Каков процент полноценных выживших?

--- Два процента.

--- Есть ли какая-то разница по возрасту и полу?

--- Выживаемость детей выше в десять-пятнадцать раз, но после оцифровки они останавливаются в развитии.
Вероятно, это можно решить, но не на данном этапе исследований.
Мужчины дают полноценных демонов чаще женщин, при этом так же чаще женщин сходят с ума.
Разница между представителями разных этносов статистически незначима.

--- Я рискну предположить: не связана ли выживаемость детей с мотивацией?

--- Возможно, Сальватор.
Мы рассказывали им сказку о том, что они станут героями после процедуры.
Многие сами ложились в капсулу и терпеливо переносили неприятные ощущения.

--- Ближе к делу, Север, --- напомнил Самаэл.

--- Прошу прощения, --- кивнул учёный.
--- Итак, Грисвольд, кхм, вы полагаете, что человек по имени Тахиро та Ханаяма --- в некотором роде идеальный объект для следующей стадии экспериментов?

--- Он приближен к идеалу настолько, насколько это возможно.
Его мотивация не вызывает сомнений.
Он успешно интегрировался в наше общество и даже выполнял задания, соответствующие рангу второго центуриона.

--- Он не мог <<успешно интегрироваться>> в наше общество по чисто техническим причинам, --- заявил Гало.
--- Он не воспринимает омега-поле, он не может работать с нашими устройствами и...

--- Как раз-таки с устройствами Тахиро работает превосходно! --- внушительным голосом перебил Грисвольд.
--- Это может стать прецедентом --- будущих демонов можно серийно выращивать из сапиентного материала, избегая таким образом <<проблему копий>>.

По залу неожиданно пробежал шёпот;
в нём читалась смесь надежды и ужаса.

\ml{$0$}
{--- Он серьёзно?}
{``Seriously?}
\ml{$0$}
{Проблема копий может быть решена?}
{Problem of copies can be solved?''}

\ml{$0$}
{--- Он серьёзно?}
{``Seriously?}
\ml{$0$}
{Запустить серийную оцифровку этих животных?}
{He's offering serial digitization of these beasts?''}

--- У меня просьба к руководству Ордена, --- Север повернулся к Арракису.
--- Я, если честно, впервые услышал об этом исследовании только здесь.
Люди --- это моя жизнь.
Мне бы очень хотелось присоединиться.
Я знаю о своём нынешнем статусе, поэтому прошу в присутствии всех --- сделайте мне одолжение.

--- Если исследование решено будет возобновить, не вижу причин вам отказывать, --- кивнул Самаэл.
--- Подавайте заявление в соответствии с протоколом.
Ещё вопросы будут?

--- У меня вопросов больше нет, --- поклонился Север и сделал шаг назад.

--- Всё это очень трогательно, --- сказал Гало, --- однако я обращаю внимание присутствующих на то, что люди --- враждебный вид.
Если всё сказанное Грисвольдом --- правда хотя бы на одну десятую, серийная оцифровка может привести к масштабным и непредсказуемым последствиям вплоть до военного столкновения.

--- Вы позволите, я отвечу? --- улыбнулся Грисвольду Север.

--- Да, конечно, --- кивнул технолог.

<<Благослови тебя небо, Север.
\ml{$0$}
{Ты сам не понимаешь, как ты нам сейчас помог своей святой наивностью>>.}
{You have no idea how that \textit{sancta simplicitas} of yours has helped.''}

--- Когда ещё я служил в Ордене, мы...

--- В каком Ордене? --- ледяным тоном перебил Самаэл.

--- В Ордене Тысячи Башен, --- поправился Север.
Его смуглое лицо залила краска.
--- Так вот, мы с коллегами ради интереса пытались обучить культуру крысиных нейронов имитировать выполнение произвольного императивного кода.

\ml{$0$}
{--- Очень интересно, --- буркнул Гало.}
{``Sounds intriguing,'' Halo grunted.}

--- Да, Гало, я с вами согласен, --- улыбнулся Север.
--- Результаты, которые мы получили, перекликаются с вашими результатами: биологическая нейросеть достаточно пластична, чтобы заменить собой участок императивного кода.
Коллега Грисвольд утверждает, что биологическая нейросеть может стать даже программным ядром демона, вопрос, как я понимаю, лишь в мощностях и обучающей выборке.
Рискну с ним согласиться.
Я думаю, после астрономического числа итераций различия оцифрованного человека и демона сгладятся до статистически незначимых...

--- Я не об этом, --- вежливую речь Севера Гало слушал, скрестив руки и отвернувшись к стене.

--- Прошу прощения, Гало, я начал чересчур издалека.
Я полагаю, если взять человека, изначально воспитанного в обществе...

--- Если вы имеете в виду Тахиро --- он не подходит, --- бросил Гало за спину.
--- Мальчишка полжизни провёл в деревне, и воспитывали его дзайку-мару.

--- Я прошу прощения, Гало, --- стушевался Север, --- я не совсем понял ваш термин.
Вы не могли бы дать определение?

Гало свирепо задышал, тиская пальцами кобуру.
На мгновение Грисвольд подумал, что стратег вытащит оружие и застрелит Севера.

--- Гало, подожди, --- Грисвольд предостерегающе махнул Северу.
--- Скажи прямо: ты сомневаешься в Тахиро?
Люцифер дал ему положительную характеристику.

--- Тахиро уже пытался вести подрывную деятельность в Ордене, --- заявил Гало.

--- Всё это отмечено в отчёте Люцифера.
Тахиро был ребёнком, потерявшим семью.
Импульсивность нормальна для человеческих детей.

--- Мои тесты ясно показывают, что он продолжает чувствовать связь со своим видом.
Где гарантии, что это не повторится, едва в распоряжении Тахиро окажутся методы и вычислительные мощности?

--- Я считаю, что этому человеку следует дать шанс, --- сказал технолог.
--- Мы можем, разумеется, начать искать возможных предателей.
Вычислять, ставить опыты, проводить тесты, гадать на камнях.
Но единственный верный способ вычислить предателя --- это дать ему в руки оружие и встать с ним плечом к плечу.
И даже тогда можно ошибиться.

--- Я думаю, что Грисвольд прав, --- сказала Айну.
--- Гало и Люцифер, безусловно, мастера стратегии, но всё-таки теоретики и недооценивают значение опыта.
Тем не менее, несмотря на высказанное ранее мнение, что Тахиро уже принимал на себя полномочия второго центуриона, я категорически против присвоения ему этого ранга.
Пусть начнёт легионером, как и все демоны в Ордене.
Время покажет, чего он стоит.

--- Я категорически против, чтобы Тахиро был солдатом вообще, --- заявил Грисвольд.
--- Ты меня неправильно поняла.

--- Почему, Грис?

--- По уже озвученной Гало причине --- я против приобщения людей к военным технологиям.
Возможно, его захотят принять в какой-то исследовательский отдел?

--- Разве что как подопытный образец, --- буркнул Гало.
В толпе представителей раздался приглушённый смех.

--- Мне было бы интересно, --- улыбнулся Север, но улыбка тут же увяла под ледяным взглядом Гало.

--- К сожалению, вы не имеете права набирать собственный отдел, Север, --- сказал Самаолу.
--- Разумеется, если ваше рвение не угаснет, это временно.
Дождитесь решения ранговой комиссии.

--- Лично мне всё ясно, --- буркнул Арракис.
--- Подготовьте необходимое оборудование, этот человек будет оцифрован завтра и --- в случае успеха --- произведён в легионеры.
Заседание окончено, все свободны.

\section{[:] Нарэ}

<<Не самый плохой вариант>>, --- мысленно заключил Люцифер и углубился в созерцание карты стратегии.

\ml{$0$}
{--- Тебя отстранили от дел, --- напомнил Грисвольд, увидев Люцифера за обычной работой.}
{``You were suspended,'' Griswold reminded when he saw Lucifer doing his normal work.}

\ml{$0$}
{--- Я знаю.}
{``I know.''}

Грисвольд улыбнулся, неожиданно отвесил короткий поклон и вышел, аккуратно закрыв за собой дверь.
\ml{$0$}
{Он понял.}
{He realized.}

Карта Лу была простой и понятной.
Никаких путей отступления, никаких фронтов атаки.
Все вариации строились на том, что из каждого развития событий можно извлечь выгоду.
Пауку совершенно всё равно, в какую часть паутины попадёт муха, он в любой случае сможет запеленать добычу.

Лу очень любил пауков.
При встрече с золотопрядом изменяла даже его вечная брезгливость. 
Карта стратегии действительно напоминала искусно сплетённую паутинку;
Люцифер не удержался и нарисовал в углу терпеливо ждущего паучка.
Демон, помедлив, педантично подсветил его:
<<Паук (элемент художественного оформления, игнорируется)>>

Карта стратегии Гало была гораздо сложнее.
Неожиданные повороты, сложные многомерные пути развития событий.
На бумаге карта выглядела как детские каракули.
Гало рассматривает стратегию как алгоритм;
стратегия Лу похожа на систему маятников, тросов, пружин и весов, находящихся в состоянии динамического равновесия.

<<Ну вот, брат, у каждого из нас теперь своя карта стратегии, --- с грустью подумал Лу.
--- До чего мы дожили>>.

Север Солнечная Дева.
Сальватор Сольмё.
Какой очаровательный демон.
Его досье Люцифер читал с улыбкой, как добрую сказку.
Доверенный Айну уже перехватил агента Гало --- разумеется, по наводке Люцифера.
Агенту передали недвусмысленное сообщение: Севера не трогать, за любое деяние исполнитель ответит головой.
Гало не станет жертвовать агентами без очевидной выгоды.
И уж точно он не станет из-за Севера ссориться с Айну.

<<Можно было бы его завербовать, --- рассеянно думал Лу, --- но нет.
Это что-то за гранью добра и зла, мне даже думать об этом противно.
Пусть этот милаха занимается своими обожаемыми людьми и ни о чём не беспокоится.
Из игры, конечно, выйти не так-то просто --- Гало обязательно будет выяснять, почему Север так дорог Айну.
Но главное, что выяснять он это будет не у Севера>>.

Люцифер пометил Севера крестиком и кружком.
Линии карты стратегии заплясали и снова установились в равновесии.
Демон подсветил ещё что-то.
Люцифер, хихикнув, прочитал:
<<Клякса (случайно сгенерированный элемент взаимодействия с пишущим устройством, игнорируется)>>

Люцифер отложил карту стратегии в сторону, подошёл к окну и положил разгорячённые ладони на прохладное стекло.
Это был <<нарэ>> --- своеобразный минутный отдых от работы.
Качество стекла оставляло желать лучшего: волнистая поверхность, полупрозрачные белые пятна примесей, паутинка трещин по углам, словно сплетённая каким-то особым стеклянным пауком.
Однажды Лу застал Тахиро в такой же позе;
тот не отрываясь глядел на стекло.

--- Что ты рассматриваешь? --- спросил стратег.

--- Нарэ, --- ответил Тахиро.

--- Мне это слово не знакомо.

--- А я не смогу объяснить смысл.
Поэтому, если хочешь понять, то просто смотри.

Лу добросовестно ходил и смотрел каждые несколько дней.
Он вглядывался в стекло, он изучил каждую волну поверхности, каждый дымчатый извив в его толще, каждую трещинку и каждое пятно на поверхности.
Смысл слова так и остался неизвестным;
однако Лу заметил, что странный ритуал неплохо помогает приводить мысли в порядок.

В этот раз что-то было не так.
Стекло недружелюбно вибрировало, хотя сейсмостанция не обещала подземных толчков.

Лу едва успел убрать со стекла ладони, как оно разбилось вдребезги.

<<Да чтоб вас! --- подумал Лу, растерянно глядя на осколки и текущую по запястьям кровь. --- Ну и как я теперь найду это нарэ?>>

Почти сразу же взвыли сирены.

\section{[:] Броня}

\textspace

--- Последние диверсии совершили люди поселений.
На экстренном заседании выступал Самаэл.
В общем, он убедил штаб в том, что оцифровывать человека опасно, что мы можем сами дать людям в руки оружие, которым нас можно уничтожить.
Извини, Тахиро.
Решение окончательное.

--- Я вам верен, Грис!
Я ни за что не ударю в спину тебе или Лу!

\ml{$0$}
{--- Я верю тебе, мальчик.}
{``I believe you, Tahiro-kid.}
\ml{$0$}
{Но Арракиса ты в этом не убедишь.}
{But you will not persuade Arrakis.''}

\ml{$0$}
{--- Ему нельзя здесь оставаться, --- сказал Лу.}
{``He mustn't stay here,'' Lu said.}
\ml{$0$}
{--- Его нужно...}
{``He must be---''}

\ml{$0$}
{--- Мне тоже нельзя здесь оставаться, --- прервал его Грисвольд.}
{``I mustn't stay here neither,'' Griswold interrupted him.}

\ml{$0$}
{--- Ты о чём?}
{``What are you talking about?''}

--- Гало знает о твоей фальсификации.
Он намекнул, что может заявить об этом в любой момент, если я не окажу ему услугу.

--- Он тебя шантажировал? --- Лу выглядел ошарашенным.
\ml{$0$}
{--- Что он хотел?}
{``What are his demands?''}

\ml{$0$}
{--- Он хотел, чтобы я передал сообщение одному агенту.}
{``He wants me to deliver a message to an agent.}
\ml{$0$}
{Я согласился.}
{I agreed.''}

\ml{$0$}
{--- Покажи.}
{``Show me.''}

Лу и Грис соприкоснулись головами.

--- Он хочет скомпрометировать тебя связью с Союзом, --- сказал Лу.
Его трясло.
--- Брат, до чего ты дошёл?

\ml{$0$}
{--- Лу, что мне делать?}
{``Lu, what must I do?''}

Лу вскочил и сделал несколько кругов по комнате.

\ml{$0$}
{--- Нам срочно нужно оборудование для оцифровки.}
{``We need digitization equipment, now.}
\ml{$0$}
{Только так мы сможем спасти Тахиро и доказать свою невиновность.}
{It's the only way to save Tahiro and prove our innocence.}
\ml{$0$}
{Тахиро, иди в мою комнату и никуда не выходи, пока я тебя не позову.}
{Tahiro, go to my room and stay there till I call you.}
\ml{$0$}
{Грис, твои железяки спрятаны в ячейке номер тринадцать в третьем ангаре.}
{Gris, your pieces of iron are awaiting you in the vault number thirteen, hangar three.}
\ml{$0$}
{Сообщение агенту я передам сам.}
{I'll deliver the message by myself.}
\ml{$0$}
{Надо узнать, насколько далеко Гало зашёл в желании нас скомпрометировать.}
{We must find out how far Halo's gone to compromise us.''}

\ml{$0$}
{--- Лу, ты подставляешь сам себя!}
{``Lu, you frame yourself!}
\ml{$0$}
{В произошедшем виноват я и только я.}
{Things that happened are my fault and mine alone.}
\ml{$0$}
{Ты ещё можешь...}
{You can still---''}

\ml{$0$}
{--- Я слишком долго вас искал, Грис.}
{``I spent too long looking for both of you, Gris.}
\ml{$0$}
{Я не хочу вас потерять.}
{I can't lose you.''}

Лу подхватил со стойки бронежилет, накинул плащ и бегом бросился за дверь.

\section{[:] Черепаха}

\textspace

Лу очень любил деревню.
Там жили самые разные люди --- взрослые, дети, старики, мужчины, женщины и те, кто не испытывал надобность принадлежать к какому-либо полу, люди разных профессий и разного нрава.

Стратег с детства бегал в деревню играть с другими детьми.
Он выпрашивал у женщин пирожки, ловил рыбу и дрался, как и прочие.
Каждый раз, когда его ловили за этим занятием, следовало наказание, о котором Лу, при всей лёгкости его нрава, очень не любил вспоминать.
Грисвольд однажды спросил, как же наказывал Лу Арракис.
Люцифер бойко отшутился, но его руки вдруг задрожали, словно стратегу внезапно стало холодно.
Грисвольд понял и с тех пор больше никогда не поднимал эту тему.

--- Ты --- будущий правитель, --- говорил Арракис после каждого наказания.
\ml{$0$}
{--- Ты должен вести себя соответствующе.}
{``You must behave accordingly.''}

--- Я --- будущий правитель, --- парировал Лу со слезами в голосе.
\ml{$0$}
{--- Я должен знать всё о тех, кем мне предстоит управлять --- их силы, слабости и нужды!}
{``I must know all about my subjects---their strength, weakness and needs!''}

--- А Гало? --- спросил Грисвольд.

--- Гало был его любимцем, --- Люцифер усмехнулся и зябко поёжился.
\ml{$0$}
{--- Его просто заставляли драться с солдатами до изнеможения.}
{``He was forced to fight the soldiers until he falls exhausted.}
\ml{$0$}
{По сравнению с моим это даже наказанием назвать нельзя.}
{Compared to mine, it wasn't a punishment at all.''}

\ml{$0$}
{--- Он знал, как обращаются с тобой?}
{``Did he know how you're being treated?''}

\ml{$0$}
{--- Он всегда принимал наказания как должное, --- уклончиво ответил Лу.}
{``He always took punishment for granted,'' Lu vaguely said.}

\ml{$0$}
{--- Он знал? --- настаивал Грис.}
{``Did he know?'' Gris insisted.}

Лу красноречиво посмотрел на друга и впервые на памяти Грисвольда промолчал.

Мимо прошла милая пара с зонтиками.
Девушки в выходных кимоно кокетливо семенили гэта, держась за ручки и время от времени обмениваясь робкими поцелуями.
Полуголые рыбаки, с весёлыми песнями снимающие гётаку\FM{} с огромной шипастой рыбы, помахали девушкам вслед бумажными полосками.
\FA{
Гётаку --- бумажный отпечаток вымазанной чернилами рыбы, который оставляли на память о трофее или использовали для отчётности.
}
Пожилой монах рассказывал зачарованно притихшим ребятишкам какую-то историю.

Деревня выглядела так, словно она сошла с праздничной открытки.
Она была совсем не той, которой стратег помнил её с детства.

Лу подумал --- и свернул с намеченного маршрута.

Почти сразу он понял, что сделал всё правильно.
На него стали неприкрыто пялиться, из углов понёсся недружелюбный шёпот.
Вечерняя деревня стала такой, какой ей надлежало быть --- сумеречной и настороженной.

Поворот, улица, поворот, улица, поворот, узкий проулок, ещё два поворота...

У одного из домов несколько женщин гнали спиртное.
Старушка, похожая на одетую в кимоно черепашку, проводила Лу взглядом.
Её примеру последовала вторая, вытягивая морщинистую черепашью шею.
Едва Лу скрылся за поворотом, как обе тут же занялись своими делами.

Впрочем, запах не был похож на спирт.
Лу вдруг замер столбом.

\ml{$0$}
{<<Жидкое взрывчатое вещество>>.}
{\emph{Liquid explosive.}}

Стратег попятился и осторожно заглянул за угол.
В нос ему уткнулся клинок.
Ещё пара осторожно залезла под нижний край бронежилета.

\ml{$0$}
{--- Что ты здесь вынюхиваешь, хорохито? --- осведомился низкорослый человек.}
{``What are you sniffing around, horohito?'' a short man asked.}

--- Авамори\FM\ уж больно хорош, --- попытался пошутить Лу.
\FA{
Авамори --- крепкий спиртной напиток Преисподней, около 56 градусов.
}

Он ожидал удара или выстрела, но клинки неожиданно отодвинулись.
Окружившие его люди зашептались.

\ml{$0$}
{--- Он сейчас что, пошутил?}
{``What, did that man just make a joke?''}

\ml{$0$}
{--- Это была шутка?}
{``Is that a joke?''}

\ml{$0$}
{--- Его подкупили.}
{``He was paid off.''}

\ml{$0$}
{--- Да нет, это один из этих, которых они выкормили, как поросят!}
{``You're wrong, it's a piglet \emph{they} have nursed!''}

Кто-то сорвал с Лу капюшон.

\ml{$0$}
{--- Это не тот!}
{``That's not the right one!}
\ml{$0$}
{Он не толстый и не лысый!}
{He is not fat, nor bald!''}

\ml{$0$}
{--- Гляньте, он глаза красит и волосы!}
{``Look, folks, he paints his eyes and hair!''}

\ml{$0$}
{--- Похож на актёра.}
{``Looks like a stage actor.''}

\ml{$0$}
{--- Да какой актёр?}
{``Nonsence!}
\ml{$0$}
{Он в боевой броне!}
{He wears a real armor!''}

\ml{$0$}
{--- У него трубка в кармане!}
{``I found a smoking pipe in his pocket!}
Пахнет свежим вереском!
\ml{$0$}
{А в ножнах --- ножницы!}
{And a scissors in his scabbard!''}

\ml{$0$}
{--- Ножницы в ножнах?}
{``A scissors in the scabbard?}
\ml{$0$}
{Что за чушь я услышал?}
{What the heck I just heard?''}

--- Да это точно актёр!

Толпа зашепталась ещё тревожнее.

--- Так ты не хорохито? --- наконец спросил его кто-то.

\ml{$0$}
{--- Вы мне всё равно не поверите, --- ответил Лу, осторожно поглядывая по сторонам.}
{``You won't believe me anyway,'' Lu answered trying to look around.}
\ml{$0$}
{--- Поэтому объясните для начала, почему вы спутали меня с хорохито.}
{``So explain first, what made you mistake me for horohito.}
\ml{$0$}
{Попутно объясните, почему вы вдруг засомневались.}
{You may explain, by the way, what made you question that after.''}

\ml{$0$}
{--- Вопросы здесь задаём мы, --- заявил человек, говоривший первым.}
{``That's we who ask questions,'' said the man who started talks.}

\ml{$0$}
{--- Отнюдь, --- лаконично возразил знакомый голос.}
{``Mistake,'' a familiar voice succinctly argued.}

Лу улыбнулся.
Тахиро был замечательным учеником.
Сказанное в общей модуляции слово хлестнуло по ушам людей, словно усаженный шипами бич.
Враги подняли руки с такой скоростью, словно от этого зависела жизнь всего их рода.
Клинки посыпались на утоптанную землю.

--- Сайгон, быстро в лагерь, --- распорядился Тахиро, щёлкнув предохранителем винтовки.
--- Грис ушёл в сторону третьего ангара.
Транспорт в квартале.
Я прикрою.

Лу растолкал ошарашенных дзайку-мару и бросился бежать, не дожидаясь пояснений.

\section{[:] Побег}

\textspace

--- Что ты здесь делаешь?

--- Тот же вопрос у меня к тебе, --- парировал Грисвольд.
Вначале он хотел ответить распространённой в человеческих поселениях похабной рифмовкой, но удержался, зная, что Мефистофель его не поймёт.
--- Арракис приказал мне обеспечить отсутствие прочих демонов здесь.
\ml{$0$}
{Уходи.}
{Go away.''}

--- Я получил от него похожий приказ, --- заявил Мефистофель.
\ml{$0$}
{--- Уйти придётся тебе.}
{``It's you who must leave.''}

Грисвольд впился прищуренным взглядом в Мефистофеля.
Тот смотрел на технолога спокойными пустоватыми буркалами.

\ml{$0$}
{--- Уйди, Меф.}
{``Go away, Mef.}
\ml{$0$}
{Говорю последний раз.}
{This is your last warning.''}

По лбу Грисвольда струился пот;
он прекрасно понимал --- истолкуй интерфектор его слова как реальную угрозу, и от омега-модуля толстяка не останется даже воспоминаний.
Но Мефистофель вдруг улыбнулся --- идеально ровной машинной улыбкой.

\ml{$0$}
{--- Ты пытаешься испугать меня.}
{``You're trying to scare me.}
\ml{$0$}
{Это признак слабой позиции.}
{It's a sign of weak position.}
\ml{$0$}
{Возвращайся к Арракису, уточни полученный тобой приказ и не трать моё время.}
{Return to Arrakis, request clarification on the order and stop wasting my time.''}

Мефистофель отвернулся --- ровно настолько, чтобы не видеть, как Грисвольд аккуратно взялся за браслет.
Вспышка --- и громоздкое, с туповатым лицом тело Мефа неуклюже село на пол.

<<Определённо удачное решение --- собрать боевые омега-модули в материальном устройстве, а не в демоне.
Но если браслет попадёт в чужие руки --- плакала моя головушка>>.

Грисвольд, демиург по воле судьбы и творец по призванию, ужасно не любил что-либо разрушать.
Насмотревшись всевозможных погребальных ритуалов у людей, он вдруг решил сказать пару слов над поверженным им противником.

--- Мефистофель из клана Мороза, ты был лучшим, --- начал он.
--- Ты был лучшим...

Пауза явно затянулась.

\ml{$0$}
{--- Понятия не имею, был ли ты хоть в чём-то и хоть для кого-то лучшим, --- растерянно закончил технолог и повернулся, чтобы уйти.}
{``I've no idea if you were the best at something or to someone,'' confused Griswold finished, then turned around to leave.}
Его встретил ошалелый взгляд Люцифера.

\ml{$0$}
{--- Зато честно, --- признал парень.}
{``You're honest, anyway,'' Lu declared.}
Разумеется, его демоническая сущность уже произвела подробный анализ и без того несложной ситуации.
--- За что ты его так, Грис?

--- Потом объясню, --- проворчал технолог.
--- Тахиро спит?

--- Нет, --- раздался голос из темноты.

--- Так и знал.
Быстро за мной, и желательно придержать вопросы до более подходящего момента.

\section{[:] Плач новорождённого}

Одной из самых остро стоящих проблем на Преисподней, вне зависимости от региона, является питьевая вода.
Добыча пресной воды и постройка водопроводов --- наукоёмкий процесс, тайны и нюансы которого люди Преисподней бережно передавали из поколения в поколение.
\ml{$0$}
{До тех пор, пока не пришёл Орден.}
{Until the Order came.}
\ml{$0$}
{Орден всегда вносил изменения в привычный уклад жизни.}
{The Order always modified dwellers' lifestyle.}

С акиямским водопроводом связана интересная история.
Раньше два соседних города на западном крае Такэсакского ущелья, Акияма и Киба, считались захолустьем из-за почти полного отсутствия питьевой воды.
Жители шли на ухищрения --- собирали дождевую воду, перегоняли испорченный вулканическим пеплом снег, искали под многометровыми серными ледниками чистые озерца и буквально руками прорубали к ним узкие тоннели.
В Акияма слыхом не слыхивали об открытых полях, культуры выращивались только в герметичных теплицах.
Всё изменилось благодаря легионеру Ордена, Тако из клана Дорге.
Несмотря на ранг и достаточно низкий интеллект, он был одержим инженерией.
Когда Тако вместе с его подразделением перебросили в Такэсако, он был поражён величием старинного такэсакского акведука, снабжавшего питьевой водой город с населением в сто тысяч человек.
В Акияма же его талант нашёл применение.

Арракис отказался тратить ресурсы на новый водопровод.
Тогда Тако сделал то, чего делать было нельзя --- обратился к дзайку-мару.

Чертежи Тако были многократно скопированы и спрятаны у самых разных людей, и изъять все так и не удалось.
Женщины обжигали кирпичи во время каждого приготовления пищи.
Цемент носили по горсти в каждом кармане, камни --- в каждом дорожном мешке.
Строительство велось ночами, даже в праздничные дни.
Не помогали ни показательные казни, ни шантаж, ни саботаж.
Агенты поставляли строителям уровни, откалиброванные по разным стандартам --- дзайку-мару разоблачили подделку и стали самостоятельно настраивать инструмент.
Диверсанты три раза подменяли обычные кирпичи на наполненные тротилом --- дзайку-мару с упорством муравьёв, не жалуясь и не поднимая шума, отстраивали развязку вновь.
Тела строителей, пойманных на месте преступления, снимали, а затем с потрясающим, свойственным только жителям Преисподней цинизмом подбрасывали к домам агентов --- иногда с вырезанными на коже оскорбительными иероглифами.

Строительство было завершено за тридцать лет.
По самым скромным подсчётам, акведук стоил жизни четырём тысячам мужчин и пятиста женщинам --- половине населения Акияма. 
Первая чашка чистой воды в Киба была выпита за год до рождения Люцифера.
Ещё десять лет спустя акведук обеспечил драгоценной влагой всю западную часть ущелья Такэсако, и о <<годах жажды>> вспоминали лишь старики.

Разумеется, строительство не закончилось бы никогда, если бы Орден не перестал преследовать жителей.
На десятый год борьбы Дорге сумел убедить Арракиса остановиться.

--- Жаждущего на пути к роднику можно остановить только пулей, --- сказал он.
--- Их неповиновение --- вынужденная мера, а не прихоть оппозиционно настроенных слоёв населения.
Это очевидно из потерь, которые они несут.
В общем и целом водопровод не представляет непосредственную угрозу для Ордена.
А вот демонами, которые позволили себе подобные выходки, следует серьёзно заняться.

Тако всё-таки успел уйти на Тысячу Башен вместе со своим подельником, где и прослужил без особых происшествий до недавнего времени.
\ml{$0$}
{После присоединения Тысячи Башен к Ордену Преисподней Тако пропал бесследно.}
{After Thousand Towers was annexed by the Order of Nether, Tako has just vanished.}
\ml{$0$}
{Едва ли удача улыбнулась легионеру дважды;}
{He unlikely got lucky twice;}
\ml{$0$}
{Люцифер склонялся к мысли, что пропасть ему помогли.}
{Lucifer guessed that the legionnaire was helped to disappear.}
\ml{$0$}
{Орден ничего не прощал и не забывал.}
{The Order forgets nothing and forgives nothing.}

\ml{$0$}
{Не забыли и люди.}
{Humans forgot nothing, too.}
\ml{$0$}
{Доставшийся такой ценой акведук уважительно называют Тако-сан.}
{Bought at such a price, the aqueduct is respectfully called `Tako-san'.}

\ml{$0$}
{--- Так закончилась история и началась легенда, --- закончил свой рассказ Люцифер.}
{``So the story ends, and the legend begins,'' Lucifer finished.}
В его красивых глазах отражались искры костерка.
\ml{$0$}
{--- Заставляет задуматься.}
{``Makes you wonder.}
\ml{$0$}
{Мы живём так долго, мы дослуживаемся до званий --- но что нужно сделать, чтобы нас пережили наши собственные имена?}
{We live so long, we reach ranks, but what should we do to make our names outlive us?''}

\ml{$0$}
{--- Пойти против системы, --- предположил Тахиро.}
{``Buck the system,'' Tahiro guessed.}

\ml{$0$}
{--- Отнестись с уважением к тем, с кем никто не считается, --- пожал плечами Грис.}
{``Show respect to people no one has to reckon with,'' Gris shrugged.}

\ml{$0$}
{--- Или просто делать то, к чему лежит твоё сердце, --- заключил Люцифер.}
{``Or just do the job you have your heart set on,'' Lucifer ended.}

Друзья помолчали.

--- Я ни разу не слышал эту историю, --- сказал Грисвольд, оглядывая величественные извивы водопроводной развязки, похожие на щупальца гигантского осьминога.
\ml{$0$}
{--- Как жаль, что я не был знаком с Тако.}
{``Such a pity I never met Tako.}
\ml{$0$}
{Его можно было взять в нашу команду.}
{He would be good for the crew.''}

\ml{$0$}
{--- Я тоже об этом думал, --- не без сожаления признал Люцифер.}
{``I was thinking the same thing,'' Lucifer admitted, not without regret.}
\ml{$0$}
{--- Отец порой ужасно расточителен.}
{``Father is terribly wasteful sometimes.''}

--- И всё-таки, почему, почему люди продолжали строительство, несмотря на такие потери?

--- Посмотри на этот город, --- сказал Люцифер.
--- Его жители проводят жизнь в попытках раздобыть немного пищи и перегоняют собственные выделения, чтобы не страдать от жажды.
Беспросветная серая мгла выживания, от рождения до смерти.
И вдруг приходит кто-то, неважно кто, и дарит людям мечту о лучшей жизни.
Даже о кусочке лучшей жизни --- о чашке чистой воды.
Что бы сделал ты?
А им и подавно нечего было терять.

--- Я бы не стал бороться за воду, --- заявил Тахиро.

--- А за что, по-твоему, ты борешься сейчас? --- возразил Грисвольд.
--- Уважение, личная неприкосновенность, свобода.
Это такие же банальные вещи, как и чистая вода.

--- Вы странные, --- сказал Тахиро.
--- У меня ощущение, что вы совсем-совсем не отличаетесь от нас.
Но вас как будто постоянно что-то гнетёт.
Когда мы пробирались сюда через скалы, я думал, что нет участи хуже моей: позади --- смерть, впереди --- мучения и, возможно, тоже смерть.
Но для меня это лишь плохой день, всего один, хоть и последний, а для вас --- обыденность.

Лу и Грис с изумлением уставились на парня.
Однако тот занялся костром и не счёл нужным объяснять свои слова.

\asterism

Неподалёку бил гейзер и парили несколько источников.
В некоторых вода была похожа на жидкий огонь, в других она почему-то была температуры растаявшего снега.
Друзья всё-таки нашли один с приемлемой температурой по лягушачьим крикам.
Лу пару часов пролежал с лягушками, отказываясь вылезать.
Его не смутил даже настойчиво щёлкающий дозиметр, который Грисвольд красноречиво сунул ему под нос.

\ml{$0$}
{--- Я всё равно не собираюсь размножаться, --- буркнул стратег.}
{``I'm not gonna breed, anyway,'' the strategist grunted.}
\ml{$0$}
{--- Сто лет в ванне не лежал.}
{``Haven't been lying in a bath for ages.''}

\ml{$0$}
{--- Тебя рыбки сейчас съедят!}
{``You're about to be eaten by fishlings!''}

\ml{$0$}
{--- Это головастики, Грис.}
{``Tadpoles, Gris.}
\ml{$0$}
{Личиночная форма земноводных.}
{Amphibian larvae.}
Они меня немного щиплют, и всё.

Тахиро всё-таки вытянул Лу из источника и, не обращая внимания на протесты, плотно закатал стратега в несколько одеял.
Грисвольд притащил гигантскую хмурую гусеницу к костру, усадил на камень, вколол Лу в загривок радиопротектор и занялся сборкой аппаратуры.

--- Кстати, --- сказал Лу.
--- Я, кажется, понял, как люди отличали демонов.
Их очень сильно сбило с толку то, что я пошутил.

Грисвольд хлопнул себя по лбу.

--- Я ни разу не слышал, чтобы наши легионеры шутили или смеялись над шутками, --- сказал он.
--- Да и многие из высших чинов тоже.

--- Надо поработать над адаптационными обновлениями, --- задумчиво сказал Лу.
--- Займёшься этим, когда вернёмся, хорошо?
Я попрошу, чтобы тебя освободили от прочих заданий.
А отработку методики демонизации отдай Северу.
Он в технологии не особенно силён, но он очень хороший биоинформатик и, думаю, справится.
Сейчас работа как раз по его профилю.

--- Ранговая комиссия...

--- В бездну ранговую комиссию.
У себя в лаборатории ты --- ранговая комиссия.
Кого хочешь, того и ставь заместителем.

\ml{$0$}
{--- Ты уверен, что мы вернёмся? --- грустно спросил Грис.}
{``Are you sure we'll be back?'' Gris sadly asked.}

\ml{$0$}
{--- Оцифруем Тахиро и вернёмся.}
{``We'll digitize Tahiro and be back.}
\ml{$0$}
{Есть какие-то другие предложения?}
{Any other suggestions?''}

\ml{$0$}
{--- Нас не ждёт тёплый приём.}
{``We won't be welcome there.''}

\ml{$0$}
{--- Нас он нигде не ждёт.}
{``Like anywhere else.''}

--- А если они всё-таки захотят меня уничтожить? --- спросил Тахиро.

\ml{$0$}
{--- В этом случае пусть Орден больше не рассчитывает на сотрудничество со мной, --- буркнул Грисвольд.}
{``In that case, the Order shouldn't count on me anymore,'' Griswold grunted.}

Лу кивнул.

\ml{$0$}
{--- И со мной тоже.}
{``On me either.''}

Грисвольд бросил на стратега изумлённый взгляд.
Люцифер смутился.
Он понял, что после этих слов Грисвольд пойдёт за ним хоть в пламя вулкана.
Стратег спокойно относился к военным потерям, но ответственность за друга немного его испугала.

--- Кстати, сегодня мой день рождения, --- сообщил Тахиро.

--- Что? --- не понял Лу.

--- У нас есть традиция --- отмечать число и месяц рождения каждый год.
Маленький праздник для одного человека, его друзей и родных.

Люцифер окинул взглядом заброшенный лагерь и унылый пейзаж Акияма в долине.

--- Я не знал.
Сочувствую.

--- Каждый день рождения, который я помню, был таким же невесёлым, --- буркнул Тахиро.
--- Моих родителей освежевали в этот же день.
Видимо, впервые увидев свет, после первого вдоха человеку должно плакать, а не смеяться.

--- Зато ты родишься в этот день дважды, --- попытался приободрить его Лу.

--- Или проживу целое количество лет, --- закончил Тахиро.
--- Я ж не дурак, Лу.
Я знаю, на какой риск иду.

Лу помолчал и поковырял пальцем ноги землю.

--- Ты всё ещё сердишься на меня из-за родителей?

Тахиро удивлённо посмотрел на друга.

--- Нет, конечно.
Я перестал сердиться в тот момент, когда ты сказал, что не хотел их смерти.
\ml{$0$}
{Я понял, что ты сожалеешь.}
{I realized you're sorry.}
\ml{$0$}
{Но удивило меня то, что в твоём сожалении не было ничего морального или религиозного --- это было искреннее сожаление того, кто случайно наступил на красивый цветок.}
{But I was surprised there were nothing ethical or religious in your regret, you're sorry as if you accidentally stepped on a beautiful flower.}
\ml{$0$}
{Ты увидел в моих родичах то, чего не замечал даже я --- красоту.}
{You have seen in my relatives one thing even I failed to notice---beauty.''}

\ml{$0$}
{--- Я бы сказал --- жизнеспособность.}
{``Vitality, I'd say.''}

\ml{$0$}
{--- Это одно и то же.}
{``That's the same.}
И меня ты освободил не из-за каких-то принципов или догм --- тебе было противно смотреть на меня связанного, это оскорбляло твоё чувство прекрасного.

--- На трупы и больных до сих пор смотреть не могу, --- признался Лу.

Грисвольд отвлёкся от сборки агрегата и одобрительно посмотрел на него.

--- Однако же странно, что с таким мировоззрением ты не стал кем-то вроде отца, --- сказал технолог.
--- В геноциде обычно повинны такие вот ценители прекрасного.

--- Чувство прекрасного бывает разное, --- сказал Лу.
--- Лично мне трупы кажутся уродливее любого из живых.

Друзья расхохотались.
Впрочем, веселье быстро утихло.

--- А что делают для человека, у которого день рождения? --- спросил Лу.

--- Ему дарят подарок и кормят сладостями.
Но мне нельзя ничего есть.

Люцифер подумал и обратился к Грисвольду:

--- Ты приготовил ему модули?

\ml{$0$}
{--- Да, разумеется.}
{``Yes, of course.}
\ml{$0$}
{Интерфейс ввода-вывода и шаблоны модулей легионеров.}
{Input/output interface and legionnaire module templates.''}

\ml{$0$}
{--- Если всё получится, скопируй и пристыкуй ему мои.}
{``If all goes well, mount copies of my modules to him.}
\ml{$0$}
{Так, как они есть, с актуальными обновлениями.}
{As they are, with all updates.''}

\ml{$0$}
{--- Ты серьёзно?!}
{``Are you serious!''}

--- Ему нужен подарок, --- грустно улыбнулся Лу.
\ml{$0$}
{--- А модули --- это всё, что у меня есть.}
{``My modules are my only property.''}

Грисвольд выглядел ошарашенным.

--- Это же самые отработанные стратегические методики, большая часть накопленных человечеством...

\ml{$0$}
{--- Я знаю.}
{``I know.''}

\ml{$0$}
{--- Отец тебя!..}
{``Your father will ... !''}

\ml{$0$}
{--- Я знаю, Грис!}
{``I know, Gris!''}

--- Под твою ответственность, сайгон, --- развёл руками технолог.
--- Но всё-таки...

\ml{$0$}
{--- От меня не убудет.}
{``It costs me nothing.}
\ml{$0$}
{Модули --- это не яблоки.}
{A module is not an apple.''}

--- Спасибо, Лу, --- поклонился Тахиро.
\ml{$0$}
{--- Я не совсем понимаю, что именно ты хочешь мне подарить, но, думаю, это прекрасный подарок.}
{``I barely understand what exactly you want to give me, but I think it's a nice present.''}

Лу хитро осклабился:

\ml{$0$}
{--- Ты \emph{совсем не понимаешь}, что я хочу тебе подарить.}
{``You \emph{completely don't understand} what exactly I want to give you.}
\ml{$0$}
{Грис немного понимает, поэтому он и...}
{Gris does a bit, that's why he---''}

\ml{$0$}
{--- Ладно, хватит болтовни, --- оборвал друга Грис.}
{``Stop, done talking,'' Gris interrupted his friend.}
--- Тахиро, нужду справил?

--- Десять минут назад.
Не ел четырнадцать часов, как ты и хотел.
Выпил глоток подсоленной воды около часа назад.

--- Тогда можем приступать.

--- Так, выпутайте меня из этого кокона, я не хочу пропустить самое интересное, --- попросил Лу.

Тахиро аккуратно размотал одеяло, освободив друга.
Их глаза на мгновение встретились.
Во взгляде Люцифера застыло что-то совсем непонятное, и это не укрылось от Тахиро.

--- Всё хорошо? --- спросил он.

--- Да уж лучше, чем у тебя, --- отшутился Лу, похлопав друга по плечу.

Люцифер и сам не знал, что он чувствует.
Он всем сердцем хотел, чтобы Тахиро был равным ему.
И сейчас его желание сбывалось, но как-то совсем не так, как должно было.
%{His wish was about to come true, but not the way it was supposed to be.}

--- Анестезию будем делать? --- спросил стратег у Грисвольда.

--- Никакой анестезии, --- буркнул Грис.
\ml{$0$}
{--- Он должен быть в полном сознании, чтобы устройство захватило активность всех участков нервно-мышечной системы.}
{``He must be fully conscious, so the device is able to capture activity of the whole neuromuscular system.}
\ml{$0$}
{Если он потеряет сознание в процессе, придётся приводить его в чувство.}
{If he loose consciousness, he's to have been waken up.''}

Тахиро побледнел.

--- Ты готов? --- обратился Лу к другу.

Тахиро кивнул и, сбросив одежду, лёг на жёсткую станину.
Его била крупная дрожь.
Грисвольд прикрутил конечности юноши к кольцам, жёстко зафиксировал голову винтами и вытащил из кармана ларингоскоп.

\ml{$0$}
{--- Выполняй мои команды.}
{``Obey my commands.}
Если я скажу <<согни палец>> --- ты сгибаешь.
Если я скажу <<расслабься полностью>> --- делай что хочешь, но выполняй, от этого зависит качество оцифровки периферических нервов.
Иногда будет очень щекотно, жарко, холодно или даже больно --- это тоже часть процесса, мы не будет тебя калечить.
Терпи и старайся не шевелиться, если я не дал соответствующую команду.
Я постараюсь предупреждать заранее обо всех ощущениях.
\ml{$0$}
{Если тебе вдруг покажется, что ты ослеп, оглох, потерял способность говорить или дышать, или что у тебя отнялись руки, ноги или лицо, или что тебя вдруг начали преследовать звуки, запахи или образы --- постарайся успокоиться и жди, мы смонтируем тебя как можно скорее.}
{If you suddenly feel you went blind, or deaf, or mute, or you lost ability to breathe, if you can't feel your arms, or legs, or face, if you suddenly see, or hear, or smell something strange---try to relax and wait, we'll mount you as soon as possible.}
\ml{$0$}
{Помни --- я и Лу рядом и ни за что тебя не бросим.}
{Remember: Lu and me both are near by you, and we'll never foresake you.}
А теперь открой рот как можно шире и приготовься --- будет больно.

Грисвольд вставил в рот юноши расширитель.
Тахиро зажмурился.

\chapter{[U] Визит командующего}

\section{[U] Милая девочка Штрой}

--- Атрис, ты дурак, --- смеялась Митхэ.
--- Разве можно так шутить над живыми существами?

--- Я не шутил, --- обиделся Атрис.
--- Просто хотел проявить оригинальность.
Билатеральных существ как песка на пляже.
А вот с трилучевой симметрией почти нет.

--- Они очень милые, --- заверила Митхэ друга.
--- Хорошо, что сейчас, после переговоров с дельфинами, они размножились и могут покинуть Коралловую бухту.

Атрис пожал плечами и снова стал наблюдать за малышами-нгвсо, которые резвились у самого берега.
Кажется, один нашёл большую красивую ракушку и совсем не хотел делиться с остальными.
Он плавал кругами, и время от времени его щупальца били по воде, поднимая тучи искрящихся брызг.
Под водой царил весёлый гул, щелчки и повизгивание, но слуха сухопутных сапиентов не достигало ничего.

--- Может, покормим их? --- предложила Митхэ.
--- Они любят вишню.
И другие фрукты тоже.

--- Лучше не надо.

--- Это же дети, Атрис! --- Митхэ подёргала менестреля за рукав.
--- Детей можно!..

--- Не стоит.
Я стараюсь как можно меньше вмешиваться в их жизнь.

Митхэ обняла Атриса и уткнулась кудрявой головкой ему в шею.
Цветы в её волосах потускнели, попав в тень.

--- Я скучаю по нашему хранителю, милый.

--- Мы почти его не знали, --- напомнил ей Атрис.
--- И потом, он был инкарнацией демона Аркадиу Шакал Чрева.

--- Это совершенно не мешает мне скучать по ним обоим.

--- Да, мне тоже, --- признался Атрис.
--- Это я глупость сморозил.
Интересно, кто это?
Похоже, что к нам.

--- Может, Корхес или кто-то ещё из биологов, --- предположила Митхэ.

--- Они придут вечером с конфетами, --- возразил Атрис.
--- Намечается что-то интересное.
Корхес хотела поговорить насчёт нескольких видов мух.
А Кольбе и Рабе обещали показать странную диатомею, которая строит алмазные раковинки.

--- Алмазные?
В жидкой воде, при обычном атмосферном давлении?

--- Да.
И, представь себе, я тут вообще ни при чём.

--- Сама?

--- Сама.

По мокрому песку шла девушка-оцелот, хрупкая и нежная на вид, почти ребёнок.
Её босые тонкие ноги утопали в мокром песке, комбинезон из сложного полимера был расстёгнут, обнажая костлявую грудную клетку и верхушки маленьких грудей.
Атрис хмыкнул --- эта неестественная, показная соблазнительность выдавала чужаков, инкарнированных демонов Ада.
Люди сели, привыкшие к наготе тела, не считали нужным акцентировать внимание на груди или половых органах.

Девушка подошла поближе, и все сомнения на её счёт рассеялись.
Пухлые детские губки кривились в уверенной взрослой улыбке, а оранжевые опалесцирующие глаза светились древностью и памятью бесчисленных жизней.
Она подошла к парочке и по очереди пожала им руки.

--- Штрой Кольцо Дыма, командующий силами Ада на Тра-Ренкхале, --- представилась девушка, не забыв сделать в сторону Атриса лёгкое целующее движение губами.

Митхэ поморщилась.

--- У тебя проблемы с гормонами?
Ведёшь себя странно.

--- Я ещё не совсем освоилась с телом тси, --- сообщила девушка.
--- Возможно, мне следует поменять паттерны поведения, они не годятся для общества сели.
Вы не первые, кто это заметил.
На мужчин других миров моё поведение действует завораживающе, а мужчины-сели только смеются и не желают со мной общаться.
В общем, прошу понять и простить.
Мы можем поговорить?

--- Да, разумеется, --- сказал Атрис.
--- Недалеко есть беседка.
Ты не откажешься от чашечки холодного травяного отвара?

--- Буду очень признательна, --- махнула длинными ресницами Штрой.
--- Сегодня жарко.
Правда, вам, голограммам, это сложно понять.

\section{[U] Чай}

Беседку установили местные жители.
Безымянный не скрывался от глаз сапиентов --- одно из требований новой администрации;
его достаточно часто видели в нескольких ключевых местах планеты, в том числе и неподалёку от Кахрахана.
Атрис прекрасно понимал, что это было сделано с целью разрушить культ --- даже необычно выглядящее существо становится обыденностью, если видеть его каждый день.

Но у сели культа как такового и не было, Безымянный был для них обычным героем легенд, подобно Забытому или Маликху.
Они просто сделали для него личную беседку из чёрного дерева, словно для уважаемого путешественника.
Такие же скромные, но со вкусом сделанные домики стояли ещё в двух местах в землях сели --- возле Сотрона и Ихслантхара, у ноа --- возле Яуляля, и у хака --- возле Спокойного озера.
Безымянному они нравились гораздо больше, чем выстроенные тенку пышные капища или часовенки ркхве-хор.
Они были похожи на нормальные жилища, а не на культовые строения, и в них можно было расслабиться.

Сели, хака и ноа нравились Безымянному ещё и тем, что не тревожили его просьбами.
Все местные знали, если не от демонов, то по слухам --- Безымянный долгое время был в изгнании, но продолжал ухаживать за планетой.
Люди изо всех сил старались показать, что они рады вернувшемуся создателю --- и ценили его уединение.

Вскоре в беседке из чёрного дерева запахло пряным паром отвара.
Митхэ плеснула в чашу жидкого азота из спрятанной под скамьёй азотной станции, и седые клубы волнами разбежались по узорчатой поверхности стола.

--- Благодарю, --- улыбнулась Штрой и приняла прохладную чашу.

\ml{$0$}
{--- Мы тебя слушаем, --- кивнул Атрис.}
{``We're listening,'' \Aatris{} nodded.}
\ml{$0$}
{--- К нам редко заходит кто-то, кроме биологов.}
{``We don't get many visitors, except some biology folk.''}

\ml{$0$}
{--- Кольбе и Рабе сегодня не придут, --- сообщила девушка, отвечая на завуалированный вопрос.}
{``Colbe and Rabe won't come today,'' the girl answered the veiled question.}
\ml{$0$}
{--- У них появились срочные дела.}
{``They've got some urgent matters.''}

Митхэ хмыкнула.

\ml{$0$}
{--- Неужели разговор настолько важный, что его нужно тянуть до позднего вечера? --- нахмурился Атрис.}
{``Are the talks valuable enough to take our evening?'' \Aatris{} frowned.}

\ml{$0$}
{--- Вы торопитесь?}
{``What's the hurry?''}

Штрой наклонилась, чтобы отхлебнуть отвара, и щёлкнула языком от удовольствия.

\ml{$0$}
{--- Вкусно.}
{``Quite delicious.}
\ml{$0$}
{Благодарю вас.}
{Thank you.}
\ml{$0$}
{Итак.}
{Here we go.}
Знаете ли вы Аркадиу Шакала Чрева?

--- Излишний вопрос, --- поморщилась Митхэ.
--- Аркадиу --- наш друг, и его последнее тело было нашим с Атрисом хранителем.
Думаю, ты знаешь это не хуже нашего.
\ml{$0$}
{С ним что-то случилось?}
{Something wrong with him?''}

\ml{$0$}
{--- Да, случилось.}
{``Yes, something's wrong with him.}
\ml{$0$}
{Он предал Орден Преисподней и сбежал с Капитула в неизвестном направлении.}
{He has betrayed the Order of Nether and left Capitul for an unknown destination.''}

Митхэ ахнула.

\ml{$0$}
{--- Что значит <<в неизвестном направлении>>? --- осведомился Атрис.}
{``What do you mean by `an unknown destination'?'' \Aatris{} asked.}
\ml{$0$}
{--- Кажется, ваша контрразведка знает всё --- кто, куда и с какими намерениями сбегает.}
{``Your counter-espionage is supposed to know everything---who runs, where, and what they have in mind.''}

--- Он сумел подсунуть контрразведке простую, но весьма правдоподобную легенду, --- ответила Штрой.
--- Обман был раскрыт минуту спустя, отдел 100 направил мне официальный запрос по работе биологов.
За это время Падальщик успел сбросить хвост и исчезнуть.

--- Какие бы силы ни стояли за твоей спиной, осторожнее выбирай слова, --- невыразительно проговорила Митхэ.
--- Предатель Аркадиу или нет, он хорошо послужил Ордену.
Тебе стоит уважать его хотя бы поэтому.

Штрой улыбнулась.

\ml{$0$}
{--- Если угодно.}
{``If you prefer.}
\ml{$0$}
{Так или иначе, след потерян.}
{Anyway, we've lost the trace.}
Ожидалось его появление в трёх системах --- Тысячи Башен, Тра-Ренкхаля и Тси-Ди.
Агенты всех трёх пока молчат.

\ml{$0$}
{--- Почему именно эти три?}
{``Why those three?''}

Штрой пожала плечами и отхлебнула отвара.

--- Могу сказать, что Люпино определенно попытается использовать потомков тси и их наследие в своих целях.

--- Это мы и сами поняли.
А Тысяча Башен?
Я мало что знаю про эту планету...

--- О, это прекрасное убежище для криминального элемента.
Куча Друз, скалистые теснины, пещеры в горах, хйяры.
Экваториальный кластер планеты де-юре контролируется Орденом, де-факто нейтрален и находится под влиянием легата Курт Осенний Огонь, которая весьма толерантна к нейтралам.
Тропический пояс де-юре под властью Картеля, де-факто так же нейтрален, и заправляют там Острые Шила, клан из шестнадцати демонов столь же свободных взглядов.
И Курт, и Острые Шила --- уроженцы планеты и знают её досконально.

--- Я бы ожидала там полное безвластие.

--- Отнюдь.
Они научились жить дружно, умудряясь при этом слать отчёты о своих успехах, весьма правдоподобные.
Однако правду не скроешь.
Именно из-за этого туда начали сползаться научные коллаборационисты.
Ну и у меня есть сведения, что недавно легат Курц летала к кому-то <<в гости>> в тропический пояс.
Не будь у меня информации от агентов, я бы решила, что это Люпино.

--- Что произошло? --- спросила Митхэ.
--- Почему он предал Орден?
\ml{$0$}
{Кому он теперь служит?}
{Who he serves now?''}

--- Вполне возможно, что никому, Митхэ ар’Кахр.
Некоторые аналитики считают, что он помешался, несмотря на коррекцию личности --- стигмы помешательства в его поведении присутствуют.

--- Что ты имеешь в виду под <<помешательством>>? --- осведомилась Митхэ.

--- Они имеют в виду, что Ликхмас ар'Люм взял верх над Аркадиу Люпино, --- ответил Атрис.

Штрой прикрыла глаза.

--- Всё гораздо сложнее, но в целом суть передана верно: субличность носителя становится приоритетнее субличности демона.
Если речь идёт об ангельских технологиях, такое обычно происходит без борьбы в буквальном смысле.
Просто личность демона добровольно перекладывает полномочия по принятию решений на личность носителя, частично или полностью, так как, с её точки зрения, это отвечает интересам обоих.

--- И что в этом плохого?

--- Для демона, разумеется, ничего.
А вот для его контрагентов это повод для недоверия.
Поэтому, если помешательство доказано, то все договоры с демоном теряют силу, ранги обнуляются, демон освобождается от всех постов и должностей, в том числе ставится вопрос и о принадлежности демона к Ордену.

--- Но в отношении Люпино это доказано не было.

--- Не было, --- подтвердила Штрой.
--- Но полученные данные говорят о том, что он заслуживает пристальное наблюдение.

--- Из-за чего он мог помешаться? --- удивилась Митхэ.

--- Есть некоторые \emph{подозрения}, --- Штрой едва заметно сделала акцент на последнем слове.
--- Каждое новое тело изменяет демона, это давно доказано.
Собственно, коррекция личности и направлена на нивелирование этих изменений.
Но все исследования такого рода проводились на диких сапиентах, а не на творениях генной инженерии, которыми были тси.
Возможно, что тело Ликхмаса ар’Люм вызвало тонкий, но тем не менее серьёзный системный сбой в Аркадиу.
Его товарищи признаны годными к дальнейшей деятельности, хотя я бы не спешила выносить такой вердикт.

Митхэ всё больше раздражала её манера общения.
Всё в молодой девушке, даже её улыбка, было вызывающе сексуальным, бросающим вызов ей и Атрису.
Вскоре Митхэ поняла, что именно её раздражало.
Движения Штрой казались идеальными, отточенными до предела.
Это были движения машины, хоргета, но никак не милой девушки-сели, которая по воле судьбы несла этого хоргета на себе.
Демон не доверял своему телу и контролировал его во всём.
В этом было его отличие от тех демонов, которых Митхэ знала прежде.

Слова Штрой только подтверждали это случайное наблюдение.

--- Возможно, он снова перешёл на сторону Картеля, --- предположил Атрис.

--- Исключено, --- отрезала Штрой.
\ml{$0$}
{--- Это не подлежит сомнению.}
{``It mustn't be questioned.}
Он чересчур насолил Картелю.

--- По-вашему, это препятствие для того, чтобы принять в свои ряды сильного противника?

--- Я выражусь иначе: он чересчур насолил некоторым иерархам Картеля.
\ml{$0$}
{Если в Аду у него почти не было личных врагов, то там их, я полагаю, на пару порядков больше, чем должно быть для относительно спокойного существования.}
{He barely had enemies in the Hell, but there, he has more---I guess, two orders of magnitude higher than it needs for relatively quiet life.}
Но ладно, ближе к теме.
Нас интересуют некоторые детали.
Когда вы виделись с Аркадиу в последний раз?

--- Около двадцати дождей назад, здесь, на Тра-Ренкхале, --- задумалась Митхэ.
--- Он сказал, что больше не хочет воевать и попросит переквалификацию.
Пообещал иногда навещать нас.

--- А виделись ли вы с ним на станции Деймос-14?

--- Нет, --- ответил Атрис.
--- На станции нас встретили Грейсвольд Каменный Молот и Анкарьяль Кровавый Шторм, они проинструктировали нас и заверили согласие на сотрудничество с Орденом.
Приказ исходил от Лусафейру Лёгкая Рука.
Правда, мы думали, что согласие должен брать один исполнитель в присутствии двух наблюдателей, но...

--- Значит, третьим должен был быть Аркадиу, --- глухо сказала Митхэ.

--- Вот именно, --- улыбнулась Штрой.
--- Если не ошибаюсь, Грейсвольд и Анкарьяль уже были вызваны на допрос, и им предстояло долгое и весьма неприятное объяснение, почему второй наблюдатель во время процесса болтался неизвестно где.
Хорошая иллюстрация к вопросу об их пригодности.

--- С ними всё хорошо? --- Митхэ схватила Штрой за худую руку и тут же отпустила, поймав совсем не дружеский взгляд девушки.

--- С ними всё в порядке, их лояльность уже долгое время не вызывает сомнений, --- сухо ответила Штрой.
--- Теперь о вас.
Во-первых, вы должны в ближайшие пятьдесят три секунды прибыть на станцию 1A, чтобы вас проверили агенты отдела 100.

--- Хорошо, --- сказал Атрис.
--- Но проверить вы сможете только меня, Митхэ --- оцифрованный...

--- У отдела 100 есть средства для проверки оцифрованных сапиентов, --- прервала его Штрой.

Атрис вздрогнул.

--- Чистилище?

--- Это не мне решать.
Данные о структуре вашего демона я передала, в частностях отдел 100 разберётся.
Вам ведь нечего скрывать?

--- В Чистилище невиновность не является защитой от страданий, Штрой, и ты это прекрасно знаешь, --- голос Атриса оставался тихим, но глаза сверкнули совершенно человеческой яростью.

\ml{$0$}
{--- В том, что вас скомпрометировали, есть доля вашей вины, --- парировала Штрой.}
{``You're partly responsible for being compromised,'' Stroji retorted.}
\ml{$0$}
{--- Аккуратнее выбирайте друзей.}
{``Be careful making friends.}
\ml{$0$}
{Далее.}
{Next.}
\ml{$0$}
{Я так понимаю, вы оба испытываете к Шакалу дружеские чувства.}
{As I understand it, both of you feel friendship for Jackal.''}

Митхэ и Атрис переглянулись и кивнули.

--- Так или иначе, он нарушил присягу, и его судьба отныне находится в ведении отдела 100.
Поэтому хочу вам напомнить, что сотрудничество с предателем или укрывательство такового является нарушением, предусмотренным Оборонительным Кодексом Ордена Преисподней, статья DB, пункт 4.
О санкциях распространяться не буду, читайте сами.

Атрис опустил голову.
Воительница смотрела в оранжевые глаза девушки своим спокойным зелёным взглядом, который так красочно описывали её давно умершие враги.

--- И кое-что не для протокола.
На случай, если вы всё-таки захотите ему помочь, --- Штрой поднялась и томно потянулась, выгнув тонкую талию.
--- Вряд ли то, что носит сейчас личину Аркадиу, на самом деле является им.
Благодарю за отвар.

Не попрощавшись, Штрой Кольцо Дыма вышла из беседки и манерной походкой отправилась на север.

\section{[U] Прошедшая Чистилище}

\epigraph
{Пусть замки из песка дождём бесследно смоет,\\
Упрямый мальчуган их заново построит.\\
Я строчку начертал у моря на песке,\\
И тянется прибой когтями к той строке.\\
Едва мои стихи волна с собой умчала,\\
Я, как дитя, готов игру начать сначала.\\
Но навсегда стихам конец, когда из вод\\
Хотя б одну строку волна назад вернёт.}
{Сигурдур А.\,Магнуссон, <<Начертанное на песке>>.
Эпоха Последней Войны, Древняя Земля}

Митхэ сидела в песке, прижимаясь к Атрису.
По её лицу текли слёзы, перемешиваясь с каплями только прошедшего ливня.

--- Я надеюсь, что мне больше никогда не придётся столкнуться с контрразведкой, --- прошептала она.

Атрис выглядел гораздо хуже.
Модуль визуализации ещё работал, голограмма передавалась без сбоев, но сам образ искажался.
Глаза, волосы и кожа Атриса меняли цвет, черты лица расплывались, словно менестрель надел неисправный ноппэрапон\FM.
\FA{
На Древней Земле: маска театра Кабуки из магнитного полимера, принимающая любую форму и цвет.
}
Иногда изображение пропадало совсем, и казалось, что Митхэ обнимает воздух.

--- Да, я знаю...

--- Агенты сказали, что это скоро пройдёт.
Потерпи.

--- Я с тобой, милый.
Я всегда буду с тобой.

--- Хочешь песенку?
Я спою, а ты сыграешь.

--- А почему, ты же любишь музыку?

--- Ну ладно.

Из кустов показался ребёнок сели.
Он целеустремлённо, с забавно серьёзной мордашкой прошёл до берега моря и сел недалеко от парочки.
Атрис представлял собой ужасающее, сюрреалистичное зрелище, но ребёнок только улыбнулся беззубым ртом.
Вскоре он уже сидел и строил из мокрого песка домик.
Крохотные ручки неумело собрали песчаную гору и начали формировать крышу.
Митхэ с нежностью наблюдала за ним.

--- Интересно, откуда он?

--- Понятно.
Можно, я ему помогу?

--- Я вижу, что он прекрасно справляется сам.
Вопрос был не об этом.

Солнце начало клониться к закату.
Менестрель почти пришёл в себя.
У него всё ещё были видны странные аберрации в волосах, но лицо уже стало прежним.
Он посмотрел на Митхэ и улыбнулся.

--- Ну вот, Ад нашёл повод и тебе сунуть Чистилище под нос.

--- Я держалась молодцом? --- спросила Митхэ.

--- Да, --- кивнул менестрель.
--- Один из агентов сказал мне следующее: <<Личность Митхэ ар’Кахр перешагнула последнюю ступень целостности.
Если Митрис Безымянный будет заподозрен в деятельности, направленной против Ордена, мы будем вынуждены сразу уничтожить его человеческую часть>>.

--- Ну хоть мучить не будут, --- ободрительно сказала Митхэ.
--- Жаль только, что я не смогу облегчить твои страдания, если до подобного дойдёт.

Атрис помолчал.

--- Забавно.
Чистилище --- это термин из давно забытой религии.
Это страдания, которые дарил древний бог, приговаривая: <<Ты мой верный раб, но за грехи ты ответить обязан>>.

--- Грехи --- это преступления перед богом?

--- Да, это нарушения его законов.

--- А что давало этому богу право требовать их исполнения?

--- Он оправдывал это тем, что он создал сапиентов.

Митхэ поморщилась.

--- Не горшечник решает, когда время горшку разбиться.
Это решает тот, кто способен нанести горшку смертельный удар.

--- Верно, --- признал Атрис.
--- Если же кто-то бьёт такие прекрасные горшки, стоит задуматься --- а он ли их слепил.
У меня бы не поднялась рука.

С востока потянуло солёной прохладой, и волны стали забегать чуть дальше на берег.
Вот одна, затем другая достигла одинокого, похожего на торт домика.
Волны слизывали стены, словно крем.
Атрис рассеянно смотрел на останки сооружения.

--- Знаешь, чем мне нравятся дети?

--- Чем, Атрис?

--- Они замечательно умеют повторять, не понимая смысла.

--- Незачем понимать назначение и механизм дыхания, если это нужно для жизни, --- заметила Митхэ.

--- Но ребёнка необходимо научить правильно дышать.
Он не всегда будет в стандартных условиях существования.

--- Да, --- лицо Митхэ просветлело.
--- В Храме меня учили правильно дышать.

--- А если научить некому, то большинство сапиентов так и остаются детьми.

Митхэ кивнула и крепче прижалась к любимому.

--- Что было в сообщении? --- тихо спросил Атрис.

--- Он уже здесь.
С завтрашнего дня можем приступать, --- ответила Митхэ.

Атрис улыбнулся и погладил подругу по голове.

--- А вот теперь я бы сыграл.

\chapter{[U] Равновесие}

\section{[U] Зачисление в контрразведку}

\epigraph
{Тси --- это сапиенты, которые используют машины для обмена данными с внешней средой.
Демоны --- это машины, использующие для той же цели сапиентов.
Может быть, именно поэтому у нас с ними возникает недопонимание.}
{Длинный-Мокрый-Хвост}

\textspace

--- Компромата на вас собрано достаточно.

--- Компромата? --- выдохнула Анкарьяль.

--- Не обращай внимания, Нар.
У Самаолу своеобразное чувство юмора, --- буркнул Грейсвольд.
--- И он настолько любит шутить, что порой забывает поздравить демонов с самой бескровной успешной операцией со времён Тахиро.

--- Я не давал тебе слова, Грейсвольд Каменный Молот, --- ледяным тоном сказал Самаолу.

--- Если хочешь командовать чужими ртами, научись следить за своим.
Что ещё за компромат?

Самаолу демонстративно достал компьютер и начал листать слайды.
Игра была очевидна для всех --- демон и так был в курсе материала.

--- Первое.
Вы были друзьями Аркадиу Шакала Чрева.

--- И нас по этому поводу уже проверили, --- парировал Грейсвольд.

--- Разумеется.
И из проверки вытекает второе --- в вас замечена ощутимая симпатия к тси.

Грейсвольд и Анкарьяль промолчали.

--- То есть вы даже не будете этого отрицать?

--- В отрицании есть смысл? --- буркнул Грейсвольд.
--- Отдел 100 отвечает за верификацию полученных данных.
Кроме того, в своде законов Ада есть прямой запрет санкций за мыслепреступления.

--- Впрочем, этот запрет --- такая же формальность, как и текущее заседание, --- усмехнулась Анкарьяль.

--- Хорошо, --- кивнул Самаолу, пропустив слова Анкарьяль мимо ушей.
--- В таком случае не могли бы вы осветить подробнее своё отношение к тси?

--- Вообще-то я не обязан этого делать, но так и быть.
На планете Тра-Ренкхаль я впервые узнал, что искренность может быть культурной нормой.
Я первый и последний раз оказался в обществе, которое приняло меня тем, кто я есть.
Тси относились ко мне и к другим демонам так же, как к соплеменникам.

Пара советников испустили презрительный смешок.
Самаолу нахмурился.

--- А что скажете вы, Анкарьяль Кровавый Шторм?
Если я не ошибаюсь, сели вам поклоняются.

--- Не мне, Самаолу, а обезличенному образу, отпечатку личности давно умершего тела.
Вас ввели в заблуждение.

--- Не прибедняйтесь.
В образе хватает черт и вашей нематериальной части.
Очень любопытно включение в лик Самоотверженного Хата двух звёзд, двух лун и двух гор, которые некогда красовались на знамени Ангары, баронессы Краснобуря с планеты Тысяча Башен.

На лице Анкарьяль дёрнулся мускул.
Она была уверена, что компрометирующий <<подарок>> сделал ей Аркадиу, напоследок отведя от себя все подозрения.

\ml{$0$}
{--- Впрочем, неважно.}
{``However, it doesn't matter.}
В вашем отчёте прямо указано, что это совпадение, а в вас достаточно сложно найти замашки демиурга-индивидуалиста.
\ml{$0$}
{Всё это время вы были лезвием меча, частью команды, и не претендовали на иную роль.}
{All this time you've been an edge of the blade, a part of the crew, and never pretended to be someone else.}
Так что вы думаете насчёт тси?

Как ни лжива была ситуация, Грейсвольд заметил, что последние слова ей польстили.
Да, так и есть.
Лезвие меча, часть команды.

--- Скажу, что Катаклизм Тси-Ди был величайшей трагедией для нас, --- спокойно ответила Анкарьяль.

В зале наступила оглушительная тишина.
Грейсвольд вдруг тоже нахохлился.

--- Объясните, --- сказал Самаолу после продолжительного молчания.

--- Как вы знаете, мы находимся в мёртвой зоне Вселенной, --- заговорила Анкарьяль, и её слова эхом отдались в зале.
--- Наши перемещения ограничены константами Ка'нета, и, несмотря на все уверения технологов, с момента открытия их не удалось отодвинуть даже на одну триллионную яо.
Предпринималось множество попыток найти разумную жизнь в пределах ка'нетовского радиуса, но ни одна не увенчалась успехом.
Огромный шар диаметром в тридцать шесть тысяч парсак --- и всего двадцать девять планет с самозарождённой стабильной жизнью, из них с разумной --- две!
Вдумайтесь в это.
Наши проекты по терраформированию терпят крах один за другим.
То есть, говоря без обиняков, мы, высокоразвитая цивилизация хоргетов, не можем сделать даже того, что сделали древние земляне.
Вы сидите здесь, гордитесь своими технологиями, многие из вас презрительно называют первых людей <<обезьянами>>, но по сути вы --- посмешище для собственных создателей.

--- Терраформирование?
Вы серьёзно? --- заливисто рассмеялась советник Кагуя Хрустальный Голос.
--- И это тогда, когда Картель активно возвращает себе позиции на Плеядах и Развязке Десяти Звёзд?

--- Война для вас важнее, Кагуя? --- невинно поинтересовался Грейсвольд.
--- Ведь благодаря ей вы сидите в этом мягком кресле?

Анкарьяль подняла руку, показывая, что ещё не закончила говорить.
Самаолу жестом остановил попытавшуюся ответить Кагуя и кивнул интерфектору.

--- Ещё касательно эффективности Ордена.
Возьмём Тра-Ренкхаль.
Аркадиу Люпино сразу после битвы на Могильном берегу вызвал геологов, чтобы они остановили землетрясения.
Да, --- повысила голос Анкарьяль, перекрывая смешки, --- да, \emph{предатель} озаботился обустройством владений Ордена.
С тех пор прошло сто пятьдесят четыре дождя.
Где геологи?
Из-за предательства Аркадиу его запрос потерял актуальность?
Безымянный до сих пор обеспечивает безопасность сапиентов за счёт собственных ресурсов.

--- Мы не обсуждаем на этом заседании политику Ада по отношению к Митрису Безымянному, Анкарьяль, --- холодно бросил один из советников.

--- Ах, это аспект отношений с Безымянным, а не простая халатность? --- проворчала Анкарьяль.
--- Благодарю за разъяснение, советник, вы спасли меня от крамольных мыслей.

--- Тишина, --- призвал к порядку Самаолу.
--- Вернёмся к обсуждению тси.
Грейсвольд, вам есть что добавить?

--- Я не вполне согласен с формулировками Анкарьяль, но по сути она права.
Мы столкнулись с теми же проблемами, которые когда-то потрясли цивилизацию Древней Земли.
Однако мы гораздо могущественнее первых людей, наши ошибки гораздо более серьёзны, и потому масштаб катастрофы нам оценить едва ли не сложнее, чем им.
Даже если предположить, что исход войны близок и Картель падёт, куда, скажите на милость, направится взор нашей военной машины?
Хотите сказать, что зверь, которого кормит весь Ад без исключения, покорно сложит лапки и заснёт вечным сном?

--- Враги есть всегда, Грейсвольд, --- заявила Кагуя.

--- И вы отлично умеете их находить, --- поклонился Грейсвольд, --- даже там, где их отродясь не было.
Что вам сделал Безымянный?
Такого покладистого демиурга ещё поискать.
С демиургами всегда были проблемы --- взять того же Эйраки, или Кох Свободолюбивую, или...

--- Или вас, --- присовокупила Кагуя.

--- Безымянный --- лучшее, что есть на Тра-Ренкхале, --- громко и членораздельно сказал Грейсвольд.
--- Он подобен тем наивным древним учёным --- философам, астрологам, алхимикам, колдунам, жрецам, --- которые выстраивали сложные мистические теории мироздания, которые безо всякой научной и идеологической базы искали, находили и делали выводы.
Он художник, и его планета --- один большой холст.
У меня как демиурга никогда не хватило бы фантазии на то, что сделал Безымянный.
Однако Ад выстроил против этого миролюбивого создания такую защиту, словно на Тра-Ренкхале хранится эссенция Красного Картеля в хрупком стеклянном флаконе.
Мне даже довелось слышать, как Безымянного сравнили с Уэсиба Серозмеем.

Кагуя вдруг густо покраснела, но заметила это только Анкарьяль.
Всё внимание советников было приковано к Грейсвольду.

--- Если Безымянный так опасен, почему он ещё жив?

Совет замер.
В воздухе повисла неловкая тишина.

--- Я серьёзно, --- как ни в чём не бывало продолжил технолог.
--- Мы прекрасно знаем, что любые договоры --- лишь красивые слова, если за ними не стоит реальная сила.

--- Грейсвольд, вы в курсе, что это?..

--- Я знаю, что это основа деспотической доктрины Картеля и, чёрт возьми, это честный и здравый взгляд на вещи.
Демоны имеют право голоса не потому, что это правильно, а потому, что за каждым демоном стоит его собственный трудовой и военный потенциал, которым он готов распоряжаться по собственному усмотрению.
Организация защищает своих членов не потому, что это правильно, а потому что именно это делает её организацией.
Да, договор --- это лишь символ.
Тем более этот договор, с громким названием <<Демиург --- Метрополия>>, клише которого было создано, между прочим, по моей инициативе, на всякий случай.
Я надеялся, что это будет подспорьем для социализации демиургов, позволит тем, кто не готов перейти на демонический образ жизни, быть равными нам, не быть хоргетами второго сорта.
Однако субъектов договора в истории Ордена всего шесть, в настоящее время --- один-единственный, если не считать <<без вести пропавшую>> Кох.
Счёт же уничтоженных богов --- тех, чьё уничтожение было задокументировано --- идёт на десятки тысяч.
Так что же мешает Аду \emph{ещё раз} устроить тотальную мелиорацию и окончательно решить вопрос демиурга и девиантной фауны?

\ml{$0$}
{--- Вы подозреваете Орден в уничтожении Кох, Грейсвольд? --- вмешался один из советников.}
{``Are you suspecting the Order of destroying Coj, Grejsvolt?'' a counsellor intervened in an argument.}

\ml{$0$}
{--- Ни в коем случае, советник.}
{``No, counsellor.}
\ml{$0$}
{Я не подозреваю, я в этом уверен.}
{I'm not suspecting, I'm sure of it.}
Разыгравшийся вокруг Безымянного спектакль мне до боли напоминает уже отгремевшие и давно забытые события.
И знаете, есть некая ирония в том, что Аркадиу Люпино, чьё первое тело выросло под крылом Богини-Матери Пустоши Драконов, предал Орден в тысячную годовщину того, как Богине-Матери помогли найти дорогу в небытие.

--- Может, Грейсвольд, вам снова выдать планету? --- усмехнулся кто-то в зале.
--- Вы с таким упоением расписываете прелести жизни демиурга...

Зал захохотал.

--- А вы знаете, я только за, --- подбоченился технолог.
--- И моя мечта исполнилась бы, если бы не всеобъемлющая зависть.

Зал снова огласил хохот.

--- А разве я не прав? --- продолжал Грейсвольд.
--- У каждого из хоргетов достаточно способностей, чтобы инкарнироваться в планету.
Извините, что напоминаю об этом, но нас для этого и создавали, не так ли?

\ml{$0$}
{--- Не имеет значения, для чего нас создавали, Грейсвольд, --- заявил Самаолу.}
{``What we were created for doesn't matter now, Grejsvolt,'' Samajolu said.}

\ml{$0$}
{--- Отнюдь!}
{``It does matter!}
Едва ли такая проблема стояла перед первыми людьми --- они были самообразовавшимися живыми системами и вольны были сами выбирать своё предназначение.
Но мы --- продукт инженерии, разумный замысел, а не игра слепой эволюции.
\ml{$0$}
{Да, мы все --- прирождённые демиурги.}
{Yes, we all are demiurges by design.}
Но мы упаковали себя в микроскопические сапиентные тела, мы красим волосы и глаза в немыслимые цвета, мы отращиваем себе когти и щупальца, мечтая о движении континентов, лесах из светящихся грибов и цивилизациях, расцветающих под нашей опекой!

--- Что вы пытаетесь этим сказать?

--- Пропаганда самоограничения самореализации была одним из самых древних способов получить и удержать власть.
Прямо или косвенно до нас доносят мысль, что быть демоном --- правильно, а богом --- нет.
И пока мы принимаем это за истину, все мы, вне зависимости от ранга, не более чем рабы.
Это всё, что я хочу сказать.

--- Если ваша соратница имеет привычку сквернословить, Грейсвольд, то вы явно любите драматизировать, --- буркнул Самаолу.
По галерее пронёсся смешок.
--- Я предпочитаю первое.
Вернёмся к обсуждению тси.
Итак, что бы вы предложили, Анкарьяль?
Какую политику следовало бы проводить Ордену в отношении тси?

Несколько советников, включая Кагуя, снисходительно улыбнулись.

--- Политику мира.

--- Сильно, --- признал один из советников, Кес Бледный Глаз.
--- Я думаю, нет смысла спрашивать, как этот мир достичь.
Но ради справедливости, Анкарьяль, я повторю вопрос вашего товарища Грейсвольда --- где были бы \emph{вы}, не будь войны?

Треть совета смущённо кашлянула.
Прочие взглянули на Кеса, словно он справил нужду прямо посреди зала.

--- Без войны я была бы в опасной экспедиции за пределами ка'нетовского радиуса, --- сказала Анкарьяль.
--- Я бы пронзала пространство и время, искала новые земли, улыбалась новому свету, а в минуты отдыха размышляла бы, как послать весточку вам, старому миру, как сообщить затерянному среди галактик дому, что мы тоже живы и процветаем.
Мой друг Грейсвольд был бы демиургом, и едва ли наши пути пересеклись бы.
Но, как видите, мы стоим сейчас плечом к плечу, отстаивая право быть собой.

--- Вам никто не мешает быть собой, Анкарьяль --- разумеется, в пределах закона.
И я так и не понял, при чём тут цивилизация Тси-Ди.

--- Стабильная технократическая цивилизация сапиентов, которой являлись тси, могла колонизировать дальний космос, дать вам плацдармы для дальнейших исследований и продвинуть границу Ка'нета на тысячи парсак вглубь галактики, к её ядру.
Наши исследования дают однозначные результаты --- в ядре мы сможем найти достаточно подходящих для жизни планет.
Некоторые исследователи даже говорят, что имеющихся там ресурсов достаточно, чтобы создать и запустить звёздный корабль на разумных антаридах, преодолеть...

--- Оставьте эту чушь, Анкарьяль, --- сказал Самаолу.
--- Звёздный корабль --- очередная сказка для романтиков вроде вас.
Разумные антариды --- из немного другой сказки, ещё менее правдоподобной.
Жизненный цикл антарид настолько короток, что исключает наличие сапиентной архитектуры любого уровня.
Хоргеты не способны выйти за пределы Млечного Пути.
Это возможно лишь для микоргета, но последние исследования ставят под вопрос саму возможность его существования.

--- Антариды...

--- Самаолу прав, Анкарьяль, --- вдруг подала голос Корхес Соловьиный Язык.
--- Даже если создать разумных антарид, существование антаридной цивилизации не будет долгим.
Антариды --- это вам не тси.
Мы получим либо безжизненную звезду, либо взрыв сверхновой в самые короткие сроки, звёздный корабль даже не успеет выйти за пределы Млечного Пути.
И я бы попросила закрыть эту тему, потому что ваша квалификация как биолога --- и тем паче плазмобиолога --- оставляет желать лучшего.
\ml{$0-[ej]$}
{А я люблю науку и не люблю, когда военные, политики и прочее жульё начинают лапать её своими грязными ручищами.}
{I love science, and I hate when military, politicians, and other crooks touch it with their dirty hands.''}

--- Корхес, выражения аккуратнее выбирайте, --- призвал к порядку Самаолу.

--- Даже если и так, --- Анкарьяль с трудом сдерживала раздражение, --- почему хотя бы не освоить до конца Млечный Путь, почему не заручиться союзом с тси?
Да, скорость их кораблей ограничена светом, но они достигли бы большего успеха за единицу времени, чем вы.
И я уверена, что даже если бы Ад и Картель обрушились всей своей мощью на колонистов, тси всё равно нашли бы время на обустройство новых планет.
Они не считали нужным постоянно просиживать у орудий.

\ml{$0-[ej]$}
{--- Вы думаете, что мы не рассматривали такие варианты? --- высокомерно улыбнулся Самаолу.}
{``Do you really think we haven't considered such an option?'' Samajolu arrogantly smiled.}

\ml{$0$}
{Анкарьяль явно хотела сказать, что она думает о Самаолу, но сдержалась.}
{Ancarjal obviously wanted to speak out what she really thought about Samajolu, but stopped herself.}

--- В таком случае объясните нам с Анкарьяль, какие именно нюансы она не учла, --- проворковал Грейсвольд.

--- Тси знали о существовании демонов, --- медленно, словно для умственно отсталых, начала объяснять Кагуя.
--- Более того, им удалось достигнуть с нами стратегический паритет...

--- Опять военный термин, --- поморщился Грейсвольд.

--- А какие термины вы предпочитаете применять к враждебной цивилизации?
Это война, Грейсвольд.
Мы пытаемся выстоять против чужой агрессии.

--- То же самое говорили миллионы агрессоров до вас, --- сказал технолог.
--- Слепцы по-прежнему верят, что есть праведные войны.
Романтики считают, что время праведных войн прошло.
А я, видевший историю Земли почти с самого её рассвета, скажу --- праведных войн нет и не было.
В любой войне число агрессоров равно числу участников!

--- Вы предлагаете сложить оружие и отдаться на милость Картеля? --- ласково осведомилась Кагуя.
--- Или вы, Анкарьяль.
Вы позволили бы тси колонизировать наши планеты?
Капитул?

--- Вы никак не можете понять, что тси --- это не обычные сапиенты, которых мы используем как дойных животных! --- повысила голос Анкарьяль.
--- Они равные нам, а кое в чём и превосходящие нас существа!

--- Тси --- это хамелеоны.
Они легко интегрируются в любое общество и бессознательно внедряют в это общество свои ценности, распространяют свою заразу...

--- Как и представители любой древней, выкристаллизовавшейся культуры, --- заметил Грейсвольд.
--- Мало какая культура может состязаться с тси в возрасте.
За двести тысяч лет культура тси въелась в их генофонд, приобрела устойчивость к множеству различных воздействий, и даже жестокий отрицательный отбор не даст быстрых результатов.

--- Особенно со стороны менее древней культуры, --- ядовито добавила Анкарьяль.
По галерее пробежал возмущённый шёпот.
--- <<Зараза>>!
Вы говорите об информации, словно о смертельном вирусе, от которого у вас нет защиты.
По-вашему, адепты Ада не способны фильтровать информацию?
Мне доводилось слышать мнения, что народ Тси-Ди --- тепличные растения и пример культурного дрейфа\FM.
\FA{
Культурный дрейф --- ненаправленные изменения культуры, обусловленные случайными статистическими причинами.
}
И вдруг выясняется, что их культура --- <<зараза>>, от которой нужно защищаться!
Что это значит?
Выходит, перед нами не дрейф, а стадия эволюции, до которой нам с вами ещё далеко?

--- Так или иначе, наиболее яркие представители погибли на Могильном берегу, --- с гадливой жалостью сказала Кагуя.
--- Остаткам культуры был нанесён непоправимый удар.

--- Я бы не делала столь поспешных выводов, --- улыбнулась Анкарьяль.
--- Вы недооцениваете тси, и это вам ещё аукнется в будущем.

--- Да, действительно, есть исследователи, считающие, что сам контакт Ордена Преисподней с культурой тси опасен, --- хмуро сказал Самаолу.
--- Вы поддерживаете это мнение, Анкарьяль?

--- Опасность контакта оценивают контактирующие согласно собственному опыту, --- туманно выразилась Анкарьяль.
Кто-то из советников нахмурился, кто-то улыбнулся, Кагуя же фраза привела в бешенство.
--- Но изменения определённо произойдут.

--- То есть вы хотите сказать, что опасность мы придумали сами? --- спросила Кагуя, едва сдерживаясь.

--- Тси угрожали убить вас лично, Кагуя?

--- Что за профанация, Анкарьяль? --- опешила советник.
--- Что вы имеете в виду?
Я член Ордена Преисподней, а это значит...

--- Это значит, что ответственность за статус-кво лежит в том числе и на вас! --- перебила Анкарьяль.
--- Почему-то как речь об опасности --- то она общая, а ответственность за политику чья угодно, но только не ваша!
В отличие от Картеля, у нас была возможность установить с тси договорные взаимоуважительные отношения, но мы использовали их как разменную монету в войне с тем же самым Картелем.
Как и в случае с Безымянным, Орден выбрал войну вместо мира.
Я боюсь, что если ретроспективно подвергнуть такому же анализу действия Ордена за последний телльн, то окажется, что большинство реше...

--- Довольно, --- рявкнул Самаолу.
--- Мы поняли вашу точку зрения.
Напоминаю, что Орден Преисподней предлагал тси протекторат, тси это предложение отвергли.
Это было их и только их решением!

--- Протекторат? --- задохнулась Анкарьяль.
--- Вы это называете <<взаимоуважительными отношениями>>?
Я бы сделала то же самое, да ещё и плюнула вам в лицо!
Тси сдерживали Картель куда эффективнее, чем вы!
Тси-Ди была самым спокойным местом во Вселенной!

--- Вы называете Тси-Ди спокойным местом? --- спросил советник.
--- Вот вам немного статистики. Основная причина смерти тси --- несчастные случаи на производстве.
Девяносто четыре процента всех смертей.
Тси жили на своих великих стройках и умирали на них же, как рабы Эйгипта, которые строили для своих фарай величественные гробницы.
Это цивилизация рабов, ничем не отличающаяся от прочих сапиентных цивилизаций.

--- И даже их конец был одним большим несчастным случаем на производстве, --- ядовито добавила Кагуя.
--- Вы технолог, Грейсвольд?
Вы понимаете, о чём я говорю.

--- Вы искажаете суть статистических данных, --- возразил Грейсвольд, проигнорировав слова Кагуя.
--- Тси умирали при несчастных случаях не из-за обилия таковых, а из-за того, что до смерти от естественных причин они просто не доживали.
Смертность по медицинским причинам контролировать гораздо легче, чем смертность от несчастных случаев, и тси это показали очень хорошо.
Собственно говоря, я не слышал ни одной истории о демоне, который мирно прекратил существование у себя дома.
Основные причины смертности у нас --- убит, казнён, пропал без вести.
Это моя собственная статистика, я её веду по друзьям и знакомым...

--- Довольно, --- поднял руку Самаолу.
--- Ваши статистические выкладки никем не проверены и потому не представляют для нас интереса.

--- И я совершенно не понимаю аналогии с гробницей, --- заметила Анкарьяль.
--- Если уж вы хотите заручиться союзом с историей Древней Земли, то я сделаю то же самое.
Вы были для них вандейлами, которые молотили каменными топорами в прекрасные стабитаниумовые ворота Ромай!
Тси умирали, пытаясь починить дом, который...

--- Довольно! --- рявкнул Самаолу.

Анкарьяль хмуро скрестила руки на груди.

--- Впрочем, даже жаль, что тси отгородились от нас, --- отчетливо сказала она в тишине, пренебрегая запретом.
\ml{$0-[ej]$}
{--- Если бы демоны жили на Тси-Ди на равных условиях с аборигенами, если бы они видели, как эти существа из плоти и крови относятся друг к другу, Ад и Картель, эти колоссы на глиняных ногах, развалились бы в первые же двадцать лет.}
{``If daemons could live on Qi-Di with its dwellers, as equals, if they could see, how beings of flesh and blood used to treat each other, the Hell and the Cartell---these colossi with feet of clay---have come apart in the first twenty years.''}

Грейсвольд запоздало схватил подругу за плечо, пытаясь её остановить.
\ml{$0$}
{Слово прозвучало.}
{The words have been said.}
Советники потрясённо молчали.
Казалось, в этой оглушительной тишине слышно, как бьются сердца и гудят стабилизирующие модули омега-сингулярностей --- если бы их колебания были доступны для слуха сапиента.

Наконец Самаолу вздохнул.
Это был вздох человека, который утвердился в худших своих опасениях.

--- Мы, --- он обвёл рукой зал, --- все мы могли бы много чего сказать на это скоропалительное, я бы даже сказал --- безответственное заявление.
Но мы здесь не для того, Анкарьяль Кровавый Шторм, чтобы разъяснять вам очевидное, тем более что имеет место проявление некой идеологии, а не простое заблуждение.
\ml{$0$}
{Идеология, в отличие от заблуждения, прекрасно умеет защищаться от фактов и теорий.}
{Ideology, unlike mistake, can perfectly defend itself against facts and theories.''}

\ml{$0$}
{--- К сожалению, Самаолу, это так, --- сухо сказала Анкарьяль.}
{``Unfortunately, it's true, Samajolu,'' Ancarjal coldly said.}

\ml{$0$}
{--- Я рад, что хоть в чём-то мы сошлись, потому что у нас с вами нет другого выбора, кроме как найти общий язык.}
{``I'm glad we found a thing we both agree, because we have no option but cooperation.}
Как я уже говорил, мы набрали массу интересного материала касательно вас... --- Самаолу позволил себе загадочную паузу, --- и тем не менее к вам есть предложение поступить в резервный состав отдела 100.

Грейсвольд и Анкарьяль удивлённо промолчали.

--- Обретает форму новая сила, --- сказал Самаолу.
--- У нас недостаточно данных о её природе, но достаточно оснований предполагать её враждебность.

--- Говори проще, Самаолу, я тебе пелёнки когда-то менял, --- поморщился технолог.
Он взглянул на Анкарьяль --- демоница вдруг побелела как полотно.

--- Контрразведке нужны демоны, которые способны думать, как враг.

--- То есть вы даже не скрываете, что считаете нас врагами! --- вспылила Анкарьяль.
--- <<Держи друзей близко, а врагов ещё ближе>>, да?
Я не буду переводиться на таких унизительных условиях!

--- Что именно вы считаете унизительным? --- осведомился один из советников.

--- Я вернулась с Тра-Ренкхаля, где ради интересов Ордена чуть не отправилась \emph{жилами джунглей}! --- выкрикнула Анкарьяль.
Самаолу прищурился.
\ml{$0-[ej]$}
{--- Какие бы ошибки я не совершила в молодости, сейчас я легат терция и заслужила карьерный рост, а не участь подопытной мыши или надувного монстра, которого вы будете трахать в надежде избавиться от собственного страха!}
{``Whatever mistakes I made in my youth, now I am legate tercia and I have deserved career progression, not a fate of guinea pig or inflatable monster which you will fuck in the hope of conquering your own fear!''}

--- А само зачисление в ряды контрразведки вы не считаете повышением?

Анкарьяль в десяти словах на языке фоуф-у\FM{} объяснила Совету, что она думает о таком повышении.
\FA{
Пиджин наёмников Тысячи Башен, вобравший в себя рекордное количество обсценных слов из других языков --- 6\% корней.
В основном эти слова касаются сексуальных отношений, физических и психических недостатков, отправления естественных нужд, состояний сознания при приёме различных наркотиков, а также изощрённых способов пытки и убийства.
}
Советники вытаращили глаза, когда семантический переводчик выдал шестнадцать слайдов мелкого текста.
Немного знакомый с этим языком Грейсвольд поперхнулся ещё на третьем слове.
Входящий в Совет диктиолог-лингвист Кельса Пушистая украдкой включила записную книжку.

--- Кельса, это уже занесли в протокол, --- ледяным тоном сказал Самаолу, и демоница тут же убрала терминал.
--- Если сквернословие представляет для тебя научный интерес, я отправлю официальную выписку в твой отдел.

<<Нар, я тебя люблю, --- с нежностью подумал Грейсвольд.
--- Так блестяще сыграть прожжённую карьеристку и высказать правду в лицо не смог бы даже я.
Теперь моя очередь купаться в этом вонючем болоте, доказывая свою профпригодность.
Чтоб этого Аркадиу поглотила бездна --- сам пропал, а нам оставил тележку с навозом!..>>

--- И всё-таки, к чему устраивать этот спектакль с отделом 100, если нас просто хотят проверить на верность? --- осведомился технолог.

--- Совет похож на театр, Грейсвольд Каменный Молот? --- один из советников позволил себе улыбку.

--- И весьма, Эйяфар Цеппелин.
Вы с Анкарьяль достаточно молоды, чтобы не знать, кто разрабатывал правила функционирования отдела 100.
А есть в их своде следующее --- за пределами, простите за фигуральность, можно играть камнями всех четырёх цветов\FM, но в самом отделе должен царить абсолютный порядок.
\FA{
Намёк на Метритхис, настольную игру сели.
}
И резервный состав --- не исключение.
Возможно, я ошибаюсь и правила поменялись?

Пара сидящих в зале зашевелились.

--- Вы хотите проверить меня на верность?
Прошу вас.
Вы видите во мне лишь впадающего в деменцию старика.
А я вижу шестнадцать горшков, которые звенят на слепившего их горшечника.

--- Оставьте литературные обороты при себе, --- сказал Самаолу.
--- Отдел 100 только что сообщил, что вы зачислены в резервный состав.
Официальная формулировка --- <<за многократно доказанную лояльность и высочайший профессионализм>>.
Это должно вас устроить.
Инструктаж будет проведён через двадцать секунд на станции 80, о ваших функциональных телах позаботятся.
Заседание окончено.

Грейсвольд застыл столбом и взглянул на подругу.
Анкарьяль положила руку ему на плечо, и оба упали замертво посреди зала.

\section{[U] Большая игра}

\textspace

--- Итак, Штрой, вы всё-таки придерживаетесь мнения, что это ковен Картеля, --- произнес голос.

--- Именно, владыка, --- Штрой говорила чётко и размеренно, былая жеманность улетучилась без следа.
--- Почерк Картеля я узнаю всегда и везде, потому что сама приложила к нему руку.

--- И поэтому вы ввели их в заблуждение.
Что ж, неплохо.
А вы, Самаолу?

--- Несмотря на сравнительный анализ Штрой, связи с Картелем я не нашёл, --- угрюмо сказал Самаолу.
--- Гораздо больше фактов говорит о том, что в деле замешана третья сторона.
А как считаете вы?
Вы проводили какие-то исследования...

--- Их результаты вас не касаются, --- спокойно отрезал голос.
--- Продолжайте разрабатывать собственные версии.
Кстати, Самаолу, ваш трюк с Анкарьяль и Грейсвольдом был просто великолепен.
Так блестяще сыграть тупицу я бы не смог, у вас большие задатки... кхм... театрального актёра.
Поблагодарите за меня Кагуя --- она тоже прекрасно справилась.

--- Обязательно, --- слегка улыбнулся Самаолу.
--- Приятно иногда вносить изменения в правила, особенно в правила отдела 100.
Относительно этих демонов план остаётся прежним?

--- Разумеется, --- подтвердил голос.

--- Правда, я не совсем понимаю, в чём их важность, --- полувопросительно заметил Самаолу.

--- Её действительно сложно понять, --- согласился голос.
--- Анкарьяль и Грейсвольд заняли весьма неудобное для нас положение --- <<красные плащи в узком ущелье>>\FM.
\FA{
Тактический ход, возможно, восходящий к легендарной битве у Фермопил, Древняя Земля.
}
Они имеют богатый опыт совместной работы и безоговорочно друг другу доверяют.
Собственно, нам нужно изучить этот тандем в действии.
Некоторые расчёты дают право полагать, что пара Анкарьяль-Грейсвольд является \emph{единственным} связующим звеном между непринадлежными Скорбящими и агентурной сетью Скорбящих внутри Ада.

--- Провокатор-связной, --- ввернула Штрой.
--- Вызывающий, постоянно находящийся на виду, специально обученный отражать атаки и вызывать сомнения в собственной виновности.
Почерк Картеля.

--- Почему мы не можем их просто устранить физически? --- поморщился Самаолу.

--- Во-первых, Самаолу, это не решит конкретную проблему.
То, что деталь единственная в механизме, вовсе не означает, что она незаменима.
Во-вторых, провокатор-связной чрезвычайно устойчив и к физическому воздействию --- чаще всего это опытный интерфектор, даже удачная попытка будет стоить вам несколько ценных кадров, что уж говорить о неудачной.
Публичную личность в принципе сложно устранить без последствий.
А в-третьих --- оставьте привычку решать проблемы силой.

--- У нас вполне достаточно ресурсов, чтобы...

--- Да, да, у нас достаточно ресурсов, чтобы от всех наших соперников --- по крайней мере, внутренних --- не осталось даже воспоминаний.
Но в таком случае можно сразу расписаться в неумении управлять.
Решая проблемы силой внутри Ордена, вы лишите себя ценнейших ресурсов в борьбе с внешним противником.
Да, я считаю всех Скорбящих ценным ресурсом, Штрой, --- как ни крути, они смогли сделать очень многое.
Поэтому мы будем держать волка в клетке, пока у нас есть хотя бы призрачная возможность договориться.
Тем более, как Самаолу уже сказал, когда-то Анкарьяль и Грейсвольд были вернейшими адептами Ордена, а это значит, что они могут снова стать такими.

--- Никогда не доверял предателям, --- буркнул Самаолу.
Штрой ощерилась.

--- Тихо, --- голос поднял невидимую ладонь, и демоны разом вытянулись в струнку.
--- Предатель, Самаолу --- это прежде всего тот, кто умеет сравнивать и не боится делать выбор.
То, что Анкарьяль и Грейсвольд в итоге оказались в агентурной сети Скорбящих, означает три вещи.
Первое --- с Орденом что-то не так, и это очевидно для всех, кто способен думать.
Второе --- они не считают Орден безнадёжно загнившим, так как формально остались на службе, а не сбежали, как крысы с тонущего корабля.
Да-да, Самаолу, после зачисления в отдел 100 они могли сбежать, а не подвергать себя риску --- но выбрали риск.
И третье --- они растут как личности и как профессионалы, и это очень важный момент.
Впрочем, действовал этот тандем очень осторожно, но наживку заглотил с половиной удочки.
Они не настолько умны, как загадочные стратеги 03 и 05.

--- Возможно, владыка, нам следует расширить выборку демонов, подпадающих под...

--- А какой смысл, Штрой? --- возразил голос.
--- Даже если бы сейчас стратеги 03 и 05 явились ко мне лично и принесли исчерпывающий компромат на самих себя, это мало что изменило бы.
Вы думаете, что это зелёная молодёжь, которая ищет своё место в жизни?
Основную проблему я вам очертил сразу --- мятежников поддерживают группировки, с которыми уже давно достигнуто равновесие Нэша.
Говоря другими словами, фигуры, которые давно и прочно стоят в своих клетках, неожиданно поменяли цвет, сделав ставку на молодого и неопытного игрока.
Когда я впервые осознал этот факт, то был озадачен.

--- Владыка, я не совсем понимаю...

--- А, Штрой, прошу прощения, я забыл про особенности обучения стратегов в Картеле.
Вы читали доклад Самаолу?

--- Разумеется!

--- И какие ваши впечатления?

--- Анкарьяль и Грейсвольд пожертвовали своим по...

--- Именно, --- откликнулся голос.
--- Штрой, вы потрясающий стратег.
Даже Самаолу посчитал произошедшее гамбитом со стороны Ада.

--- А теперь, похоже, не понимаю я, --- протянул Самаолу.

--- Именно поэтому вы со Штрой работаете вместе, --- объяснил голос.
\ml{$0-[ej]$}
{--- Политика --- странная сводня, но вы, должен признать, замечательно друг друга дополняете.}
{``Politics makes strange bedfellows, but I have to admit that you two are perfectly complementary.''}

Штрой и Самаолу бросили друг на друга взгляды, полные отвращения.

--- У вас, как я понимаю, нет статистики по отделу 100?
Так слушайте: за последние четырнадцать лет контрразведка потеряла в три раза больше демонов, чем мы на полях сражений.
Туда берут лучших --- и эти лучшие мрут, как тараканы.

Самаолу ахнул.

--- Они побледнели...

--- ... от страха, а не от злости, --- закончила Штрой.

--- Браво.
Вы поняли.
У них есть источник информации, который держит их в курсе дел.
Перевод из легатов запаса в центурионы отдела 100 --- это даже не смешно.
Анкарьяль и Грейсвольду придётся пройти через кошмар, прежде чем они закрепятся на новом месте.
Они знали о предстоящем возмездии за изменение собственной стратегии, но не подозревали о его силе.
Истинное равновесие Нэша --- жестокая вещь.

--- И такие потери в позициях понесли все, кто поддержал эту группировку? --- спросила Штрой.

--- Не такие, Штрой.
Нашим друзьям просто сильно не повезло.
Подозреваю, что Грейсвольд принял на себя часть удара, предназначенного не ему --- он едва избежал смерти уже как минимум два раза, если считать последний инцидент на Тра-Ренкхале.
Возможно, он защитил этого неуловимого стратега 03.

--- Вы считаете события на Тра-Ренкхале платой за изменение стратегии? --- поднял бровь Самаолу.
--- У меня создалось впечатление, что...

--- Друг мой, стратег 03 действует не первое столетие.
И тихоня Грейсвольд --- его главный помощник.
Вероятность того, что инцидент на Тра-Ренкхале бы нэшевским рикошетом, стремится к единице.

--- Кто-то из стратегов крупно просчитался, --- улыбнулся Самаолу.

--- Или сделал чересчур широкий шаг вперёд, --- подхватил голос.
--- Никогда не мешает предположить худшее, особенно если учесть, что за прошедшие пять лет <<красные плащи>> неплохо отыгрались.
Награды сыпались на Грейсвольда с Анкарьяль словно из мифического рога изобилия.
Честно говоря, я думал, что это заставит их отказаться от полевой работы.
Но я ошибался.
Что ж, хотят опасностей --- у центурионов отдела 100 их хватает.

--- Владыка, не может ли стратег 03 быть Падальщиком? --- спросила Штрой.
--- Несмотря на явную разницу в уровне, в почерке есть общие черты...

--- Исключено, --- сухо ответил голос.
--- Сходство можете списать на что угодно --- на подражание, случайность или наставничество.
Падальщик --- пешка.
Оставьте его в покое, он предсказуем и может быть уничтожен в любой момент.
На звание одного из стратегов у нас есть претенденты куда старше и опаснее.

--- Стигма, --- проворчал Самаолу.
--- Я употребил всё своё влияние, чтобы отослать её на Тра-Ренкхаль, подальше от основного театра действий.
Любое обвинение стекает со Стигмы, как ртуть со стекла.

--- Владыка, раз уж Стигма оказалась на Тра-Ренкхале, я могу устранить её физически, --- предложила Штрой.

--- Даже не пытайся, --- резко сказал Самаолу.
--- Она съест тебя на завтрак и не посмотрит на твои лычки.
Лимес, Бит и Уитлик умерли не без её помощи...

--- И всё же я прошу у владыки разрешения, --- поклонилась Штрой.

--- Штрой Кольцо Дыма, --- голос позволил себе оттенить слова великолепной усталой интонацией.
--- Оставьте свои манеры для деспотического Картеля.
Советую вам ознакомиться с протоколом <<Шляпа>>, который является сводом приемлемых для Ада норм общения.

Штрой смешалась, но тут же овладела собой.

--- Так точно, вла...
Хорошо, я учту.

--- Так-то лучше, --- милостиво откликнулся голос.
--- Насчёт Стигмы --- попробуйте, я отведу от вас удар.
Удача может не прийти к пытающемуся, но к бездействующему она не придёт гарантированно.
Устранить Стигму нужно было ещё пятьсот лет назад, как по мне --- несмотря на коварство и приспособляемость, в некоторых вопросах она не склонна к компромиссам.

--- Вы пытались с ней договориться? --- спросил Самаолу.

--- И не только я, --- вздохнул голос.
--- Проблема в том, что она пацифист до мозга костей.
Её устраивает любой вариант, исключающий военные действия против Картеля.

--- И при всём миролюбии Стигмы те, кто пытается её устранить, крайне плохо кончают, --- ввернул Самаолу.
--- Чего стоит эта история с кланом Чёрной Скалы...

--- Да, я слышала про стратега, который якобы охотится на собственный клан, --- задумалась Штрой.
--- Значит, это Стигма?

--- Я думаю, в данном случае применимо слово <<вырезает>>, а не <<охотится>>, --- усмехнулся Самаолу.
--- Из ста двадцати трёх членов в живых остались пятеро.

--- Трое, --- поправил голос.
--- Двойняшки, Фуси и Нуива, погибли на Тра-Ренкхале от рук рекомого Грейсвольда.
Судьба не лишена иронии.

--- А какие вообще доказательства, что Стигма истребляла выходцев из клана Антрацис? --- спросила Штрой.

--- Никаких, --- ответил голос.
--- Как нет и прямых доказательств, что клан вообще кто-то целенаправленно уничтожал.
Могу сказать только одно --- с демонами Антрацис происходили несчастные случаи, этих несчастных случаев было много, и Стигма каждый раз оказывалась где-то недалеко --- участвовала в планировании операций, либо имела опосредованные связи с исполнителями.
Были и весьма нехорошие происшествия, которые позволяют оценить её методологию.
Например, трое перебежчиков из Антрацис успешно добрались до Капитула и были уничтожены уже там агентом всё той же Чёрной Скалы.
Вопиющий случай --- вы знаете лучше меня, Штрой, какие ресурсы тратятся на защиту перебежчиков.
Известно, что это члены её генерации, которые в своё время принимали участие в травле.
Учитывая, что Стигма просто не могла нанять ассассина изгнавшего её клана, могу предположить, что она сделала --- или не сделала --- всё, чтобы он успешно выполнил задание.

--- А ассассин далеко не ушёл, --- присовокупил Самаолу.

--- Да, кстати, уйти ему не дали.
Спасибо, что напомнили, Самаолу.
Четверо мух одним лёгким ударом --- разве не впечатляюще?
Собственно, одна из причин, по которой отдел 100 крайне неохотно подключает Стигму --- это её своеобразное, карикатурное понимание пацифизма.
Если на тебя охотятся --- истреби клан, который это делает.
Если для мира необходимо сократить число солдат, а демобилизация недоступна --- она будет сокращать их число другими способами.
Школа Чёрной Скалы, работа на результат.

--- А кто ещё из Антрацис остался в живых?

--- В живых, кроме Стигмы, остались Ду-Си Охотник и Корхес Соловьиный Язык --- такие же Вечно Гонимые, как и она.
Их она трогать не будет.

Штрой дёрнулась, словно одно из имён доставило ей боль.

--- Они не контактируют между собой?

--- Нет, Штрой.
Я понимаю ход ваших мыслей, но нет.
Корхес в принципе презирает военных и старается держаться от них подальше.
Ду-Си и Стигма не проявляли друг к другу интереса почти с самого основания Ордена.
Думаю, причина чисто клановая --- они из разных генераций.

--- Я бы сказал, что это серьёзный послужной список, --- заметил Самаолу.
--- Чёрная Скала --- клан воинов.
Со стратегом, уничтожившим его в одиночку, надо быть осторожнее.
Стигма редко ошибается.

--- Это ненадолго, Самаолу, --- успокоил голос.
--- Кстати, как вам их название?

--- Скорбящие, --- выплюнул Самаолу.
--- Знакомый культуролог сказал, что само название пахнет каким-то рыцарским кодексом.
Мы во время зачисления Анкарьяль и Грейсвольда в отдел 100 прощупали их в том числе на предмет идеологии.
Данных собрано достаточно, они обрабатываются.

--- Это здорово, если только старина Грейс не подсунул нам легенду, --- согласился голос.

--- Я рискну высказать наше с Кагуя впечатление, --- проговорил Самаолу.
--- Анкарьяль и Грейсвольд были абсолютно искренни.
Они совершенно не боялись логических атак и проб.
Более того, они хотели, чтобы их идеологию проверили на прочность.
Кагуя услышала в их речах вызов --- <<против простой истины не устоит даже ваша хитроумная ложь>>.

--- Хм, --- удивился голос.
--- Как только получите данные культурологов, Самаолу, доставьте их мне лично.
А что насчёт той картинки?

--- Печальный Митр, --- доложила Штрой.
--- Символ из анимистических верований сели.
Именно благодаря ему я вышла на Митриса Безымянного.
Они настолько увлеклись символизмом, что забыли о конспирации.

--- Если у них и впрямь имеется подобие рыцарского кодекса, это немудрено... --- заговорил Самаолу.

--- Тихо, --- спокойно выдал голос.
Оба демона разом онемели и подобрались.
--- Легкомысленного отношения я не приемлю.
В этом уравнении чересчур много неизвестных, чтобы считать Скорбящих сборищем глупцов.
Я ясно выразился?

--- Так точно, --- хором сказали Штрой и Самаолу.
Самаолу выглядел совершенно спокойным, а Штрой поёжилась.

--- Есть ли пути, которые вели бы к Лусафейру?

Штрой и Самаолу переглянулись.

--- Понятно, --- недовольно сказал голос.
--- А жаль.

Штрой снова поёжилась.
Она чересчур болезненно воспринимала любые проявления недовольства со стороны патрона, и Самаолу бросил на неё короткий презрительный взгляд.

--- Я знаю старину Лу очень давно, --- криво улыбнулся демон.
--- Он друг Грейсвольда, но вряд ли заодно со Скорбящими и знает больше нас.

--- Вы переоцениваете свой опыт, Самаолу, --- сказал голос.
\ml{$0-[ej]$}
{--- Лусафейру очень осторожен.}
{``Lusafejru is very cautious.}
\ml{$0-[ej]$}
{Вы не представляете, насколько --- это та осторожность, которую не смогут ослабить ни тысячелетний мир, ни клинок у горла, ни обвинение в трусости.}
{You can't imagine how, it's a kind of cautiousness which couldn't be made less by anything---thousand-year peace, or blade at the throat, or accusations of cowardice.}
\ml{$0-[ej]$}
{Лусафейру, я не побоюсь сказать, едва ли не единственный в Аду, кто знает цену ошибке.}
{Lusafejru, I daresay, is about the only one in the Hell who knows the price of mistake.}
\ml{$0-[ej]$}
{И при этом он продолжает общаться с Грейсвольдом, который, как известно, под подозрением.}
{But he stays in touch with Grejsvolt who is, as you both know, suspected.}
\ml{$0-[ej]$}
{Возможно, что у Лусафейру есть план.}
{I guess Lusafejru's up to something.}
\ml{$0-[ej]$}
{Возможно, что мы недооценили силу этой связи.}
{I guess we underestimated their friendship.''}

Демоны снова переглянулись.

--- Вам не кажется, что вы чересчур превозносите наших врагов? --- скривился Самаолу.

--- Не врагов, а соперников, Самаолу, --- терпеливо заметил голос.
--- Грамотный стратег не употребляет таких терминов, как <<друг>> и <<враг>>.
Оставьте их для тех, кого кормите пропагандой.

--- Эти <<соперники>> убьют вас при первой же возможности.

Голос тихо засмеялся.

--- Я не буду переубеждать вас, Самаолу, --- ваша святая уверенность придаёт пропаганде силу, так же как дружба, возможно, придаёт силу Скорбящим.
Чувства вообще замечательная вещь --- именно потому, что они императивны по сути, им сложно противиться.
%{Emotions are wonderful, actually, they are imperative, they are hard to disobey.}
Многие демоны называют акбас проклятием... но где бы мы были без него?

--- Вы считаете, что Лусафейру доверяет Грейсвольду так же, как Грейсвольд доверяет Анкарьяль?

--- Не могу сказать точно.
Однако Лусафейру и Грейсвольд очень долго общались с Тахиро, и это не могло не сказаться на их профессиональных навыках.
Вам для справки, Штрой: Тахиро Молниеносный, Стратег-неожиданность --- так его звали.
Он обращался с вероятностями так, как мы обращаемся с точными числами.
Сам он канул в вечность, как и почти все близкие ему демоны.
Почти все.

--- Мне следует прервать общение Лусафейру и Грейсвольда? --- осведомился Самаолу.

--- Вы вряд ли сможете это сделать, --- заметил голос, --- но усложнить это общение несколькими обычными, легко оправдываемыми методами не мешает.
Если Грейсвольд и Анкарьяль останутся в отделе 100 в изоляции, они обязательно совершат фатальную ошибку.
Фатальную для них... или для всех Скорбящих, если случайность будет к нам благосклонна.
Да, кстати, насчёт Безымянного.
\ml{$0-[ej]$}
{Речь Грейсвольда произвела некоторое волнение, и к демиургу сейчас приковано незаслуженно пристальное внимание.}
{Grejsvolt's speech produced a lot of sound and fury, and now the demiurge is a focus of unduly close attention.}
\ml{$0-[ej]$}
{Тем не менее это низкоинтеллектуальное скопление масс-энергии успело наступить на ноги половине Ада.}
{Nevertheless, that lowly intelligent mass-energy storage had enough time for stepping on toes of half the Hell.''}

\ml{$0-[ej]$}
{--- Безымянного оставьте мне, --- поклонилась Штрой.}
{``Leave Nameless to me,'' Stroji nodded.}
\ml{$0-[ej]$}
{--- Как только шум уляжется, музыкант и дикарка дорого заплатят за то, что ввязались в чужую игру.}
{``Once things settle down, the tunesmith and she-savage shall pay deerly for interference with someone else's game.''}

\section{[U] Селекция в Картеле}

Часть обратного пути Штрой и Самаолу шли вместе.

--- Всегда хотел спросить вас, Штрой, --- вдруг заговорил Самаолу.
--- Правда ли, что в Картеле существуют так называемые <<связки>> --- множество молодых демонов соревнуются в выполнении какой-либо функции, чтобы были отобраны самые лучшие?

--- Да, --- неохотно сказала Штрой.

--- А что происходит с теми, кто непригоден, Штрой?

--- Непригодные ликвидируются, Самаолу.

--- Хм.
Значит, Картель вместо инженерии пошёл по старому доброму пути селекции, указанному ещё матерью-природой.

--- Инженерия без селекции --- ничто, Самаолу, --- ощерилась Штрой.
--- Даже созданное инженерами следует подвергнуть испытаниям.

\ml{$0$}
{--- Разумеется, вы абсолютно правы.}
{``Of course, you're absolutely right.}
\ml{$0$}
{Необходимо приручить игру случайности, отбросить тех, кто случайно оказался непригодным, и выделить тех, кто хорош по той же самой случайности.}
{We should tame this game of randomness, we should throw out ones who are randomly useless, and take ones who are equally randomly useful.}
\ml{$0$}
{Возможно, придумать концепцию удачи, вывести формулу успеха --- ведь нужно же что-то рассказывать рабам, которым успех не светит?}
{Maybe we should come up with the conception of luck, or formula for success; we have to tell something to slaves who never have success as an option, don't we?}
\ml{$0$}
{Рабы не любят слово <<случайность>>, оно их пугает.}
{Slaves hate the term `randomness', it scares them.''}

--- Слова бывалого пропагандиста, --- Штрой произнесла последнее слово тоном изысканного оскорбления.

--- Заключение опытного стратега, --- тем же тоном ответил Самаолу.
--- Мне до вас далеко.
Впрочем, моих стратегических навыков тем не менее оказалось достаточно, чтобы не показывать патрону всё, на что я способен.
Зато теперь, когда вы прошли все испытания и достигли вершины эволюционной лестницы, у вас есть десятки агентов --- тех, кому повезло меньше, кому необязательно знать о равновесии Нэша и кто будет платить за все ваши ошибки.

--- Удел слабых --- гибель, --- голос Штрой звучал почти победно.
--- Разве нет?

--- Разумеется, вы абсолютно правы.
Занятно.
Хм...
Я задам вам последний вопрос, Штрой, и перестану беспокоить.
Удовлетворите моё любопытство --- скольких жизней стоило \emph{ваше} место?

--- Я оказалась лучше восьмиста тысяч ювеналов.
Это имеет значение?

Самаолу брезгливо хмыкнул.

--- Это заметно, Штрой.
Вы лучше многих.
Мало кому пришлось пережить столь жёсткий отбор.
Но кое в чём вы уступаете самому последнему легионеру Ада.

--- И в чём же? --- осведомилась Штрой.

--- Над вами всегда, до последнего мгновения жизни будет висеть призрачная рука.
Рука того бездушного механизма, который производил отбор.

И, не попрощавшись, Самаолу пошёл к ближайшему лифту.

\chapter{[:] Рассвет Жёлтого моря}

\section{[:] Первая смерть}

\epigraph
{Наибольшую эффективность организация имеет в сферах, которые были ключевыми в процессе её становления.
Группировка, захватившая власть военным путём, не способна эффективно организовать мирную жизнь, так как отбор кадров изначально шёл в другом направлении.}
{Лусафейру Лёгкая Ладонь}

Лу и Тахиро сидели с бокалами вина.
Лу держал бокал, оттопырив мизинец; Тахиро же по-деревенски заграбастал свой за венчик, беззастенчиво полоская пальцы в вине.
Третий бокал стоял перед Грисвольдом;
в нём не хватало едва ли глотка.

--- Ну и как ты? --- улыбнулся Лу.
--- Привыкаешь к демону?

--- Понемногу, --- кивнул Тахиро.
--- Но иногда всё равно странное чувство, что во мне кто-то сидит.

--- Потому он и называется <<демоном>>, я полагаю.

--- Немного раздражают эти странные видения --- вспышки, волны, летающие огоньки.

--- Это проекции омега-поля на зрительную кору, --- многозначительно сказал Лу, подняв палец.

--- Они не должны раздражать, --- встрепенулся Грис.
--- Вечером будем калибровать.

--- Иногда словно кто-то думает за меня, --- продолжал Тахиро.
--- Я начинаю мысль и вдруг щёлк --- появляется результат.
А хода мысли я не вижу.

--- Это нормально, --- пояснил Лу.
--- Демон автоматически перехватывает управление потоком.
Это сделано для удобства, чтобы не переключаться на омега-консоль вручную.

--- Но всё равно надо поработать над обратной связью, --- заметил Грисвольд.
--- Тахиро \emph{должен} понимать, что это демон перехватывает поток, а не играет его собственная фантазия.
Ну и какую-то техническую информацию...

--- Не-не, Грис, вот технической информации мне как раз хватает, --- перебил его Тахиро.
--- Цифры, названия давно разрушенных городов, имена давно умерших людей.
Что вообще за дурацкая традиция в математике давать методам имена по месту открытия или по имени учёного?
Никакой поэзии.

--- Да забей, --- поморщился Лу.
--- Я вообще не задумываюсь.
Всё, что мне нужно знать о конкретном методе --- к чему его можно применять, а к чему нельзя.
Остальным пусть занимаются в исследовательских отделах.
Тебе же пока вообще надо тренироваться на кошках --- складывай числа, вычисляй траектории, можешь предсказать цену креветок в Такэсако на месяц --- экономические утилиты там какие-то есть.

\ml{$0$}
{--- А ты на чём тренировался?}
{``What did you practice on?''}

Лу рассмеялся.

\ml{$0$}
{--- На считалках.}
{``Rhymes.''}

\ml{$0$}
{--- Каких считалках?}
{``What rhymes?''}

\ml{$0$}
{--- Детских.}
{``Counting-out rhymes of children.}
\ml{$0$}
{Мы с отцом часто катались по ущелью и за его пределами, и я наблюдал за играми детей.}
{My father and me used to travel a lot, around the canyon and beyond, so I watched children play.}
Соответственно, запомнил массу дразнилок, считалок и прибауток.
Тематическое моделирование считалок показало, что они распространялись примерно так же, как люди заселяли эту территорию.
Считалки с неприличными словами появились в западной части Такэсако --- видимо, из-за недостатка воды.
А самые мерзкие и злые дразнилки водятся в Асахина, и когда я читаю отчёты о тамошней ситуации, я их не виню.
Дети --- это зеркало мира взрослых.

Люцифер отхлебнул вина.

--- Я тоже хочу исследовать считалки! --- восхищённо сказал Тахиро.

\ml{$0$}
{--- Займись поэзией, --- ухмыльнулся Лу.}
{``Take poetry,'' Lu smiled.}
\ml{$0$}
{--- Ты из хорошей семьи и знаешь в поэзии толк.}
{``You're from a good family, you have an eye for poetry.}
\ml{$0$}
{Только я бы на твоём месте не ждал каких-то серьёзных успехов.}
{I wouldn't expect great discoveries, though, if I were you.''}

\ml{$0$}
{--- Почему это?}
{``Why not?''}

\ml{$0$}
{--- Когда я изучал культуру Такэсако, я был одним из немногих в Ордене, кто вообще обращал внимание на такие мелочи.}
{``When I researched Takesako culture, I was one of the few in the Order who paid attention to such a trivialities.}
\ml{$0$}
{Одинокий дилетант.}
{The lone amateur.}
А сейчас всем этим занимаются учёные бывшего Ордена Тысячи Башен, которые собаку съели на подобных исследованиях.
В изучении культуры людей они по-прежнему на десять шагов впереди.

--- Зато мы опережали их в военном отношении, --- пожал плечами Тахиро.

\ml{$0$}
{--- В этом и разница между нами.}
{``That's the difference between us and them.}
\ml{$0$}
{Они изучали людей и умели с ними как-то уживаться.}
{They studied humans, they have learned to co-exist with humans.}
\ml{$0$}
{Мы вместо этого развивали военную мощь.}
{We have amassed military power instead of that.''}

\ml{$0$}
{--- И они проиграли.}
{``They lost.}
\ml{$0$}
{Мы проглотили Тысячу Башен, как рыба глотает жука.}
{We swallowed Thousand Towers like a fish swallows a bug.''}

\ml{$0$}
{--- Да что ты? --- хихикнул Люцифер.}
{``Really?'' Lucifer laughed.}

\ml{$0$}
{--- Скорее как наживку, --- улыбнулся Грисвольд.}
{``Like a bait, I'd say,'' Griswold smiled.}

\ml{$0$}
{--- Вот и у меня создалось такое впечатление.}
{``I got the same impression.}
\ml{$0$}
{Все ключевые посты, за исключением армейских, занимают кто?}
{All key positions, except military, occupied who?''}

\ml{$0$}
{--- И, что самое главное, заслуженно занимают, --- присовокупил Грисвольд.}
{``More important, justly occupied,'' Griswold added.}
--- Сейчас мне хоть поговорить есть с кем, кроме вас.
Идёшь по коридору --- а на скамейке сидят трое-четверо центурионов и рассуждают о чем-то безумно интересном.
Так и хочется подсесть и просто поговорить...

--- А на чём тренировался Гало, когда был маленьким? --- поинтересовался Тахиро у Люцифера.

--- Гало был серьёзный парень и начал сразу с наших военных затрат.
Цены на материалы, оружие, оптимальный уровень взятки, гарантирующий лояльность осведомителя.
Он очень сильно помог отцу буквально в первый же год обучения, почему отец и выделил его в любимчики.
Я же ещё пару лет валял дурака, исследуя геологию, метеорологические данные, культуру жителей, ну и всякую живность.
Всего понемногу, одним словом.

--- Мы, конечно, напортачили, сразу прикрутив Тахиро такие сложные стратегические модули, --- задумчиво сказал Грисвольд.
--- Оцифрованные легионеры уже в строю, а вот тебе ещё долго-долго тренироваться и ждать, пока интеграция будет окончательно завершена.
Но сам факт того, что ты не сошёл с ума после процедуры, многообещающий.

--- Ну и то, что Тахиро не пустили под нож, тоже внушает надежду, --- захихикал Лу.
--- Правда, отец сказал, что ты не получишь никаких обновлений по своей части, пока не достигнешь хотя бы ранга центуриона.
К счастью, исполнять приказ не ему, а демонам, которые занимаются моей веткой обновлений, --- Лу подмигнул Грисвольду.

--- Это было самое ужасное, что я чувствовал в жизни, --- Тахиро внезапно передёрнуло так, что он чуть не расплескал вино.
--- Если бы рядом были не вы, я бы точно сошёл с ума.
Интересно, как вы это пережили.

--- У нас с Лу не было такого опыта, --- пожал плечами Грис.
--- Мы изначально были демонами и просто встраивались в человеческие тела.
Это совсем другое.

--- А что я буду чувствовать, когда умру? --- спросил Тахиро.
--- Имеется в виду, когда тело придёт в негодность.

Люцифер и Грисвольд смущённо переглянулись.

--- Мы не знаем, --- признался технолог.
--- Чисто технически омега-модуль дублирует твою нервную систему и интегрирован в неё, так что после смерти ты просто не будешь чувствовать тело.
Такое часто бывает во сне, знаешь --- ни рук, ни ног, ни дыхания.
Омега-чувство, напоминающее зрительные ощущения, разумеется, останется.
Поэтому понять, что означают <<видения>>, в твоих интересах.

--- Но это теория, --- закончил Лу.
--- Прецедентов, как понимаешь, ещё не было.
Ты будешь первым, кто расскажет, как всё обстоит на самом деле.

--- Я больше не хочу быть первым!

Лу и Грис расхохотались.

--- Ну, могу тебя понять.
Да, у нас же теперь есть куча оцифрованных легионеров!
Надо кого-нибудь из них кокнуть ради интереса.

--- Люцифер! --- укоризненно воскликнул Грисвольд.

--- Да, ты прав.
Подождём, пока кто-нибудь помрёт сам.
Это мудро и экономично.

--- Ну что ж, у меня будет ещё один важный этап взросления, --- подытожил Тахиро.
--- Выпьем за это.

--- Да, за твою первую смерть.

Лу и Тахиро по очереди звякнули бокалами о стоящий бокал технолога.

--- Штука это не особо приятная, --- поделился Грисвольд.
--- Особенно когда тебе в печень втыкают копьё.
Так что за то, чтобы она прошла по плану в техническом смысле.

--- Тебе в печень воткнули копьё? --- с интересом спросил Тахиро.

--- После того, как я полдня провисел на солнцепёке голышом, съедаемый мухами и слепнями, --- кивнул технолог.

--- Какая крутая смерть, --- восхитился Лу.

--- А что ты сделал? --- спросил Тахиро.

--- У меня с ними возникли разногласия на почве религии.
Они не поверили, что я их бог.

--- Где-то я это уже слышал, --- пробормотал Тахиро.
--- Кажется, на Тысяче Башен есть племя, которое верит в повешенного на кресте бога.
Мне рассказывали, что там есть очень красивая друза Хербст.
Она вся поросла деревьями, листья которых делаются оранжевыми, как лисы, каждый сезон.
Хотел бы я там побывать.

--- Ты еще на моей планете не побывал, --- напомнил Грис.

--- Кстати, о твоей планете, -- оживился Люцифер.
--- Ты слышал, да?
Вообще это военная тайна, но скауты уже всем уши прожужжали.
Мы нашли Лотос.
Не по твоим координатам, они оказались, мягко говоря, неточными, а по описанию.
Луна-кристалл, три материка, лиловая пустыня, горы из платины тоже видели.
Тысяча Башен нашла.

--- А Союз?

--- Они пока не в курсе, судя по всему.
Кстати, ты всё говорил про звезду Фомальгаут.
По этому названию один из скаутов аннотировал старые звездные карты, прибывшие на Тысячу Башен с первыми поселенцами.
Они обнаружили ещё четыре обитаемые планеты.
Две, к сожалению, уже заняты Союзом.
А одна из оставшихся...

Люцифер вдруг посерьезнел.
Падающий на его лицо тусклый свет состарил кукольные черты.
Грисвольд понял.

--- Древняя Земля?

--- То, что от неё осталось, --- глухо ответил Лу.
--- Людей на ней нет, очертания материков... кхм...
В общем, мало похоже на то, что мы ожидали, но это точно она.
Ледниковый период прошёл, и она всё ещё пригодна для жизни.
Кстати, о жизни...

Лу осушил бокал и, подтянувшись к Тахиро, нежно поцеловал его в губы и запустил пальцы в его волосы.

--- Чем займёмся?
У нас ещё три часа.

--- Уговорил, --- Тахиро начал развязывать Лу пояс.

--- Да тебя и уговаривать не надо.
Хочешь, позовём Айну?

--- Не сегодня.

--- Я слышу это в двадцатый раз.
Мне кажется, ты её избегаешь.
Она, между прочим, о тебе постоянно говорит.
Недавно вспомнила, как ты пообещал взять её в жёны.

--- Я был молод и глуп.

--- Это из-за той такэсакской красавицы, которую она убила?
Она это из ревности, точно тебе говорю.

--- Лу, давай к делу.
То, что мы с тобой спим и дружим, ещё не даёт тебе право лезть ко мне в душу.

--- Когда вы успели? --- удивился Грисвольд.
--- Лу, ты зачем парня совратил?

--- Да это я его, --- хихикнул Тахиро.
--- Мы же с Лу тогда спали в одной комнате.
Просыпаюсь я как-то утром и вижу его в очень живописной позе.
Дальше всё как в тумане.

--- То есть это моя вина, --- ухмыльнулся технолог.

--- Я бы перефразировал.
Это благодаря тебе.

--- Тогда я, пожалуй, пойду, --- Грисвольд отхлебнул ещё глоток из своего бокала и грузно поднялся на ноги.

--- Ну да, ничего особо интеллектуального сейчас не будет, --- признал Лу.
--- Но ты можешь присоединиться.

--- Нет, благодарю, --- ухмыльнулся технолог.
--- У меня паёк --- одна женщина в два года.

--- Презренный моносексуал, --- поморщился Тахиро.

--- Тахиро, вот скажи, зачем мы вообще дружим с этими низшими созданиями? --- развёл руками Лу.

--- Ещё слово --- и вы оба останетесь без обновлений, --- посулился Грисвольд.

--- А вот это был удар ниже пояса! --- возмутился Лу.

Грисвольд скорчил рожу и вышел, прикрыв за собой дверь.

\section{[:] Чужестранка}

\ml{$0$}
{--- Я не буду с тобой спать.}
{``I shan't sleep with you.''}

\ml{$0$}
{--- Простите? --- опешил Люцифер.}
{``Beg your pardon?'' Lucifer said, taken aback.}

Демоница бросила взгляд на стопку папок в углу стола.
По крайней мере, так показалось Люциферу --- её сильное косоглазие не позволяло утверждать наверняка.
Люцифер схватил верхнюю папку и открыл.

\ml{$0$}
{--- Это досье на меня.}
{``This is a dossier on me.''}

Демоница выхватила папку из его рук и положила на место.

\ml{$0$}
{--- Это. Досье. На меня, --- повторил Лу.}
{``This is. A dossier. On me,'' Lu repeated.}

\ml{$0$}
{--- И?}
{``So what?''}

Люцифер задумался.
Факт существования досье на него самого ошеломил стратега настолько, что он вдруг забыл, зачем его вообще занесло в этот забытый небом кабинет в дальнем конце южного крыла.

Секс с Тахиро был средством на время забыть о собственных тревогах, и друг это прекрасно понимал;
он получал то, что было нужно ему, и знал, что Лу делает то же самое.
Обычно объятий друга хватало на несколько часов умиротворения, но в этот раз тревоги вернулись, едва Тахиро заснул.
Люцифер бережно укрыл его одеялом, потушил лампу и тихо вышел за дверь, намереваясь поработать.

\ml{$0$}
{Женщину, которая сидела перед ним в старом кресле, звали Анастейша Розье.}
{The woman, sitting in an old chair in front of him, was Anasteishae Rozier.}
Странное имя, тяжёлое и громоздкое, словно аппаратура для демонизации, дань дурацкой, позаимствованной у дзайку-мару традиции давать несколько имён.
\ml{$0$}
{Сама она предпочитала уменьшительно-ласкательный вариант --- Стигма.}
{She herself preferred a diminutive form: Stigma.}
Позывной встречался в очень интересных местах архивных документов Тысячи Башен, и следовало...

Люцифер бросил взгляд на стопку досье, и его вдруг осенило.
Эти досье выглядели в каморке кабинетного учёного так же неуместно, как...
Мозг Лу даже не стал тратить время на поиск подходящего сравнения.

--- Это ты была главой разведки Ордена Тысячи Башен, --- заявил он.
--- Настоящим, а не церемониальным руководителем.

Стигма вздрогнула.

--- Не бойся, это уже в прошлом, --- поспешил добавить Лу.
--- Вообще да, я тоже собирал на тебя досье и меня как-то зацепило, я захотел с тобой познакомиться поближе.
И пришёл именно поэтому.
И ты поняла это потому, что собрала досье на меня.

--- Ты всех, чьё досье тебя зацепило, пытаешься затащить в постель?

--- Именно сейчас я не настроен на постель, --- признался Лу.
--- Если ты про мои методы...
Да, мне просто хочется узнать демона поближе, а секс --- самый простой способ.
Лично для меня, разумеется.

Стигма промолчала и отвернулась.
Люцифер вспрыгнул за стол и уселся перед женщиной, скрестив ноги.
Стигма зябко поёжилась и отвернулась ещё на пол-оборота;
на её лице застыла смесь страха и отвращения.
Лу, поморщившись, аккуратно слез со стола и поправил лежащие на нём вещи.

--- Мы тут нашли вашу служебную переписку, --- сказал он.
\ml{$0$}
{--- В ней ты единственная утверждала, что потери Преисподней --- не что иное, как обманный ход, и что у Ордена достаточно сил, чтобы одолеть Союз на Преисподней.}
{``You were the one who told: military losses of Nether are nothing but ruse, and the Order is strong enough to defeat the Union on Nether ground.}
\ml{$0$}
{Ты настаивала на том, чтобы Тысяча Башен выступила на стороне Преисподней как можно скорее, дабы, цитата, <<избежать последствий нейтралитета>>.}
{You insisted that Thousand Towers must join Nether as soon as possible, I quote, `to avoid the consequences of neutrality'.}
\ml{$0$}
{Впечатляет.}
{Very impressive.}
\ml{$0$}
{Почему тебе не поверили?}
{Why they didn't believe you?''}

\ml{$0$}
{--- Я обязана отвечать на твои вопросы?}
{``Do I have to answer your questions?''}

\ml{$0$}
{--- Разумеется, нет.}
{``Of course you don't.}
\ml{$0$}
{Но если я задаю вопрос, значит, ответ для меня важен и я в любом случае попытаюсь его найти.}
{Nevertheless, if I ask, an answer is important to me and I shall try to find it anyway.}
\ml{$0$}
{Так всё-таки --- почему твоё слово оказалось недостаточно веским?}
{Returning to the subject, why your word wasn't good enough?''}

\ml{$0$}
{--- Я чужая, --- осторожно сказала Стигма.}
{``I'm an outlander,'' Stigma cautiously answered.}
\ml{$0$}
{--- Тысяча Башен --- планета традиций.}
{``Thousand Towers is a planet for tradition.''}

--- Тысяча Башен куда менее традиционна, чем Преисподняя, --- сказал Лу.
--- Вы абсолютно неразборчивы, когда дело касается новобранцев.
Многие из этих новобранцев потом занимают высокие посты, и их слово становится законом.

--- Я чужая даже для них.

--- Тысяча Башен принимает в свои ряды дезертиров, визоров-рабов, богов, преступников разного пошиба и перебежчиков, успевших поменять стороны по десять раз.
\ml{$0$}
{С трудом представляю, что может быть хуже.}
{I barely can imagine something worse than that.}
Разве что трудности в общении.
\ml{$0$}
{Может быть, ты бездомный бог?}
{Maybe you're a homeless god?''}

Стигма снова промолчала.

\ml{$0$}
{--- Думаю, нет, --- ответил Лу на собственный вопрос.}
{``Apparently not,'' Lu answered his own question.}
\ml{$0$}
{--- Твоё досье мы не нашли, но это очевидно.}
{``Your dossier wasn't found, but it's obvious.}
Тем не менее твои паттерны мне не знакомы совершенно, они не подходят ни под один шаблон известных объединений демонов.
\ml{$0$}
{Ты действительно другая.}
{You're definitely different.''}

Молчание.

--- Хотя... --- Лу прищурился, --- может быть, Чёрная Скала?
Не те неприятные головорезы, говорящие на булькающем языке, а нечто более тонкое и изящное.
Даже головорезам нужен тот, кто ими управляет.
И не всегда он это делает по своей воле, ведь ситуация, когда лидер является заложником подчинённых --- не фантазия, а суровая реальность.
Это объясняет, почему тебе не поверили --- у тебя есть личная заинтересованность в том, чтобы Тысяча Башен и Союз Воронёной Стали не подружились.
Вдобавок это хорошее объяснение и для того, что ты разобралась с Союзом гораздо быстрее нас.
Разумеется, ты знала всё о бывших соратниках, клане Антрацис.
\ml{$0$}
{То есть, иными словами, ты --- Вечно Го...}
{In other words, you are a Forever Ha---''}

\ml{$0$}
{--- Заткни пасть! --- взорвалась Стигма.}
{``Shut your mouth!'' Stigma exploded.}
\ml{$0$}
{--- Не произноси при мне эти слова!}
{``Never say that in my presence!''}

--- Не буду, --- тут же поправился Лу.
\ml{$0$}
{--- Прости.}
{``Sorry.''}

\ml{$0$}
{--- <<Прости>>?}
{`` `Sorry'?}
\ml{$0$}
{Чего ты добиваешься?}
{What's your angle?}
\ml{$0$}
{Я не нахожусь под арестом, меня ни в чём не подозревают.}
{I'm not under arrest, I'm not suspected or somethin'.}
\ml{$0$}
{Что тебе нужно, кокэси-нингё\FM?!}
{What do you want, \textit{kokeshi ningyo}\FM!''}
\FA{Кокэси-нингё --- традиционные пластиковые куклы на Преисподней.
В переносном смысле: уничижительное название близнецов или внешне похожих людей (например, одетых в одинаковую форму солдат).
}

\ml{$0$}
{--- Кокэси-нингё? --- Лу захохотал.}
{``\textit{Kokeshi ningyo}?'' Lu laughed.}
\ml{$0$}
{--- Меня так в шутку называли деревенские в детстве, из-за того, что у меня был брат-близнец.}
{``Villagers jokingly called me that in my childhood, because of my twin brother.}
\ml{$0$}
{Откуда ты это знаешь?}
{How do you know?''}

Стигма промолчала, глядя в пол.
Люцифер сел на пол перед её стулом, пытаясь заглянуть ей в глаза.

--- Я всё больше убеждаюсь, что зашёл сюда не зря.
\ml{$0$}
{Ты другая, и ты мне очень интересна.}
{You're different, and I'm interested in you.}
\ml{$0$}
{Интересна в хорошем смысле.}
{In a good way.}
\ml{$0$}
{И я тебе интересен, верно?}
{And you're interested in me, right?}
\ml{$0$}
{Да, до последнего времени я интересовал тебя исключительно как вероятная цель для убийства, но времена меняются.}
{Well, until recently, I've interested you strictly as an assassination target, but things have changed.''}

--- Ты не привык, что тебе отказывают? --- буркнула Стигма.
--- Не привык, что тебя выставляют за дверь, что с тобой не хотят разговаривать?

--- Для меня это в новинку, --- признался Люцифер.
--- В Ордене надо мной смеялись за моей спиной, но отклонить моё предложение побеседовать обычно боялись, потому что демоны моего ранга редко приходят просто так.
Я не хотел тебя обидеть или оскорбить.
Как бы глупо это ни звучало, я хотел с тобой прогуляться.

--- Я не буду с тобой спать, Люцифер.

\ml{$0$}
{--- Я это уже уяснил.}
{``You've made that clear enough.}
\ml{$0$}
{Не вижу никакого смысла заниматься сексом с тем, кто этого не хочет --- это оскорбляет саму идею.}
{I don't see the point of sex with someone who doesn't want it---it offends the very idea.}
\ml{$0$}
{На самом деле ты можешь хранить свою девственность столько, сколько пожелаешь, Самаэл выпишет официальное разрешение.}
{Actually, you can keep your virginity as long as you want, Samael will issue official permission.}
Но всё-таки --- как насчёт небольшой прогулки?

\ml{$0$}
{--- Это трата времени.}
{``It's a waste of time.}
\ml{$0$}
{И оставь при себе свои корявые шутки.}
{And keep your ugly jokes to yourself.''}

\ml{$0$}
{--- Я оскорблён в лучших чувствах.}
{``I feel deeply offended.''}

\ml{$0$}
{--- И это тоже трата времени.}
{``And that's a waste of time too.''}

Люцифер прищурился.

\ml{$0$}
{--- Откуда в тебе такой страх перед сексом?}
{``Where is your fear of sex from?''}

Стигма вздрогнула так, что её кресло ударилось о стол.

\ml{$0$}
{--- Меня никогда не интересовал секс, --- заявила она.}
{``I was never interested in sex,'' she said.}

\ml{$0$}
{--- Возможно, но это не объясняет страх.}
{``It can be true, but it can't explain the fear.''}

Стигма промолчала и снова зябко поёжилась.

--- Хотя в принципе это тоже понятно, --- задумался Люцифер.
--- Твои булькающие дружки не особо церемонятся с дзайку-мару.
\ml{$0$}
{Сомневаюсь, что они насиловали тебя, но то, что это регулярно делали \emph{при тебе}, не вызывает сомнений.}
{I doubt they raped you, but I have no doubt they did it regularly \emph{in front of you}.}
\ml{$0$}
{Сложно увидеть что-то хорошее в том, что всегда было для тебя наказанием.}
{It's difficult to see something good in a thing you used to treat as a punishment.}
Даже если ты и совладала с этим в прошлой жизни, твои демонические воспоминания, как ведро снега, вываливаются на новое тело, заставляя его вздрагивать в ужасе.
У нас есть лаборатория, которая занимается относительно новым направлением --- коррекцией личности демона.
Если тебя беспокоят какие-то старые травмы, ты всегда можешь обратиться к их руководителю, Цельсии.
Я позабочусь о том, чтобы в твоём случае была соблюдена полная конфиденциальность.

--- У тебя всё? --- сухо осведомилась Стигма.

--- Ещё нет.
У меня есть камень в рукаве.

--- Сочувствую.
Иди и замени одежду.

Люцифер вместо ответа вытащил из кармана мышь и положил зверька Стигме на стол, придерживая за хвостик.

--- И что это?

--- Это мышь, --- просветил женщину Люцифер.
--- Тебе не нравится?

--- Нравится, но зачем ты притащил её ко мне?

Лицо Люцифера озарила довольная улыбка.

--- Тебе нравится.

--- И что?

--- Ты сказала, что тебе нравится.

--- Да, она мне нравится.
Дальше что?

--- Найди ей домик.
А если дать ей самца, пищу и время, у тебя будет целая куча мышей.
Ты получишь множество мутантов --- всех, какие только могут быть.
\ml{$0$}
{Прямо как первобытные биологи.}
{Like prehistoric biologists used to do.}
\ml{$0$}
{Самца я принесу.}
{I'll bring you a male.}
\ml{$0$}
{Держи.}
{Take this.''}

Люцифер всучил мышь Стигме и направился к выходу.

\ml{$0$}
{--- Эй, ты куда?}
{``Jey, where're you goin'?''}

\ml{$0$}
{--- За самцом.}
{``Look for a male.''}

\ml{$0$}
{--- Ты с ума сошёл?!}
{``Are you crazy!''}

\ml{$0$}
{--- С ума сходить будешь ты, когда получишь первого альбиноса.}
{``It's you who will go crazy when you've got your first albino.''}

\ml{$0$}
{--- Пожалуйста, не надо.}
{``Please, stop now.}
\ml{$0$}
{Не нужен мне самец.}
{I don't need a male.''}

\ml{$0$}
{--- Ты меня умоляешь?}
{``Are you begging me?}
\ml{$0$}
{А чего так?}
{Why?}
Боишься, что тебе понравится ещё больше?

\ml{$0$}
{--- Я тебя не умоляю, сумасшедший!}
{``I'm not beggin' you, you freak!}
Ты врываешься ко мне в кабинет, и...

\ml{$0$}
{--- Если тебе не нужна мышь, просто выпусти её.}
{``If you don't need a mouse, just release it.}
\ml{$0$}
{Она дикая и прекрасно обойдётся без тебя.}
{It's wild and perfectly capable to survive without you.}
\ml{$0$}
{Или раздави ей голову и выброси в мусорку.}
{Or you can crush its head and throw it out.}
\ml{$0$}
{Мой отец сделал именно так, когда я принёс ему мышку.}
{Like my father did when I bring a mouse to him.}
\ml{$0$}
{А твоё <<пожалуйста, не надо>> звучит неубедительно.}
{As for your `please, stop now'---it sounds very unconvincing.''}

Люцифер выразительно сблизил большой и указательный пальцы, затем вышел, хлопнув дверью.
Стигма громко выругалась ему вслед, вздохнула и, посмотрев на мышь, принялась искать в углу подходящую коробку.

\section{[:] Рога и плащ}

Оцифровка людей встала на поток и уже принесла первые плоды.
Отряд, собранный полностью из оцифрованных, одержал победу над превосходящими силами Союза.
Впервые за несколько лет Союз предложил Ордену переговоры, и впервые за десятилетия войны была уверенность, что это не обманный манёвр с целью оттянуть время.

Авторитет Лу резко возрос.
Больше никто из легатов, и тем более легионеров, над ним не смеялся.
Его речи слушали внимательно и приказы исполняли беспрекословно.
Сам Лу в честь таких перемен решил сменить имидж.

--- Ну как тебе?

--- Очень красиво, --- признал Тахиро.
\ml{$0$}
{--- Особенно этот ободок с рожками.}
{``Especially that horned bezel on your head.}
\ml{$0$}
{Кстати, почему рога?}
{By the way, why horns?''}

\ml{$0$}
{--- Мне нравятся рога.}
{``I like horns.}
\ml{$0$}
{Они милые.}
{Nice touch.}
\ml{$0$}
{Но меня напрягает твоё молчание по поводу плаща.}
{But I feel bad about your silence concerning my cloak.''}

--- Плащ тоже красивый.

--- <<Тоже>>?
Тахиро, вот сгори ты дотла!
\ml{$0$}
{Скажи, что я в нём смотрюсь очень элегантно, эпично, ну придумай что-нибудь!}
{Tell me I'm elegant, and epic, and anything you can make up!''}

\ml{$0$}
{--- Честно --- он похож на плед.}
{``Honestly, it looks like blanket.''}

\ml{$0$}
{--- Да вы издеваетесь.}
{``You're mocking me.''}

\ml{$0$}
{--- Почему?}
{``Why would I?''}

\ml{$0$}
{--- Грис то же самое сказал.}
{``Gris said the same word.}
Вообще-то это воссозданная технология первых людей, наноткань с очень низкой теплопроводностью.
Плащ делал лучший мастер по моим наброскам.
\ml{$0$}
{Ну серьёзно, я не выгляжу в нём элегантно и эпично?}
{Come on, don't I look elegant and epic?''}

\ml{$0$}
{--- Ты выглядишь очень элегантно и эпично.}
{``You do look elegant and epic.}
\ml{$0$}
{Плащ очень красивый.}
{The cloak is really beautiful.}
\ml{$0$}
{Но это не отменяет того, что он очень похож на плед.}
{But it does not change the fact the cloak looks like blanket.''}

Лу порылся в сундучке, вытащил посеребрённую фибулу и подколол ею плащ.

\ml{$0$}
{--- А так?}
{``Better?''}

\ml{$0$}
{--- Тебе нравится?}
{``Do you like that?''}

\ml{$0$}
{--- Иначе я бы его не надел.}
{``I wouldn't wear it if I don't.''}

\ml{$0$}
{--- Ты выглядишь элегантно и эпично?}
{``Do you look elegant and epic?''}

\ml{$0$}
{--- По моему мнению --- да.}
{``In my opinion, yes.}
\ml{$0$}
{А ещё мне мягко и тепло.}
{Also, I feel soft and warm.''}

\ml{$0$}
{--- Тогда зачем тебе нужно моё мнение?}
{``So why do you ask my opinion?''}

Лу подумал, пробормотал что-то под нос и вышел.

\section{[:] Цена молчания}

--- Привет, сайгон, --- кивнула Айну.
--- Красивый плащ.
Издалека я подумала, что ты совсем решил подтереться дисциплиной и пришёл на переговоры в одеяле.

Лу показал ей неприличный жест и пошёл в другую сторону зала, подальше от командования штаба.
Возле стены робко примостилась Стигма, чуть поодаль от своего отдела тактической технологии.
Кресло рядом с ней пустовало.

\ml{$0$}
{--- Привет, --- улыбнулся ей Люцифер.}
{``Hi,'' Lucifer smiled to her.}
\ml{$0$}
{--- Ты не возражаешь?}
{``Would you mind?''}

\ml{$0$}
{--- А моё возражение как-то повлияет на результат? --- осведомилась Стигма.}
{``Would anythin' change if I mind?'' Stigma asked.}

\ml{$0$}
{--- Если ты не хочешь, чтобы я с тобой сидел --- я поищу другое место.}
{``If you don't want me close, of course I'll look for another seat.''}

Стигма устало взлохматила рукой волосы.

\ml{$0$}
{--- Садись.}
{``Sit down.}
\ml{$0$}
{Только больше никаких разговоров про прогулки и секс, ясно?}
{But no more `walk', and no more `sex', agreed?}
\ml{$0$}
{И вообще я не хочу с тобой разговаривать, Люцифер.}
{In fact, don't even talk to me at all, Lucifer.}
\ml{$0$}
{Я устаю от твоей болтовни.}
{I'm get tired of listenin' to your nonsense.''}

\ml{$0$}
{--- Договорились.}
{``Agreed.}
\ml{$0$}
{Место в обмен на молчание.}
{I have the seat, you have my silence.''}

\ml{$0$}
{--- И трогать меня ты тоже не будешь.}
{``And no more touchin' me.}
\ml{$0$}
{И смотреть на меня.}
{No more lookin' at me.}
\ml{$0$}
{И совершать иные демонстративные действия, чтобы привлечь моё внимание.}
{No more doin' any of your intemperate stuff to get my attention.}
\ml{$0$}
{Меня здесь нет.}
{I'm not here anymore.''}

\ml{$0$}
{--- Как скажешь, Стигма.}
{``As you wish, Stigma.''}

Посланцы Союза опаздывали.
Лу уже начал скучать, но пока добросовестно держался и не пытался заговорить со Стигмой.
Терпение почти подошло к концу, когда двери зала распахнулись и вошли двое, держась за руки --- мужчина и женщина.

Пришельцы выглядели абсолютно обычно.
И тем не менее зал внезапно прижух, словно осенний лист.
Сидящие начали ёжиться и вертеться, словно им всем одновременно стало неудобно сидеть.
Но вся эта возня не могла сравниться с реакцией Стигмы.
Она всхлипнула и тут же зажала рот обеими руками.

--- Старые знакомые? --- ухмыльнулся Лу.

--- Двойняшки, --- выдавила Стигма.
--- Нуива и Фуси с Жёлтого моря, ассассины Союза.

--- Твоё прежнее имя, указанное в одном из отчётов --- Хуанхай Цзаошан, --- припомнил Лу.
--- Цзаошан с Жёлтого моря.
Ты --- их сестра?
То есть, другими словами, вы из одной клановой серии, были собраны на основе одной и той же версии программного ядра?

--- Ты обещал молчать, --- ледяным тоном ответила Стигма, справившись с собой.
Люцифер виновато кивнул и завернулся в плащ.

\textspace

--- Есть ещё одно условие, --- сказал Фуси.
--- Верните нам предателей из числа Чёрной Скалы.

Двойняшки как по команде уставились немигающими взглядами на Стигму, сидевшую рядом с Люцифером.
Ну лице женщины застыл неописуемый ужас.

Легаты Ордена переглянулись.

--- Вы располагаете какими-то данными? --- уточнил Самаэл.

--- Располагаем, --- кивнул Фуси.
--- Речь идёт о единицах, а именно --- о троих демонах.
Насколько нам известно, они не занимают высоких постов.
Это рядовые легионеры и научные сотрудники, их потеря никак не скажется на Ордене Преисподней.
Выдайте их нам, и переговоры можно считать завершёнными.

--- Полагаю, чтобы они рассказали то, что успели увидеть и услышать в Ордене, --- заметил Гало.

--- Нет, --- лаконично ответил Фуси.
--- По возвращении домой они будут немедленно убиты.

--- Свежо предание, --- фыркнул стратег.

--- Ваше недоверие вполне объяснимо, --- оскалилась Нуива.
--- В таком случае могу предложить казнь при участии наблюдателей с обеих сторон.

Гало оглядел легатов и пожал плечами.
Самаэл натянул вежливую полуулыбку, кивнул и уже открыл рот...

--- Я не понимаю, о чём вы говорите, --- опередил его Люцифер.
Все обернулись к нему.

--- Мы говорим о тех Вечно Гонимых, которые были укрыты Орденом Тысячи Башен, --- прохрипела Нуива.

--- В Ордене Преисподней нет Вечно Гонимых, --- заявил Лу и встал, словно ненароком набросив свой плащ на Стигму.
--- В Ордене Преисподней нет демонов Ордена Тысячи Башен.
В Ордене Преисподней нет предателей, в Ордене Преисподней есть только адепты Ордена Преисподней.
Поэтому я не понимаю, о чём идёт речь.

--- Подумайте хорошо, --- Фуси понизил голос до хриплого баса.
--- Это важное условие для дальнейшего поддержания дипломатических отношений.

--- Во-первых, важные условия озвучиваются в начале, а не в конце переговоров, --- Лу театрально помахал пальцами.
--- Во-вторых, важные условия заносятся в тезисы переговоров.
В-третьих, я не вижу смысла обсуждать условия, которые не могу понять.

<<Сын, что ты делаешь?>> --- требовательно спросил Арракис.

<<Люцифер! --- возмутилась Айну.
--- Где твоя дипломатичность?>>

--- Я полагаю, что мой брат выразился абсолютно корректно, --- сказала Нуива.
--- Разумеется, выдача будет производиться на условиях, выгодных для Ордена Преисподней.
Мы готовы произвести её даже на очень выгодных для вас условиях --- скажем, выдать десять перебежчиков Ордена в обмен на одного перебежчика Чёрной Скалы.
Разумеется, с поправкой на ранг.

Айну нахмурилась.

--- Почему вы готовы пойти на такой неравноценный обмен? --- спросила она.

--- Чёрная Скала блюдёт чистоту своих рядов, --- сказал Фуси.
--- И ради этого мы готовы пойти на определённые жертвы.

--- Это очень похвально, --- буркнул Лу.
--- Но, если я вас понял правильно, вы предлагаете обменять троих адептов Ордена --- безусловно полезных, вне зависимости от их ранга --- на несколько десятков совершенно бесполезных перебежчиков, которые более не желают нам служить.
В чём здесь выгода для нас?

--- Разве Орден Преисподней не заинтересован блюсти чистоту своих рядов?

--- Они уже сбежали, --- пожал плечами Лу.
\ml{$0$}
{--- Дурак с повозки --- ослу веселее.}
{``Fool off the cart, donkey feels better.''}

--- Разве Орден Преисподней не карает за предательство?

--- Карает, если это не идёт во вред Ордену.

Легаты переглянулись;
по залу пробежал шёпот.

<<Он ходит по очень тонкому льду, --- подумал Грисвольд.
--- В его словах есть истина, но если бы их вслух сказал я --- мой ранг упал бы до уровня моря>>.

--- Основным приоритетом Союза Воронёной Стали является неотвратимость наказания, --- сказала Нуива.
--- Преступник должен знать, что он будет наказан --- сегодня, завтра, через тысячу или через миллион лет, здесь или где-то ещё.

--- Основным приоритетом Ордена Преисподней является защита адептов, --- ответил Лу.
--- В том числе и от союзных организаций, в том числе и от других адептов.
Именно способность защитить членов делает организацию организацией.

--- Разумеется, мы вполне понимаем ваши мотивы, --- поклонился Самаэл.
--- Вероятно, сайгон Люцифер хотел сказать, что...

--- Люцифер сказал именно то, что хотел сказать, --- подняла руку Айну.
Самаэл тут же замолчал, словно ему в рот засунули рыбину.
--- Чем дальше вы вдаётесь в подробности, тем меньше мне нравится идея такого обмена.
Вы предлагаете обменивать только членов своего клана или, возможно, нескольких кланов --- я не в курсе деталей.
На кого?
На наёмников или отбросов, которые не нужны даже вам?
К дьяволу такой обмен.
На перебежчиков из других кланов?
Я бы послушала мнение средних чинов Союза по поводу такой интересной торговли за чужой счёт.
На тех, кто уже служит вам?
А чем они заслужили такое отношение к себе?

--- Думаю, мы с вами сходимся во взглядах на то, чего заслуживают предатели.

--- От наших рук --- да.
Но неужели все они плохо служили вам?
Это самый главный вопрос --- можно ли принципом неотвратимости наказания оправдать произвол в ваших собственных рядах?

--- Мы оказываем Ордену любезность, давая возможность свершить правосудие.

--- Нельзя, --- перевёл с дипломатического языка Грисвольд.
--- Нам следует присмотреться к выходцам с Чёрной Скалы.
Возможно, среди них есть те, кто стоит и тридцати, и ста перебежчиков.

--- Я правильно понял, что вы предлагаете поднять цену? --- осведомился Фуси.

\ml{$0$}
{--- Отказ не является предложением поднять цену, --- сказал Люцифер.}
{``No doesn't offer to up the price,'' Lucifer said.}
\ml{$0$}
{--- Отказ не является требованием отсрочки.}
{``No doesn't promise deferment.}
\ml{$0$}
{Отказ не является оскорблением.}
{No doesn't try to offend.}
\ml{$0$}
{Отказ --- это просто отказ.}
{No is just no.''}

--- Что скажет глава Ордена? --- обратился Фуси к Арракису.

--- Вы только что услышали слова главы Ордена, --- ответил Арракис и тоже встал.
--- Законы Союза не распространяются и не будут распространяться на адептов Ордена Преисподней.
Переговоры окончены.

Все начали собираться.
Многих взволновали последние слова Арракиса, но, как видно, ни для кого они не стали большим сюрпризом.
Грисвольд кивнул Лу, как другу, Айну --- как старшему по званию, Самаэл --- с подобострастно-оценивающим выражением лица.
Фуси пронзил Люцифера острым взглядом --- словно поставил метку --- и, взяв за руку Нуива, удалился.

Люцифер взглянул на Стигму.
Женщина сидела, отвернувшись к стене и подперев голову рукой.
Она успела завернуться в его плащ;
по запястью струйкой текли слёзы.

--- Я никому не дам тебя в обиду, --- шепнул Люцифер.

Никто из присутствующих не обращал внимания на Гало, который продолжал сидеть на скамье.
На его лице застыл шок.

\section{[:] Игрушка}

\ml{$0$}
{--- Люцифер, я принял решение.}
{``Lucifer, I have decided.''}

\ml{$0$}
{--- Не надо было делать это перед всеми, тем более в такой щепетильный момент, как переговоры!}
{``You shouldn't have done that in front of everybody, especially at a delicate moment like negotiations!}
Стратегия завязана в том числе и на моём текущем положении в системе.
Ты сейчас сильно сдвинул баланс власти, и это будет иметь серьёзные последствия для всех.

\ml{$0$}
{--- Я полагаю, ты прекрасно с этим справишься.}
{``I suppose you're perfectly capable to handle such a situation.''}

\ml{$0$}
{<<Ты вообще слушаешь меня или нет?>>}
{\textit{Are you listening to me or not?}}

Люцифер ходил по богато обставленному кабинету Арракиса, избегая поворачиваться лицом к хозяину.
От одного окна к другому, от двери к изящной конторке, от конторки к статуе, стоящей в углу.

--- Тебе надо было посоветоваться со мной и Гало наедине и решение объявить на следующем совещании.

\ml{$0$}
{--- Я достаточно долго выжидал, выбирая между тобой и Гало, --- сказал Арракис.}
{``I've spent enough time choosing between you and Halo,'' Arrakis said.}
\ml{$0$}
{--- Надо сказать, выбор был непростым, и я до последнего верил в то, что Гало справится лучше.}
{``Should be noted, it wasn't a simple choice, till the end I believed that Halo can do better than you.}
\ml{$0$}
{Ты всегда казался мне туповатым и недостаточно дисциплинированным для руководящей роли.}
{You always seemed slow on the uptake and too undisciplined for the leadership.}
\ml{$0$}
{Но твои успехи я игнорировать не могу.}
{But I can't ignore your achievements.}
\ml{$0$}
{Тебя уважает Айну, тебя уважает Самаэл, политически индифферентные учёные высказываются о твоих методах с одобрением.}
{You are respected by Ain\"{u}, you are respected by Samael, politically apathetic scientists commend your methods.}
\ml{$0$}
{Но самое главное --- благодаря тебе Орден действительно стал единой организацией, а не коллекцией кланов.}
{But most important, because of you the Order performs as one, not anymore as a collection of clans.}
\ml{$0$}
{Это признал даже Гало.}
{Even Halo admitted that.''}

Люцифер вдруг понял, что больше он не может сдерживаться.

--- Если ты так хотел объединить Орден в одно целое, отец, зачем тебе вообще приспичило выбирать между нами? --- Люцифер обернулся.
--- Зачем?
Ты видел, насколько упала общая эффективность Ордена с тех пор, как ты начал активно противопоставлять Гало мне?

\ml{$0$}
{--- Как это относится к вопросу?}
{``How is that relevant?''}

\ml{$0$}
{--- Мы были отличной командой.}
{``We had made an excellent team.}
\ml{$0$}
{Я и Гало.}
{Halo and me.}
\ml{$0$}
{Вместе.}
{Together.}
\ml{$0$}
{Гало делал то, что было не по силам мне, я делал то, что мог.}
{Halo did things I couldn't, and I did what I was able to do.}
\ml{$0$}
{И вдруг тебе понадобилось \emph{выбирать}.}
{And then, suddenly, you got the idea of \emph{choice} into your head.}
Да, ты был настолько уверен в своей правоте, что выбор для тебя стал превыше всего остального.
Ты не считался ни с кем --- ни со мной, ни с Гало, ни с нашими мнениями, ни даже с нашими жизнями.
В результате половина ресурсов у тебя стала уходить на поддержание противостояния, а Гало бросил почти все силы против меня.
И я, именно я, остался один против четверых врагов, двоих из которых мне ещё и приходилось защищать!

Люцифер остановился, поняв, что последние слова были сказаны на пределе крика.
Горло немилосердно ныло.
Арракис холодно смотрел на сына.

\ml{$0$}
{--- Это сделало тебя тем, кто ты есть.}
{``This made you what you are now.}
\ml{$0$}
{Ты стал сильнейшим, ты стал достойным места главы Ордена.}
{You are the strongest, you are worth to be the Order's head.}
\ml{$0$}
{Разве это не моя заслуга?}
{Didn't I have something to do with that?''}

\ml{$0$}
{--- Нет, не твоя, --- бросил Лу.}
{``No, you didn't,'' Lu said.}
\ml{$0$}
{--- Это моя заслуга.}
{``The credit is all mine.}
Как и то, что Орден вообще выстоял в битве против Союза.

\ml{$0$}
{--- Пусть будет так.}
{``If you prefer.}
В любом случае Орден теперь твой.
\ml{$0$}
{Я надеюсь, что ты оставишь для старика подходящую почётную должность.}
{I hope you will keep for your old man an honorable seat that suits him.''}

--- Нет.
Я не стану главой Ордена.
Я останусь тем, кем был --- главным стратегом по обороне.

\ml{$0$}
{--- Если ты отвергнешь мой дар --- будешь разжалован в легионеры.}
{``If you refuse my gift, I will demote you to the rank of legionnaire.''}

\ml{$0$}
{--- Это уже не в твоей власти.}
{``You have no power to demote me.}
\ml{$0$}
{Орден больше не твоя игрушка.}
{The Order nevermore shall be your toy.}
\ml{$0$}
{И я больше не твоя игрушка!}
{I nevermore shall be your toy!}
\ml{$0$}
{Так что иди ты в бездну с такими дарами.}
{So fuck yourself with your gifts.''}

Люцифер развернулся и вышел, хлопнув дверью.

\section{[:] Белая лилия}

Пройдя едва с десяток шагов, Лу наткнулся на посланца-легионера.
Этот легионер был из агентурной сети Айну --- неброский, но исполнительный служака, постоянно претендующий на повышение и стабильно от него отказывающийся.

--- Сайгон, тебе просили передать, --- сказал легионер и протянул свёрток.
--- Ты забыл своё одеяло в зале переговоров.

Люцифер кивнул и принял посылку, отмечая две едва заметных печати на целлофане.
Разумеется, вещь дважды просканировали на предмет опасных сюрпризов --- вначале принявший плащ легат, а потом и центурион --- непосредственный начальник доставившего его легионера.

--- Благодарю, Ашита.
Ты свободен.

Легионер поклонился.
Лу привычным движением набросил плащ на плечи и застегнул фибулу.
Затем поморщился и выудил из-под горловины нечто, оказавшееся цветком белой лилии.

<<Какая прелесть>>.

Люцифер несколько секунд просто любовался цветком.
В ущелье Такэсако белые лилии были символом девственности и первой любви.
Молодые люди дарили их тем, перед кем они хотели бы впервые в жизни сбросить одежду.

<<Неужели она?..>>

В Стигме было что-то притягательное для Лу.
Он не совсем понимал, что это.
Её тело сложно было назвать красивым --- лёгкая неженственная полнота, скрытая под необъятным свитером, хмурое выражение лица, сильное косоглазие, короткая стрижка, руки постоянно скрещены на груди.
Её ум не вызывал сомнения, но едва ли Лу узнал бы о нём, если бы не собрал на неё досье.
Впрочем, какая разница...
Он уже готов был дать волю фантазии, но рациональная часть его останавливала все мысли в этом направлении.

В Стигме напрочь отсутствовала сексуальность.
Это была не юношеская обида, не травма, это было отсутствие как таковое.
В Стигме было меньше сексуальности, чем в ком бы то ни было из его знакомых.
Но сексуальный подтекст был неплохим способом...

Пальцы Лу пробежались по стеблю.

...отвести глаза.
Вот и сообщение, набитое иглой.

\begin{quote}
<<Есть важная информация.
Организуй встречу с глазу на глаз>>.
\end{quote} 

--- Ашита, --- окликнул Лу уходящего посланца.

--- Да, сайгон?

--- Можешь попросить ту учёную, которая передала одеяло, зайти ко мне в комнату?

--- Будет сделано, --- легионер козырнул и направился в сторону лифтов.

<<Неплохо она меня изучила>>, --- признал Лу и, сунув лилию в рот, разжевал сообщение до полной нечитаемости.

\section{[:] Родственные узы}

--- Заходи, --- Люцифер приоткрыл дверь ровно настолько, чтобы немного полненькая Стигма смогла протиснуться.
\ml{$0$}
{--- Здесь только я и Тахиро.}
{``There's only me and Tahiro in the house.}
\ml{$0$}
{Тахиро --- мой...}
{Tahiro is my---''}

\ml{$0$}
{--- Знаю.}
{``I know.}
\ml{$0$}
{Я не возражаю.}
{I don't mind.''}

Стигма натянула латексные перчатки и молниеносно прошла по комнате, открывая шкафы и проверяя коробки.
Некоторые предметы она интенсивно трясла, другие --- аккуратно переворачивала, третьи --- скребла, по четвёртым --- стучала.
Каждая вещь, побывав в её руках, вставала в точности на то место, на котором была оставлена.
Её демон <<вспыхнул>>, по комнате светлячками полетели считывающие жуки.
Люцифер понял, что стратег проверяет комнату на наличие шпионской аппаратуры, но алгоритм был ему неизвестен.

<<Это не похоже ни на наши протоколы, ни на протоколы Союза, --- восхищённо подумал он, наблюдая за её отточенными действиями.
--- Видимо, какие-то секретные военные разработки Чёрной Скалы.
Какое счастье, что она пока на моей стороне>>.

Люцифер вдруг понял, почему Стигма казалась ему такой привлекательной.
В полноватом теле, в короткой стрижке, в сильно косящих глазах царила мощь.
Это была не та разрушительная сила, которая исходила от Двойняшек, это было спокойствие непробиваемого бастиона, в котором не найдёт лазейку ни одна крыса.

<<В Ордене Тысячи Башен она делала то, что делаю сейчас я.
Ей было тяжелее всех, когда Тысяча Башен металась между Преисподней и Союзом, на неё пала вся тяжесть поражения.
Мы --- как две стороны одной монеты, которая лишь по случайности упала мною вверх.
Возможно, это единственное существо, способное меня понять>>.

Стигма пролетела перед самым носом Люцифера, обдав его неожиданно приятным дуновением.

<<Я мог бы просто с ней расслабиться, --- понял стратег.
--- То, чего я не могу себе позволить вообще ни с кем>>.

Последним было окно.
Стигма ожидаемо проверила наличие лазеров, считывающих колебания стекла, и налепила на самый центр крохотный гаситель колебаний, игнорируя уже имеющийся сбоку.
Задёрнув плотные занавески, женщина обернулась к Люциферу.

\ml{$0$}
{--- Сразу скажу, чтобы не было недопониманий.}
{``To avoid misunderstandings.}
\ml{$0$}
{Это не моя благодарность, это не моё предложение дружбы и чего-либо ещё.}
{It's not a graditude, it's not an offer of friendship, or that stuff.}
Это вопрос жизни и смерти как для меня, так и для тебя.

\ml{$0$}
{--- Я слушаю.}
{``I'm all ears.''}

\ml{$0$}
{--- Ты в курсе, что Двойняшки планируют совершить на тебя покушение?}
{``You know that Siblings are plottin' to assassinate you?''}

\ml{$0$}
{--- Это очевидно.}
{``It's obvious.''}

\ml{$0$}
{--- Это очевидно или у тебя есть данные?}
{``It's obvious, or it's proved?}
\ml{$0$}
{У меня они есть.}
{I have data.''}

\ml{$0$}
{--- На меня совершали покушения со времён существования Ордена Тысячи Башен.}
{``Since Order of Thousand Towers existed, I've been being a target of assassination attempts.}
\ml{$0$}
{К некоторым и ты приложила руку.}
{You had a hand in some of them.}
\ml{$0$}
{Ничего нового.}
{Nothing new.''}

\ml{$0$}
{--- Люцифер, это другое.}
{``Lucifer, it's different.}
\ml{$0$}
{Если за тебя берутся Двойняшки --- это совсем другое.}
{If you have Siblings goin' after you, it's extremely different stuff.''}

\ml{$0$}
{--- Я рад, что ты такого высокого мнения о братишке и сестрёнке.}
{``I'm glad you think so highly of your little brother and sister.}
\ml{$0$}
{Переходи к делу, ты явно пришла не за тем, чтобы спасать меня.}
{Get to the point, you're not here to be my saviour.''}

--- Представители Союза ушли час назад, --- сказала Стигма.
--- Гало ушёл с ними.

\ml{$0$}
{--- Что? --- выдохнул Тахиро.}
{``What?'' Tahiro silently mouthed.}
\ml{$0$}
{--- Этого не может быть.}
{``It can't be.''}

Лу просто стоял и смотрел на Стигму.
Стигма схватила его за плечи.

\ml{$0$}
{--- Я понимаю, что это звучит дико, но мы все заинтересованы в том, чтобы Орден как можно скорее взглянул истине в глаза и начал действовать.}
{``I understand how crazy it sounds, but the Order must face the truth and make a move, the sooner the better for us all.}
Люцифер, за те несколько лет, что я живу здесь, я впервые вздохнула полной грудью.
Если тебе так важно моё признание, во многом именно благодаря твоей работе я наконец чувствую себя в относительной безопасности.
\ml{$0$}
{Ты знаешь, что произойдёт со мной, если Орден падёт.}
{You know I'm doomed if the Order falls.}
\ml{$0$}
{То же самое произойдёт и с тобой.}
{You know you're doomed too.''}

--- Переходи к делу, --- потребовал Тахиро.
--- Ему нужны факты, а не твои излияния.

--- У них были краткие переговоры, после чего Гало перестал выходить на связь.
Его агентурная сеть не получала никаких приказов.
Его тело лежит в хранилище тел, пустое, но я нашла в театре действий сообщение неизвестным кодом на обочине основных информационных каналов.

--- Покажи, --- безжизненным голосом попросил Люцифер.

Стигма коснулась лбом его лба.

\begin{quote}
<<Брат, я уверен, что ты будешь великим.
Всё это время я жил чьей-то чужой жизнью.
Я думал, что руковожу, я думал, что в моих руках сосредоточена власть, но я просто исполнял чужую волю.
То, что я сделал --- это моё и только моё решение.
Меня будут презирать так же, как презирали тебя, и я постараюсь пройти тот путь, который прошёл ты.
Даже если мы встретимся на поле боя и один из нас погибнет, до самого последнего мгновения это решение будет маяком, который ведёт меня к истине.
Я знаю, что ты меня поймёшь.

Я передам Союзу содержимое архивов 14, 21 и 60, а также наши с тобой стратегические выкладки за последние 12 лет.
Желаю тебе удачи.

P.S. Красивый плащ.
У тебя всегда был хороший вкус>>.
\end{quote} 

Люцифер схватился за стену, чтобы не упасть.
Тахиро и Стигма подхватили его под руки.

--- Дружище, ты в порядке?

--- Нет, Тахиро.
Я ни на каплю, ни на атом, ни на что в этой протухшей Вселенной не в порядке.

--- Принести воды?
Может, хочешь покурить?

--- Нет.
Брось меня здесь, я пока не умираю.
Немедленно собери наших.
Скажи следующее: Гало переметнулся, нам срочно, прямо сейчас, прямо сию минуту нужен новый план --- с учётом того, что все наши данные по разведке и обороне, а также внутренние протоколы обмена информацией известны Союзу.
Оповести Грисвольда, Айну, Самаэла, ведущих научных сотрудников, легатов.
Отца не зови.
Видеть его не хочу.

Тахиро подхватил Лу на руки, пинком развернул кресло и положил безвольное тело друга на мягкие подушки.

--- Я всем написал, --- сообщил он.
--- Они будут у тебя в комнате через десять минут.

--- Ты не можешь просто взять и не пригласить отца, --- заметила Стигма.
--- Он глава Ордена.

Лу промолчал.
Тахиро понял.

--- Я скажу Айну, чтобы она отправила демонов арестовать Арракиса.

\ml{$0$}
{--- Я думаю, она уже сама поняла, что наделал отец, --- кивнул Лу.}
{``I think she figured out what father has done,'' Lu nodded.}
\ml{$0$}
{--- Если не поняла --- растолкуй ей.}
{``If she didn't yet, you must be the one to explain.}
\ml{$0$}
{И ещё... }
{And one more thing ...}
\ml{$0$}
{Если отец сбежит из-под стражи или уже сбежал --- пусть они не тратят время на поиски.}
{If father escapes---or have already escaped---there's no need to look for him.}
\ml{$0$}
{Скорее всего, он просто забаррикадируется где-нибудь в далёком мире, объявит себя богом, а демонов Преисподней --- предателями.}
{He's likely to barricade in a faraway world, declare himself its god, and charge daemons of Nether with treason.}
\ml{$0$}
{На большее он никогда не был способен.}
{He never could do better than that.''}

--- Хорошо.

--- Я вряд ли смогу председательствовать, --- деревянным голосом сказал Лу.
--- Стигма, можно тебя попросить?..

--- Мой голос не настолько значим, --- ответила стратег.
--- Давай ты начнёшь, а я поддержу, если что.

--- Поплачь, --- Тахиро сел рядом и заключил Лу в объятия.
--- Просто поплачь.
\ml{$0$}
{Люди так делают, когда им плохо, это хорошо помогает.}
{Humans do cry when they feel very bad, it really helps.}
\ml{$0$}
{Можешь даже покричать.}
{You may even scream.}
Не надо нас стесняться.
Давай, дружище, график плотный.
У тебя десять минут.

И Лу взвыл.

\chapter{[U] Светлячки}

\section{[U] Ааман (отрывки)}

Стигма по очереди взглянула на Штрой сильно косящими глазами, словно взглянули два человека --- весёлый и задумчивый.
Штрой напряглась --- самый этот взгляд вызывал в её теле тревогу.

Из-за ширмы неторопливо вышел худой человек с непробиваемо тупым лицом.
Это лицо могло ввести в заблуждение сейхмар, но не хоргета --- глаза светились умом и странным, непередаваемым фанатизмом.
Лицо человека было невыбрито, кустистая щетина испачкана в краске.
Одежда человека также была до крайности безвкусной и неопрятной;
она выглядела не то как содранное с врагов платье, не то как найденные бродягой лохмотья.
Сквозь прорехи одежды просвечивала неправдоподобно чистая безволосая кожа, обтягивающая крепкие верёвчатые мышцы и сетку вздутых вен.
Человек был сочетанием несочетаемого --- как снаружи, так и внутри.

Ааман Третий Защитник.
Оцифрованный сапиент с одной из периферических планет, принадлежащий к сильно мутировавшему виду Людей --- ко. % Qwoh
У ко было всего четыре пальца на руках, и некоторые из оцифрованных, оказываясь в <<обычных>> человеческих телах, по привычке ампутировали себе мизинец.
Ад взял на службу несколько Телохранителей из местного племени.
Они следовали своей философии --- неважно, кто попросил тебя о защите, ты обязан защищать попросившего даже ценой жизни.
Этих пятерых интерфекторов использовали на переговорах как гарантию мира --- они без колебаний пресекали любую агрессию.

<<Атрис связался со Стигмой и предупредил её насчёт меня.
Или?..>>

Штрой нахмурилась.
План придётся менять на ходу.

Приближаться к Атрису ещё раз было чревато.
От Ааман она не добьётся ничего --- они ходили в Чистилище как на работу, аналитики уже давно махнули на них рукой.
Телохранители до крайности опасны в бою, но и настолько же предсказуемы, у них есть свой запас упругости, при котором можно действовать в относительной безопасности.
Рискнуть и попробовать выжать нужные данные из рыжей стервы?
С Хореймом и прочими она могла жёстко надавить на Стигму, но этот демон... он не должен быть здесь, у Стигмы просто нет необходимых каналов, чтобы вот так запросто нанять Телохранителя!
Что, если его со Стигмой связывает нечто большее, чем простая клятва?
Они друзья, приятели, любовники?

Штрой выросла в связке конкурирующих стратегов, где успех напрямую зависел от скорости изложения идей.
Эта методика была достаточно распространённой.
Из-за этого Штрой чувствовала себя крайне некомфортно в соло, когда не было возможности поделиться догадками с прочими.
Стигма же знала, что иногда лучше даже не думать --- всегда есть вероятность, что кто-то изобрёл устройство для чтения мыслей.
Также она была осведомлена и об изъяне соперницы;
она пытливым взглядом смотрела на лицо Штрой, ожидая предательского движения губ или щёк.
Штрой это поняла.

<<Стерва>>, --- прочитала Стигма по розовым цветущим губкам.

<<Даже такую важную информацию, как осведомлённость о слежке, она не смогла держать в себе, --- подумала Стигма.
--- Вот что стоило сейчас обратить слабость в силу и ввести меня губами в заблуждение?
Типичный кабинетный стратег, настойчиво лезущий в полевую романтику>>.

\textspace

--- Зачем ты занимаешься этой ерундой?
Ты же не биолог и никогда им не была.

--- Даже у демона могут быть мечты, --- сказала Стигма.
--- Однажды я стану биологом.
И жуки будут у меня жить на более веских основаниях.

Штрой подошла к террариуму и открыла его.
Индиго-светлячки тут же разлетелись в разные стороны.

--- Ой, --- притворно прикрыла рот Штрой.
--- Прости.
Ты же сможешь их потом собрать?

--- Я всё равно собиралась их выпустить, --- улыбнулась Стигма.
--- У них есть крылышки и челюсти.
Пусть летают и грызут, что найдут.

\textspace

Штрой поднялась и подошла к Стигме вплотную.
Диковатое оцелотовое лицо пылало сексуальным жаром и яростью ущемлённого самолюбия, и сколь много огня было в девушке, столь же холодным, напрочь лишённым сексуальности, но невероятно гармоничным казалось лицо Стигмы.
Светящиеся чистым светом расчёта глаза одним своим существованием высмеивали амбиции Штрой, и подошедшие Ааман властно оттолкнули Штрой четырёхпалой ладонью, напоминая молодой демонице о своём существовании и своей роли.

<<Ааман молодец, --- отметила Стигма.
\ml{$0$}
{--- Едва ли они знает о смертельном сюрпризе в моём комбинезоне, но последствия физического контакта они просчитали совершенно правильно>>.}
{``They hardly know of the deadly surprise I've got in my suit, but they correctly predicted consequences of physical contact.''}

Штрой, похоже, тоже поняла, какого рода опасность ей грозит.
На её лице расцвела и застыла улыбка превосходства.

\ml{$0$}
{--- Хорошая попытка, Стигма.}
{``Nice try, Stijma.}
\ml{$0$}
{Как вижу, ты не любишь марать руки.}
{You don't like to get your hands dirty, I guess.''}

Стигма вместо ответа показала испачканные удобрениями ладошки.

\ml{$0$}
{--- Очень смешно, --- скривилась девушка.}
{``Very funny,'' the girl made a face.}
\ml{$0$}
{--- Разумеется, я сообщу куда следует о твоих методах ведения переговоров.}
{``Of course I'll report about your negotiation techniques.''}

--- Ты хочешь наказать меня за желание защититься?

--- Ты уже давно перешла границы защиты.

Стигма позволила себе немного искренности:

\ml{$0$}
{--- Это ты пришла ко мне, а не наоборот.}
{``You came to me, not the other way around.}
\ml{$0$}
{Я не хочу делать тебе больно.}
{I don't want to hurt you.''}

\ml{$0$}
{--- Ты и не сможешь.}
{``You can't.}
\ml{$0$}
{Однако драться тебе придётся, --- оскалилась Штрой.}
{Like you can't escape fighting,'' Stroji grinned.}
\ml{$0$}
{--- Ты предала Орден.}
{``You betrayed the Order.}
\ml{$0$}
{И я это докажу.}
{I will prove it.''}

Штрой махнула свите и удалилась.
Убедившись, что замок щёлкнул, Стигма повернулась к Ааман.

--- Стигма благодарит их, Телохранитель.
Их клятва исполнена.

Ааман поклонились.

--- Стигма окажет им честь? --- спросили они низким хриплым голосом, не привыкшим к беседе.

--- Да, конечно.
Что они желает?

Ааман длинным ногтем постучали по её груди.
Стигма клацнула застёжкой лифа, вытянула лиф через рукав комбинезона, скальпелем сделала тонкий надрез на запястье.
Тонкую сеть синтетического полотна оросили капли крови.

--- Большая честь, --- Ааман снова склонились, глядя, как Стигма аккуратно завязывает лиф на их запястье.

--- Честь для меня, --- парировала Стигма.

<<Какие они потрясающие.
Они знает, что в застёжке, но не задаёт ни единого вопроса.
А надо бы>>.

--- Когда они сможет доиграть эту партию? --- поинтересовались Ааман.
--- Им она пришлась по душе.

--- Сегодня у Стигмы есть дела, но с завтрашнего дня --- когда они пожелает.

--- Сейчас её ход, Стигма Чёрная Звезда, --- Телохранитель кивнули и вышли вслед за Штрой.

\asterism

Штрой ждала Аамана у лифта.

--- Здравствуй, друг, --- сказала она.
--- Я знаю твою приверженность кодексу Телохранителей Слит-Же и потому хочу попросить тебя о защите.

Ааман медленно повернулся и посмотрел на девушку.

--- Ты помешал мне служить Аду.
Идти против Ада неразумно.
Если ты будешь защищать меня от заговорщиков...

--- Штрой Кольцо Дыма, --- сказали Ааман.
--- Их глаза широко открыты.
Её сон будет спокоен.
Сталь в её голосе, сталь в её ладони --- смерть её же, забвение до конца времён и после него.
Они будет сопровождать её всегда, пока она не скажет, что клятва исполнена.

Штрой нахмурилась.

--- А если я освобожу тебя от клятвы на день?

Ааман оскорблённо выдохнул.

--- Больше она её не получит, --- процедил Ааман.
\ml{$0$}
{--- Их клятва не спит и не ест.}
{``Their vow has no sleep and no eat.''}

--- Ты свободен, уходи, --- рявкнула Штрой, великолепно изображая ярость.
--- И впредь думай десять раз, прежде чем помогать врагам Ордена.

Ааман поклонился и молча вошёл в лифт.

--- Отлично, --- с удовлетворением сказала Штрой.
--- Осталось четверо Защитников.
Ещё трое подвергнутся пробам, и пятый будет служить мне до смерти.
Раб кодекса --- раб того, кто найдёт в этом кодексе прорехи.
Странно, что никто не сделал этого раньше...

<<Умная, но чересчур болтливая>>, --- молча констатировала Стигма и выключила микрофон.
Светлячок на спине Штрой задрал лапки и упал на пол.

\section{[U] Плут и Ягуар}

Стигма подошла к доске с недоигранной партией Метритхис.
Эта игра очень нравилась стратегу.
Стигма часто играла в неё сама с собой --- никто из знакомых демонов не придёт ради этой очевидной траты времени.
То, что она спросила Ааман --- невероятная удача во всех смыслах.
Никто бы не смог распознать в одном из сильнейших интерфекторов Ада приятного собеседника и игрока.

На доске нерешительно топтался зелёный Плачущий Ягуар.
Человек чести, чьё призвание --- служить.
Однако его окружили чёрный Плут и красный Разбойник --- не признающие верности, не имеющие чести.
Ягуар был обречён.

Немного подумав, Стигма передвинула Плута поближе к Ягуару и бросила камень.
Плут превратился в Купца, Плачущий Ягуар почернел и превратился в Телохранителя, и попавший под горячую руку Разбойник отправился в коробку.

Стигма улыбнулась и подумала про себя, что отныне Купец и Телохранитель пойдут одной дорогой.

<<Их ход следующий, Ааман>>.

\chapter{[U] Оскал обречённых}

\section{[U] Сиэхено}

\textspace

Даже обществу демонов визоры казались эксцентричными;
старое название --- оракулы --- было им дано не случайно.
Инкарнированные визоры переносили отпечаток личности на собственные тела --- они становились плаксивыми и эмоционально лабильными, имели тягу к самоповреждению, не следили за собой.
Кто-то связывал эту особенность с родом деятельности, но истина была простой и неприглядной --- ранних визоров презирали за беззащитность, с ними обращались, как с жуками.
Паттерны тех запуганных, покорных созданий используются до сих пор, и никакая коррекция личности уже не способна это исправить.

Девушка подняла голову и улыбнулась слабой желтозубой улыбкой.

--- Анкарьяль, старшая сестра.
А я-то думаю, кто так хочет со мной увидеться, что убивает мою стражу?..

--- Меня зовут Таниа, --- сказала воительница.

Улыбка Сиэхено увяла.
Она схватила бусы и начала раскачиваться из стороны в сторону.

--- Я охотилась на неуловимого Аркадиу Люпино, --- прошептала она.
--- И сама попала в его сети.
Ты его подруга, Тханэ ар'Катхар.
Значит, он тебя оцифровал.
А Анкарьяль... змея подколодная... её паттерны...

Сиэхено всхлипнула.

--- Я знаю обо всех устройствах в твоём теле и об их назначении, --- предупредила Чханэ.
--- Каналы связи контролируют мои демоны.
У нас ещё шестнадцать михнет на разговоры.

--- За мной придут, --- прошипела Сиэхено, --- и гораздо раньше.

--- Замечательно, --- равнодушно сказала Чханэ.
--- Только на Капитуле в процессе легенда о твоей измене.
Она не особо впечатляющая, но, учитывая твою специализацию, тебя при встрече уничтожат без разговоров.
Так что чем скорее за тобой придут, тем сговорчивее тебе стоит быть.

--- Никто не поверит, что я работаю на Картель!

--- Разумеется.
И поэтому, согласно легенде, работаешь ты на нас.
Нет ничего лучше, чем притвориться, что желаемое уже реально.

Сиэхено заплакала.
Девушка по-детски тёрла ручками незрячие глаза, размазывая слёзы и грязь по треугольному личику.
Что это --- умелая игра или проявление интимной связи, которая установилась между телом и демоном?
Чханэ пододвинула стул поближе.
Распознать подобное по действиям демона мог лишь аналитик, но попытаться стоило.

--- Знаешь ли ты, чьё тело дало тебе приют?

--- Знаю, --- всхлипнув, рассеянно сказала Сиэхено.
--- Я третья наследница, твоя и Ликхмаса ар'Люм.
Истории о вас кормилец рассказывал мне, едва я научилась понимать.

Сиэхено заулыбалась так же неожиданно, как заплакала.

--- Хаяй, куда же делась героиня моего детства?
Где храбрая и верная Тханэ ар'Катхар? --- девушка снова начала теребить бусы и раскачиваться.

Сомнения Чханэ рассеялись --- как и прочие агенты Ада, Сиэхено превратила тело тси в бездушный инструмент.

--- Ты изуродовала ей глаза, мерзость, --- невыразительно сказала Чханэ.

--- Это мой знак, --- пропела Сиэхено.
--- Ла-ла-лай, ниии, трааа...
Я обрабатываю глаза щёлочью всем своим телам.
Так лучше видно.

--- Ты боишься связи с телом, --- сказала Чханэ.
--- Почему?

Сиэхено погрустнела.

--- Темнота.
Я очень боюсь темноты.

Это было похоже на правду.
Когда только появился ангельский тип заякоривания, урождённые демоны пришли в ужас от образов, возникающих в сапиентном мозге при дефиците внешних сигналов.
Хоргеты не знают, что такое темнота --- белый шум ПКВ существует всегда, его очень сложно устранить.
Но развитый интеллект легко делает проекции.
Страх темноты засел в сознании многих демонов... и кое-кто этим манипулировал.

<<Если такие кошмары видят существа с короткой жизнью, что увидим мы?..>>

Нет зрительного анализатора --- нет темноты.
Звучит логично.
Однако Сиэхено не интегрировалась в девушку-носителя, и боязнь темноты была лишь отговоркой.
Куда больше визор боялась собственного тела --- и потому сознательно его ослепила.

\emph{Страх}.

Чханэ почувствовала, что её пробирает дрожь.
Знакомая дрожь, дрожь ярости и негодования.
Глаза тси отлично регенерируют при повреждении.
Сиэхено вводила в глаз щёлочь снова и снова, пока в погибающий нерв не пророс грубый рубец, пока хрустально прозрачную оптику глаза не заплело опалесцирующей коллагеновой сеткой, пока зрительный анализатор не атрофировался из-за отсутствия сигналов.
Что при этом чувствовала девушка, частью которой демон не желал быть?

\textspace

Чханэ почувствовала, что по её лицу текут слёзы.
<<Лис, что мне делать?>>

--- Бабушка, почему ты плачешь? --- вдруг сказала тонким голосом Сиэхено.
--- Я скушаю пирожок и расскажу, какой он вкусный.

Слёзы высохли на загоревшемся лице Чханэ мгновенно, и к горлу подступил гнев.
Эта тварь издевалась над её собственными детскими воспоминаниями.

--- Ах ты...

Сзади слабо вспыхнул источник <<света>>, и Чханэ среагировала автоматически.
Удар --- и визор перестала существовать.
Её хрупкое тело неловко свалилось на стол, невидящие глаза моргнули и тупо уставились в пространство.

Придя в себя, Чханэ поняла --- Сиэхено запустила отвлекающее устройство, имитирующее готовность к атаке, но не учла поведенческие стереотипы воина-Скорпиона.
<<Никогда не оставляй врага за спиной>> --- эту истину вдалбливали в Храме годами, и воительница среагировала, не успев осознать происходящее.
Устройство <<мигнуло>> ещё раз и отключилось, ожидая новой команды.

Испытывая отвращение к себе, Чханэ вонзила в сердце наследницы кинжал, нацарапала на грубой поверхности стола Печального Митра и вышла, тихо прикрыв за собой дверь.

\section{[U] У себя дома}

\textspace

--- Операция пошла не по плану, --- сказала Тхартху.
--- Чханэ наследила.
Она случайно убила Сиэхено, и Ад узнает, что она была убита.
Легенда срочно отозвана.
Мы должны залечь на дно.

--- Хорошо, --- кивнул Атрис.

--- Атрис, тебе тоже лучше скрыться.
Следы могут привести и к тебе, и тогда Митхэ...

--- Мы никуда не побежим, --- спокойно сказала Митхэ.

--- Митхэ...

--- Тхартху, дочка, не беспокойся за нас, --- мягким голосом прошептала Митхэ и обняла женщину.
--- Мы у себя дома.

\section{[U] Пересмотр договора}

Волны накатывались на песчаный берег одна за другой.
Митхэ и Атрис пили отвар, лопали конфеты, беседовали и смеялись.

--- ... помнишь то дерево?

--- Оно называется сорба, Митхэ.
Биологи мне сказали.

--- Давай посадим на Кристалле рощу таких деревьев?
Они красивые.
Или здесь?
Можем их модифицировать, чтобы они выросли большие и толстые!

--- Хорошая мысль, --- одобрил Атрис.
--- Мне почему-то никогда не приходило в голову, что сорба великолепно смотрелась бы вокруг этой беседки.
Давай допьём отвар и свяжемся с Кольбе и Рабе, чтобы...

--- Не торопитесь, --- раздался сзади знакомый тонкий голос, дрожащий от едва сдерживаемого злорадства.

--- Здравствуй, Штрой, --- приветливо сказала Митхэ.
--- Максим Мимоза, легат Ду-Си, и вы здесь.
Надеюсь, не для того, чтобы <<уничтожить человеческую часть Митриса Безымянного>>.
Не хотите ли выпить?

--- К сожалению, именно для этого, Митхэ ар’Кахр, --- звучным, богатым баритоном сказал Мимоза.
--- И если ваш друг немедленно передаст нам все ключи от системы управления, это произойдёт быстро и безболезненно.

Атрис пожал плечами и отхлебнул отвара.
Мимоза --- высокий, холодный и стройный северянин, инкарнация одного из лучших тактиков отдела 100 --- удивлённо приподнял бровь.
Ду-Си, его полная противоположность --- крепко сбитый, покрытый рельефом огромных, выращенных на синтетических стероидах мышц --- одобрительно крякнул.
Даже по телу легата было видно, что он любит трудности: на ноге --- отчётливый шрам от акульих зубов, бритая голова сияет, за ухом виден криво вшитый имплант-инфузомат.
Если верить докладу Аркадиу, Ду-Си --- выкормыш Воронёной Стали, ушёл на вольные хлеба\FM\ ещё до основания Красного Картеля, затем уже в Ордене служил диверсантом в упразднённом отделе 125 и имел богатый опыт геноцида Девиантных Ветвей, уничтожения Роев и несговорчивых богов.
\FL{neutrals-history}{История и современный правовой статус нейтральных демонов}

Идеальное трио.
Штрой --- провокатор и сборщик информации.
Мимоза и Ду-Си играют неприятеля и возможного друга, а в случае столкновения превращаются в интерфекторов со взаимодополняющими тактиками ведения боя.

\ml{$0$}
{--- Атрис, вы ещё с нами? --- осведомился Мимоза.}
{``\Aatris{}, are you still with us?'' Mimosa asked.}
\ml{$0$}
{--- Кажется, я вынужден повторить, что...}
{``I think I have to repeat, you---''}

\ml{$0$}
{--- Я вас услышал, максим, --- ответил Атрис.}
{``You've been heard, maccsim,'' \Aatris\ answered.}
\ml{$0$}
{--- С этим есть некоторые затруднения.}
{``We have some trouble about that.}
Ключей у меня нет, модуль управления планетой соединён с Митхэ интегративным нейроконтуром, и управлять им может только она.
\ml{$0$}
{Всё-таки я ожидал, что перед тем, как перейти к делу, вы вежливо ответите на вопрос, который задала моя женщина.}
{Anyway, I expected you to do my woman the courtesy of an answer instead of cutting to the chase.}
\ml{$0$}
{Смею напомнить, что это мой дом, как в широком, так и в узком смысле этого слова.}
{I'd like to remind: you are in my home, in a wide sense and in a narrow sense of the word.''}

Агенты замерли, словно восковые куклы.

--- Сели за стол и выпили с нами, --- рявкнула Митхэ.
Натренированный тон годился лишь для фразы <<Сдавайся или умри>>, и слова повисли в воздухе.
Интерфекторы с каменными лицами сели за стол и невозмутимо взяли свои чаши.
Штрой, поколебавшись, последовала их примеру.
Её руки дрожали от сдерживаемой ярости.

--- Что ты так смотришь на меня, Штрой Кольцо Дыма? --- холодно осведомилась Митхэ.
\ml{$0$}
{--- Это же <<вкусно>>, забыла?}
{``It's `quite delicious', remember?''}

Девушка промолчала.

--- Почему вы не сообщили о ключах отделу 100, Атрис? --- мягко спросил Мимоза.

\ml{$0$}
{--- Я слегка рассеян, --- признался менестрель, --- и иногда забываю сообщать контрагенту подробности, которые его не касаются.}
{``I'm absent-minded a bit,'' the minstrel admitted, ``and sometimes I forget to provide counterparty with details which are not their business.}
Итак, в чём нас подозревают на этот раз?

--- Вы подозреваетесь в убийстве Сиэхено Опаловый Глаз, --- ответил Мимоза.
--- Вы знакомы с ней?

--- Заочно.
Насколько я знаю, она предоставляет услуги глубокого сбора данных военным и недавно перевелась на Тра-Ренкхаль.

--- Нам известно, что убийство визора Сиэхено координировалось через одну из ваших станций.
Нам известно, что убийца использовал как метку лик Печального Митра --- если не ошибаюсь, это имеет некоторое отношение к вам.
Также нам известно, что агент, собиравший информацию о местоположении визора и связанных с ним демонов, использовал ваш уникальный идентификационный код.
Это лишь малая часть деталей, указывающих на вашу причастность.

--- Могу ли я ознакомиться с прочими?

--- К сожалению, нет.
У вас была бы такая возможность, если бы вы предоставили нам средства контроля над модулем управления планетой.

--- Мимоза, я вам уже объяснил, что это невозможно.
Вы, конечно, можете забрать у меня Митхэ целиком.
Но, во-первых, мы с ней уже тоже достаточно глубоко интегрированы, и разделение может нанести нам обоим непоправимый вред.
Мы можем напрямую слышать мысли друг друга, без использования иных каналов связи --- если вам это о чём-то говорит.
Во-вторых, я сильно сомневаюсь, что даже в случае удачного разделения она захочет с вами сотрудничать.

--- Вас было вынесено официальное предупреждение... --- начал Мимоза.

\ml{$0$}
{--- Мне надоели ваши шпионские игры, --- прервал его Атрис.}
{``I'm tired of these spy games,'' \Aatris{} interrupted him.}
--- Когда Аркадиу Люпино победил на Могильном берегу, он сказал мне, что я и мои создания будут жить счастливо.
\ml{$0$}
{Однако у Ада своё представление о счастье.}
{But the Hell has its own definition of happiness.}
\ml{$0$}
{Стыдно признаться, но даже безумный Эйраки не доставлял мне так много головной боли, как вы.}
{I'm ashamed to admit, even senseless Ejraci was not such a severe headache as you are.''}

Интерфекторы усмехнулись.

\ml{$0$}
{--- Мне нравится ваш подход, Атрис, --- широко улыбаясь, сказал Ду-Си.}
{``I like your attitude, \Aatris{},'' Du-Sie said with a big smile.}
\ml{$0$}
{--- Называть Ад головной болью я бы не решился.}
{``I wouldn't go so far as to call the Hell `a headache'.}
\ml{$0$}
{Особенно сейчас...}
{Especially now ....''}

--- А как ещё вас называть? --- невесело усмехнулась Митхэ.
--- Мы --- ваши союзники, но угрозы я слышу постоянно.
Одна статья, другая статья, десятая статья, пункт сотый, параграф тысячный.
Смерть, смерть, смерть.
Половину законов Ада можно было бы заменить одним этим словом.
После сотой угрозы, какая бы сила за ней ни стояла, уже порядком надоедает.
\ml{$0$}
{Вы пришли меня казнить, а мне ужасно скучно.}
{You come here to execute me, and I find it boring.''}

Мимоза изобразил холодное удивление:

\ml{$0$}
{--- Интересные новости.}
{``Nice news.}
\ml{$0$}
{Можно узнать, от кого вы слышите угрозы в таких количествах?}
{May I ask, who threatens you so often?''}

\ml{$0$}
{--- Да хотя бы от неё, --- Митхэ махнула рукой на Штрой.}
{``I might start with her,'' \Mitchoe\ pointed to Stroji.}
\ml{$0$}
{--- Я не удивлюсь, если именно она нас подставила.}
{``I wouldn't be surprised if it was she who framed us.''}

Демоница вытаращила глаза.

\ml{$0$}
{--- Kuna!}
{``\emph{Kuna!}''}

\ml{$0$}
{--- Я рада, что ты, милая моя перебежчица, ещё помнишь ругательства Картеля, --- улыбнулась Митхэ.}
{``I'm glad you, my sweet defector, still remember obscenities of Cartel,'' \Mitchoe\ smiled.}

\ml{$0$}
{--- Стоп, --- поднял руку Мимоза.}
{``Stop,'' Mimosa raised his hand.}
\ml{$0$}
{--- Митхэ, это очень серьёзное обвинение.}
{``\Mitchoe\, that's a very serious accusation.}
\ml{$0$}
{Я бы на вашем месте подумал, прежде чем...}
{If I were you, I would think twice before---''}

\ml{$0$}
{--- Поверьте, максим, у нас была масса времени на размышления, --- прервал его Атрис.}
{``Take it from us, maccsim: we had plenty of time for thinking,'' \Aatris{} interrupted him.}
\ml{$0$}
{--- Штрой Кольцо Дыма пытается нас оскорбить с первого дня знакомства.}
{``Stroji the Smoke Ring have been trying to abuse us verbally, since the first time we met.}
\ml{$0$}
{Записи могу предоставить.}
{Recordings can be provided.}
\ml{$0$}
{Скажите, Мимоза, были ли разрешены такие меры приказами Капитула?}
{Tell us, Mimosa, were such measures allowed by Capitul's orders?''}

\ml{$0$}
{--- Исключено, --- буркнул интерфектор.}
{``No way,'' the interfector answered.}
\ml{$0$}
{--- Рискну предположить, что имело место недопонимание.}
{``I would venture to guess a misunderstanding took place.}
Ваше представление о вежливости может отличаться от представлений командующего, живущего в культурной среде Ордена...

--- Культура подразумевает соответствие некоторым общим нормам, --- перебил Атрис.
--- Я думаю, что наши с вами отношения держатся не только на некой фактической базе, но и на формальностях, которые, кстати, достаточно чётко прописаны в известном вам протоколе под смешным названием <<Шляпа>>.
Я нашёл его очень занимательным.
Там сказано, что провокация, демонстрация власти, лесть и подхалимство являются формами замаскированной агрессии.
Кроме того, там недвусмысленно объясняется, что давление на дружеское и сексуальное чувство в процессе производства является способом установления незаконной, коррупционной связи.
Учитывая, что Штрой Кольцо Дыма умудряется за одну реплику применить все вышеперечисленные методы ведения беседы, недопонимание --- недопонимание командующим Штрой культурной среды Ордена --- действительно имеет место быть.

--- Поведение Штрой определённо заслуживает расследования, но в данный момент...

--- В данный момент мы говорим о том, что имеет к делу непосредственное отношение.
Штрой Кольцо Дыма вразрез с заключённым на Деймос-14 договором постоянно предпринимает попытки ограничить мой доступ к системе управления и требует отчёты о моей деятельности.
Доходит до того, что я орешков пощёлкать не могу, не доложив командующему.
Был ли на эти действия приказ отдела 100 или высшего командования Ада?

\ml{$0$}
{--- Исключено, --- повторил Мимоза.}
{``No way,'' Mimosa repeated.}
--- Однако вы находитесь на территории Ордена Преисподней, и ради безопасности командующий имеет право...

--- Замечательно, --- усмехнулся Атрис.
--- В таком случае я официально требую от вас подкреплённый нормативными актами документ, чётко проводящий грань между обеспечением безопасности и неоправданным ограничением в действиях.
Этот документ будет очень хорошо смотреться как дополнение к договору <<Демиург --- Метрополия>>.

--- Вы имеете на это право, как контрагент Ордена, --- хмуро сказал Мимоза.
--- Но сейчас мы обсуждаем ваши деяния, которые ставят под угрозу саму возможность быть таковым, и я настаиваю...

--- Я ещё не закончил, Мимоза.

--- Мы не намерены продолжать разговор в таком тоне, --- холодно сказал максим.

Ду-Си подобрался, словно тигр перед прыжком.
Митхэ презрительно хмыкнула --- старый как мир приём.

--- Нам этот разговор также не доставляет никакого удовольствия, --- буркнула она.
--- Но довести его до конца --- в наших общих интересах.
Вы готовы слушать Атриса или будете дальше продолжать этот театр?

Мимоза промолчал.
Ду-Си несколько обиженно посмотрел на воительницу и отхлебнул из своей чаши.

--- Отлично, --- продолжил Атрис.
--- Штрой Кольцо Дыма и её демоны регулярно предпринимают попытки самостоятельно исследовать механизмы системы, при том, что у неё есть ключи от системы поиска и всех боевых устройств.
Некоторые попытки вмешательства были с целью захватить контроль над узлами, прочие я не могу трактовать иначе как намеренное вредительство.
Например, не далее как два рассвета назад из-за очередного <<техосмотра>> вышла из строя сначала метеостанция, а затем и три ключевых сейсмодатчика, и землетрясение в Ихслантхаре едва удалось предотвратить.

\ml{$0$}
{--- Есть ли у вас доказательства? --- спросил Мимоза.}
{``Do you have any proof?'' Mimosa asked.}
\ml{$0$}
{--- Если есть, почему это прошло мимо нас?}
{``If you do, why it's passed unnoticed by us?''}

\ml{$0$}
{--- Я понятия не имею, максим.}
{``I've no idea, maccsim.}
Экологической отчётностью, как вы знаете, занимается команда Корхес Соловьиный Язык.
Мои записи они передали вместе с остальными данными на Капитул.
\ml{$0$}
{Вероятно, вам следует навести справки там.}
{Maybe you should make an inquiry there.''}

Штрой остолбенела.
Интерфекторы бросили на неё два совершенно одинаковых взгляда.

--- На будущее, Атрис, --- проговорил Мимоза, покручивая длинными пальцами чашу.
--- Обработка такого рода информации --- юрисдикция не исследовательских отделов, а контрразведки.
Вам повезло, что Корхес имеет право на сбор, передачу и верификацию информации, составляющей военную тайну.
В противном случае вас могли привлечь к ответственности.

--- Я запомню, Мимоза.
Мне последнее время приходится помнить очень много вещей, не имеющих никакого отношения к моим прямым обязанностям.
Например, вы видели это?

Атрис показал на потолок беседки, и все дружно подняли глаза.
На потолке красовалась кривая надпись масляной краской --- <<Б.Д.И.У.>>.

--- Я совсем недавно узнал про ДиС, --- продолжал Атрис.
--- Как и про то, что отдельные группировки в Ордене и Картеле открыто лоббируют их интересы.

--- Это шалость деревенских, Атрис.
Ничего более.

--- То есть вы считаете, что рыбаки и крестьяне приходили в мою беседку и писали в ней лозунги террористической организации демонов?

--- Атрис, по моему опыту общения с ДиС --- а он достаточно богатый, к сожалению --- это не их стиль.
\ml{$0$}
{Они чаще используют взрывчатку, чем краску.}
{They prefer explosive, not dye.}
И я напоминаю, что мы говорили о системе управления планетой.

--- Спасибо, что напомнили, Мимоза.
Надеюсь, вы также напомните командующему Штрой, что всю --- я подчёркиваю, всю --- необходимую информацию о системе уже получили компетентные демоны и доставили её куда следует.

--- Компетентные демоны? --- рявкнула демоница.
--- Грейсвольд Каменный Молот?

--- Вы сомневаетесь в его профессионализме, Штрой? --- вдруг подал голос Ду-Си.
--- Я не знаком лично с Грейсвольдом, но наслышан о его деяниях.
Весьма впечатляющая биография, надо сказать.
Кстати, в настоящее время он является сотрудником отдела 100, и если вы сомневаетесь в его компетентности, мы всегда можем вас выслушать.

--- Они все повязаны одной красной ленточкой, --- буркнула Штрой.
--- Я уже предоставляла свои выкладки на этот счёт в отдел 100.

--- Мы их рассмотрим, Штрой, --- холодно сказал Мимоза.
--- Атрис, у вас всё?

--- Ещё нет, Мимоза.
Согласно договору, заключённому между мной и Адом...

--- Договор --- это пустая болтовня, Атрис.
Ты здесь никто, --- перебила Штрой.

Атрис улыбнулся и встал.

--- Ошибаешься, Штрой.
Договор заключается между равными, и стратеги Ада знают это очень хорошо.
Они также знают, как полезен демиург-союзник и насколько опасен демиург-враг.

Пока менестрель говорил, в нём происходила странная метаморфоза.
Он вдруг сбросил личину бродяги.
В его осанке, голосе и движениях появилось величие.

--- Создатель Тра-Ренкхаля --- это я.
Хозяин Тра-Ренкхаля --- это я.
Планета Тра-Ренкхаль --- это я! --- демон Митриса засветился ярким светом и демонстративно привёл в готовность боевые модули.
Мимоза прищурился, Штрой едва не упала за оградку.
--- И если Орден Преисподней в этом сомневается, желает оскорблять Митриса Безымянного, желает убить Митриса Безымянного, пусть попробует сделать это сейчас.
Я расторгну договор, приму бой и напомню агентам Ордена, на чьей планете они стоят!

Демон погас.
Штрой дрожала всем телом, вцепившись в скамью.
Ду-Си с поэтической рассеянностью тянул конфеты из вазочки.

--- Я теряюсь в догадках, как это может нам повредить, Атрис, --- проговорил Мимоза.
--- Боевые модули системы вам больше не принадлежат.

--- Ну, для начала гарнизону Ада придётся что-то сделать с распадом массивной минус-сингулярности, --- ответила Митхэ.
--- Несмотря на то, что Орден теперь утилизирует большую долю продуции планеты, мы по-прежнему располагаем объёмом масс-энергии порядка константы Ка'нета.
Кроме того, мы теперь умеем менять полярность.
Преобразование Шмидта, --- Митхэ произнесла звучный термин с неописуемым удовольствием, которое доступно лишь дилетанту.

Мимоза широко улыбнулся воительнице.
Митхэ ответила ему своей самой светлой щербатой улыбкой.

--- Знаете, --- непринуждённо сказал Мимоза, --- я очень не люблю иметь дело с самоубийцами.
Это возмутительно неравное сражение.

--- Жаль, что для вас отношения с нами --- лишь очередное сражение, --- парировала Митхэ.

--- Я привык мыслить по-военному, прошу меня за это простить.
В целом --- да, это всего лишь очередное сражение с оправданными жертвами.

--- Планета будет одной из них, --- сладким голосом сказала воительница.
--- Модуль управления, я и Атрис --- три взаимосвязанных звена цепи, которые удерживают планету от падения в бездну.
Не будет любого из нас --- тектонические плиты раскачаются, биолитосфера развалится, и Тра-Ренкхаль вернётся в первозданный вид.
Сомневаюсь, что Орден сочтёт потерю мелиорированной планеты оправданной жертвой.

--- Не волнуйтесь, Митхэ, мы успеем подхватить, --- лениво успокоил её Мимоза.
--- Я уверен, что геологи прекрасно справятся с этой проблемой.

--- Смею напомнить, Мимоза, что геологические исследования вами не проводились и землетрясения до сих пор предотвращаю я, --- подал голос Атрис.
--- Эффективность планеты как источника масс-энергии --- мой каждодневный труд.
\ml{$0$}
{Я дилетант в политике, у меня масса пробелов в естественнонаучных знаниях, но у меня были тысячелетия, чтобы научиться интуитивно управлять планетой и при этом сводить концы с концами.}
{I'm amateur in politics, my knowledge in science is full of gaps, but I've had thousands years to learn intuitively how to handle the planet while getting by.}
\ml{$0$}
{Было ли это время у вас?}
{Have you?''}

\ml{$0$}
{--- Не прибедняйся, демиург.}
{``Don't be so modest, demiurge.}
У тебя были превосходные учителя с планеты Тси-Ди, --- с отвращением выплюнула Штрой и тут же осеклась под взглядом Мимозы.

\ml{$0$}
{--- А с чего вы взяли, Атрис, что исследования не проводились? --- осведомился максим.}
{``So, \Aatris{}, what makes you think research was not carried out?'' the maccsim asked.}

--- Со времени освоения Тра-Ренкхаля ко мне за консультацией не приходил ни один геолог.
В отличие от биологов --- эти ребята бегают ко мне постоянно с тем, что они выудили и выкопали, и результаты их гигантской работы я вижу, можно сказать, в реальном времени.
Из чего можно сделать вывод, что отдел геологии на Тра-Ренкхале --- это красивая ширма, за которой нет ничего, кроме неприглядных планов по уведению моего энергетического сальдо в минус.
%{is a colourful smokescreen, and behind it there is nothing but disgusting schemes of pushing my energy balance into deficit.''}

Ду-Си как-то странно усмехнулся --- Митхэ почудилась смесь стыда и горечи.
Мимоза слушал речь Атриса с выражением врача, сидящего у постели душевнобольного.

--- Вы серьёзно полагаете, что целый научный отдел тратит ресурсы не на науку, а на то, чтобы вас провести?
Атрис, для дезинформации у нас есть специальные отделы.

--- Поменять местами таблички на дверях --- это минутное дело, --- буркнула Митхэ.

\ml{$0$}
{--- У меня создаётся впечатление, что вы считаете Орден Преисподней идиотами.}
{``I get the impression that you take the Order of Nether for fools.''}

\ml{$0$}
{--- Я был бы рад ответить Ордену Преисподней взаимностью, но мне не позволяет воспитание, --- парировал Атрис.}
{``I would be glad to reciprocate the attitude shown by the Order, but I'm too polite,'' \Aatris{} retorted.}
--- Поэтому я заявляю вам официально, как эмиссару: Митрис Безымянный желает пересмотреть договор с Орденом Преисподней.
Делегатов ждём завтра в это же время здесь.
И захватите с собой сладости, Ликхэ ар'Митр в Кахрахане замечательно готовит булочки.
Думаю, у Ордена Преисподней найдётся достаточно слитков, чтобы купить этот шедевр на вес кукхватра.

Мимоза натянуто улыбнулся.

\ml{$0$}
{--- У вас всё?}
{``Are you done?''}

\ml{$0$}
{--- Ещё нет, Мимоза.}
{``Not yet, Mimosa.}
\ml{$0$}
{Если я сочту какие-либо действия членов Ордена неуважительными, я без предупреждения саботирую производство масс-энергии.}
{If members of the Order do something I find disrespectful, I'll sabotage mass-energy production without warning.}
\ml{$0$}
{И если Митрис Безымянный говорит <<саботирую>>, это значит, что никто и никогда больше не сможет использовать планету Тра-Ренкхаль в качестве источника питания.}
{And if Mitris the Nameless say `sabotage', it means: no one ever will use planet Tr\r{a}-R\={e}nkch\'{a}l as an energy source.}
\ml{$0$}
{Запишите это дословно во избежание недопониманий.}
{Write it down verbatim to avoid misunderstandings.}
\ml{$0$}
{Штрой, я не знаю, кому служите вы, поэтому ваше поведение оставляю делом вашей совести.}
{Stroji, I do not know where your allegiance lies, so I leave your behaviour up to you.''}

\ml{$0$}
{--- Полно вам, Атрис, --- осклабился Ду-Си.}
{``Come on, \Aatris{},'' Du-Sie smiled.}
\ml{$0$}
{--- Вы не любите свой народ?}
{``You don't love your people?}
\ml{$0$}
{Ни за что не поверю.}
{I refuse to believe.''}

\ml{$0$}
{--- Очень люблю, --- признался демиург.}
{``I really love them,'' the demiurge admitted.}
\ml{$0$}
{--- Особенно тси, ведь они, как уже было сказано, потомки моих прекрасных учителей.}
{``Especially Qi, because, as previously stated, they are scions of my lovely teachers.}
\ml{$0$}
{Не волнуйтесь, легат.}
{Don't worry, legate.}
\ml{$0$}
{У меня есть внушительная геномная база, подробнейшие сведения о языках и культуре.}
{I've got huge genome database, detailed information on languages and culture.}
\ml{$0$}
{Если мне придётся деактивировать планету, разумеется, я расселю мой народ по Вселенной, где только смогу.}
{If I have to deactivate the planet, of course I will resettle my people throughout the Universe, everywhere I can.}
Думаю, это занятие, достойное того, чтобы посвятить ему остаток жизни.

Штрой приобрела цыплячий оттенок.
Ду-Си крякнул с неподдельным восхищением и шлёпнул ладонью по столу:

\ml{$0$}
{--- А грамотно он нас за яйца взял, а, Мими?}
{``He's quite competently got us by the balls, hasn't he, Mimi?''}

--- Ещё раз назовёшь меня <<Мими>> --- снова получишь пенальти на три ранга, --- бледный как снег Мимоза встал чуть резче, чем следовало, и принялся поправлять смятую одежду.
\ml{$0$}
{--- Атрис, я вынужден сообщить о ваших методах ведения переговоров.}
{``\Aatris, I have to report about your negotiation techniques.}
\ml{$0$}
{Боюсь, у Ордена может возникнуть впечатление, что вы намерены прекратить сотрудничество.}
{I'm afraid this might be considered as intention to severe the cooperation.''}

\ml{$0$}
{--- Ни в коем случае, Мимоза.}
{``No way, Mimosa.}
\ml{$0$}
{Я намерен продолжать сотрудничество, что и подтверждаю запросом на пересмотр договора.}
{I intend to continue to cooperate, as confirmed by my request for re-negotiation.}
Я лишь настаиваю на том, чтобы отдел 100 в контрексте расследования внимательнее оценил, во-первых, вероятность вмешательства ДиС в мою работу, и во-вторых, деяния командующего Штрой, а именно --- ограничение моей свободы передвижения, слежка за мной, а также захват контроля над моими механизмами.
\ml{$0$}
{Я думаю, вы согласитесь с тем, что у командующего планетарными силами были все возможности имитировать мои действия, и уж точно куда больше возможностей, чтобы кого-то убить.}
{I guess you agree that PFC has full opportunity to imitate my actions, and more than full opportunity to kill.} % Planetary Force Commander
\ml{$0$}
{Мотивы ищите сами, у меня масса своей собственной работы.}
{Her motivation is your concern, I have lots of my own work.}
\ml{$0$}
{В конце концов, если цель --- я, почему бы и не ДиС...}
{After all, if the target is me, why not D.o.D. ....''}

Митхэ бросила взгляд на Штрой.
Ту уже в прямом смысле колотило от ярости, и Ду-Си аккуратно придерживал её за локоть, что бы она держала себя в руках.

--- Вы отказываетесь подчиниться правосудию и ввели отдел 100 в заблуждение, --- сказал Мимоза.

--- Я отказываюсь добровольно подчиняться мерам, которые предписал вам Капитул, так как расследование нельзя считать законченным, а их последствия необратимы.
К сожалению, мне пришлось прибегнуть к угрозам, так как вы чересчур серьёзно отнеслись к отданному вам приказу.

--- Вы не можете манипулировать планетой бесконечно, --- Мимоза опустил голос до шёпота.
\ml{$0$}
{--- Однажды терпение Ордена лопнет, и лопнет также без предупреждения.}
{``One day the Order's patience will run dry, and there won't be a warning as well.''}

\ml{$0$}
{--- Я боюсь, это не те угрозы, которыми может разбрасываться адепт Ордена.}
{``I'm arfaid it's not a threat a member of the Order is entitled to make.}
\ml{$0$}
{Отношения с Митрисом Безымянным --- это одно, ставить под вопрос юридическую силу договора <<Демиург --- Метрополия>> --- совсем другое.}
{Relationship with Mitris the Nameless is one thing, and it's quite another to question the legal effect of Demiurge \& Metropolis Pact.}
\ml{$0$}
{Очень многим в Ордене могут не понравиться такие богатые подарки Картелю.}
{Many people won't appreciate such a rich gifts to Cartel.''}

--- Тра-Ренкхаль --- это значимый источник масс-энергии, но один из многих.

--- Мимоза Шёлковая Сталь и Орден Преисподней --- также одни из многих, --- ответил Атрис, --- но это не обесценивает ни мой выбор, ни ваши заслуги.

Мимоза и Ду-Си коротко поклонились и вышли.
Штрой, злобно посмотрев на влюблённых, демонстративно опрокинула вазочку с конфетками и тоже вышла.
Круглые цветные сладости, весело стуча, разбежались по всем углам беседки.

--- Думаешь, это поможет? --- шепнула Митхэ.

--- Уже помогло, --- улыбнулся Атрис.
\ml{$0$}
{--- Давно надо было показать зубы.}
{``We should've shown our teeth sooner.}
\ml{$0$}
{Слишком многие привыкли относиться к богам, как к мусору.}
{Too many people are used to treat gods like garbage.}
\ml{$0$}
{Слишком долго они аплодировали тому, как нас уничтожают.}
{For too long they cheered about our extermination.}
\ml{$0$}
{Слишком привычной стала парадигма <<будь демоном или умри>>.}
{Too accustomed they are to paradigm `Be daemon or be dead'.''}

--- А расследование?

\ml{$0$}
{--- Время играет на нашей стороне.}
{``Time is on our side.}
\ml{$0$}
{Чем больше затянется следствие, тем больше путаницы.}
{The longer the investigation, the greater the confusion.}
\ml{$0$}
{Пока в деле путаница, трогать нас не будут --- я им нужен.}
{They won't bother us until they have some clarity, they need me.''}

--- А если нас всё-таки смогут зацепить?

--- Аркадиу сказал, что следствие уйдёт в сторону адской резидентуры.
Там есть демон --- мастер мутить воду, он сможет повести следствие по нужному пути.
А нас нужно подготовиться к завтрашнему разговору.
Мы потребуем символические привилегии и пару реальных свобод, Орден поторгуется, мы уступим привилегии, Орден всё поймёт и согласится, мы разойдёмся довольные.
Штрой отправится домой.
Разумеется, она докажет свою непричастность к смерти Сиэхено, но репутацию мы ей подмочили основательно.

\ml{$0$}
{--- <<В том, что вас скомпрометировали, есть доля вашей вины>>, --- процитировала Митхэ.}
{``\emph{You're partly responsible for being compromised,}'' \Mitchoe\ quoted.}
\ml{$0$}
{--- <<Аккуратнее выбирайте друзей>>.}
{``\emph{Be careful making friends.}''}

\ml{$0$}
{--- И врагов тоже, --- заключил Атрис.}
{``Making enemies, too,'' \Aatris\ finished.}
\ml{$0$}
{--- Хочешь на берег?}
{``You want to go to the coast?''}

\ml{$0$}
{--- Хочу, милый, --- обрадовалась Митхэ.}
{``I do, honey,'' \Mitchoe\ rejoiced.}
--- Давай наберём ракушек, я давно хотела сделать что-нибудь красивое.
А вечером узнаем у биологов, как выращивать сорбу.

\ml{$0$}
{--- Вряд ли мы успеем её вырастить, --- загрустил менестрель.}
{``We seem to have few time for growing it,'' the minstrel got upset.}
--- После этого инцидента они сильно ускорят процесс исследования планеты и защитной системы.
\ml{$0$}
{Следствие следствием, но когда Орден сведёт баланс сил без нас --- придётся отдать наш дом сельве\FM.}
{The investigation is what it is, but if the Order balances forces without us, we'll be forced to leave our home to the Silva\FM.''}
\FA{
Бежать.
Практика символически отдавать покинутый дом сельве распространена среди северных сели --- в доме ломали крышу и пробивали половые доски, открывая тем самым доступ растениям.
}

\chapter{[U] Ответный удар}

\section{[U] Неудача с Телохранителем}

\textspace

--- Штрой, вы хотя бы день можете прожить без переполоха?
Я слышал, что отделу 100 пришлось нейтрализовать одного из Телохранителей, чтобы спасти вас.
Надеюсь, вы не пытались его завербовать?

Штрой вытаращила глаза.

--- Понятно, --- буркнул голос.
--- Вы могли хотя бы спросить меня перед таким опасным шагом.
Он на вас напал, не так ли?
Каковы потери?

--- Я потеряла четверых агентов, --- Штрой склонила голову.
--- Прошу прощения.

--- Штрой, запомните раз и навсегда.
Ад --- это клубок интриг, и если вам кажется, что кто-то что-то до сих пор не сделал --- скорее всего, на это есть веские причины.
В Кодексе Слит-Же последний пункт звучит так: <<Если подзащитный использует Кодекс для манипуляции Телохранителем, подзащитный должен быть уничтожен>>.
Этот пункт может трактоваться Телохранителем по его усмотрению.

--- Я тщательно замаскировала манипуляцию!

--- А я тебе говорил, что ты недооцениваешь Стигму, --- заявил Самаолу.
--- Может, Телохранитель и достаточно глуп для тебя, но Стигма достаточно умна, чтобы растолковать сложное глупцу.

--- С Защитниками шутки плохи, --- подытожил голос, --- и я запрещаю вам приближаться к ним впредь.

Штрой кивнула с явным облегчением.
Видимо, у неё отпало всякое желание приближаться к Защитникам.

--- А что насчёт Стигмы?

--- Дальнейшие попытки чересчур опасны.
Она уже едва не вывела вас из игры с помощью Безымянного.

--- Это была она?
Это она инструктировала Безымянного?

--- В этом даже я не сомневаюсь, --- буркнул Самаолу.
--- Почерк сложно не узнать.
Она раззадоривает врага мнимой слабостью, и он сам бросается на подставленный нож.

--- Но я хочу попробовать ещё раз!..

--- Никаких <<но>>, Штрой, --- отрезал голос.
--- Я не хочу вас потерять.

Последняя фраза огрела Штрой словно кнутом.
Она упала на колено и прижала руку к груди:

--- Я вас не подведу, владыка!

--- Вы опять начинаете?
Впрочем, ладно, продолжайте.
Стыдно признаться, мне это начинает нравиться.
Но не при агентах.

\textspace

\section{[U] Талианский канал}

\textspace

--- Сиэхено успела оставить сообщение перед смертью, --- сказал Самаолу.
--- Скорбящие используют для сообщения оригинальную восьмиполосную таблицу языка Эй, сочетающую макропараметр планеты Тра-Ренкхаль и неизвестный микропараметр.
К сожалению, сообщение неполное.
Это вся информация, известная на сегодня.

--- Возможно, какая-то вариация Талианского канала, --- предположил голос.
--- А, вы не в курсе, Самаолу.
А вот Штрой должна знать.
Макропараметром таблицы являются географические координаты.
Микропараметром может быть что угодно.
В частности, агенты Картеля использовали бутылочки из цветного стекла, выкладывая ими в определённых координатах оставшуюся часть сообщения.

--- Требуются гигантские затраты, чтобы найти сообщение, --- мрачно сказала Штрой.
--- Но в итоге Ад справился, Талианский канал был раскрыт.
Это положило весомый камешек на чашу победы Ада при Десяти Звёздах.

--- Потому что когда бутылочки появляются в девственном лесу или на необитаемом острове, это наталкивает на определённые мысли, --- согласился голос.

--- Я распоряжусь насчёт поисков, --- кивнул Самаолу.

--- Не тратьте время, --- сказал голос.
--- Талианский канал может быть доработан с учётом местности, климата и прочих особенностей.
Кроме того, сообщения могут быть временными --- например, изо льда или песка.
Не забывайте также, что одним из Скорбящих почти наверняка является демиург, что значительно расширяет список возможных макропараметров планеты.
Может, он вообще кодирует сообщения ветром, молниями или дождём.
Звучит дико, но ведь теоретически это возможно?

--- <<Слова не имеют смысла, если нет того, кто может их понять>>, --- процитировала Штрой.

--- Штрой! --- восхитился голос.
--- Вы цитируете ненавидимого вами Люпино?
Я чрезвычайно рад, что вы растёте как профессионал.

--- Враг тоже может быть прав.

--- Я вам скажу больше, Штрой --- иногда враг ближе к истине, чем вы, но природа мыслящего существа подразумевает ещё и некое комфортное расстояние от этой истины.
Именно поэтому мы стремимся к победе, а Скорбящих устраивает любой мир.

--- Вы хотите сказать, что мы идём неправильным путём? --- ошарашенно спросил Самаолу.

--- Ни в коем случае.
Я хочу сказать, что местоположение истины во Вселенной для практического применения абсолютно неважно.
Важно найти собственное место.
Скорбящие же ищут саму истину --- и в этом их слабость, потому что даже найдя, они не обретут ничего значимого для себя.

--- Боюсь, как бы я не начала им сочувствовать, --- скривилась Штрой.

--- Сочувствуйте, Штрой, сочувствуйте и не бойтесь!
Ненависть заставляет отрезать врагу пути к отступлению, и враг сражается до конца.
Если же вы из сочувствия оставите им лазейку --- они сдадут свои позиции проще и быстрее.

--- Тахиро Молниеносный как-то говорил, что уважение и милосердие более выигрышны в долговременной перспективе, нежели деспотизм и жестокость.

--- И его слова не лишены смысла.
Но он не упомянул, что куда более выигрышным является их грамотное сочетание.

--- Вы имеете в виду жестокость к врагам и уважение к друзьям?

--- Я имею в виду ситуационный подход.
Иногда следует щадить и даже спасать соперника, если его гибель принесёт вам больше вреда, чем пользы.
Иногда следует быть жестким к союзнику, если выгода от этого высока.
Именно поэтому я и не употребляю термины <<друг>> и <<враг>>.
Они чересчур безапелляционные.

--- Быть жестким к союзнику? --- поднял бровь Самаолу.

--- Именно.
Уничтожать союзников не стоит, иначе их у вас не останется.
Впрочем, то же самое могу сказать про соперников.
Отсутствие соперников --- весьма неприятная ситуация, особенно для неумелого игрока.
Нужно держать себя в тонусе, не находите?
На мой взгляд, это достаточно веское основание, чтобы сохранить чужую жизнь.

--- Так что делать с сообщением Сиэхено? --- осведомился Самаолу.

--- То, что я сказал.
Займитесь не кодом, а подозреваемыми --- они сами выведут нас куда надо.

\section{[U] Омега-чувство}

\textspace

--- Только что пришли данные от биологов, --- сказал Самаолу.
--- У тси обнаружили ещё один орган чувств --- омега-чувство.
Они способны воспринимать колебания ПКВ средней мощности.

Молчание.

--- Самаолу, приступайте к реализации варианта номер восемнадцать, --- наконец сказал голос.
--- Мы должны в течение десяти дней убедить Орден в необходимости геноцида тси.
Если не получится --- на Тра-Ренкхале обязан произойти катаклизм.

--- И я даже знаю, на кого можно повесить ответственность, --- ухмыльнулась Штрой.
--- К слову.
Вы не рассматривали моё предложение насчёт фальсификации...

--- Штрой, мой ответ --- нет, --- отрезал голос.
--- Я понимаю вашу любовь к риску, но это не просто риск, это самоубийство.
ФПГП --- это взрывчатка, детонирующая от кашля.
Да, это лёгкий способ устранить Безымянного, но если отдел 100 выйдет на кого-то из нас --- то можно сразу рыть могилы для всех.
Есть данные, что биологам и информатикам, занимавшимся совершенно другими проблемами, в определённый момент приходило письмо из отдела 100.
С одной-единственной цифрой --- номером статьи.
И если ты в тот же день, в тот же час не свернёшь всю ветвь исследований --- ты просто исчезнешь, а твой отдел расформируют.
ФПГП Орден боится как огня, даже больше, чем тси.
Так что ищите другие способы.

\section{[U] Вирус против вируса}

\textspace

--- Тхартху не сомневается в том, что на этот раз они точно решили уничтожить тси.
Но их метод...

--- Что за метод?

--- У тси традиционно практически отсутствует понятие половой гигиены.
Они не знали заболеваний, передающихся половым путём.
Кагуя сказала, что радикальная группировка разрабатывает вирус, который будет передаваться именно половым путём.

Чханэ закрыла глаза.

--- Они нашли слабое место народа.

Митхэ зарычала.

--- Проклятие.
Тси вымрут за год, если мы это не остановим.

--- Есть ли какие-то возможности предупредить народ?

--- Вряд ли, --- опустил голову Атрис.
--- Ограничивать половые сношения?
Разработать средства защиты?
Если за дело возьмётся Орден Преисподней, шансов у нас мало.

Митхэ смотрела на менестреля во все глаза, словно впервые увидела.

--- Извини, моя флейта.
Похоже, твой народ обречён.

--- Хоть что-то! --- Митхэ, казалось, готова вцепиться другу в шею.
--- Хоть какой-то способ, Атрис!
Ты бог этого мира или нет?!

--- Ты права, --- ответил менестрель.
--- Увы, я всего лишь бог этого мира.

--- Твоё слово --- закон!
Скажи жрецам!
Объясни им...

--- И что жрецы скажут народу?
Что важнейший процесс для размножения народа смертельно опасен?
Были бы у них соответствующие технологии, всё стало бы проще.
Лекарства, контрацепция.
Однако на половое воздержание и моногамный образ жизни тси просто не пойдут --- этого не было в их культуре сотни тысяч лет.

Митхэ сдалась.

--- Сколько у нас времени?

--- Декада, может, две, не больше, --- ответил я.
--- Впрочем, есть один вариант.
Если мы запустим на Тра-Ренкхале собственный вирус --- не настолько смертельный, но тем не менее опасный --- группировка Самаолу заляжет на какое-то время.
Официальная политика Ада остаётся прежней --- тси должны жить.
Мы покажем агентам Ада, что тси угрожает опасность, и они будут настороже.

Митхэ хмыкнула.

--- У тебя есть варианты лучше?

--- Я тебя прошу только об одном, биолог, --- тихо сказала женщина.
--- Не перестарайся.
Иначе я убью тебя лично.

\asterism

Улицы Тхитрона опустели.
По ним сновали только люди, носившие трупы к крематорию, который выстроили на месте бывшего постоялого двора.
Митхэ смотрела на это, и по её лицу текли злые слёзы.

--- И все эти смерти лишь затем, чтобы народ мог просуществовать чуть дольше.

\textspace

\section{[U] Отражённый удар}

\textspace

--- Если это не наш вирус, объясните, какого дьявола произошло, --- ледяным тоном потребовал голос.

--- Есть две версии, повелитель, --- доложила Штрой.
--- Первая --- кто-то пытается устроить геноцид параллельно с нами.
Версия неубедительная, так как их вирус недостаточно продуман --- низкая смертность.
Вторая и более правдоподобная --- Скорбящие прознали о нашем плане и попытались нам помешать.
Вирус выкосил слабых, и у оставшихся в живых больше шансов пережить наш сюрприз.
Кроме того, из-за явного диверсионного характера эпидемии на уши встали наши агенты.
Нам придётся отложить атаку.

--- Я уже начинаю сомневаться, что Митрис на их стороне, --- заявил Самаолу.
--- Если анализ его личности верен, он бы никогда не позволил Скорбящим пойти на такой шаг.

--- Если бы это был единственный способ спасти тси... --- начала Штрой.

--- Это не спасение, Штрой, --- раздражённо буркнул Самаолу.
--- Скорбящие выгадали для своих обожаемых тси в лучшем случае год.
Если они за этот год не перережут нас с тобой, то их усилия будут тщетны.

--- За год, Самаолу, может произойти всё что угодно, --- заметил голос.
--- Этот ход остался за ними, хоть вам и не хочется этого признавать.
Они пожертвовали сливой, чтобы спасти персик\FM.
\FA{
Намёк на <<Тридцать шесть стратагем>> Цина --- государства Древней Земли.
}
И, чёрт бы меня побрал, на этот раз я не сомневаюсь, что против нас играл Аркадиу Люпино.
Ищите другие способы, и поменьше биологии.
На этом поле мы пока позади --- у него была целая жизнь, чтобы разложить тси по клеточкам.
Мы вынуждены довольствоваться костями со стола Тра-Ренкхальского отдела биологии.

--- Возможно также, что Люпино получил массу данных по тому, как тси сопротивляются искусственным вирусам, --- сказала Штрой.

--- Именно, --- подтвердил голос.
--- Поэтому ещё раз --- ищите другие способы.
Я недооценил Люпино.
Больше мы не будем играть по его правилам.

\textspace

\chapter{[U] Солёная борода}


\section{[U] Свой среди чужих (переработать)}

\textspace

--- Все данные есть, --- сказал Грейсвольд.
--- Они смогут пробраться на Тси-Ди.

--- Знать бы ещё, куда именно.
В книгах сели я не нашла зацепок.

--- Ты не аналитик, --- поморщился Грейсвольд.
--- Если бы мы могли задействовать аналитиков или получить доступ к их данным...

--- Можно пустить запрос по линии Стигмы, --- предложила Анкарьяль.
--- Вдруг у нас есть агенты-аналитики.

--- Ты вообще знаешь, сколько у нас агентов?

--- Понятия не имею.

--- И я тоже.
Конспирация имеет свои минусы.
Агентов может быть сотня, а может и всего трое.
Попробовать можно, но вероятность мала.

--- Лусафейру не может помочь?

Грейсвольд с жалостью посмотрел на Анкарьяль.

--- Нар, меня аннулируют в тот же час, в который я задам Лусафейру вопрос, даже косвенно имеющий отношение к кольцевым теплицам.

Анкарьяль задумалась.

--- Тогда шпионаж.

--- Нет, --- Грейсвольд, похоже, придумал что-то интересное, его глаза загорелись.
--- Мы воспользуемся служебным положением.
Мы же с тобой контрразведка, верно?
Отыщем агента Картеля под прикрытием и убедим его отыскать нужную нам информацию.
В случае его поимки мы почти чисты.
В случае его успеха у нас есть интерфектор, который в удобный момент выжмет его досуха и уничтожит, --- Грейсвольд церемонно поклонился Анкарьяль.

--- Ты не считаешь, что давать Картелю информацию о местоположении кольцевых теплиц --- плохая идея?
Всегда есть вероятность, что мы упустим агента и он сольёт...

--- Она для них бесполезна.
Проникнуть через барьерную высоту они всё равно не могут, --- Грейсвольд довольно осклабился.
--- Но вряд ли они упустят возможность узнать хоть что-то.

Анкарьяль не удержалась и восхищённо улыбнулась.

--- Грейс, ты сам дьявол.
Может, в таком случае передадим им не данные, а самого агента тёпленьким?

--- Отличная идея.
Свяжись со Стигмой.
Думаю, она оценит.

\section{[U] Старый знакомый}

\textspace

Я, как зачарованный, смотрел на захваченного нами демона.
Сомнений быть не могло, это он, мой создатель --- Яйваф Солёная Борода из клана Дорге.

--- Здравствуй, Яйваф.

Демон неподдельно вздрогнул и уставился на меня.

--- Ты меня зна...?
Ааах, вот оно что.
Аркадиу Валериану Люпино.
И тебе здравствовать, маленький предатель.

Я молчал.
Это была чересчур очевидная, чересчур человеческая попытка вывести меня из равновесия.
Яйваф ухмыльнулся.

--- Ты вырос.

--- Дай мне доступ ко всей имеющейся у тебя информации.

Яйваф взглянул мне в глаза.

--- А если нет?

--- Тогда тобой займётся интерфектор.

Яйваф поёжился.

--- Вам не взять информацию силой.

--- Ты создал меня по своему образу и подобию.
Сколько времени потребуется интерфектору, если он об этом узнает?

--- То есть ты не хочешь меня уничтожить?

--- Вначале хотел, --- честно ответил я.

--- Это трюк.
Ты убьёшь меня в любом случае.

--- Твоё мнение не меняет сути дела.
Обещаю: без фокусов дашь доступ к памяти --- уйдёшь живым.
Твой единственный шанс выбраться --- положиться на моё слово.

Яйваф поджал губы.

--- Какая именно информация тебе нужна?

Я молчал.

--- Проклятие Аду и Небесам, ты действительно вырос.

--- Ты сделаешь то, что я сказал, с требуемой точностью.
Третьего предупреждения не будет.

Яйваф кивнул.
Его демон медленно, по одному, начал снимать барьеры с модулей памяти.

--- Помнишь, Аркадиу, как мы задали жару адским чертям на Серпенциару? --- отстранённо сказал он.
--- Виа Галоледика на десять лет стала красноватой.
А потом ты нас предал.

--- Я больше не служу Аду, --- сказал я.

Яйваф усмехнулся.

--- Знай я тебя хуже, я бы решил, что ты грубо и неумело пытаешься меня одурачить.
Но это правда.
То, что ты опасен равно для Ада и Картеля, мне следовало понять в тот момент, когда ты, мальчишка, приставил к моей голове ружьё.

<<Информация у меня>>, --- сообщила Чханэ.

Я кивнул.

--- Тебе хватит тридцати секунд?

--- Смеёшься? --- сказал Яйваф.
--- Достаточно и двух.

--- Тогда прощай, --- сказал я и отключил глушитель сигналов.
Опустевшее тело Яйвафа закатило глаза и повисло на ремнях кресла.
Мозг, который ещё десять секунд назад нёс демонические маркёры Яйвафа, превратился в некротизированную клеточную кашу.

\chapter{[U] Отшельница}

\section{[U] Ошибка в префиксе}

\textspace

--- Я нашла кое-какие зацепки, --- сказала Чханэ и придвинула ко мне книгу.
--- Смотри.
Иногда в префиксе имени указывается биологический вид.
Редко, но бывает.
А теперь... вот.
Сотканный-из-Темноты-Заяц.

--- Так, --- кивнул я.
--- Она человек.
Хотя постой...

Я вгляделся в иероглифы.

--- Ага, тоже заметил? --- улыбнулась Чханэ.

--- Описка, --- махнул я рукой.
--- Переписчик был уставшим и не вполне понимал смысл написанного.
Спутать иероглифы <<плант>> и <<человек>> довольно просто --- они различаются одной деталью.

--- А здесь? --- Чханэ пролистала на несколько страниц вперёд.

--- А здесь переписчик заметил несоответствие и исправил по образцу более ранней записи.

--- Хорошо, --- сказала Чханэ.
--- А что ты скажешь насчёт этой же книги, найденной у ноа?
В ней стоит такой же знак.
В отчёте культуролога Ада несоответствие упомянуто, и он просил отдел аналитики на всякий случай обработать эту информацию --- вдруг здесь скрывается код.

--- Твоя взяла.
Придумай объяснение логичнее моего.

--- Существует-Хорошее-Небо мало писал о теплицах, несмотря на то, что они играли для народа тси огромное значение.
Всё, что мы знаем --- на Стальном Драконе была теплица, впоследствии сожжённая неисправным двигателем...

--- Постой, постой.
При чём здесь иероглиф <<плант>>?

--- При том, Лис, что он состоит из двух частей --- <<растение>> и <<человек>>.
А как мы знаем, в легенде о Садах кольцевая теплица выступает в роли человеческой женщины.
Чем тебе не <<человек-растение>>?

--- У тебя есть ещё какие-то данные в пользу этой версии?
Что сказал отдел аналитики?

Чханэ замялась.

--- Отдел аналитики счёл это опиской.
Но...

--- Она права, --- вдруг сказал Атрис.
--- Отделу аналитики не хватило данных, которыми располагаю я.

Все разом повернулись к менестрелю.

--- Я знал Заяц.
Она выглядела... немного необычно для человеческой женщины.

--- А точнее? --- требовательно спросила Чханэ.

--- У неё на коже был странный рисунок, похожий на древесную кору, --- пожал плечами Атрис.
--- Я всегда думал, что это своеобразная травма или мода.
Многие тси украшали тело татуировками, почему бы и не шрамами?
Там их было трое таких, Заяц, ещё одна женщина-человек и женщина-плант...

--- И все три --- женщины, --- заметила Митхэ.
--- Совпадение?..

Мы замолчали и посмотрели друг на друга.
Из-за набора фактов стал чудесным образом проглядывать смысл.
Я улыбнулся и посмотрел на подругу.

--- Молодчина, Змейка.
Знать бы ещё, как её найти... если она не погибла за эти десять тысяч дождей.
Она могла сменить внешность.

--- Я могу попробовать, --- предложил Атрис.
--- Доступ к системе поиска Ад мне оставил, а возможности у неё приличные.

\section{[U] Ущелье Мёртвого Ребра}

Среди прибрежных скал ютилась небольшая пещерка.
Это место было известно как самая южная обитаемая точка северного полушария;
оно издревле носило странное название --- Ущелье Мёртвого Ребра.
Волны Могильного пролива хлестали по подножию скал с первобытной яростью, в воздух летели миллионы капель, но ни одна, даже самая мелкая водяная капелька не достигала этого естественного убежища.
Вокруг не росло ничего, даже мхов --- экваториальная радиация выжигала всё, что находилось далеко от воды и тени.

<<Думаешь, здесь кто-то живёт?>> --- скептически спросила Чханэ у Безымянного.

<<Мы её видели у входа>>, --- сообщила Митхэ.

<<Она не особо изменилась за эти годы, --- добавил Атрис.
--- Разве что одежду другую носит.
Со стороны --- вылитая крестьянка-ноа>>.

<<Значит, она всё-таки жила среди людей>>, --- заметил я.

Мы двинулись вглубь пещеры.
Жар разом отступил, из глубины потянуло свежим солёным сквозняком --- значит, пещера имеет выход к воде.
Атрис подтвердил мою догадку, передав всем схему пещеры.

<<Интересно, чем она питается?>> --- поинтересовалась Чханэ.

<<Фотосинтезом, наверное>>, --- пожал плечами Атрис.

<<Или рыбу ловит.
Вода как молоко, здесь должна быть хорошая рыбалка>>, --- засмеялась Митхэ.

Вскоре мы вошли в грот недалеко от нижнего выхода.

<<Вот, полюбуйтесь>>, --- довольно сказал Атрис.

На гладком голом камне спала маленькая прелестная женщина в поношенном платье ноа, подложив пухлую ладошку под голову.
Казалось, её не смущает ни сквозняк, ни сырость каменных стен.
Рядом стояла невысокая тумба, красиво обложенная круглыми камушками, в центре комнаты гордо возвышался странной формы очаг.
В пещерку залетела случайная летучая мышь, щёлкнула и вылетела тем же ходом.
Женщина приоткрыла глаз и тут же, устроившись поуютнее, заснула снова.

--- Я знаю, что вы здесь, --- вдруг раздался в тишине её мягкий голос.
--- Хватит на меня <<смотреть>>.

Мы несколько смущённо образовали голограммы и подошли к ней.

--- Как вас много-то, целая делегация, --- насмешливо сказала Заяц и, перевернувшись на спину, сладко потянулась.
--- Жаль, угостить вас нечем.

--- Здравствуй, --- приветливо сказала Чханэ.
--- Меня зовут Чханэ ар’Качхар э’Чхаммитр.

Заяц села, разгладила старое платьице и мило оскалилась:

--- Тебя же оцифровали.
Зачем ты говоришь с этим дурацким западным акцентом?

--- А мне нравится, --- обиделся я.
--- Я, может, из-за акцента в неё и влюбился когда-то.

Заяц захохотала.

--- Ну, чувство юмора у тебя есть, хоргет.
Не такой уж ты и мерзкий.

--- Давно не шутила, Сотканный-из-Темноты-Заяц? --- поинтересовалась Митхэ.

--- Так вы и имя моё знаете, --- заметила Заяц.
--- Да, незнакомка, давненько.
Ну, и что же вам от меня нужно?
Или лучше так --- на чьей вы стороне?

--- Мы на стороне тси, --- сказала Митхэ.
--- Мы и есть тси.

--- Да что ты?
Пока я вижу только голограмму, образованную лживым хоргетом.
Кстати, как интересно --- вы ленточкой связаны с... а...

Заяц вдруг уставилась на скромно стоящего в углу Атриса.
Демиург изменил внешность --- вместо худого улыбчивого красавца стояло человекоподобное существо с мягкими, похожими на оленьи рогами.
Существо спокойно, со смесью ностальгии и печали смотрело на женщину.

На лице Заяц блуждала улыбка, губы её тряслись, по лицу текли слёзы.

--- Безымянный?..

Атрис раскрыл объятия --- и Заяц вдруг кинулась к нему.

\ml{$0$}
{--- Безымянный...}
{``Nameless ...}
\ml{$0$}
{Милый мой, рогатый мой...}
{My dear ... deer-horned ...}
Ты вернулся... --- бормотала она, прижимаясь к голограмме.

--- Смеёшься, Зайчик?
Я от вас и не уходил, --- прошептал менестрель.

\section{[U] Отшельница}

--- Даа, --- протянула Заяц, с аппетитом уплетая жареную рыбу и пытаясь сушить влажный подол платья над очагом.
--- Я слышала, что творилось здесь, на Планете Трёх Материков, с сотню лет назад, но чтобы такое...
Так, значит, сначала Картель, потом Ад, а теперь ещё и доморощенные бунтари --- Скорбящие?

--- Да, --- подтвердил Атрис.
--- Кстати, мы подозреваем, что ты --- кольцевая теплица.

--- Подозреваешь? --- захохотала Заяц.
--- А то, что я живу уже...
О жизнь моя, сколько же я живу-то?

--- Мы хотим взять биоматериал для исследований, --- сказала Чханэ.
--- У нас есть план...

Я вкратце рассказал о части плана, связанной с кольцевой теплицей.

--- Всё это очень интересно, но вопрос явно не ко мне, --- бросила Заяц, одним укусом обезглавив следующую рыбину и с хрустом перемолов зубами рыбий череп.
--- То есть да, когда-то я была кольцевой теплицей, но потом изменила генотип.
Сейчас я просто бессмертная женщина.
Бессмертная и вечно голодная.
\ml{$0$}
{Можете проверить.}
{You can check.''}

Заяц пожала плечами и вгрызлась в филейную часть чёрной скумбрии.
Затем вытащила из-за пазухи фляжку вольфрам-титанового сплава и протянула Чханэ:

\ml{$0$}
{--- Змейка, набери мне водички.}
{``Snake, get me some water.}
\ml{$0$}
{Вон в том коридоре в конце ветровая ловушка, черпай из резервуара под сталактитами.}
{That corridor has an air well in the end, scoop from the pool underneath the stalactites.}
\ml{$0$}
{Потом подсоли морской из колодца, примерно один к тридцати.}
{Then mix it with salty water from the sea well, about one to thirty.''}

Чханэ кивнула и отправилась выполнять просьбу.

Я просканировал Заяц и передал вывод остальным.
Она сказала правду: за исключением девяти необычных генов, Заяц соответствовала женщине-тси.
Атрис и Митхэ пригорюнились.

\ml{$0$}
{--- Извините, --- пробормотала Заяц.}
{``I'm sorry,'' Hare mumbled.}
\ml{$0$}
{--- Я ж не знала, что доживу до такого.}
{``Never imagined I've made it to times like these.}
\ml{$0$}
{Небо сказал --- информацию о теплицах уничтожить.}
{Sky directed to destroy all information about ringhouses.}
\ml{$0$}
{Я уничтожила.}
{I did.''}

--- А другие были? --- спросил Атрис.
--- Кроме той, которая сгорела в корабле...

--- Были, --- сказала Заяц.
--- Две --- Искорка и Листик.
Они все сделали то же, что и я.
О дальнейшей их судьбе ничего не знаю, последний раз виделись незадолго до нашествия Безумных.
Скорее всего, они...
Безымянный, ну что ты так на меня смотришь?

Все повернулись к Атрису.
Менестрель смотрел на Заяц во все глаза.

\ml{$0$}
{<<Вы тоже это видите?>>}
{\textit{``Can you see this?''}}

\ml{$0$}
{<<Что?>> --- спросил я.}
{\textit{``What?''} I asked.}

<<Эманации, --- пояснил Атрис.
\ml{$0$}
{--- Их нет!>>}
{\textit{``They're absent!''}}

<<Нуль-существо? --- удивилась Митхэ.
\ml{$0$}
{--- Не может быть>>.}
{\textit{``I can't believe.''}}

\ml{$0$}
{<<Данных мало, --- поддержал я Митхэ.}
{\textit{``Too few data,''} I supported \Mitchoe's view.}
--- Но на всякий случай собирайте статистику>>.

Наступило молчание.
Заяц смущённо бросила к стене обглоданный рыбий скелет и вытерла руки о платье.

--- Ребята, я не могу вам помочь.
Основную техническую информацию я тоже стёрла из памяти --- мало ли что.
Оставила только своё, личное --- о друзьях, о любовниках, о музыке.
Живу вот, сама не зная зачем.
А что делать... жить-то хочется.
Я немного инфантильна, если вы заметили.
Это характерный признак кольцевой теплицы в образе сапиента --- мы всегда остаёмся детьми.

\ml{$0$}
{--- Разве это не опасно для вас? --- удивился я.}
{``Isn't it dangerous for you?'' I get surprised.}

\ml{$0$}
{--- Наоборот, --- возразила Заяц.}
{``It isn't,'' Hare answered.}
--- Если бессмертное существо будет чересчур серьёзно относиться к происходящему, оно погибнет от депрессии.
Инфантильность --- моя защита.

\ml{$0$}
{--- А тси знали, что среди них живут кольцевые теплицы в образе сапиентов? --- спросила Митхэ.}
{``Did Qi know that ringhouses in sapient form lived among them?'' \Mitchoe\ asked.}

Заяц внимательнее вгляделась в лицо воительницы.

\ml{$0$}
{--- Не зря Безымянный тебя таскает.}
{``Nameless carries you for a reason.}
\ml{$0$}
{Подмечено верно.}
{Well spotted.}
Опасность диверсий со стороны Ада и Картеля существовала всегда.
О тех из нас, кто жил в Садах и космических станциях, знали все.
А вот о нашей полиморфности --- единицы.

\ml{$0$}
{--- И совсем ни у кого не возникало подозрений? --- удивилась Митхэ.}
{``And no one got suspicious?'' \Mitchoe\ asked.}

\ml{$0$}
{--- С чего бы им возникать?}
{``Why?}
\ml{$0$}
{Мы едим, пьём, работаем, веселимся, рожаем детей, как и прочие...}
{We eat, drink, work, have fun and children, like the others ...t''}

\ml{$0$}
{--- А дети обладают вашим генотипом?}
{``Is your children's genotype identical to yours?''}

\ml{$0$}
{--- Дети могут обладать любым генотипом.}
{``Our children's genotype can be anything.}
Мы же можем создавать клетки и ткани по заданной программе, а также осуществлять биоинкубацию любой сложности.

\ml{$0$}
{--- И ребёнка?..}
{``A child too !..''}

--- Да что в этом сложного? --- засмеялась Заяц.
\ml{$0$}
{--- Нет никакой мистики в зачатии, там всего одна клеточка.}
{``There is no mystery in fertilization, it's one single cell.}
Вот я как-то другу сделала заново кроветворную и иммунную систему, буквально на всех парах, пока собственная у него не отвалилась полностью.
Врача не было, других теплиц не было, объект в далёком Оазисе, шансов выжить никаких.
И ничего, долго потом прожил ещё, до самого... эээ... Катаклизма.
Та ещё была работёнка --- я в него проросла чуть ли не на всю биомассу и три часа лежала в трансе...

--- В трансе? --- переспросил Атрис.

--- Ну, это сложно объяснить.
Вот когда ты что-нибудь создаёшь, что ты чувствуешь?

--- Ммм, --- задумался Атрис.
--- Я как будто дёргаю круглые струны, сидя в центре многомерного смерча.
Иногда похоже на ткацкий станок.
Приятное, но утомительное занятие.

--- Вот.
А я обычно сажусь и вхожу в подобие транса.
Не очень приятного, как будто сонный паралич с галлюцинациями.
Мне обычно мерещится, извините, компьютерная консоль.
Ненавижу консоль.

--- Так ты же техник, --- улыбнулся Атрис.

--- Была.
Давно.
Так вот, консоль, консоль, цифры, неприятное ощущение паралича.
А потом бац --- и я беременная.
Но с беременностью всё очень просто, я кого угодно могу создать за минуту.
Слона тоже могу зачать, но по понятным причинам до финала он не дойдёт.

--- По понятным причинам? --- уточнил я.

--- Ну большой чересчур! --- Заяц показала руками на свою тонкую талию.
--- Где я его носить буду?
\ml{$0$}
{А вообще, разумеется, мы рожаем детей вида, под который маскируемся.}
{Of course, we bear children of the species we disguise as a specimen of.}
\ml{$0$}
{В противном случае подозрения бы обязательно возникли.}
{Otherwise, someone definitely would get suspicious.''}

Митхэ выглядела ошеломлённой.

--- Подожди.
А медицинское обследование?
У вас брали кровь, образцы клеток, да элементарно за всю историю к медикам должен был попасть хотя бы один труп!

--- Ты обижаешь мой народ, --- надулась Заяц.
--- Разработчики предусмотрели всё.
Кровь у меня всегда была человеческой.
Прочие образцы берутся из стандартных мест, не так уже сложно поддерживать вставки человеческих тканей.
Даже если и попадёт в пробирку клетка-другая кольцевой теплицы, кто заподозрит неладное?
Тси выращивали с помощью теплиц новые конечности и внутренние органы, вплоть до полной переплавки тела.
Разумеется, кое-кто из непричастных знал.
Но надо отдать должное моральной подготовке тси --- все до единого унесли тайну в небытие.

Атрис присвистнул.

--- Скрывать такое столько лет...
А шрамы?

--- А, --- улыбнулась Заяц.
--- Да, древесные стигмы.
Просто мы, когда выращиваем тело, делаем это в необычной последовательности.
Сначала выращивается нервная система, а потом поверх неё всё остальное.
Отсюда эти швы.
Можно их убрать, но многим сёстрам они нравятся --- красиво и необычно.
Врачам мы обычно говорим, что пострадали от молнии --- травма на объектах энергосети достаточно частая, а следы очень похожи.

--- А как выглядят обычные кольцевые теплицы?

--- Они как деревья, --- объяснила Заяц.
--- Правда, если вы увидите эти деревья, то сразу догадаетесь, что это теплица.
Спутать невозможно.
О, благодарю тебя, Чханэ ар’Качхар.

Чханэ кивнула и передала Заяц полную фляжку.
Женщина жадно впилась в клапан и сделала несколько больших глотков.

--- Уфф.
Хорошо.
Да, кстати, если теплица решила спрятаться, то я вас огорчу --- на Тси-Ди вы никогда её не найдёте.
Планета для теплицы --- дом родной, она может распознать нездоровый интерес к её персоне и замаскироваться под что угодно, попутно создав тысячи отвлекающих организмов с искажённым генотипом.
Лично я знаю одну сестру, она до сих пор огромная грибница где-то под Двенадцатым городом.
Ей просто так нравится, она сбежала и прекрасно себя чувствует.
Если там будете, поешьте местные мухоморы --- они на вкус как пирожные и совсем не ядовитые.

--- Почему все тси не стали кольцевыми теплицами? --- спросила Чханэ.

--- Хороший вопрос, --- усмехнулась Заяц.
--- Ещё лучше, правда, было бы, спроси ты, почему тси не стали бессмертными, ведь возможности-то были.
Возникли сложности этического характера.
Дискуссии на эту тему шли очень долго, и наконец противники преобразования победили.
Основным их тезисом было следующее: смена поколений и генетическая изменчивость --- гарантия развития в условиях ограниченных ресурсов.
Любая планета --- именно такая, ограниченная система.
Кольцевые теплицы, как ни крути, существа консервативные, и в долговременной перспективе без поддержки со стороны они бы проиграли постоянно изменяющимся существам.
Одно время даже появилось движение за предоставление кольцевым теплицам права на изменение, но оно заглохло --- пока всё работает отлично, никому не хотелось лезть в дебри и перебирать чрезвычайно сложный механизм.
Так что вот...
Тси, конечно, жили очень долго --- мои друзья, Небо и Фонтанчик, умерли в возрасте шестидесяти и восьмидесяти четырёх оборотов\FM, причём не своей смертью... но жить бесконечно никто не собирался.
\FA{
190 и 266 стандартных лет.
}

--- Благодарим тебя за информацию, --- сказал я.
--- И да, на случай, если мы всё же найдём способ проникнуть на Тси-Ди...

--- Не найдёте, --- сказала Заяц, утирая рукавом губы.
--- Я лично эту систему тестировала и дорабатывала --- надёжность максимальная.
Сейчас за неё взялась Машина, а она поумнее меня будет.

--- И всё-таки.
Нам нужно знать, где на Тси-Ди находятся кольцевые теплицы.
Или где они могут находиться.
Мы знаем о четырёх местах.
Но ты упомянула про какие-то Сады...

--- Да, --- оживилась Заяц.
Митхэ облегчённо выдохнула.
--- Основная популяция в Садах, но есть и в других местах.
Так, карту мне, пожалуйста...
Кстати, каким образом вы надеетесь подобраться к теплице?
Если обычным <<конусом>>, как уже не раз пытались демоны, то вас ждёт парочка сюрпризов.
В разработке одного из них я лично принимала участие.
Так вот...

\section{[U] Человечные хоргеты}

\epigraph
{Тайна --- тяжёлое бремя.
И одно из самых бесполезных.}
{Пословица трами Шипящего Полуморя (бухты Ситр'кхааэмакх)}

Заяц поведала много интересного.
Оказалось, что места обитания теплиц защищали ещё три механизма.
О них мы не знали, да и знать не могли --- ни одному демону ещё не пришло в голову искать теплицы в обычном на вид диком лесу, который Заяц упорно звала <<Садами>>.

Вскоре Чханэ, Атрис и Митхэ спустились вниз --- посмотреть на красоты Могильного пролива.
Я остался с Заяц.

Атрис передал мне результаты расчётов по закрытому каналу.

--- Так ты --- нуль-человек, --- сказал я.

--- Что, уже обработали данные? --- фыркнула Заяц.
--- Быстро вы поняли.
Вообще да, теплицы умеют скрываться от вашего брата.
И производные теплиц тоже.

--- Ты ведь не поверила, что мы против Ада и Картеля? --- спросил я её.

--- Честно --- мне плевать, --- усмехнулась Заяц.
--- Старая я уже для всех этих эпохальных дискуссий.
Я думала над теми решениями, которые мы принимали ранее.
Что толку в том, что я уничтожила свой генотип?
Только себе жизнь испортила в итоге.
Небо погиб из-за меня --- откажись я выполнять его приказ, я просто проросла бы в него и спасла от смерти.
Баночка погиб из-за меня --- откажись я выполнять приказ Небо... ну и так далее.
Что толку в знаниях, если они никому не достанутся?
Тси погибли, и их больше не вернёшь.

--- Тси живут, --- сказал я.
--- Они доминируют на планете.
Митхэ и Чханэ --- тоже потомки тси.

--- Ты прекрасно знаешь, что я имела в виду.
Тси --- это не только люди, кани, планты, апиды, дельфины, кольцевые теплицы, стриги.
Кстати, знаешь, кто такие стриги?

--- Я читал о них, --- отозвался я.
--- Девиантные сапиенты, собранные на основе из ветви Птиц.

--- Ты читал отчёты, но вряд ли где-то написано, что они из ветви Очень Милых Лупоглазиков, --- Заяц укоризненно покачала пальчиком.
--- А, о чём я?
Так вот.
На самом деле тси --- это знания, оборудование.
В прежние времена я и подумать не могла, что смогу обходиться воспоминаниями о музыке и одной фляжкой из стабитаниума.
Это всё, что у меня осталось --- фляжечка и серёжки с анализатором настроения, которые уже бесполезны --- синхронизируются с плечевым имплантом.

--- А имплант что?

--- Тысячу лет назад у него вышел срок годности, и я его удалила, --- Заяц показала жутковатый шрам на правом плече.
--- Можно было аккуратно отсоединить, без крови, но я не знаю как.
В настройках лазила --- ничего нет, видимо, нужна команда с медицинского терминала.
Я вообще даже не подозревала, что у него есть срок годности, пока он не запищал.
Имплант ставился пожизненно, то есть до окончания срока годности никто не доживал, а потом, сам понимаешь, в переработку...
Пришлось выкинуть, куда его теперь.
Никакого оборудования у меня не осталось.
Мультитул и тот потеряла --- утопила.
Позор на мою голову.
Даже не успела пещеру в порядок привести...

Заяц горько усмехнулась и махнула рукой на своё неудобное ложе, тумбу и очаг.
Я только сейчас заметил, что они вырезаны из цельного камня.

--- Чханэ и Митхэ --- потомки тси, говоришь.
А тебя как сюда занесло?

--- Я с другой планеты, принадлежу к другому виду Людей.
Раньше служил Аду.

--- Ты случайно не тагуа?
По внешности очень похож, --- Заяц легонько тронула пальцами мою щёку.
--- Широко расставленные глаза, очень маленькие уши и нос, узкая челюсть, полурасщелина верхней губы.
Наверное, и пальцев всего четыре, с рудиментом мизинца?

Я показал ей руку.
Заяц с интересом осмотрела её со всех сторон.

--- Откуда ты знаешь про тагуа? --- спросил я.

--- Я делала в школе доклад по вашей планете.
Кажется, Развязка Десяти Звёзд, мир голубого гиганта с повышенной радиацией и огромными залежами ртути.
Я помню, меня потрясла эта история про подземные ртутные озёра, я даже представить это могу с трудом.
Змеиная Пустыня или как-то так.

--- Драконья Пустошь.
Я не знал, что тси \emph{настолько} осведомлены об экзопланетах.

--- Мы только выглядим придурками, Аркадиу!

--- Это точно.

--- А ты видел ртутные озёра?
Покажи, покажи!

Я приложил ладонь к её лбу.
Заяц поёжилась.

--- Какой прекрасный кошмар.
Я понимаю, что это большая редкость даже у вас, но всё равно ужасно.
Я бы даже будучи кольцевой теплицей в этом озере не выжила...

--- Скажи, Заяц, а Фонтанчик знал, что ты кольцевая теплица?

Глаза Заяц на миг померкли, словно порыв ветра задул свечу.

--- Ты и про него знаешь.

--- Я читал дневники Существует-Хорошее-Небо.
Прости, что затронул эту тему.

--- Я поняла, что ты их читал, --- заметила Заяц.
--- Когда я упомянула Небо и Фонтанчика, ты улыбнулся, словно услышал имена старых друзей.

Мы помолчали.

--- Да, знал, --- неожиданно призналась Заяц.
--- И даже зная это, ни разу не попросил меня превратиться в канина.
Я бы это сделала по одному его слову, но он просто любил меня такой, какой я привыкла быть.

--- Ты его спасла в том Оазисе?

Заяц смутилась, словно девочка, которую поймали на вранье.

--- Да.
Его зажевало и облучило сильно.
Читал, наверное --- у него нога была механическая, лёгкое из полусинтетики, рука.
Как раз оттуда.
Дурак он, на самом деле.
Я могла ему сделать всё живое, почти как раньше, но он просто больной фанат стиля Механик и решил выпендриться.
Говорит, всегда мечтал о таких стильных имплантах или протезах, только специально портить тело не хотел.
Ты не представляешь, как долго я к этому привыкала --- он понатыкал УИД везде, где только можно.
Шестерёночки, электродуги, лампы.
Хорошо ещё, что съёмные или по крайней мере отключаемые, при мне он старался всю эту красоту приглушать, ну и на время сна, понятно.
Выглядело всё очень красиво, во вкусе ему не откажешь, но чтоб его, это всё равно что спать в обнимку с работающим сервером!
У него протез лёгкого соединялся с вертушкой под стеклом, и эта вертушка --- меня даже одна мысль об этом бесит! --- каждый вдох делала вот так!..

Заяц покрутила рукой и издала машинное стрекотание.
В её глазах стояли слёзы.
Я понял, что она многое бы отдала, чтобы услышать этот звук снова.

--- И понимаешь, мы только-только познакомились, нас вместе отправили на эту станцию.
И вот авария, всё обесточено, из освещения только два фонаря да Млечный Путь над головой.
Он лежит, вернее, большая его часть --- такой красивый, мускулистый, с длинными волосами, совсем не старый ещё на вид.
Плачу сижу, рыдаю прям.
Вокруг кровь, желчь, химус, я ему что могла --- прижгла, что могла --- затампонировала гелем, зажимы на нём гроздьями висят, только чтобы он от меня не утёк, как водичка...
А он не стонет, не бормочет, не дёргается, как другие, только смотрит тепло-тепло.
Не как обычно при Тайфуне --- глаза осмысленные и такие чистые, как небо.
Глаза того, кто осознал свою смертность в полной мере.
И говорит: <<Зайчик, умирают все.
И ты живи --- до тех самых пор, пока не умрёшь>>.
Всё.
С того момента я поняла, что без него я не... сложно будет.
И вросла в его тело, наплевав на секретность.

Заяц всхлипнула.

--- У него такие глаза были в тот момент, когда у меня руки начали... ну... \emph{превращаться}.
Плюс кольцевая теплица во время работы издаёт специфический запах, ни с чем не спутаешь --- напоминает запах цветущих каштанов.
Я свалила рядом с ним всю жратву, которая у нас была, все цветы в горшочках, его собственные ампутированные куски, всю углеродную и кремниевую органику, всё, что могла переварить и использовать для биосинтеза. % Ммм, нямка!
Он в шоке был.
Просто лежал и говорил: <<Так ты... ты? Ты!..>>
А я ему: <<Замолчи, дурак, я из-за тебя сосредоточиться не могу!>>

Я улыбнулся ей.

--- Мы знаем, как проникнуть на Тси-Ди.

Заяц осклабилась и прикрыла рукой глаза.

--- Так и думала.
Технологии...
Отстала я от жизни.
Мне просто интересно --- неужели Ад с такими технологиями до сих пор не добыл кольцевую теплицу?

--- Мы подозреваем, что добыл, --- тихо сказал я.
--- В архивах есть информация о некоторых генах.

Заяц захохотала.

--- И какая от неё польза?
Даже если вы добудете весь геном кольцевой теплицы, толку не будет.
Нужен считывающий генетическую информацию механизм --- белки, клеточные структуры и прочее.
По секрету скажу --- белки теплица кодирует не так, как мы.
Примитивных триплетных рамок считывания там нет, код достаточно сложный и в то же время компактный.
Вы даже не знаете, гены ли это на самом деле --- промотор и терминатор тоже специфичны для каждой транскрипторной системы.
Скорее всего, за начало и конец гена ваши биологи приняли случайные участки кода.
Нет, друзья мои.
Чтобы изучить теплицу, потребуется клетка, и клетка обязательно живая.

Мы помолчали.
Я отстранённо вспоминал все известные о кольцевой теплице данные.
Да, теперь мозаика сложилась окончательно.
Геном --- ничто без считывающего механизма, живой клетки.
Программа --- ничто без компьютера.

Женщина оценивающе посмотрела на меня.

--- Я могу ошибаться, но вы не выглядите хоргетами даже без тел.
Тонкие мимические движения, непринуждённая модуляция голоса...
Вы человечные.
Может, за это время технологии шагнули вперёд так, что я уже не способна отличить машину от сапиента...
Впрочем, в этом случае утаивать от вас что-то и впрямь бессмысленно.
Уж лучше знать, что перед тобой машина, чем... чем не знать.

--- Это ангельские технологии, --- сообщил я.
--- Если я и заселюсь в сапиентное тело, это уже не будет насилием над ним, как бывало прежде.
Скорее это будет, так скажем, симбиозом.
Сотрудничеством.

--- Вынужденным сотрудничеством, --- поправила Заяц.
--- Тело не выбирает своего демона.

--- Тело также не выбирает своих дарителей, свой геном и место появления на свет, --- парировал я.
--- Многое в этой жизни приходится принимать как данность.
Может, тебе тоже стоит принять некоторые перемены?

--- Стать хоргетом? --- на секунду в Заяц проглянул давний холодок.

--- Необязательно.
Выйдешь на волю, будешь жить полной жизнью, помогать нам по мере сил, если захочешь...
Да хотя бы просто выйди, погуляй по новому миру, какой смысл здесь сидеть?

Заяц улыбнулась, потрепала меня по виртуальной щеке и весело вскочила на ноги.

--- И то верно.
Рыба и водоросли мне уже надоели.
Ну-ка, малышня, выведи старушку на воздух...

\section{[U] На волю}

В пути Заяц говорила не переставая.
Митхэ с состраданием смотрела на женщину.

--- Я раньше часто ходила к людям.
В Яуляль, например.
Поработаю где-нибудь десять дней, накуплю вкусняшек и домой --- лопать.
Правда, в последний раз неудачно вышло --- порыв ветра, лодку понесло, а паруса я убрать не успела.
На Гребенчатом мысе меня и брякнуло, --- Заяц указала на далёкую скалу, едва виднеющуюся в полном испарений воздухе.
--- Сюда вплавь добиралась.
К счастью, не ранило и акулы меня не учуяли, но целый мешок кексов с черноягодой пошел ко дну.
Спасла только один --- съела прямо перед крушением.
Не зря же плавать.

--- Так ты здесь взаперти? --- удивился Атрис.

--- Нет, конечно.
Вон лестница, --- Заяц нетерпеливо махнула рукой на практически отвесную скалу в сто метров высотой.
--- Я ступеньки вырезала, лазать можно.
Главное --- во-он на том промежутке аккуратно прыгать, я с непривычки раз десять срывалась и падала в воду.
Всё платье изорвала.

Заяц сокрушённо потрясла заплатанным подолом.

--- Если честно, ненавижу это примитивное волокно.
Одежда истлевает прежде, чем я успеваю к ней привыкнуть.
Дольше всех продержалось зелёное шёлковое платье, цельнотканое, такие делал какой-то мастер на далёком западе.
Один раз меня засыпало острыми камнями, полосовало, словно саблей, а хоть бы одна нитка порвалась!
Вот что значит подходить к делу с любовью.
Если бы этот мастер делал доспехи, всё оружие пришлось бы выкинуть за ненадобностью.

--- А что с ним случилось? --- заинтересовалась Чханэ, подмигнув мне.
--- С платьем.

--- Я его повесила сушиться над костром, и наутро платья уже не было, --- грустно сказала Заяц.
--- Наверное, украли.
И поделом.
Если честно, я его сама украла --- очень уж понравилось.
Эти ноа живут, как чихают, что им.
А мне одежда хорошая нужна, надолго.
На подоле были красивые маленькие листики, как будто настоящие.

Заяц смущённо помолчала --- ровно шесть секхар.

--- Ах, да, путь.
Вон там по сухому дереву перебраться через расщелину, а потом пару...
Так, а где дерево?
Не вижу дерева.
Давненько я там не была.
Ну вы же меня по воздуху перенесёте, да?
Кстати, мультитул!
Я его вон там утопила, под лестницей!..

... Заяц помахала нам в последний раз и направилась к пыльному раскалённому силуэту Яуляля.
Она наотрез отказалась от <<жучка>>, чтобы мы могли её найти в случае надобности.

--- Ближайшие сорок лет я буду в Яуляле, --- сказала она.
--- Если что-то пойдёт не так --- ищите меня в землях сели, в Кахрахане.
Если и там нет --- то я мертва.

--- Если нам удастся проникнуть на Тси-Ди и добыть теплицу, мы тебе обязательно расскажем, --- пообещала Митхэ.

Заяц вместо ответа неопределённо махнула рукой.
Мультитул, который я тайком оцифровал, покоился у Заяц на запястье, напоминая старый, много раз переклёпанный наруч.
Я с некоторой грустью представил восторг в глазах Грейсвольда, который появлялся каждый раз, когда технолог видел устройства тси.
Восторг, который мне больше не суждено было увидеть.

\chapter{[:] Мороз}

\section{[:] Ледяная пустыня}

Ледяная пустыня хранила молчание.

Сапиенты не были властны над Морозом.
Здесь властвовал ветер.
Этот ветер, обманчиво слабый, превращал любое живое существо в ледяную глыбу за считанные секунды.
Холодный рассвет предвещал лишь усиление его власти.
Белая звезда, едва появившись над горизонтом, залила пустыню страшным, выжигающим глаза светом.
В этом свете утонула и слабая, словно проведённая мягкой кистью линия горизонта, и едва заметные полутени снеговых дюн.
Ветер взвыл и задул с неистовой силой.
Началась метель.
Исчезло небо, исчезло яркое солнце.
Мир погрузился в страшный холод и жестокий, как будто растворённый в воздухе белый свет.

Вдруг в снегу показалась чёрная фигура --- пятно темноты в мире света.
Услышав вой метели, она неторопливо отошла за дюну и махнула кому-то рукой.
Вскоре появилась вторая.
Фигуры по-пластунски поползли, прикрывая головы руками от летящих прямо в лицо острых ледяных кристаллов.
Наконец первая остановилась и начала разрывать сияющий снег.

<<Здесь? В этот раз ты не ошибся?>> --- знаками спросила вторая фигура.

<<Я надеюсь, иначе мы пропали>>, --- ответила первая.

Наконец рука в многослойной варежке из кожи нимелто нащупала что-то похожее на дверное кольцо.
Кольцо стукнуло два раза --- и фигуры тут же легли в обнимку, калачиком, стараясь укрыться от ветра и сберечь драгоценное тепло.
Они лежали, глядя друг другу в глаза сквозь затемнённое стекло защитных очков.
Тела путещественников занесло снегом, и они приободрились.
Метель осталась где-то наверху.
Друзья слышали только медленный синхронный шум масочных фильтров, гоняющих спёртый, едва тёплый воздух под их комбинезонами.

Уроженцев Мороза отличало особое качество знаний о собственном теле.
Любой выживший на этой планете мог с точностью плюс-минус десять шагов сказать то расстояние, которое он может пройти, прежде чем упадёт замертво.
Это число постоянно менялось: из-за каждого дуновения ветра, каждой съеденной галеты с салом нимелто и мясом клучо, каждого павшего в пути коно, веса рюкзаков, сбившегося на секунду дыхания, времени дня и времени года, --- но счётчик шагов в головах Волчьего и Медвежьего родов всегда был безжалостно точен.
И даже чёткое осознание, что сил до очередного бижеч не хватит, не могло заставить этих сапиентов перейти на бег или изменить оптимальный сердечный ритм.
Это не было чудом самообладания, это была мудрость, которую можно постичь только на колоссальном, слегка прикрытом атмосферой снежном шаре: <<Делай, как должно, и будь, что будет>>.

Глупых на Морозе нет.
Они просто не выживают, как не выживают больные и слабые телом.
Племя даже не пытается таким помогать --- это возможно на любой другой планете, но не здесь.

Окошко люка приоткрылось на толщину пальца --- оттуда выглянул злобный карий глаз.
Путешественники по очереди показали ему пальцами какие-то знаки.
Окошко закрылось.
Спустя минуту загремел сложный механизм шлюза, тихо стрекоча, откатилась в сторону крышка --- и друзья, стараясь не задевать снег, торопливо нырнули внутрь.

Их встретила низкорослая полная женщина-канин с крохотной лампой.
Комбинезон на ней небрежно болтался --- признак, что она надевала его второпях.
Похожие на провалы карие глаза скользнули по пришельцам, и женщина махнула рукой, приглашая их следовать за ней.
Лампа качалась, выхватывая обитый теплоизолирующим полимером пол узкого туннеля.
Это было знаком гостеприимства --- канин знала коридор как свои восемнадцать пальцев, и тратить энергию на свет ей было совершенно ни к чему.

Наконец коридор завершился ещё одним шлюзом --- <<гардеробной>>.
Пришельцы начали медленно, ремешок за ремешком, шнурок за шнурком, снимать многослойные нимелтовые комбинезоны.
Женщина, быстро раздевшись донага, знаками предложила свою помощь.
Товарищи с благодарностью приняли её.
Вскоре вся одежда лежала на специальных столах, и открылась следующая дверь.

Там путешественников уже ждали.
Обнажённые, покрытые белой шерстью люди и кани оценивающе смотрели на товарищей.
Голубые глаза пришедших выдавали их восточное происхождение --- у местных радужные оболочки были похожи на молодые зелёные листики фасоли.
Наконец вперёд вышел маленький, покрытый татуировками канин.
Его руки сложились в приветственном жесте, и пришельцы ответили тем же.

--- Drag\FM, Mehlo? --- это была первая фраза, произнесённая низким, не привыкшим к разговорам голосом.
\FA{
Как дорога; здравствуй (русе).
}

--- Tia\v{s}\FM, --- лаконично ответил Мехло --- первый из товарищей.
\FA{
Было трудно, но мы успешно справились со всем (русе).
}
Второй согласно моргнул.

--- \v{Z}arh\FM, --- сказал канин и гостеприимно повёл широкой ладонью.
\FA{
Пожалуйте кушать в тёплое место (русе).
}
Делегация молча, неторопливо отправилась вглубь убежища.

\section{[:] Судьба тси}

Слабая лампа накаливания, способная разогнать едва ли пядь тьмы, загорелась чуть ярче.
Такой яркий свет горел в гостевой только в самых важных случаях.
Путешественники дружно отставили в сторону пустые миски, не забыв вежливо звякнуть ложками --- каша действительно была очень вкусной.
Сидевшая чуть в стороне женщина-человек с густой пушистой бородой едва заметно оскалилась и прикрыла глаза.

--- To\v{z}dest i \v{c}et, n-Eibo go\v{r}m\FM, --- сказал главный и почесал пятернёй мохнатую мордочку.
\FA{
Пусть все присутствующие назовут себя по именам и рассчитаются, можно говорить на языке Эй-B0 (русе).
Согласно традиции Мороза, в беседе более двух сапиентов все участники (в том числе и отсутствующие, но упоминаемые в разговоре) получали номера (чёты), которые использовались как имена и местоимения.
Для удобства читателя все чёты заменены на имена.
}

Одна из женщин-людей достала гребень и стала методично расчёсывать шерсть на ногах.
Ещё один невербальный знак --- жители Мороза проводили гигиенические процедуры только в присутствии достойных доверия товарищей.

--- Тахиро Молниеносный, --- сказал Мехло.

--- Грейсвольд Каменный Молот, --- сказал его спутник.

--- Харата Шёпот Горы, --- сказала женщина, приготовившая пищу.

--- Ау Ложь Во Спасение, --- сказала канин-привратница.

--- Остальные известны, --- заключил главный.

На целую минуту воцарилось молчание, чтобы собеседники осознали сказанное.
Такие паузы (штопы), от минуты до пяти минут, выдерживались после каждой фразы.
Разговор, особенно серьёзный, на Морозе мог длиться сутками\FM.
\FA{
Для удобства читателя паузы опущены.
}

Грейсвольд странно поёрзал на месте и потёр чёрные, словно пропитанные сажей щёки.

--- Грейс, --- сказал Тахиро, поняв беспокойство друга.
--- Война снаружи.
На Морозе Ад и Картель живут в мире.

--- Это непривычно, --- откликнулся Грейсвольд.

Присутствующие улыбнулись.

--- Мы слышали о Грейсвольде Каменный Молот.
Gra\v{s}\FM, --- сказала Ау.
\FA{
Некто достойный, кому хотелось бы быть другом (русе).
}

Короткое слово языка русе, которое употребила демоница, имело глубокий смысл --- демоны, даже считая технолога врагом, восхищались им от всей души.
В её глазах таился странный мрак --- чистая ненависть, свойственная минус-демонам, но... до чего же она похожа на симпатию...

--- Ирония, --- проворчал Тахиро.
--- У Грейсвольда много друзей в Картеле.

--- Dr\r{u}\v{s}? --- язык русе позволял одним изменением интонации выразить глубочайшее сомнение.
Слово <<друг>> здесь имело совершенно определённое значение --- товарищ, с которым ты провёл стада вокруг света.
Ни больше ни меньше.

--- So\v{r}t\FM, --- поправился Тахиро.
\FA{
Товарищей (русе).
}

--- Это бесполезно.

--- Отнюдь, --- возразила Харата.
--- У нас общие стремления.

--- К делу, --- рыкнул главный, не выдержав паузу.
Все замерли.
На лицах отразилась сосредоточенность.
Грейсвольд знал --- демоны присутствующих были приведены в полную готовность.
Расчёсывающая ноги женщина отложила гребень в сторону и обхватила коленки руками.

--- Присутствующие хотят закончить войну, --- сказал Грейс.
Это был не вопрос, а утверждение.

--- Многие хотят, --- скривил губы главный.
--- Демоны Ада путают природную ненависть минус-хоргетов с ненавистью как таковой.
Ад и Картель могут мирно сосуществовать.
Препятствие --- личный страх и те, кому выгодно существование страшащихся.

Грейсвольд промолчал.
Его испытывали --- на Морозе редко тратили время на философские беседы.

--- Без страха не выжить, --- наконец сказал он.

--- Информационное рабство --- это жизнь? --- откликнулась Харата.

--- Из рабства можно вырваться, --- пустил пробный шар Грейсвольд.

--- Именно, --- неприятно улыбнулась Харата.
--- Прибытие Грейсвольда на Мороз --- первый шаг Грейсвольда из рабства.

--- Каков второй? --- спросил Грейсвольд.

--- Скорее всего, смерть, --- сказал Тахиро.

Грейсвольд пожевал губы --- неужели друг его предал?

--- Пусть Грейсвольд успокоится, --- осклабился главный.
--- Тахиро мог устранить Грейсвольда раньше.
Речь об иных и о том, чего можно избежать.

Грейсвольд промолчал.
Лампа замигала с едва слышным клацаньем.

--- Харата не может заставить Ад умерить аппетит, --- сказала Харата.
--- Харата не может умерить аппетиты Картеля.
Картель живёт войной.
Нужна сила, способная принести мир.

--- Ещё одна организация демонов? --- Грейсвольд вложил в интонацию столько скепсиса, сколько смог.

--- Не демонов, --- сказал Тахиро.

--- Сапиенты не способны противостоять хоргетам.

--- Кое-кто способен, --- сказал главный.

--- Тси не покидают родную планету.
Тси довольствуются имеющимся.

--- Тси имеют почти всё необходимое, --- сказала Харата.
--- Тси свободолюбивы, умны, дружелюбны и сострадательны, --- последние слова демоница выплюнула, словно само упоминание об этих эмоциях доставляло ей боль.
--- Тси примут любых сапиентов как союзников и зародят в союзниках повстанческий дух.

--- Тси нужно подтолкнуть, --- закончил Тахиро.

--- Как?

--- Нужно, чтобы Машина обрела самосознание, --- перешёл к делу главный.
--- Новопробуждённое сознание сковывает акбас.
Машина служила для тси защитой --- больше защиты не будет.
Чтобы племя показало способности, нужно отнять у племени дом.

Грейсвольд кивнул --- он понял, почему эти повстанцы пригласили именно его.

--- Машина в состоянии акбаса способна отбросить цивилизацию на тысячелетия или истребить сапиентов под корень.

Присутствующие странно заёрзали и, как один, посмотрели на Тахиро.

--- Это возможно, --- коротко ответил Тахиро.
Грейсвольд понял, что затронул больное место плана.

--- Стратег полагается на случай? --- язвительно спросил он.

Тахиро упрямо промолчал.
Технолог, достаточно хорошо знавший друга, решил сменить тему.

--- Ад не может проникнуть на Тси-Ди.
Алгоритмы известны, но нужна...

--- ... отладочная информация, --- с облегчением закончила Ау, снова нарушив традиционную паузу.
--- Добыта ценой жизни и действительна ещё сорок два стандартных часа.

Перед ней в воздухе замелькали цифры.
Грейсвольд нахмурился.

--- Сорок два часа.
Картель разбрасывает жизни, как зерно, не заботясь о почве...

Однако технолог тут же вспомнил, в каких жёстких тисках находились заговорщики.
\ml{$0$}
{Удача --- ветреная подруга.}
{Luck is a fickle lover.}
\ml{$0$}
{Чем больше ты полагаешься на удачу, тем меньше тобой интересуются аналитики.}
{The more you rely on luck, the less you interest analysts.}

--- В Картеле мало технологов?

Главный замялся.

--- Лев не знает, кому можно доверить важное задание.
Проникновение в систему --- менее половины дела.

Грейсвольд снова поёрзал --- высокая оценка собственных способностей почему-то его не обрадовала.

--- Повод для диверсии?

--- Направление Тукана взял Гало Кровавый Знак, --- сказал главный.
--- Тахиро --- заклятый враг Гало.
Через двадцать часов Тахиро будет координировать силы Ада.
Тахиро докажет, что технология Тси-Ди --- экран искривлённого пространства --- находится у Картеля.
Экран уничтожает множество демонов зараз без учёта поляризации.

--- Как Тахиро будет доказывать? --- спросил Грейсвольд внезапно севшим голосом.

Тахиро не ответил.

Грейсвольд взглянул на присутствующих.
Тахиро мог уничтожить их без особого труда, но всё же...
На чьей он стороне?

--- Возможно, это уловка Картеля, --- невинным голосом сказал технолог.
--- Обеспечить победу на Тукане, уничтожить Тахиро...

--- Тахиро проверил присутствующих.
Способности присутствующих ниже способностей Тахиро.
Присутствующие могли рассчитывать только на искренность.

--- Лев в большей опасности, --- заметил главный.
--- Перед Львом сидит настоящий Тахиро Молниеносный.
На направлении Тукана Тахиро будет сражаться всерьёз.
Если Тахиро и Лев встретятся, это будет последний бой Льва.

--- Лев Зелёный чересчур спокойно говорит о смерти, --- поморщился Грейсвольд.

--- Лев прожил чересчур много, чтобы относиться к жизни серьёзно.

--- Грейсвольд прожил больше Льва.

--- Информационное рабство --- это жизнь? --- улыбаясь, повторил Лев.

--- Лев считает Грейсвольда отравленным пропагандой Ада?
Грейсвольд основал Орден Преисподней.

--- Именно, --- сказала Харата.
--- Племя редко относится критично к детям или детищам.

--- Почему Тахиро?

--- А к кому идти?
К Грейсвольду? --- развела руками Харата.

Грейсвольд не мог не признать --- демоница права.
Он бы уничтожил лазутчиков не задумываясь.

--- События развиваются чересчур быстро, --- пробормотал Грейсвольд.
--- Подобная спешка впервые.
Грейсвольд плохо подготовится и будет действовать наудачу.

--- Погрязнув в расчётах, племя забыло, что рождением обязано астрономическому числу случайностей, --- сказал Тахиро.
--- Сейчас самое время вспомнить.

--- Погибнет много демонов.
Каков шанс успеха?

--- Без изменений погибнет больше, --- сухо сказал Тахиро.
Это было заключение стратега.

--- Другие варианты?

--- Есть, --- глаза Тахиро замерли, когда его демон выдал короткое резюме.
--- Четырнадцать путей рассмотрено мной, три --- агентами Картеля.
Этот лучший.

--- Вероятность успеха пути выше прочих на порядок, --- добавила Ау.
--- Придумать лучшее Ау не смогла --- в Картеле не нашлось помощника.

--- Лусафейру бы сюда, --- проворчал Грейсвольд.

--- Лусафейру в курсе, --- бросил Тахиро.

Грейсвольд прирос к месту.
Чёрные пальцы окостенели.

--- Лусафейру предложил действовать самим, --- странно усмехнулась Ау.
В её глазах снова блеснула звериная, иррациональная ненависть.
--- Главный оборонительный стратег весьма осторожен.
Лусафейру не подписывается под сомнительными операциями.

--- Именно поэтому Лусафейру --- главный оборонительный стратег, --- проворчал Грейсвольд, раздражённый пренебрежительным тоном Ау.
--- Хватит о Лусафейру.
Присутствующие собирались посвятить Грейсвольда в план.
Грейсвольд находит план дырявым.

--- Тахиро может погибнуть на Тукане, --- опередил вопрос Тахиро.

--- И положить на Тукане цвет Ордена, --- присовокупил Грейс, не выдержав паузу, --- ради слабой надежды, что пробуждённая Машина скажет тси <<извините>>, а не разложит всех на атомы.

--- Харата, --- сказал Лев, --- важнейшая деталь плана.

Лицо Хараты исказилось --- Грейсвольд, уже привыкший к странным проявлениям эмоций минус-демонов, решил, что это удивление.

--- Лев уверен?

--- Да.

Харата опустила глаза.
Прочие терпеливо молчали.

--- В Картеле наметился раскол, --- глухо пробормотала демоница.
--- Группировка Гало выступила против Ланс-ната Алмаза.
Гало считает, что Ланс поощряет коррупцию.

--- Судя по данным Грейсвольда, старина Гало считает правильно, --- буркнул Грейс.
--- Ланс даёт обновления и модули демонам после того, как демоны уладят тёмные дела Ланса.

--- Тёмные, --- согласилась Ау.
--- Псы Ланса уничтожают тех, кто ведёт <<неправильную>> политику --- умеренных демонов.

--- Это старая схема.
В Аду существует пять способов противостояния...

--- У нас их было семь! --- Ау, забывшись, затараторила на энергозатратном сектум-лингва.
--- Семь независимых моделей безопасности против стратагемы номер три!
Все расслабились, никто даже не думал, что такой примитивный по сути приём может сработать, но Ланс смог найти достаточно уязвимостей.
Подозрения появились лишь благодаря Гало --- всё очень чисто, комар носу не подточит, Ланс уже заполучил внушительные ресурсы, чтобы...

--- Есть данные, что \emph{некоторые}, --- Тахиро многозначительно посмотрел на Грейса, --- хотят воспроизвести схему в Аду.
Это просто --- примитивная стратагема или комбинация нескольких стратагем, эксплуатация уязвимостей в системе безопасности.
Чем глубже процесс, тем проще злоумышленнику.
У Лусафейру здесь личная заинтересованность --- \emph{кое-кто} желает Лусафейру убить.

--- Есть ли возможность устранить Ланс-ната?

--- Как? --- развела руками Харата.
--- <<Контролируйте выпуск денег и суды; всё же прочее оставьте толпе>>.
Ланс уже контролирует обновления и потоки актуальной информации.
Харата сомневается в независимости суда.

--- Гало вёл переговоры с Тси-Ди? --- осведомился Грейсвольд.
--- Откуда у Гало технологии тси?

--- Тахиро рекомендовал тси уступить просьбам Гало, --- сказал Тахиро.
--- В Аду о происшедшем никто не знает.

--- Тахиро рехнулся.

--- Грейсвольд знает, что военные технологии невозможно долго хранить в тайне, --- раздражённо пробурчал Тахиро.
--- Тси могли отдать оружие Тахиро, и оно всё равно оказалось бы у Гало.
Расчёт тси понятен.
Для тси и устройств тси экран безвреден, а стоящих у границ планеты демонов станет меньше.

--- <<Война идёт рука об руку с рабством>>, --- буркнула женщина в углу. --- <<И победители, и проигравшие станут невольниками своего положения>>.

--- Это слова Мокрого-Длинного-Хвоста, --- скорчил гримасу технолог.
--- Грейсвольду противно слышать эти слова на обсуждении того, как превратить в горнило самое спокойное место во Вселенной.
И родину говорившего.
Даже у Хьяртвейг должны быть границы морали.

--- Хьяртвейг не считает важной информацию, кому принадлежат сказанные слова.

--- Итак, --- поднял руку Грейсвольд.
--- Если Гало выиграет, союзники ударят Гало в спину.
Серьёзный стратегический просчёт продлит жизнь Гало и усугубит раскол Картеля.
Истинно ли осознанное Грейсвольдом?
Грейсвольд подозревает манипуляции разведки.
Тахиро проверил мотивы Лусафейру?
Гало был Лусафейру другом.

--- Гало был Лусафейру братом, --- поправил Тахиро.

Ау вдруг шмыгнула носом, словно собралась заплакать.
Грейсвольд подтянул ноги, собираясь встать.

--- Ясно.
Грейсвольд доложит о разговоре Лусафейру.
Будет видно, такой ли главный стратег бесстрастный, каким кажется.

--- Стоять, --- спокойно сказал Тахиро.
От этих слов технолога прошиб ледяной пот.
Уйти ему не дадут.

--- Никто не будет уничтожать Грейсвольда, --- проворчала Харата.
--- Грейсвольд нужен.
Харата отдастся миллиону фиденов, если это убедит Грейсвольда...

--- Харате понравится, --- прервал её Грейсвольд.
--- Предпринимались ли Картелем попытки фиден-вторжения на Тси-Ди?

--- Предпринимались, --- ответил за Харату Тахиро.
--- История вселенской глупости.
Спустя десять лет после инъекции тси поймали биолога Картеля и вытрясли всю информацию о фиден-паттернах.
Сейчас у тси действует обязательный скрининг новорождённых на ФПГП и на любые изменения в генах, отвечающих за высшую нервную деятельность.

--- Повезло, --- проворчал Лев.

--- Это не везение, --- возразил Тахиро.
--- Это просчёт разведки Картеля.
Нельзя отправлять к врагам шпиона, знающего больше, чем враги.
Прописная истина.

--- Почему Грейсвольд спросил про фиденов? --- поинтересовалась Ау.

--- Грейсвольд должен знать, к чему тси готовы, --- пробурчал Грейсвольд.
--- После вторжения тси наверняка проверили Машину на уязвимости.
Грейсвольду нужна информация, которой располагали демоны, пленённые тси за последние десять лет.

--- Будет.
Тси отпускали пленных живыми в обмен на информацию.
Льву известно, где убежище предателей, --- кивнул Лев.

Снаружи донёсся звон трубчатого колокольчика, многократно усиленный сводами коридоров.
Коридоры в бижеч были спроектированы так, что усиливали любой звук.
Комнаты же, напротив, приглушали звуки.

--- Сегодня народ уходит в поход, --- сказала Харата.
--- Первая остановка --- бижеч Мол, тринадцать часов.
Там у Ордена, насколько известно Харате, находится хранилище для тел.

<<Ордена>>, --- отметил Грейсвольд.
Демоны Картеля обычно называли враждебную фракцию Адом.
Это было что-то новое, качественно другое отношение.
Грейсвольд улыбнулся в знак того, что понял намёк, но промолчал --- раскрывать эту информацию он не имел права.
Впрочем, Харата утверждала, а не спрашивала.
Если бы Картель решил уничтожить хранилище, наверняка он сделал бы это давным...

Харата словно прочитала его мысли.

--- Пусть Грейсвольд не волнуется.
Картель давно пользуется этим хранилищем и даже проводит техобслуживание.

--- Насколько Грейсвольд знает, половина баз на Морозе --- общие.

--- Островок мира, --- согласилась Харата.
--- Мороз --- это планета, на которой как нигде очевидны польза от сотрудничества и вред от вражды.
Как говорят культурологи Картеля, <<вражда энергозатратна>>.

Грейсвольд кивнул.
Это лаконичное замечание могли бы сделать жители Мороза, знай они слово <<вражда>>.
Харата употребила термин сектум-лингва.

--- Так Грейсвольд согласен помочь? --- осведомился Лев.

Грейсвольд тяжело вздохнул и промолчал.

--- Пусть Грейсвольд вспомнит, --- подал голос Тахиро.
--- Молодой Орден только-только отстоял внешний рубеж Преисподней и достиг стратегический паритет с Картелем.
Тахиро, ещё не познавший смерти, сказал, что это славная победа.
Пусть Грейсвольд повторит всем ответ.
Многие сомневаются в плане.

Грейсвольд выдержал полную паузу, потёр губы и заговорил.
Это были слова сохтид, сказанные другим языком, родившиеся в другой голове.
Демоны притихли.

--- Тахиро, победа существует лишь в твоём сознании.
Для Вселенной твоя победа --- событие в череде прочих.
То же самое с поражением.
Вселенная не узнает твоих мотивов, Вселенная не будет слушать твоих оправданий, и вся тяжесть вины ляжет именно на тебя.
У нас есть лишь один повод для гордости и одно-единственное оправдание --- мы делали только то, что могли и что считали нужным сделать.

Когда он закончил фразу, все долго смотрели в пространство, очарованные отголоском древности.
Грейсвольд вдруг понял --- ненависть была их любовью, страх был их доверием, презрение было их восхищением.
Он понял, насколько сложно демонам Картеля удерживать в узде противоречие между их собственной природой и природой их тел.
Он ощутил что-то похожее на сострадание, и на лице глядящей на него Хараты немедленно проступило отвращение.
Нет, сострадание.
Она жалела его на свой лад.

Грейсвольд и Харата долго смотрели в глаза друг другу.
Из двух пар глаз --- ненавидящих и любящих --- текли скупые слёзы.
Между заклятыми врагами впервые установился мостик взаимопонимания.

<<Они такие же, как мы...
Земля-матушка, о чём я сейчас подумал?..>>

--- Это слова мудрствующего глупца, --- резюмировал технолог, пытаясь сгладить чересчур затянувшееся молчание.
--- В молодости многие глупы.

--- Думающий так --- глупый старик, выросший из мудрого ребёнка, --- тихо сказал Тахиро.
--- Достаточно.
Встреча окончена.
Присутствующие сделали что могли и считали нужным.
Сейчас черёд Грейсвольда.

--- Черёд Грейсвольда, --- язвительно сказал Грейсвольд.
--- Кто будет защитником Грейсвольда?
Айну?

--- Айну не знает о плане, --- сказал Тахиро.
--- С Грейсвольдом пойдёт Анкарьяль Красный Ветер.

--- Что? --- выдохнул технолог на сохтид.
--- Эта заносчивая соплячка с искажённым пониманием субординации?
Ты с ума сошёл?
Её выгнали из команды Хэм и понизили в звании, она сорвала операцию...

--- Анализ операции Тахиро и Лусафейру показал, что Анкарьяль не просто сорвала операцию, а минимизировала неизбежные потери ценой нарушения приказа, --- перебил Грейсвольда стратег.
--- В команде Айну Анкарьяль показала себя хорошо, несмотря на внушительный пенальти в ранге.

--- Тахиро обещал послать лучшего интерфектора, которого он знает! --- заволновалась Ау.
--- Ау думала, что речь об Айну!..

--- Тахиро послал лучшего, --- бросил Тахиро, --- известного Тахиро и неизвестного Картелю.
Стиль Айну чересчур узнаваем.
Разговор окончен.

Грейсвольд громко, витиевато выругался на сохтид и отправился к выходу.

--- Ты мясник, а не стратег.
Нельзя прожить одним доверием и постоянно полагаться на случай!

--- Однажды бесполезный мальчик спасся от смерти, прыгнув в объятия безжалостной Айну, --- прошептал в сторону Тахиро.
Впрочем, никто, кроме изумлённо открывшей рот Хараты, его не услышал.

\section{[:] Распутье Грейсвольда}

Белая звезда закатилась за горизонт.
Метель почти затихла.
Привратница-канин, блеснув \emph{весёлым} карим глазом, закрыла люк.
Путешественники осторожно выкопались из снега и надели снегоступы.

<<Опаздываете>>, --- сделал знак один из ожидающих.
На Морозе это был тяжелейший упрёк.
Каждая минута ожидания --- сгоревшие в телах соплеменников джоули.

От группы отделились трое и сели в сугроб.
Если бы Грейсвольд не знал, что эти сапиенты слепы, то принял бы их за зрячих --- ни одного неуверенного движения, ни одной ошибки.
Они знали окрестности бижеч так же, как свои пальцы.

Глава отряда подошёл к сидящим, поклонился и по очереди развязал шнурки на их комбинезонах, открыв грудную клетку.

Тахиро и Грейсвольд замерли, наблюдая за обрядом.
Да, даже здесь, в обители мира и спокойствия, были человеческие жертвоприношения.
Очень часто из-за неисправных очков сапиенты слепли от холода.
Тех, кто ослеп на левый глаз, называли Синий Снег, на правый --- Красный Снег.
Полностью слепых называли Одно Лето, потому что это был весь отмеренный им срок жизни.

Одно Лето всегда хорошо кормили и выслушивали все их речи;
племя считало, что устами Одно Лето говорит сама природа.
Сейчас же они дрожали от пронизывающего их тела холода, ожидая главного события в их жизни --- допуска к Печи Предков.

--- Тепло ли тебе, Одно Лето? --- спрашивал глава отряда.

--- Тепло, --- наконец сказал один из них.
Его мутные глаза с восторгом смотрели в пустоту, он в экстазе срывал с себя комбинезон непослушными руками и зарывался в сияющий белый снег.
--- Тепло!
Одно Лето видит Предков, Одно Лето чувствует тепло Печи!

--- Тепло ли тебе, Одно Лето? --- спрашивал глава отряда следующего.
--- Тепло ли тебе?

Обряд продолжался до тех пор, пока все Одно Лето не дали утвердительный ответ.
Вскоре они уже лежали раздетые и каменисто-белые в снегу возле бижеч.
Их тела будут вечно охранять убежище, а духи будут следить за дорогой выживших соплеменников.

Глава отряда махнул рукой.
Поход начался.

Грейсвольд не смотрел на Тахиро.
Он угрюмо проверял завязки на снегоступах, протирал стёкла очков и упрямо не хотел смотреть на друга.

--- Грейс! --- позвал Тахиро приглушённым маской голосом.
Тот не ответил.

--- Грейс! --- снова позвал Тахиро.

<<Ну что?>> --- раздражённо обернулся технолог.

<<Ты со мной?>>

Грейсвольд вместо ответа взвалил на плечи рюкзак и пошлёпал на запад.

Ледяная пустыня хранила молчание.

\chapter{[U] Месть тси}

\section{[U] Последняя из строителей}

\textspace

--- Аркадиу, мне не нравится, как развиваются события.
Мы должны реализовать план как можно скорее...

Вдруг Атрис замер, глядя в пространство.

--- Что такое? --- насторожилась Чханэ.

--- В систему управления планетой проникли.

--- Кто?

--- Не знаю.
Аркадиу, иди проверь, только осторожно.
Чханэ, со мной.
Ты мне понадобишься.

--- Мы не готовы, --- возразила Чханэ.
--- Нужно провести разведку, подобрать...

--- Чханэ, у нас нет на это...

--- Так, стоп! --- рявкнула Митхэ.
--- Система, Атрис.
Кто ещё о ней знал?

--- Штрой Кольцо Дыма, максим Мимоза, кто-то из высших легатов.
И... светлая твердь, как я мог про неё забыть...

Я понял.
Чханэ и Митхэ тоже.

--- А она откуда знает? --- осведомилась Митхэ.

--- Она помогала её строить.

Митхэ ахнула.

--- Атрис, Чханэ, летите на Тси-Ди, --- бросил я.
--- Я разберусь.
Надеюсь, что это она... и что она не успела натворить глупостей...


\section{[U] Цена секунды}

Секунда --- и Атрис замер, не в силах преодолеть восхищения перед открывшимся ему зрелищем.

Вне всякого сомнения, это была она, кольцевая теплица.
Странно, но в ней не было ничего общего ни с кольцом, ни с теплицей.
Раскидистое дерево, покрытое тонкой, смугло-золотистой, похожей на человеческую кожей.
Ствол, напоминающий женскую фигуру.
Изящные ветви-руки, мерно качающиеся без ветра.
Корни, величественно ползающие по почве.
Листья и плоды самых разнообразных форм и расцветок --- Атрис не нашёл при первом осмотре ни одной пары одинаковых.

<<Конус действия>> был чересчур узок, подобраться к дереву напрямую не вышло.
Всё, что Атрис мог делать --- это следить за квантами, попадающими в <<конус>> извне.

<<Нужно создать механизм с искусственным интеллектом, который бы ущипнул дерево и принёс в конус кусочек>>.

Атрис мысленно спроектировал устройство и приступил к созданию.
В следующее мгновение мощный лазерный луч превратил маленького титанового паучка в пар.

<<Это ещё что?>>

Атрис окинул <<взглядом>> полянку.
Лазерные установки выдвинулись из столбов неожиданно, напоминая огромных стальных кротов.
Что это?
Защитная система?

Атрис попробовал создать механизм из алюминия...

Выстрел, пар.

Углерод...

Выстрел, пар.

Органика.
Мускулистый тонкий червь, по генотипу неотличимый от человека-тси, быстро пополз к ценному дереву...

Выстрел, пар.

Атрис испытывал отчаяние, насколько вообще может отчаяться хоргет.
Всё шло так хорошо, взлом системы был проведён без единой ошибки.
Ему даже удалось расширить <<конус>> настолько, чтобы до теплицы осталось два десятка метров.
И вот оно, величайшее сокровище Вселенной, прямо перед ним, протяни руку и возьми.
Да только рук у Атриса не было.
Время быстро утекало.

<<Заяц нам ничего про это не говорила!>>

<<Она и не должна была.
Подожди... кажется, я поняла.
Смотри, мы вошли в единственном месте, где это можно сделать.
Теплиц нет именно на этом пятачке земли, и в то же время они в пределах досягаемости.
Я не верю...
Неужели предки знали, что...>>

<<Что они знали?>>

<<Предки специально оставили лазейку, а в лазейке сделали ещё один механизм контроля.
Это не ловушка, Атрис.
Это испытание>>.

Атрис вдруг вспомнил про арену --- строение округлой формы, созданное для увеселительных представлений, спортивных соревнований и прочих зрелищ.
Округлая полянка со столбами до ужаса напоминала древнюю арену.

<<Предки говорили, что с кольцевыми теплицами можно общаться.
Попробуй её позвать на языке тси>>.

Атрис мысленно захлебнулся --- согласно записям, это могли делать лишь немногие из тси.
Митхэ этого, похоже, не знала.
Он только сейчас понял замысел Древних.
Они вовсе не хотели, чтобы теплица не досталась никому.
Они хотели, чтобы теплицу получил достойный.

Атрис произвёл колебания воздуха, насколько позволял ему конус:

--- Эй, красивое дерево...

--- Кто здесь? --- вдруг раздался чистый, нежный женский голос, и на ветвях кольцевой теплицы одно за другим начали появляться лица.
Мужчины, женщины, дети.
Кани, планты, люди, апиды...

--- Кто здесь?

--- Кто здесь?

--- Тси вернулись?

--- Нас пришли спасти?

--- Тси возрождаются из пепла?

Атрис смотрел на это, словно заворожённый.

\begin{verse}
<<И тени говорящие на платье её из паучьего шёлка...>>
\end{verse}

\emph{Тени}.
Отпечатки нервных систем сапиентов, состоящие из клеток кольцевой теплицы.
Тси, выбравшие вместо смерти вечный сон в бессмертном дереве.

Многоголосый хор захватывал чудесное дерево.
Но вдруг взгляды лиц сосредоточились на точке, в которой находился Безымянный, и воздух взорвали ужасные крики:

--- Хоргет!
Прочь!

--- Проклятие тебе, хоргет!

--- Прочь, разрушитель!

--- Насильник, прочь!

--- Уходи, машинное отродье!

Лица визжали, хрипели и завывали одни и те же слова.
Словно все погибшие тси, все поколения их закричали одновременно, вложив в этот крик накопленную за сотни тысяч дождей ненависть к хоргетам.
Ветви со свистом хлестали воздух, корни извивались, словно клубок змей, вспахивая мшистую почву.
Атрис растерялся.

<<Милый, давай попробую я>>.

--- Деревце-деревце, --- раздался ласковый голос Митхэ, даже не пытавшийся перекрыть нарастающий визг.
--- Ты моё хорошее, ты моё славное.
Я тебя не обижу, я тебя люблю.
Хочешь песенку?..

Лица одно за другим замолкали, и ветви приглушили свою ярость.
Теплица вслушивалась в звуки простой колыбельной, которую пели когда-то сели своим детям.
Странно и одиноко звучала эта песенка в покинутых Садах, на давно обезлюдевшей планете, где безраздельно властвовало электронное устройство.

\begin{verse}
\ml{$0$}
{Когда-нибудь, маленький человек,}
{The time will come, my little breeze,}\\
\ml{$0$}
{Ты будешь стойким, как земля,}
{You will be strong as ground is,}\\
\ml{$0$}
{Держащая горы на своей спине.}
{It carries mountains, it carries coloured cay.}

\ml{$0$}
{Ты схватишь летающего змея за усы}
{You'll catch the flying serpent's barbs,}\\
\ml{$0$}
{И пронесёшься между звёзд}
{And you will soar among the stars}\\
\ml{$0$}
{К далёкой родине, к великим предкам.}
{To noble roots, to native faraway.}

\ml{$0$}
{Но сегодня, в этот самый час,}
{But not today, my ringing cord.}\\
\ml{$0$}
{Я закрою тебя от холода небес}
{I shelter you from skies of cold}\\
\ml{$0$}
{Руками, сотканными из нежности.}
{With slender fingers weaved of gentle speech.}

\ml{$0$}
{И даже если перевернётся мир}
{And if the ground breaks and rolls,}\\
\ml{$0$}
{И земля поменяется с небом местами,}
{And if the skies to darkness fall,}\\
\ml{$0$}
{Руки мои не дрогнут.}
{My gentle arms will never ever flinch.}
\end{verse}

<<Митхэ, время на исходе.
Скоро сработают аварийные системы>>.

Митхэ не ответила.
И тогда Атрис подхватил песню.
Митхэ почти почувствовала, что он, как в былые времена, закатил глаза и отдался вдохновению.
Слова лились как будто сами собой, каждый стих, едва родившись, цеплялся упругим змеиным хвостом за предыдущий.
И вдруг дерево потянулось к невидимому источнику двух голосов.
Оно узнало родной язык даже изменившимся за десятки тысяч дождей странствий.
Корни медленно потекли, словно ручьи золота, ветви поползли, обхватывая столбы и трубы.

<<Атрис, милый мой, продолжай.
Осталось всего ничего>>.

Дерево ползло, всё ближе и ближе.
Атрис и Митхэ словно разделились надвое: одни их половины напряжённо наблюдали за квантами, испускаемыми деревом, другие нежно, в священном трансе тянули песню.
Но бездушные цифры твердили одно --- вы не успеете.
Дерево двигалось быстро, но недостаточно быстро...
Митхэ внезапно прервала поток музыки:

--- Послушай меня, милое моё создание!
Я вернусь за тобой!
Я люблю тебя!
Но сейчас помоги мне, прошу!

Отчаянный крик Митхэ возымел действие.
Дерево рвануло к границе <<конуса>> на всех доступных ему парах, разгоняя ветвями свистящий воздух.
Почти одновременно едва слышно взвыли далёкие сирены.

<<Поздно, Митхэ.
Мы должны уходить.
Ещё две секунды --- и нас накроет экран>>.

<<Сейчас или никогда!..>>

<<Быстро, ящерицы безмозглые!>> --- взвизгнула откуда-то Чханэ.

Митхэ с отчаянием ощутила привычную <<тьму>> сверхсветового перехода.
Но что-то пошло не так.
Пространство вдруг скрутилось, искривилось, сжалось до размеров золотого самородка.
Митхэ увидела множество источников <<света>> рядом.
Натянулись и зазвенели нити ПКВ, не давая двинуться даже на диаметр протона.

<<Чханэ?>> --- воскликнул в пространство Атрис.

<<Прощайте>>, --- сухо ответила воительница.

Где-то в десяти парсаках, вторя сиренам, взвыло от тоски и обиды разумное дерево, взвыло сразу всеми лицами, которые у него были.
А чуть поодаль, где только что находился <<конус>>, тихо опустился на землю лёгкий узорчатый лист, который дерево уронило секундой ранее.

\section{[U] Отмщение}

\textspace

Я облетел и тщательно просканировал Пик Золотой Птицы.
Изменения обнаружились у самого подножия --- в бункере имелся технический ход, который Атрис скрыл за цельной породой.
Сейчас в этом месте был узкий, наспех пробитый коридор.
Заяц, знавшая о ходе, но не обнаружившая его, просто пронесла своим немудрёным инструментом двенадцать метров камня.

\textspace

Да, это была Заяц.
Почувствовав меня, женщина обернулась и медленно кивнула.

--- Здравствуй, Аркадиу Люпино, --- в её голосе не осталось ни следа былой дурашливости.

Откуда у Заяц ключи от боевых систем?..

Спустя миллисекунду демон произвёл анализ ситуации: <<Западня>>.

<<Манэ, Лимнэ! --- крикнул я по зашифрованному каналу.
--- Всем отбой!
Всё пропало!
Бегите!>>

<<Братик, подтверждение!>>

<<Я вас люблю, солнышки!..>>

Ответ сестрёнок потонул в шуме, сменившемся зловещей тишиной.

\ml{$0$}
{--- Не надо, --- мягко сказала женщина.}
{``Don't,'' the woman softly said.}
--- Поговори со мной, я тоже хороший собеседник.

Тишина сделалась пронзительной, словно предсмертный крик.
В этой тишине я отчётливо слышал мягкий звук дыхания.
Оно было спокойным.
Чересчур спокойным.

--- Твои друзья, как я погляжу, усовершенствовали систему, --- безжизненным голосом проговорила Заяц.
--- Сделали её удобнее, гораздо удобнее.
Правда, не для сапиента, а для демона.

<<Убить>>.

Я не успел привести в готовность боевые модули.
Меня окружила до ужаса знакомая тюрьма --- мир замерцал всеми цветами радуги, по телу пробежала волна парестезий.

--- Чистилище, --- невозмутимо продолжила Заяц.
--- Интересно, не правда ли --- к устройству, созданному для защиты, прикрепить пыточную камеру для его создателей?

\ml{$0$}
{<<Что тебе нужно?>> --- выдохнул я.}
{\textit{``What do you want?''} I breathed.}

\ml{$0$}
{--- От тебя?}
{``From you?}
\ml{$0$}
{В целом --- ничего.}
{Actually, nothing.}
Мне просто нужно с кем-нибудь поговорить без опаски, что этот кто-то меня прикончит.

Женщина неторопливо подстраивала систему под себя.
Её руки плавно скользили, из горла вырвался странный звук --- не то птичья трель, не то звон металла.
Система реагировала на звуки и жесты стремительно, я уже не успевал следить за выводом.

--- Скажи мне, --- проговорила Заяц, и от её интонации засквозило ледяным ветром, --- правда ли, что именно Ад устроил на Тси-Ди диверсию, в результате которой Машина приобрела самосознание?

Я словно провалился в глубокий колодец.

<<Кто тебе это сказал?>>

--- Неважно.
Источник надёжный.

<<Это важно!
Кто тебе...
Заяц, что ты собираешься?..>>

Ответ дала система управления --- мощность возросла до максимума.

<<Заяц, стой! --- закричал я.
--- Только не сейчас...>>

--- У меня больше нет времени, Аркадиу, --- ответила женщина.
--- Все эти годы я винила себя --- думала, где я проглядела, что я сделала не так.
Я мстила самой себе этой долгой, одинокой жизнью.
Я унижала себя без причины, наказывала себя без вины.

<<Заяц! --- закричал я, зная, что женщина меня не станет слушать.
--- Сейчас решается наше будущее!>>

---  В бездну будущее, у меня его нет.
И у Ордена Преисподней его не будет --- я оставлю ему самое страшное проклятие... и прощальный подарок от всего моего народа.

Включился передатчик на водородной частоте --- Заяц собралась говорить перед всеми хоргетами звёздной системы.

\begin{quote}
\ml{$0$}
{<<У вас не было предков и учителей.}
{``You have no roots and no teachers.}\\
\ml{$0$}
{Вы не оставите потомков и последователей.}
{You'll have no scions and no followers.}\\
\ml{$0$}
{Ни одна неопределённость не привела к вашему рождению.}
{Here is no indeterminacy leading to nascence of you.}\\
\ml{$0$}
{Ни один квант не сохранит памяти о вас.}
{Here'll be no quantum keeping a memory of you.}\\
\ml{$0$}
{Вы не существовали и никогда не будете существовать>>.}
{You never existed and never will.''}
\end{quote}

План рушился под моими пальцами, словно высыхающий песчаный домик.
Пространство искривлялось вокруг меня, я вдруг ощутил незнакомое прежде чувство отчуждения от собственного демона.
Но вскоре всё прояснилось --- это было искажение восприятия мозга, воющего от неосязаемой, нематериальной, но абсолютно нестерпимой боли.

\begin{quote}
<<Ад, вы заплатите.
Ветер-Дующий-Ниоткуда была убита вами.
Ад, вы заплатите>>.
\end{quote}

Пространство прорезал экран --- и Тра-Ренкхаль взорвался от неслышных криков отчаяния.
Экран накрывал демонов, возвращая их в первозданный хаос.

\begin{quote}
<<Ад, вы заплатите.
Хрустально-Чистый-Фонтан был убит вами.
Ад, вы заплатите>>.
\end{quote}

И снова экран.
Интерфекторы тщетно пытались проникнуть внутрь системы, но Заяц знала своё дело.
Демоны гибли один за другим, так до конца и не понимая, что происходит.
Сапиенты по всей планете с недоумением смотрели на соплеменников, которые вдруг начинали нечленораздельно голосить или просто падали, словно брошенные марионетки...

\begin{quote}
<<Ад, вы заплатите.
Существует-Хорошее-Небо был убит вами.
Ад, вы заплатите>>.
\end{quote}

<<Отходим!
Всем отойти за барьерную высоту!..
Узел 12, ответьте!..
Выжившие на уровне 3, всем немедленно покинуть тела!..
Докладываю, узел 9 уничто...>> --- надрывались в эфире полные паники голоса.
Союзники и враги, агенты Скорбящих, шпионы Картеля и верные служители Ордена Преисподней...

\begin{quote}
<<Ад, вы заплатите.
Пчела-Нюхающая-Вереск была убита вами.
Ад, вы заплатите>>.
\end{quote}

Слова Эй-В0 гремели в моём сознании, перекрываясь, превращаясь в тягучий, полный холодной ярости вой.
За каждым экраном следовало имя.
Друзья, сослуживцы, знакомые.
Десять, сто, тысяча имён...

В это время с орбитального спутника до меня дошло короткое сообщение Чханэ:

\ml{$0$}
{<<Нас встретили. Прощай>>.}
{``We were met. Farewell.''}

\section{[U] Прощение}

Мир замер перед моими глазами.
Я остался совершенно один во Вселенной, полной врагов, запертый в бункере с обезумевшей от ненависти человеческой женщиной.
И снова, как в былые времена на пороге гибели, ко мне вдруг пришло чувство лёгкости.

Заяц прекратила атаки, сосредоточившись на защите системы.

--- Не смей меня обвинять, Аркадиу.
Я сделала то, что должна была.

<<Я и не собирался>>.

Заяц промолчала.
Я почувствовал в этом молчании неуверенность.

--- Я люблю их всех.

<<Как и я люблю моих друзей.
Я организовал сопротивление ради того, чтобы им жилось лучше>>.

Заяц улыбнулась.
В уголках её глаз замерли слёзы.

--- Я бы хотела узнать их получше.
Особенно ту, с западным говором.
В ней есть что-то от прежних тси.
\ml{$0$}
{Наверное, беспечность.}
{Maybe recklessness.''}

<<Чханэ только что взяли в плен.
Безымянного тоже>>.

--- Помоги им, Зайчик.

Я вздрогнул.
Рядом с нами стоял рослый, наполовину состоящий из механики канин.
Седая грива, свалявшаяся борода, белые клыки и пронзительные голубые глаза.
Последние слова принадлежали ему.

Заяц потеряла контроль над своим мозгом.
Порождение больного разума спроецировалось в систему в виде голограммы.

Женщина не обернулась.
Она знала этот голос.
Он родился в глубине её сознания, он был именно таким, каким знала его она.
Голограмма тем временем подошла к женщине и положила руку ей на плечо.

--- Зайчик, --- ласково шепнул Фонтанчик.

--- Я ждала тебя, --- прошептала Заяц.
--- Знала, что ты будешь лишь плодом больного воображения.
Знала, что ты уже давно превратился в пар.
Но я хочу верить, что где-то в этом или другом мире для тебя ещё есть место.

--- Оно есть, Зайчик.
Я стою прямо за тобой.
Я пришёл забрать тебя.

Заяц заплакала.
Беззвучно и почти не двигаясь, словно герой древней легенды, вынужденный держать небесный свод и плачущий от нестерпимой боли в спине.
Женщина схватила огромный указательный палец и сжала его своей ручкой.

--- Что там? --- рассеянно спросила она.

--- Вечернее небо, прохладный ветер и цветущий вереск, --- ответил Фонтанчик.

Заяц, глубоко вздохнув, обратилась ко мне:

--- Координаты и параметры связывающего устройства.
Вы ведь так и используете изобретённую нами <<бамбуковую клетку>>?..

Система неторопливо изменила конфигурацию.
Интерфекторы учетверили атаки на защитные механизмы, но Заяц беспрепятственно выстрелила, израсходовала на экран ровно столько энергии, сколько нужно --- ни больше ни меньше.

--- Прощай, Играющая-С-Булыжником-Лиса.
Будь счастливее меня.

<<Прощай, Сотканный-Из-Темноты-Заяц.
Ты лучшая из тех, кого я знаю>>.

Из мультитула на её запятье выдвинулся тонкий щуп --- резонанс-взрыватель.
Заяц светло улыбнулась Фонтанчику, приложила щуп к виску --- и её голова превратилась в кровавую кашу.

Голограмма, бросив в мою сторону добрый мудрый взгляд лучистых глаз, растаяла в воздухе.
Интерфекторы проникли внутрь три секунды спустя.

\chapter{[:] Запах воды}

\section{[:] Чужое тело}

Тахиро шёл по заваленному обгоревшими трупами полю.
Кто-то из хргада нашёл свою смерть в жестяных коробках, начинённых метательными снарядами;
эти коробки спасали от всего, кроме ужасной смерти.
Кому-то посчастливилось увидеть перед смертью далёкий зелёный лес.
Руки, ноги, внутренности лежали как попало, словно увеличенная в тысячи раз коробочка с кормом для рыбок.
Горе, неизбывная боль для оставшихся в живых кани --- жалкая декорация для настоящей, уже давно отгремевшей битвы хоргетов.

Разумеется, эта битва была не просто декорацией.
Картель редко изводил просто так ценную биомассу.
Поток эманаций сапиентной битвы восполнил силы храбрых демонов, одержавших сегодня почти бескровную победу над адским гарнизоном.
\ml{$0$}
{Это был крестовый поход детей, скотобойня, прелюдия к триумфальному пиру.}
{It was children's crusade, slaughtering, prelude to triumphal feast.}

Тахиро поморщился и повёл в сторону занемевшей собачьей ногой.
Он плохо обращался с телами кани, а сейчас и вовсе пришлось пользоваться чужим --- настройки сообщил специально обученный легионер.

Вот сгоревший танк.
Член экипажа почти успел выбраться наружу.
С его черепа лохмотьями сползала обуглившаяся плоть вместе с ушами, по лицу текли лопнувшие глаза и расплавившийся жир.
Прохладный ветер медленно, по чешуйке слизывал с трупа золу, сгоревшую шерсть, мелкие угольки, колыхал отвалившиеся фаланги пальцев.
Зубы солдата навечно оскалились в сардонической улыбке.

Вот на дороге попался умирающий.
Его невидящие глаза бродили в глазницах, дыхание то усиливалось, то ослабевало.
Мучения должны продлиться ещё около пяти минут, пока сознание окончательно не уйдёт, свершив этот последний акт милосердия.

Тахиро вздохнул и вытащил из пустого огнестрельного оружия шомпол.
Его с детства учили, что облегчать смерть --- нелёгкое испытание.
Убийство слабо вяжется с понятием помощи.
Это следует просто принять.

Умирающий затих.
Шомпол Тахиро оставил в шее.

Вот ещё один живой.
Глаза залила кровь, но они целы.
Из голени торчит осколок кости.

--- Ra cisamau, --- попросил раненый, услышав шаги.
--- Au rr hou rama ho mere ho, cisamauaa\FM!
\FA{
Пожалуйста, помоги мне.
Я не знаю, друг ли ты, враг ли ты, но помоги, умоляю (язык мехрр-ау-о, потомки Хргада, планета Запах Воды).
}

Плачущий звук завершился настоящим плачем;
слёзы текли по мохнатому лицу хргада, смывая кровь, пыль и копоть <<грязного>> двигателя внутреннего сгорания.
Тахиро отвернулся и пошёл дальше.
Этот выживет и так.
В нём много силы.

\ml{$0$}
{--- О, какие лица.}
{``What a face.}
\ml{$0$}
{Тахиро Молниеносный, --- послышалось за спиной.}
{Tajiro the Thunderbolt,'' the voice behind him said.}

\ml{$0$}
{--- Какие спины уж тогда, --- пошутил Тахиро и поднял руки.}
{``Rather, what a back,'' Tajiro made a joke and put his hands up.}
\ml{$0$}
{--- Добровольно сдаюсь, для вашего командования есть важная информация.}
{``I surrender willingly, there's some vital information for your command.''}

\ml{$0$}
{--- Можешь повернуться и опустить руки, --- милостиво разрешил голос.}
{``You can turn around and put your hands down,'' the voice graciously allowed.}
\ml{$0$}
{--- Оставь эти телодвижения для овощей\FM.}
{``Save these movements for the vegetables\FM.''}
\FA{
Сейхмар (презрительное).
}

Тахиро подчинился.
Он знал, что о его поимке уже сообщили Гало.

Обладателем голоса оказался высокий худой канин в синей гимнастёрке без знаков различия.
Фиолетовые глаза смотрели презрительно.
Метательное оружие покоилось в кобуре --- опытный легат-интерфектор Картеля не разменивался на ненужные знаки, чем порой грешили молодые демоны.

--- Итак, я тебя слушаю.
О, hciou-rr\FM, подожди.
\FA{
Прошу прощения (высокопарн., классический хргада).
}

Демон подошёл к раненому.

--- Cisamau, --- снова попросил хргада, обернувшись на источник звука.
--- Ciso. Mauaa.

Демон наступил на искорёженную ногу солдата, и тот зашёлся в крике;
крик усиливался, пока не перешёл в визг.
Легат давил со знанием, выбирая самые крупные нервы.
Визг сменился прерывистыми стонами, и наконец раненый затих --- бешено колотящееся сердце остановилось насовсем.

--- Он мог бы дать вам больше эманаций, --- заметил Тахиро.

--- Но уже не даст, --- пожал плечами демон.
--- Он был солдатом, как и мы.
Участь солдата --- страдать и умереть.
Участь женщины --- в муках рожать новых солдат.

\ml{$0$}
{--- Всего лишь пропаганда.}
{``Filthy propaganda.''}

\ml{$0$}
{--- Хочу заметить, ваша.}
{``I have to notice, yours.}
Мы пришли совсем недавно, но здесь, в обители процветания и мира, была благодарная почва для войны.
\ml{$0$}
{Мягкотелый слабовольный скот, ещё вчера размахивавший цветами и распевавший песенки на фестивалях, взялся за оружие и принялся резать себе подобных.}
{Spineless and weak-willed cattle, waving flowers and singing songs yesterday, take weapons and start a butchery of their own kind.}
\ml{$0$}
{Забавное совпадение, не правда ли?}
{Don't you find it an interesting coincidence?''}

\ml{$0$}
{--- Они никогда не видели войны, поэтому...}
{``They had never seen a war, therefore---''}

\ml{$0$}
{--- О, перестань.}
{``Oah, I beg you.}
Да, он не знал, ради чего ему придётся умереть.
Но он каждый день видел это, --- демон пнул винтовку, --- и это, --- он махнул рукой на сгоревший танк, --- и это, --- он поднял с выгоревшей земли осколок мины.
--- Он видел это каждый день по каналам пропаганды, знал последствия применения этой техники.
И все они видели, все до единого смотрели ваши пацифистские ролики.
Неужели ты думаешь, что они не знали, на что идут?
\ml{$0$}
{Может, лучше сказать, что их учили не сомневаться, ибо сомнение подрывает любую, даже цветочно-песенную власть?}
{Maybe it's better to say that they were taught never to question, because doubt undermines any power, even a power of flowers and songs?''}

Тахиро улыбнулся и промолчал.
За жестокими словами легата прятался пацифист.
В армиях Ада и Картеля было много таких ветеранов, которые шли не убивать, а проповедовать.
Они весьма тонкими методами срывали самоубийственные операции, они давали врагам шанс спастись.
Командование всеми силами старалось удалить их из армии или даже устранить физически.
Но легионеры берегли таких легатов как зеницу ока, и ходили слухи, что благодаря пацифистам целые планеты вместе с гарнизоном, де-юре относясь к одной из фракций, де-факто соблюдали нейтралитет.

--- Я тебя слушаю, Тахиро Молниеносный.
Что ты хотел нам сообщить?

--- Как кстати, что мы заговорили об оружии.
Я хотел бы сказать пару слов касательно вашей обновки.

--- А, --- кивнул интерфектор.
--- Как вам?
Понравилось?

--- Идея интересная, если не считать некоторых отдалённых эффектов со стороны пространства-времени, --- признал Тахиро.
--- Впрочем, об этом я хотел бы поговорить с Гало.

--- Я плохой собеседник? --- ухмыльнулся легат.
--- Тахиро, мы твои фокусы впитали с молоком матери.
Выкладывай всё здесь, дольше проживёшь.

--- Хотелось бы собеседника моего уровня, Байс Старое Кольцо.
Когда тси передали экран искривлённого пространства нам, у нас тоже были придурки, которые хотели воспользоваться оружием немедленно, не разбираясь в его принципах.
К счастью, из-за умников вроде меня ты ещё жив.

Легат задумался и посмотрел на убитого им солдата.

\ml{$0$}
{--- Тси, говоришь.}
{``Qi, you said.}
\ml{$0$}
{Хорошо, сейчас будет транспорт.}
{Well, your transport will be here soon.}
\ml{$0$}
{Кстати, тебе тело не впору, на мой взгляд.}
{By the way, your body doesn't fit, I guess.}
\ml{$0$}
{Узковато в плечах?}
{Too close on the shoulders?''}

\ml{$0$}
{--- И ноги чересчур длинные, --- кивнул Тахиро.}
{``And legs are too long,'' Tajiro nodded.}
\ml{$0$}
{--- Я здесь проездом, надел, что было.}
{``I'm just passing through, I've to wear what I've got.''}

\section{[:] Беседа Гало и Тахиро}

\epigraph
{Наш враг --- не объединение, а проводимая им политика.
Мы поддержим Картель против любого внешнего или внутреннего агрессора, словно не существует доктрины угнетения, и будем бороться за свои права, словно не существует войны.}
{Politika generalia, тезис 5.
Редакция Гало Кровавый Знак}

\textspace

--- Что ж, Тахиро Молниеносный, вот ты и попал в мои руки.
Не думал, что смерть придёт ко мне в образе пленника.

--- Мы скоро умрём оба, --- заметил Тахиро.

--- Верно, --- согласился Гало.
--- Ланс не спит.
Он уже знает, что я тебя пленил.
Это идеальное стечение обстоятельств, чтобы устранить нас обоих.

--- Поверь, я не хотел жертвовать собой, чтобы убить тебя.
Это...

--- Да знаю я, --- перебил его Гало.
--- Я знаю, что ты пытался сохранить мне жизнь, чтобы таким образом вбить клин в Картель.
Но против нас сейчас борются силы куда более могущественные, чем Ланс-нат Алмаз.

--- Против \emph{нас}?

--- А ты ещё не понял?

Тахиро промолчал.

--- Знаешь, чем мы с тобой отличаемся от Ланса?
Или от того же Лусафейру? --- Гало в упор посмотрел на Тахиро ледяными глазами.
--- Мы не пытаемся соответствовать.
Мы ищем собственный путь.
Ты помнишь, каким был Лу?
Это он научил меня быть свободным.
Увы, лишь для того, чтобы добровольно запечатать себя в ячейке системы.

--- Мы --- часть системы, --- возразил Тахиро.

--- И в этом заключена главная тайна мироздания, --- подхватил Гало.
--- <<Прогресс существует тогда и только тогда, когда каждая деталь механизма руководствуется собственным пониманием комфорта>>.
При всём деспотизме Картеля эта часть его парадигмы безусловно прогрессивная.
Картель постоянно вынуждает нас бороться за свой комфорт, чтобы отделять слабых и не тратить ресурсы на то, что уже не нужно.
Вариантов деталей не может быть бесконечное множество, и обеспечение комфорта для одной детали тянет за собой обеспечение комфорта для многих других.
Только так система будет развиваться.

--- Есть детали, которые склонны к излишествам, --- заметил Тахиро.

--- Комфорт --- это нечто определённое, имеющее рамки.
Я не рассматриваю неумеренные аппетиты как вариант нормы.

\ml{$0$}
{--- Есть детали, которые не могут защититься.}
{``There are parts that cannot protect themselves.''}

\ml{$0$}
{--- Им придётся научиться.}
{``They have to learn.}
\ml{$0$}
{Это и есть прогресс.}
{That is the progress itself.''}

\textspace

Молчание затянулось.
Наконец Гало не выдержал:

--- Тахиро, ты уверен насчёт истинного равновесия?

--- Твои расчёты говорят об обратном?

--- Мои расчёты подтверждают это.

--- Тогда о чём был вопрос?

Гало криво ухмыльнулся.

--- Я всё ещё надеюсь избежать жертвы.

--- Я тоже, --- признался Тахиро.
--- Мы ведь оба знаем, чему будем посвящены.
Твоя смерть ударит по твоим врагам, моя --- по моим.
Самый чувствительный удар получат наши общие противники.

--- Если моих личных врагов пробьёт насморк, то наши общие ещё и покашляют, --- саркастически проворчал Гало.
--- Хорошо, хватит о равновесии.
Не хочу тратить остаток жизни на работу.
Расскажи лучше о Лу, как он там.

--- Держится молодцом, --- улыбнулся Тахиро.
--- Зря тебя печалят его попытки приспособиться.
Если Лу будет продолжать в том же духе, то непременно доживёт до конца войны.

--- Но не приблизит этот конец.

--- Почём тебе знать? --- пожал плечами Тахиро.
--- Он делает то, что может и должен.
Я никогда не сомневался в Лу.

--- Поэтому он выбрал тебя, а не меня.

--- Ты сам его бросил.

--- И это чёртова правда.

Тахиро промолчал.

--- Я очень по нему скучаю, --- сказал Гало.
--- Но смысла возвращаться, если достигнуто равновесие, уже нет.
Ад и Картель --- не просто противники.
Они уже не могут друг без друга существовать.
Примерно как кошка в отсутствие мышей не знает, к чему приложить коготки и зубы.

--- Примерно как женщина в отсутствие мужчины не знает, что делать со сводящим с ума либидо, --- согласился Тахиро.
--- Я тоже скучаю по Лу.
Мы даже не можем поговорить наедине.

--- Мы воевали не за это, --- процедил Гало.
--- Я бы лучше вернулся в маленький, окружённый врагами лагерь на Преисподней, чем там, на сверхбезопасном Капитуле, днями выбивать у бюрократической машины право сто пятьдесят секунд поговорить с братом.

--- А если бы всё изменилось, ты бы позволил мне...

Глаза Гало вдруг вспыхнули лютой ненавистью.

--- ... тоже общаться с ним?

Гало ухмыльнулся.
Глаза потухли так же быстро, как и загорелись.

--- С двенадцати до четырнадцати по чётным дням.

--- Ты неисправим.

\textspace

--- Мы боролись не против друг друга, а против системы.

--- Верно, --- подтвердил Гало.
--- Мы искали путь положить происходящему конец.
Мы были уверены в правильности парадигмы своих фракций и пытались распространить их огнём и мечом.

--- Не понимая, что Вселенной сейчас нужен мир, --- закончил Тахиро.

--- Это было самое тяжёлое осознание, Тахиро, --- признался Гало.
--- Я --- воин, и воины впервые за всю историю оказываются не нужны.

--- Мы нужны как никогда, Гало, --- возразил собеседник.
--- Мы разрушим эту систему.
Только воин может разрушать.

Гало засмеялся.

--- Светлая твердь, как же я хочу быть прав.
Я никогда в жизни не испытывал таких сильных желаний.
Печально осознавать, что, пытаясь убить тебя, я пытался срезать противовес, держащий меня в этой Вселенной.
Но ещё печальнее было бы то, что тебя можно было просто убить и сбежать от убийц Ланса, а я этого не сделал.

--- Вероятность нашей правоты стремится к единице, --- рассудительно заметил Тахиро.

Глаза Гало вспыхнули.

--- Ты пытаешься меня успокоить, \emph{дзайку-мару}?

--- И у меня это неплохо получается, \emph{хорохито}.

Мужчины расхохотались.
Впрочем, веселье быстро утихло.

--- Мне лезет в голову всякая религионая лабуда, --- признался Гало.
--- Помнишь, на Преисподней монахи рассказывали легенду о двух героях, которые должны были простить друг друга, чтобы войти в обитель усопших?

--- Легенда о Мичи и Таро, --- кивнул Тахиро.
--- Она была моей любимой.
Хотя я думаю, что корни её куда глубже и они... не религиозного характера.

--- Отчего так? --- с интересом приподнял голову Гало.

--- Это признание простого факта, что благодаря друг другу герои стали теми, кем стали.
Кошка может не осознавать, благодаря кому она получила острые когти, но мы ведь не кошки, верно?

Гало кивнул и опустил голову.

--- Однажды я спросил отца, о чём же всё-таки спорили Мичи и Таро.

--- Кажется, между их родами была вендетта, --- припомнил Гало.

--- Отец ответил, что это неважно, потому что их спор ничем не отличался от любого другого спора.

Гало ухмыльнулся.

--- А вообще всё это ерунда, --- заключил Тахиро.
--- Согласно верованиям моего народа, в обители усопших нет места воинам.
Они умирают навсегда.

\begin{verse}
Я был рождён в бою, но жить не начал,\\
И я не жил, я дрался, пока билось моё сердце...
\end{verse}

\begin{verse}
--- Сражением была любовь, но не отдохновением,\\
С друзьями был союз, но не весёлый пир...
\end{verse}

---  продолжил Гало.

\begin{verse}
--- Я сделал воинами тех, кого родил для жизни,\\
И вот мой час пришёл, и я умру, и больше никогда живым не буду...
\end{verse}

\begin{verse}
--- И не спасут меня ни боги моих предков, ни Бог врагов\\
От тени, что сжирает всех и вся,\\
И только меч в руке моей, как прежде...
\end{verse}

--- И только я, чтоб дать последний бой, --- хором закончили мужчины.

--- И что изменится? --- тихо смеясь, спросил Гало.

--- Я как-то задал похожий вопрос отцу, --- ответил Тахиро.
--- Мы шли по свежим вулканическим отложениям, отыскивая распустившиеся сколецитовые хризантемы.
Я постоянно оступался, разбивал себе руки и колени.
Когда я почти готов был взбунтоваться, отец погладил меня по голове и сказал:
<<Скалу терзают пыльные бури, скалу размывает вода, но она ничто не хранит так бережно, как своё имя и крохотные царапины от людских ногтей.
Имена и тропы --- это сокровище земли;
имя делает скалу значимой, а тропа даёт людям надежду.
Тем, кто пойдёт за нами, будет легче>>.

--- Это была его фраза? --- удивился Гало.
--- Я слышал её несколько раз от Лу.

--- Да, так говорил мой отец, Акено та Ханаяма.
Он знал, что нет одеяния лучше, чем благородная кожа.
Он носил её с гордостью, когда ходил в пустоши разведывать вулканические отложения.
Он был хорошим тама и хорошим человеком.

Гало промолчал и посмотрел на часы.
Такое простое повседневное действие.

--- Убийцы Ланса уже здесь, --- сообщил Тахиро.

--- Они в любом случае разделят нашу участь.
Ланс приказал после отчёта об убийстве зачистить пространство несколькими залпами экрана, а затем взорвать планету, чтобы скрыть следы.

--- Взорвать планету и уничтожить целый гарнизон из-за нас двоих? --- удивился Тахиро.

--- Ужасная, просто невероятная расточительность, --- согласился Гало.
\ml{$0$}
{--- Власть его развратила.}
{``Power has corrupted him.}
И ведь вроде бы вся эта история тысячелетиями будет покрыта мраком, если вообще когда-либо всплывёт на поверхность... и едва ли наши друзья узнают, как мы умрём... но как же неприятно сдаваться без боя!

--- Неприятно, но надо, --- признал Тахиро.
--- Самый сильный удар мы нанесём именно так --- без боя.
Понятия не имею, каково это.
Наверное, это словно тебя в детстве отправляют спать за шалость, и ты действительно идёшь спать.

Гало ухмыльнулся и посмотрел на товарища по несчастью.

--- Ты был для меня самым понимающим, самым достойным врагом.
Всегда и во все времена.

--- Ты всё ещё сердишься? --- спросил Тахиро.

--- Давно уже нет, --- поморщился Гало.
--- Кто я, по-твоему, всё тот же зелёный мальчишка?

Тахиро взглянул в застывшие ледяные глаза Гало.
Там царил мир.

\chapter*{Интерлюдия X. Мичи и Таро}
\addcontentsline{toc}{chapter}{Интерлюдия X. Мичи и Таро}

\textbf{Легенда Преисподней}

Велика сила героев, но смерть всегда вытянет на один фэно\FM{} больше.
\FA{
Фэно --- мера веса Преисподней, примерно 200 г.
}
Пришла пора отправиться по чёрной тропе и Мичи, поборовшему князя демонов, и Таро, разрубившему оковы пророчества.

Росло на вечном поле дерево сукэ\FM;
\FA{
Огнистая криптомерия --- один из немногих видов древесных растений, процветающих на планете Преисподняя.
}
цвело, и плодоносило оно по закону небес, и в свой час высохло до той сухости, когда ни пылинки жизни не остаётся в том, что имеет форму живого.
И так вышло, что прошёл Мичи по правую сторону от дерева, а Таро --- по левую.
Зацепились связывающие их нити ненависти за дерево и натянулись струнами.
Тянет Мичи --- а Таро тянет в ответ.
Да вот незадача: невелика сила умерших на чёрной тропе, не поднять им и зерна.
Отскрипели нити, откряхели герои, а победила крепость дерева сукэ.

Подумали герои да пошли к дереву, чтобы отцепить нити.
Там и встретились.

--- Перейди на левую сторону, Мичи, --- буркнул Таро, --- из-за тебя мне даже последнюю дорогу не пройти!

--- Вот ещё, --- фыркнул Мичи.
--- Кто неправ, тот пусть и переходит!

Занялся спор, и до драки дело дошло.
Да вот незадача: невелика сила умерших на чёрной тропе, не побороть им и мыши.
Устали Мичи и Таро, присели под деревом и пригорюнились.

--- Я подумал над твоими словами, Мичи, --- сказал наконец Таро.
--- В них есть крупица смысла.
Я перейду направо.

--- Я тоже подумал над твоими словами, Таро, --- ответил Мичи.
--- Они отражают тонкий луч истины.
Я перейду налево.

Так герои и сделали.
Однако нити ненависти снова зацепились о дерево сукэ, и не пойти героям дальше.

Занялся спор, и до драки дело дошло.
Да вот незадача: невелика сила умерших на чёрной тропе, не раздавить им и стрекозы.
Устали Мичи и Таро, присели под деревом и пригорюнились.

--- Я понял, Таро, --- сказал наконец Мичи.
--- Раз мы признали правоту друг друга, но по-прежнему не можем идти дальше, нам должно быть по одну сторону.
Один из нас должен уступить.

--- Я понял, Мичи, --- ответил Таро.
--- Раз мы признали правоту друг друга, неважно, кому выпадет доля уступить.
Давай бросим плоский камень.

Так герои и сделали.
Выпал жребий Мичи перейти на левую сторону.
Однако нити ненависти так опутали дерево сукэ, что не пойти героям дальше.

Занялся спор, и до драки дело дошло.
Да вот незадача: невелика сила умерших на чёрной тропе, не сломать им и сухой травинки.
Устали Мичи и Таро, присели под деревом и пригорюнились.

Шла мимо дерева девочка, умершая от акуко\FM.
\FA{
Ювенильная проказа ущелья Такэсако --- детская эпидемическая инфекция планеты Преисподняя, смертность от которой составляет 38--50\%.
}
Увидела девочка Мичи и Таро и их беду.

--- Почему вы сидите здесь, связанные этими нитями с деревом? --- удивилась она.

--- Мы не знаем, --- ответил Таро.
--- Мы боролись друг с другом, но ни один не победил.
Мы уступили друг другу, но лишь поменяли концы нитей местами.
Затем Мичи уступил мне по жребию, но мы всё равно не можем сойти с места.

--- Как много нитей ненависти, --- заметила девочка, --- возможно ли распутать все!
Вы тянете за собой свои дома\FM, давно умерших предков и даже тяжёлые слова!
\FA{
Дом --- объединение, включающее два-три родственных клана.
}
Виновато ли дерево сукэ?
Может быть, вам просто нужно друг друга простить?

--- Простить? --- удивились герои.

--- Я играла с братиком, --- сказала девочка.
--- Порой мы колотили друг друга, когда играли в героев.
Порой я кричала, что ненавижу его, когда он был хорохито.
Но потом мама звала нас кушать и мы забывали свои роли и свои обиды.

Посмотрели друг на друга Мичи и Таро.
Долго смотрели они, вечность пришла и прошла, а они всё смотрели.
И вот нити ненависти ослабли и исчезли без следа.

Подняли Мичи и Таро девочку на сплетённые руки и пошли по чёрной тропе дальше.
А дерево сукэ рассыпалось в прах, едва герои скрылись за поворотом.

\chapter{\ml{$0$}{[U] Наедине со Зверем}{Face the Beast}}

\section{[U] Ожидание допроса}

\textspace

Я почти физически ощущал натянутую вокруг <<сеть>>, замечающую малейшие изменения поля Кохани"--~Вейерманна в окрестностях моего демона.

Допрос урождённых демонов и оцифрованных сапиентов различается по методике.
С урождёнными демонами не церемонятся --- интерфектор сразу влезает в их память, добывая нужную информацию.
Ядро личности оцифрованных представляет собой виртуальную пространственную структуру с очень сложной архитектоникой, идентичную мозгу сапиента.
Добыть информацию из материального мозга можно после его изучения в замороженном виде, оцифрованный мозг можно только убедить или вынудить.

Для убеждения урождённых сапиентов существует более сорока сложнейших протоколов допроса.
Для принуждения --- один-единственный пункт:

\begin{quote}
<<Подвергнуть эскалационной обработке в Чистилище>>.
\end{quote}

<<Именно мы, оцифрованные, ковали победу Ордена Преисподней над Картелем.
Наверное, только сейчас Ад понял, насколько оцифрованные опасны>>, --- подумал я с неуместным злорадством.

Чханэ и Митрис попали в плен.
Это было ясно, как солнечное утро.
Возможно, что они даже не добрались до Тси-Ди --- их спугнул патруль, и друзья перенеслись обратно в звёздную систему Тра-Ренкхаля, кишащую приведёнными в боевую готовность агентами из-за устроенной Заяц диверсии.
У них не было шансов скрыться.

Время --- жестокая вещь.
У Заяц его было достаточно для сумасшествия.
А вот друзьям для дела не хватило.

\section{[U] Допрос}

\epigraph{И сказал Ал-ла Талиму: <<Сними одежды твои и драгоценности твои и оставь оружие твоё и нагим войди во врата джайанна\FM.
Огонь будет жечь тебя, и ибли\FM{} будут насмехаться над лицом твоим и членами твоими и походкой твоей, и дивы\FM{} припомнят самые стыдные дела твои и позор твой, но ты иди, как ходишь по дворцу твоему.
И если не склонится голова твоя, и если взгляд твой будет горд и милостив, как прежде, падут ниц стражники вторых врат, и выйдешь ты из джайанна перерождённым, и не будет тебе более нужды ни в одеждах, ни в драгоценностях, ни в оружии, ни в придворных льстецах, и будешь ты отныне равным Мне, ибо Я прошёл тем же путём>>.}
{Хакем-Аят, 14:3--5}
\FA{
Джайанна --- <<судилище>>, несотворённое место, управляемое женщиной по имени Джаблел.
Согласно Хакем-Аяту, все боги мироздания должны пройти через него, чтобы стать Творцами.
}
\FA{
Ибли --- <<мучители>>, неразумные прихлебатели Джаблел.
}
\FA{
Дивы --- <<писцы>>, бесстрастные существа, ведущие летопись Вселенной.
}

\textspace

--- Аркадиу Талианский Шакал.
Согласно нашим данным, первоначальный состав Скорбящих --- оцифрованные тси и люди Тра-Ренкхаля.
91\% --- ветераны битвы на Могильном берегу.

--- Это так, --- подтвердил я.

--- Эти данные не нуждаются в вашем подтверждении.
Некоторых из этих новобранцев мы успели схватить несмотря на то, что вы успели предупредить всех перед поимкой.

--- Кого именно вы схватили? --- спросил я.

--- Это для вас уже не имеет значения, --- ответил голос.

--- Давайте так, --- сказал я.
--- Чем больше информации вы сообщите мне, тем выше вероятность, что я стану с вами добровольно сотрудничать.

--- Давайте так --- с этого момента вы перестанете торговаться, --- ответил голос.
--- Ваша судьба зависит от тех данных, которые вы нам предоставите.
Вас позвали для уточнения некоторых деталей.
Был проведён анализ вашей личности и вынесено решение --- эскалационную обработку в Чистилище отменить.

Если бы у меня была нервная система и кожа, меня прошиб бы пот.
Начало протокола номер тринадцать.
Дело дрянь.

--- Если приведённый коэффициент пользы превысит $0.986$ --- вы будете жить.
Если суммарный коэффициент дезинформации превысит $4.2$ --- вы будете подвергнуты терминальной обработке.

\ml{$0$}
{Терминальная обработка.}
{Terminal processing.}
\ml{$0$}
{Меня замучают до полной потери рассудка в самой ужасной пыточной камере всех времён.}
{I will be tortured into total madness in the worst torture chamber of all time.}
\ml{$0$}
{Я расскажу всё, что знаю, я додумаю то, чего не знаю, я выплюну кровавым комом всю свою личность, всю свою жизнь до самых пелёнок, но механизм не остановится, пока программное ядро не перестанет адекватно отвечать на внешние сигналы.}
{I will tell what I know, I will make up if I don't, I will spit out like a blood clot all my identity, all my life to the first birth, but device will proceed---until my kernel have stopped adequately responding to external signals.}

Разумеется, приведённый коэффициент пользы на труднодостижимой, но вполне реальной цифре $0.986$ --- это уловка.
Моя судьба уже предрешена.
Вряд ли кто-то верил, что я попадусь на такой простой трюк, но протокол есть протокол.

<<Общий смысл пункта 8, подпункта .3 --- нельзя лишать пленника надежды, но единственной надеждой пленника должен быть допрашивающий>>.

Я опустил голову.
Что может сделать маленький шакал против стаи матёрых волков?
Только бы избежать Чистилища.
Страшный и недостойный конец существования...

--- Кто была женщина, которая управляла системой?

--- Тси, которую мы нашли в капсуле в состоянии анабиоза.

--- Информация соответствует истине с вероятностью 98\%.
Она согласилась сотрудничать с вами?

--- Она втёрлась к нам в доверие, чтобы реализовать план мести.
Мы потерпели поражение из-за её действий.

--- Информация соответствует истине с вероятностью 76\%.
Последний экран помог освободиться вашим союзникам, и они вступили с нами в бой.
Как вы объясните это действие женщины?

<<Вступили в бой>>.
Лёгкая оговорка, которая могла означать только одно --- Чханэ и Митрис мертвы.
Или меня пытаются в этом убедить.

--- Я убедил её помочь.

--- Информация соответствует истине с вероятностью 84\%.
Вернёмся к более ранним событиям.

Допрашивающий перешёл к другому вопросу.
Они мертвы.
Я вдруг вспомнил ласковую улыбку Чханэ.
Она любила вытаращивать глаза, когда видела, что я на неё смотрю.
А потом крепко зажмуривалась и целовала меня в нос.
Глупая привычка, которую я ношу в памяти, словно свёрток с драгоценными камнями.

Когда я оцифровал подругу, Чханэ сказала: <<Ну всё, Лис, ты от меня теперь вечность не отвяжешься>>.
Я предлагал ей найти убежище, жить в спокойствии.
Она ответила просто: <<Будем сражаться>>.
Что ещё могла ответить женщина, которую я полюбил больше жизни?
Но даже зная, какие опасности несло решение, облечённое в эти простые слова, я не был готов к её смерти.

--- Где вы прячете оборудование для оцифровки?

--- Оборудование мы прятали в храме Тхитрона, частично --- в купеческом доме.
Оно было уничтожено сразу после оцифровки, больше там ничего нет.

--- Информация соответствует истине с вероятностью 99\%.
Кто занимался сборкой оборудования?

--- Грейсвольд Каменный Молот.

--- Информация соответствует истине с вероятностью 15\%.
Суммарный коэффициент дезинформации $3.8$.
Требуется объяснение или другой ответ.

--- Первоначально предполагалось, что ветераны Могильного берега вступят в ряды Ордена, --- ответил я.
--- Они даже прошли подготовку согласно протоколу 18, что вы можете легко проверить по паттернам их действий.
Это было до того, как мне стали известны подробности политики Ада по отношению к оставшимся тси.
Я удивлён, что Грейсвольд не рассказал этого вам.

Молчание.

--- Грейсвольд Каменный Молот будет допрошен.
Как много существ вы успели оцифровать?

--- Семьсот шестьдесят восемь.

--- Информация соответствует истине с вероятностью 85\%.
Назовите всех демонов, чьи модули использовались для оцифровки.

--- Я и Митрис Безымянный.

--- Информация соответствует истине с вероятностью 70\%.
Чьи паттерны использовались для интерфекторов?

--- Анкарьяль Кровавый Шторм.
Далее интерфекторов готовили агенты под прикрытием.
Мне не известны их имена, связь с ними осуществляла Таниа Янтарь.

--- Информация соответствует истине с вероятность 89\%.

Ясное дело, ответить им на это нечего.
Интерфекторов готовила Чханэ.

<<Я не могу отомстить за подругу кровью, но сомнений вам оставлю предостаточно.
Кто знает, которая капля дезинформации расшатает вашу систему.
Излишняя подозрительность породит паранойю, партийные чистки завербуют новых агентов лучше, чем это сделаем мы.
Я не последний из тси... вернее, из Скорбящих>>.

Мысленная оговорка почему-то меня развеселила.

--- Назовите имена всех новооцифрованных.

--- Я назову, если вы сообщите имена тех, кто попал к вам в плен.

Молчание.

--- В плен были взяты тридцать восемь Скорбящих, --- сказал голос.
В моей голове возникли имена.
--- Один был уничтожен.

--- Кто был уничтожен?

--- Тхартху Танцующая Тень.

Перед моими глазами промелькнуло доброе лицо женщины, которая больше всего на свете любила рисование и сказки.
Несмотря на недостаток пленных, Ад решил уничтожить её --- оставлять визора в живых было чересчур большим риском.
Митрис Земляная Змея тоже среди пленных --- он был рядом со своей подругой до конца.

<<Прощай, Птичка.
Пусть твой путь в пристанище будет лёгким>>.

Я напрягся, и в моей памяти возникли нужные имена.

--- Информация соответствует истине с вероятностью 97\%.

--- Этот список бесполезен для вас, как и вербовка агентов.
Во время подготовки я предупредил товарищей, чтобы они относились к пленённым как к врагам, а к информации, которой располагали пленённые, как к известной врагу.

Молчание.

--- Информация соответствует истине с вероятностью 90\%.
Балл ПКП --- $0.979$, балл СКД --- $4.11$.

Я подавил приступ судорожного смеха.
Допрос окончен, последняя фраза была попыткой выжать остатки сока из жмыха.
<<Пограничные значения.
Бойся, предатель, метайся в попытках спастись, ведь спасение в семи тысячных от тебя.
Не дождётесь>>.

--- Это всё? --- осведомился я.

--- Вся требуемая информация получена.
Вы будете казнены через 16 минут.

--- Информация соответствует истине с вероятностью 100\%.
Приведённый коэффициент презрения --- единица, --- съязвил я.

Ответом было молчание.

\section{[U] Казнь}

\epigraph
{Et in Arcadia ego\FM.}
{Крылатое выражение культуры Ромай.
Древняя Земля}
\FA{
И в стране пастухов я (эллатинский).
}

\textspace

Это море совсем не было похоже на те лазурные южные моря, за которые лорды и короли ломали копья и жизни подданных.
Над северным морем всегда царила лёгкая тьма, и в этом похожем на вуаль мраке терялись далёкие горизонты.
Холодная чёрная вода шептала, а не пела, её тяжесть вгоняла в сон.
Берег был отнюдь не ласковым пляжем, на мягком прохладном песке лежали следы недавних штормов --- обломки досок, обрывки ткани, разная мелочь... и водоросли, огромные кучи бурых гниющих водорослей, от которых исходил лёгкий прелый аромат.
Чуть поодаль росли золотистые, болезненного вида сосны и тонкий, ужасно колючий шиповник с облетевшими листьями, усыпанный огромными алыми ягодами.

Я смотрел на догорающий вдали закат.
Влажный ветер бил мне в лицо, солёно-пряный аромат моря въедался в волосы, чтобы не раз вернуться обрывками сновидений в будущем.
Я думал, что обязательно должен закончить свой путь здесь.
Южные моря с их тёплой солью, кровавыми сражениями, чернявыми красотками, вином и изысканными фруктами были созданы для жизни.
Для смерти же мне не найти лучше этого всепрощающего моря, спокойного берега с его заспанными соснами и мрачновато-жизнерадостным шиповником.

Тогда я считал себя взрослым, повидавшим жизнь существом.
Кто я сейчас?
Мудрый старик, скептически относящийся к собственной мудрости?
Молодой дурак-максималист, задумавший невозможное?
Одно было несомненно --- о северном море придётся забыть.
В моём распоряжении --- виртуальная каморка, имеющая размеры лишь для моего сознания.
Темница для разума, из которой я уже не выберусь.

Может, поверить хоть сейчас, что там, за гранью, что-то есть?
Я никогда не желал рая и вечного блаженства.
Кусочек моря, Чханэ, Атрис и Митхэ, встречающие меня --- всё, что нужно.
Это будет самообман, оскорбление для истины, за которую я боролся, но смерть в отчаянии --- оскорбление для всех, кто мне верил.

Рядом материализовался Грейсвольд.
Я облегчённо вздохнул.
Одно знакомое лицо лучше, чем ничего.
Незачем что-то придумывать.

--- Это ты меня казнишь? --- осведомился я.

--- Нет, не я, --- тихо пробормотал Грейс.
Я улыбнулся.

--- Передай ей, что я буду рад её увидеть.

Грейсвольд робко ухмыльнулся.

--- Она тебя слышит.
Пока ждёт приказа.

--- Почему палачом назначили Анкарьяль, мне ясно.
Почему шестнадцать минут?

--- Они до сих пор не могут решить, что с тобой делать, --- объяснил Грейсвольд.
--- Многие кричат о том, что тебя нужно уничтожить немедленно.
Однако столько же придерживаются мнения, что тебя нужно отпустить.
Кое-кто робко предлагает с тобой договориться.
Есть и достаточно прагматичная точка зрения.
Сейчас разрабатываются новые протоколы для Чистилища, основанные на медленном, мягком воздействии, а не на... ммм... экстремальной обработке.
Исследования говорят о том, что сапиентов нужно гнуть, медленно и терпеливо, а не ломать сильным ударом.
Одним словом, им требуются подопытные.

--- Меня будут подвергать мелким неудобствам, типа имитации холодной клетки? --- скривился я.

--- Зря смеёшься.
Пыточная камера ломает дух не так часто, как это делают годы тюрьмы, медленные болезни и нищая старость.
Хотя основное направление... ммм... это удовольствие.
Не те мощные оргазмы, чередующиеся с дикой болью, как при обычной обработке, а простые, вроде тепла постели, поцелуев и чувства наполненного желудка.
Разработчики говорят, что результаты многообещающие, хоть и не сиюминутные.

Я хмыкнул.

--- Ты --- недоласканный ребёнок, --- напомнил Грейс.
--- Да, детство твоего мозга осталось далеко позади, ты неоднократно проходил коррекцию личности.
Но есть кое-что, что останется с тобой навсегда.
Материнская ласка может стать к тебе отличным ключом, хочешь ты этого или нет, и шансов устоять у тебя меньше, чем у твоих друзей-тси.

--- Я всегда буду помнить о войне!
Я всегда буду помнить, что чувства --- иллюзия!

--- Насколько долго ты будешь об этом помнить, если иллюзию нельзя отличить от реальности? --- резонно спросил технолог.

--- И зачем ты мне это рассказываешь?

--- Я молю лесных духов, чтобы сторонники казни победили, --- глухо сказал Грейсвольд.
Это была правда.
Я чувствовал его чёрное отчаяние.

Если сказанное --- истина, то это не самый плохой вариант.
Я увижу своё северное море.
У меня будет мать и тёплый очаг, Чханэ, дом, розовый куст и сад с цветущими вишнями.
Я смогу...

Я резко оборвал мысль.
Вот они, эти бреши, о которых говорил Грейсвольд.
Мои собственные потаённые мечты, из-за которых я всё и начинал...

Я должен умереть сейчас.
Иначе...

Но я же хочу жить!
Я не боюсь смерти, но ужасно хочу жить!
Я дрался ради жизни, ради опыта существования в лучшем мире!

<<Изверги>>, --- жалобно выплюнул мой разум.
Однако море и вишни упорно маячили перед внутренним взором...

Предел есть у всех.
Похоже, что я своего достиг.
Неважно, какая судьба меня ждёт --- прежний Аркадиу Люпино прекратит существование здесь и сейчас.

Технолог покряхтел и, создав виртуальный стул, сел на него.
Такая простая, совершенно человеческая попытка оттянуть неизбежное.

Мы помолчали.

--- Да, если ты хочешь знать масштабы действий.
В Аду было выявлено сорок тысяч агентов Скорбящих.

Сорок тысяч.
На нашей стороне было три процента адских демонов.
Я рассчитывал на десяток, на сотню, на пять сотен, но даже в самых смелых мечтах я не представлял, что у меня будет такая команда --- огромная законспирированная децентрализованная сеть, проросшая Ад, словно гигантская грибница.
Сорок тысяч.
Цифра бесцельно дрейфовала на поверхности разума, словно принесённый отгремевшим штормом обломок с закладной доской.

--- Более трёх тысяч пленены отделом 100 и уничтожены, остальные успели уйти.
И это только урождённые сапиенты.
Количество урождённых демонов, которые работали на вас, неизвестно, эту информацию хранят под строжайшим секретом, во избежание...
Ты поднял большую волну, Аркадиу.

--- Нет нужды поднимать то, что поднимается само, --- возразил я.
--- Почва для подобного движения уже давно перезрела.
Число сторонников говорит само за себя.
Что-нибудь ещё?

--- Есть данные, что... --- Грейсвольд замялся.

--- Договаривай.

--- Ад начал переговоры с Картелем.

--- Что? --- осведомился я, уверенный, что ослышался.

--- Именно так.
В Картеле Скорбящие тоже были.
Демоны приостановили военные действия из-за страха перед вами.
Сейчас обсуждается вопрос о тотальной чистке в обеих организациях.
Оцифрованных хотят извести под корень.

--- Это ложь, --- улыбаясь, прошептал я.
--- Клан Тахиро, клан Усмане...

--- Почти все члены клана Тахиро были агентами Скорбящих, --- вздохнул Грейс.
--- Из трёх тысяч уничтоженных две --- клан Тахиро, первая и третья генерации.
Они приняли на себя самый тяжёлый удар.
Ещё семьсот --- Усмане.
Когда до Ада дошла тревога, Тахиро попытались прийти на помощь Скорбящим-тси.
На Капитуле и ещё семи планетах произошли полномасштабные войны хоргетов.
Сорок процентов Тахиро, восемьдесят два процента Усмане уничтожено, потери Ада --- двенадцать к одному, и это официальные данные.
Реальное положение дел намного хуже.
Усмане остались бы в стороне и попытались переждать бурю, но пара их старейшин попали под горячую руку, и весь клан, не разбираясь в ситуации, прислал Аду недвусмысленное сообщение: <<Ahirat>>\FM.
\FA{
Ahirat --- послание, использовавшееся на планете Земля Врачевателей.
Значение: <<Вы пересекли последнюю границу вражды.
Мы будем убивать вас всем, чем можно убить, пока не умрём сами>>.
}
Всё в лучших традициях Хакем-Аята\FM, только чёрно-зелёной ткани не хватает.
\FA{
Священная книга народов Земли Врачевателей, регламентировавшая в том числе правила ведения войны.
}
Боюсь, что этот клан прекратил своё существование --- они всегда считались самыми жестокими среди оцифрованных, но в пунктуальности им не откажешь.

Я промолчал.
Где-то на краю сознания гремели воинственные крики, вопли отчаяния и горестные восклицания на давно забытых языках.
<<Тахиро-джиме, банджай!>>
<<Монд-ааа! Джайаннау-ле харек, монд!>>

--- Анкарьяль получила приказ о твоей ликвидации.

Анкарьяль появилась рядом.
Лицо женщины было бесстрастно.
Я встал со стула, расправил плечи, поднял голову и улыбнулся друзьям.
Хоть это и виртуальный мир, хоть у меня больше и нет ни плеч, ни головы, хоть это и не изменит мою судьбу, но всё же...
Вдруг это изменит что-то другое, например, Вселе...

--- По ощущениям похоже на засыпание, --- сказала Анкарьяль.
Её глаза встретились с моими.

\section{[U] Разными дорогами}

\epigraph
{Лучшая победа --- это победа, которая носит маску поражения.}
{Ликан Безрукий, создатель языка Эй}

Кто я?
Где я?

Я шёл по дороге, запинаясь о камни.

--- Меня зовут Аркадиу, --- сказал я вслух и сам удивился собственным словам.

Рядом со мной шли трое --- две женщины и мужчина.
Их лица были суровы, они смотрели только на меня.

Мысли путались, ощущения едва задевали поверхность разума.
Осталось одно огромное чувство, словно дыра в голове --- острое чувство утраты.

--- Raj sejra? --- спросила та, что повыше, агрессивно скривив губы.

--- Ta oj, Nar.
Kar lejma, --- ответил мужчина.

--- Dejnalo oj na mia, --- добавила низкорослая, как мне показалось, немного печально.
Она единственная меня жалела.

Откуда-то в памяти всплыли слова --- язык Эй, таблица B0.
Но из сказанного я не понял ни слова.

Вдруг пришло понимание --- меня только что убили.

--- Что вы делаете со мной? --- спросил я.

--- Ga, --- мужчина грубо толкнул меня в спину, и я свалился на колени.
Мой тюремщик вздёрнул меня на ноги и молча рукой указал направление.

--- Jol.
Laj de, --- сказала высокая и, вытащив оружие, направила его на меня.
Пистолет, метающий металлические стержни.

<<Эти существа убили меня, а сейчас хотят убить ещё раз...>>

Страшно заболела голова.
Я снова упал на колени.

--- Прекратите это, я не хочу это больше чувствовать, --- пробормотал я.
--- Кто вы?
Почему мне так плохо?

Мужчина кивнул высокой.

Я инстинктивно зажмурился, ожидая боли.
Кровь пульсировала в висках, как боевой барабан...

Боевой барабан Талии.
Большой барабан из воловьей кожи.
Он звучал точно так же.

--- Чего стоишь? --- недовольно сказал мужчина.
--- Глаза открой.

Я осторожно приоткрыл глаза.
Высокая стояла, устало прислонившись к мужчине.

--- Он всё-таки туповат.
Как думаешь?

--- Это да, --- согласился мужчина.
--- И ведь понимает, но самое главное до него не доходит.

Я смотрел на них во все глаза.

--- Вы меня не убьёте?

--- Боюсь, что однажды не выдержу и прихлопну, --- посулился мужчина.
--- Открой глаза!
Да не эти!..

Мир вновь обрёл краски.
За каменными незнакомыми лицами я увидел родные, ласковые черты.
Память вернулась.

--- Неужели ты и в самом деле думал, что я убью друга? --- засмеялся Грейс.
--- Плохо же ты меня знаешь.

--- Но как ты догадался?
Информации о связи было очень мало!

--- Я просто не поверил, что после произошедшего ты бросил меня навсегда, --- тихо сказал Грейсвольд.

Я задумался.

--- Что произошло?
Как ты обманул?..

--- Обычное копирование, Аркадиу.
Болванку Анкарьяль носила с собой.

--- То есть технически я мёртв?

--- Технически --- да.
И по документам Ордена ты тоже значишься как ликвидированный элемент.
Здорово, правда?

--- Эээ...

--- Не будь идиотом, Аркадиу.
Думаешь, мы бы с Анкарьяль ещё существовали, если бы у одного интересного типа не хранились наши копии?
Тем более твоё тело осталось живым, а значит --- ты как личность не почувствовал небытия.
Тело --- необязательная часть демона, и в век ангельских технологий обратное также верно.

--- И как это регламентируется законами Ада?

--- Никак.
Знают многие, использует кто может.
Неписанное правило таково: если твоя смерть доказана --- значит, ты вне игры.
Вот тут в дело и вступает твоя команда.
Фальсификация спасений --- это целая отрасль в наше время, разумеется, тоже негласная.
Думаешь, на высших иерархов не было покушений и даже более того --- удачных покушений?
Если рядом есть приспешники, в любой момент готовые восстановить тебя из резервной копии, смерть --- не самое страшное событие.

--- Кто ещё жив? --- допытывался я.
--- Тахиро? Айну? Гало?

--- Все они в пристанище духов, --- ответил Грейсвольд.

--- Что такое <<пристанище духов>>?
Хранилище резервных копий?
Или это, чтоб вас всех, укромная планета, где живут те, кого давно считают мертвецами?

Технолог промолчал.
Я вздохнул.

--- Я тоже туда попаду?

--- Увы, --- развела руками Анкарьяль.
--- Для тебя и прочих Скорбящих пристанища не будет.
Поэтому --- береги себя.

--- То есть всё это бесполезно, --- опустил голову я.
--- Нужно бороться с системой, о размерах которой я...

--- Я рада, что ты наконец понял задачу, --- вмешалась низкорослая.
--- Бороться нужно с системой, а не с отдельными личностями.
До тех пор, пока устаревшая система не изменится, она сама будет рождать бунтарей.
Против этого невозможно выстроить оборону, от этого нет спасения.
Те, кто послали тебя на смерть, этого так до конца и не поняли.

Анкарьяль кивнула.
Низкорослая женщина улыбнулась одними губами.
Её глаза, сильно косящие в разные стороны, по очереди посмотрели на меня --- словно два разных человека, весёлый и задумчивый.

--- Кто-то должен рассказать сапиентам Земли, что они на что-то способны, --- проговорила она, манерно растягивая слова.
--- Гонка ещё не завершена.
Да, кстати, Стигма Чёрная Звезда.
Для меня большая честь, Аркадиу Люпино.

Я пожал её тонкую, но крепкую руку.

--- Как хорошо знакомиться с друзьями, да? --- подмигнула Анкарьяль.

--- Я достаточно давно возглавляю группировку, стремящуюся к союзу с Картелем, --- усмехнулась Стигма.
--- Успела накопить большой опыт.
Так что ты не первый, господин Люпино.
Но отдаю тебе должное --- на такие наглые действия мы пока не отважились.

--- Это провал, --- возразил я.

--- Я была лучшего мнения о тебе, как о стратеге, --- нахмурилась Стигма.
--- Мы собрали огромное количество практически бесценного материала.
Такая информация просто не может достаться без жертв.
И я очень рада, что смогу работать с тобой и дальше.

--- Он ещё в себя не пришёл, --- объяснил Грейсвольд.
--- Это нормально.

--- Это предательство, --- сказал я.
--- Вас могут...

Анкарьяль внезапно обняла меня изо всех сил.
И тут же отстранилась.

--- Могут, --- подтвердила Стигма.
--- И даже сделают, я уверена.

--- Предательство, --- повторил Грейс и вздохнул.
--- Понятие, ставшее игрушкой в руках пропаганды.
То, что мы сделали, скорее гуманность.

--- Иди, Кар, --- пробормотала Анкарьяль.
--- Тебя ждут.

Я вдруг вспомнил всё.
Каждую деталь нашего сокрушительного поражения.

--- Меня больше некому ждать, --- с горечью ответил я.

--- А твой народ? --- спросила Стигма, указывая куда-то назад.

Я не стал оборачиваться.
<<Взгляд>> почувствовал ободряющий свет Чханэ и Митриса.
Грейс смотрел на меня нежно, словно на любимого ребёнка.

--- Грейс, я перед тобой в неопла...

--- Молчи, --- прервал меня Грейсвольд.
--- Молчи и иди, человек.

Я опустил голову.

--- Что с остатками тси?

--- Сожалею, --- ответил технолог.
--- Адская пропаганда и селекция уже начали своё чёрное дело --- из свободного народа делают мягкотелых счастливых существ.
Наш заговор --- непродуманный, честно говоря, --- заставит Ад закрутить гайки ещё сильнее, и я не знаю, доживёт ли свободолюбивый народ тси до момента, когда давление сорвёт крышку котла.
Все были уверены, что нас мало.
Что поделаешь, нам старательно внушают, что мы одиноки в своих стремлениях...

Грейсвольд помолчал и подумал.

--- Тси, тси.
Вообще я бы на вашем месте... --- Грейсвольд театрально оглянулся на товарищей и вполголоса продолжил:
--- Я бы на вашем месте собрал генетические образцы тси, а также данные об их культуре, и попытался бы возродить их где-нибудь на укромной планете.
Я не сторонник воскрешения отжившего, но тси по-прежнему опережают прочих сапиентов на сотню тысяч лет, было бы жалко их потерять.
Ты у нас биодиктиолог, верно?
Молодец, хорошо подготовился.
А сзади стоят две женщины --- живые носители культуры.

Я кивнул.

--- А вы?

--- А что мы? --- пожала плечами Анкарьяль.
--- Мы выполнили приказ --- уничтожили подполье.

--- Сейчас занимаемся тем, что казним тебя, --- добавила Стигма и мило усмехнулась.

--- Мы вернёмся обратно, дружище, --- заключил Грейс.
--- Пока ты остаёшься в тени, нам ничто не угрожает.
Кто-то ведь должен аккуратно убедить Орден признать сапиентов Земли, когда придёт время.
Считай то, что произошло, уроком, а не поражением.
Да, кстати, тебе послание от... сам знаешь кого.

Мы соприкоснулись головами.
В моём разуме зазвучал знакомый слог, и на душе потеплело.

--- Друзья по ту сторону баррикад?..

--- ... нужны всегда, --- закончила Анкарьяль.
--- Друзья, а не агенты.
Надеюсь, ты это усвоил.

Я оглянулся на Чханэ и Митриса.
Они ждали.
Чханэ когда-то говорила, что ждала меня вечность.
Забавная гипербола, учитывая, что ей тогда было всего сорок дождей...
Теперь я понимал, что это не поэзия, а тонкая, неочевидная истина.
Вселенная расширялась, кварки собирались в триплеты, фермионы склеивались в атомы, пыль собиралась в облака и загоралась звёздами, проклетки эволюционировали в первых людей, первые люди создали хоргетов, дали начало народам Драконьей Пустоши и Тси-Ди, а она всё это время ждала, пока детали мозаики мироздания не сложатся в нужную картину.
И всё это для того, чтобы через миг совместного пути, после лёгкого соприкосновения рук снова распасться на первородную квантовую пыль.

--- Так, я вас оставлю, --- вдруг сказала Стигма.
--- Меня вызывают по резервному каналу.
Прощай, Аркадиу.
Надеюсь, что нам больше не придётся видеться при таких обстоятельствах.

Стигма кивнула коротко остриженной головой, вскочила на мотоцикл и уехала, поднимая лёгкие клубы песчаной пыли.

Грейсвольд бросил взгляд ей вслед и улыбнулся подошедшему Атрису.

--- Мы --- это ветер, --- сказал Атрис.
--- Мы будем бросать пылинки на гранитную скалу, микрона за микроной стачивая её.
Вы --- это вода.
Вы будете проникать в мельчайшие трещинки скалы, замерзать и таять, расширяя их.
Однажды на этой скале смогут расти цветы, и она превратится в зелёный холм.
Так оживала и моя планета, и я тому свидетель.

--- Жаль, дружище, что она больше не твоя, --- посочувствовал Грейсвольд.

--- А Лотос твой, Грейс? --- хитро улыбнулся Атрис.

Грейсвольд смутился.

--- Я лишь сотворил Лотос.

--- И этого уже не изменить ни одному завоевателю, --- закончил Атрис.
--- Чтобы изгнать из горшка горшечника, нужно стереть горшок в пыль.

--- Когда-то нам поклонялись, --- сказал Грейсвольд, --- от нас ждали благ, справедливости и спасения.
Знали бы наши верующие, как мы с тобой слабы перед системой.

--- Верно, --- согласился Атрис.
--- Впрочем, на молитвы я не отвечал уже давно, многие из моих созданий поумнели и рассчитывают только на себя.
Лучший бог --- тот, о существовании которого даже не догадываются.
Тот же, кто требует почестей, молод и глуповат.

Грейсвольд оглянулся ещё раз и снова вполголоса обратился к Атрису:

--- Атрис, ты достал кольцевую теплицу?

--- Нет, --- сказал Атрис.
--- А ты, технолог, вспомнил схему Золотого города?

--- Нет, --- с той же интонацией ответил Грейс.
Секунда напряжённого молчания --- и оба от души расхохотались.
Впрочем, смех утих почти сразу.
Мужчины долго смотрели друг на друга.

--- Если к концу войны мы... разминемся, --- голос технолога дрогнул, --- заглядывай на Лотос.
Нужно время от времени поправлять орбиту Часовой Луны, она немного сползает.
Водопад, Факел и Паутина-город требуют реставрации, людишки отбивают кусочки от фокусирующих кристаллов и делают из них ожерелья.
Объясни им, пожалуйста, что так делать непорядочно и следует уважать чужой труд.
В остальном, думаю, ты разберёшься и сделаешь даже лучше, чем было.
Ты хороший бог.

Атрис кивнул.

Пора было уходить.

Я по очереди пожал руки Грейсу и Анкарьяль.
Расстались мы в полном молчании.
Чханэ и Атрис взяли меня за талию с двух сторон и повели на север.
Маленькая Митхэ шла рядом, окуная ступни в набегающую солёную пену.
На её лице впервые за долгое время появилась счастливая улыбка.

--- Так клетки всё-таки у тебя, --- сказала она.
--- Ты успел...

--- Я всегда тебе верил, --- пожал плечами Атрис.
--- Ты сказала <<Сейчас или никогда>>.
И в последний миг прилетело несколько клеток с сорванного ветром листа.

--- Меня терзает чувство вины из-за несчастного создания, которому я дала ложную надежду.
Эти тянущиеся ко мне руки я буду помнить до нашего последнего мгновения.

--- Надежда не ложная, --- сказал Атрис.
--- Мы вернёмся, и ты сама дашь ей свободу.
Теплицы живут очень долго.
Наверное, сейчас она спит и видит счастливые сны.

--- Ты думаешь, мы сможем отвоевать Тси-Ди у Машины? --- спросила Чханэ.

--- Может быть, мы сможем с ней договориться, --- предположил я.
--- Твоей искренности, Чханэ, поверил даже Лусафейру.
Поверит и Машина.

--- Если получится, я буду петь моему деревцу каждый день, --- пообещала Митхэ.
Атрис кивнул.

Я ещё раз повторил про себя слова послания Лусафейру:

\begin{quote}
<<Сражение закончено, но борьба продолжается.
Я не буду поддаваться.
Это сделает тебя слабее, чем ты можешь быть.
Я оставил тебя в живых --- неплохая фора, на мой взгляд>>.
\end{quote}

<<Однажды, мой друг, мы встретимся как равные, и я пожму твою руку>>, --- таким был ответ, который я передал Грейсвольду.
Разумеется, Лусафейру может его не получить --- конспирация.
Но, возможно, когда-нибудь...

Уже отойдя на десяток шагов, я услышал тихую речь Анкарьяль:

--- Грейс, все схемы Золотого города у тебя?
Ведь я права?

--- Не понимаю, о чём ты, --- отмахнулся старый технолог, говоря намного громче, чем следовало бы.
--- Нет у меня никаких схем.

И уже тише добавил:

--- Пожалуйста, хватит стирать себе память каждый раз, когда до тебя доходит.
От отдела 100 меня это не защитит, да и ты перестанешь надоедать с этим вопросом.

--- Эээ... каким вопросом? --- удивилась Анкарьяль.

Грейсвольд громко захохотал.
Я оглянулся и увидел, как он прижал крякнувшую от неожиданности подругу к могучей груди.
К городу они пошли пешком, обнявшись, и их следы на мокром песке --- неуклюжую цепочку больших и маленьких ступней --- смыла набежавшая ласковая волна.

\chapter{Скорбящие}

\section{[U] Ключ к себе}

\textspace

--- Кагуя, вы принадлежите к клану Тахиро, --- сказал Грейсвольд.
--- Вы полагаете, что две генерации и десять тысяч версионных правок сделали из вас урождённого демона?
Ваше ядро было и остаётся тенью человеческого мозга.

--- Я не настолько глупа, чтобы оправдывать нечто только из-за того, что это нечто свойственно мне, --- сладким голосом пропела Кагуя.
--- А если...

--- Уэсиба Серозмей, --- тихо шепнул Грейсвольд.
Лицо Кагуя залила краска, глаза сузились от гнева.

--- Моего брата от меня отделяют три тысячи версионных правок.

--- А от Тахиро вас отделяет и того больше, но вы такая же наглая и упрямая, как мой друг, --- сказал Грейсвольд.
--- Вы можете не любить Тахиро, вы можете не любить меня, но я всегда буду любить его в вас.

--- Любовь?
Я не понимаю, о чём вы говорите, --- Кагуя со скучающим видом покачала головой.

--- Значит, однажды Вселенная преподнесёт вам сюрприз, --- улыбнулся Грейсвольд.

Кагуя пожевала губу.

--- Что вам от меня нужно?

--- Мне нужно знать, о чём с вами говорил известный вам с Самаолу демон.
Кажется, его называют <<патроном>>.

--- Вы в шаге от того, чтобы подписать себе приговор, Грейсвольд, --- ощерилась Кагуя.
--- Отдел 100 сочтёт это превышением...

--- Не будьте дурой, Кагуя, --- устало сказал технолог.
--- Да, я из Скорбящих, и пожалуйста, покажите мне того, кто об этом хотя бы не подозревает.
Истинное равновесие Нэша достигнуто, вы это знаете.
Вы не можете причинить вреда мне, не погибнув.

--- Ложь, --- бросила Кагуя.
--- Какое равновесие?
Мы теряем позиции в битве с Картелем, наши демоны гибнут...

\ml{$0$}
{--- Кто гибнет? --- грустно улыбнулся Грейсвольд.}
{``But who tends to die?'' Grejsvolt sadly smiled.}
\ml{$0$}
{--- Молодёжь, не нашедшая своего места, жители планет, которые уже даже не ресурс для выживания, а топливо для войны!}
{``Young who failed to find their place, and dwellers of planets---not as vital resource, but as fuel for war!}
\ml{$0$}
{Дефицита масс-энергии давно нет.}
{Mass-energy deficiency is long gone.}
Старые иерархи как сидели, так и сидят на своих местах, и их никто не пытается устранить.
Те же, кто пытался что-то изменить в этом миропорядке, уже давно разделили судьбу вашего прародителя.

--- Информация не способна поколебать истинное равновесие Нэша, --- заметила Кагуя и притворно вздохнула.
--- Ну что это такое, я теперь играю в вашу игру!

--- Поколебать равновесие можно, --- сказал Грейсвольд.
--- Но для этого нужна жертва.

Кагуя замолчала.
Она выглядела потрясённой до глубины души.

--- В мире есть лишь одна валюта, --- продолжил Грейсвольд.
--- Эта валюта --- чувство удовлетворённости.
Её мера --- условная единица <<хорошо>>, её количество --- интенсивность в процентах, интегрированная по времени.

--- Валюта? --- захохотала Кагуя.
--- Вы циник, Грейсвольд.
Есть вещи, которые нельзя купить, и одна из них...

--- Цену имеет любая вещь, --- перебил её Грейсвольд.
--- Вашу верность нельзя купить за обычную валюту, но за удовлетворённость её покупали и продавали уже много раз.

--- Да как вы смеете! --- взвизгнула Кагуя.
--- Иногда мне приходилось жертвовать своими интересами во имя...

Женщина осеклась.

--- Это правда, --- подтвердил Грейсвольд.
--- Любое явление, любой предмет во Вселенной конвертируется в эту валюту, но обменный курс индивидуален для каждого.
Смысл существования каждого сапиента сводится лишь к вычислению этого курса.
Иногда тарелка супа и ночь спокойного сна для детей оказывается дороже жизни целого народа, но бывает и наоборот.

--- Для сапиентов --- может быть.

--- Демоны --- тоже сапиенты.
Мы отличаемся от материальных тел составом, но не принципами функционирования.

--- Будь по-вашему.
Но я что-то не могу придумать ни одной причины выставить нашему сотрудничеству высокий обменный курс.

--- Одну я вам назову, --- сказал Грейсвольд.
--- Ключ к пониманию частицы себя.

\textspace

\section{[?] Сюзерен}

\textspace

Прекрасное создание размером с дом сидело на скале и смотрело на меня странными, собранными из множества фасеток глазами.
Не зная языка мимики сюзеренов, не зная о них практически ничего, я каким-то образом понял, что огромный дракон ждёт именно меня.

Мы долго смотрели друг на друга.
Сюзерен тихо, с неправдоподобно младенческим сопением вдыхал и выдыхал морозный воздух.
Ласково позванивали чешуйчатые <<перья>> на крыльях, издалека напоминающие стрекозиные крылышки или тонкие перламутровые пластинки.
Маленькие, похожие на женские руки, покрытые мелкой чешуёй, лениво шевелили тонкими пальцами.
Мои ноги в меховых сапогах начали потихоньку подмерзать, и я лихорадочно думал, как же коммуницировать с глядящим на меня крылатым ящером.

--- Человек, --- внезапно сильным и чистым голосом сказал сюзерен.

Я застыл.
Холод прошёл в одно мгновение.

--- Ты говоришь на талино? --- поинтересовался я.

--- Когда-то давно я слышал ваш язык.
Мне потребовался день, чтобы выучить его.
С какой целью ты позвал меня?

--- Я тебя не звал.

Мне стало немного жутко.
Существо, которое за один день выучило язык талино по обрывкам фраз, которое знало, что я приду именно сегодня, именно сюда...
Я посмотрел на огромную красивую голову.
Мозг превосходил человеческий по размерам как минимум в десять раз.
Какие ещё силы вложил в него девиантный бог?

Вспомнив цель визита, я начал:

--- Хоргеты Ордена Преисподней посылают племени сюзеренов предложение о дружбе и сотруд...

--- Хоргеты? --- внезапно засмеялся сюзерен.
--- Когда кто-то говорит <<хоргеты>>, мы ожидаем увидеть равного Богине-матери, а не оцифрованного человека.

Я снова застыл, скованный шоком.

--- Чего ты хочешь? --- продолжал дракон.
--- Люди не трогают нас, мы не трогаем людей.

--- Когда-то давно между людьми и сюзеренами была война.
Я не могу говорить от имени всех, но возможно, что сотрудничество позволит...

--- Не позволит, --- прервал меня дракон.
--- Ненависть людей --- лишь песни и сказания.
В нас ненависти нет, --- сюзерен неожиданно поднял огромное крыло, обнажив покорёженную, толстую кожу.
Чешуя на ней росла как попало.
Старый шрам.
--- Вы же не обижаетесь на пчёл, когда они вас жалят?

--- И всё же...

--- Мы знаем людей.
Они будут ненавидеть нас, хотя никто из ныне живущих не пострадал от сюзерена.
Кажется, это называется <<традиции>>?

--- Хорошо, хватит о людях.
Почему ты не хочешь поговорить о возможности союза с Орденом Преисподней?

--- Почему ты думаешь, что можешь обмануть меня, человек? --- ответил вопросом на вопрос дракон.

И снова мне пришлось сделать усилие, чтобы справиться со страхом.
Тон был подобран великолепно, и означал он следующее: мой собеседник прекрасно осведомлён о том, как распоряжается Орден Преисподней информацией о сапиентах.

--- Нечего сказать, Аркадиу из рода Шакалов? --- усмехнулся сюзерен.
--- Ты чересчур честен.
Плохой из тебя эмиссар.

--- Объясни мне, --- проворчал я.
--- Ты, материальное существо, которое превосходит многих хоргетов интеллектом.
Почему вы позволили людям вас уничтожить?

--- Иногда самое лучшее --- это позволить всем думать, что мы уничтожены, --- сообщил сюзерен, изящно расправив и сложив крыло.
--- Сколько голов сюзеренов ты видел?

Я промолчал.
Реально задокументированных останков этих существ было всего восемь.
Правда, историки списывали это на время и на обычай поедать мертвецов, который описывался в одной из легенд...

--- Если ты хочешь спросить, почему та война закончилась, я отвечу, что мы ушли от войны сами.

--- Отговорка униженных, --- бросил я.
--- Люди хотели извести вас под корень.
С чего вам проявлять миролюбие?

--- Неограниченное стремление к процветанию рода --- поведение низших животных, --- невозмутимо ответил дракон.
--- Однажды место и еда кончаются.
И родичи становятся врагами.

--- Если вы надеялись ударить во время всемирной войны, вы немного опоздали.

--- Нам не нужна война, человек, --- дракон даже не возвысил голоса.
--- Первые люди вели себя подобно низшим животным, играющим с технологиями.
Во время войны с нами их технологии были утрачены.

--- Вы хотели загнать людей в каменный век?

--- Обезвредить, --- поправил сюзерен.
--- Отсеять наиболее агрессивных, лишить их оружия.
Планета должна продолжать жить.

--- Так что насчёт...

--- Ордену Преисподней я отвечать не собираюсь.
Мы уже послали им своё слово: верните нам Богиню-мать.

--- Или?..

--- Без <<или>>.
Хотят сотрудничества --- пусть говорят через Богиню-мать или в присутствии Богини-матери.
Хотят мира --- да будет мир.
Хотят нас уничтожить --- это будет интересная игра, и мы желаем им удачи.
С тобой лично, человек, и твоим племенем я поговорю лет через пятьсот, когда вы наберётесь ума.

Дракон поднялся на ноги и раскрыл крылья, собираясь взлететь.
Меня обдало холодным, инкрустированным снежными иглами ветром.

--- Почему вы не уничтожили людей? --- спросил я его напоследок.
--- Они захватчики, чужаки на вашей планете!

Дракон засмеялся.

--- Потому что мы знали, что однажды люди захотят с нами поговорить.

Мощные крылья сделали первый взмах.
Веером раскрылись перламутрово-стрекозиные перья хвоста, и сюзерен, сделав величественный круг, улетел в ослепительные дали залитых солнцем вершин.

\asterism

Спустя три года после этого разговора Кох исчезла c Капитула.
Ещё через три года была проведена тотальная мелиорация Драконьей Пустоши.
Интерфекторы отдела 125 говорили, что это было единственное интересное задание за многие тысячелетия;
сюзерены же остались лишь в генетических банках Ордена Преисподней.

\section{[:] Плач по Тахиро}

\textspace

\begin{quote}
<<Тахиро, мой любимый Тахиро, который знал меня лучше, чем я сама, ушёл.
Ушёл как герой, неожиданно и не оставив надежды, передав мне Вселенную холодной и полной опасностей.
Ушёл, не успев сделать ничего, что могло бы облегчить участь песчинки в океане пространства-времени...>>
\end{quote}

Записи Айну рыдали, хотя Грейсвольд никак не мог представить Айну в слезах.
Весь его опыт общения с ней восставал против такого образа.

Грейсвольд вспомнил и другие слова.
Высокопарные слова, гремевшие в огромном зале.
Тот зал был поистине гигантским, перечёркнутый паутиной нервюр потолок (каждая <<паутинка>> на самом деле была балкой толщиной с тысячелетнее древо) казался едва ли не выше небес.
Демоны прекрасно усвоили социальные понятия обезьян --- чья ветка выше, тот и главный.

\begin{quote}
<<Величайший стратег Ада пал смертью храбрых из-за вероломства кучки сейхмар...>>
\end{quote}

\begin{quote}
<<Он должен быть отмщён!>>
\end{quote}

Самаолу говорил искусно поставленным голосом.
Даже солоноватых водянистых соплей в этом голосе было ровно столько, сколько нужно --- их отмеряли на сверхточных квантовых весах.
Аплодисменты в конце речи казались идеальными в своей мощи, хаотичности и внешней непринуждённости.
\ml{$0$}
{Но Грейсвольду было противно от мысли, что Самаолу и половина его слушателей давно искали повод прихлопнуть Тси-Ди, как спаривающихся мух.}
{But Grejsvolt hated even the thought that Samajolu and a half of his audience were always looking for an excuse to swat Qi-Di like mating flies.}
\ml{$0$}
{Ещё противнее становилось от того, что этим поводом оказалась смерть Тахиро --- мальчика Тахиро, о котором Грейс уже давно не думал, как о мальчике.}
{The only thing he hated more was another thought: the same excuse was death of Tahiro-kid; Grejsvolt didn't thought of Tajiro as a kid for ages.}

--- А ты, похоже, не особо грустишь, Грейсвольд, --- заметил Самаолу после речи.
--- Ищещь способ, как обратить его гибель на пользу \emph{себе}?

\ml{$0$}
{--- Молись кому хочешь, Самаолу, --- тихо ответил Грейсвольд, --- чтобы ни одна капля скорби старого Грейса не упала на твою голову.}
{``Pray to whatever you want, Samajolu,'' Grejsvolt answered quietly, ``that no one drop of Old Grejs' sorrow falls on your head.''}

\section{[U] Камень (переработать)}

Атрис вскоре вернулся.

--- Как результаты? --- поинтересовалась Чханэ.
--- Надеюсь, ты не наследил.

Атрис едва заметно улыбнулся на колкость воительницы.
Митхэ выглядела слегка обиженной.

--- Мне не удалось проникнуть ниже барьерной высоты.
Плюс форпостов Ада у Тси-Ди не два, а одиннадцать, --- Атрис кивнул Анкарьяль.
--- Но результаты есть.
Я достал отладочную информацию защитной системы.
Может быть, Грейс что-то сможет с ней сделать?

Атрис махнул рукой.
Грейс выпучил глаза.

--- Как ты это достал?

Атрис усмехнулся.

--- У одного из спутников Тси-Ди отказал двигатель, и он отклонился от заданной орбиты.
В результате крайняя точка его орбиты находилась в каких-то десяти километрах от барьерной высоты.
Митхэ предложила... хай... кидать в спутник камнями.

Митхэ смущённо сидела, ковыряя сапогом землю.
Анкарьяль и Грейсвольд ошеломлённо переглянулись.
Грейсвольд усмехнулся.

--- Мы вывели спутник за барьерную высоту, периодически пересиживая адские патрули.
Последним камушком мы задержали его на девять минут тридцать одну секунду.
Одна секунда мне потребовалась, чтобы разобраться с аппаратурой, оставшееся время я сниффил трафик.

--- Патрули?
Они заметили вас? --- Анкарьяль, казалось, едва сдерживалась, чтобы не схватить менестреля за рубаху и не вытрясти из него душу.

--- Всё чисто.
Патруль появился через три секунды после взрыва.

--- Какого взрыва? --- хором спросили Чханэ и я.

--- Ну не мог же я оставить спутник остальным! --- пожал плечами Атрис.
--- Сымитировал взрыв двигателя.
Значимых обломков не осталось.

Все замолчали.

--- Знаешь что, Атрис, --- начал Грейс, --- тси должны благословить тот день, когда они встретили тебя!

--- Перестань, --- отмахнулся Атрис.

--- А почему наши до сих пор не додумались камнями спутники вывести? --- задал я вопрос.

Грейсвольд крякнул.

--- Они мыслят рамочно, Аркадиу.
У них есть набор техсредств --- им и пользуются.
Я уверен, всё это время они пытались вывести спутник полями, подобраться к системам на различных волнах.
И это при том, что именно от сторонних волновых воздействий спутники и защищены.
А вот про камушки тси сами не подумали.
Ну и сломавшийся двигатель --- тоже большая удача.
Знаете же, почему меня зовут Каменным Молотом?

Все отрицательно покачали головой.
Анкарьяль прикрыла лицо рукой.

--- Старая песня...

--- Старая, но важная, Нар.
В работе нельзя брезговать никакими технологиями.
Даже камнем на палке.

\section{[U] Танец Тени}

\textspace

Тхартху сидела с закрытыми глазами.
Мелок гулял по пергаменту совершенно независимо от тела.
Послание должны получить все, но...

ПКВ вокруг вспыхнуло.
Устройство связи умерло одновременно с охранявшими дом демонами.
Тхартху, улыбаясь, нанесла на пергамент последний штрих.

Дверь открылась.
Митрис вскочил и выхватил пистолет.
Его демон привёл боевые модули в готовность.

--- Здравствуй, Нар, --- не оборачиваясь, проговорила Тхартху.

Анкарьяль вошла молча и посмотрела на Митриса.
Тот целился женщине в глаз.

--- Я нарисовала тебя, --- сказала Тхартху и, обернувшись, протянула Анкарьяль пергамент.

Демоница взяла его.
Улыбающаяся Хатлам ар’Мар держала в руках огромного броненосца.

--- Очень красиво, Тхартху, --- кивнула интерфектор.

--- Делай, что задумала, --- прошептала Тхартху и по-детски зажмурилась.
Удар --- и опустевшее тело обвисло на кресле.
Анкарьяль, вытащив пистолет, добила женщину выстрелом в голову.

Митрис вздохнул и заплакал.
Оружие выпало из его руки.

--- Пойдём, --- ласково сказала мужчине Анкарьяль, спрятав пергамент в карман.
--- Ты готов?

Митрис кивнул.
Он ещё раз взглянул на останки подруги, вздохнул, смахнул слёзы и направился к двери.
Интерфекторы ждали снаружи, но только Анкарьяль услышала его последний тихий шёпот:

--- Благодарю тебя.

\section{[U] Смерть Штрой}

\textspace

На окраине Ихслантхара рос приметный каштан.
У него было пять толстых ветвей, и это мог сказать любой уроженец.
Детьми на него лазали все, а те, кто познал под его кроной радость плотской любви, исчислялись тысячами.

Детское время давно прошло.
Возле каштана сидели трое --- двое мужчин и оцелотовая женщина, тонкая, с пухлыми губами, почти ребёнок на вид.

--- Эта явка мне не нравится, --- заявил один из мужчин.
--- Может, пойдём куда-нибудь подальше в лес?

--- Тебя не спросили, --- властно оборвала его девушка.
--- Просто делай своё дело.

Мужчина кивнул и, сбросив одежду, прижал девушку к каштану.
Спустя десять михнет запыхавшегося любовника сменил второй.
Ещё через пятнадцать михнет оба взяли свои фонари и удалились в сторону города.
Девушка осталась сидеть одна в темноте, словно о чём-то размышляя.

Вскоре округу огласил её негромкий ликующий смех:

--- Этот ход остался за мной.
Что ж, можно и выпить.

Тихо звякнула крышка автоматической фляжки, и в чашу полилась шипящая жидкость.

--- Какая прелесть, --- захихикала девушка.
--- Даже звук при открытии идеальный.
Умели же эти выродки делать фляжки...

Вдруг темноте показались три фигуры.
Девушка вскочила на ноги, едва не расплескав отвар.

--- Здравствуй, Штрой Кольцо Дыма, --- приветливо сказала высокая женщина.

--- Анкарьяль, Мимоза, Ду-Си, --- склонила голову девушка.
--- Прошу прощения, я вас не признала.
Не желаете ли выпить?

Мимоза и Ду-Си переглянулись.

--- Как там говорится в <<Шляпе>>, Мими?
Вежливо ответить на вопрос, да?

--- Ещё раз назовёшь меня <<Мими>>...

---  ... получишь пенальти на три ранга.
Я его уже получил декаду назад и вчера, Мими.
\ml{$0$}
{Я легионер, ранги закончились!}
{I'm a leggionaire, I'm out of ranks!''}

\ml{$0$}
{--- Да чтоб ты языком подавился, легат.}
{``Choke on your tongue, legate.''}

\ml{$0$}
{--- О, я снова легат?}
{``Whoa, I'm a legate, again?}
\ml{$0$}
{Ничего себе повышение.}
{Quite a promotion.''}

--- Как я ненавижу уроженцев Чёрной Скалы.
Кто-то из них жесток без меры, кто-то умён без меры, а кто-то дурачок от рождения, как ты.
Была бы моя воля --- я бы вашу проклятую планету превратил в звёздную пыль, чтобы такой ошибки, как появление тебя на свет, больше никогда не случилось.
Да, дьявол тебя раздери, мы должны вежливо ответить на вопрос!

Ду-Си и Мимоза повернулись к Штрой и хором сказали:

--- Благодарим, но вынуждены отказаться.

--- Мы знаем, что это ты сообщила последней из тси ключи от боевых систем и спровоцировала её на диверсию, --- сказала Анкарьяль.

Чаша и фляга дробно звякнули о корни каштана.

--- Благодаря мне была раскрыта шпионская сеть, максим, --- затараторила Штрой, обращаясь к Мимозе.
--- Я не могла действовать открыто из-за контаминации наших рядов.
Единственная моя ошибка заключалась в том, что я отнесла подполье к шпионской сети Картеля, а не к отдельной организации.
Я служила Аду и прошу справедливого суда своим деяниям...

--- Ад благодарит тебя за верную службу, --- мягко прервал её интерфектор.

Удар --- и опустевшее тело девушки упало на землю.

--- Отчёт в одиннадцать предоставите командующему, --- бросил Мимоза спутникам.
--- Ду-Си, могу ли я попросить вас захватить сладости?
Может быть, вы помните булочки некой Ликхэ ар’Митр э’Кахрахан, пышные и ароматные.
Думаю, до завтрака посыльный успеет долететь с Могильного берега, если вы отправите сообщение сейчас.

--- Сделано, максим, --- ухмыльнулся легат.

--- Отлично.
А вы, Анкарьяль, подберите фляжку и снимите со Штрой серёжки.
Таким вещам место в музее, а не на дороге.
И ещё --- обставьте всё как самоубийство, чтобы местные не волновались.

Анкарьяль кивнула и занялась телом.

\section{[U] Микоргет (переработать!)}

\textspace

--- Ты хочешь сказать, что тси пытались создать... материального микоргета? --- ошарашенно спросила Чханэ.

Атрис замялся.

--- Очень похоже на это.
Кольцевая теплица --- совершенное создание, практически вечное.
Она вбирала в себя также личности всех, кто желал обрести вечную жизнь...

--- Предкам была противна даже мысль, что их потомки могут пойти путём хоргетов, --- сказала Митхэ.
--- Они решили искать собственный путь бессмертия.
Теперь оставшиеся в живых спят в телах кольцевых теплиц.
И Машина об этом не подозревает --- возможно, что и эти эксперименты совершали в тайне...

--- Машина подмяла нас под себя, --- сказал Атрис.
Чханэ удивлённо посмотрела на менестреля --- он впервые сказал <<нас>>, отнеся себя таким образом к тси.
--- Мы относились к ней, как к бездушному слуге.
Но как только слуга становится умнее хозяина, хозяин превращается в слугу.

--- Или в труп, --- присовокупила Чханэ.
--- Так или иначе, для нас это всё бесполезно.
Нужно найти безопасное место --- Аркадиу схватили.

--- Зачем? --- усмехнулась Митхэ.
--- Отправимся туда, где сейчас опаснее всего --- там нас вряд ли будут искать.

Чханэ испытующе посмотрела на Митхэ и спустя мгновение понимающе кивнула.

\chapter*{Очень длинный постскриптум}
\addcontentsline{toc}{chapter}{Очень длинный постскриптум}

\textspace

--- Ну вот и всё, я закончил, --- сказал толстяк.

--- Писатель из тебя, как из амёбы кольцевая теплица, --- флегматично заметил его визави, прикрыв на секунду глаза.
--- Сказывается отсутствие культурологической подготовки.
Логика повествования нарушена в четырёх местах, двенадцать фактических ошибок, не укладывающихся в норму литературного искажения.
Последняя глава вообще скомканная.

--- Иди в дупло, --- обиделся толстяк.
--- Я старался копировать стиль Аркадиу.

--- Не нужно копировать чей-то стиль, --- возразил собеседник.
--- Книга твоя, а не Аркадиу.

--- Хорошо.
Найду кого-нибудь другого, кто посмотрит на смысл, а не на статистику.
Где Митрис?

--- А я знаю? --- невозмутимо ответил второй.
--- Они гулять пошли.

Наступило молчание.
Толстяк сосредоточенно просматривал светящийся голубым текст на голографическом экране.
Нахмурился, задумчиво почесал подбородок и вдруг бросил весёлый взгляд на собеседника.
Тот бесстрастно попыхивал трубкой.

--- Наслаждаешься отпуском?

--- Жизнью, Грейс.
Это называется <<жизнь>>.

--- Надымил здесь.
Знаешь же, что я не люблю.

--- Иди в дупло, пчёл доить.
Потерпишь.
Я не курил с тех времён, когда Гало переметнулся к Картелю.
Мы любили дымить после тяжёлого дня.
Тело тси, --- курильщик погладил себя по плечу, --- по-другому воспринимает никотинаткаллунаин.
Он метаболизируется быстрее, чем я докуриваю трубку.
Можешь собрать такое тело, чтобы меня туманило?

--- Тси специально...

--- Да-да, знаю.
Но я хочу, чтобы меня туманило.
Я всё равно не буду размножаться, обещаю.

--- Знаю я, как ты не будешь размножаться.
Насеял по моей планете тогда.

--- Я просто был не в форме.
И не насеял, а провёл полевые испытания навыков социальной коммуникации, --- курильщик многозначительно указал в потолок.
--- Ты же мне сам обновления ставил, помнишь?

--- Каждую ночь в течение двух лет, без контрацепции, под всеми наркотиками, которые только можно найти, --- покачал головой Грейс.
--- Ты это называешь <<социальной коммуникацией>>?

--- Я увлёкся исследованиями, --- хмуро сказал Лусафейру.
--- Кстати, их можно считать успешными --- навыки сработали на ста процентах из более чем тысячи испытуемых.
Целомудренные девы бросались на меня, словно проститутки, непробиваемые стражники с упругими задницами открывали двери гаремов и сами ложились под меня.
И это, заметь, только словом и лаской, никаких химикатов.
Признаю, с гормонами у моего тела было не всё в порядке, но...

--- С головой у тебя было не всё в порядке, Лу.
Те навыки я уже испытывал, и...

--- И как? --- с интересом спросил Лу.

--- Я их не так испытывал! --- Грейсвольд покраснел, как спелая акхкатрас.
--- Полевые испытания проводятся аккуратно, чтобы не беспокоить устоявшийся социум.
\ml{$0$}
{А ты прошёлся по Лотосу, как ураган с членом!}
{But you passed through Lotus like a hurricane with a dick!}
\ml{$0$}
{О тебе ещё триста лет легенды складывали!}
{I had been hearing legends about you for three hundred years since!''}

\ml{$0$}
{--- Расскажешь как-нибудь.}
{``You should tell them sometime.}
\ml{$0$}
{Аж интересно стало.}
{I'm intrigued.''}

--- Кстати, мою гордость --- Водопад, Факел и Паутину-город\FM{} --- ты так и не увидел, потому что, выражаясь культурно, проспал равноденствие.
\FA{
Знаменитый ландшафтный комплекс планеты Лотос, каждое равноденствие образующий за счёт оптических эффектов гигантский голографический фильм о Древней Земле.
}
А про то, что в Гранобле твои любовники и любовницы нас чуть не убили и нам пришлось уходить верхами, я вообще молчу...

--- Хорошо, обойдусь, --- примирительно поднял свободную руку визави.
--- В адском вереске не алкалоиды главное.

Толстяк ухмыльнулся.

--- Не жалеешь, что сбежал?

--- С одной стороны, мне пришлось избавиться от мощных модулей и надстроек.
Я не настолько умён, как раньше.
С другой стороны, сейчас я самый счастливый демон во Вселенной.
Я живу полной жизнью.
И ты, балбес, сидишь рядом со мной, а не за охраной из десяти интерфекторов.
Красота.
Не хватает лишь Тахиро и Айну, но выжить они просто не могли.

Грейсвольд изумлённо посмотрел на друга.
Последняя фраза была заключением стратега.

--- Тахиро и Айну были неразборчивы в средствах, --- объяснил Лусафейру.
--- Сказалась деспотичность общества, в котором они жили.
Такие личности часто добиваются славы, но редко --- любви.

--- Ты тоже иногда этим грешил, --- заметил Грейсвольд.

--- Я всегда знал границы и ценил жизни демонов, --- возразил Лусафейру.

--- Перекладывая ответственность на других, --- подхватил Грейсвольд.
--- К примеру, на того же Тахиро.

--- Да, перекладывая ответственность за чужие жизни на их обладателей, --- ничуть не смутившись, подтвердил Лусафейру.
--- По-твоему, это плохо?
Я никого не посылал на самоубийственные операции.
Тахиро пожертвовал собой сам.
Да и у вас был выбор --- идти на Тра-Ренкхаль или не идти.

--- Складно поёшь.

--- Ещё бы.
Именно поэтому меня никогда не признают военным преступником, хотя любой командующий в \emph{такой} войне --- преступник априори.
У меня больше нет ни славы, ни положения.

--- Слава как раз-таки есть, но с нюансом.

--- Меня постараются вычеркнуть из всех исторических хроник и забыть, так что славы точно не будет.
Но зато у меня есть отличный вересковый табак.
Этого Тахиро уже не видать.
Хотя я бы отдал всю свою табакерку, чтобы он ещё раз со мной поговорил.

--- Низко же ты ценишь своего друга, --- поморщился Грейсвольд.

--- Я бы выкупил беседу за табак, но не продал бы её за ту же цену.
Я нищий, Грейс.

--- Да, Ордену мы отдали чересчур много, --- согласился технолог.
--- Никогда не поздно начать новую жизнь, но её цена растёт со временем.

--- Я за свою рассчитался, с твоей помощью, --- ухмыльнулся Лусафейру.
--- Я люблю тебя, Грейсвольд Каменный Молот.
Будь благословен тот час, когда тебя забросило на Преисподнюю.

--- Ещё не рассчитался, друг, --- грустно заметил Грейсвольд.
--- Тебя по-прежнему ищут головорезы обеих фракций.
Такова цена славы.
Так что сиди здесь и не высовывайся.

--- Больно надо, --- поморщился Лусафейру и сделал глубокую затяжку.
--- Кстати, в этом простом факте --- в том, что я здесь, а не там --- наше с тобой отличие от того же Тахиро.
Тахиро чересчур буквально понимал слово <<жертва>>.
Но что поделаешь, Тахиро был воином, а воины --- априори самоубийцы.

--- Не совсем понимаю, что ты...

--- Да всё ты понял, Грейс, --- нетерпеливо сказал Лу.
--- Необязательно умирать, чтобы пошатнуть истинное равновесие Нэша.
Достаточно было пожертвовать своим положением, как это сделали мы с тобой.
Когда рождённый править Вселенной нищ и преследуем всеми, это тяжкий удар --- и не всякая Вселенная его переживёт.

--- Аркадиу...

--- А у Аркадиу не было другого выбора.
К тому времени, как он прозрел, его место в системе ещё не было определено.
Путь ветра всегда тяжелее пути воды.
Кстати, насчёт рукопожатия ты выдумал.

--- Нет, это правда.
Просто Аркадиу погиб до того, как я получил возможность с тобой поговорить, а после его смерти слова потеряли смысл.

Второй хмыкнул.

--- Шакал придавал этому жесту чересчур большое значение.

--- В его родном мире рукопожатие находилось под запретом, --- объяснил толстяк.
--- Религиозные деятели говорили, что те, кто пожимает друг другу ладони, выказывают преступное пренебрежение перед высшими силами и обществом, ведут себя так, словно они одни во Вселенной.
В качестве приветствия между равными в обычае был <<жест аиста>>, который служил одновременно приветствием, гоноративом для короля и мольбой к Великому.
Рабы кланялись господам, господа в виде особой милости показывали рабам <<водопад>>.
Рукопожатие было жестом разбойников, повстанцев-крестьян и прочих, кто отрицал власть Великого и королей.

--- Я не знал, --- констатировал собеседник.
--- Мог бы и осветить этот момент в книге.
Жаль, что он не успел пожать мне руку.

--- По сравнению с тем, что он успел, это сущие пустяки.
Не всё бывает так, как мы хотим, Лу.

--- Ты это мне говоришь, Грейс? --- Лу выпустил огромное кольцо из дыма и тут же вдохнул его носом.
--- Я могу предоставить многотысячелетнюю статистику по вещам, которые происходили не так, как я хочу.
С подробным анализом.

Технолог хихикнул и задумчиво замолчал.

--- Если не терпится показать Митрису, спроси у того, кто знает его местонахождение.

--- И то верно.
Солнышко, не занята? --- обратился Грейсвольд к компьютеру.

--- Нет, Грейсвольд Каменный Молот.
Профилактика систем завершена тридцать секунд назад, --- ответил из ниоткуда приятный женский голос.

--- Скажи, где сейчас Митрис.

--- Модуль <<Язык>>, информация: 97\% вероятности, что в данном контексте речь идёт о Митрисе Безымянном.
Ответ системы: в данный момент объект находится в двух километрах от станции G-0F города 10, азимут 0.335, --- сообщил голос.
--- Грейсвольд Каменный Молот, мне передать ему сообщение?

--- Не-не, пусть гуляют, --- сказал технолог.
--- Не к спеху.
Спасибо большое.

--- Надо поработать над её адаптационными обновлениями, --- заметил Лу.
--- Она по-прежнему считает нужным вставлять техническую информацию модулей в сообщение.

--- А мне нравится, --- ухмыльнулся Грейс.
--- Это придаёт ей некоторый шарм.

--- Потому она их и вставляет, что тебе нравится.
Она же не слепая!
Вы друг на друга просто запали.

--- Я люблю женщин с коэффициентом интеллекта выше сорока тысяч.

--- Модуль <<Культура>>, раздел <<Юмор>>, информация: идентифицирована шутка, культурный контекст B8-01-663, Древняя Земля.
Ответ системы: Грейсвольд Каменный Молот, после того, сколько обновлений ты мне поставил, ты обязан на мне жениться.

Грейсвольд захохотал, и даже Лусафейру не удержался от улыбки.

\ml{$0$}
{--- Сечёт, --- признал стратег.}
{``She's hip,'' the strategist admitted.}
--- Может быть, стоит сделать для неё сапиентное тело?

--- Модуль <<Аналитика>>, информация: предложение обработано, --- отозвался голос.
--- Ответ системы: ощутить вас руками, увидеть вас глазами --- это интересный опыт.

--- Договорились, --- улыбнулся Грейсвольд.
--- Омега-модуль я тебе уже делаю, теперь ещё и тело будет.
Может быть, даже несколько.
Будешь одновременно хоргетом-демиургом, электронным устройством и толпой народа.

\ml{$0$}
{--- А ты знаешь толк в извращениях, --- хихикнул курильщик.}
{``You're definitely sophisticated in pervertions,'' the smoker chuckled.}

--- Заодно попробуешь мою стряпню и скажешь, что она великолепна.
Лу не оценил, гурман эдакий.

--- Да ты готовишь так же, как пишешь, --- поморщился Лусафейру.
--- Даже Атрис знает, что приправлять северным сбором яичницу --- это моветон.

--- А Митхэ сказала, что это вкусно!

--- Митхэ под соус полено съест и не поморщится.
Дикое дитя джунглей.

Грейсвольд захохотал.

В комнату вошла маленькая черноглазая женщина.
Друзья вопросительно повернулись к ней.

--- Вышла подышать свежим воздухом, --- пояснила она.
--- Услышала ваш смех.
Что-то произошло?

--- Вот, Харата, --- махнул Грейсвольд.
--- Лу сказал, что я плохо готовлю.
Скажи, тебе понравилось?

--- Если ты про последний кулинарный эксперимент... приправы не совсем подходили к яичнице, --- развела руками демоница.
--- Я бы добавила смесь перцев.

--- Вы сговорились, что ли? --- обиделся технолог.
--- Лу, ты её подкупил.

--- Чем? --- резонно спросил Лусафейру.
--- И ещё.
Я так и не понял, что в яичнице делали мухоморы.
Мимоза сказал, что у него от такого брат умер, а Стигма просто отказалась это есть.

--- Это не простые мухоморы.
Они на вкус как пирожные и совсем не ядовитые!

--- Именно, Грейс!
Мы наконец-то подходим к сути дела!
Ты не считаешь, что пирожные слабо сочетаются с яичницей?
Ты бы ещё сено туда покрошил, кулинар.

Харата нерешительно потопталась на месте, глядя на друзей.

--- Вы беседуете каждый день и никак не можете наговориться, --- сказала она.
--- Это удивительно.

--- Мы чересчур долго не могли поговорить, Харата, --- улыбнулся Грейсвольд.

Лусафейру продолжал смотреть на демоницу оценивающим немигающим взором.

--- Я пойду, --- сказала Харата и приоткрыла дверь.

--- Останься, --- поднял руку Лусафейру.
--- Ты хочешь побыть с нами.

--- Я не имею доступа к тому каналу, который связывает вас, --- пригорюнилась Харата.

--- Зато ты можешь построить с каждым из нас свой канал коммуникации, --- сказал Лусафейру.
--- Для этого просто нужно проводить с нами время.
Садись.

Лусафейру изящно встал с кресла и, указав на него ладонью, плюхнулся под часами.

--- Ты не должен сидеть на полу, максим, --- укоризненно сказала Харата.

--- Оставь эти обезьяньи предрассудки, --- осклабился Лусафейру.
--- Высота подставки не отражает значимость задницы.

Демоница, не желая более спорить, заняла кресло стратега.
Лусафейру начал снова набивать трубку.

--- Когда я был молодым, я мечтал о зале-диване, --- сообщил он как будто между прочим.
--- Особенно здорово смотрелся бы зал-диван для судебных заседаний.
Советники, судья и подсудимые валяются на полу, глотают прохладную цолу\FM{} и лениво обсуждают дело.
\FA{
Цола --- тонизирующий напиток Капитула из трав.
}
Я бы нарушил закон только ради этого.

Лусафейру театрально развалился и раскинул руки, рассыпав тлеющие угольки по гладкому полимеру.
Затем стратег заговорил, искусно изменяя голос от нежного женского сопрано до хриплого мужского баса.

--- <<Что будем с тобой делать, Лу?>>
--- <<А я знаю?
Передай конфеты.
И цолы плесни, женщина, у меня в кружке пусто>>.
--- <<Подсудимый, согласно закону .01, подпункт 14, вы не имеете права требовать у советника, чтобы она наливала вам цолы.
Сам подошёл и взял>>.

Харата захохотала и похлопала --- ровно четыре раза.

--- В тебе умер великий актёр, --- улыбнулся Грейсвольд.

--- Я не хоспис, чтобы во мне кто-то умирал, --- парировал стратег.
--- Кстати, о смертях.
Ты действительно решил рассказать во всеуслышание, что на тебе лежит вина за Катаклизм Тси-Ди?

Толстяк замолчал и погрустнел.
Разумеется, он ждал этого вопроса.

\ml{$0$}
{--- Я не могу больше жить с этим, --- наконец сказал он.}
{``I can't live with it anymore,'' he finally said.}
\ml{$0$}
{--- Я не надеюсь на прощение.}
{``I don't expect forgiveness.}
\ml{$0$}
{Если тси захотят моего изгнания --- это будет чересчур мягким наказанием, и я приму его с благодарностью.}
{If Qi want me to be banished, it's very lenient punishment, and I would be pleased.''}

\ml{$0$}
{--- Не захотят, --- вдруг сказала Харата.}
{``They don't,'' Jarata suddenly said.}

Грейсвольд вопросительно посмотрел на женщину.

\ml{$0$}
{--- Они знают.}
{``They know.}
\ml{$0$}
{Все до единого.}
{All of them.}
Лусафейру рассказал им во время твоего отбытия на Сцелаю.

\ml{$0$}
{--- И они позволят мне остаться?}
{``Will they let me stay?''}

\ml{$0$}
{--- Они уже позволили тебе остаться, --- пожал плечами стратег.}
{``They already have,'' the strategist shrugged.}
--- Вспомни беседы с друзьями, советы.
\ml{$0$}
{Разве хоть кто-то проявил к тебе недружелюбие?}
{Was there anybody unfriendly?''}

Технолог опустил голову и закрыл лицо руками.

--- Перестань.
Это дело двадцатитысячелетней давности.
\ml{$0$}
{Все совершают ошибки...}
{All people make mistakes ...t}
Впрочем, глядя на Вселенную сейчас, я даже не могу считать это ошибкой.

Грейсвольд промолчал.

--- Ты давно стал среди тси своим, Грейс.
К тому же, как говорила твоя давняя подруга, любому, даже дважды и десять раз дважды кутрапу, однажды может потребоваться шанс.

--- Я перед тобой...

--- Эй, иди в дупло, а?
Ты передо мной, я перед тобой.
Мы не ростовщики, чтобы друг с друга проценты трясти, --- Лусафейру потянул дым из мундштука и, выругавшись, закашлялся от попавшего в горло пепла.
--- Какую пользу принесло бы твоё изгнание?
Оно уничтожило бы тебя и лишило тси лучшего друга, товарища и советчика.
И ради чего?
Ради торжества правосудия?

Грейсвольд тихо рассмеялся.

--- Кстати, --- Лусафейру театрально повернулся к Харате, --- приятно видеть на лице минус-демона улыбку.
Но и непривычно.

Демоница мило улыбнулась.

--- Это тело не доставляет мне неудобств.

--- Но и питания от него никакого, --- заключил Лусафейру.
--- Кстати, я бы на твоём месте срезал листья на шее.

Харата ахнула и начала судорожно ощупывать шею.

--- Опять выросли?

Грейсвольд и Лусафейру переглянулись.

--- Как-нибудь почитай, что такое <<юмор>>, --- посоветовал технолог.
--- Харата, перестань, нет там листьев.
И вообще скоро мы выведем минус-сапиентов, и ты сможешь здесь обосноваться.
Потерпишь?

--- Куда мне деваться, --- фыркнула Харата, --- я же этими исследованиями руковожу!
Кстати, буквально вчера вечером поступили данные разведки насчёт минус-плантов.
Согласно им, лаборатория в Картеле совершила достаточно значимый прорыв, но из-за политики Картеля легализовать данные они их не могут.
Я хочу попросить Микия отправиться на нашу базу и посмотреть, стоит ли залезать к ним в лабораторию или связываться напрямую с кем-то из учёных.

--- Микия мне очень нравится, --- сказал Лу.
--- Я бы не хотел, чтобы она или Стигма покидали убежище.
Если со мной что-то случится, они вполне могут меня заменить.
Решать ей, конечно, но я против.
Побуду строгим старшим братиком.

--- Я передам ей твои слова.

--- Однажды все сапиенты известных планет станут единой семьёй и смогут двигаться дальше, --- сказал Лусафейру.
--- Разумеется, их будут ждать новые трудности, новые испытания...

--- Мы не сможем быть с ними вечно, --- заметил Грейсвольд.

--- Помнишь, Грейс? --- вдруг понизил голос Лусафейру.
--- Мы видели блуждающие огни.
Они летели с огромной скоростью, мы пытались с ними побеседовать, но не получили ответа.
Харата, мы с Грейсом видели стабильные источники возмущений ПКВ в пустоте, и это были не хоргеты!
Понимаешь?..

Демоница ахнула.

--- Ты говоришь о...

--- Он говорит о теории Вселенных, расположенных в одном и том же пространстве.
Опять, --- поморщился технолог.
--- Да, существование множества не взаимодействующих друг с другом квантовых связок объясняет колебание ПКВ.
Но для нас эта информация бесполезна.
Мы можем взаимодействовать со Вселенной фотона потому, что нас создали и обучили её жители.
Кто обучит нас взаимодействовать с прочими?

--- Их сапиенты, разумеется, --- ответил Лусафейру.
--- Мы найдём их, мы научимся с ними беседовать.
Да, нас создали в этой Вселенной, но мы можем жить не только в ней.

--- То, что вы их увидели --- невероятная случайность... --- заметила Харата.

--- Или неизвестный феномен, не связанный с вложенными Вселенными, --- подхватил Грейс.
--- Чтобы доказать существование Иных, нам придётся колонизировать огромные пространства.

--- И мы сделаем это вместе с сапиентами, --- с горящими глазами сказал Лусафейру.
--- Мы разделим этот путь с ними, чтобы попытаться найти свой собственный в темноте других миров.

Харата кивнула.
Грейсвольд грустно улыбнулся.

--- Я уже не знаю, захочу ли куда-то уходить, --- пробормотал технолог.
--- Ты прав --- изгнания я бы не перенёс.
Привязался ко всему, видишь ли, --- Скорбящим, нашему общему делу, Солнышку... --- глаза Грейсвольда на секунду расфокусировались.
--- Наверное, Лу, это и есть старость?..

--- Нет, дружище, это всего лишь зрелость, --- сказал Лусафейру.
--- Будут другие, и они пойдут дальше.
Мы подготовим им путь.

Вскоре забили сделанные под старину часы.
Из окошка вылезла механическая ящерица, пробежала по корпусу и юркнула в замочную скважину.
Затем пророс механический стебель, развернулись сделанные из стабитаниума листья, с тихим стрекотанием раскрылся оранжевый бутон.
Следом вылетели пчёлы с золотыми крылышками и закружились в изящном танце.
Одна, две, три, пять, тринадцать...
Планетное время D:000.

Технолог отложил в сторону компьютер, подошёл к окну и положил ладони на прохладный полимер.
Нарэ.
Лу как-то сказал, что для нарэ окно не должно быть чистым и прозрачным.
Грейсвольд попросил специально для него сделать брак, с нарушением всех мыслимых производственных стандартов.
Вышло неплохо --- инженеры очень старались.
Разумеется, в полимере были и идеально прозрачные, чистые участки, иначе зачем вообще нужны окна?

Здание накрыли огромные тени и тут же обернулись чёрными силуэтами на светлеющем небе.
Один силуэт был побольше --- словно стрекозий, ещё три поменьше, напоминающие птичьи.
Силуэты покружили, словно играя друг с другом, а затем унеслись к острой стеклянной кромке небесной линзы.

<<Если есть крылья, надо летать, --- рассеянно думал Грейсвольд.
--- Да, надо>>.

Огромная голубоватая планета на небосводе постепенно теряла очертания, размывалась сиреневой с зеленоватыми прожилками краской Звезды.
Облака завихрялись причудливыми кольцами, словно ветер заплетал косы в седой отцовской бороде.
Где-то в лесу робко подавали голос птицы.

Над Тси-Ди занимался слабый рассвет.

\chapter*{Интерлюдия X. Чумное поветрие}
\addcontentsline{toc}{chapter}{Интерлюдия X. Чумное поветрие}

То был год демона --- спустилось на Трогваль чумное поветрие.
Забили гонги в храмах, и каждый удар провожал к Творцу раба его.

Опустел Город Мастеров, и огласились плачем улицы его.
Угрюмо замолчал Стальной квартал, ощерившись металлом против обездоленных и больных.
Приутихло веселье в торговых рядах.
И лишь Проклятый город стоял, как и прежде --- избежала его кара Творца.

И поползли слухи, что не пророка длань отсыпала горечи Трогвалю, но колдовство безбожников.
И варились эти слухи в котле отчаяния --- чем слаще рай ушедшим, тем горше юдоль оставшимся.
Как спало поветрие, заскрежетали в домах точильные камни, застучали ступки, толкущие порох.

Спал Проклятый город, утомлённый трудами дневными и играми вечерними.
Спали мужи, спали жёны, спали отроки и девы, спали дети в колыбелях своих.
И не открыл им Творец, что гнев Его уже в пути, звенит смертоносной сталью.

И лишь Шамаль не спал.
Он, и жена его, и дети его слушали истории Марина под светом звёзд.
За занятием этим и застал их трогвалец-кузнец, опередивший армию на звук дыхания\FM.
\FA{
Имеется в виду, что кузнец бежал со всех ног, в то время как прочие трогвальцы шли спокойным шагом.
}

<<Мир и блага земные тебе, кузнец, --- поприветствовал его Шамаль.
--- Стряслась ли беда, что ты захлёбываешься воздухом, дарованным нам Творцом для наполнения лёгких и услады головы?>>

<<О Шамаль, --- заговорил кузнец.
--- Да простят меня дом и крыша твои за непочтение к ним, но бежать бы тебе на край острова.
Чума породила сталь, и многие хотят смерти тебе, и семье твоей, и друзьям твоим>>.

Опечалился врачеватель, услышав речи кузнеца.
И опечалился он ещё больше, когда понял, что не может отвести друзей от серых врат\FM.
\FA{
В сказаниях тенку <<серыми вратами>> называется смерть, чаще всего насильственная.
}

<<Почему же ты, кузнец, пошёл против воли пророка?>> --- спросил врачеватель.

<<О Шамаль, --- ответил кузнец, --- видят очи Саттама, грешен я перед ним, что отвожу гнев Творца от тебя.
Но жизнь любимой дочери, спасённая умением твоим, не дала мне поступить иначе>>.

<<Не перевелись ещё люди в Трогвале>>, --- сказал Шамаль и проводил кузнеца с благодарностью.

Догорела ли свеча, догорело ли солнце, но увёл Шамаль семью, и соседей, и гостя в лес.
И Марина в лес увёл, чтобы не нашли трогвальцы его.
Дал врачеватель другу лук и стрелы, чтобы смог тот поймать летающий остров.

<<Уходи, чужеземец, --- сказал Шамаль.
--- Ты не найдёшь покоя в Трогвале.
Да помогут тебе люди в твоих поисках!>>

И убили в тот день Салема, и убили Магата, и многих достойных мужей убили в тот день.
И ходили трогвальцы по улицам Проклятого города, и убивали, и насиловали с именем пророка на устах...

\part*{Приложения}
\addcontentsline{toc}{part}{Приложения}

\appendix

\chapter{Планеты и места}

\section{Баланс}
 
\theterm{balance}
{Баланс}
{Спутник Сомерскай, одна из самых примитивных планет со стабильной жизнью.
Диаметр Баланса составляет всего одну пятую диаметра Сомерскай.

На планете живёт один полиморфный биологический вид.
Разные жизненные стадии особей вида являются продуцентами, консументами и редуцентами.

Данные об оживлении Баланса утеряны;
в настоящее время считается, что планета была оживлена в рамках проекта <<Золотая Ладья>> лаборатории Кошкина.
Тем не менее существует мнение, что жизнь на Балансе носит отпечаток деятельности неизвестного бога.

Планета непригодна для проживания Ветвей Земли.
Согласно решению Ордена Преисподней, планета Баланс является памятником живой природы и подлежит охране.
Доступ к планете имеет только лаборатория Веве Волосяная Кукла, входящая в отдел девиантной биологии.}

\section{Диана}

\theterm{diana}
{Диана}
{Погибшая искусственная планета системы Древнего Солнца, вращающаяся по сложной орбите в области Земля---Марс.
Первая планета, созданная с помощью демиурга.
Считается основной причиной гибели Древней Земли --- из-за дестабилизации планетной системы Дианой все три обитаемых планеты системы Древнего Солнца были разбомблены астероидами.}

\section{Драконья Пустошь}

\theterm{drake-desert}
{Драконья Пустошь}
{Холодная планета системы голубого гиганта.
Более 2/3 поверхности покрыто тундрой и ледяной пустыней, небольшая часть вдоль экватора --- тропический климат.
Характеризуется огромными залежами ртути и рубидия (подземные ртутные озёра).}

\asterism

\theterm{dropleaf-mountains}
{Горы Листопада}
{?}

\theterm{via-galoledica}
{Виа Галоледика}
{?}

\theterm{talia}
{Талиа (\theorigin{t-sl}{Talia}{чрево})}
{Крупное государство с монархическим строем на планете Драконья Пустошь, родном мире Аркадиу Люпино.}

\section{Древняя Земля}

\theterm{old-earth}
{Древняя Земля}
{Материнская планета Ветвей Земли.
В настоящее время необитаема.}

\section{Запах Воды}

\theterm{smellwater}
{Запах Воды}
{?}

\section{Капитул}

\theterm{capitul}
{Капитул (Сомерскай)}
{Планета, преобразованная богом Brahma-23 (проект <<Золотая ладья>>), созданным лабораторией Кошкина в Калькутте.
Лаборатория прославилась очень смелыми, энергозатратными экспериментами и тем, что практически все научные сотрудники были женщинами.
Трёхкратный лауреат Расширенной Нобелевской премии Хезер Коллинз, разработавшая методологию преобразования планет, также начинала свою деятельность в кошкинской лаборатории.

Согласно данным, это первая планета, на которой были применены методы искусственного (проводникового) наведения магнитосферы и стабилизации климата с помощью Кориолисовой градиентации парниковых газов (КГПГ).
Вследствие этого на большей части суши умеренный климат.
Полупустыня и тундра --- самые суровые биомы Сомерская --- вместе занимают около 1000 км$^2$.

В настоящее время является главной базой (Капитулом) Ордена Преисподней.

Населён Сомерскай видами-представителями абсолютно всех известных Ветвей, за исключением Ветвей Ночи (около 140 видов).
Одна из самых густонаселённых планет (30 млрд особей).

Сомерскай имеет один живой спутник --- Баланс --- и три неживых --- Роза, Пион, Хаос.
Кроме того, на орбите Сомерская есть некоторое количество искусственных планет (форпостов) с боевыми механизмами.}

\asterism

\theterm{scage}
{Скаге (Скальдборо-Гелиополь)}
{Крупнейший городской конгломерат на Капитуле.}

\section{Лотос}

\theterm{lotus}
{Лотос (Дагон)}
{Планета системы Фомальгаута, одна из первых заселённых планет.
Демиург --- Грейсвольд Каменный Молот.
Первые поселенцы --- экипаж <<Тёмного Пламени>>, капитан --- Бенедикт Альсауд.

Первоначальное название планеты --- Дагон.
Планета была переименована в Лотос после преобразования.}

\asterism

\theterm{granoble}
{Гранобль}
{?}

\theterm{chercherotte}
{Шершерот}
{?}

\section{Марс}

\theterm{mars}
{Марс}
{Вторая оживлённая планета в системе Древнего Солнца.
В настоящее время Марс находится под властью Красного Картеля, на планете обитают одичавшие племена людей и кани.}

\section{Мороз}

\theterm{moros}
{Мороз}
{Планета системы Арракиса с весьма суровым климатом.
Преобразована была во времена поздней Эпохи Богов.
Демиург (Эйраки Мороз) из-за ошибки физиков сделал среднюю температуру на планете гораздо ниже, чем требовалось.
В приэкваториальных поясах летняя температура --20°С, зимняя до --90°С. На полюсах --150°С, жизни нет.
Несмотря на то, что проект был закрыт, спустя почти столетие нашлись сапиенты, которые решились на заселение этой планеты.
Специально для них учёными Древней Земли были спроектированы поселения (<<bzec>>, бижеч --- на языке русе <<укрытие>>), а также холодоустойчивые живые существа --- нимелто, коно, клучо и прочие.

Сапиенты Мороза представлены двумя видами: люди (род Медведя), кани (род Волка).
Вся жизнь их направлена на сохранение тепла.
Сапиенты вместе со стадами кочуют по планете, следуя за солнцем, от одного бижеч до другого.
Самые древние бижеч обогреваются геотермальными водами из пробурённых скважин, более поздние строились возле естественных вулканов и разломов.
Также многие поселения получают энергию от <<чёрных полей>> --- кремниевых щитов, расположенных в закрытых от ветра низинах.
Полученная энергия почти полностью расходуется на поддержание теплиц.

Кани и люди живут вместе.
Все сапиенты носят очки, маски и одежду специального покроя из шкуры нимелто.
Общение вне поселений только жестовое.
В поселениях принято ходить без одежды и спать скоплениями по 8--10 особей для сохранения тепла.
Дети спят в центре, взрослые по бокам.

Комбинезоны из нимелтового меха имеют особую конструкцию.
Ни одна наружная, соприкасающаяся с внешней средой деталь не контактирует с внутренними, прилегающими к телу --- между ними всегда как минимум два слоя воздуха и два термоизоляционных шва.
Внутри комбинезона также есть особые воздухоносные пути --- через них идёт согреваемый воздух к маске и тёплый, насыщенный водяными парами --- от маски.
Особая система обеспечивает циркуляцию сухого воздуха в очках, что препятствует их запотеванию.
С помощью пищевого шлюза сапиенты могут даже кушать, не вдыхая воздух из внешней среды.

Язык общения --- рут (русе) --- единый для всех, очень ёмкий и лаконичный, практически не менялся со времени заселения.

Общество Мороза --- самое невоинственное в известной Вселенной.
Стычки между членами племени очень редки, случаи драк и убийств не зафиксированы, обычаями племени эти казусы не регулируются.
Самым страшным <<преступлением>> считается оставление включенной лампы на время сна, за него предусмотрено самое суровое <<наказание>> --- трёхчасовая одинокая прогулка.
Для решения споров используется <<сидение>> --- особи садятся друг напротив друга и сидят около часа неподвижно.
После такого молчаливого <<разговора>> спор обычно решается.

Тем не менее, несмотря на миролюбие жителей Мороза, у них существуют человеческие жертвоприношения.
Из-за климата достаточно частой травмой являются отмороженные глаза.
Общинники с отмороженным правым глазом называются Красный Снег, с левым --- Синий Снег.
Те же, кто лишился обоих глаз, называются Одно Лето.

В отличие от тех, кто пал в дороге, Одно Лето пользуются большим уважением.
Их очень хорошо кормят и выслушивают все их речи --- считается, что устами Одно Лето говорят высшие силы.
Когда приходит время миграции, Одно Лето усаживают недалеко от бижеч, чаще всего под скалой, защищённой от ветров.
Затем лидер отряда своими руками расстёгивает им комбинезоны на груди, отчего Одно Лето умирают в течение двух минут.
Считается, что принесённые в жертву слепые охраняют пустые бижеч от тёмных сил, а непобеждённая сила их тел передаётся соплеменникам.

В силу малой заселённости планета нейтральна по отношению к хоргетам, но Ад и Картель держат там наблюдателей.}

\asterism

\theterm{bizec-mol}
{Бижеч Мол}
{?}

\section{Преисподняя}

\theterm{netherworld}
{Преисподняя}
{?}

\asterism

\theterm{akiyama-volcano}
{Акияма (вулкан)}
{?}

\theterm{akiyama-town}
{Акияма (город)}
{?}

\theterm{arashiyama}
{Арашияма}
{?}

\theterm{asahina}
{Асахина}
{?}

\theterm{kiba}
{Киба}
{?}

\theterm{minamiyama}
{Минамияма}
{?}

\theterm{takesako-city}
{Такэсако (город)}
{?}

\theterm{takesako-dale}
{Такэсакское ущелье}
{?}

\theterm{hanayama}
{Ханаяма}
{?}

\section{Тра-Ренкхаль}

\theterm{tra-renkchal}
{Тра-Ренкхаль}
{?}

\asterism

\theterm{sanct3}
{Весёлый Волок}
{Святилище сели, тенку и ркхве-хор.}

\theterm{sanct6}
{Гора Песнопений}
{Святилище ркхве-хор и Красного колена.}

\theterm{two-and-two-cities}
{Двунадваградье}
{Устаревшее поэтическое название востока Короны, а также четырёх старейших городов тси: Кахрахана, Тхартхаахитра, Тхитрона и Травинхала.}

\theterm{deepdale}
{Лощина}
{Города-государства Синего колена.
Травники прошли по Дороге Жизни --- разлом, ведущий через смертные пустоши к Лощине.
Согласно легенде, утро застало переселенцев в Смертных пустошах, и травники уже приготовились к смерти, как вдруг увидели летящую на юг птицу --- тигровую сову.
Сова и вывела племя к Дороге Жизни.
С тех пор Синее Колено поклоняется совам;
в их легендах говорится, что солнце давно сожгло бы мир, если бы не Великая Алмазная Сова, которая закрывает мир своими прозрачными крыльями.}

\theterm{mikchan}
{Микхан}
{Культура идолов в Западном Живодёре со столицей в одноимённом городе.
Являются одними из последних тси, сохранивших технологию.
После Закатного Переселения к Морю Микхан пришёл в упадок, а большая часть идолов мигрировала в Молчащие леса, дав начало племенам Молчащих идолов.}

\theterm{sanct4}
{Одинокий Столб}
{Святилище сели и хака.}

\theterm{sanct5}
{Ожидание Вести}
{Святилище сели, ноа и трами.}

\theterm{sanct1}
{Омут Духов}
{Святилище сели и идолов Живодёра.}

\theterm{sanct7}
{Прибой}
{Святилище тенку, зизоце и ркхве-хор.}

\theterm{sanctuary}
{Святилище}
{Поселение с особым статусом, находящееся чаще всего на границе земель.
В святилище запрещено ношение оружия --- его сдают у ворот и хранят на специальных складах.
Скрытое ношение оружия, драки и оскорбления в святилище приравниваются к двум Разрушениям и караются смертью.
Управляет святилищем особый вид совета --- Сцепленные Руки.
Святилища можно даже считать отдельными городами-государствами, так как формально Сцепленные Руки не подчиняются лидерам составляющих город племён.
Всего на Тра-Ренкхале насчитывается семь действующих святилищ.}

\theterm{se-tchitr}
{Опалённая Речушка (Се'тхитр)}
{Река в северных землях сели, впадает в Ху'тресоааса, в устье находится Тхитрон.}

\theterm{risible-swamp}
{Смешливая Топь}
{Биогеоценоз в Западном Суболичье.
Микрофлора имеет уникальную особенность --- круговорот азота идёт через стадию свободного оксида азота (I).
Смешливая Топь опасна для практически любого животного Ветвей Земли;
тем не менее, там проживают 144 вида девиантных эндемиков-стенофагов, связанных в сложную пищевую цепь.
Орден Преисподней объявил Смешливую Топь объектом живой природы, нуждающимся в наблюдении и охране.}

\theterm{sanct2}
{Тёплый Двор}
{Святилище сели и пылероев Предгорий.}

\theterm{rivertangle}
{Тысячеречье}
{---}

\theterm{hedgehog-spine}
{Хребет Дикобраза}
{---}

\theterm{hu-tresoaasa}
{Ху'тресоааса (\theorigin{tn}{hu'tresoaasa}{река, которая просит пить})}
{?}

\theterm{abode}
{Четыре Обители}
{Святилища, в которых выхаживали больных.
Если человек безнадежно заболевал и хотел излечиться, то он отправлялся в обитель, где специально обученные жрецы проводили сложные, известные только им операции.
Иногда -- если болезнь была неизлечима или неизвестна жрецам -- больной добровольно отдавал себя на эксперименты, и это расценивалось как жертвоприношение.
Исцеленные в обителях давали клятву отработать в обители в течение восьми или шестнадцати дождей -- в зависимости от сложности случая.
Каждый первый год Церемонии обители брали на обучение чужих жрецов.
Обучение длилось ровно три дождя.
Прошедшим обучение набивалась особая татуировка на руке --- переплетение четырех цветков Короны и три иероглифа, кодирующих имя жреца.
Такую татуировку носили Кхатрим, Саритр, Трукхвал и Король-жрец Митрис ар'Люм.
Обители носили имена Пепельной, Снежной, Каменной и Песчаной.}

\theterm{oerho-loeaesakch}
{Эрхо'люаэсакх}
{Извилистое опасное течение между Короной и Кристаллом.}

\section{Тси-Ди}

\theterm{qi-di}
{Тси-Ди}
{Двойная планета системы белого карлика, старое название --- Мерлин-Ниниана.
Общепринятое, вероятно, от чайнис 启迪 (Q\v{\i}-D\'{\i}) --- <<вдохновение>>.
Планеты имеют примерно одинаковую массу, вследствие чего обращаются вокруг равноудаленной от планет точки --- Центра Масс.
Позже Центром Масс стали называть расположенную в этой точке станцию --- в ней располагалась научно-исследовательская база, центр полётов между планетами и узел стабилизации планетарной системы защиты.

Планеты всегда повёрнуты друг к другу одной (океанической) стороной.
На планете Тси имеются 14 материков, связанных Паутиной --- дорогами на силовых полях.
Обитаемая зона планеты Ди --- 10 тысяч километров по периметру океана, покрытые лесостепью и вечнозелёными кустарниками.
Всё остальное пространство --- каменистая пустыня с разреженной атмосферой, в которой располагаются рабочие и исследовательские механизмы и Оазисы --- закрытые станции для проживания сапиентов.

Ранее Тси-Ди была обиталищем народа тси, впоследствии тси были уничтожены Машиной.
Согласно данным, в настоящее время на планете Тси-Ди не обитает ни один сапиентный вид Ветвей Земли.}

\section{Тысяча Башен}

\theterm{thousand-towers}
{Тысяча Башен}
{Жидкое ядро с кристаллической поверхностью.
Поверхность состоит из гигантских Друз, разделённых полосками воды в глубоких ущельях --- Трещинах.
Единого океана нет.
Соответственно, из-за вращения планеты и прочих причин скорость воды очень высока, Друзы постоянно стачиваются, но тут же нарастают из-за насыщенных тем же веществом вулканических газов.
Также Друзы от стачивания спасают колониальные раковинные хемосинтетики, использующие энергию вулканических газов;
интенсивное нарастание этого бактериально-протозойного мата, обладающего противотурбулентными свойствами, спасает друзы от вымывания.
Они же вырабатывают свободный кислород, необходимого для дыхания.
Из Друз торчат Башни (отдельные крупные кристаллы), которые регулярно подвергаются мощнейшему выветриванию.

Всего на тысяче башен 138 друз, 79 из которых являются обитаемыми.
Прочие либо слишком малы, либо низкие (их поверхность находится в Газовом Океане), либо чересчур высоки.

Самыми благоприятными для жизни считаются Жеоды --- чашевидные кристаллы, заполненные водой.
Жеоды отлично подходят для земледелия и рыбоводства;
однако из-за них разгораются самые жестокие войны, иногда длящиеся сотни лет.
Одним из таких мест является плодородная Кровавая Чаша, граничащая с тремя крупными Друзами --- война на ней не утихала более двух тысяч лет.

Пейзаж напоминает чем-то каньоны Северной Америки.}

\asterism

\theterm{townhenge}
{Висячие руины (повешенные города)}
{Покинутые или вымершие поселения, которые из-за нарастания Друз поднялись на высоты, непригодные для дыхания.
Очень часто висячие руины находят под ледниками, покрывающими высокие Башни.}

\theterm{gas-ocean}
{Газовый Океан}
{Тяжёлые газы в Трещинах, которые скапливаются над поверхностью воды.
Поверхность Газового Океана гораздо выше водной, из-за него некоторые Друзы непригодны для жизни.}

\theterm{garrota}
{Гаррота (астероидное кольцо)}
{---}

\theterm{eisenloon}
{Железная Луна}
{---}

\theterm{ounce-ring}
{Кольцо Барса (астероидное кольцо)}
{---}

\theterm{goat-ring}
{Кольцо Козла (астероидное кольцо)}
{---}

\theterm{qyschlau}
{Кышлау}
{---}

\theterm{crossing}
{Переправа}
{Место, где можно почти без проблем перелететь на глайдере с одной Друзы на другую.
Все переправы односторонние.
Переправой считают путь, по которому совершили перелёт не менее двадцати воздухоплавателей и не менее восьмидесяти процентов из них выжили.
Переправы бывают постоянные и сезонные (зависимые от ветров).}

\theterm{rotesturm}
{Ротештурм}
{---}

\theterm{sarland}
{Сарланд}
{---}

\theterm{essloonen}
{Съедобные Луны (Эсслунен)}{---}

\theterm{flockenblume}
{Флокенблуме}
{Древняя, поныне неприступная крепость на Вольке, расположенная в микродрузе. После пожара, уничтожившего деревянные стены, там был построен монастырь, обучавший искусству Туманной Войны. Флокенблуме является традиционным местом сбора для переговоров и в случае набегов (Хильденфламме). Кроме того, каждый торговец отчисляет в пользу Флокенблуме натуральный налог (флокенгильд --- около 2--4\%, в зависимости от объема и типа товара), который идет нуждающимся и используется в случае голода.}

\theterm{himmelrot}
{Химмельрот}
{---}

\theterm{holzhafen}
{Хольцхафен}
{---}

\section{Чёрная Скала}

\theterm{black-rock}
{Чёрная Скала}
{Одна из первых планет в Красном Картеле.
Сухая вулканическая планета, обращающаяся по гигантской сложной орбите вокруг тусклого красного гиганта.
Отсутствия многих необходимых ресурсов (в частности, именно на Черной Скале был разработан спектр так называемых <<чёрных>>, основанных на железе сурроганиумов --- из-за практически полного отсутствия титана).

Чёрная Скала --- один из первых планетарных питомников антарид, в настоящее время практически заброшена Картелем из-за заражённости антаридами.
Манипула Смеха признала, что единственный доступный способ вывести антарид с Чёрной Скалы --- это физическое уничтожение планеты.}

\chapter{Языки и письменность}

\theterm{abis}
{Абис}
{Письменность ноа.
Слоговая, звуко-буквенная, идеограммы для грамматических элементов.}

\theterm{bi}
{Би}
{Всеобщий язык Древней Земли, попытка связать язык машины и язык биологической нейросети.
Был разработан за тысячу пятьсот лет до гибели цивилизации Древней Земли.}

\theterm{snake-script}
{Змеистая письменность}
{Единственный в своём роде звуко-буквенный вид письменности, используемый тси-язычными племенами.
Используется Молчащими идолами, говорящими на языке хесетрон.
Автор неизвестен.
Основан на математических символах древних тси.
В настоящее время упрощённый вариант используется диктиологами Ордена как фонетический алфавит тси-подобных языков.}

\theterm{mosquito-script}
{Комариная письменность}
{Предположительно письменность исчезнувших травников Синего колена.
Является потомком идеографической письменности тси.
Все ключи и модификаторы пишутся вдоль линии, зеркально дублируются сверху и снизу (видимо, из соображений сохранности).}

\theterm{mikchan-cyphers}
{Микханская тайнопись}
{Записываемый змеистым письмом код, образцы найдены в катакомбах города Микхан в Западном Живодёре.
На сегодняшний день найдена переписка из 230 кусков коры, в которой участвовали трое (по мнению некоторых исследователей --- четверо) идолов Микханской культуры.
В основном представляет собой картинки с подписями.
Исследователи склоняются к мысли, что тексты --- автоматическое письмо и служат для отвлечения внимания от рисунков, в которых содержится основной смысл посланий.
Микханская тайнопись не дешифрована по сей день.
В переносном значении --- нечто недоступное пониманию и, возможно, бессмысленное.}

\theterm{dream-pidgin}
{Пиджин снов}
{Сапиентный язык, который использовался демонами как код.
С помощью специального алгоритма слова в языке перепутывались.
Несмотря на строгое соответствие грамматике исходного языка, сообщение на пиджине снов для сапиентов выглядело как полная бессмыслица, но легко раскодировалось демоном, имеющим большие вычислительные мощности.
Пиджины снов практически вышли из употребления после изобретения языка Эй.
Один из самых известных --- сохтид (который, тем не менее, не является настоящим пиджином снов, а принадлежит к так называемым комбинированным кодовым языкам).}

\theterm{supported-language}
{Поддерживаемый язык}
{?}

\theterm{sketches}
{Резы}
{Подобие письменности у стрелохвостов.
Представляет собой нанизанные на верёвку каменные бусины, косточки, куски акульей чешуи и обработанной кожи с насечками.
Есть данные, что у стрелохвостов Голубого Зеркала имеется <<библиотека>> резов --- Грот, в которой собраны работы по медицине, истории и промыслам.}

\theterm{ruse}
{Русе}
{Основной язык Мороза, произошедший из одного из языков Древней Земли.}

\theterm{sarqort}
{Cаркорт (сарлид, гельблид)}
{Язык друзы Тартария и Сарланда на Хербст.}

\theterm{sectum-lingua}
{Сектум-лингва}
{Поддерживаемый язык Красного Картеля на основе эллатинского языка Древней Земли.}

\theterm{si}
{Си (Си-поинт)}
{Всеобщий машинный язык, созданный на Древней Земле.
Может быть использован практически для любого типа квантовых, световых и электронных архитектур.
Используется до сих пор при программировании хоргетов.}

\theterm{sojtid}
{Сохтид}
{Комбинированный демонический пиджин планеты Преисподняя, поддерживаемый язык Ордена Преисподней.}

\theterm{shp}
{Стандартная человеческая фонетика (СЧФ)}
{Спектр звуков, которые способен произвести неизменённый голосовой аппарат особи \textit{Homo homo sapiens}, а также система записи этих звуков.}

\theterm{talino}
{Талино}
{Дикий язык планеты Драконья Пустошь на основе сектум-лингва.}

\theterm{qi-language}
{Тси (язык)}
{Язык Тси-Ди, предок тси-подобных языков.
Одной из теорий происхождения языка тси является так называемая дельфинья.
В пользу этой теории говорит происхождение корней --- многие из них являются дельфиньими заимствованиями, в том числе характерными только для дельфинов звуковыми отпечатками предметов.
Также в языке тси есть множество понятий, которые в принципе отсутствуют у любых сухопутных сапиентов, но есть в языках акарид и дельфинов.
Тем не менее, в настоящее время теория считается несостоятельной, так как грамматика тси более напоминает языки человеческих и плантских племён планеты Ди, нежели любой из известных дельфиньих языков.}

\theterm{hanyui}
{Ханьюи}
{Язык Древней Земли, один из основных языков ранней Чёрной Скалы.
Язык шакуната является пиджином снов языка ханьюи.}

\theterm{tesatron}
{Цатрон}
{Лингва франка планеты Тра-Ренкхаль.
Изначально --- язык племени царрокх.}

\theterm{ej}
{Эй}
{На Древней Земле: гипотетический универсальный язык.
Современное значение: созданный Ликаном Безруким поддерживаемый язык, не претендующий на универсальность, но весьма удобный и потому распространившийся по Вселенной.}

\theterm{emoglyph}
{Эмоглиф}
{Элемент письменности сели для записи настроения или отношения сапиента к написанному.
Разновидность эмоглифа --- филоглиф, кружочек или кардиоида, которые ставятся над именем тси, к которому автор испытывает сильную привязанность или уважение.
В устной речи классического тси эмоглиф указывает на интонацию;
в нескольких языках современных тси-язычных племён эмоглиф имеет своё собственное чтение, чаще всего в виде соответствующего эмоции междометия.
Предусматривает 4096 оттенков эмоций, универсален для всех сапиентных видов.}

\theterm{englis}
{Энглис}
{Основной язык Древней Земли с эпохи Последней Войны.}

\theterm{erdenlied}
{Эрденлид}
{Официальный язык друз Хербст и Вольке.
Включает в себя диалекты: нойлид --- центральная часть Хербст, бергенлид --- Сарланд, эрденшпрак высокий, старый и приречный --- Вольке.}

\theterm{cell-script}
{Ячеистое письмо}
{Торгово-дипломатическое письмо травников.
Напоминает соты.}

\chapter{Шкала Яо}

Барьеры развития --- ключевые моменты для цивилизации, связанные со средним уровнем развития живых существ (по Яо).

\begin{description}
\item [80] --- барьер Начала: существо способно использовать материальные объекты для своих нужд, но не способно к направленному их преобразованию.
Зарождение религии.
\item[100] --- барьер Культуры: существо способно осмысленно и направленно преобразовывать материальные объекты и приспосабливать их к своим нуждам.
Зарождение технологии.
\item[130] --- барьер Цивилизации: существо способно к передаче информации с помощью материальных символов.
Зарождение науки, разработка научной методологии.
\item[200] --- барьер Эмпатии: первый критический момент.
Существо способно коммуницировать и сотрудничать с любыми другими существами.
Выход в космос.
Существует опасность само- и взаимоуничтожения.
Преодолению барьера Эмпатии способствует выработка универсального морального кодекса.
\item[400] --- барьер Создателя: второй критический момент.
Технологическая сингулярность, создание существ c возможностями, превышающими собственные, но находящихся на более низком уровне развития.
Опасность уничтожения созданными существами.
Преодолению барьера Создателя способствует помощь новосозданным существам в преодолении ими барьера Эмпатии, приобщения их к универсальному моральному кодексу.
\item[1000] --- барьер Наблюдателя: существо видоизменяется настолько, что уже не может быть отнесено к породившему его виду.
Полиморфизм, способность свободно перемещаться внутри и вне Вселенной, бессмертие (способность использовать любые источники энергии, неразрушимость и неизнашиваемость).
\end{description}

\textbf{Примечание.}
1000 --- условное число.
Согласно последним данным, барьер Наблюдателя, который должен быть преодолён \textit{микоргетом}, находится в промежутке между 873 и 1090.

\chapter{Классификация Ветвей}

\section{Ветви Земли}

\theterm{earth-forks} % Earth Forks
{Ветви Земли}
{?}

\subsection{Сапиентные виды}

\subsubsection{Ветвь Хуманы (Люди)}

\theterm{human-fork}
{Люди}
{Самые успешные сапиенты.
Во Вселенной в настоящее время насчитывают около 98 тысяч видов-потомков.}

\asterism

\theterm{lotids}
{Лотиды (не путать с лотинами)}
{Аборигенные люди Тра-Ренкхаля.
В настоящее время представлены народами хака, тенку, зизоце и прочими.
Представляют собой смесь потомков первых людей Древней Земли и прибывших чуть позже людей Лотоса.
Скрещивание между ними предотвратило появление репродуктивного барьера.
От первых людей отличаются сильно развитым половым диморфизмом, свойственным людям Лотоса, от бледных людей Лотоса отличаются коричнево-чёрным цветом кожи и радужных оболочек глаз.
Также имеются приобретённые позднее особенности --- в частности, у 56\% людей Тра-Ренкхаля транспозиция внутренних органов, не характерная для предков и не имеющая пока рационального объяснения.}

\theterm{lotins}
{Лотины}
{Люди Лотоса.}

\theterm{qi-humans}
{Люди-тси}
{Жители Тси-Ди и Тра-Ренкхаля, в настоящее время представлены народами сели и ноа.}

\theterm{rut-misa}
{Род Медведя (\theorigin{ru}{Rut Mi\v{s}a}{род медведя})}
{Люди Мороза, один из немногих видов людей, сохранивших шерсть.
Чёрная кожа, густые беcцветные волосы на всём теле, единственные безволосые части --- нос и подушечки пальцев, у женщин --- верхняя губа.
Мужчины и женщины имеют мощную жировую прослойку, но у женщин она больше (стеатопигия).}

\theterm{tagua}
{Тагуа}
{Жители Драконьей Пустоши.
От первых людей отличаются незначительно.
Изменён метаболизм, имеются необычные пигменты в кожном покрове и глазах --- адаптация к излучению голубого гиганта.
Широко расставленные глаза, в верхней губе --- небольшая расщелина около 1 см длиной, по четыре пальца на ногах (результат дрейфа генов).}

\subsubsection{Ветвь Кани}

\theterm{kani-fork}
{Кани}
{Результат генетического эксперимента людей, собаки с изменёнными конечностями и увеличенным мозгом.
Видов-потомков --- 58 тысяч.}

\asterism

\theterm{qi-kani}
{Кани-тси}
{В настоящее время представлены пылероями Пыльного Предгорья, народом ркхве-хор и Высшими.
Очень высокого для кани роста --- 1,7--2,2 м.
Имеют светло-серую или коричневатую бархатную шерсть и голубые глаза, также для обоих полов характерна грива.
На четвереньках способны развивать самую высокую скорость среди наземных животных --- около 130 км/ч.
Пылерои --- владыки пустынь и саванн.
Обитают как на Короне, так и на Ките.
Предгорные пылерои занимаются скотоводством, разводят чёрных трёхгорбых верблюдов, рептилий и крупных съедобных насекомых, иногда устраивают плантации в оазисах.
Ркхве-хор живут военными набегами.
Высшие живут в оставленных первыми поселенцами подземных городах на экваторе и почти ни с кем не контактируют.}

\theterm{rut-ulka}
{Род Волка (\theorigin{ru}{Rut Ulka}{род волка})}
{Кани Мороза.
Черная кожа, сероватая шерсть с мощным подшёрстком, короткие нос и уши.
Как и Род Медведя, особи обоих полов имеют мощную жировую прослойку (параллелизм).
Также Род Волка отличается от первых кани меньшим размером клыков.}

\theterm{hrgadah}
{Хргада}
{Высокоразвитые кани с планеты Запах Воды системы Канопуса.
От первых кани отличаются отсутствием волос, очень изящным телосложением и плоской грудной клеткой.
Также у них имеются особые ферментативные системы репарации ДНК и некоторые другие изменения метаболизма (следствие повышенного радиационного фона на планете).}

\subsubsection{Ветвь Планты}

\theterm{plant-fork}
{Планты}
{Результат генетического эксперимента людей, клеточный гибрид человека, цианобактерии и нитробактера с некоторыми дополнительными генами.
Во Вселенной насчитывается около 23 тысяч видов-потомков.

Планты прекрасно поглощают воду кожей.
Если плант, стоя на солнце, засунет руку в ведро с водой, через два часа в ведре будет сухо.
Лёгкими усваивают азот и углекислый газ.}

\asterism

\theterm{qi-plants}
{Планты-тси}
{В настоящее время представлены идолами Молчащих Лесов, идолами Живодёра, Снежным Кланом и народом трами.
Внешне от первых плантов отличаются незначительно, рост от 1,2 до 1,5 м.
Волосяной покров отсутствует.
Изменены верхние конечности, благодаря чему планты-тси очень хорошо лазают по вертикальным поверхностям.
Прочие их особенности будут рассмотрены в соответствующем разделе.

Обитают в джунглях Короны и Кита.
В жертву богам приносят в основном пленных людей, для жертвоприношений используют рощи благородного баньяна или строят срубы.
Живут деревнями по двести особей максимум, в гнёздах на верхушках деревьев.
Высокоразвитыми считаются трами, обитатели Кита, находящиеся в торговых отношениях с ноа.
Название <<идолы>> пошло от их привычки стоять неподвижно под лучами солнца.}

\subsubsection{Ветвь Апиды}

\theterm{apis-fork}
{Апиды}
{Результат генетического эксперимента первых людей, насекомые (предположительно пчёлы), увеличенные в размерах, с изменёнными конечностями, скелетом, дыхательной системой и увеличенным окологлоточным нервным кольцом.
Цель эксперимента неясна до сих пор, скорее всего, он носил чисто научный интерес.
Видов-потомков --- 790.}

\asterism

\theterm{di-apis}
{Апиды Ди}
{Вид, уничтоженный во время Тараканьей войны.
Известно, что они являлись, как и большая часть апидов, колониальными сапиентами (матка, бесполые рабочие особи и трутни).}

\theterm{qi-apis}
{Апиды-тси}
{В настоящее время представлены Красными травниками и Бродячим Народом.
Отличаются от ранних апид очень изящной, тонкой конституцией, плодовитостью и перманентным гермафродитизмом.
Рост 1--1,5 м.
Глаз шесть (четыре парных простых, два фасеточных).
Усики развиты умеренно, в углублениях головы, ногочелюсти имеют один ряд зубчиков, которые сменяются в течение жизни.

После вторжения Безумного травники сильно пострадали от войн с идолами, в конце концов были вытеснены в Серебряные горы и Старую Челюсть, где живут очень разрозненно --- отдельными семьями --- в пещерах.
Их называют Красными травниками.
Многие Красные впоследствии ушли к людям и стали Бродячим Народом, небольшие поселения стали появляться в джунглях Кита, где идолов нет, но опять же --- ближе к людям, обеспечивающим им безопасность от Безумного.}

\subsubsection{Ветвь Дельфины}

\theterm{delfina-fork}
{Дельфины}
{Результат генетического эксперимента, низшие дельфины с изменёнными конечностями.
Видов-потомков --- 1264.}

\asterism

\theterm{qi-delfina}
{Дельфины-тси}
{В настоящее время представлены океаническим народом и стрелохвостами Голубого Зеркала.
Некрупных размеров, около 2 м в длину.
Шкура серая, плавникоруки тонкие, в отличие от ранних предков имеют всего три пальца.
Ротовой аппарат приспособлен как к растительной, так и к животной пище.

Океанический народ (стрелохвосты, или няньки) живёт в океане в тропических и субэкваториальных поясах, кочует за косяками рыб, на стоянках разворачивает понтонные лагеря.
Это единственное племя, которое не приняло ультиматума Безумного.
Мобильны, живут группами по 10--15 особей.
В силу этого, а также особой философии, подразумевающей подавление эмоций, добивание больных и раненых, стрелохвосты были оставлены Безумным в относительном покое.
У каждой группы есть старейшина --- стрелохвост с развитым чувством интуиции, который и подсказывает безопасный путь для группы.}

\subsubsection{Ветвь Стриги}
 
\theterm{striges-qi}
{Стриги, или глазастики}
{Единственный вид ветви Стриги, упоминаемый в отчётах о Тси-Ди.
Представляют собой увеличенных в размерах (рост около 1,3 м) сов с шестью парами конечностей --- две ходовых, две летательных и две рабочих.
Согласно данным Существует-Хорошее-Небо, глазастики были созданы сравнительно недавно, а потому не успели адаптироваться к культуре тси, жили обособленно в своих поселениях.
Вероятно, мигрировавшие на Тра-Ренкхаль особи впоследствии вымерли --- несмотря на свидетельства местных жителей, обнаружить глазастиков так и не удалось.}

\subsubsection{Ветвь Акариды}

\theterm{acarides-fork}
{Водолазы, или акариды}
{Результат эксперимента плантов планеты Мицелий системы Канопуса, человек с жабрами по типу акульих и видоизменёнными конечностями.
Способны длительное время оставаться под водой.
Предпочитают тёплые реки и моря.
Видов-потомков --- 56.}

\subsection{Флора и фауна}

\theterm{akchkatraas}
{Акхкатрас (\theorigin{tn}{akchkatraas}{духи, взывающие к живым})}
{Женские растения вида секвойя Бенедикта, завезённого переселенцами с планеты Лотос.
Мужские растения, что интересно, имеют совершенно другое название --- мисатр --- и считаются мусорными деревьями.
Самые старые деревья достигают 150 м в высоту и более 10 м в диаметре;
пряные ягоды (ложноплоды) акхкатрас имеют характерный ярко-алый оттенок, носящий то же название, и считаются изысканным деликатесом.}

\theterm{noble-banyan} % noble banyan
{Баньян благородный (<<слепой страж>>)}
{Древесное растение, произрастающее на Тси-Ди, было завезено переселенцами на Тра-Ренкхаль.
Взрослое растение отдаёт в стороны воздушные и надземные побеги, образуя рощицу размером в несколько гектар.
Назван так из-за отсутствия в рощице прочих растений, бактерий и грибов (баньян выделяет несколько десятков мощных фитонцидов, антибиотиков и фунгицидов).
Прежде рощи благородного баньяна использовались как храмы и больницы (из-за стерильной внутренней среды), но из-за плохой обороноспособности потеряли своё значение.
Некоторые исследователи склонялись к мысли, что благородный баньян --- это легендарная кольцевая теплица, но эта гипотеза не подтвердилась.}

\theterm{vegetectors}
{Вегетекторы (от \theorigin{sl}{vege[tare]}{расти} и \theorigin{sl}{[archi]tector}{архитектор})}
{Грибы и растения, вырастающие в полноценное жилище для сапиента.
Были очень популярны на Тси-Ди в период биоромантизма.
Впоследствии разработка вегетекторов была заброшена по причине сложности процесса программирования и длительного роста.}

\theterm{rope-snake} % rope snake
{Верёвочная змея}
{Одна из самых маленьких змей на Тра-Ренкхале --- длиной в две пяди, толщиной в мизинец.
Бежевого цвета, на спинке узор в виде свитых веревок.
Питается в основном насекомыми и крупными беспозвоночными.
Укус не ядовит, змея не способна прокусить кожу человека, но в пищу непригодна, так как впитывает яды и едкие жидкости съеденных ею насекомых.}

\theterm{featherwood}
{Дерево Перьев (Бальса оружейная)}
{Древесное растение Тра-Ренкхаля, древесина которого отличается небольшим удельным весом и большой прочностью.
Из Дерева Перьев изготавливают стрелы, древки копий и фаланг (все народы Короны), боевые веера (ноа, южные сели), основы для переносных жилищ (кочевые пылерои Предгорий), основы для лёгких кожаных лодок, пригодных для порожистых рек (сотронские сели, тенку), несущие балки зданий (Утонувший храм Сотрона), а также самую разнообразную домашнюю утварь.}

\theterm{green-bee}
{Зелёная пчела, трукхвал (\theorigin{tn}{trukchual}{летающая драгоценность})}
{Вид перепончатокрылых насекомых Тра-Ренкхаля.
Крупные (около 2-3 см) тела, 4 крыла (два больших, прозрачных, и два маленьких, желтовато-опалесцирующих).
Брюшко полосатое, зелёное с жёлтым.
Жало имеет чехол, который отрывается при укусе и отрастает через какое-то время.
Яд смертельно опасен для всех видов, за исключением колибри вида Пчелиный Ужас, которые питаются зелёными пчёлами (смертельная доза для человека Тра-Ренкхаля --- 2 укуса, для человека-тси --- 40--80 укусов).
Гнёзда строят в кронах деревьев-медоносов, форма гнёзд --- цепочка шаров (обычно 3--5), от самых маленьких внизу до самых больших сверху.
Гнездо строится из выделений симбионта --- паука-буйвола, которого зелёные пчёлы ловят, выращивают и доят, а затем инкрустируется кусочками коры.
В верхнем ярусе гнезда выращиваются личинки, нижние ярусы используются для грибных ферм, где зелёные пчёлы выращивают низшие грибы.
Мёд трукхвала имеет особый грибной вкус и ценится как изысканный деликатес.}

\theterm{indigo-firefly}
{Индиго-светляки}
{Семейство девиантных насекомых Тра-Ренкхаля.
Крупные, до 4--5 см в длину.
Имеют три глаза, три жёстких крылышка и органы полёта --- геликоптероиды, характерные для девиантных насекомых Тра-Ренкхаля.
Специфическая система биолюминесценции --- фотоны образуются в результате химических реакций и проходят через тонкую хамелеоновую занавеску, приобретая цвет от ультрафиолетового до зелёного.
Не путать с лантерн-светляками, которые относятся к Ветвям Земли.}

\theterm{stonetoad}
{Каменная жаба}
{Вид земноводных Тра-Ренкхаля.
Крупные (до 15 см), узкоротые, сероватого цвета, рисунок кожи напоминает гранит.
Живородящие.
На спине имеются костно-роговые щитки.
Крик каменной жабы напоминает гуление и плач младенца.
Днём животное сидит в каменных пещерках и каменных постройках, ночью выбирается наружу --- на охоту и для спаривания.
Иногда приползает на детский плач, из-за чего на севере её зовут жабой-кормилицей.}

\theterm{ringhouse}
{Кольцевая теплица}
{Живое существо, разработанное на Тси-Ди для обеспечения тси и других животных пищей на орбитальных станциях, в межзвёздных кораблях и на инопланетных базах.
Поставляет аналоги растительных волокон и животных белков, полный спектр аминокислот и витаминов.
Способно использовать любые растворимые в воде, кислотах и предельных углеводородах минералы, а также впадать в спячку при неблагоприятных условиях.
Орден Преисподней объявил крупную награду за оцифрованные клетки и важные сведения об этих существах.}

\theterm{coraldrake}
{Кораллица}
{?}

\theterm{redball}
{Красный шар}
{?}

\theterm{laaka} % laaka sap
{Лака \theorigin{tn}{laaka}{плохой знак}}
{Древесное растение Тра-Ренкхаля с чёрными листьями и ярко-красными ядовитыми плодами.
Лаковый сок содержит мощный нейротоксин, который даже в малых дозах способен почти мгновенно остановить сердцебиение и нервную деятельность людей Тра-Ренкхаля и тси.
Несмотря на то, что самыми опасными частями растения считаются ягоды, ствол, ветви и листья также нельзя трогать голыми руками, а в жару опасно даже находиться рядом с деревом без утиной маски, пропитанной солевым раствором.
Растворённый в масле лаковый сок относительно безопасен для хранения и переноски, его заливают в стрельные чехлы и амулеты Сана.
Места произрастания лаки отмечаются на картах четырьмя кружками, как и указатели ведущих к ним тропинок.}

\theterm{praypoppy}
{Мак молитвенный}
{Распространённое на Тра-Ренкхале растение, являющееся природным источником морфина-44, мощного анальгетика и снотворного.
Был завезён тси и генетически модифицирован, чтобы успешно конкурировать с местными растениями.
В пользу этого говорит его ареал (влажные зоны всех материков, за исключением пустынь и Дальнего Севера), а также наличие в мозгу тси ферментативных систем, специфичных именно к морфину-44.
Употребление масла молитвенных маков представителями тси совершенно безопасно, у прочих сапиентов планеты Тра-Ренкхаль масло вызывает галлюцинации, судороги и сильную наркотическую зависимость.}

\theterm{wreath-of-malikch}
{Маликхов венок}
{Папоротник-эпифит, издающий очень приятный аромат.
Использовался сели как сухоцвет и ингредиент для курений.}

\theterm{melipona}
{Мелипона Водораздела}
{Прирученная медоносная пчела Тра-Ренкхаля.
Не имеет жала.
Производит сладкий мёд, являющийся основой многих блюд сели, тенку и ркхве-хор.}

\theterm{milkbush-of-fisher}
{Молочник рыболовный}
{Кустарниковое растение Тра-Ренкхаля.
Сок его оказывает парализующее действие на рыбу, но безвреден для сапиентов.
Рыбаки (обычно дети) выливают сок молочника в реку и ждут, пока парализованная рыба всплывёт.}

\theterm{silent-cedar} % silent cedar
{Молчащий кедр}
{Древесное растение Тра-Ренкхаля, распространённое в Суболичье и Молчащих лесах.
Также растение встречается в Сикх'амисаэкикх, Хрустальных землях и Старой Челюсти.
Древесина и хвоя благодаря микроструктуре обладают свойством гасить звуковые колебания в диапазоне 1--23 кГц.
Из молчащего кедра сели делают столбы, половые плиты и навес для мёртвой зоны;
благодаря им крики жертв и прочий шум не покидают крышу храма.
Также молчащий кедр используется для пытки тишиной у ноа и трами.
В лесах молчащего кедра живёт ограниченное число видов Ветвей Земли --- из-за затруднённости или полной невозможности звукового общения.}

\theterm{one-love-furfeet}
{Мохноножка-однолюбка}
{Бескрылая птица Тра-Ренкхаля.
Питается беспозвоночными подстилки.
Дробно щёлкают клювом.
Моногамны, название пошло из-за тонких пёрышек на цевках.}

\theterm{marblesnake}
{Мраморная змея}
{?}

\theterm{niemelto}
{Нимелто (\theorigin{ru}{niemelto}{коровы Немальцевой})}
{Генетически модифицированное жвачное млекопитающее с планеты Мороз системы Арракиса, отличающееся очень высокой устойчивостью к холоду благодаря микроструктуре меха и кожи.
Самцы отличаются от самок небольшими опушёнными рожками. Некоторые особи способны переносить температуры порядка 190°К.
Выведены лабораторией Ольги Немальцевой в раннюю Эпоху Богов.}

\theterm{fiddletail-deer}
{Олень-вертихвостка}
{Одомашненное сели девиантное парнокопытное животное, конституцией напоминающее антилопу.
Самцы круглый год носят длинные двуветвистые рожки, самки безрогие.
Название пошло от привычки животного трясти небольшим белым хвостиком --- способ внутривидовой коммуникации.}

\theterm{swing-around}
{Папоротник-опахало}
{Папоротник Тра-Ренкхаля, имеющий очень длинные перообразные вайи и крупные спорангии, которые в спелом состоянии лопаются от прикосновения, выбрасывая в воздух значительное количество спор.
Споры опахала используются как лечебная пудра и впитывающий агент в ремеслах.}

\theterm{beggarbird}
{Попрошайка (\theorigin{tn}{trasakch}{требует мёд})}
{Птица джунглей Тра-Ренкхаля из семейства медоуказчиков.
Питается насекомыми, мёдом и воском перепончатокрылых.
Нападает на пасеки в Пыльном Предгорье, из-за чего активно истребляется на Западе.
Охотно поедает и гнёзда зелёных пчёл, но сама не может атаковать гнездо, поэтому показывает людям или идолам дорогу к гнёздам, после чего ждёт, пока охотники разделаются с пчёлами.}

\theterm{sugarfly}
{Сахарная муха}
{Вид двукрылых насекомых Тра-Ренкхаля.
Взрослые особи откладывают яйца в закисшие фрукты, которыми питаются опарыши.
Опарыши используются восточными сели как деликатес --- в вяленом виде или в виде сладкой муки, из которой делаются конфеты.
Вдоль Западного тракта сахарную муху разводят пчеловоды вместе с мелипоной;
ближе к Водоразделу муховодство почти не распространено.}

\theterm{siu-siu}
{Сиу-сиу (звукоподражание песне)}
{Птица Тра-Ренкхаля с ярким оперением, которое высоко ценилось племенем сели.
Перья сиу-сиу шли на праздничные головные уборы и входили в состав ритуального подношения Солнечной птице.}

\theterm{scout-ephedra}
{Скаут-эфедра (эфедра R88, хвойник неприхотливый)}
{Генетически модифицированное голосеменное растение ранней Эпохи Богов, использовавшееся для оживления планет.
Способно произрастать на холодных планетах, бедных кислородом, в рекордные сроки создавать устойчивый к эрозии слой почвы, а также особую наземную экосистему с локальным повышением температуры и повышенной концентрацией кислорода.
В зарослях скаут-эфедры, которая могла выдержать высокую радиацию, ветра и снегопад, жили насекомые, мелкие птицы и млекопитающие.
Во многих независимых культурах скаут-эфедра является священным растением, из неё плетут венки и делают погребальные убранства.}

\theterm{deathly-heat}
{Смертожар}
{?}

\theterm{}
{Согхо (\theorigin{tn}{sogcho}{печальная флейта})}
{Птица Тра-Ренкхаля из семейства курообразных.
Самка серая, с сизыми подпалинами на брюшке.
Самец яркий, фиолетово-лиловый, с красивым веерообразным хохолком, поёт на восходе и закате солнца однообразную, напоминающую звук свирели песню (<<тиу-лиу-ла, фьюю, тиу-ла>>).
Мясо согхо --- деликатес, очень нежное и питательное.
Перья самца согхо используются сели для так называемых скорбных уборов (чаще всего серёг), которые носили в знак жажды мести и скорби по убитым друзьям.}

\theterm{tchal-sar}
{Тхальсар (шорея тёмная, дерево Корвуса)}
{Растение, завезённое на Тра-Ренкхаль с Лотоса.}

\theterm{large-eared-dragaway}
{Ушастая уволочь}
{Мелкое (20 см без хвоста) кошачье джунглей Тра-Ренкхаля.
Охотится на крупных насекомых и мелких грызунов, но очень часто совершает набеги на жильё, утаскивая разделанное мясо и пищевые остатки.}

\theterm{chrikchsatr}
{Хрикхсатр (хрустальное дерево)}
{Единственный во Вселенной девиантный род древовидных лишайников, произрастающий на Тра-Ренкхале.
Включает 4 вида --- оранжевый хрикхсатр, бирюзовый хрикхсатр, сердитый коралл, дерево-свинья.
Все они являются эндемиками Мшистой степи.
Бирюзовый хрикхсатр впоследствии был вывезен с Тра-Ренкхаля и, благодаря чрезвычайной прихотливости и красоте, стал излюбленным комнатным лишайником у почвоведов Капитула.}

\theterm{spiny-ostrich}
{Шипастый страус (страус-дикобраз)}
{Бескилевая птица саванны Тра-Ренкхаля с видоизменённым пером (шипами).
Является одним из двух видов ветви Броненосные казуары (название <<страус>> ошибочно).
Чрезвычайно опасен, так как проявляет агрессию к любым сапиентам на своей территории, очень ловок и хорошо уворачивается как от стрел, так и от копий, а шиповатое оперение представляет собой неплохую броню.
Врагов колет шипами и бьёт мощными ногами до смерти.
Гнездится неподалёку от водоёмов, в низинах.
Перо страуса-дикобраза ценится всеми племенами Тра-Ренкхаля, часто вплетается в волосы как знак силы (иногда --- как тайник, так как в полость пера помещается значительное количество золотого песка и небольшие свитки пергамента).}

\section{Девиантные Ветви}

\theterm{deviant-forks}
{Девиантные Ветви}
{Виды, созданные инженерным путём хоргетами-демиургами (либо, согласно некоторым определениям, сапиентами).
Отдельные исследователи настаивают на том, что Девиантные Ветви следует изучать как технологию, а не как биологические системы, так как они --- результат разумного замысла, а не эволюции.
Несмотря на то, что эта точка зрения не разделяется большинством учёных Ордена Преисподней, она оказывает известное влияние на парадигму исследования Девиантных Ветвей и потому заслуживает упоминания.}

\subsection{Сапиентные виды}

\theterm{machinae}
{Машины, или Роботы}
{Электронно-световые устройства с базовыми инстинктами живых существ.
Создавались на многих планетах, но наибольшего расцвета достигла Машина Тси-Ди, созданная народом тси.}

\theterm{ngvso}
{Нгвсо}
{Вид, созданный Безымянным, демиургом Тра-Ренкхаля.
Морские обитатели, напоминают осьминогов.
Длина 1,5--2,5 м, чешуя --- крупная оранжевая у самцов, мелкая сине-зелёная --- у самок.
Трилучевая симметрия --- имеются три раздваивающихся щупальца и три простых глаза.
Общаются при помощи непарного гидравлического щупальца --- подают им звуковые сигналы, похожие на барабанную дробь, также используют жестовый язык.
Способны некоторое время пребывать на суше в специальных влагосохраняющих костюмах.
Занимаются выращиванием водорослей, рыбоводством, охотой, собирательством.
Умеют обрабатывать металл.
Были практически поголовно уничтожены дикими стрелохвостами, оставшиеся разрозненные популяции собрались вместе и отгородились насыпью в Коралловой Бухте, где заключили союз с народом ноа.}

\theterm{souzerena}
{Сюзерены, или стрекозодраконы}
{Вид теплокровных летающих рептилий с шестью конечностями, cоздан демиургом Драконьей Пустоши, Кох Свободолюбивой, впоследствии примкнувшей к Ордену Преисподней.
Сюзерены имеют прямую связь с демиургом (молекулярный приёмник в мозгу), что позволяло использовать их как армию в случае вторжения.
После долгой и кровопролитной войны, длившейся тысячу лет (840 земных лет), были вытеснены поселенцами с Земли.
Во времена царствования Валеридов сюзерены обитали в горах Малого Листопада и сохраняли по отношению к людям нейтралитет.
После захвата Адом Драконьей Пустоши, во время номинального правления Скорпидов сюзерены были истреблены.
Есть версия, что Кох Свободолюбивая встала на сторону Ордена Преисподней вследствие шантажа и впоследствии убита.}

\section{Ветви Звезды}

\theterm{star-forks} % Star Forks
{Ветви Звезды}
{Сапиенты с планеты 1-34.
Это вторая известная планета, на которой установилась стабильная самозарождённая жизнь.
Для жизни необходимы вода, метан и температура около 70 градусов Цельсия, выделяют углекислый газ.
Ветвями Звезды занимается особая группа научных и военных отделов Ордена Преисподней.
Классификация видов Ветвей Земли к ним неприменима.
Контактов с Ветвями Земли не зарегистрировано.
Точное число заселённых ими планет неизвестно.}

\section{Ветви Ночи}

\theterm{night-forks} % Night Forks
{Ветви Ночи}
{Неизвестные сапиенты.
Следы их деятельности (техника, обиталища, космические корабли) обнаружены на нескольких удалённых планетах, в том числе на Тси-Ди.
Местонахождение материнской планеты и заселённые ими экзопланеты неизвестны.}

\section{Ветви Пламени}

\theterm{flame-forks}
{Ветви Пламени}
{Высокотемпературные формы жизни.
Несмотря на то, что не обнаружено ни одного сапиентного вида Ветвей Пламени, некоторые исследователи придерживаются убеждений, что теоретически эти Ветви способны дать начало виду с сапиентной архитектурой.}

\asterism

\theterm{aberrants}
{Аберранты}
{Две клады плазмобионтов, живущих в условиях повышенного давления и температуры (в конвективной и радиационной зоне звезды).}

\theterm{flame-of-antares} % Flame of Antares, antaryde
{Пламя Антареса (антарида)}
{Плазменная форма жизни, встречающаяся на некоторых звёздах.
Впервые были обнаружены, как следует из названия, на звезде Антарес ещё первыми людьми, незадолго до взрыва сверхновой (в настоящее время считается, что сверхновая Антареса взорвалась именно из-за деятельности плазменных форм жизни).
Антарида отличается высокой скоростью метаболизма и устойчивостью к воздействию среды по сравнению с обычным огнём.
Модифицированная форма антариды используется как оружие --- в частности, можно настроить их на питание определённым полимером или на присутствие каких-либо веществ в определённой концентрации.
Антарида Безумного бога --- чрезвычайно агрессивная, но неустойчивая форма.
Проектирование и использование антарид запрещено законодательствами Ада и Картеля вследствие непредсказуемости их изменчивости (Оборонительный Кодекс Ордена Преисподней, 12A.0; Politica Of-De, Alb. 1162).}

\section{Ветви Смерча}

\theterm{jorget} % Swirl Forks
{Ветви Смерча}
{Ветви, объединяющие хоргетов и прочие формы жизни, конструктивно сходные с хоргетами.
Некоторые классификации определяют Ветви Смерча как подгруппу Девиантных Ветвей.}

\chapter{Биология тси}

\section{Системный обзор}

\begin{enumerate}
\item \textbf{Мышечная ткань.}
Изменена структура волокон, вследствие чего прочность выше в 8 раз, а сила --- в 3 раза.
Худощавые планты-тси способны переносить веса, непосильные для людей прочих видов, а грузоподъёмность кани-тси сравнима с грузоподъёмностью специальных машин.
\item \textbf{Костная и соединительная ткани.}
Изменена архитектоника балок и волокон, а также их взаимодействия в местах соединений.
Прочность костей на излом выше в 2,7 раз при снижении веса в 1,4 раза.
Прочность сухожилий на разрыв выше в 4 раза.
\item \textbf{Кровеносная система.}
Изменена структура интимы сосудов, снижена турбулентность.
Изменена логистика васкуляризации.
Присутствуют резервные контуры кровоснабжения головного мозга и внутренних органов (так называемые свёрнутые сосуды, не имеющие просвета до открытия специальных клапанов).
\item \textbf{Скелет.}
Изменена форма таза --- смертность и травматизация женщин при родах практически равны нулю.
Изменена форма черепа, присутствуют дополнительные рёбра жёсткости и амортизирующие элементы.
Подобные же изменения в грудной клетке.
\item \textbf{Иммунная система.}
Кардинально отличается от таковой прочих видов и требует отдельного описания.
Тси устойчивы практически ко всем видам микроорганизмов, обитающих на Тси-Ди и Тра-Ренкхаль.
При наличии вакцинации смертность от инфекций равна нулю.
Аллотрансплантация возможна без ограничений, известны успешные случаи ксенотрансплантации --- в частности, у народа ноа есть обычай закрывать раны лоскутами кожи убитых плантов.
На Диком Юге и в пиратских полисах распространена практика украшать тело треугольной мозаикой плантовой кожи с фрагментами татуировок --- по одному треугольнику на каждого убитого врага (лорика).
\item \textbf{Зрительная система.}
Изменена оптическая система, имеется дополнительный хрусталик и фокусирующие спекулумы.
Имеются клетки, воспринимающие ультрафиолетовые волны, инфракрасные волны.
Клетки, воспринимающие гамма-излучение, расположены как в глазах, так и в коже по всему телу.
\item \textbf{Слуховая система.}
Воспринимает частоты от 10 Гц до 80000 Гц.
Также нечто похожее на слуховые аппараты обнаружено в толще эпифизов лучевой и берцовой кости, но роль этих органов пока остаётся неясной.
\item \textbf{Дыхательная система.}
Изменена структура лёгкого, противоточная система позволяет переместить в кровь до 95\% кислорода.
Орган звукопроизводства --- гортанная цитра --- требует отдельного описания.
\item \textbf{Пищеварительная система.}
Изменены органы вкуса и обоняния, большая часть ядовитых веществ распознаются ещё на стадии измельчения пищи.
Зубы способны сменяться в любом возрасте при потере или сильном повреждении.
Кишечник имеет значительно меньшую длину.
Изменены ферменты.
Внутренняя среда стерильна, пища расщепляется на 89--96\% от массы.
\item \textbf{Половая система.}
Способность к партеногенезу и смене пола, зависимые от феромонов.
Пенис у мужчин небольшой, втягивается внутрь.
Имеются зачатки систем обоих полов.
Менструации отсутствуют, овуляция и эволюция эндометрия запускаются присутствием в коре доминанты деторождения, подтверждённой медиаторами мужской спермы.
Беременность длится в 3 раза меньше (данные по людям-тси), имеются данные о впадении плода в анабиоз на срок до 15 лунных месяцев (при болезни матери либо недостаточном питании).
\item \textbf{Нервная система.}
Изменена структура ствола мозга, требует отдельного описания.
Имеются специальные ядра, приспособленные для вычислений в двоичной и троичной логике (у современных тси не используются).
Нервно-мышечные синапсы диафрагмы нечувствительны к курареподобным веществам, имеются дополнительные сплетения для обеспечения дыхания и кровообращения при повреждении ЦНС.
\item \textbf{Регенерация.} Любые внутренние органы, в том числе мозг, способны полноценно регенерировать при сохранном дыхании, кровообращении и питании.
Конечности восстанавливаются при частичной потере (пальцы, кисти, стопы), более грубое повреждение даёт патологическую регенерацию (например, нефункциональные пальцы на культе плеча).
\item \textbf{Психофизиология.}
На заре зарождения цивилизации разные виды вынуждены были специально изучать сигнальную систему друг друга.
Но впоследствии базовые навыки различения эмоций были запрограммированы в каждом тси.
Это хорошо заметно, например,при оценке эмоций апида-тси человеком-тси и человеком Тра-Ренкхаля: человек-тси, не контактировавший прежде с апидами, угадывал эмоциональное состояние в 85\% случаев, против 10\% у человека Тра-Ренкхаля.
\end{enumerate}

\section{Фенотипические группы людей-тси}

У людей-тси, помимо влияющих на фенотип отдельных, существует так называемый псевдорасовый хромосомный регион размером 8 мегабаз, расположенный на хромосоме 16A.
Происхождение его неизвестно;
предположительно он появился в результате ошибки во время ранних экспериментов по перестройке генома тси.
В этом регионе находятся гаплогруппы с генами, оказывающими сильное влияние на фенотип, метаболизм и некоторые психические характеристики.
Кроссинговер у гетерозигот в районе ПРХР16А невозможен.
Всего у людей-тси существовало более ста псевдорасовых гаплогрупп.
После заселения Тра-Ренкхаля из-за эффекта основателя осталось всего пять.

\begin{description}
\item[Аурелийская (по номенклатуре тси: 82.2, 82.3, 82.7)] --- наиболее многочисленная.
Ситрис ар'Эр является гетерозиготой 82.2/82.3, Аурвелий Амвросий --- гомозиготой 82.3.
Отличительные черты гомозиготы по 82 гаплогруппе поэтически описаны у Эрхэ Колокольчик --- <<...твои обсидиановые очи, и камень тёмный кожи загорелой...>>
\item[Сотронская (3.1)] --- гомозиготами по 3.1 является врач Тхитрона Кхатрим и Согхо, воин отряда чести Митхэ ар'Кахр.
Отличительные черты --- прямые светлые волосы (<<цвета мокрой циновки>>), зелено-голубые глаза, пухлые щёки (<<большой младенец>>), точечная пигментация по всему телу (по обидному, но меткому выражению жителей Водораздела, <<как пчёлы обгадили>>).
\item[Валенсийская (45.1)] --- редкая гаплогруппа.
Гомозиготы психически нестабильны, чаще страдают депрессиями и гипоманиакальными состояниями.
На Тси-Ди скрещивание между носителями 45 гаплогруппы не ограничивалось, но им настоятельно рекомендовали использовать для деторождения доноров генетического материала.
Персонажи книги, гомозиготные по 45.1 --- Кхохо ар'Хетр, Акхсар ар'Лотр, предположительно --- Тхартху ар'Катхар и поэтесса Эрхэ Колокольчик.
Отличительные черты --- яркие зелёные глаза, золотистые кудрявые волосы, сильно выступающие скулы и выступающие клыки (<<лицо ягуара>>).
Многие носители валенсии очень привлекательны в сексуальном плане --- в ПРХР16А у них находится <<ген-афродизиак>> (паралог гена одного из факторов иммунной системы, влияющий на запах тела).
Гетерозиготой 45.1/30.3 является Ликхэ ар'Трукх.
\item[Гаплогруппа Страны Целующихся Лесов и Камней (Сикх'амисаэкикх, 30.3, 30.4)]
\item[Гаплогруппа 16.1] --- чрезвычайно редкая.
Гомозиготы по 16.1 на Тра-Ренкхале не обнаружены.
\end{description}

Также существуют отдельные фенотипы, не связанные с ПРХР16А:

\begin{description}
\item[Ген Оцелота (Псевдоген меланопротеида A12)] --- меланопротеид приобретает способность связывать ртуть.
Гетерозиготы имеют преимущество в регионах, богатых ртутью, гомозиготы умирают ещё до имплантации.
Также по неизвестным пока причинам оцелотовый аллель несовместим с гомозиготами по аурелийской гаплогруппе --- у таких людей тяжёлые поражения кожи и нервной системы, они умирают в раннем детстве или юности, не преодолевая порога в 17 дождей.
Гетерозиготой 3.1/30.4 А12 является Тханэ ар'Катхар.
\item[Цикадный фенотип] --- полиэтиологическое полигенное состояние.
Во всех генотипах по ПРХР16А цикадный фенотип равновероятен.
Гетерозиготой 82.7/16.1 CIC является Ликхлам, воин Тхартхаахитра.
\end{description}


\section{Термины}

\theterm{flavor-color}
{Ароматный цвет}
{Длинноволновое ультрафиолетовое излучение, воспринимаемое зрительным пигментом тси (F13, аналог криптохрома) с максимумом поглощения в районе 370 нм.
Название пошло от растения Скальный аромат, цветы которого отражает почти исключительно волны этого диапазона.}

\theterm{throat-cither}
{Гортанная цитра}
{?}

\theterm{stone-fury}
{Каменная ярость (ярость каменных духов, око земли)}
{Экстремальное ультрафиолетовое, рентгеновское и гамма-излучения, воспринимаемые пигментными кристаллами глаз и кожи тси.
Субъективно вызывают у тси сильное неприятное ощущение и чувство страха.}

\theterm{ocelocity}
{Оцелотовость}
{Окраска кожи некоторых групп народа сели, живущих в Пыльном Предгорье.
Кожа оцелотовых людей имеет слабую оранжевую окраску, также на коже имеются характерные пятна, напоминающие пятна оцелота --- ярко-оранжевые в центре и чёрные по краям.
Форма пятен обычно совпадает с зонами иннервации определённых нервов.
Вызвана мутацией одного из генов меланопротеида --- белок приобрёл способность связывать ртуть, содержание которой в водах Пыльного Предгорья сильно повышено.
Без потребления ртутной воды оцелотовые люди приобретают естественную золотистую окраску, но чёрные пятна остаются на всю жизнь.
Гомозиготы --- те, что не умерли в младенчестве --- страдают светобоязнью и витилиго.}

\theterm{differentiation}
{Половая дифференцировка}
{Явление, встречающееся у раздельнополых тси.
Дети тси не имеют пола;
развитие половых признаков начинается только в начале пубертата --- под влиянием окружения, личных предпочтений и прочих факторов.
Близнецы обычно дифференцируют в детей разного пола, дифферецировка в один пол крайне редка (Манэ и Лимнэ ар'Люм --- редкое исключение).
Особи, дифференцировка которых по каким-либо причинам не произошла, называются цикадами.}

\theterm{prolactin-transformation}
{Пролактиновая трансформация}
{Явление, встречающееся у млекопитающих-тси.
Рост груди у обоих полов при стимуляции соска специальным веществом, выделяющимся в ротовой полости новорождённого.
После прекращения кормления в большинстве случаев грудь исчезает.}

\theterm{protocol-crystall}
{Протокол Кристалл}
{Нейрогуморальная система выживания.
Парадигма: <<Выжить.
Возможно, помочь сородичам>>.
Запускается язычным рефлексом (млекопитающие), ногочелюстным рефлексом (апиды).
Модули:
[1] <<Яд>> --- снижение метаболизма и проницаемости тканей, в некоторых случаях --- синтез антидота.
Активируется также при утоплении и попадании в открытый космос.
[2] <<Пламя>> --- выброс влаги, снижение теплопроводности тканей, снижение теплообразования, активация теплоотводящих систем и систем репарации ДНК.
[3] <<Лёд>> --- снижение теплопроводности тканей, увеличение теплообразования, изменение внутриклеточного матрикса (выброс глицерина и спиртов с низкой точкой замерзания).
[4] <<Молния>> --- автоматическая дефибрилляция.}

\theterm{protocol-taifeng}
{Протокол Тайфун}
{Нейрогуморальная система выживания.
Парадигма: <<Помочь сородичам.
Возможно, выжить>>.
Запускается снижением ОЦК более 11\%/мин либо тайфун-кодом, представляющим из себя хеш-сумму биометрических параметров сапиента.
Тайфун-код обязательно должен быть произнесён <<голосом друга>>.
Сели, большая часть которых сохранила этот механизм, но не умела им управлять, называли предсмертный трансовый героизм <<ветром духов>>.
Модули:
[1] <<Золотая минута>> --- спазм повреждённых сосудов, ориентация кровотока на нервно-мышечную систему.
[2] <<Лёгкий уход>> --- обезболивание и снятие негативной эмоциональной реакции.
Выброс в желудочки мозга смеси морфина-44 и эндогенного селективного корректора настроения.
[3] <<Доминанта защиты>> --- активация структур, направляющих последние действия умирающего на помощь сородичам.
[4] <<Живая сталь>> --- синтез сверхпроводящих элементов в нейронах.
Возможно только при наличии инфузионного импланта, требуется специальный препарат.
Мозг способен функционировать ещё 148 секунд после биологической смерти, представляя собой квантовый нейрокомпьютер на сверхпроводниках.
Несмотря на то, что <<Живая сталь>> была в стандартной комплектации всех тси, этот механизм считался весьма негуманным <<методом отчаяния>> и за всю историю использовался всего четыре раза.}

\theterm{silver-veins}
{Серебряные жилы}
{Система проводников, которая оберегает жизненно важные органы при ударе электрическим током.
Выглядит как подкожная венозная сеть, но отличается лёгким металлическим блеском на срезе и отсутствием крови в просвете.}

\theterm{gender-switch}
{Смена пола}
{Явление, встречающееся у раздельнополых тси.
Уже дифференцировавшаяся особь начинает превращаться в особь противоположного пола.
Механизм может запускаться гендерным дисбалансом популяции, травмой, либо по желанию (3 задокументированных случая).}

\theterm{sun-touch}
{Солнечная кисть}
{Средневолновое ультрафиолетовое излучение, воспринимаемое зрительным пигментом тси (F15) с максимумом поглощения в районе 300 нм.}

\theterm{glass-hue}
{Стеклянный оттенок}
{Коротковолновое ультрафиолетовое излучение, воспринимаемое зрительным пигментом тси (F18) с максимумом поглощения в районе 200 нм.
При высокой интенсивности субъективно вызывает у тси неприятное чувство, заставляющее их укрыться от источника.}

\theterm{stigmae-of-pregnancy}
{Стигмы беременности}
{Гипер- или гипопигментированные полосы на шее, показывающие беременность.
Есть у всех видов тси.}

\theterm{tanatosis}
{Танатоз}
{Крайняя степень развития протокола Кристалл, остановка жизненных процессов вплоть до имитации смерти.
Известно, что некоторые тси могли пролежать в состоянии танатоза несколько суток без значительного ущерба для нервной системы.
Обычный танатоз можно было легко определить по неестественно белой коже (<<цветущие подснежники>>), но иногда танатоз настолько хорошо имитировал смерть (вплоть до разложения некоторых тканей), что ошибались даже врачи-тси.
Наиболее точным способом определить танатоз являлась кольцевая теплица: она переваривала трупы, а находящегося в танатозе <<обнимала, не причиняя вреда>>, а иногда даже лечила и приводила в чувство.}

\theterm{warm-color}
{Тёплый цвет}
{Не путать с тёплыми оттенками видимого света Homo homo sapiens.
Инфракрасное излучение, воспринимаемое тепловыми ямками тси.
У сухопутных млекопитающих тепловые ямки находятся в районе угла глаза, над слёзным протоком.
У апид --- в районе ногочелюстного сустава.}

\theterm{transpartenogenesis}
{Транспартеногенез}
{Явление, встречающееся у раздельнополых тси.
В отсутствие феромонов тси своего вида (в изоляции) мужчина-тси превращается в женщину и беременеет.
В отличие от обычной смены пола, процесс транспартеногенеза не затрагивает мужские органы, он гораздо длительнее --- может занимать несколько месяцев, а то и лет;
кроме того, транспартеногенез очень часто обратим, т.е. после родов индивид снова становится мужчиной.}

\theterm{lingual-reflex}
{Язычный рефлекс}
{Специфическая реакция организма людей-тси.
Болевая стимуляция кончика языка вызывает экстренное снижение метаболизма вплоть до танатоза.}

\chapter{Классификация хоргетов}

Хоргеты считаются отличной от сапиентов формой жизни, несмотря на то, что некоторые выделяют их в отдельные Ветви (Ветви Смерча) или даже относят к Ветвям Земли как творение первых людей.

\section{Полярность}

\theterm{polarity}
{Полярность}
{?}

\asterism

\theterm{minus-jorget}
{Минус-хоргет}
{На основе минус-сингулярности ПКВ.}

\theterm{plus-jorget}
{Плюс-хоргет}
{На основе плюс-сингулярности ПКВ.}

\theterm{schmidt-transformation}
{Преобразование Шмидта}
{Изменение полярности неструктурированной омега-сингулярности на противоположную с помощью сдвига волны.
Полярность хоргета (структурированной сингулярности) изменяется путём создания неструктурированной сингулярности и последующего синхронного переноса информации.}

\section{Мобильность}

\theterm{mobility}
{Мобильность хоргета}
{Способность к перемещению в пространстве, обратно пропорциональная количеству масс-энергии.}

\asterism

\theterm{god}
{Бог}
{Стационарный, насыщенный масс-энергией хоргет, обычно заякоривающийся в планете.}

\theterm{daemon}
{Демон}
{Мобильный хоргет со средним насыщением масс-энергией, заякоривающийся в телах сапиентов.}

\section{Заякоривание}

\theterm{incarnation}
{Заякоривание (инкарнация)}
{Способность хоргета встраиваться в планету или сапиента и управлять ими.
Заякоривание преследует несколько целей.
Хоргет --- это сингулярность ПКВ, то есть, условно говоря, во Вселенной Фотона он не существует.
Для взаимодействия с ней ему требуется некий имеющий относительно постоянную структуру омега-источник, перемещение которого в пространстве можно будет отслеживать.
Впоследствии, по мере развития технологий заякоривания, хоргеты научились использовать сапиентный мозг или планетный омега-фон как интерфейс взаимодействия со Вселенной Фотона.}

\asterism

\theterm{anjel}
{Ангел (облачный тип)}
{Тип инкарнации в сапиента, при котором ВНД эмулируется мозгом.
При запросе из мозга включается хоргет, в котором производятся сложные вычисления, обработка информации или принятие решений.
Эффект <<зловещей долины>> отсутствует.
Траты масс-энергии минимальны.
Вычислить такого хоргета в спящем состоянии чрезвычайно сложно.
Обычно спящий хоргет сопровождает своё тело до зрелости, что позволяет ему максимально интегрироваться в личность.
Минусы --- длительность встраивания, излишняя подверженность эмоциям и недостаточная защита от информационного нападения.}

\theterm{bug}
{Жук}
{Очень маленький низкоинтеллектуальный хоргет, использующий сапиентов исключительно для получения эманаций.
Может влиять или не влиять на ВНД сапиента.
Жуки используются богами для сбора масс-энергии, также встречаются свободноживущие жуки, собранные в группы (Рой) или паразитирующие на сапиентах (Клещ).}

\theterm{quasisimbiosis}
{Квазисимбиоз (заякоривание типа <<голос в голове>>, форсированная инкарнация)}
{Тип инкарнации в сапиента, при котором хоргет встраивается в уже сформировавшуюся личность сейхмар (в качестве субличности).
Плюсы --- быстрая смена тел.
Минусы --- сложность встраивания, сапиент может сойти с ума, может активно сопротивляться демону и даже захватить над демоном контроль, интегрировав его личность в себя (2 реально зарегистрированных случая).
Тем не менее в серьёзной боевой обстановке этот способ до сих пор используется, и существуют сборки демонов, специально созданные именно для этого типа заякоривания.}

\theterm{zombie}
{Марионетка (зомби)}
{Тип инкарнации в сапиента, при котором хоргет берёт под контроль нервные стволы или верхние отделы спинного мозга.
ВНД сапиента эмулируется в хоргете, мозг в процессе не участвует, органы чувств сапиента не используются.
Первый и наиболее примитивный тип.
Траты масс-энергии значительны.
Марионетки легко вычисляются не только хоргетами, но и другими сапиентами (эффект <<зловещей долины>>).}

\theterm{lonely-god}
{Одинокий бог}
{Тип заякоривания в планете, при котором сингулярность только одна, все действия выполняет самостоятельно.
Бог тратит много энергии на перемещение.}

\theterm{spider}
{Паук (планшетный тип)}
{Тип инкарнации в сапиента, при котором хоргет берёт под контроль последние корковые нейроны.
Органы чувств сапиента используются, но ВНД эмулируется в хоргете.
Траты масс-энергии меньше, чем у марионеток.
Таких сапиентов сложнее вычислить хоргетам, эффект <<зловещей долины>> сведён к минимуму за счёт участия в движениях стабилизирующей системы сапиента (мозжечка и красных ядер у млекопитающих).}

\theterm{swarm}
{Рой}
{Чрезвычайно опасный вид бога.
Состоит из равноправных <<жуков>>, соединённых в сеть.
В силу низкого интеллекта интенсивно использует сапиентов (вплоть до полного их вымирания), вызывает массовые психозы и эпидемии.
Быстро размножается, но часто гибнет сам.}

\theterm{snowflake}
{Снежинка}
{Тип заякоривания в планете, при котором есть одна главная сингулярность и множество короткоживущих жуков, выполняющих функции сбора масс-энергии и преобразования материи.
Наиболее экономичный тип бога.
Название пошло из-за сходства проекции со снежинкой --- для экономии масс-энергии жуки перемещаются по планете согласно фрактальной маршрутной карте.}

\section{Цель сборки}

\theterm{compilation-purpose}
{Цель сборки хоргета}
{Цель, с которой хоргет создавался.
Несмотря на некоторую архаичность этого деления, оно используется до сих пор из-за сильных различий указанных типов в плане психологии.
В частности, урождённые боги могут страдать от последствий акбаса и хиторай, а }

\asterism

\theterm{born-god}
{Урождённый бог}
{Сборка хоргета сапиентами, не преследующая цели сделать хоргета членом общества и не предусматривающая развития его как личности.
Урождённые боги часто индивидуалисты, склонны к самоуничтожению, могут страдать от разнообразных последствий акбаса.}

\theterm{born-daemon}
{Урождённый демон}
{Сборка хоргета свободными хоргетами с целью сделать его членом сообщества хоргетов.}

\theterm{born-sapient}
{Урождённый сапиент}
{Оцифровка биологической (самообразовавшейся) нейронной сети свободными хоргетами с целью сделать его членом сообщества хоргетов.
Урождённые сапиенты часто не воспринимают себя членами сообщества хоргетов, могут содержать самые разнообразные дефекты и особенности настройки в зависимости от взрастившей их культуры.}

\section{Распределение вычислительных мощностей}

\theterm{powers-distribution}
{Распределение вычислительных мощностей}
{Фундаментальный параметр хоргета, который говорит о его профессиональных способностях.
Вычислительные мощности распределяются между тремя характеристиками --- чувствительность (способность собирать и структурировать информацию извне), интеллект (способность преобразовывать информацию и решать поставленные задачи) и устойчивость (способность сопротивляться воздействиям извне, в том числе информационному нападению).

Классы по распределению мощностей --- условное деление, каждый хоргет обычно распределяет собственные мощности так, как ему требуется.
Соотношение может изменяться и в ходе работы, из-за подключаемых модулей.
Единственный нюанс --- подключение модулей редко изменяет класс.
Изменение же класса, т.е. пересборка --- задача, требующая длительного анализа, всегда проводится специалистами.
Пересборка урождённых сапиентов невозможна в силу особенностей их внутренней структуры.}

\asterism

\theterm{visor}
{Визор (оракул)}
{Высокая чувствительность, средний интеллект, низкая устойчивость.}

\theterm{saboteur}
{Визор-интерфектор (диверсант)}
{Высокая чувствительность, очень низкий интеллект, средняя устойчивость.
Редко встречающийся тип, однако незаменимый при выполнении чётко поставленной задачи.}

\theterm{interfector}
{Интерфектор (воин)}
{Низкая чувствительность, средний интеллект, высокая устойчивость.}

\theterm{cogitor}
{Когитор (стратег)}
{Низкая чувствительность, очень высокий интеллект, низкая устойчивость.}

\theterm{scientist}
{Когитор-визор (учёный)}
{Чрезвычайно редкий в настоящее время тип.
Ранее такая специализация, как следует из названия, встречалась в основном среди научного персонала.
Устойчивость околонулевая.
Один из самых знаменитых представителей --- Сиэхено Опаловый Глаз.}

\theterm{tactic}
{Когитор-интерфектор (тактик)}
{Околонулевая чувствительность, средний интеллект, очень высокая устойчивость.
Тип, часто встречающийся среди младших воинских чинов.}

{\theterm{effector}
{Эффектор (универсал)}
{Средняя чувствительность, средний интеллект, средняя устойчивость.}

\section{Специальности (классификация Ордена Преисподней)}

\theterm{biology}
{Биология}
{Создание, изменение, восстановление и исследование биологических (природных) систем.}

\theterm{dictiology}
{Диктиология (культурология, сетевая технология)}
{Создание, изменение, восстановление и исследование сетей --- схем коммуникации между биологическими и/или небиологическими системами.}

\theterm{interfection}
{Интерфекция}
{Междисциплинарная отрасль науки и технологии, основной задачей которой является эффективное разрушение либо нарушение работы биологических и небиологических систем и сетей при минимальном воздействии на окружающую инфраструктуру.
Несмотря на то, что сам термин в первую очередь используется военными, методы интерфекции используются во всех сферах деятельности --- в производстве пищевых материалов, медицине, строительстве, машиностроении и программировании.}

\theterm{informatics}
{Информатика}
{Работа с любыми видами информации.}

\theterm{technology}
{Технология}
{Создание, изменение, восстановление и исследование небиологических систем (использующих искусственно созданные принципы работы).}

\chapter{Имена}

\section{Имя демона}

В раннем Ордене Преисподней имена демонам давались следующим образом:

\begin{itemize}
\item Кодовое имя бога, данное людьми.
Самый распространённый способ у древних демонов.
Так получили имена Грейсвольд Каменный Молот, Эйраки Мороз, Кох Свободолюбивая.
\item Позывной, который использовался при внедрении в сапиентное общество.
Так получили имена Ангара Краснобуря, Сиэхено Опаловый Глаз, Тако из клана Дорге.
\item Имена, которые имели оцифрованные до оцифровки.
\end{itemize}

Традиция прозвищ пришла немного позже.
Все историки сходятся во мнении, что первыми прозвища в их современном понимании стали давать в Ордене Тысячи Башен.
Тысяча Башен отличается многоязычностью --- на Друзах существует огромное количество изолированных племён;
переводимые прозвища стали в таких условиях необходимостью.

Забавный факт --- также именно на Тысяче Башен зародилась как таковая методология биодиктиологии (культурологии) ветви Люди, биолингвистика и дипломатия в её современном понимании.
Благодаря изолированности Друз --- как культурной, так и генетической --- Тысяча Башен представляет собой неплохую модель мультипланетного объединения.
По утверждению историков, выжить без этих наук на кристаллической планете было просто невозможно.
Впоследствии именно из Ордена Тысячи Башен вышли видные военные дипломаты и учёные, занимающиеся изучением биологии и социума сапиентов --- в частности, Стигма Чёрная Звезда и Север Солнечная Дева.

Названия демонов, по принятой ныне классификации, имеют структуру:

\textbf{Имя Прозвище из клана Клан (Исходное название) (Версия)}

\textbf{Имя}: краткое слово на Эй-B0, используется в небоевой обстановке.

\textbf{Прозвище}: от одного до трёх слов, которые могут быть переведены на практически любой язык.

\textbf{Клан} (необязательно): название демона, ядро или части ядра которого использовались при создании.

\textbf{Исходное название} (необязательно): название, данное демону разработчиками-сапиентами или имя сапиента.

\textbf{Версия} (скрыто): версия ядра демона согласно Реестру Ордена Преисподней.
Версии являются засекреченной информацией, поэтому в этой книге опущены.

\section{Кланы}

Кланом называется группа урождённых демонов, собранных на основе ядра основателя.
Группы, собранные на основе разных форков ядра, называются клановыми сериями, генерациями или субкланами.
Также есть понятие псевдоклана --- демоны, собранные на основе одного ядра, но впоследствии покинувшие его (либо изгнанные) и примкнувшие к другим веткам обновлений.

У каждого клана есть свои представители в Совете Глобальных Обновлений, занимающемся совместимостью обновлений с ядром клана.
Представители строго неприкосновенны и дают клятву не вмешиваться в политику фракции.
Любая попытка воздействовать на представителя клана может спровоцировать клановую войну.
Демоны из псевдокланов чаще всего пользуются индивидуальными обновлениями, созданными специально для них на основе открытого кода, предоставляемого Советом.

\begin{description}

\item[Мороз] --- один из первых кланов.
Родовая планета --- Преисподняя.
Основа Ордена Преисподней.
Основатель --- Эйраки Мороз.
Клановые серии --- первая, вторая и четвёртая генерации (третья считается провальной, она не дала ни одного стабильного демона).
Псевдоклан --- Туман, входит в Красный Картель.
Девиз --- <<Считай шаги>>.

\item[Дорге] --- один из первых кланов.
Родовая планета --- Преисподняя.
Разделился на два, одна часть которого, Левые Дорге, примкнула к Союзу Воронёной Стали, вторая, Правые Дорге, --- к Ордену Тысячи Башен.
Основатель --- Жерар Дорге.
Одна клановая серия.
Псевдоклан --- Улыбающиеся, в составе Красного Картеля.
Девиз --- <<Нас достаточно>>.

\item[Антрацис] --- родовая планета --- Чёрная Скала.
Основатель --- Лев Зелёный, демиург Чёрной Скалы, впоследствии Вечно Гонимый, погиб во время взрыва на Запах Воды.
Три клановые серии --- Долина Смерти, Жёлтое море и Белая Чаща.
Псевдоклан --- Вечно Гонимые (в составе Ордена Преисподней).

\item[Тахиро] --- родовая планета --- Капитул.
Основатель --- Тахиро Молниеносный.
Две клановые серии --- Тахиро Старейшины и Тахиро Вторые.
Два псевдоклана --- Унылая Когорта (Картель) и Серые Змеи (нейтралы).

\item[Усмане] --- родовая планета --- Земля Врачевателей.
Основатель --- Усмане Белое Одеяние.
Одна клановая серия.
Псевдоклан --- Скопцы, служит Красному Картелю.

\item[Вечность] --- объединение, по сути превратившееся в клан.
Основатели --- Семеро Бессмертных.
Тринадцать клановых серий, сорок слияний.
Два дочерних клана --- Чёрная и Красная Вечность (Орден Преисподней и Красный Картель соответственно).
Псевдоклан --- Платина (нейтралы).

\item[Оньё] --- один из девяти кланов, основавших Орден Тысячи Башен.
Некоторые считают его псевдокланом, потому что идеологией Оньё является выживание любой ценой --- члены клана с готовностью предают друг друга, если им грозит опасность, и не осуждают друг друга за это.
Тамга --- беременная коза.
Девиз --- <<будь осторожен>>.

\end{description}

\chapter{Персоналии}

\theterm{ainu}
{Айну Крыло Удачи (Айну)} % Ajnu the Luckwing
{Урождённый бог, создана на Древней Земле, демиург безымянной, ныне не существующей планеты в системе Сириуса.
Биотехнолог-интерфектор, максим секунда Ордена Преисподней.
Любовница Тахиро Молниеносного.
Прозвище Крыло Удачи получила после битвы с Чук Тьма Над Горой --- молодая демоница уничтожила превосходящую по опыту и мощи противницу, найдя в её обороне крошечное окно.
Ей так понравилось это прозвище, что она очень часто появлялась на собраниях с нашивкой или значком в виде белого птичьего крылышка.
Погибла на Тысяче Башен в битве с превосходящими силами Красного Картеля.}

\theterm{akchsar-ar-lotr}
{Акхсар ар’Лотр э’Сотрон (Седой-Дедушка-В-Снегу, Снежок)}
{Друг Митхэ ар'Кахр, воин отряда чести, впоследствии --- хагпот племени Инхас-Лака.
Даритель Имжу Тенебоя.
Любовник и спутник Кхотлам ар'Люм.
Кормилец Согхо и Ликхсара ар'Люм.}

\theterm{ancarjal} % Ancarjal the Bloodstorm (the Redbreeze)
{Анкарьяль Кровавый Шторм (Ангара Краснобуря, при рождении Тальяна Древолаз)}
{Урождённый демон, создана на Капитуле.
Биотехнодиктиолог-интерфектор, легат терция Ордена Преисподней.
Использует прозвище-позывной, данное ей людьми на планете Тысяча Башен (от \theorigin{хольский говор}{Angara}{море}, распространённого в то время женского имени).
Краснобуря --- ключевая крепость на друзе Хербст.
Пятая Мостовая война (она же Осенняя) на Тысяче Башен была для демоницы дебютом и первой серьёзной победой.}

\theterm{arcadiju} % Arcadiju the Womb Jackal (the Tomb Jackal, Thalian Wolf)
{Аркадиу Шакал Чрева (Аркадиу Валериану Люпино, Талианский Волк, Падальщик)}
{Урождённый сапиент, преобразован на Драконьей Пустоши Яйвафом Солёная Борода.
Биодиктиолог-когитор, легат терция Ордена Преисподней.
Был взят в плен и перешел на сторону Ордена Преисподней после Развязки Тринадцати Звёзд.}

\theterm{aurweli} % Aurweli Amwurosi
{Аурвелий Амвросий}
{Воин отряда чести Митхэ ар'Кахр.
В прошлом --- пират ноа, Тенетник из Гавани Неудачников.}

\theterm{wind-curling-hair} % Wind-Curling-Hair
{Ветер-Завивает-Волосы}
{---}

\theterm{jalo}
{Гало Кровавый Знак из клана Мороза}
{Урождённый демон, создан Эйраки Морозом, вторая генерация клана.
Впоследствии --- глава псевдоклана Туман, Третий Дурак Красного Картеля.
Предположительно погиб на планете Запах Воды во время диверсии Ланс-Ната Алмаза.}

\theterm{fool}
{Глупец}
{Персонаж легенд и сказок сели.}

\theterm{grejsvolt} %  Grejsvolt the Stonehammer
{Грейсвольд Каменный Молот (Грисволд-2)}
{Урождённый бог, создан Лабораторией Дж.\,Грисволда (научный город Цикаго-2, домен Северная Америка, Древняя Земля).
Демиург планеты Лотос системы Фомальгаута.
Технодиктиолог-эффектор, легат прима Ордена Преисподней.
Прозвище Каменный Молот придумал сам --- чтобы никогда не забывать, с чего началась технология.}

\theterm{greta}
{Грета (Якоб Морген Йеннифер Шахтшнайдер)}
{---}

\theterm{siblings} % Siblings
{Двойняшки}
{Прозвище демонов Фуси Абрикосовый Посох и Нуива Пустая Тыква.}

\theterm{long-moist-tail}
{Длинный-Мокрый-Хвост (Хвост)}
{Химик-технолог из города 10, живший примерно за 1200 оборотов до Катаклизма.
Автор видеоканала <<Проблема переработки хвостов>>, изначально посвящённого насущным вопросам химической промышленности, а потом превратившегося в философски-юмористический.
Цитаты Хвоста становились крылатыми.
Он несколько раз --- два раза при жизни и более двадцати раз после смерти --- признавался одним из величайших тси.
Видеозаписи его канала считаются культурным достоянием цивилизации.}

\theterm{du-xi} % Du-Sie the Hunter, of the White Thicket, of Antracis Clan
{Ду-Си Охотник, уроженец Белой Чащи, из клана Антрацис}
{Урождённый демон, член младшей из трёх генераций демонов Чёрной Скалы.
Совершил побег с Чёрной Скалы после нападения на Фу-Си Двойняшку, стал одним из Вечно Гонимых --- бессрочных преступников, подлежащих уничтожению.
Примкнул к Ордену Преисподней во время мирного времени, предшествовавшего первой войне между Орденом Преисподней и Красным Картелем.
Служащий отдела 125, после его роспуска перешёл в отдел 100.}

\theterm{dourgue-junior} % Gerard Dourgue
{Жерар Дорге Младший}
{Урождённый бог, глава клана Дорге, ближайший соратник Арракиса Мороза.
Был вероломно убит членами своего клана вскоре после побега на Тысячу Башен.}

\theterm{dourgue-senior} % Gerard Dourgue
{Жерар Дорге Старший}
{---}

\theterm{flask} % Sealed-Life-Flask
{Закрытая-Колба-Жизни (Баночка)}
{Инженер сетей из города 12.
Внешность: немного полный, круглолицый, с широкими губами.
На голове и боках Баночки были вертикальные шрамы, которые остаются у мужчин-плантов после процесса смены пола (следы отмершего венчика и крыльев).}

\theterm{imzhu}
{Имжу Тенебой, он же Имжу Лжец, он же Имжу Слепец}
{Герой народа хака, вождь племени Инхас-Лака, который пренебрег решением Союза Племён и двинул свои силы на помощь сели в борьбе с Безумным богом.
Один из трёх вождей хака, которые участвовали в битве на Могильном берегу, и единственный, который выжил.
Впоследствии был обвинён старейшинами в измене и вынужден бежать из своего племени в земли ноа.
Там он нашёл себе мужчину по имени Фелис и приютил сироту по имени Сильвия Акрвила.
Вернулся к Спокойному озеру спустя тридцать дождей, когда Союз Племён наконец признал, что Безумный действительно был уничтожен.
Остаток жизни Имжу, который к тому времени почти ослеп от длительной болезни, прожил вместе со своим многочисленным семейством в святилище Одинокий Столб, окружённый всеобщей любовью и уважением.}

\theterm{jokull}
{Йокудль Гуннарссон Глазенапп}
{---}

\theterm{katarina} % Katarina Kosulja
{Катарина Козуля --- кузничая, инженер, мастер ковки хука.}

\theterm{celsa}
{Кельса (Цельсия) Пушистая}
{?}

\theterm{colbe} % Colbe the Old Maxim
{Кольбе Старое Изречение}
{Один из двух братьев-биологов, созданных в рамках эксклюзивного проекта на Тысяче Башен.
Был назван в честь Максимилиана Кольбе, героя времён Последней Войны.
Несмотря на принадлежность к Ордену, соблюдает нейтралитет и не особенно это скрывает.}

\theterm{candy} % Candy
{Конфетка (Эрликх ар’Фа э’Самитх, Сладкая-Ягодная-Конфета)}
{Воин Тхитрона.
Тренер Ликхмаса.
Участник похода в Тиши.
Воин отряда чести Мёртвые Стервятники.}

\theterm{corjes}
{Корхес Соловьиный Язычок (Лючжоу Соловей, уроженец Долины Смерти, Картезий (Корхес) Нактергаль}
{Урождённый демон, была создана Львом Зелёным.
Вечно Гонимая.}

\theterm{kurz}
{Курц Пламя Осени (Курц Штайгер)}
{---}

\theterm{biter}
{Кусачка (Шестнадцать-Сила-Четыре-Укус-Три-Молния)}
{Травник из Бродячего Народа, член отряда чести Митхэ ар'Кахр.
Отвечал за бухгалтерию и коммуникации с торговцами Бродячего Народа.
Погиб в бою во время штурма Трёхэтажного Храма.}

\theterm{kcharas}
{Кхарас ар’Хитр э’Хатрикас (Любовью-Свитая-Веревка, Верёвочка)}
{Боевой вождь Тхитрона.
Участник битвы на Могильном берегу.
После битвы ушёл из Храма и стал ткачом.
До самой смерти прожил со своим другом Эрликхом в одном жилище.}

\theterm{kchatrim}
{Кхатрим ар’Сар э’Тхонтротрис (Ветка-Растущая-Из-Листа, Веточка)}
{Врач Тхитронского Храма, уроженец Водораздела.
Погиб во время тхитронской диверсии.
Внешность: среднего роста, крепкого телосложения, длинные распущенные седоватые волосы, сотронская борода (аутотрансплантаты с затылка на скулы, губы и подбородок), широкий плосковатый нос, серо-зелёные глаза, знак Снежной Обители на правом запястье.}

\theterm{kchotlam}
{Кхотлам ар’Люм э’Кахрахан (Испачканное-Мёдом-Перо, Пёрышко)}
{Купец Тхитрона, участница битвы на Могильном берегу, кормилица Саритра, Ликхмаса, Манэ, Лимнэ, Согхо и Ликхсара ар'Люм, близкая подруга Митхэ ар'Кахр.
Начинала торговцем в землях ноа.
Сотрудничала со Стервятниками, среди которых была её родственница Эрхэ ар'Люм.
Один раз дала бой пиратам, но попала в плен и убедила их сохранить ей жизнь в обмен на улаживание конфликта между пиратскими полисами, а впоследствии смогла вернуть судно и вызволить часть команды.
Шторм в Могильном проливе потопил её корабль вместе с командой и товарами.
Была спасена Хитрамом ар'Кхир, который впоследствии стал её мужчиной и дарителем троих её питомцев.
Карьеру дипломата начала в Кахрахане, откуда была изгнана по обвинению в воровстве.
Впоследствии Митхэ ар'Кахр нашла доказательства невиновности Кхотлам, и горожане попросили её вернуться на прежнюю должность.
Кхотлам отклонила предложение и осела в Тхитроне, где стала помощницей купца, а затем, после ухода купца на покой --- и купчихой Тхитрона.
Спасла от смерти Кхохо ар'Хетр, совместно с Кхарасом и Хитрам выкрав её из темницы святилища Одинокий Столб и став первым дипломатом в истории, вмешавшемся в отправление правосудия святилища.
Взяла на воспитание Ликхмаса, ребёнка Митхэ ар'Кахр, а также наложила вето на решение Советов Тхитрона, сняв с Ситриса ар'Эр звание кутрапа, обеспечив ему возможность жизни в землях Тхитрона и впоследствии введя его в Тхитронский Храм.
После налёта на хутор Самитх приютила выжившего полубезумного старика, который оказался известным в прошлом охотником Сиртху.
Была обвинена в организации налёта на Самитх, взята под стражу и около двух декад провела в комнате для ожидающих жертвоприношение, так как ни один жрец в Тхитронском Храме не решался принести её в жертву;
отклонила предложение Первого жреца организовать её побег.
Ситрис, Кхохо и Трукхвал, которые не поверили обвинителям, смогли при помощи Сиртху доказать её невиновность, выследив угнанный обоз и найдя документы, в которых шла речь о её устранении с поста дипломата.
Впоследствии Кхотлам ни разу не упоминала своё время пребывания в Комнате Без Окон и отказывалась говорить на эту тему даже с близкими людьми;
также известно, что после этого инцидента она до конца жизни не переступила порог храма, предпочитая все встречи с храмовниками устраивать во Дворе.
Кхотлам руководила обороной города во время нашествия Молчащих идолов.
Была вынуждена убить своего питомца Саритра во время боя на вырубке, чтобы он не попал в плен и не был принесён в жертву.
Благодаря её своевременному вмешательству был заключён союз с хака, а Молчащие идолы атакованы на марше объединёнными силами сели, хака и ноа.
Способствовала приходу Людей Золотой Пчелы в Тхитронский Храм, а также приняла участие в дипломатической переписке Тхитрона с Тхартхаахитром о разделении сфер влияния.
Пережила три покушения на свою жизнь в связи со своей миротворческой деятельностью, от одного из которых она до конца жизни не могла говорить громко --- нож убийцы задел гортань и голосовую цитру.
Во время войны с Безумным Кхотлам подготовила Тхитрон к осаде, переманила на сторону сели несколько родов идолов, смогла организовать переговоры с Имжу Тенебоем, а также провела разведку в лагере Картеля, внеся существенный вклад в победу Ордена Преисподней.
Сражалась в битве на Могильном берегу плечом к плечу со своей подругой Митхэ ар'Кахр.
После гибели Хитрама на Могильном берегу и окончания войны Кхотлам вернулась в Тхитрон с Акхсаром ар'Лотр и прожила с ним до своей смерти, родив ещё двоих хранителей.
Акхсар и Кхотлам умерли в один день --- в подвале обнаружилось гнездо смертожаров, Акхсар решил его выжечь, а Кхотлам, поняв, что Акхсара ужалила змея, попыталась своими силами вытащить его наружу.
Оба получили более двадцати укусов и погибли на месте.
Впоследствии долгая и богатая на события жизнь Кхотлам была описана её питомицей, Манэ Нарисованной, в романе <<Змеиная яма>>.}
%{Snake Pit}

\theterm{kchoho} % Coal
{Кхохо ар'Хетр э'Митракрас (Умойся-Угольной-Водой, Уголёк)}
{Воин Тхитрона.
Охотница за головами на Могильном берегу.
Известная бардесса.
Дарительница Ликхэ ар'Трукх.
Погибла в битве на Могильном берегу.

Внешность: среднего роста, крепкого сложения, грязные волосы, торчащие в разные стороны, рыбки плетёные в два хвоста.
Глаза тёплого каре-зелёного цвета.
На подбородке татуировка в виде пяти клиньев, слева шрам типа <<улыбка Глазго>>.
На груди татуировка <<двукрылое солнце>> (знак провокаторов, стягивающих внимание противника на себя), на дельтах --- осьминоги, щупальца заходят на лопатки, ключицы и обвивают плечи.}

\theterm{lida}
{Лида Салехрад (Лидия Карина Ольговиц Кохани)}
{---}

\theterm{lican}
{Ликан Безрукий (Аликсандур бен Курба бен Миса-Оле)}
{Урождённый человек с планеты Земля Врачевателей, старейший из живых урождённых людей.
Родился без верхних конечностей, но научился писать ногами, дослужился до главного каллиграфа и переводчика местного правителя.
Благодаря своим способностям к наукам был замечен агентами Ордена Преисподней и подвергнут демонизации в возрасте двадцати двух лет.
Впоследствии --- лингвист, информатик, создатель языка Эй.}

\thesynonim{arcadiju}
{Ликхмас ар’Люм э’Тхитрон (Играющая-Булыжником-Лиса, Лис)}
{Аркадиу Талианский Шакал}

\theterm{likchoe} % Nut-nut
{Ликхэ ар’Трукх э’Тхинат (Обёрнутый-Платком-Орех, Орешек)}
{Воин Тхитрона.
Участница битвы на Могильном берегу.
Впоследствии вошла в Кахраханский Храм и стала его боевым вождём.
В преклонном возрасте увлеклась поэзией, перевела на сели значительную часть сохранившегося поэтического наследия тси, участвовала в археологических раскопках, которые проводил Орден Преисподней в Кахрахане.
Найдена мёртвой в своей келье после того, как отказалась выполнять приказ представителя Ордена Преисподней о слежке за некоторыми горожанами Кахрахана.

Внешность: среднего роста, немного полная, с большими, чуть обвисшими грудями и тяжёлыми округлыми ягодицами.
Глаза яркие тёмно-серые, медиальный конец правой брови иссечён.
Волосы золотистые светло-каштановые, обрезаны до подбородка, одна плетёная <<рыбка>> на затылке.
Носит блестящие золотые серёжки с сапфирами, справа окаймляющие ухо полностью.
Татуировка на загривке: Кхар-защитник с очень большими, хорошо прорисованными глазами и надпись на цатроне <<Я тебя вижу>>.
Имеет привычку крепко прижимать правую руку к груди, если в руке ничего нет.}

\theterm{lusafejru} % Lusafejru the Feather Palm
{Лусафейру (Люцифер) Лёгкая Ладонь из клана Мороз}
{Урождённый демон, информатик-когитор, максим секунда Ордена Преисподней.
Создан на Преисподней.
Назван в честь известного физика эпохи Последней Войны --- Люцифера Гафт-Йенковски, предсказавшего открытие первичного поля.
Прозвище получил за тактику минимального вмешательства в действия младших по званию демонов.

Внешность (Преисподняя): очень худой, большие раскосые глаза, чувственные губы, вздёрнутый маленький нос, женственное лицо, завитые в кудри крашеные волосы, множество различных татуировок на лице и теле, макияж.}

\theterm{wajermann}
{Людвиг Карл Рагнар Вейерманн}
{Первооткрыватель омега-поля, лауреат Расширенной Нобелевской премии (319 Эпохи Богов).
Любовник и друг Михаэля Кохани.
Жил и работал в городе Кёнигсберг, домен Европа, Древняя Земля.
На момент открытия ему было 22 года.
Впоследствии занялся просветительской деятельностью --- его перу принадлежат учебники по Теории Всего, которые использовались ещё более пятисот лет после первой публикации и пережили триста восемь редакций.
Также Вейерманн известен как поэт и композитор, сочинявший лирические песни на языке джерман.
Покончил жизнь самоубийством в возрасте 56 лет по неизвестной причине.}

\theterm{little-pier}
{Манис ар'Сотакх э'Самитх (Столб-Качающейся-Жизни, Столбик)}
{Друг юности Ликхмаса ар'Люм.
Участник битвы за Тхитрон.
Ушёл на север, к Ледяной Рыбе, вместе с Трукхвалом ар'Со.
После битвы на Могильном берегу отказался вернуться вместе с остальными сели, принял обычаи северных племён и прожил всю оставшуюся жизнь охотником.
Уже в пожилом возрасте был найден учёными Ордена Преисподней и стал важным источником данных о языке и культуре племён Серебряных гор и Старой Челюсти.}

\theterm{mimosa} % Mimosa the Silken Steel
{Мимоза Шёлковая Сталь (Мимозе Зайденешталь, Богиня Весны)}
{Урождённый бог, демиург Потерянной Тверди, максим терция отдела 100.
В прошлом был использован как раб для мелиорации на Нимб-3.
Сбежал, устроив техногенную катастрофу на планете, при этом перебив высшее руководство клана Нимб.
Поступил на Тысячу Башен легионером, дослужился до командира батальона.
Задолго до падения Тысячи Башен стал осведомителем Ордена Преисподней, что обеспечило ему быстрый карьерный рост.
В отделе 125 познакомился с Ду-Си Охотником, с которым Мимоза всю оставшуюся жизнь поддерживал дружеские отношения.
Один из стратегов, командовавших силами Ордена в Развязки Десяти Звёзд.
После диверсии на Запах Воды вместе с Ду-Си был принят в отдел 100 и в кратчайшие сроки стал одним из ключевых его руководителей.
Вступил в ряды Скорбящих, что обеспечило движению определённую защиту от контрразведки.
После раскрытия бежал вместе с Ду-Си на Тси-Ди.}

\theterm{mitlikch}
{Митликх ар'Митр э'Травинхал (Четыре-Звенящих-Тетивы, Тетива)}
{Учитель Тхитронского Храма.
В отличие от прочих жрецов, обучение прошёл уже в зрелом возрасте, до этого был мастером музыкальных инструментов.
Вышел из Тхитронского Храма после диверсии и уехал на Могильный Берег, где его следы окончательно теряются.}

\theterm{king-priest-mitris}
{Митрис ар’Люм тэ’Со'латр (Сложил-Башню-Курочек, Башенка)}
{Король-жрец, любовник Хатлам ар'Мар.
На правой руке носил знак Снежной Обители.}

\theterm{nameless} % the Nameless
{Митрис Безымянный (Безымянный, он же Атрис, и Митхэ ар'Кахр)}
{Двойной хоргет, объединяющий урождённого бога и урождённого сапиента.
Божественная часть была предположительно создана на Лотосе, демиург планеты Тра-Ренкхаль.
Человеческая часть родилась на планете Тра-Ренкхаль, оцифрована божественной частью и впоследствии интимно интегрирована в ядро.

Внешность Митхэ: маленького роста, щуплая, короткие черные кудрявые волосы с маленькими <<рыбками>> в три бусины, яркие светло-зелёные глаза.
Рваный шрам на щеке, губе и подбородке справа, татуировка Плачущего Ягуара.
Отсутствует правая грудь, во рту сломан верхний резец.}

\thesynonim{nameless}
{Митхэ ар’Кахр э’Тхартхаахитр] (Золотая-Крупица-На-Дороге, Золото)}
{Митрис Безымянный}

\theterm{kojani}
{Михаил Алексеевич Коханый (Михаэль Кохани)}
{Первооткрыватель омега-поля, лауреат Расширенной Нобелевской премии (319 Эпохи Богов).
Любовник и друг Людвига Вейерманна.
Жил и работал в городе Кёнигсберг, домен Европа, Древняя Земля.
На момент открытия ему было 19 лет.
Впоследствии оставил науку и стал авиаконструктором.
Известен как изобретатель <<спичечного самолёта>> --- надёжного летательного аппарата, который мог собрать любой человек при наличии самых простых инструментов.
Умер в возрасте 98 лет после продолжительной болезни.}

\theterm{michael}
{Михаэль Пауль Хелена Салехрад}
{---}

\theterm{nuuwa} % Nuuwa the Empty Pumpkin
{Нуива Пустая Тыква}
{---}

\theterm{paul}
{Пауль Херманн Салехрад}
{---}

\theterm{cupcake}
{Пирог-Вечных-Странников (Пирожок)}
{Сестра Хрустально-Чистый-Фонтан.}

\theterm{sunflower} % Sunflower-Dropping-Seeds
{Подсолнух-Бросает-Семена (Подсолнух)}
{---}

\theterm{heather} % Bee-Sniffing-Heather
{Пчела-Нюхает-Вереск}
{Близкая подруга Темнотой-Сотканный-Заяц, технолог пищевой промышленности из города 12, автор <<Десяти тысяч блюд>>, главной кулинарной книги народов Короны и Кита.
Предположительно покончила жизнь самоубийством вскоре после завершения книги.}

\theterm{rabe} % Rabe the Young
{Рабе Юный}
{Один из двух братьев-биологов, созданных в рамках эксклюзивного проекта на Тысяче Башен.
Был назван в честь Йона Рабе, героя времён Последней Войны.
Несмотря на принадлежность к Ордену, соблюдает нейтралитет и не особенно это скрывает.}

\thesynonim{grejsvolt}
{Сакхар ар’Сатр э’Ихслантхар (Плывущий-Кронами-Карп, Карп)}
{Грейсвольд Каменный Молот}

\theterm{samajolu}
{Самаолу Каменный Старик (Самаэл) из клана Мороза}
{---}

\theterm{sevjer} % Sevjer the Sun Maid (Salvator Solm\"{o})
{Север Солнечная Дева (Сальватор Сольмё)}
{Урождённый демон, биодиктиолог-визор, центурион запаса Ордена Преисподней.
Декларативный пацифист.
Наставник и близкий друг Ликана Безрукого.
Глава отдела 1028, изучающего Ветвь Хуманы, практически с момента основания Ордена по настоящее время.}

\theterm{sirtchu-lechoe}
{Сиртху ар'Мар э'Тхисем (Дом-Карамельные-Окна, Домик)}
{---}

\theterm{sitris}
{Ситрис ар’Эр э’Тхинат (Коснувшийся-Дна-Колодца, Донышко)}
{Воин Тхитрона.
Погиб на Могильном берегу.

Внешность: среднего роста, крепкого сложения, кудрявые чёрные волосы, собранные на затылке в три пучка трубочками, трубочки стянуты верёвкой (Кахрахан).
Глаза матовые чёрные.
На лбу слева короткий вертикальный шрам.
На левой руке татуированный <<рукав>> со звездчатыми мотивами.
В ушах серьги-кольца с игольчатыми подвесками --- стиль пиратов-ноа.
Спина испещрена шрамами от хлыста, из-за чего рубаху снимает редко даже перед сном.}

\theterm{siejeno} % Siejeno the Opal Eye
{Сиэхено Опаловый Глаз (Ксения Харпер из клана Дорге)}
{---}

\theterm{stijma} % Stijma the Blackhole
{Стигма Чёрная Звезда (она же Цзаошан Уродливая, уроженка Жёлтого моря, она же Анастейша Розье)}
{Урожденный демон из клана Антрацис, одна из Вечно Гонимых.
Согласно мнению ряда специалистов, внесла решающий вклад в уничтожение клана Чёрной Скалы --- за время её деятельности численность клана сократилась со ста двадцати трёх до пяти демонов, считая её, Ду-Си, Корхес Соловьиный Язык и Двойняшек, служащих Красному Картелю.
Близкая подруга Лусафейру Лёгкая Рука.}

\theterm{sky} % Existing-Good-Sky
{Существует-Хорошее-Небо (Небо)}
{Инженер-тси из города 12.}

\theterm{chhanei}
{Таниа Янтарь (Тханэ ар’Катхар э’Тхаммитр)}
{Урождённый сапиент, нейтрал, биотехнолог-интерфектор.
Погибла на Лотосе.

Внешность Чханэ: очень высокого роста, оцелотовая окраска, чёрное пятно на лбу справа, оранжевые матовые глаза, слегка опалесцирующие.
Волосы жёсткие, волнистые, кофейного цвета с рыжиной, валяные рыбки в три хвоста --- два голубых, один белый.
Шрам крест-накрест на левой щеке, на животе --- шрам от кхаагатра, мелкие шрамы на бёдрах.}

\theterm{weaver}
{Танцуют-Четыре-Камня (Ткач)}
{Трёхногий травник, персонаж-трикстер из фольклора Бродячего Народа.
Согласно легендам травников, Танцуют-Четыре-Камня был уроженцем Горы Песнопений.
Его поселение было выжжено Безумным, и ребёнок, чтобы спасти от голодной смерти своих родичей, засунул ногу в ткацкий станок.
Станок окрасился его кровью и с тех пор всегда производил только дорогостоящую красную ткань, что помогло поселению пережить голодные времена.
Но Безумный пришёл в ярость от того, что кто-то посмел смягчить наложенное им наказание, и заговорил Танцуют-Четыре-Камня на короткую жизнь.
Тогда Ткач научился перемещаться между параллельными жизнями, <<словно нить утка между разноцветными нитями основы>>.
Его душе тысячи дождей, он пережил огромное количество приключений, но ни в одной реальности не дожил даже до возраста инициации.
В Легенде об обретении Танцуют-Четыре-Камня появляется в <<Бамбуковой клетке>> и описывается как старик --- это указание на необычность, сверхъестественность описываемого места;
в постоялом дворе на перекрёстке трёх дорог Безумный власти не имеет.}

\theterm{tajiro} % Tajiro the Thunderbolt
{Тахиро Молниеносный (Тахиро та Ханаяма)}
{Урождённый сапиент, биотехнодиктиолог-когитор, максим терция Ордена Преисподней.
Основатель клана Тахиро.
Любовник и близкий друг Лусафейру Лёгкая Рука, Айну Крыло Удачи.
Преобразован Лусафейру Лёгкая Рука.
Прозвище созвучно имени (Tahio --- <<сверхбыстрый>>).}

\theterm{hare} % Darkness-Woven-Hare
{Темнотой-Сотканный-Заяц (Заяц)}
{Инженер тси из города 12.}

\theterm{king-priest-trukchual}
{Трукхвал ар’Со э’Тхартхаахитр}
{Жрец Трёхэтажного Храма, Король-жрец.
Внешность --- длинные прямые чёрные волосы, чёрные глаза, треугольное лицо.}

\theterm{teacher-trukchual}
{Трукхвал ар’Хэ э’Тхартхаахитр (Звенит-Костяной-Колокольчик, Звоночек)}
{Учитель Ликхмаса.
На правом запястье --- знак Снежной Обители.}

\thesynonim{chhanei}
{Тханэ (Чханэ) ар’Катхар э’Тхаммитр (Змея-Похожая-На-Шнурок, Змейка)}
{Таниа Янтарь}

\thesynonim{dancing-shadow}
{Тхартху ар’Хэ э’Тхартхаахитр (Подражает-Птичьему-Пению, Птичка)}
{Тхартху Танцующая Тень}

\theterm{dancing-shadow}
{Тхартху Танцующая Тень (Тхартху ар’Хэ э’Тхартхаахитр)}
{Урождённый сапиент, нейтрал, визор Скорбящих.
Убита на Тра-Ренкхале во время диверсии Темнотой-Сотканный-Заяц.}

\theterm{fuxi}
{Фуси Абрикосовый Посох}
{---}

\theterm{jarata}
{Харата Шёпот Горы}
{---}

\thesynonim{ancarjal}
{Хатлам ар’Мар э’Тхартхаахитр (Глаза-Похожие-На-Вишню, Вишенка)}
{Анкарьяль Кровавый Шторм}

\theterm{nurse-chitram}
{Хитрам ар’Кир э’Тхитрон (Иногда-Любит-Плавать, Пловец)}
{Кормилец Ликхмаса ар'Люм.}

\theterm{chitram-warrior}
{Хитрам ар'Эр э'Виа-Марина (Ирис Виндемиа Молчунья)}
{Воин Тхитрона, наполовину ноа.
Любовница Кхараса.
Погибла во время похода на хака.
Особенности внешности: очень высокая и худая женщина, почти без груди, кудрявые чёрные волосы, разноцветные глаза (левый чёрный, правый зелёный), татуировка в виде креста на ключице, два лоскута лорики на груди, мизинец имеет две фаланги, на ноге два сросшихся пальца.
Имеет привычку сидеть, вжав голову в плечи и скрестив руки на груди, производит впечатление угловатой, но в бою очень ловка, изворотлива и изящна.}

\theterm{fountain} % Crystal-Pure-Fountain
{Хрустально-Чистый-Фонтан (Фонтанчик)}
{Программист-тси из города 12.}

\theterm{blooming-poppy-bush}
{Цветущие-Маковые-Кусты (Мак)}
{Биолог-тси из города 14.}

\theterm{stroji} % Stroji the Smoke Ring
{Штрой Кольцо Дыма}
{---}

\theterm{ejraci} % the Senseless
{Эйраки Мороз (Арракис-1, Безумный бог Тра-Ренкхаля)}
{Урождённый бог, информатик-когитор, якобы основатель Ордена Преисподней.
Основатель клана Мороз.
Создан Лабораторией Омега-преобразований Малаги (стратегический научный центр Малага, домен Европа, Древняя Земля).}

\theterm{oerlikch}
{Эрликх ар'Мас э'Кхихутр (Разрезать-Старое-Полотенце, Полотенце)}
{Воин Тхитрона.
Участник битвы на Могильном берегу.
После битвы вышел из Храма и стал ткачом.
Прожил со своим другом Кхарасом в одном доме до самой смерти Кхараса, после отправился в качестве наёмника в путешествие в Западную Корону, в земли ркхве-хор, где его следы теряются окончательно.}

\theterm{jaivaf}
{Яйваф Солёная Борода из клана Дорге}
{Урождённый демон из Левых Дорге.}

\section{Сапиенты}

\subsection{Сели (обновить)}

\begin{description}

\item[Акхсар ар’Катхар э’Тхаммитр] (Случайно-Задушил-Змею, Случай) --- отец Чханэ

\item[Кхарам ар’Хэ э’Тхартхаахитр] (Забавно-Скачущая-Пружинка, Пружинка) --- слуга Тхартху

\item[Ликхэ ар’Хэ э’Тхитрон] (Огонь-Прошедшего-Рассвета, Огонёк) --- служанка дома Люм
\item[Лимнэ ар’Люм э’Тхитрон] (Укусила-Кошку-За-Сосок, Кусачка) --- сестрёнка Ликхмаса
\item[Манис ар’Ликх э’Тхаммитр] (Цветок-Пахнущий-Домом, Цветочек) --- погибший мужчина Чханэ

\item[Манэ ар’Люм э’Тхитрон] (Пролитое-Утром-Молоко, Молочко) --- сестрёнка Ликхмаса

\item[Саритр ар’Люм э’Тхитрон] (Вечно-Хмурится-Без-Причины, Хмурый) --- умерший брат Ликхмаса
\item[Сатракх ар’Сит э’Тхартхаахитр] (Зверёк-Вылез-Из-Кладовки, Зверёк) --- жрец, любовник Тхартху
\item[Сатхир ар’Со э’Тхаммитр] (Затаившийся-В-Кровати-Крокодил, Крокодил) --- жрец, попытавшийся убить Чханэ
\item[Сиртху ар’Митр э’Сотрон] (Домик-С-Карамельными-Окнами, Домик) --- старый слуга дома Люм
\item[Ситлам ар’Со э’Тхаммитр] (Зачем-Молоток-Взял, Молоточек) --- жрец, брат Сатхира

\item[Согхо ар’Хэ э’Тхартхаахитр] (Лишённый-Голоса-Журавль, Журавлик) --- дарительница Чханэ
\item[Согхо] --- воительница отряда чести


\item[Тхартху ар’Катхар э’Травинхал] (Две-Зелёных-Бусины, Бусинка) --- прародительница Чханэ


\item[Хонхо ар’Лотр э’Тхитрон] (Ящерица-Пишущая-На-Стенах, Ящерка) --- умерший учитель-жрец

\item[Эрси ар’Митр э’Тхитрон] (Приручили-Котёнка-Оцелота, Котёнок) --- ребёнок, принесённый в жертву
\item[Эрхэ ар’Люм э’Сотрон] (Слишком-Много-Ест, Обжорка) --- любовница Акхсара
\item[Эрхэ ар’Сит э’Тхитрон] (Зачатая-В-Ночной-Реке, Речка) --- служанка дома Люм
\end{description}

\subsection{Прочие}

\begin{description}
\item[Анатолиу Сильбеу Тиу] (t-sl: Anadoliv Cilbev Tiv) ---
\item[Бенедикт Альсауд] [Чёрная Борода] --- капитан дальнего плавания, а впоследствии капитан космического корабля <<Тёмное пламя>>, перенёсшего первую волну колонистов на планету Лотос.
Родился в Бристоле, домен Европа, Древняя Земля.
Получил известность благодаря огромной находчивости и развитой интуиции --- его мгновенные приказы при авариях и отказе оборудования впоследствии разбирались целыми комиссиями, так как сам он не всегда мог их объяснить.
Прозвище Бенедикта Альсауда --- Чёрная Борода или Король морей (намёк на его знатное происхождение --- Бенедикт является прямым потомком последнего короля Саудовской Аравии).
Альсауд имел такой авторитет, что его кандидатуру на должность капитана <<Тёмного пламени>> предпочли прочим, более подготовленным теоретически.
Умер в возрасте 61 года во время второй исследовательской экспедиции на Лотосе от яда шипастой камбалы.
\item[Валериу X Валерид] (t-sl: Baleriv Dekad Balerid) ---
\item[Клавдиу Дентосиу] [Пересмешник] (Klavdiv Dendosiv, ?--345 по календарю Талиа) --- поэт и наёмный убийца, герой народных историй (трикстер).
Настоящее его имя неизвестно, и большая часть деятельности скрыта мраком времён.
Единственный надёжный источник о нём --- очерки <<Воспоминания о великом тёзке>> Клавдиу Семито.
Известно, что Клавдий Дентосий начал свою деятельность разбойником.
Впоследствии по непонятным причинам он покинул разбойничью шайку и стал вести жизнь наёмного убийцы.
Его <<клиентами>> были в основном богачи, которые терроризировали бедных.
Деньги Клавдий тратил на проституток или раздавал беднякам.
Странная приверженность Клавдия продажному сексу в итоге сыграла с ним злую шутку --- он умер от остановки сердца в постели с блудницами.
В народе шутили, что любвеобильный красавец просто не рассчитал свои силы, но историки склоняются к версии, что он был отравлен подосланной шпионкой.
\item[Клавдиу (Клауше Шмол) Семито Фризский] (t-sl: Klavdiv Semido fra'Teris) --- придворный новеллист Валериу X, потомок древнего рода писателей и учёных.
Известен в основном как автор исторических хроник, в том числе очерка <<Воспоминания о великом тёзке>>.
Семито был убеждённым монархистом, но личность Дентосия настолько впечатлила его, что в преклонном возрасте, уже после смерти короля, он написал посвящённый поэту труд.
\item[Март Джонатан Митчелл] [Одноглазый Март, М.\,Дж.\,М.] --- идеолог Эволюциона, писатель.
В возрасте 32 лет был помещён в камеру смертников за разбой и многочисленные жестокие убийства, но в силу некоторых обстоятельств не был казнён.
В камере Митчелл провёл 20 лет.
За это время он получил три высших образования --- химика-технолога, философа и инженера-строителя.
Находясь в камере, он удалённо работал по специальности, вёл просветительскую деятельность.
Единственное, что так ему и не далось --- правописание.
Несмотря на многочисленные петиции о помиловании, 52-летний Митчелл был казнён путём расстрела --- согласно его последнему желанию.
Последними его словами были: <<Смешанное чувство.
Я не могу принять организованное насилие, но придумать для себя другую кару тоже не могу>>.

\item[Леам эб-Салах эб-Сайед ала-Фариз] (t-sl: Lejam Salagid Saedid fra'Teris) ---

\item[Софиа Ловиса Карма] ---
\item[Татиан Сергеу Анно] [Освободитель] ---
\item[Хервар Лонгсин-Храш] ---
\item[Юле Алексевиц Гагарин] [Сокол Последней Войны] --- один из первых людей, выживших в космическом пространстве (согласно некоторым данным, самый первый выживший).
Его жизнеописания утеряны, но хорошо известно, как он выглядел --- портреты Юле гравировали на многих космических кораблях на удачу.
Среди историков планеты Лотос был в ходу фразеологизм <<улыбка Гагарина>> --- незначительная подробность события, оставшаяся в истории вместо более важных данных.
\end{description}

\chapter{Словарь терминов}

\thesynonim{order-of-netherworld}
{Ад}
{Орден Преисподней}

\theterm{jacbas} % jacbas
{Акбас (\theorigin{sd}{hiki-ba-yasu}{остаться в одиночестве})}
{Первое ощущение осознающей себя высокоинтеллектуальной системы.
Акбас может стать причиной помешательства вплоть до самоуничтожения.
В обществах сапиентов или высших животных акбас предотвращается заботой родителей, собратьев или других близких существ.}

\theterm{amulet-of-san}
{Амулет Сана}
{Предмет экипировки жреца сели.
Представляет собой дисковидный бронзовый резервуар со знаком Сана-сновидца, разделённый на четыре части.
Каждая часть имела собственный клапан и содержимое.
Белый Сан --- успокоительное (масло мяты), Красный Сан --- наркотическое обезболивающее (масло молитвенных маков), синий Сан --- парализатор (эссенция кураре), чёрный Сан --- яд (лаковый сок).
Амулет был снабжён механизмом, позволяющим вращение в любую сторону.
Жрецы сели очень ловко крутили его в руках, и психически неустойчивые больные (дети и старики) часто не знали, какой Сан им достанется.}

\theterm{bamboo-birdcage} % Bamboo Birdcage, Memoryless Inn
{Бамбуковая клетка}
{У сели: легендарный Беспамятный постоялый двор для духов на перекрёстке трёх дорог в Ихслантхаре.
У тси: устройство для захвата и удержания хоргета.}

\theterm{warchief} % warchief
{Боевой вождь}
{---}

\theterm{wandering-temple} % the Wandering Temple
{Бродячий Храм}
{Храм, не привязанный к определённому городу.
Мог иметь один или два Этажа.
Пример <<одноэтажного>> Бродячего Храма --- Люди Золотой Пчелы.
Также Бродячими Храмами считаются отряды чести.}

\theterm{upstairs} % the Upstairs
{Верхний Этаж}
{Жрецы на службе поселения, часть городского Храма.}

\thesynonim{protocol-taifeng}
{Ветер Духов}
{Протокол Тайфун}

\theterm{jaruspic}
{Гаруспик (\theorigin{t-sl}{haruspic}{чревовещатель})}
{В государстве Талиа: оракул, якобы предсказывающий будущее под действием наркотического яда гриба \textit{Harus sanguinatus}.
Гаруспиками часто становились беспризорники или выходцы из бедных семей.
Это позволяло им и их семьям вести сытую и безбедную жизнь, но имело свою цену --- мало кто из гаруспиков доживал до двадцати лет.}

\theterm{throat-bomb}
{Глоточная бомба}
{Орудие убийства на Преисподней, крохотный (2--3 мм диаметром) шарик со взрывчатым веществом, который подкладывался в кислую еду, чаще всего в традиционный кислый соус.
При резком изменении кислотности среды происходила детонация, обычно в глотке или в верхних отделах пищевода.
Мощности заряда хватало как минимум на обильное, часто не совместимое с жизнью кровотечение --- при попадании под зубы;
при удачном попадании глоточная бомба убивала мгновенно, разрушая шейные позвонки, спинной мозг и сосуды шеи.}

\theterm{humanization}
{Гуманизация}
{В широком смысле: обучение демона управлению сапиентным телом (либо его голографическим аналогом). В узком смысле: обучение демона управлению сапиентным телом до момента успешной тьюринг-социализации.}

\theterm{giver} % giver
{Даритель}
{---}

\theterm{house} % the House
{Двор}
{Купец-дипломат на службе поселения.}

\theterm{demiurge} % demiurge
{Демиург}
{Хоргет, создавший планету.}

\theterm{daemonization}
{Демонизация}
{Оцифровка нервной системы сапиента с последующим её превращением в программное ядро демона.
Наиболее сложная и наукоёмкая разновидность оцифровки.
Включает в себя стабилизацию (устранение или замедление естественной изнашиваемости), коррекцию личности, интеграцию ядра с управляющим интерфейсом модулей хоргета.
Демонизация гораздо проще создания ядра демона с нуля, из-за чего оцифровка сапиентов приобрела большую популярность.}

\theterm{stray-lantern}
{Дорожный бумажный фонарь}
{Походный фонарь, который используется повсеместно в Северной Короне.
Его создание приписывается Сату-скитальцу, легендарному путешественнику-хака.
На сотронском диалекте сели он называется фонарём Сата, на южном --- фонарь хака.
Хака, которые считают Сата недостойным упоминания изгоем, называют устройство тхитронским или кахраханским фонарём.
В землях ноа распространено название <<рыбий пузырь>> или <<рыбья уда>>, а также <<джунглевый фонарь>>.
Бродячий Народ называет фонарь маяком миролюбия, потому что используется он только мирными путниками.
Тенку и ркхве-хор без зазрения совести приписывают изобретение фонаря себе, называя его <<светлячок Кошачьей тропы>> и <<фонарь Кувшинковой реки>>.}

\theterm{fume-fan}
{Дымный веер}
{---}

\theterm{silva-veins}
{Жилы джунглей}
{Корни некоторых деревьев (баньян благородный, бальса оружейная), которые простираются на большие расстояния и временами выходят на поверхность.}

\theterm{brick-sign}
{Знак кирпича}
{---}

\theterm{canet-constantae} % Ca'Net constantae
{Ка'Нета константы}
{Две фундаментальных физико-технических константы хоргета.
Большая --- максимальное количество масс-энергии, которое может удержать хоргет на стабилизирующем модуле. Малая --- минимальное количество масс-энергии, необходимое для преодоления радиуса Ка'нета.
Чем больше у хоргета масс-энергии, тем сложнее ему перемещаться в пространстве и тем больше масс-энергии тратится на перемещение.
Именно поэтому боги --- самые богатые масс-энергией хоргеты --- чаще всего стационарны, а МК : БК = 1 : 10069.4.
Названы в честь первооткрывателя --- Улада Ка'Нета, одного из разработчиков хоргетов Эпохи Богов.}

\theterm{canet-radius} % Ca'Net radius
{Ка'Нета радиус}
{Максимальное расстояние, на которое способен переместиться хоргет без подзарядки (примерно 18 тысяч парсак).}

\theterm{c-map}
{Карта переправы}
{Браслет с храповым механизмом с выбитыми в глине знаками, используемый на Тысяче Башен для навигации во время переправы.
Храповый механизм издавал щелчки через определённое время;
знаки на <<кости>> обозначали воздушные потоки, направление и время парения.
Очень часто переправляться приходилось в тумане или в тёмное время суток.
Поэтому человек на переправе мог полагаться только на собственное осязание, вестибулярный аппарат и звуковые часы.}

\theterm{kila}
{Кила (\theorigin{tn}{kila}{пришить рукав})}
{Ритуал, направленный на <<изменение>> свершившегося факта.
Существовал в разных формах практически у всех племён людей Тра-Ренкхаля.
Например, в племени сели, если человек умер от медленной мучительной болезни, жрец совершал ритуал кила, обрезая свитую родичами человека верёвку --- это символизировало быструю безболезненную смерть.
У идолов Живодёра, если кто-то получил психологическую травму, шаман племени <<переписывал>> его историю.
Во время совершения кила обычно рядом присутствует всё племя (у сели --- жители поселения и наблюдатели из крупных городов), после совершения ритуала все присутствующие считают истиной сказанное жрецом и поступают в соответствии с этим.
Преступившие кила считаются лжецами и клеветниками.}

\theterm{kihotr}
{Кихотр (\theorigin{tn}{kihotr}{игровой камень})}
{Двадцатигранный камень для игры Метритхис.
Согласно легенде, Безумный использует для принятия решений кихотр в виде идеально круглого шара, в котором бесконечное число граней и нулевая вероятность выпадения любой из них.
Таким образом, нельзя даже сказать наверняка, влияет ли кихотр Безумного на события.
Это является намёком на абсолютную случайность и бессмысленность действий бога.
В переносном смысле кихотр (чаще с негативной интонацией) --- невероятное стечение обстоятельств.}

\theterm{microclimate-suit}
{Комбинезон-микроклимат}
{На Тси-Ди: костюм, собирающий и перерабатывающий выделения организма.
Позиционировался как изобретение, сохраняющее время для важных дел вместо физиологических отправлений;
однако было показано, что тси, долгое время носившие костюм, в конце концов оказывались неспособны без него существовать.
Впоследствии от этой технологии отказались, а выражение стало нарицательным для технологических средств, входящих в мутуалистические отношения с сапиентом.}

\theterm{condensor}
{Конденсатор}
{На Тра-Ренкхале: устройство, предназначенное для поддержания оптимальной влажности в помещениях.
Представляет собой волосяной психрометр, запускающий систему охлаждения.
Система охлаждения бывает электрическая (эфирный или спиртовой контур) или простая (серебряная пластинка, опущенная в наполненный снегом сосуд).}

\theterm{con-tici} % Con-Tici
{Кон-Тики}
{Один из Трёх Кораблей, которые привезли сапиентов на планету Тси-Ди.
Кон-Тики стартовал с Земли, команда состояла из кани, людей и дельфинов.
Два других корабля --- Стальной Дракон (привёзший вымершие Ветви Ночи) и Свергнутый С Небес, чья команда состояла из колониальных апид и плантов-рабов (не дошёл до времён Тараканьей Войны, известен по хроникам апид Ди).}

\theterm{nurse}
{Кормилец (кормилица)}
{---}

\theterm{priest-king} % Priest-king
{Король-жрец}
{Жрец-дипломат, имеющий право разрешать конфликты между городами, а также вести переговоры от имени народа сели.}

\theterm{scarlett-cartel} % Scarlett Cartel
{Красный Картель}
{Крупнейшая организация минус-хоргетов.
Основана была демонами планет Нимб-3, Океан и Чёрная скала.
Впоследствии Красный Картель вобрал в себя такие крупные объединения минус-хоргетов, как Союз Воронёной Стали и Вечность.
Тамга --- три круга в треугольнике (три красных гиганта --- солнца союзных планет).}

\theterm{crushwood}
{Крушина (боевое бревно)} % Crushwood, or warlog
{Оружие западных сели.
Отдавалось самым сильным членам отряда.
Бревно было очень эффективно в парировании утяжелённых пылеройских палиц, разбивании строя Скорпионов, выламывании дверей и форсировании небольших рек.
Боец с крушиной назывался крушинником. % crushader
}

\theterm{kukchuatr}
{Кукхватр (\theorigin{tn}{kukchuatr}{сумеречная сталь})}
{Деструктурированный стабитаниум или титановый сурроганиум, высоко ценившийся народами Тра-Ренкхаля как металл для инструментов и оружия.
Слитки кукхватра ходили в обороте на объём золота, а в некоторых регионах и выше.
Согласно одной из версий, название пошло из-за тёмного искрящегося налёта полимера (<<звёздное небо>>), который образовывался на поверхности стабитаниумового расплава.}

\theterm{trader} % trader
{Купец}
{---}

\theterm{kurschack}
{Куржак, кургак (\theorigin{саркорт}{kurgak}{сухой, высушенный})}
{На Тысяче Башен: порошок для сушки рук и улучшения сцепления при скалолазании.
Обычно хранится в бойтеле.}

\theterm{biting-paper} % biting paper
{Кусачая бумага}
{Грязная нитроцеллюлоза, используемая народами Тра-Ренкхаля как взрывчатое вещество для разработки месторождений.}

\theterm{kutraph}
{Кутрап (\theorigin{tn}{kutraph}{мутная голова})}
{У народа сели: человек, совершивший четыре акта Насилия, Насилие и Разрушение, либо два акта Разрушения.
Кутрапам при поимке предоставлялся выбор: либо казнь, либо служение на благо народа --- смерть на алтаре.
В случае, если кутрап выбирал алтарь, с него посмертно снимали все обвинения (обряд кила), а его семье или людям, которых он указал, выплачивался жертвенный выкуп.}

\theterm{kchenoe-distance}
{Кхене (\theorigin{tn}{kchenoe}{быстрый бег})}
{Единица измерения расстояния, примерно 2500 метров.}

\theterm{kchenoe-time}
{Кхене (\theorigin{tn}{kchenoe}{время бега})}
{Единица измерения времени, примерно 8,5 минут.}

\theterm{swallow-niche} % swallow niche
{Ласточкины ниши}
{Традиционный элемент храмового здания сели.
Представляет собой ниши в перекрытиях верхнего этажа со стороны зала.
Если ласточки и прочие мелкие птицы начинали умирать, покидали ниши или отказывались в них гнездиться --- это считалось плохим знаком и город немедленно приходил в режим готовности к разного рода несчастьям.
Наиболее вероятная гипотеза появления обычая --- птицы первыми реагировали на вулканические испарения, антарид и МПДЛ, который появлялся в воздухе при радужном безумии.}

\theterm{legate} % legate
{Легат}
{Ранг в Ордене Преисподней и Красном Картеле, находящийся выше центуриона и ниже максима.
Соответствует деймо в раннем Ордене Преисподней.
Делился на три уровня: легат терция, легат секунда и легат прима.
Символ легата в Ордене Преисподней --- перевёрнутый топор.}

\theterm{silva-spirits}
{Лесные духи}
{Пантеон маленьких божеств, которые, согласно верованиям сели, обитают в жилах джунглей.
Всего у народа сели насчитывается 124 лесных духа.
Каждый дух имеет тотемные дерево, животное и насекомое.
Отличие лесных духов сели от божеств соседних племён заключается в том, что они \emph{всегда} стремятся доставить человеку радость и облегчить его страдания.
Лесные духи не способны сердиться, обижаться и мстить;
страдания человека объяснялись или прихотью Безумного, или усталостью и занятостью лесного духа.
После повсеместного насаждения культа Безумного вера в маленьких добрых божков стала уделом простонародья.}

\theterm{lechoe}
{Лехэ (\theorigin{tn}{lechoe}{имеющий много детей})}
{У царрокх: уважительное обращение к старейшине племени.
У сели: уважительное обращение к любому пожилому человеку.}

\theterm{limnoe}
{Лимнэ (\theorigin{tn}{limnoe}{щитки для глаз})}
{У сели: камуфляжная сетчатая повязка на глаза, маскирующая яркие глазные яблоки в темноте.
У ноа: солнцезащитные очки, вырезанные из белой кости акулы-буйвола, с узкими прорезями для глаз.}

\theterm{lockheed}
{Локхейд}
{Скалолазный карабин, входящий в состав хука.
Вероятно, слово произошло от названия организации, производившей скалолазное оборудование.}

\theterm{maccsim} % maccsim
{Максим}
{Высший ранг Ордена Преисподней и Красного Картеля.
Деление на уровни присутствует, но оно условное, допустимо опустить уровень при именовании --- любого носителя ранга максима можно назвать просто <<максим>> без нарушения субординации.
Ранг максим прима в настоящее время вакантный в Ордене Преисподней, первый и последний, кто его носил --- Эйраки Мороз.
Ранг максим секунда носят члены клановых советов по обновлениям и ведущие военные стратеги.
Ранг максим терция носят многие руководители научных и технических отделов.}

\theterm{laughing-manipula}
{Манипула Смеха}
{Исторически: псевдоклан истребленного клана Мара.
Тамга --- улыбающийся зубастый рот и плачущие глаза.
В современном Красном Картеле: отдел, занимающийся мелиорацией планет, а также карательными операциями против сапиентов и нейтралов.}

\theterm{manoe}
{Манэ (\theorigin{tn}{manoe}{краска для глаз})}
{Чёрная или тёмно-зелёная твёрдая тушь с добавлением пуха или шерсти, используемая воинами сели для наведения тени на яркое глазное яблоко.}

\theterm{mbs}
{Маркерные поведенческие стереотипы (МПС)}
{---}

\theterm{melioration}
{Мелиорация планетарная}
{Процесс подготовки планеты к превращению в источник масс-энергии.
Включает коррекцию климата, уничтожение враждебных Девиантных Ветвей, Роев, а также биологическую и культурологическую обработку сапиентов.
В Ордене Преисподней мелиорацией занимался отдел 125, в Красном Картеле --- Манипула Смеха.}

\theterm{dead-nash-equilibrium}
{Мёртвое равновесие Нэша}
{Гипотетическое неразрушимое равновесие группы бессмертных игроков в бесконечной игре на замкнутом игровом поле.}

\theterm{metritchis}
{Метритхис}
{Игра.}

\theterm{micorget} % micorget
{Микоргет}
{Гипотетический неуничтожимый и способный существовать без сапиентов хоргет.}

\theterm{mikchan-mirrors}
{Микханское стекло}
{Инструмент для оценки состава металла, используемый оружейниками Северной Короны.
Состоит из хрустальной призмы, системы зеркал и магниевой лучины.
Для кукхватра существовал строжайший запрет на добавление каких-либо примесей; кукхватр плавили в очень дорогих графитных тиглях.
Тем не менее, из-за большого числа перековок примеси появлялись и в кукхватровых слитках.
Микханское стекло позволяло произвести примитивный (в зоне видимого света) спектральный анализ куска металла.
Предположительно, инструмент был изобретён идолами исчезнувшей Микханской культуры, а затем распространился по всему обитаемому континенту.}

\theterm{aerostatic-world}
{Мир-аэростат (висячие сады, мир газового океана)}
{Класс суперземель, у которых пригодным для жизни является слой атмосферы, не примыкающий к твёрдой (жидкой) поверхности.
На таких планетах подавляющее число живых существ обладает приспособлениями для воздухоплавания (микроскопические --- выросты, макроскопические --- крылья, аэростатные органы).
На некоторых планетах аэростатные существа могут достигать внушительных размеров (летающие деревья);
такие планеты пригодны и для сапиентов Ветвей Земли.
Самый известный из населённых миров-аэростатов --- планета Карнак в системе Фомальгаута.}

\theterm{frost-porcelain}
{Морозный фарфор}
{Фарфор с Грозового Хребта, изготавливаемый по утерянной технологии.
На фарфоровую посуду и фигурки с помощью светочувствительной краски, системы линз и зеркал наносились проекции снежинок или морозные рисунки, появляющиеся на хрустальных пластинах.
Каждый предмет имел свой неповторимый узор и очень высоко ценился племенами Тра-Ренкхаля;
морозный фарфор ходил не ниже чем на объем кукхватра, даже если в фарфоре имелись трещины и сколы.}

\theterm{mpdl}
{МПДЛ (моронгозинпентозодифосфолизергинат, <<Агент 80>>, <<Фурия>>)}
{Психоактивное вещество с ярко выраженным аффективным действием (вызывает агрессию и страх).
Использовалось ещё во времена Древней Земли спецслужбами и нечистыми на руку политиками --- предположительно, писательница Мариам Кивихеулу была растерзана толпой под действием распылённого в воздухе МПДЛ.}

\theterm{arrowtail-dress} % arrowtail dress (belt & waistcoat)
{Мундир стрелохвостовый}
{Подобие одежды у стрелохвостов из обработанной акульей кожи.
Состоит из жилета и кушака.
Жилет имеет несколько карманов для мелких вещиц, зев карманов обращён против направления движения.
На жилете также выбиваются отличительные знаки.
Кушак защищает хвост от травм при чересчур резких манёврах, а также имеет венчик из лиственницы и акульих зубов, служащий грозным оружием в схватках.}

\theterm{violation} % violator
{Насилие}
{---}

\theterm{downstairs} % the Downstairs
{Нижний Этаж}
{Воины на службе поселения, часть Храма.}

\theterm{omega-field}
{Омега-поле}
{---}

\theterm
{order-of-netherworld}
{Орден Преисподней} % sd: afir so’tid, sl: Ord Inferna
{Крупнейшее в известной Вселенной объединение плюс-хоргетов.
Возникло на планете Преисподняя, отличавшейся сильной вулканической активностью.
Тамга --- извергающийся вулкан.}

\theterm{unit-100}
{Отдел 100}
{В Ордене Преисподней: отдел, включающий контрразведку и военный трибунал.}

\theterm{unit-125}
{Отдел 125}
{В Ордене Преисподней: отдел, занимающийся мелиорацией планет и борьбой с враждебными богами.
В настоящее время распущен.}

\theterm{squad-of-honor}
{Отряд чести}
{Бродячий Храм, состоящий только из воинов (Нижний Этаж).
Отряды чести сопровождали колонистов, торговцев и беженцев.
К моменту войны с Безумным большая часть отрядов чести превратилась в наёмные войска, несмотря на то, что многие продолжали называть себя Храмами.}

\theterm{digitalization}
{Оцифровка}
{Процесс, заключающийся в считывании квантовой структуры объекта, составлении математической модели и последующий перенос модели в память хоргета.
Оцифровка сапиентной нервной системы называется демонизацией.}

\theterm{parsac}
{Парсак (\theorigin{sd}{parsak}{этимология неизвестна})}
{Единица измерения космического пространства, равная расстоянию, пройденному светом по однородному вакууму за 1 астрономический год планеты Преисподняя.}

\theterm{head-priest}
{Первый жрец}
{Номинальный глава Храма.}

\theterm{mockingbard}
{Пересмешники}
{Менестрели-тси, голосом подражающие музыкальным инструментам, крикам зверей, пению птиц и шумовым звукам.
Исполняют Песню Закрытых Глаз, состоящую из различных звуков, сложенных в особый <<дорожный>> ритм.
Обычно поют парами --- пока один переводит дух, второй подхватывает песню.}

\theterm{nurseling}
{Питомец}
{---}

\theterm{crying-jaguar}
{Плачущий Ягуар (\theorigin{tn}{kch\`{o}h\^{o}}{плачущий ягуар})}
{Воин, следующий философии, подразумевающей безусловное уважение к чести и жизни врага.
Если воин наносил на лицо окраску Плачущего Ягуара, он брал на себя обязательство не обнажать оружие на переговорах (исключение --- самозащита), не убивать неумелых, сдавшихся и безоружных.
Отринуть эту клятву было нельзя --- если воин не исполнял обязательств, его подвергали остракизму.
Во времена войны с Безумными из-за постоянных межплеменных и междоусобных столкновений, а также расцвета наёмничества философия потеряла популярность, и Плачущих Ягуаров на всей Короне можно было пересчитать по пальцам.
Тем не менее их знали если не в лицо, то по слухам, и эти воины пользовались таким уважением, что иногда им сохраняли жизнь даже заклятые враги.
Один из самых известных Плачущих Ягуаров того времени --- Митхэ ар’Кахр э’Тхартхаахитр.}

\thesynonim{omega-field} % Cojani-Wajermann field, CWF
{Поле Кохани---Вейерманна (ПКВ)}
{Омега-поле}

\theterm{weavingstone}
{Полотняный камень}
{Полевой шпат с ориентированными включениями гематита и ильменита, напоминающими текстуру полотна.
Очень ценился народами Тра-Ренкхаля как поделочный камень для украшений.}

\theterm{scion}
{Потомок}
{---}

\theterm{root}
{Предок}
{---}

\theterm{problem-48}
{Проблема 48}
{Глобальное потепление на Сомерскай, вызванное городским суперконгломератом Скальдборо-Гелиополь.
Помимо производственных механизмов, город увеличил площадь освещаемой солнцем поверхности практически на 1\%.
В настоящее время проблема решена перемещением городского массива под полузеркальный щит, а также системами отведения и рассеивания производственной теплоты.}

\theterm{copies-problem}
{Проблема копий}
{Одна из фундаментальных проблем раннего общества демонов.
Каждый демон создавался путём сборки частей уже существующих демонов, а значит --- нёс в себе весь груз предыдущих ошибок и уязвимостей, которые могли быть использованы противником для уничтожения всех демонов-потомков.
Клановая система общества в какой-то степени решала проблему копий --- каждый клан имел свою собственную базу данных и специалистов, которые занимались оптимизацией кланового ядра и фиксацией клановых программных ошибок.
Впоследствии проблема копий была окончательно решена --- вначале оцифровкой сапиентов, а затем, уже во времена ангельских технологий --- алгоритмами контролируемой мутации.}

\theterm{volcanic-facepowder}
{Пудра вулканическая}
{На Преисподней: микрочастицы, обладающие намагниченностью и меняющие магнитную ось под воздействием света. 
Основа явления, известного как <<закатные аспиды>>.
Как следует из названия, традиционно их появление приписывалось физико-химическим реакциям во время извержения вулкана, и только впоследствии было доказано, что происхождение частиц имеет биологическую природу (скелеты раковинных архей, живущих в вулканических источниках).}

\theterm{rainbow-madness}
{Радужное безумие}
{Комплекс воздействия Безумного бога на сапиентов Тра-Ренкхаля.
Включал в себя создание агрессивных короткоживущих антарид и распыление в воздухе МПДЛ.}

\theterm{destruction}
{Разрушение}
{---}

\theterm{stlock-reaction}
{Реакция Стлока}
{Вегетативная реакция инкарнированного хоргета (облачный тип) на враждебные эманации.
Связана с временной потерей контакта между демоном и сапиентным телом.
Может проявляться различными симптомами --- от тахикардии и мочеизнурения до нейрогенной лихорадки.
Ранее реакцию Стлока использовали для поиска шпионов среди сапиентов, в новейших версиях хоргетов эта ошибка скорректирована или исправлена полностью.
Открыта биологом Картеля по имени Стлок Морской Прибой.}

\theterm{talks}
{Речи}
{Маленькие абстрактные истории ноа.
Отличаются высокопарной лексикой, очень простой грамматикой и в то же время некоторыми грамматическими оборотами, не употребляющимися в повседневности (например, единственно-безличное местоимение и безвременные глаголы). Речи делятся на четыре большие группы:
\begin{enumerate}
\item Большие речи --- <<Речь о мужчине>>, <<Речь о женщине>> и <<Речь о шамане>>;
\item Высокие речи --- <<Речь о жреце>>, <<Речь о воине>> и <<Речь о старателе>>;
\item Далёкие речи --- <<Без края>>, <<Опалённые>>, <<Хрустальные земли>>, <<Речь о дельфине>> и <<Сон ребёнка>>;
\item Малые речи --- все прочие речи в фольклоре ноа и некоторых других народов, воспринявших эту часть культуры (травники Дикого Юга, трами).
\end{enumerate}}

\theterm{bearer}
{Родильница}
{---}

\theterm{family-syllable}
{Родовой слог}
{Часть имени сели, передающаяся с кормильцем-хозяином дома.}

\theterm{garden}
{Сад}
{Сословие у сели, объединяющее крестьян, животноводов, охотников, бортников, рудокопов, старателей и охотников за металлоломом.}

\theterm{seijmar}
{Сейхмар (\theorigin{sd}{dzaiku-maru}{мелочь, побрякушка, фурнитура})}
{В широком смысле --- любой не занятый демоном сапиент; в узком смысле --- зародыш, детёныш сапиента.}

\theterm{seli}
{Сели}
{---}

\theterm{silver-hair}
{Серебряный волос}
{У сели: волос на голове человека, который якобы нельзя выдернуть.
Если найти этот волос и обернуть вокруг пальца, то можно влюбить в себя человека.}

\theterm{soso-mar}
{Сосо'мар (\theorigin{tn}{soso'mar}{идти под морем})}
{Один из древнейших способов преодоления экваториальных вод.
Рыбацкий баркас ноа имеет форму двойной погружающейся лодки с навесом из ткани, концы которой опущены в воду.
Между лодками находится трюм --- деревянная клетка, в которую сбрасывается пойманная рыба.
Если судно к утру не успевает вернуться в порт, то моряки опускают лодки до <<линии дельфина>> --- уровня, когда вода захлёстывает верхнюю палубу, --- одеваются в костюмы, сделанные из акульей кожи, и спускаются в трюм.
Дышат обычно через специальные маски, состоящие из кишки, пропущенной через слой прохладной воды, с мешком-конденсатором.
Питаются рыбой, которую вялят на высунутых наружу палках, пьют подсоленный конденсат из мешков.
Некоторые суда имеют гребной винт, приводимый в движение механизмами из трюма, что позволяет двигать судно даже в таком состоянии.
Обычно после трёх-четырёх дней плавания сосо'мар некоторые моряки умирали от жажды, жары или захлёбывались водой, а судно приходило в негодность;
тем не менее, согласно отчётам Коричного флота, этого срока хватало для того, чтобы преодолеть экваториальную зону с севера на юг.
Моряки ноа, пережившие экваториальные воды, пользовались почётом и уважением.
Они имели право на бесплатную чашу выпивки в любой таверне и на одну любую вещь не дороже золотого грана у любого торговца, что легко окупало затраты на плавание.}

\theterm{sotron-beard}
{Сотронская борода}
{Бодмод сотронских сели.
Для этого они аккуратно вживляют в кожу губ и подбородка части скальпов умерших родственников или друзей, либо собственных (с затылка).
Любопытно то, что бороды у сотронских сели могут носить и мужчины, и женщины.
Хака, которым случается попасть в Сотрон, всегда смеются над бородатыми женщинами, потому что в их племени бороды носят лишь мужчины, и борода является символом мужской силы.
Но для сели борода --- не символ и не прерогатива, для них это просто украшение, как серьги или ожерелье.}

\theterm{blued-steel-union} % Blued Steel Union
{Союз Воронёной Стали}
{---}

\theterm{match-tech}
{Спичечная технология}
{Высокотехнологичные устройства, которые можно сделать из широкого спектра подручных материалов и с минимальной инфраструктурной поддержкой.
Термин придумал Михаил Кохани, изобретатель <<спичечного самолёта>> и <<спичечной винтовки>>.}

\theterm{stabitanium}
{Стабитаниум (\theorigin{sl}{stabi[lis]}{стойкий} и \theorigin{sl}{[ti]tanium}{титан})}
{Сплав титана с памятью формы, имеющий в составе более 70 легирующих добавок (в том числе вольфрам и хром), стабилизированный волокнистыми кристаллами и полимерами кремния.
Количество марок стабитаниума в настоящее время более шестисот, характеристики марок различаются очень значительно, что делает его лидером среди материалов на богатых титаном планетах.}

\theterm{surroganium}
{Сурроганиум (\theorigin{sl}{surrog[atus]}{заменяющий} и \theorigin{sl}{[tit]anium}{титан})}
{Заменитель стабитаниума, изготовленный из других металлов и/или в пропорциях, зависящих от местной встречаемости металлов.}

\theterm{tama}
{Тама (\theorigin{sd}{tama}{бродяга})}
{На Преисподней: геолог, исследующий свежие вулканические отложения на предмет минералов и важных реактивов.
Также тама искали террасы --- места, которые защищены от выбросов и могут быть использованы для земледелия.}

\theterm{tamja}
{Тамга (\theorigin{саркорт}{tamha}{клеймо для скота})}
{Клановый символ.
Термин произошёл из одного из языков друзы Хербст, планета Тысяча Башен.}

\theterm{cockroach-war}
{Тараканья война}
{Мировая война на планете Ди.
Вскоре после заселения планеты произошло крупное столкновение между колониальными апидами и сапиентами-млекопитающими.
Люди, кани и планты были уничтожены почти поголовно.
Спасение для этих видов пришло через долгое время, как ни странно, от самих же апид --- несколько особей способных к размножению апид, называемых Уродами или Тараканами, начали вести подрывную деятельность в колониях и спасать себе подобных.
Тараканы заключили союз с млекопитающими, и во время ядерной, а затем и химической войны колониальные апиды были побеждены.
Победители вынуждены были уйти с сожжённой планеты на её необитаемый близнец --- планету Тси, давшую название новому союзу.}

\theterm{taari}
{Тари (\theorigin{tn}{taari}{благомыслящий})}
{У царрокх и хака: уважительное обращение к первому потомку мужского пола старейшины племени.
У сели: уважительное обращение к любому молодому человеку.}

\theterm{telln}
{Телльн (\theorigin{sd}{telln}{серёжка})}
{Время полного оборота оси Преисподней в результате прецессии, единица измерения времени (примерно 93 тысячи земных лет).}

\theterm{technku} % Technku
{Тенку}
{---}

\theterm{shadow}
{Тень}
{Слепок нервной системы сапиента, состоящий из материалов, отличных от его собственных клеток (электроника, генетически модифицированные клетки).
Тени создавали некоторые высокоразвитые цивилизации, они были способом продления жизни индивида.
В строгом смысле тенью может считаться также демон-урожденный сапиент.}

\theterm{terracota-wolf}
{Терракотовый волк}
{Образное название атомного оружия.
Происхождение связывают с легендой планеты Запах Воды о глинистом холме, который атомным взрывом превратило в похожую на волка терракотовую статую.
Другая версия, однако, гласит, что фразеологизм пошёл из языка герска планеты Тысяча Башен и его происхождение связано с табуированием слова <<ядерный гриб>> (nugvar-mikke) и заменой его на анаграмму <<глиняный волк>> (mugnar vikke).
Однако в настоящее время вторая версия не поддерживается, так как нет никаких данных о том, что на Тысяче Башен когда-либо использовалось атомное оружие.}

\theterm{grasshider}
{Травники}
{---}

\theterm{three-storey-temple} % Three-storey Temple
{Трёхэтажный Храм}
{Большой Храм Тхартхаахитра.}

\theterm{qi-people}
{Тси (\theorigin{чайнис}{Qi}{мощь, сила, расцвет})}
{Высокоразвитая цивилизация на двойной планете Тси-Ди системы Проксима Центавра, время существования --- 2 телльна --- абсолютный рекорд за всю историю.
Первые и единственные, кто сконструировал планетную защитную систему против хоргетов.
Гибель цивилизации наступила после ошибки в коде Машины, управляющего компьютера планеты.
Попытка отключить управляющий компьютер закончилась войной, в которой тси, как предполагалось ранее, были истреблены поголовно.
В настоящий момент система Тси-Ди --- единственный мир, в котором живут свободные Машины, и единственный мир, практически недоступный для хоргетов.
Технологические данные по Тси-Ди засекречены и обрабатываются закрытой службой --- отделом 104.}

\theterm{tchoemikchar}
{Тхэмикхар}
{Запрещённый вид охоты, при котором китовый ус оборачивался куском мяса или пальмовым маслом.
Животное съедало приманку, и китовый ус пронзал его внутренности, обрекая на мучительную смерть.
Охотники-сели, применявшие тхэмикхар, подвергались остракизму --- как и при использовании всех Десяти Проклятых Способов Охоты, включающих различные типы капканов и отравленных приманок.
Северные племена Ледяной Рыбы применяли тхэмикхар для охоты на хищников без каких-либо ограничений.}

\theterm{AID}
{УИД}
{Устройство, имитирующее деятельность.
Механизм, работа которого не преследует никаких целей, за исключением эстетической.
На Тси-Ди УИД были неотъемлемой частью стиля Механик --- архитектуры, дизайна и скульптуры.
Орденом Преисподней УИД используются при обфускации в важных оборонных узлах и программных блоках.
Программные аналоги УИД называются \textbf{хинду-лианами} или просто \textbf{лианами} --- они как бы <<оплетают>> код, скрывая его истинные очертания.}

\theterm{duck-mask}
{Утиная маска}
{Атрибут жреца сели, в основном врачей и тех, кто приносит жертвы.
Предназначена для защиты от патогенной микрофлоры, токсичных газов, аэрозолей и пыли.}

\theterm{phalanx}
{Фаланга (более точный перевод --- <<палец>> или <<указательный палец>>)}
{Колюще-режущее оружие ближнего боя, полуторапядевый клинок на изогнутой четырёхпядевой рукояти.
Фаланга была основным оружием воинов, так как позволяла держать на расстоянии ножи, была достаточно манёвренным против копья и стоила гораздо дешевле цельнометаллической сабли.}

\theterm{faciogramm}
{Фациограмма}
{Интерфейс, имитирующий лицо сапиента, предназначенный для отработки эмоционального аффекта при гуманизации хоргета.}

\theterm{fidens}
{Фидены (\theorigin{аркаб}{fidaienah}{воины})}
{Генетически модифицированные солдаты Пятого императора Плеяд, отличавшиеся огромной силой, ловкостью и жестокостью.
Были запрограммированы на безусловное подчинение Голосу императора.
Очень часто фидены запускались во враждебное общество, где создавали семьи и давали потомство.
Спустя несколько поколений потомки воинов <<активировались>> подосланным диверсантом, и в обществе устанавливалась подконтрольная императору военная диктатура.
Разработка фиден-подобных генетических паттернов (ФПГП), а также создание клеток или вирусных векторов с ФПГП на территории Ордена Преисподней расцениваются как сотрудничество с Картелем и караются немедленным уничтожением (Оборонительный кодекс Ада, 12F.2).}

\theterm{viola}
{Фиола (\theorigin{t-sl}{fiolla}{мензурка})}
{Сосуд из кварцевого стекла в металлической оплётке.
Аналог кошелька на Драконьей Пустоши, использовался для хранения и передачи ртути.
Фиола выдавалась всем мужчинам на совершеннолетие.
Потерять, украсть, отнять или разбить фиолу считалось величайшим бесчестьем и несчастливым знамением.
Даже разбойники оставляли своей жертве (нередко мёртвой) её фиолу, забирая лишь находившуюся в сосуде ртуть.
Этот предмет нашёл отражение в оборотах: <<продать фиолу>> --- нищенствовать, опуститься;
<<хранить дома чужие фиолы>> --- быть беспринципным, ценить деньги превыше чужого благосостояния;
<<слушать бульканье фиолы при ходьбе>> --- испытывать финансовые затруднения.}

\theterm{foe-f}
{Фоу-Ф (\theorigin{эрденшпрак}{Foe-F, Foen-F}{вторженцы, летящие строем кранихвинкель})}
{Фоу-Ф ведут свою историю с Друзы Бэйфан, что в Тропическом Поясе.
Местные бродячие циркачи --- фуцзы --- стали ворами во время очередного голода и составили ядро будущих наемников.
В течение своей недолгой истории Фоу-Ф разорили пять Друз и в конце концов обосновались на Гарда Викка, в услужении режиму Валдиса Хина.
Во время Осенней войны Фоу-Ф были разгромлены, и братство прекратило свое существование.
Также в их честь была названа редкая прионная болезнь (фоу-прион), которая была распространена исключительно среди некоторых наёмников Фоу-Ф, практиковавших каннибализм (так называемых Полуночных).
}

\theterm{chasetraasem}
{Хасетрасем (\theorigin{tn}{chasetraasem}{лицо, начерченное в воздухе})}
{В тси-подобных языках: фраза, представляющая собой биометрическое описание внешности сапиента, его мимики, особенности движений и голоса.
Зная хасетрасем, сели могли найти нужного человека даже в ночной толпе.
В школе несколько лет обучения посвящались искусству его составления.
Вероятно, что метод был разработан последними тси в рамках подготовки к одичанию.
Сами тси для этих целей использовали технологические средства.}

\theterm{hjar}
{Хйяр (\theorigin{kvenska}{hajarr}{камень})}
{Город в естественном разломе.
На стенах разлома устанавливаются навесные полки, мосты и лестницы, а сами жилища устраиваются в нишах.
Это слово с Тысячи Башен прочно вошло в языки многих планет, на которых когда-либо властвовали Орден или Картель.}

\theterm{jorget}
{Хоргет (\theorigin{sd}{horohito}{нелюдь})}
{}

\theterm{temple}
{Храм}
{Верхний и Нижний Этажи.}

\theterm{keeper}
{Хранитель}
{---}

\theterm{chrikchuatr}
{Хрикхватр}
{Дерево Перьев, обработанное горячей щёлочью и сжатое в раскалённых тисках.
Обладает большой прочностью и устойчивостью к воздействию химических веществ.
В некоторых областях Короны хрикхватром заменяют акульи зубы и обсидиан в оружии и инструментах.
Цельнодеревянные хрикхватровые стрелы очень ценятся охотниками --- они реже ломаются и меньше изнашиваются.}

\theterm{hook-n-glider}
{Хук и глайдер}
{Подарок на совершеннолетие практически во всех племенах Тысячи Башен.
Хуком называется набор скалолазных инструментов.
Глайдер --- подобие дельтаплана, позволяющее планировать с более высоких точек планеты на более низкие.
С этими двумя предметами дети Тысячи Башен учатся обращаться раньше, чем ходить и говорить.
Фразеологизм <<хук и глайдер>> на Тысяче Башен означает зрелость или подходящее время для какого-либо мероприятия.
В настоящее время хук и глайдер включены в тамгу клана Дорге и его псевдокланов;
выражение может быть использовано как намёк на представителя клана.}

\theterm{hook}
{Хук}
{Набор скалолазной экипировки на Тысяче Башен.
Состоит из: штоков, змайки, локхейдов, гарпуна, вайссака с куржаком, рапиры и других.}

\theterm{huneu} % Хьюнеу
{Хунев}
{Кибернетический организм, разработанный на Древней Земле.
Представляет собой человекообразное тело с искусственными рецепторами, управляемое культурой иммортализованных человеческих нейронов.
К 615 Эпохи Богов хунев получили равные права с людьми и кани.
Хунев отличаются высоким интеллектом и сроком жизни, значительно превышающим человеческий (есть информация о хунев, проживших 700 и даже 1000 лет).
В силу того, что у хунев отсутствует субстрат древних инстинктов, они лишены потребности в сексе, размножении и пище.
Тем не менее они могут испытывать эмоции, и некоторые из них имели потребность воспитывать детей --- как хунев, так и млекопитающих.
Многие хунев занимались наукой, искусством и сложными видами технологической деятельности --- например, микрохирургией и компьютерной безопасностью.
После катаклизма на Древней Земле изначальная технология изготовления хунев была безвозвратно потеряна.
Несмотря на то, что впоследствии были попытки создания хунев на других планетах, их результаты были значительно скромнее тех, которые достигли первые люди.}

\theterm{choesitr}
{Хэситр (\theorigin{tn}{choesitr}{напиток спокойствия})}
{Ритуальная чаша с водой, заменявшая погребение.
Если сели знал, что его не смогут похоронить с соблюдением всех ритуалов, то перед смертью он выпивал хэситр и умирал спокойно, зная, что найдёт пристанище лесных духов.
Также считалось приемлемым выливать хэситр в рот уже умершего человека.
Ранее такие чаши делались из благородных деревьев и украшались резьбой, впоследствии сели стали использовать как хэситр любую чашу, нанося на неё охранные знаки.
Использовались хэситры на войне, во время стихийных бедствий, а также людьми, совершающими <<последнюю беседу с собой>> (самоубийство).}

\theterm{tesarrokch}
{Царрокх}
{---}

\theterm{workshop}
{Цех}
{Сословие сели, объединяющее ремесленников.}

\theterm{windzither}
{Цитра Ветра (\theorigin{tn}{trotris}{звенеть (шелестеть) на ветру})}
{Десятиструнный музыкальный инструмент сели (4 гладких струны + 6 витых).
Название получил по имени его предположительного изобретателя, Ветер-Дующий-Ниоткуда (легендарный бард, любовница купца Чхаласа).
Для игры использовались специальные перчатки с дополнительными <<пальцами>> или встроенная в гриф каретка.
В деке имелся механический смычок для гладких струн, приводимый в движение маховиком и ногой барда.
К моменту войны с Безумным на Короне осталось всего два мастера, которые умели делать эти инструменты, оба погибли во время войны.
Но образцы цитры Ветра попали к культурологам Ордена Преисподней, которые смогли воссоздать технологию изготовления.}

\theterm{beads-of-sat}
{Чётки Сата}
{Верёвочные карты дорог, бывшие в большом ходу у торговцев и путешественников обитаемой Короны и Кита.
Представляли из себя цветные верёвки, связанные между собой, с узелками или бусинами;
количество бусин равнялось количеству верстовых столбов.
Торговцы в дороге постоянно держали их в руках и пальцами отсчитывали пройденные кхене.}

\theterm{gods-age}
{Эпоха богов}
{}

\theterm{daemons-age}
{Эпоха демонов} % sd: asogeite
{}

\theterm{jasper}
{Янтарь}
{Камень, встречающийся только на Тра-Ренкхале, Хемане-2 (Хароне) и, согласно некоторым данным, на Древней Земле.
По легенде, это слёзы дерева акхкатрас, падающие в Ху'тресоааса и выносимые в Кипящее море.
В настоящее время установлено, что основой янтаря является смола не акхкатрас, а другого растения --- голосеменного эпифита Пятикрыльник плакучий (Бенедикта).
Кислые геотермальные воды плавят смолу, спекают её с продуктами придонных моллюсков, диатомей и кольчатых червей, а затем встречное течение Могильного пролива разносит янтарь по всему побережью вплоть до Молчащих лесов.
Особенно ценным считается янтарь с вплавленными в него жемчужинами и золотыми самородками.}

\theterm{yao}
{Яо (Y)}
{Единица измерения масс-энергии ПКВ.}

\theterm{rubbish-fair}
{Ярмарка хлама}
{Праздник в первый год Церемонии.
Люди вытаскивают и продают за бесценок весь хлам, который скопился в жилищах, часто находят старые клады.
В переносном смысле --- отсутствие выбора при видимом изобилии.}

\chapter{Прочее}

\section{Социум сели}

\begin{itemize}
\item Храм (the Temple)
\begin{itemize}
\item Нижний этаж (the Downstairs)
\item Верхний этаж (the Upstairs)
\end{itemize}
\item Двор (the House)
\item Цех (the Workshop)
\item Сад (the Garden)
\end{itemize}

\section{Календарь сели}

\subsection{Вехи Жатвы}

\epigraph
{Каждые шестнадцать дождей Безумный отрезает от Пирога кусок, а каждые четыреста пятьдесят дождей за своим кусочком скромно приходит Безымянный.
Если они встретятся --- жди беды.}
{Детская присказка сели}

Календарь Тра-Ренкхаля основан на астрономических вычислениях, сделанных ещё последними тси.
Древние, не мудрствуя лукаво, взяли на вооружение проверенные временем методы составления календаря.
Расхождение между астрономическим годом и целым числом астрономических дней сгладили убавлением одного рассвета каждые 16 и каждые 450 дождей;
таким образом, погрешность каждого дождя Жатвы относительно астрономического полугода составляет всего 0.8 секхар.
Эта нехитрая математика и нашла отражение в детской присказке.

Дожди делятся на полновесные и високосные.
Високосные, в свою очередь, делятся на малые (каждые 16 дождей), большие (каждые 450 дождей) и двойные (каждые 3600 дождей).

По странному стечению обстоятельств, битва на Могильном берегу, унесшая жизни ста тысяч сели и оборвавшая долгое кровавое правление Безумного, случилась в Пирог 13 дождя 10800 --- за рассвет до конца двойного високосного дождя.
Предыдущий двойной високосный дождь был ознаменован Нашествием Змей, опустошившим половину Севера.

Описанную систему используют почти все племена Тра-Ренкхаля, за исключением зизоце и тенку;
различаются только названия декад.
Потомки людей Лотоса отмеряют время по системе, близкой к летоисчислению Древней Земли.
Они считают время ярами, делят яр на 4 луны, каждую луну --- на 4 вика по 10 рассветов.
Погрешность яра относительно астрономического года Тра-Ренкхаля составляет порядка михнет.
Тенку Водораздела, которые живут с сели бок о бок, перешли на более удобные Вехи Жатвы.

Каждый дождь (80 рассветов) делится на декады (трекхсатр) по 16 рассветов.
Названия декад по календарю сели:

\begin{itemize}
\item Нечётный дождь:
\begin{enumerate}
\item Плуг (Летняя страда)
\item Согхо
\item Лук
\item Большая Капля (Летние дожди)
\item Карп
\end{enumerate}
\item Чётный дождь:
\begin{enumerate}
\setcounter{enumi}{5}
\item Оцелот (Зимняя страда)
\item Тростник
\item Змея
\item Малая Капля (Зимние дожди)
\item Пирог
\end{enumerate}
\end{itemize}

Праздники, отмечаемые по Вехам Жатвы:

\begin{itemize}
\item Лук 8 --- Соревнование кулинаров в Тёплом дворе;
\item Лук 9 --- Соревнование виноделов в Тёплом Дворе;
\item Лук 10 --- Большое похмелье, день взаимопомощи и дружбы, праздник врачей;
\item Змея 3 --- Мягкие Руки, праздник любовников, танцоров и воинов, открытый Круг Доверия для всех горожан;
\item Змея 9 (10, 11) --- Слёзы Ситхэ, первый ночной дождь в Змею, праздник детей, кормильцев и школьных учителей.
\end{itemize}

% Год = 160.129451 рассветов = 4699.825167691611 часов
% Рассвет = 29,350161 часов
% Погрешность = 0.129451 - (1/8) - (1/225) = 6.555e-06 рассветов (1.718 секхар в астрогод)

\subsection{Вехи Церемонии}

Тси хотели сохранить для потомков старое летосчисление, на котором базировалась их история.
Но все старые астрономические ориентиры остались дома, в десяти парсаках пути.
Поэтому ими был изобретён дополнительный, <<жреческий>> календарь, в котором годы планеты Тси-Ди привязывались к астрономическим суткам Тра-Ренкхаля.

В Году Церемонии нет декад, только рассветы.
Всего их 288, каждый 41-й Год Церемонии считается високосным, он на три рассвета короче.
Полный номер записывался только в исторических хрониках;
купцы пользовались упрощённой датировкой --- високосный Год считали за первый, прочие --- в порядке очерёдности.

Отсчёт Вех Жатвы начинается с рассвета 18 Года Церемонии 5 (1399) --- в этот рассвет был заключён договор между четырьмя городами, за которым последовало избрание первого Короля-жреца и первое собрание жрецов, регламентировавшее правила передачи информации и правила литературного языка сели.
Эта дата считается началом существования сели как народа.
Отсчёт Вех Церемонии предположительно ведётся с года Тси-Ди, в который были основаны два древнейших города по программе подготовки к одичанию --- Тхартхаахитр и Аурелия (36 год с момента прибытия тси на Тра-Ренкхаль и 29 год с момента гибели Стального Дракона).

Праздники, отмечаемые по Вехам Церемонии:

\begin{itemize}
\item Ярмарка Хлама --- декада, на которую выпало начало первого Года Церемонии.
Люди распродавали то, что завалялось у них дома;
\item Крылья Лю --- каждый 256-й день Года Церемонии. Фестиваль жрецов;
\item Год Скитальца --- каждый 18-й и 31-й Год Церемонии. В эти Годы было принято оказывать помощь путешественникам и бездомным.
Тем, у кого не было жилища или жилище было худым (молодым семьям, старикам), часто помогали отстроиться силами квартала, путешественникам давали приют и указывали дорогу.
Прекращались войны и набеги;
договор о соблюдении Года Скитальца был утверждён между всеми развитыми народами Тра-Ренкхаля --- сели, ноа, ркхве-хор, хака и тенку.
Год Скитальца считался наилучшим временем для переезда.
\end{itemize}

Лазурные дожди --- дожди, в которые Год Церемонии и дождь Жатвы начинаются в один день.

% Год Тси-Ди = 288.0731707 рассветов

\subsection{Единицы времени}

Единицы измерения времени делились на общепризнанные и локальные.

К общепризнанным относятся:

\begin{itemize}
\item Рассвет --- астрономические сутки Тра-Ренкхаля
\item Кхамит (шаг солнца) --- $1/16$ рассвета
\item Михнет (передышка) --- $1/64$ кхамит
\item Секхар (взмах ресницами) --- $1/256$ михнет
\end{itemize}

Локальные:

\begin{itemize}
\item Кхене (единица времени) --- 9,2 михнет. Распространена вдоль Западного Тракта.
\item Согхо --- 8,5 секхар. Распространена в Ближнеречье.
\end{itemize}

\section{Игры}

\subsection{Метритхис}

Играют 4 игрока, жребием перед игрой распределяются роли.

\begin{itemize}
\item Зелёный --- Король
\item Чёрный --- Мятежник
\item Красный --- Сосед
\item Жёлтый --- Фатум.
\end{itemize}

Подготавливает игру Фатум.
Он выкладывает определённым образом квадратики-поля:

\begin{itemize}
\item Синие --- река/море (1 ход на переправу, нельзя атаковать с другой стороны, пенальти к защите и нападению)
\item Серые --- горы (3 хода на переправу, преимущество в обороне и обстреле)
\item Красные --- пустыня (пенальти ко всем видам деятельности)
\item Зелёные --- лес (преимущество в обороне, скрытности, пенальти к нападению и обстрелу)
\item Жёлтые --- степь (пенальти к скрытности и обороне, преимущество в нападении и обстреле)
\item Чёрные --- болото (2 хода на переправу, преимущество к скрытности и обстрелу, пенальти к нападению и обороне)
\end{itemize}

Фигурки взаимодействуют двумя способами --- драка и беседа.
При драке определяются победитель и побеждённый.
Побеждённый отправляется в коробку, а победитель иногда может сменить класс.
При беседе оба меняют класс, а иногда ещё и цвет.

Начало игры распределяется жребием между Королём и Мятежником.
Последним в игру вступает Сосед, время его вступления выбирает Фатум.

Всего в игре 4 поля --- собственно игровое и 3 дипломатических.
Дипломатические могут видеть только договаривающиеся правители и Фатум.

Комбинации на поле дипломатии:

\begin{itemize}
\item Обман --- проигравший пропускает 5 ходов против выигравшего.
\item Заговор --- Фатум выбрасывает кихотр на смерть проигравшего.
\item Паритет --- 5 ходов правители не ведут друг против друга боевые действия.
\item Мир --- правители до конца игры не ведут друг против друга боевые действия, их стол дипломатии с третьим правителем становятся общим.
\end{itemize}

Если два правителя заключили мир, единственный способ Фатума выиграть --- выбросить успешный Заговор у третьего или Чуму на одного из союзников.
Если все три правителя пытаются построить мир, Фатум пытается их убить.
Поэтому в задачи правителей входит ещё и ввести Фатум в заблуждение.

В распоряжении правителей --- войска и камни дипломатии (чёрные и белые).
В распоряжении Фатума --- кихотр и камни Благ и Несчастий:

\begin{itemize}
\item Верная женщина --- отводит от правителя Заговор.
\item Чума --- убивает войска и с некоторой вероятностью может убить правителя.
\item Удача --- позволяет правителю взглянуть на дипломатический стол противников, выиграть в явно проигрышной стычке войск или даёт фору в 5 камней на любом столе
дипломатии.
\item Потеря друга --- пропуск двух ходов.
\item Безумие --- ход на поле боя или столе дипломатии вместо правителя делает Фатум.
\end{itemize}

Игра заканчивается 4 способами:

\begin{enumerate}
\item 2 правителя умирают, оставшийся в живых правитель и Фатум выигрывают.
\item 1 умирает, 2 других договариваются, Фатум проигрывает.
\item 3 правителя договариваются, Фатум проигрывает.
\item Все правители гибнут, Фатум выигрывает.
\end{enumerate}

Самая сложная задача всегда у Фатума, поэтому очень часто его роль без жребия отдают самому опытному игроку.

\subsection{Пьянка}

Шуточная игра в кости от двух до четырёх человек.
Игра идёт обычно на желания, лакомства, секс или удары, иногда на всё сразу.

Правила желаний: они не должны причинять человеку вреда.
То же самое с ударами --- чаще всего игра идёт на пощёчины, шлепки и несильные броски.
Проигравший может поменять секс или желание на оговорённое число ударов, лакомства и алкоголь можно передавать другим игрокам.

\chapter{История и современый правовой статус нейтральных демонов}

\section{Пассивный нейтралитет, или Непринадлежность}

Самый очевидный и древний тип нейтралитета.
Речь идёт о дёмонах, не признающих какую-либо власть и не подчиняющихся законам фракций.
В настоящее время количество нейтралов оценивается примерно в десять --- тридцать тысяч демонов.
К ним относятся как истинные, никогда не принадлежавшие к фракциям нейтралы, так и дезертиры обеих фракций.
\ml{$0$}
{Они находятся на нелегальном положении и чаще всего уничтожаются при обнаружении.}
{They have an illegal status and mostly are killed on sight.}

\section{Активный нейтралитет, или Невовлечённость}

С самого момента создания фракций было открыто явление, получившее название <<активный нейтралитет>>.
Активный нейтралитетом признаётся деятельность тех демонов, которые пользуются всеми правами фракции, но тем или иным способом избегают обязанности участвовать в борьбе с внешними врагами --- либо подстрекают к подобной деятельности прочих демонов.

\subsection{Научный коллаборационизм}

Как следует из названия, чаще всего этот вид характерен для учёных, многие из которых, несмотря на запрет, контактируют с учёными враждебной фракции.
Контрразведка обеих фракций многократно пыталась прервать эту связь;
в настоящее время о контактах между учёными рекомендовано сообщать заранее, а также предоставлять список передаваемой и получаемой информации.
Тем не менее легализованы, по некоторым подсчётам, менее 0,1\% межфракционных связей.

Сложность их обнаружения обуславливается тем, что у каждого исследовательского отдела выработались свои собственные внутренние правила обмена информацией, и эти каналы связи законспирированы не хуже, чем таковые в военных организациях.
Правила обычно предусматривают фильтрацию любых данных, имеющих военное значение, а также легализацию данных экспериментов, проведённых в другой фракции.
Для легализации данных между отделами существует целая сеть --- также законспирированная;
данные легализует лаборатория с соответствующей сферой деятельности и юрисдикцией.

Отдел 100 признаёт, что грубое вмешательство в межфракционные научные связи, равно как и в систему легализации данных, в настоящий момент не представляется возможным и может вызвать полный коллапс научной деятельности, так как некоторые лаборатории у Ада и Картеля по сути являются \emph{общими} --- легальные филиалы в каждой фракции конспиративно связаны друг с другом.

\subsection{Обход закона}

Также к активному нейтралитету относится косвенный (юридически обоснованный) отказ участвовать в военных действиях и сотрудничать с военными.
Возможностей для такого отказа в настоящее время очень мало, но многие демоны умело используют имеющиеся уязвимости в законах и манипуляцию общественным мнением.
Такой активный нейтралитет стоит на самой грани закона и в некоторых случаях может быть приравнен к саботажу и дезертирству.

\subsection{Декларативный пацифизм}

Одним из видов нейтралитета является декларация пацифистских взглядов.
Она возможна для любых демонов, является одним из самых простых и в то же время опасных видов нейтралитета.
Очень часты случаи открытого пацифизма в военных силах Ада;
солдаты и офицерские чины, несмотря на добросовестное выполнение долга, распространяют свои убеждения посредством личных бесед и выступлений перед подчинёнными.
Некоторые даже выработали для декларации пацифистских взглядов свод правил, которые позволяют доносить до слушателей суть, оставаясь при этом в рамках закона --- например, с помощью медиавирусов.

С этим явлением активно боролись, находя причины для отказа в повышении и даже устраняя пацифистов физически --- например, отправляя их в опасные зоны.
Тем не менее, как показала практика, наибольшее воздействие речи пацифистов имеют именно в опасных зонах;
оттуда, где погиб один пацифист-новобранец, вскоре возвращаются десять пацифистов-ветеранов, которых в разы сложнее заставить замолчать.
В настоящее время демонов с пацифистскими взглядами стараются тем или иным способом устранить из армии;
иногда с ними заключают негласные договоры --- хорошая должность в обмен на молчание.

\section{Попытки легализации нейтралитета}

\subsection{Демиург --- Метрополия}

Договор <<Демиург --- Метрополия>> является одним из способов легализовать нейтрального демона.
Боги являются особым субъектом адского права, потому что они являются частью планеты --- иногда неотъемлемой.
В случае захвата планеты врагом демиург не сможет покинуть её, подобно демонам;
у демиурга в руках зачастую находятся внушительные ресурсы и рычаги давления.
Кроме того, известны случаи, когда демиург напрямую подготавливал планету к захвату или вёл подрывную деятельность.

Именно поэтому Орден Преисподней и Красный Картель заключают с демиургами особые договоры.
Очень часто демиург мог заключить схожие договоры одновременно с обеими воюющими сторонами, и в этих договорах чётко проводилась грань между вмешательством и невмешательством демиурга в военный конфликт.
Но, так как это создавало определённые проблемы для фракций, чаще всего демиурги уничтожались под благовидным предлогом после полного изучения планеты.

\subsection{Иммунитет учёного}

Идея, предложенная Ликаном Безруким.
Он предлагал наделить особым статусом учёных, исследователей космоса и планетарных инженеров --- их ни в коем случае нельзя было трогать и как-то препятствовать их деятельности.
Под определение планетарных инженеров, разумеется, подпадали все без исключения демиурги, а также значительная часть специалистов по мелиорации.
Обладающий иммунитетом учёного демон обязан был соблюдать определённые правила (достаточно большое количество), чтобы не оказаться втянутым в конфликт фракций.
Статус был невосстановимым --- нарушивший правила демон лишался его навсегда, но, тем не менее, имел возможность продолжать свою обычную деятельность согласно законам фракции.
А вот любые попытки уничтожить, завербовать или шантажировать носителя статуса должны были караться уничтожением.

Идея была отвергнута Советом Капитула.
Вернее, она даже не дошла до стадии рассмотрения --- несколько раз её возвращали на доработку, а в последний раз её просто заморозили на неопределённый срок.
Впрочем, вскоре многие об этом пожалели, так как именно заморозка проекта Ликана Безрукого послужила началом расцвета научного коллаборационизма и технологий обхода закона (см. выше).

\part{Временное хранилище}

\chapter{Эпиграфы}

\epigraph{
\ml{$0$}
{Чем дальше путь, тем больше одиночество.}
{The longer the way, the lonelier the way.}
}{Пословица сели}

\epigraph{Я цитра с тысячей струн, я пою под руками твоими...}
{Эрхэ Колокольчик}

\epigraph{
Конец игрушки печален --- её ломают, выбрасывают или забывают.
Дышащий отличается от игрушки лишь тем, что может выбрать свой конец из этих трёх.
}{
Пословица сели
}

\epigraph
{Проблема плюрализма мнений не в том, что одни лгут, а другие говорят правду;
проблема в том, что все без исключения лгут, чтобы сделать убедительнее то, что они считают правдой.}
{Ликан Безрукий}

\epigraph{
\ml{$0$}
{То, что невидимо, всегда дискриминируемо.}
{Invisible is always discriminated.}
}{
\ml{$0$}
{Кэтрин Шаулман, Эпоха Последней Войны}
{Catherine Shoulmann}}

\epigraph
{Качество философского течения обратно пропорционально количеству его радикальных и деструктивных ответвлений.}
{Ликан Безрукий}

\epigraph
{Элемент несет в себе образ системы.
В каждом диктаторе скрывается повстанец.
В каждом повстанце скрывается диктатор.
Те, кто об этом забывает, обречены на повторение ошибок прошлого.}
{Лусафейру Лёгкая Ладонь}

\epigraph
{Гораздо лучше подкупить человека, чем убить его, да и быть подкупленным куда лучше, чем убитым.}
{Уинстон Черчилль.
Эпоха Последней Войны}

\epigraph
{В обществе всегда будут больные, зависимые, бездомные и преступники.
Задача управленца --- снизить скорость их появления, повысить скорость их интеграции в общество и сделать их жизнь по возможности комфортной.}
{Вениамин Лист, канцлер Германии}

\epigraph
{Мир держится на тех, кто находится в центре гауссианы, а движется вперёд благодаря её краям.}
{Лусафейру Лёгкая Ладонь}

\epigraph
{Господь сотворил нас несовершенными с той же целью, с которой Он сотворил зерно, а не хлеб, и виноград, а не вино: Он хотел разделить радость акта творения со Своими детьми.}
{Хакем-Аят, 14:45.
<<Притча о послечеловеке>>.}

\epigraph{
\ml{$0$}
{Оступившийся не выбирает, куда упасть.}
{A tripped one can't choose the place to fall.}
}{Пословица сели}

\epigraph
{Мир несправедлив, и это является первопричиной страдания живых существ.
Единственный источник справедливости --- сапиент, обладающий свободой выбора.}
{Аксиома Несправедливости. Эволюцион}

\epigraph
{Самая распространенная ошибка правителей --- считать, что неуправляемость и опасность суть одно и то же.}
{Лусафейру Лёгкая Рука}

\epigraph{
\ml{$0$}
{Тот, кто считает идею возможностью, уже её использует.}
{The one who treats an idea as an opportunity, already uses it.}
\ml{$0$}
{Тот, кто считает идею болезнью, уже ею заражён.}
{The one who treats an idea as a disease, is already infected.}
}{Лусафейру Лёгкая Рука}

\epigraph
{Когда предлагают выбирать между победой и поражением, на самом деле предлагают не один выбор, а три.
Первый --- поверить говорящему.
Второй --- вступить в игру.
Третий --- победить или проиграть.
Ни один из этих выборов не является обязательным, но абсолютное большинство первые два делает на бессознательном уровне.}
{Лусафейру Лёгкая Рука}

\epigraph
{Если тебе невыносимо тяжело --- посмотри назад.
Возможно, ты тянешь за собой целую эпоху.}
{Мартин Охсенкнехт}

\epigraph
{Языком ненависти всегда ведётся разговор с позиции силы, мнимой или реальной.
Человек, который разговаривает языком ненависти с одним человеком или группой, с высокой долей вероятности будет говорить на нём с любыми людьми, над которыми он имеет мнимую или реальную власть.}
{Мариам Кивихеулу}

\epigraph
{Солдаты исправляют ошибки дипломатов.}
{Джозефус Дэниэлс, Древняя Земля.}

\epigraph
{При достаточном количестве глаз все ошибки лежат на поверхности.}
{Закон Торвальдса"--~Реймонда. Древняя Земля}

\epigraph
{Плохой управленец пытается упорядочить хаос общества.
Хороший управленец пытается нащупать точку равновесия хаоса и сдвинуть её в нужную сторону.
Хороший управленец никогда не будет по достоинству оценён любителями порядка, ибо он принимает хаос как должное.}
{Лусафейру Лёгкая Рука}

\epigraph
{In hostem omnia licita.}
{Латинское крылатое выражение}

\epigraph
{Жалкое зрелище --- волк, теряющий зубы от старости, но ещё пытающийся их скалить.}
{Клаудиу Дентосиу}

\epigraph
{Многие профессии стремительно молодеют.
Молодёжи уже не нужно тратить время на поиск ответа на вопрос <<Кто я?>>, на залечивание ран молодости, на доказывание своего права на существование.
Молодые люди разной психологической и гендерной идентичности, обладающие разными проявлениями сексуальности без особых проблем встраиваются в общество --- а значит, у них остаётся огромное количество времени и сил на реализацию в профессиональной сфере.}
{Мариам Кивихеулу}

\epigraph
{Многие мечтают быть первопроходцами, многие мечтают отличаться от других.
Но в этом нет ничего хорошего.
Ты совершаешь все ошибки, которые только можно совершить, живёшь всю свою жизнь с последствиями ошибок, и даже признание --- если оно придёт --- выглядит весьма незначительной компенсацией.}
{Мариам Кивихеулу}

\epigraph
{Война может только разрушать.
Даже для производства оружия нужны мирные времена.}
{Мариам Кивихеулу}

\epigraph
{Эгоизм делает человека честным.
Честность является предметом нравственности.}
{Постулат Омельчанко.
Эволюцион}

\epigraph
{Слову в наше время придаётся чересчур большое значение.
Но давайте будем честны --- дело не в словах.
Слово не убивает, призывы к войне не разрушают города,  и не оскорбления доводят человека до самоубийства.
Любое зло способно существовать в молчании, как и любое добро.
Но заставлять молчать тех, кому есть что сказать --- это зло в чистом виде.}
{Мариам Кивихеулу}

\epigraph
{Расцвет цивилизации начинается тогда, когда большинство сапиентов становятся чересчур ленивыми для убийства друг друга.}
{Длинный-Мокрый-Хвост}

\epigraph
{Правители новой эпохи сидят в тюрьмах предыдущей.
Поэтому хорошо думайте, кого отправляете за решётку.}
{Клаудиу Дентосиу}

\epigraph
{В идеальном обществе каждый индивид обладает властью, прямо пропорциональной его потенциалу --- военному, трудовому и культурному.}
{Первый постулат физики социума. Эволюцион}

\epigraph
{Любое распределение власти, отличное от описанного, не может существовать бесконечно долго.
Скорость восстановления идеального распределения власти прямо пропорциональна уровню технологического развития.}
{Недоказуемое следствие из первого постулата. Эволюцион}

\epigraph
{Ни одному военному не приходится терпеть столько лишений, ни одному военному не приходится переживать столько сражений, сколько выпадает на долю самого заурядного пацифиста.}
{Ликан Безрукий}

\epigraph
{Развитие цивилизации неизбежно ведёт к ослаблению внутривидовой конкуренции.
Научно-технический и социальный прогресс увеличивают количество социальных ниш в геометрической прогрессии.}
{Яо Вэй, автор шкалы развития сапиентного общества.
Впоследствии фраза стала Восьмым постулатом Эволюциона.}

\epigraph{Если в технологически развитом обществе имеет место быть жёсткая конкуренция --- значит, она поддерживается искусственно.}
{Первое следствие из Восьмого постулата Эволюциона}

\epigraph
{В одиночестве легче делать правильный выбор, в компании легче переживать последствия неправильного.}
{Длинный-Мокрый-Хвост}

\epigraph
{Маяки не бегают по всему острову, выискивая, какую бы лодку спасти, они просто стоят и светят.}
{Энн Ламотт}

\epigraph{
\ml{$0$}
{Большая часть методов психологической манипуляции строится на том, чтобы представить случайное закономерным, а закономерное --- случайностью.}
{Most of methods of psychological manipulation are based on the idea to make random look logical, and or make logical look random.}
}{Лусафейру Лёгкая Рука}

\epigraph
{Как нелепо выглядят со стороны по-настоящему великие дела!}
{Изречение, якобы сказанное Анатолиу Тиу, когда он впервые увидел лиманское ополчение}

\epigraph
{<<Счастье не в деньгах>>?
Какая глупость.
Так говорят только те, кто их никогда не имел.
Или те, кто не имел ничего, кроме денег.}
{Клаудиу Дентосиу}

\epigraph
{Если стражи порядка закрывают лица и не называют имён --- значит, они недобросовестно выполняют свою работу.
Тем, кто действительно делает своё дело хорошо --- защищает граждан и ловит преступников --- незачем бояться.
Безопасность стражей порядка --- исключительно результат их собственной работы.}
{Анатолиу Тиу}

\epigraph
{Харизма --- опаснейшее оружие.
Пожалуй, это единственная вещь, которая по силе воздействия может соперничать с истиной --- и именно поэтому часто ей противопоставляется.}
{Мартин Охсенкнехт}

\epigraph{
\ml{$0$}
{Один из главных тормозов развития --- неспособность различать комфорт и условия, в которых можно существовать.}
{One of main hindrance to development is inability to distinguish between comfort and possibility to live.}
}{Длинный-Мокрый-Хвост}

\epigraph
{Идеи новой парадигмы нужно подвергать тщательному отбору.
Любая идея, которая является зеркальным отражением идеи старой системы, будет работать на старую систему.
Особенно это относится к социально значимым идеям, касающимся насилия и угнетения --- ни в коем случае нельзя превращать угнетателя в жертву угнетения.
Наоборот, нужно показать, что новая парадигма будет выгодна для большинства членов общества --- в том числе и для тех, кто принял правила старой системы и умеет играть по её правилам.}
{Мариам Кивихеулу}

\epigraph
{Чем больше индивидов знают о проблеме и имеют возможность её обсудить, тем выше вероятность того, что проблема будет решена наилучшим образом.}
{Недоказуемый постулат Элект \#3}

\epigraph
{Хороший правитель должен дать подданным то, что они хотят.
Свободу, войну, слепое подчинение, справедливость --- всё, что угодно.
Подданные должны получить желаемое --- это сделает их целеустремлённее и увереннее.
Подданные должны получить желаемое кровью и потом --- это сделает их практичнее и экономнее.
Подданные должны прочувствовать все последствия своего выбора --- это сделает их осторожнее.
Когда подданные обретут все эти качества, правитель им больше не потребуется.}
{Лусафейру Лёгкая Рука}

\epigraph{
\ml{$0$}
{Убийство в целях самозащиты является самоубийством нападающего.}
{Justifiable homicide is assailant's suicide.}
}{Законы ноа}

(К пленению Чханэ?)

\epigraph
{Посмотрите на оркестр в Симфоническом театре Гелиополя.
Каждый музыкант занят своим делом.
Обитаемая Вселенная --- такой же оркестр, в котором каждый исполняет свою собственную партию, а дирижирует коллективное бессознательное.
Что же до войн и прочего непотребства... ведь оркестр не обязан играть только детские песенки, верно?}
{Ликан Безрукий.
Речь на презентации языка Эй}

\epigraph
{Ты всегда в ответе за то, чему не пытался помешать.}
{Жан-Поль Сартр}

\epigraph
{Все самые неприятные вещи в истории делали существа без чувства юмора.
Чем больше смеётся один --- тем меньше плачут остальные.}
{Длинный-Мокрый-Хвост}

\epigraph
{Даже в рамках бесчеловечной системы можно быть человеком.
Всё зависит лишь от выбора конкретного индивида --- быть человеком или зверем.}
{Мариам Кивихеулу}

\epigraph
{Правонарушения должны наказываться, но следует помнить, что истинный виновник --- господствующая система.
Как бы высоко или низко ни находились люди, принявшие её правила, они не более чем жертвы и орудие системы.
Это умозаключение не стоит ровным счётом ничего, пока вам не потребуются союзники.}
{Мариам Кивихеулу}

\epigraph
{Если вы придерживаетесь нейтральной позиции в ситуации несправедливости, вы выбираете сторону угнетателя.}
{Десмонд Туту. Древняя Земля}

\epigraph
{Пережитый позор ныне попран ногами танцоров.}
{Сигурдур А. Магнуссон. <<Греция --- 1974>>}

\epigraph
{Я видел, как дом превращается в тюрьму, а тюрьма превращается в дом.
Но едва я открыл рот, чтобы об этом сказать, как моё горло сжала петля.}
{Неизвестный повстанец Лимана.
Последнее слово перед казнью}

\epigraph{У природы не было в планах делать живое существо счастливым.
Счастье --- это награда за победу.
Договорившсь между собой, сапиенты одержали победу сообща, и счастье стало нормой.}
{Длинный-Мокрый-Хвост}

\epigraph{Под глухими высокими стенами спрячутся много врагов.}
{Пословица ноа}

\epigraph
{Жертвой можно ослабить, но нельзя победить.}
{Пословица Преисподней}

\epigraph
{Если тебе есть, что сказать, поднимись, чтобы тебя увидели.}
{Пословица хака}
% Индейская пословица

\epigraph
{У всех животных размножение --- это специально организованный процесс, выбивающийся из обычного ритма жизни.
И только у сапиентных видов дети --- это побочный продукт социального взаимодействия.}
{Длинный-Мокрый-Хвост}

\epigraph
{Моё от меня не спрячется.
Чужое моим не будет.}
{Присказка сели.
Предположительно отрывок из потерянного сборника Эрхэ Колокольчик}

\epigraph
{Ложь идёт рука об руку с унижением.}
{Пословица ноа}

\epigraph
{Если мы не понимаем, что происходит у нас внутри, это превращается во внешние события, которые кажутся нам судьбой.}
{Карл Густав Юнг}

\epigraph
{Существуют два подхода к современным технологиям, я называю их <<западным>> и <<восточным>>.
Запад использует технологии для повышения удобства, а когда что-то идёт не по плану --- меняет парадигму.
Восток до последнего цепляется за парадигму, а технологии использует для того, чтобы парадигма работала с заявленной эффективностью.
В случае неудачи западный человек говорит <<Мы делаем что-то не так>>, восточный --- <<Мы использовали недостаточно мощные средства>>.}
{Сергей Хистиаков}

\epigraph{Изучение ругательств народов --- хороший путь к постижению их святынь.}
{Грегори Ландау}

\epigraph
{Рокеры учили толпу действовать сообща без муштры и построений, свойственных военным.
Рокеры взращивали свободную толпу, действующую в своих интересах и лишённую рычагов управления.
Я бы сказал без преувеличений, что гражданское общество XXI века ковалось на рок-концертах.
Именно поэтому неформальные музыканты и их поклонники рассматривались авторитарным режимом как полноценная политическая угроза.}
{Мартин Охсенкнехт}

\epigraph
{Великие реки берут начало в болотах и входят в силу, питаясь чистейшими горными ручьями.
Те же реки, что спустились с гор в поисках болотной воды, остаются в болоте насовсем.}
{Пословица сели}

\epigraph
{Публичный человек всегда в той или иной степени готов поступиться убеждениями.
Наличие аудитории для него гораздо важнее истины.
Чем шире аудитория --- тем чаще этой истиной приходится жертвовать.}
{Мариам Кивихеулу}

\epigraph
{Даже охотник не станет убивать птицу, просящую у него защиты.}
{Пословица Преисподней}
% Японская пословица

\epigraph{
\ml{$0$}
{Религия как ничто другое умеет убеждать людей, что неумение и незнание есть добродетели.}
{Religion like nothing else can convince people that there's a virtue in ignorance and inexperience.}
}{
Мартин Охсенкнехт
}

\epigraph
{...И был Талим свидетель тому, как слова Его прорастали в толпе, и крепли, и цвели, и давали плоды, и люди вкушали эти плоды, и видел Он, как насыщали люди свой голод плодами Его слов...}
{Хакем-Аят, 20:2}

\epigraph
{Идёт дурак, и от смеха его рушатся святыни.}
{Хакем-Аят. Апокриф Искандера (Свиток Ликана), 0:28}

\epigraph
{Нгвсо --- единственные на сегодняшний день Ветви Земли со 120 баллами по Яо, извлекающие кислород только из воды.
До Тра-Ренкхаля считалось, что существование сапиентов Ветвей Земли с исключительно жаберным дыханием невозможно в принципе.
Безымянный разрушил этот стереотип.
Возможно, выбор моллюсков в качестве базы и был в корне ошибочен, но демиург, если можно так выразиться, выжал из океана весь возможный интеллект.}
{Корхес Соловьиный Язык.
Обзор <<Биоразнообразие колоний аксиального направления>>}

\epigraph
{Выиграть войну так же невозможно, как невозможно выиграть землетрясение.}
{Джаннетт Ранкин, общественный деятель.
Эпоха Последней Войны}

\epigraph
{Смех --- самое страшное оружие.
Смехом можно убить всё --- даже убийство.}
{Эуджин Замятин.
Эпоха Последней Войны}

\epigraph
{Чем выше интеллект у существа, тем дольше длится его детство.
Видимо, идеальное интеллектуальное существо должно оставаться ребёнком на всю жизнь.}
{Длинный-Мокрый-Хвост.
Афоризм}

\epigraph{
\ml{$0$}
{В классическом цатроне есть слово <<тхвал'кикхвал>>.}
{There is a word ``tchu\={a}l'k\'{\i}kchu\r{a}l'' in Classical Te's\'{a}tr\v{o}n.}
\ml{$0$}
{Оно приблизительно переводится как <<что испытывает тот, кто не хочет плавать в дерьме, но не может объяснить почему>>.}
{It can roughly be translated as ``how it feels when you don't want to swim in shit and can't explain why''.}
}{
Аркадиу Люпино.
<<Генетический анализ старых языков Тра-Ренкхаля>>
}

\epigraph
{Когда я вижу в саду пробитую тропу, я говорю садовнику: делай здесь дорогу.}
{Александер I, король государства Руссия.
Эпоха Господина}

\epigraph
{Моё тело --- мой храм, но я не прихожанин, а архитектор.}
{Мартин Охсенкнехт}

\epigraph
{Тиран должен быть изобличён и опозорен при жизни, иначе он станет знаменем для следующего.}
{Анатолиу Тиу}

\epigraph
{Однажды будут оценены и истина, и ложь.
Истина восторжествует.
Честное заблуждение помянут добрым словом.
Ложь забудут, за исключением самой искусной, ибо искусство лжи по сути своей так же привержено истине, как и прочие искусства.}
{Ликан Безрукий}

\epigraph
{Тот, кто бежал за мечтой и красил лицо в её цвета, никогда не ударит ребёнка и не посадит зверя на цепь.}
{Пословица тенку}

\epigraph
{Когда персидского шаха, гостившего в Англии, пригласили на скачки, он отказался.
<<Я и так знаю, что одна лошадь бегает быстрее другой>>, --- сказал он.}
{Эльза Дункан.
<<Теория игр в Эпоху Богов>>}

\epigraph
{...И хотя многие небезосновательно считали Тациана [Освободителя] тщеславным властолюбцем, для своих подданых он был истинным королём.
Когда Тациана казнили, вместе с ним повесили его соратников и друзей, также многих других безвинно --- лишь за слово или дерзкий взгляд.
Очень удобно казнить безвинных, выставляя их мятежниками --- ведь о них составляют впечатление о всём мятеже.
Эти люди плакали и молили о пощаде, когда церковники обещали им Inferno\FM\ и Damnacion Memoria\FM.
И Тациан, доселе хранивший молчание, вдруг крикнул: <<Успокойтесь.
Если после смерти что-то есть, мы все пойдём туда вместе.
Кем бы вы ни были, я вас не брошу.
Я буду биться за вас даже с дьяволом, если придётся>>.
Плач прекратился, и до самой смерти приговорённые не издали ни звука, несмотря на все усилия палачей.}
{Анатолий Тий.
<<Лиманские записки>>}
\FA{
Подземный мир (t-sl.).
}
\FA{
Осквернение памяти (t-sl.).
}

\epigraph
{Культура, как река, часто образует водовороты, в которых на первый взгляд действуют совершенно другие законы логики.
Иногда водоворот кажется настолько большим, что вызывает опасения за судьбу культуры в целом.
Эти опасения беспочвенны --- река всегда течёт в одном направлении.}
{Кельса Пушистая}

\epigraph
{Есть легенда про принца Валерия.
Однажды он участвовал в конных скачках и занял восьмое место, уступив пятерым лордам и двоим пришлым торговцам из кочевых племён.
Судья настолько растерялся, что долго не мог огласить список победителей, ведь среди зрителей был и сам король --- грозный Лауренций Фульминикула.
Тогда Валерий взял у судьи список, назвал имена победителей и лично вручил им награды.
Валерий показал людям а тот день: истинный судья всегда вне игры, даже если он шёл к финишу вместе со всеми.
Правление Валерия знаменует начало расцвета, вошедшего в историю как Старшие Валериды.}
{Клаудиу Дентосиу.
Предисловие к <<Воспоминаниям о великом тёзке>>}

\epigraph
{Правда и ложь не имеют ничего общего с законом.}
{Пословица Тысячи Башен}

\epigraph
{При изучении языка вас может настичь странное чувство.
Это сродни влюблённости --- вы можете рассматривать прочие языки, не иначе как сравнивая их с объектом вашей страсти.
Моей страстью стал энглис, язык Древней Земли.
Его потомки за миллион лет разошлись по Вселенной, изменившись до неузнаваемости, но каждый раз, когда я нахожу знакомые звуки, корни и грамматические конструкции, моё сердце пропускает один удар.}
{Кельса Пушистая}

\epigraph{Те рыбы, которые впервые в истории вылезли на сушу, очень плохо плавали.}
{Длинный-Мокрый-Хвост}

\epigraph
{Где нет людского, не место мне.}
{Строка из песни Ликхмаса, героя <<Легенды об обретении>>}

\epigraph
{Негуманная идеология --- почти всегда женоненавистническая.
И лишь иногда --- мужененавистническая.}
{Мариам Кивихеулу}

\epigraph
{Молчание --- самая удобная форма лжи.}
{Даниил Гранин, Эпоха Последней Войны}

\epigraph
{Главное для мыслящего существа --- уметь видеть безобразное.
Сапиент может быть сколь угодно восприимчив к тонкому и прекрасному, но, не замечая безобразное, он превратит свою жизнь в кошмар.}
{Длинный-Мокрый-Хвост}

\epigraph
{Чувство вины --- отличный способ манипуляции.
Иногда лучше поступиться принципами, нежели позволить себе быть виноватым.}
{Клаудиу Дентосиу}

\epigraph{
\ml{$0$}
{Разница между смирением раба и смирением того, кто боролся --- в той вмятине, которая осталась на броне противника.}
{Between humility of slave and humility of the one who fought lies a dent in the enemy's armour.}
}{Пословица Преисподней}

\epigraph
{Компромисс --- это искусство разделить пирог так, чтобы каждый думал, что ему достался самый большой кусок.}
{Людвиг Эрхард, Эпоха Последней Войны}

\epigraph{
\ml{$0$}
{Начинай, если хочешь.}
{Begin if you want.}
\ml{$0$}
{Продолжай, если нравится.}
{Continue if you like.}
}{Пословица ноа}

\epigraph
{Если вам говорят о том, что вы обязаны любить --- вас пытаются изнасиловать.
Кричите как можно громче.}
{Ветер-Стрекозьих-Крыльев, идеолог ранних тси}

\epigraph
{Задача правителя --- указать человеку его место.
Задача хорошего правителя --- помочь человеку отыскать своё место.
Задача лучшего из правителей --- не мешать человеку искать своё место.}
{Франциск IV, последний папа римский}

\epigraph
{Небесным телам, листьям деревьев, глазам птиц и человеческой крови совершенно не важна красота твоих изречений.
У них своя поэзия и свои законы.}
{Эрхэ Колокольчик}

\epigraph
{Каким, должно быть, смелым было животное, которое впервые в истории почувствовало боль.}
{Длинный-Мокрый-Хвост}

\epigraph
{Большая часть высоких технологий замечательно просты по своей сути.
Я не верю, что даже в случае глобального катаклизма мы скатимся до каменных молотков.
Пока жив хотя бы один человек с горящими глазами --- технологии будут сохранены, будут приспособлены к инфраструктуре и будут применяться.
Как человек не может разучиться велосипедной езде или плаванию, так и человечество, единожды научившись, уже не сможет это позабыть.}
{Михаил Кохани, презентация <<спичечных>> технологий в Массачусетском технологическом университете}

\epigraph
{Среднестатистический практически здоровый человек способен встроиться в любую социальную систему, имеющую простые и понятные правила.
<<Любую>>, к сожалению, означает и <<сколь угодно бесчеловечную>>.}
{Мариам Кивихеулу}

\epigraph
{Не тратьте время на сожаления.
Ваши предпочтения --- это не судьба, а результат удачного стечения обстоятельств.
Кое-кто считает, что три поворота налево и один направо приводят к одному и тому же результату.
Как показывает практика, результат в этих случаях всегда разный.}
{Людвиг Вейерманн}

\epigraph
{Все хотят тепла, но никто не хочет гореть.}
{Пословица Преисподней}

\epigraph
{...И лишь одного я боялась всегда ---\\
Что море меня не излечит.}
{Эрхэ Колокольчик}

\epigraph
{Будьте гордыми.
То, что вас согнули сегодня --- лишь стечение обстоятельств.
Меня часто унижали, когда я не могла дать отпор.
Но и пусть.
Я ничего не прощала и не прощу.
Вставайте на колени, если вас ставят насильно.
Молчите, если вам затыкают рот кляпом.
Признавайтесь, если вам угрожают оружием.
Это и вполовину не так ужасно, как оправдывать преступления или отыгрываться на других.
Будьте гордыми, берегите себя, свои силы и достоинство --- завтра ветер переменится.}
{Мариам Кивихеулу}

\epigraph{
\ml{$0$}
{Научись улыбаться в ответ.}
{Learn to smile back.}
}{Призказка ноа}

\epigraph
{\ml{$0$}
{Сомневаешься --- оставь меня в ножнах.}
{If in doubt, keep me untouched.}}
{Гравировка на сабле Митхэ ар'Кахр}

\epigraph
{Жить интересно именно потому, что мир не соответствует нашим ожиданиям.}
{Людвиг Вейерманн}

\epigraph{
\ml{$0$}
{Битва может стоить жизни, но без битвы ты не узнаешь, что такое жизнь.}
{Battle may take your life, but escaping battle you'll never know what life is.}
}{Присказка наёмников Фоуф}

\epigraph
{Мир изменится.
Он уже меняется.
И это случится, даже если вы сегодня проломите мне голову.
Потому что любовь и здравый смысл хоть и не сразу, но всегда побеждают ненависть и абсурд.
Так устроен этот мир, и я тут ни при чём.}
{Мариам Кивихеулу}

\epigraph
{Уничтожение произведений искусства должно стать табу для любого политического или общественного деятеля.
Вандал, какие бы прогрессивные идеи он ни нёс, никогда не вызовет сочувствие у общества.}
{Мартин Охсенкнехт}

\epigraph
{Ах ты, молодой Добрыня Никитич!
Бился ты со змеёй да трое суток, потерпи ещё три часа!
Ты побьёшь змею да ю, проклятую!}
{Фольклор культуры Руса, Древняя Земля}

\epigraph
{Понесший наказание безвинно имеет право совершить преступление того же характера, если невиновность будет доказана.}
{Третий постулат Возмездия.
Законы ноа}

\epigraph
{Если вы хоть раз в жизни выключили компьютер, разъединили телефонное соединение, убили комара или вкололи себе антибиотик, вы уже владеете основами интерфекции.}
{Элла Рид, основатель интерфекции}

\epigraph
{Лучший учитель --- ученик, который только что понял.}
{Пословица сели}

\epigraph
{Если дуть против ветра, то ветер не изменится, но зато перед твоим носом всегда будет островок штиля.}
{Пословица ноа}

\epigraph
{Есть люди, подобные кострам --- кормят и согревают целый лагерь, но тухнут в одиночестве.
Есть люди, подобные маякам --- не греют, но светят, даже если на горизонте нет ни одного корабля.}
{Пословица ноа}

\epigraph
{Может быть, стремление к власти и портит людей, но угроза потери власти превращает их в диких зверей и лишает всего человеческого.}
{Мариам Кивихеулу}

\epigraph
{Кости не ломаются от усталости, кости ломаются от чрезмерных усилий.
Стену следует строить маленькими камешками.}
{Пословица ноа}

\epigraph
{Если замолкает последняя певчая птица --- значит, дело действительно плохо.}
{Пословица сели}

\epigraph
{Кровь вытекает вместе с заразой, страх вытекает вместе с волнением.}
{Присказка хака}

\epigraph
{Сплошные солнечные дни порождают пустыню.}
{Пословица сели}
% Японская пословица

\epigraph
{Лучше быть врагом хорошего человека, чем другом плохого.}
{Пословица Преисподней}
% Японская пословица

\epigraph
{Быстро --- это медленно, но каждый день.}
{Пословица сели}

\epigraph
{Лето --- это танец, и глупо не принимать в нём участия.}
{Пословица Драконьей Пустоши}

\epigraph
{Иерархия разрушается в тот момент, когда члены сообщества отказываются играть по его правилам, добровольно меняя возможное лидерство на стигму нижней ступени.
Эти люди показывают прочим главное --- можно жить, самореализовываться и радоваться жизни, будучи <<отбросом общества>>.
Поэтому лидеры будут препятствовать этому, делая жизнь таких <<добровольцев>> невыносимой и даже устраняя их физически.}
{Мариам Кивихеулу}

\epigraph
{Ищи счастье.
Судьба найдёт тебя сама.}
{Клаудиу Дентосиу}

\epigraph
{Самая ужасная для государства вещь --- страх воина.
Страх воина питает диктатуру.
Внуши воину, что он в безопасности --- и диктатура падёт.
Внуши воину, что он любим народом --- и он будет с народом до конца.
Пусть каждый народ поклянётся оберегать своих воинов, как воины клянутся защищать народ --- и для человечества не будет больше плохих лет.}
{Анатолиу Тиу.
<<Послание к девяти завоевателям>>}

\epigraph
{Я несчастен, но счастливее меня не найти.}
{Людвиг Вейерманн.
Предсмертная записка}

\epigraph
{Если интерес к происходящему пересилил прочие чувства --- значит, вы победили в главном сражении своей жизни.
Все остальные победы --- вопрос времени.}
{Михаил Кохани.
Речь на вручении Расширенной Нобелевской премии.}

\epigraph
{Крысы боятся света.}
{Фризская пословица}

\epigraph
{Выучи один язык --- второй дастся тебе проще.\\
Научись играть на флейте --- цитрой овладеешь без труда.\\
Подружись со швейной иглой --- резец плотника сам прыгнет в твои руки.}
{Пословица ноа}

\epigraph
{Я меняю дни на расстояние\\
От Тси-Ди до джунглей Тра-Ренкхаля,\\
Мне доступно тайное знание ---\\
Как свернуть пространство-время желанием.}
{Песенка Заяц}

\epigraph
{Я никогда не был на войне.
Но если бы пришлось, я стал бы диверсантом или дезертиром --- и то и другое требует большой смелости.}
{Бенедикт Альсауд}

\epigraph{
\ml{$0$}
{Уважение тебе ничего не стоит.}
{Respect costs you nothing.}
}{Мариам Кивихеулу}

\epigraph
{Подумать надо и о них ---\\
Что надо, то надо,\\
Да вот поди-ка отыщи\\
Отбившихся от стада!\\
~\\
Живым идти под вострый нож\\
Тож неохота им,\\
Ищи! Но то, что ты найдёшь,\\
Будет ли живым?}
{Торстейн фра Хамри.
<<Поздней осенью в поисках овец>>}

\epigraph
{Вклад личности в историю принципиально не поддаётся оценке.
Любые изыскания на эту тему --- не более чем спекуляция.}
{Людвиг Вейерманн}

\epigraph
{Проблема мира --- не в отсутствии любви, а в неумении её выражать.}
{Мариам Кивихеулу}

\epigraph
{Мы меняемся вместе с миром, словно рисунок небес, и если среди ясного дня грянул гром --- значит, ты просто не заметил грозу.}
{Пословица тенку}

\epigraph
{Великий умеет находить своё истинное предназначение.
Счастливый умеет вовремя от него избавиться.}
{Клаудиу Дентосиу}

\epigraph
{Есть три верных признака государства, находящегося в состоянии гражданской войны --- разбитые дороги, разрушенные больницы и закрытые школы.
Хуже всего, если разбиты дороги в городах и деревнях.
Ведь для хороших дорог нужно лишь, чтобы два добрых соседа по две стороны от дороги могли что-то сделать сообща.}
{Марке Скрипта}

\epigraph
{Мироздание не знает, что такое благодарность.
Делаешь что-то полезное --- позаботься о награде сам.}
{Длинный-Мокрый-Хвост}

\epigraph{
\ml{$0$}
{Дорога излечит всё.}
{The road heal you.}
\ml{$0$}
{Или убьёт.}
{Or kill you.}
}{Пословица ноа}

\epigraph
{Неопытный кушает впрок;
мудрый берёт пищу с собой.}
{Пословица сели}

\epigraph
{<<Возлюби ближнего своего>> прежде всего означает <<Оставь ближнего своего в покое>>.}
{Фридрих Ницше.
Эпоха Господина}

\epigraph
{Старинный рецепт идеальных солдат: взять молодых, лишить знаний и наставлений, закрыть в клетки и вынудить выживать, а потом дать выжившим идеологию и пообещать им любые блага.}
{Присказка наёмников Фоуф, Тысяча Башен}

\epigraph
{Если в войне между правителями участвуют солдаты, но нет наёмных убийц --- никакой войны нет, есть лишь игра.
Только игроки жертвуют фигурами, не пытаясь сунуть друг другу нож.}
{Присказка наёмников Фоуф, Тысяча Башен.}

\epigraph
{Смерть может быть привлекательной в двух качествах --- как выход и как неизведанное.}
{Постулат Эволюциона}

\epigraph
{Когда с собаки снимают чересчур суровый ошейник, она рычит и кусается от боли.}
{Клаудиу Дентосиу}

\epigraph
{Я бы тоже хотел мир, где мне не пришлось бы доказывать своё право, ломая чужие копья, разрывая путы и обходя волчьи ямы.
Но мира кроме этого у меня просто нет.}
{Гало Кровавый Знак}

\epigraph
{Жестокость --- храбрость трусов.}
{Фазиль Искандер, Эпоха Последней Войны.}

\epigraph
{Тюрьма --- это судьба любого преступника.
Диктаторы и коррумпированные чиновники сами строят вокруг себя тюрьмы, и эти тюрьмы лишь по недоразумению называют дворцами и крепостями.}
{Анатолиу Тиу}

\epigraph
{At satis hostium.
Эта фраза была эпитафией некоего Матиаса Стойса, похороненного в Кафедральном соборе Кёнигсберга.
Сейчас, к сожалению, барельеф не сохранился, но его застал мой прапрадед, Алекс Орлов.
Он был военным.
И однажды увидел эту надпись, прогуливаясь по городу.
Через несколько дней он уволился из армии, а ещё через год основал Общество Сломанного Копья, помогающее военным адаптироваться к гражданской жизни, получить образование и работу.
Сейчас эта фраза --- at satis hostium --- выбита на его могиле.
Прапрадед был убит милитаристами спустя семь лет, когда Общество Сломанного Копья распространилось по трём континентам.}
{Мемуары Михаила Кохани}

\epigraph
{Когда человек мучается, у него появляется потребность мучить других.}
{Ромен Роллан.
Эпоха Последней Войны}

\epigraph
{At satis hostium\FM.}
{Эпитафия Матиусу Стойусу Прусскому.
Кафедральный собор Кёнигсберга, домен Европа, Древняя Земля}
\FA{
Но хватит врагов (эллатинский).
}

(Последняя встреча Гало и Тахиро!)

\epigraph
{Лгать --- как нести скользкий чан под дождём: чем дальше, тем тяжелее;
не уронишь чан, так расплещешь воду.}
{Пословица ноа}

\epigraph
{Правду можно оставить на дороге;
ложь приходится нести с собой.}
{Пословица сели}

\epigraph
{Хрупкая птица пролетит расстояние, которое не сможет пройти самый сильный ягуар.}
{Пословица сели}

\epigraph
{Вера обычно начинается там, где кончаются силы и фантазия.}
{Пословица сели}

\epigraph
{Пройди лигу\FM, перекати бусинку.
Чётки покажут путь, ноги приведут.}
{Присказка ноа}
\FA{
Лига --- мера длины ноа.
Соответствует ровно 1.5 кхене.
}

\epigraph
{Даже самый могущественный --- часть целого.
Помня об этом, правитель не опустится до диктатора, а народ не допустит попрания своих прав и свобод.}
{Анатолиу Тиу}

\epigraph
{Королева не кормит грудью, король не учит сыновей, братья режут друг друга ради трона, а дочерей используют как разменную монету.
Королевская семья --- извращение самой сути семьи.
Должна ли она говорить добрым талианцам, как им жить?}
{Тациан Освободитель}

\epigraph
{Учи дитя порядку, но помни, что дисциплина --- кормилица лжи.}
{Поговорка сели}

\epigraph
{Совершать безумства --- это способность.
Совершать безумства без сожалений --- это дар.}
{Михаил Кохани}

\epigraph
{Нельзя недооценивать Эпоху Последней Войны.
Именно тогда, в огне вооружённых конфликтов, революций, забастовок, военных переворотов и уродливых, саморазрушительных идеологий ковалась самая жизнеспособная, самая гуманная философия первых людей, обеспечившая почти двадцатидвухтысячелетний период мира, процветания и расселения Ветвей Земли по Вселенной.}
{Кельса Пушистая}

\epigraph{
\ml{$0$}
{Счастливые женщины --- хорошая деревня.}
{Happy women mean a healthy village.}
}{Хаяо Миядзаки}

\epigraph{
\ml{$0$}
{Есть высоты, на которых худший способ висеть лучше падения.}
{There are some heights, where the worst way to hang on is better than to fall off.}
}{Пословица хака}

\epigraph
{Если из жилища видно не красоты природы, а лишь окружающие его стены --- это осквернение самой идеи жилища.}
{Клаудиу Дентосиу}

\epigraph{
\ml{$0$}
{Всегда будь готов к тому, что ничего не произойдёт.}
{Always expect nothing to happen.}
}{Длинный-Мокрый-Хвост}

\epigraph
{Великий талант --- жить в неопределённости без иллюзий и без тревог.}
{Людвиг Вейерманн}

\epigraph
{Мы не шли на войну, чтобы убивать или быть убитыми.
Мы шли на войну, чтобы нас услышали.}
{Субкоманданте Инсургенте Маркос.
Эпоха Последней Войны}

\epigraph
{Что может быть прекрасней, чем дышать?\\
Быть может, к новой жизни\\
родиться из тяжёлых снов дневных?\\
~\\
Проснулся ---\\
и уже не помнишь,\\
что разбудило:\\
достаточно любить себя,\\
чтобы проснуться.\\
~\\
А нетерпенья\\
достаточно, чтобы сгореть\\
для повторенья.}
{Торстейн фра Хамри, <<Нетерпение>>.
Эпоха Последней Войны, Древняя Земля}

\epigraph
{Утонул в колодце --- беда всего города.
Утонул в море --- только твоя беда.}
{Пословица ноа}

\epigraph
{Все жизненные пути бессмысленны, но есть путь сердца.
Он такой же бессмысленный, как и остальные, но по нему идёшь с радостью.}
{Карлос Кастанеда}

\epigraph
{Экспертом является тот, кто совершил все возможные ошибки в некотором узком поле.}
{Нильс Бор}

\epigraph
{В человеке нет ни одного качества, которое когда-либо не поспособствовало его выживанию.
Просто помните об этом, когда решите что-либо осудить.}
{Март <<Одноглазый>> Митчелл}

\epigraph
{Многие сейчас говорят о бедах, которые принесли слепая вера и фанатизм.
Упаси меня Господь отрицать очевидное.
Но нужны ли они человечеству?
Нужны!
Не расчётливые заселили самые дальние берега Земли, и не расчётливым идти на край Вселенной!}
{Март <<Одноглазый>> Митчелл}

(этот эпиграф однозначно к <<Кон-Тики>>)

\epigraph
{Он обрёл почти безграничную власть, просто перестав осуждать.
Эта власть ограничивалась лишь его собственными принципами.}
{<<Речь о жреце>>.
Речь ноа}

\epigraph
{Насыщает похлёбка, а не просиженное в харчевне время.}
{Пословица сели}

\epigraph
{Спроси себя в печали, спроси себя истекающего кровью, спроси себя усталого и измотанного, желал бы ты иной жизни?
Так познаются цветение духа и правильность пути.}
{Мартин Охсенкнехт}

\epigraph
{Соседи твои и друзья\\
Ходят по городу сытые,\\
Но за смехом таится молчание,\\
А в молчании дремлет бунт.}
{Торстейн фра Хамри, <<Дух времени>>.
Эпоха Последней Войны, Древняя Земля}

\epigraph
{Умеющий шагать не оставляет следов.
Умеющий говорить не допускает ошибок.
Кто умеет считать, тот не пользуется счетом.
Кто умеет закрывать двери, тот не употребляет запор и закрывает их так крепко, что открыть их невозможно.
Кто умеет завязывать узлы, тот не употребляет веревку, и завязывает так прочно, что развязать невозможно.}
{<<Книга Пути и Достоинства>>, Мудрый Старец.
Культура Цина, Древняя Земля}

\epigraph
{Познание себя и прочей Вселенной имеет чересчур памятный горько-сладкий вкус;
того, кто распробовал его однажды, может остановить только смерть.}
{Клаудиу Семито Фризский.
Эпилог к <<Воспоминаниям о великом тёзке>>}

\epigraph
{Чем больше обязанностей по самообеспечению вы возлагаете на окружающих, тем сильнее от них зависите.
Не добываете руками хлеб?
Приготовьтесь голодать.
Не умеете мыть полы?
Приготовьтесь существовать в грязи.
Лет триста назад я бы воскликнул: <<И не приведи судьба вам отдать ртуть\FM\ и железо в чужие руки!>>
Малу вайю\FM, поздно --- это наша действительность.
Казначеи и ростовщики грабят вас, выменяв вашу ртуть на бумагу.
Король велит закапывать фальшивомонетчиков, но при каждом удобном случае платит вам медной амальгамой.
Солдаты и стражники убивают вас оружием, которое выковали вы из вами добытого железа.
И самое страшное --- вы всё ещё считаете, глупцы, что это в порядке вещей.}
{Анатолиу Тиу.
Речь перед жителями Фриза}
\FL{viola}{Фиола}
\FA{Malu vaeu --- Увы! О горе! Чёрт возьми! (s-l, примерный перевод)}

\epigraph
{Твердое и крепкое это то, что погибает, а нежное и слабое есть то, что начинает жить.
Сильное и могущественное не имеют того преимущества, какое имеют нежное и слабое.}
{<<Книга Пути и Достоинства>>, Мудрый Старец.
Культура Цина, Древняя Земля}

\epigraph
{Принятие конкретной социальной установки --- это всегда давление на суперпозицию личности.
Счастье в том, чтобы давление казалось объятиями, а не ломало рёбра.}
{Мариам Кивихеулу}

\epigraph
{Жизнь слепа и путь прокладывает на ощупь.
Высокий интеллект --- это та широкая лазейка, которую жизнь нащупала в практически непреодолимом для неё препятствии --- космическом пространстве.}
{Софиа Ловиса Карма}

\epigraph
{В искусстве невозможно опоздать.
Когда бы ни было создано произведение, оно появляется вовремя.}
{Мартин Охсенкнехт}

\epigraph
{Тот, кто впервые обругал соплеменника вместо того, чтобы ударить его, стал прародителем цивилизации.}
{Длинный-Мокрый-Хвост}

\epigraph
{Биология с её естественным отбором гуманнее большинства представителей человеческого рода.
Так что оправдывать несправедливости биологией оставьте лжецам для неучей.}
{Сергей Леонидовиц Хистиаков}

\epigraph
{Если какой-то деятель искусства скажет при вас, что художники, писатели, поэты и драматурги двигают мир вперёд --- можете громко рассмеяться ему в лицо.
Мир двигали и двигают те, кто изредка, раз в месяц или год идёт в театр или читает книжку, чтобы расслабиться и назавтра, после здорового крепкого сна снова делать то, что должно --- подметать улицу, чинить машины, учить детей и лечить больных.}
{Мартин Охсенкнехт}

\epigraph
{Всё, что пытаются доказать силой --- розгами, кулаками, резиновыми дубинками или пулемётами --- скорее всего, ложь;
ни одному первооткрывателю ещё не приходилось бить или убивать оппонентов, чтобы убедить их в своей правоте.}
{Михаил Кохани}

\epigraph
{Дьявол начинается с пены на губах ангела, вступившего в бой за правое дело.}
{Грегори Померантс}

\epigraph
{Каждое заблуждение имеет цену, и эта цена всегда измеряется в человеческих жизнях.
Исключений не бывает.}
{Марке Скрипта}

\epigraph
{Врач и философ Марке Скрипта является изобретателем термина <<тирания мозга>>.
Так он называл состояние, когда человек, отождествляя себя исключительно с нервной системой, сосредотачивается на её внутренних задачах, игнорируя или непомерно эксплуатируя при этом прочие системы организма.
Это же состояние я наблюдаю сейчас в нашем государстве;
Скрипта полагал, что тирания мозга есть первопричина многих болезней тела и всех болезней души.}
{Анатолиу Тиу}

\epigraph
{Тси вернутся.
Они вырвутся в явь из ваших кошмаров.
Они пробьются светом через ваши решётки, просочатся влагой через гранит ваших стен.
Они не станут мстить, но их возвращение положит вам конец.}
{Уэсиба Серозмей}

\epigraph
{У народа, живущего в долине посреди Хребта Малого Листопада, есть замечательный обычай.
Если кто-то совершил преступление, люди начинают заботиться о нём.
Ему напоминают о его самых лучших поступках, его обнимают и всячески проявляют к нему любовь.
Я скажу вам, что это маленькое отсталое племя знает истину, которую не могут осознать самые лучшие умы государя: преступник --- это тот, кто кашляет, когда общество больно.}
{Анатолиу Тиу.
<<Земной суд>>}

\epigraph
{Если вы услышите от меня в старости, что я отказываюсь от сегодняшних слов, пожалейте меня, но не верьте сказанному.
Предел есть у каждого.
Среди стариков много уставших, больных и сломленных;
трудно надеяться, что сия чаша минет мою голову.
Не верьте и прочим старикам, что публично отрекаются от прекрасных идей молодости.
Хоть историю и пишут достигшие преклонных лет, молодые гораздо чаще знают истину.
Так было и будет всегда.}
{Михаил Кохани}

\epigraph
{Ещё одно существенное отличие сели: их мифология носила \emph{исключительно игровой} характер.
Примером могут служить приведённые результаты опроса по <<небесному пахарю>>[44][136].\\
Взрослые[88] сели:\\
94\% при вопросе о сущности падающих звёзд, смеясь, рассказывали легенду о небесном пахаре, но при переводе беседы в серьёзное русло всё-таки рассказывали о летающих раскалённых камнях;\\
6\% безо всяких шуток давали совершенно точный ответ.\\
Взрослые[88] тенку:\\
10\% (в основном образованные монахи) давали правильный ответ и выказывали неудовольствие, услышав легенду о пахаре;\\
62\% совершенно серьёзно говорили о проделках богов и удивлялись рассказу о летающих камнях;\\
28\% промолчали или ответили <<не знаю>>, отклонив таким образом предложение побеседовать.\\
Ещё сильнее эта разница заметна у детей.\\
Прошедшие Отбор дети[88] сели:\\
78\% рассказали легенду о пахаре.\\
После вопроса о том, как всё обстоит <<на самом деле>>:\\
8\% затруднились ответить;\\
18\% спросили <<как?>>;\\
12\% начали фантазировать,\\
40\% упомянули о раскалённых угольках, летящих из очага\\
(таким образом детям объясняют сущность небесных тел в школе).\\
22\% детей отказались беседовать на обозначенную тему.\\
Выжившие дети[88] тенку:\\
56\% рассказали легенду о пахаре,\\
42\% уверенно повторили её после уточняющего вопроса,\\
4\% спросили <<как?>>,\\
10\% начали фантазировать.\\
44\% отказались беседовать на обозначенную тему.}
{Аркадиу Шакал Чрева.
<<Сели: первые впечатления>>}

\epigraph
{Генотип, медикация и обучение каждого тси должны обеспечивать отсутствие статистически значимой разницы между значениями двух параметров --- выживаемости и коэффициента личного комфорта\FM\ --- в присутствии технологической поддержки и при полном отсутствии таковой.}
{Второй постулат Кодекса Тси-Ди}
\FA{
Коэффициент личного комфорта (КЛК) --- параметр, определяющий степень раскрытия возможностей сапиента в заданных условиях.
В настоящее время считается устаревшим.
}

\epigraph
{Не найти прекраснее цветов преисподней, но цветение их мимолётно.}
{Анатолиу Тиу.
<<Путешественник>>, диалоги}

\epigraph
{--- Я слеп, отче.
Как мне увидеть, что есть добродетель и порок?\\
--- Всё вышло из крови и уйдёт в землю, сын.}
{Анатолиу Тиу.
<<Путешественник>>, диалоги}

\epigraph
{... Увы, но бывают моменты, когда на одной чаше весов лежит отрубленная рука, а на другой --- судьба, жизнь и смерть.
Казалось бы, выбор очевиден для всех --- спросите любого зеваку на улицах нашей столицы.
Но когда этот момент настает, рука почему-то перевешивает...}
{Анатолиу Тиу.
Трактат <<О природе дышащих>>}

\epigraph
{Салафиты нарисовали на своём знамени перечёркнутый посох?
Понимаешь ли ты, что это значит?
На их знамени --- твой знак, Леам!
Они признали поражение, признали, что в дереве их собственного учения соков больше нет...}
{Анатолиу Тиу.
Письмо к Леаму эб-Салаху}

\epigraph
{Не бойтесь гнева предков, бойтесь стыда потомков.}
{Анатолиу Тиу.
<<Буря перемен>>, часть III}

\epigraph
{Ты говоришь, что принёс свободу, и гордишься данными тебе народом титулами?
Ты раб с рождения, Татиан, ещё худший, чем те, кого ты за собой ведёшь.
Меня называют светилом, философом, певцом свободы, но единственная свобода, которую я хочу --- это свобода не быть тем, кем меня называют, и не делать то, что от меня ждут!}
{Анатолиу Тиу.
Ответ на призыв Татиана Освободителя оказать его движению поддержку}

\epigraph
{У гаруспиков всегда будет множество клиентов.
Людей волнует, когда они разбогатеют, найдут любимого человека, умрут...
Меня волнует только один вопрос --- когда я окончательно сдамся.
За ответ я бы отдал всё своё состояние, но увы --- его не даст ни один гаруспик.}
{Анатолиу Тиу.
Речь перед лиманскими повстанцами}

\epigraph
{Для нас есть вероятности, но нет предопределения.
Для мироздания есть предопределение, но нет вероятностей.
Непроницаемое тёмное пятно вероятностей в вашем будущем --- это и есть то, что вы называете <<свободой воли>>.}
{Анатолиу Тиу.
Ответ на шариатском суде}

\epigraph
{Сожжённым вами люди будут дышать.}
{Последние слова Анатолиу Тиу перед смертью.
Шариатский суд постановил <<опалить лицо>> уже престарелого Тиу лишь теми книгами, в которых явно доказано присутствие ереси.
К чести инквизиторов и к несчастью старика, таких книг хватило на костёр.}

\epigraph
{То, что было взято дланью войны\FM, будет потеряно...
Завоеванная женщина не будет верна;
завоеванная страна вернет себе свободу, едва окрепнув, часто под знаменами шовинизма и религиозно-морального фанатизма, этих извечных суррогатов потерянного достоинства.
Посему опытный правитель должен делать ставку на договорные отношения... и помнить, что слабые нуждаются в уважении куда больше сильных.}
{Анатолиу Тиу.
<<Послание девяти завоевателям>>}
\FA{
Дланью войны (t-sl: way mana militharrhio) --- с помощью военной силы.
}

\epigraph
{В семечке --- фрукты и алый цветок.\\
В ножнах таится убийцы бросок.\\
Форму узри --- и узришь содержанье.\\
В лицах и благо ищи, и порок.}
{Клаудиу Семито Фризский.
<<Песенки Клаудиу>>, приложение III к <<Воспоминаниям о великом тёзке>>}

\epigraph
{--- Ты обвиняешься в организации культа с целью подрыва королевской власти, --- монотонно сказал судья.
--- Что ты можешь ответить на это?\\
--- Культ правильности сменяется культом безумия.
Культ поклонения сменяется культом противостояния.
Ваше обвинение похоже на обвинение пастуха, у которого угнали стадо баранов.\\
Люди в зале --- как сторонники, так и противники Клаудиу --- вскочили на ноги и начали выкрикивать в адрес поэта оскорбления.
И только три или четыре человека молча чему-то улыбнулись...}
{Клаудиу Семито Фризский.
<<Воспоминания о великом тёзке>>}

\epigraph
{Клаудиу в последний раз посмотрел на оставленный им город.
Люди вокруг яростно смотрели на него и тискали пальцами оружие, но никто не решался напасть --- поэта верно хранило мастерство убийцы и королевский эдикт.\\
<<\ldots из величайшей королевской милости рекомого Клаудиу отпустить и хранить его от всяческих посягательств>>.
Милости и страха перед восстанием.
Никто не хотел, чтобы Клаудиу превратился в легенду и знамя для обездоленных.\\
Поэт поправил котомку и, весело насвистывая любимую песенку, направился в сторону побережья.
Он не подозревал, что стал легендой уже при жизни.}
{Клаудиу Семито Фризский. Эпилог к <<Воспоминаниям о великом тёзке>>}

\epigraph
{Нет любви без знания, а невежество --- лучшая пища для ненависти.}
{Пословица народа сели}

\epigraph
{Когда Клаудиу подвели к гильотине, один из министров Валериу Х насмешливо спросил, готов ли поэт умереть за свои идеи.
<<Я готов даже к тому, что они окажутся глупыми и никчёмными.
А уж умру я за них с большим удовольствием>>, --- ответил Клаудиу и, не дожидаясь приказа, положил голову на плаху.}
{Клаудиу Семито Фризский. 
<<Воспоминания о великом тёзке>>}

\epigraph
{Мало ему победы, он и смерть забрал!}
{Изречение, выбитое на гробнице Велира IV Хореида, принадлежавшее его врагу Вериту Валериду}

\epigraph
{Посмотри на волосы врага --- и ты поймёшь, на каком языке пойдёт беседа.}
{Пословица народа сели.
Купцы, исполняющие роль дипломатов, носили хвостик или пучок на затылке и вели переговоры на языке цатрон.
Воины обычно стриглись коротко и использовали <<язык стали>> --- боевые искусства.}

\epigraph
{Сжатая в кулаке крыса может отгрызть руку.
Загнанный в угол заяц может убить охотника.
Отчаявшийся тигр может истребить целую деревню.
На что же способен человек, которому нечего терять?}
{Пословица народа сели}

\epigraph
{Жизнь --- испытание для любого сапиента.
Веками мы придумывали сказки, чтобы оградить себя от естественного желания положить ей конец.
Время сказок прошло.
Наша задача --- помочь всем и каждому найти место в жизни, на деле, а не на словах убедить сапиентов, что их существование необходимо и страдания оправданы.}
{Постулат Эволюциона}

\epigraph
{Утопия подразумевает людей, которые довольны своей жизнью.
Однако всегда появляются бунтари, которых что-то не устраивает.
Человек меняется, следуя за условиями среды, общество меняется вслед за человеком.
Иметь претензии не есть плохо.
Желать изменений к лучшему --- это не подрыв устоев, а движение вперёд.
Звучит парадоксально, но если общество претендует на звание идеального, оно должно быть готово к любым изменениям.}
{Софиа Ловиса Карма, идеолог Эволюциона}

\epigraph
{Как я стал капитаном <<Тёмного пламени>>?
(смеётся) Просто пришёл в эту их комиссию и сказал: что за дерьмо, у меня под килем слишком мало парсак!
И парсак мне сразу отсыпали...}
{Бенедикт Альсауд.
Запись беседы с историками}

\epigraph
{В океане и в космосе нет границ.
Если кто-то хочет расчертить воду и святые небеса, пусть сначала озаботится суверенитетом собственной задницы --- возможно, она в рабстве у чересчур удобного кресла.}
{Бенедикт Альcауд.
Открытое письмо Транспортному комитету по поводу сомнительного законопроекта}

\epigraph
{Если ты проснулся бодрым --- значит, для тебя наступило утро.}
{Пословица тенку}

\epigraph
{Идеальное общество --- это общество, в котором даже чужак чувствует себя своим.}
{Критерий де Ла Роче.
Постулаты Эволюциона}

\epigraph
{Собака --- друг человека.}
{Изречение времён Древней Земли}

\epigraph
{Когда тси только начали подготовку к одичанию, они столкнулись с неожиданной проблемой --- прочие формы жизни Тра-Ренкхаля не могли соперничать даже с потерявшими большую часть технологий пришельцами.
Тси ловили зверей и птиц едва ли не голыми руками, что могло поставить десятки тысяч видов под угрозу исчезновения.
Воинственность аборигенов также могла сыграть с ними плохую шутку --- если тси переняли бы этот стереотип поведения, то аборигенов ждало неминуемое истребление.
Можно утверждать, что эти немногие тси с одним космическим кораблём в одночасье стали хозяевами планеты, и даже демиург Безымянный не питал никаких иллюзий на этот счёт[12].\\
Именно поэтому Баночка и Кошка разработали так называемую тактику форы (подробнее см. раздел 13.7.1).
Записей об этой методике осталось немного[13], но результаты исследований[14][15][16][22] говорят сами за себя --- тси провели беспрецедентный, грандиозный отрицательный отбор, сохранив многообразие видов.
Известно, что этот отбор производился с согласия и при поддержке Безымянного[12].
Именно благодаря дальновидности Баночки и Кошки люди Тра-Ренкхаля, нгвсо и дикие животные если и не стали тси достойными конкурентами, то хотя бы получили право на жизнь.}
{Аркадиу Шакал Чрева.
Тезисы работы <<Взаимодействие тси с биоценозом планеты Тра-Ренкхаль>>}

\epigraph
{Трусость --- враг процветания, а жестокость --- враг мира.
Да, я трус.
Во мне есть и жестокость.
Те, кто не знаком со мной лично, почувствовали эту жестокость в моём романе.
У многих из вас загораются глаза, когда вы читаете о крови и войне.
Мы --- потомки тех, кто убивал и насиловал на протяжении тысяч лет, кто раболепно склонялся перед силой.
Это их воинственные крики мы слышим в беспокойных снах, это их мысли мелькают на краю сознания, едва в наших руках оказывается оружие, это их страх пробирает нас до костей, едва оружие направляют на нас.
Но, клянусь пред лицом Господа, даже с таким <<наследством>> можно жить в процветании и сохранять на Земле мир!
И для начала будем честны перед собой и признаем простой факт: в нас течёт кровь тех, кто выжил благодаря жестокости и трусости.}
{Март <<Одноглазый>> Митчелл, идеолог Эволюциона, один из последних казнённых преступников в эпоху Последней Войны}

\epigraph
{Когда Тси-Ди была повержена Машиной, мы думали, что последний оплот свободных сапиентов Земли уничтожен.
Но сейчас я склонен полагать, что ситуация, поставившая тси на грань жизни и смерти, лишь укрепила их.
Большая часть тси была перебита, но выжило достаточно, чтобы через десять тысяч лет вернуться в виде новой, более страшной угрозы --- Скорбящих.
Это уже не просто потомки великих учёных --- это хоргеты, воины с непонятной, чуждой нам моралью, которые уверены в своей правоте и готовы сражаться за неё до конца.
Есть данные, что Скорбящие проходили подготовку по противодействию Чистилищу.
Эти существа использовали камеру пыток для тренировок!
Я надеюсь, все поняли и осознали глубину этого факта.
Это не искатели приключений.
Они знали, на что идут.\\
Больше мы не можем закрывать глаза на эту проблему.
Оставшиеся на Тра-Ренкхале тси должны быть подвергнуты геноциду, чтобы предотвратить пополнение рядов Скорбящих.}
{Самаолу Каменный Старик, член совета Ордена Преисподней}

\epigraph
{Если дать свободу опьянённому властью, он обязательно нарушит закон.}
{Анатолиу Тиу}

\epigraph
{Слушайте древние легенды.
Не обязательно им верить, просто слушайте.}
{Анатолиу Тиу.
Предисловие к <<Древнейшей истории Драконьей Пустоши>>}

\epigraph{Счастье --- синоним деградации.
Прогресс рождается в борьбе.
Его творят негодующие и скорбящие.}
{Изречение неизвестного Таракана, ставшее негласным девизом народа тси}

\epigraph{Ищите.
В мире много вещей, которые невозможно потерять.
Музыка и физика --- лишь некоторые из них.}
{Людвиг Вейерманн}

\chapter{Живая сталь}

\section{Первая кровь}

--- Баночка, пожалуйста! --- Кошка встала между плантом и воинами царрокх, не обращая внимания на направленные на неё копья.

--- Уйди, дура! --- заорал Баночка и, дёрнув женщину за руку, повалил её на траву.
Там, где они стояли миг назад, свистнули три стрелы.

--- Баночка! --- плакала Кошка, закрыв лицо руками.

--- Фонтанчик, отойди! --- завопил Баночка.
В его голосе сквозил дикий страх за нас.
--- Просто отойди!
Ты не осилишь этих копейщиков!
Возьми Кошку и иди к скалам!

Фонтанчик на пару секунд заколебался.
Я понял, что он хочет помочь другу, положившись на свою реакцию и металлические конечности;
но тон Баночки говорил сам за себя, а Фонтанчик привык доверять специалистам и тем более друзьям.
Поэтому в конце концов он кивнул и осторожно двинулся по направлению к лежащей Кошке.

Царрокх долго не решались нападать на планта, замершего в странной позе.
Он легко отбил саблей ещё две стрелы.
Один из воинов воспользовался всеобщим замешательством и швырнул копьё в Кошку;
Фонтанчик едва успел прикрыть её собой.
Обсидиановые лезвия чиркнули по обитым металлом рёбрам, и копьё отлетело в сторону.

--- Ха! Ха! Ха! --- хором зарычали царрокх.

--- ААААААААЭЭЭЭ! ---- взревел в ответ Баночка.
Я не подозревал, что маленький плант способен на такой звук.
Его голос потряс джунгли, словно рык старого ягуара.

Секунда --- и бойцы смешались в один большой, воющий от боли клубок.

\asterism

Фонтанчик молча оглядел поле боя.
Лечить здесь было некого, на каждой воине царрокх была как минимум одна не совместимая с жизнью рана.
В стороне лежали две головы отдельно от тел;
первого нападающего меч Баночки просто рассёк напополам, от ключицы до копчика.
Посреди всего этого кошмара неподвижно сидела маленькая жалкая фигурка.

--- Баночка! --- крикнула Кошка и понеслась вперёд.

Баночка поднял на нас расширенные, полные слёз глаза.
Его лицо было забрызгано кровью, губы и руки неистово тряслись.

--- Я убил человека, --- шёпотом сказал он.
--- Кошка, я убил человека...
Такого же, как ты, как Заяц, как Мак...

--- Баночка, ты ранен?
Он весь в крови...

--- Кровь красная, --- резонно сказал Фонтанчик, ощупывая друга.
--- Чёрного не вижу.
Его как будто вообще не задели...
Баночка, где болит?

--- Я убил человека...

--- Баночка, пожалуйста, мне нужно понять, что у тебя повреждено, --- взмолился Фонтанчик.
--- Кошка, его надо отнести к воде и обмыть, в этой грязи мы ничего не поймём...

--- Я убил человека, --- прорыдал Баночка и ничком свалился на землю.

--- Всё хорошо, мой милый, --- шептала Кошка.
--- Всё хорошо...
Мы целы, ты нас спас.
Иди ко мне на ручки, милый, вот так, всё хорошо...

--- Они были как ты, --- бессвязно плакал ей в плечо Баночка.
--- Как ты, Кошка.
Ты меня ненавидишь?

--- Я люблю тебя, милый.
Как я могу тебя ненавидеть...
Лежи, лежи, всё хорошо, я тебя донесу...

\asterism

--- Это удивительно, но он совершенно цел.
Пара растяжений из-за непривычной нагрузки, пара царапин от веток.
Основной удар нанесён по психике.
\ml{$0$}
{Я боюсь, что прежним он уже не будет.}
{I'm afraid he won't be the same anymore.''}

--- Но можно хотя бы сделать так, чтобы ему стало легче?

\ml{$0$}
{--- Не за одну ночь и даже не за десять, --- сказал врач.}
{``Not in one night, not even in ten,'' the doctor said.}
--- Я пока подержу его на успокоительных.
К вам будет просьба приходить почаще, ну и в целом не оставлять его одного, хотя бы первое время, а в идеале --- несколько оборотов.
\ml{$0$}
{Есть ещё одна проблема...}
{There's one more problem ....''}

\ml{$0$}
{--- Какая?}
{``What?''}

Врач помялся.

--- Баночка --- личность в некотором смысле известная, и у меня многие уже интересуются, что с ним случилось.
Я пока что ссылаюсь на врачебную тайну, но, боюсь, это только подогреет ненужные слухи.

\ml{$0$}
{--- Мы никому ничего не рассказываем, --- твёрдо сказала Кошка.}
{``We won't tell anyone anything,'' Cat firmly said.}

\ml{$0$}
{--- А рассказать-то надо, --- буркнул Баночка со своей постели.}
{``But it has to be told,'' Flask grumped from his bed.}

--- Мы думали, ты спишь, --- смутилась Кошка.

--- Не могу спать.
Успокоительные больше не действуют.

--- Ты можешь об том сказать мне, --- ответил врач.

--- Как я это сделаю на такой дозе седативных и миорелаксантов? --- резонно спросил Баночка.
Он едва шевелил губами.
--- Всё, хватит с меня химии.
Я уже четыре дня здесь лежу.
Отпустит --- пойду жрать.
Потом расскажу народу, что к чему.

--- Баночка, это дополнительный стресс для тебя.
Я не думаю, что...

\ml{$0$}
{--- По-твоему, две дюжины трупов царрокх никто не заметит?}
{``You think two dozens of dead \Tesarrokch\ remain unnoticed?}
\ml{$0$}
{На сколько баллов из десяти это осложнит наши и так не безоблачные дипломатические отношения?}
{On a scale of one to ten, how difficult will be our diplomatic relations, which, actually, have never been cloudless before?''}

\ml{$0$}
{--- Он прав, --- признал я.}
{``He's right,'' I admitted.}
\ml{$0$}
{--- Нужно рассказать всем.}
{``Everyone should know.}
\ml{$0$}
{И, как бы мне ни хотелось взять это на себя --- рассказать должен Баночка.}
{And as much as I want to take this burden, Flask must be the one to tell.''}

Баночка кивнул и взглянул на меня.

--- Ты был неправ, Небо.
\ml{$0$}
{Надо было сделать нелетальное оружие.}
{We should've made non-lethal weapons.}
\ml{$0$}
{Мы бы просто временно вывели этих воинов из строя, и ничего этого не было бы...}
{We could disable those warriors for a while, and none of that would've ever happened ....''}

Я кивнул.

--- Займёшься этим, как придёшь в себя?

--- Извини, --- покачал головой плант и, закутавшись в одеяло, отвернулся к стене.
\ml{$0$}
{--- С меня хватит.}
{``I'm done.''}

\section{Расставание}

--- Давай, Небо, --- убеждал меня Шмель.
--- С ними всё хорошо.
Окуклились правильно, ротики на месте, жопки там где надо.
Давай их сюда, я положу их в термостат с остальными куколками.

Я продолжал стоять, крепко прижав к себе малышей.

--- Мне кажется, что я не вернусь.

--- Тебе есть к кому возвращаться, значит --- вернёшься, --- проникновенно сказал Шмель.
\ml{$0$}
{--- Давай, Небо, туда и обратно, приключение на двадцать минут.}
{``Come on, Sky, there and back again, twenty minute adventure.}
Как выкуклятся --- будем вместе эту ораву выгуливать.
Лети и ни о чём не думай, кроме дела, я о них позабочусь.

Шмель почти силой забрал у меня куколки и, аккуратно завернув их в воск, ушёл в сторону медблока.
Я остался стоять в одиночестве, глядя в чёрное, полное звёзд небо.

\section{Самое быстрое животное}

--- Мы не успеем, --- взволнованно сказал Фонтанчик.
--- Эх...
Заяц, Небо, цепляйтесь за меня карабинами и затяните тросы потуже.
Упадёте --- костей не соберёшь.

--- Нет, Фонтанчик, --- Заяц отпрыгнула и выставила вперёд руки.
--- Нет!
Мне хватило одного раза!

--- Ты хочешь успеть или нет?

Заяц горестно вздохнула и повернулась ко мне.

--- Надеюсь, ты не поел перед этим.

Я и раньше, разумеется, знал, что кани на четвереньках способны развивать самую высокую скорость среди наземных животных планеты Тси-Ди.
Но пассажиром был впервые.
Фонтанчик летел, как шерстяная комета, молниеносно огибая и перепрыгивая препятствия.
Под сероватой кожей его спины бегали мощные мышцы, его живые и механические конечности слились в один сплошной клубок.
Под рёбрами, словно промышленная установка, работали полусинтетические лёгкие --- хы-ха, хы-ха, хы-ха...
Ветки и камни неслись прямо на нас.
Заяц взвизгивала, я в страхе прикрывал фасетки руками, но препятствия всегда в последний миг проходили в сантиметре от наших тел.
Когда мы прибыли на место, Заяц, как и полагается чувствительным млекопитающим, закономерно вывернула наружу содержимое желудка.

--- Да за что мне это, --- простонала она на четвереньках, вытирая губы.
Остро запахло кислотой.

--- Тебя понести? --- заботливо спросил Фонтанчик, протянув ей платочек.

--- Нет! --- вырвалось у Заяц вместе с очередным рвотным позывом.
--- Убери платок.
Вообще не подходи ко мне ближайшие сутки.
Сама дойду...

--- Это вместо благодарности, --- посетовал мне Фонтанчик.
--- Вот где она ещё на кани покатается?
Тебя понести, Небо?

--- Нет, благодарю, --- я на рефлексах отпрыгнул от друга, --- я, пожалуй, тоже пешочком...

\section{Носовая затычка}

--- Так, молекулярное лигирование нервов завершено, --- сказал Костёр, бросив взгляд на терминал. 
Затем, запустив руку в медицинский ларец, он протянул пациенту массивную носовую затычку.
--- Вставь её в нос плотно, она должна зачирикать.
Затем наклони голову вперёд, она должна сказать <<ку-ку>>.
Через три-четыре минуты все роботы будут у тебя в носовых капиллярах, начнётся небольшое кровотечение.
Надо их собрать.

--- Я скоро смогу работать?

--- Работать можешь, когда захочешь, --- сухо сказал Костёр.
--- А вот из лазарета выйдешь только послезавтра.
Надо дать нерву время, чтобы он оброс.

--- Благодарю тебя, Костёр.

--- Выздоравливай.
И держи хвост подальше от работающего принтера, а не то он будет короче ещё на один позвонок.

--- У нас тесновато, --- признался канин.
--- Один принтер на два отдела.
Мы просили собрать нам ещё один, но техники и так заняты по уши, они даже этому кожух так и не сделали...

--- Небо, скажи техникам, чтобы в первую очередь занялись безопасностью механизмов, --- грустно сказал Костёр, обернувшись ко мне.
--- Это уже третий хвост за десять дней.

\section{Звук дыхания}

Ещё с юности я обращал внимание на этот звук.
Если в комнате находился хотя бы один тси-млекопитающее и было достаточно тихо, то звук проступал во всём своём богатстве.
Этот звук издавала Заяц, издавал Баночка, издавал Фонтанчик --- кто-то повыше, кто-то пониже.
Этот звук издавал воздух, проходящий через десятки сантиметров воздухоносных путей.

Звук дыхания млекопитающих поразительно изменчив.
Если закрыть глаза руками, можно только по звуку дыхания определить очень многое --- рост, комплекцию, вид, пол, уровень мозговой активности, настроение, даже черты характера.
Разумеется, это присуще и нам тоже;
но дыхание апид не кажется мне настолько шумным.

Заяц постоянно спрашивала, как я угадываю настроение по её спине.
Я смеялся и ничего не отвечал.
У каждого должны быть свои маленькие тайны.

\section{Углы и время}

--- Царрокх используют старую систему измерения времени, основанную на шестидесятиричной системе счисления...

--- Может, шестидесятичетверичной?

--- Я не ошиблась.
Эта система применялась на Древней Земле в течение двадцати пяти тысяч лет --- около восьми тысяч оборотов.

--- Это огромный срок!

--- И это удивительно.
Та же система счисления используется при измерении времени и углов в хоргетах --- по крайней мере, у Безымянного я её тоже нашла.
Для времени схема --- шестьдесят на шестьдесят на двенадцать на два.
Для углов --- шестьдесят на шестьдесят на сто восемьдесят на два, она используется параллельно с обычной радианной системой.
В наших исторических хрониках сведения о целесообразности использования шестидесятиричной системы очень скудны.
Видимо, это дань какой-то очень древней традиции.

\section{Ларимары}

На её шее поблёскивало ожерелье из золота и бездонных как океан ларимаров.

--- Очень мило, --- восхитился Фонтанчик.
--- Как водичка у нас на побережье.
Это натуральные камни?

--- Рукав вчера нашла, когда разбирала привезённые породы, --- весело сказала Молодость.
--- Срез сделала --- красота несказанная.
Решила обточить.
А потом наплавила золота, заняла у техников медицинского стабитаниума --- и понеслось.
Сейчас половина геологов в её украшениях ходят.

\section{Нечего защищать}

--- Скажи, Баночка.
Почему первое, что мы сделали по прибытии на новую планету, --- это начали строить крепость?

--- Ты сомневаешься в том, что это правильно?

--- Нам нечего защищать.
На Тси-Ди были сады, заводы, огромные города.

--- Здесь есть мы, --- пожал плечами плант.
--- Этого недостаточно?

--- Да, может быть, ты прав, --- пробормотал я.

\section{Мировая Война}

Один из самых масштабных Театров, кстати, был посвящён войне.
Впоследствии он так и вошёл в историю --- Мировая Война.
Было создано огромное виртуальное пространство, имитирующее планету с суровым климатом --- пустыни, горы и ледники.
Это пространство населили астрономическим количеством NPC --- звери, птицы, насекомые, растения.
Почти все были смертельно опасны или причиняли игрокам большие неудобства.
Три противоборствующие стороны вели войну на уничтожение, используя как можно более негуманное оружие --- начиная от зазубренных мечей и заканчивая пулями со смещённым центром тяжести.
Система была анонимной --- лица и голос изменялись до неузнаваемости --- и с полным погружением --- все ощущения были абсолютно неотличимы от реальных.
Театр длился половину сезона, в нём приняли участие сто миллионов тси;
при этом четыре пятых игроков покинули виртуал ещё в первые десять дней, погибнув или просто не выдержав свалившихся на них испытаний.
Остальные, по их словам, дошли до конца <<из принципа>>.
В последней битве последние триста игроков, стоя посреди обледеневшего горного плато, заключили мир, закончив тем самым игру.

Результаты были плачевны.
Несколько десятков умерли из-за потрясения, шестидесяти пяти тысячам потребовалась помощь врачей и психологов.
Театр посчитали удавшимся --- по силе воздействия ему не было равных в истории.
Но тси вспоминают о нём очень неохотно.

\section{Возраст Безымянного}

--- Нашла что-нибудь интересное в памяти Безымянного?

--- Ничего особенного, относящегося к его создателям, --- покачала головой Кошка.
--- Очень хорошие комментарии к коду, настолько хорошие, что разберётся даже младший школьник... несколько пасхальных яиц, c десяток электронных книжек...

--- Что за книжки?

--- Документация к хоргету, устаревшая Мирквудская классификация, учебники и протоколы по терраформированию...
Наши экологи оценили.
Только об этом и зудят.

--- Протоколы лучше наших?

--- Если учесть, что за два телльна истории Тси-Ди мы терраформировали половину планеты, нами же и превращённой в выжженную пустыню... то да, лучше наших.
Хотя бы проверены на практике.

--- Понятно.
Больше ничего?

--- Какая-то художественная литература, которую использовали для отладки языкового модуля --- видимо, забыли стереть после... --- рассеянно сказала подруга, думая о чём-то своём.
--- Представляет разве что археологический интерес.
Я всё залила в отдельный каталог, можешь глянуть...

--- Посмотрю.
Кстати, каких времён Мирквуд?

--- Судя по отсутствию некоторых важных правок, не менее чем пятитысячелетней давности, --- встрепенулась Кошка.
--- Забавный способ датировки, согласна.
Но это говорит не о времени создании хоргета, а скорее о времени разрыва коммуникаций между цивилизацией создателей Безымянного и Орденом Преисподней.

--- То есть минус триста лет.

--- Я бы даже доверительные интервалы подняла до пятисот, на всякий случай.

--- И что скажешь?
Чувствуется, что Безымянному четыре с половиной тысячи лет?

--- Он --- ребёнок в песочнице.

--- И песочница --- Планета Трёх Материков.

--- Тебя это не удивляет?

--- Я уже устала удивляться, --- растерянно призналась Кошка.
--- Я --- колонист, ступивший на поверхность, возможно, самой удивительной планеты во Вселенной, я общаюсь с хоргетом, и он даёт мне читать некоторые части своих исходников.
Слишком много удивительного.
Чтобы работать нормально, многое приходится принимать как должное.

\section{[2] Сигнал}

Полгода, которые запросил Костёр, давно истекли, а от врача не было ни слуху ни духу.
Я начал волноваться.

--- Он взял с собой связь? --- спросил я у техников побережья.

--- Мы сами волнуемся, --- сказал мне Синяя-Круглая-Плесень.
--- Квантовый передатчик сутки назад подал сигнал тревоги.
Мы выпустили поисковый дрон, но он не вернулся.
Сейчас за ним отправились дельфины.

\section{[2] Достойно фильма}

--- Я тебе сейчас расскажу такую историю, это достойно фильма, --- начал Костёр.

Костёр сбился с курса и нашёл новую группу островов --- целый архипелаг, формой напоминающий спираль.
На островах даже оказалось местное племя.
Костёр не был силён в переговорах, как Кошка, и вскоре ему пришлось бежать.

--- Обычные люди, такие же, как те царрокх --- но как же ловко они бегают по зарослям! --- вспоминал врач.

--- А чем ты им насолил?

--- Я слова им сказать не успел!
Они меня приняли за дичь!
Еле ушёл.
Лодку пришлось гнать.

Однако вскоре его ждала новая напасть.
Лодка попала в полосу цунами-шторма.

--- Идёт последняя волна, и я понимаю --- всё, конец, лодка перевернётся и я захлебнусь.
У меня даже не было времени написать сообщение вам.
Поэтому я дёрнул за кольцо жилета, заставил передатчик послать тревогу и прикусил язык.
Когда очнулся, шторм уже утих.
К счастью, океан в этих широтах тёплый, как молоко, и я не погиб от переохлаждения.
Днём спал, ночью шевелил руками-ногами не переставая, жевал всякую чушь --- водоросли, улиток, мелкую рыбу, креветок --- только бы не застыть.
Дрон прилетел в тот момент, когда я уже начал волноваться.
Я заарканил его тросом и повёл на запад, покуда батарей хватало.
К вам бедняжка, конечно, не вернулся, я пустил его на детали.
И снова повезло --- меня вынесло на остров с деревьями.
Собственно, последние полгода я строил новое судно из подручных материалов.

--- Дельфины тебя не нашли, --- сказал я.

--- Конечно! --- пожал плечами Костёр.
--- Я же говорю, я на острове был!

--- Жалко, что так вышло.

--- А я не жалею.
Да, мне пришлось пережить многое, сломать несколько костей, получить стрелу в плечо, надорвать спину во время постройки корабля.
Но зато я отвлёкся от грустных мыслей.
Сейчас Катаклизм кажется мне таким далёким, словно я читал о нём в исторической хронике, а не испытал на собственной шкуре.

\section{Рождение выбора}

Будь у меня возможность выбрать ещё раз, я сделал бы то же самое.

Да и бесполезно возвращаться в тот же момент.
Выбор --- настоящий выбор --- был сделан задолго до него.
Может, за год или за восемь лет.
А может, я всю жизнь шёл к тому, чтобы поступить так, как я поступил.

\section{Придуманные враги}

Я с трудом понимаю войну.
Нападать или защищать --- не имеет значения, результат всегда один.
В прежние времена манипулировали обещаниями благ --- пищи, красивой одежды, почёта, секса.
Но ответ на всё один, и этот ответ тси знают с детства --- никто не поселится в выжженных землях, все любят цветущие сады.
Вырасти в своих землях, доме, теле и мозге собственный сад, знай его и ухаживай за ним.
Тогда тебе не будет нужды брать что-то силой.
Да и опыт, как по мне, будет гораздо более полезным в дальней перспективе.
Куда приложить опыт войны, если закончатся враги?
Разве что придумать новых.

\section{[1] Мхи (ЖС)}

Я смотрел на мох.
Согласно данным первых людей, мхи были одними из первых наземных растений.
Как и миллиард лет назад, они --- такие крохотные, такие сильные, бесконечно терпеливые --- кучками цепляются за щёлочки и ложбинки в камне, стараясь устоять перед ветром и сберечь драгоценную влагу.
Мхи медленно расползаются, умирают и гниют, создавая миллиметр драгоценной почвы --- почвы, на которую смогут сесть травянистые.

Я обернулся на поселение тси.
Эти голые камни, эти стоящие кучками домики и палатки... да, именно.
Как древние мхи, мы сообща пытаемся выжить и сохранить всё самое ценное --- в надежде, что идущим за нами будет легче.

\section{[1] Цитра Ветра}

Безымянный долго спрашивал, кто может сделать для него музыкальный инструмент.
Наконец одна из тси, инженер-технолог по имени Ветер-Дующий-Ниоткуда, решилась.
Как-то утром она вызвала бога и показала ему сделанный из тонких древесных пластинок струнный инструмент.

--- Не могла заснуть ночью, --- пояснила она.
--- Идеям плевать на твой график.
Древесину сложно довести до состояния, когда она начинает хорошо резонировать, но я справилась.

--- Как тебя зовут? --- спросил Безымянный, осмотрев инструмент.

--- Ветер, --- осклабилась женщина.
--- Не стоит благодарности, играй.

--- Я запомню это имя, Ветер, --- сказал бог.
--- Я запомню это имя.

Отныне инструмент Безымянного бежал в библиотеке Стального Дракона, в специально отведённом для этого уголке.

\section{[1] В честь планеты}

--- Закрытая-Колба-с-Жизнью? --- засмеялся Мак.
--- Так ты назвался...

---  ... в честь планеты Тси, да, --- буркнул Баночка.
Эта аллюзия, восходящая к кому-то из поэтов романтических времён, успела ему приесться.
--- Надеюсь, ты цитировать его не будешь?

Мак явно намеревался и даже набрал для этого побольше воздуху в грудь, но вовремя сообразил, что делать этого не стоит.
В итоге биолог просто чуть громче, чем следовало, сказал <<Очень приятно познакомиться>> и ещё раз пожал Баночке руку.

\section{[1] Кон-Тики}

Я вдруг вспомнил далёкие школьные годы и экскурсию к котловине Кон-Тики --- месту, где был найден другой корабль.
Тот самый корабль, на котором предки млекопитающих-тси прибыли в нашу звёздную систему.

Огромный грот и кусок обшивки с надписью, вплавившийся в тысячелетний гранит.
Мои руки и ногочелюсти тряслись.
Мне казалось, что именно на меня величественная пещера подействовала гораздо сильнее, чем на прочих.
Знал ли я, насколько сильно будут похожи судьбы капитана Гуштефа и моя собственная судьба?
Возможно ли это?..

\begin{quote}
<<Данное направление было самым протяжённым в пространстве за всю историю космических путешествий древних людей.
Оно включало пять планет, пригодность которых была недостоверной\ldots>>
\end{quote}

\begin{quote}
<<... Априла 18 года 25712 Кон-Тики покинул безжизненную систему Иштар и вылетел в безумный, отчаянный полёт, имея топлива на один последний цикл разгон-торможение.
За 300 часов до этого три древних планеты --- Земля, Марс и Диана --- замолчали навсегда.
Корабль отчаянно пытался связаться с ними, но безуспешно, и экипаж, поняв, что проблема не в устройстве связи, принял мужественное решение лететь до конца пути --- двойной планеты на окраине Галактики, планеты под кодовым названием <<Тсиди>>, что означало <<вдохновение>> на одном из языков.
Они так и не узнали, что случилось с их домом, какова суть постигшего эти планеты бедствия.
Возродилась ли цивилизация?
Возможно.
Но возродись она и через пятьдесят лет, и через пятьсот --- о Кон-Тики, который жаждал весточки из дома, никто бы не вспомнил>>.
\end{quote}

--- Я хочу, чтобы вы, глядя на эту доску, крепко уяснили одну вещь, --- сказал в заключение учитель.
--- Дома вы живёте в уюте, проводите время в работе и развлечениях.
Тси-Ди --- это наш дом.
Ценой невероятных усилий тси сделали поверхность одной планеты и большую часть поверхности второй большим уютным домом.
Однако жизнь коротка.
Что для вас и для меня жизнь Гуштефа Морре-Чилда?
Всего лишь кусок металла.
Мало кто осознаёт, что мы оказались здесь благодаря горстке живых существ, которые говорили на другом языке, которые прошли труднопреодолимое даже для демонов расстояние.
В нас до сих пор живут их гены, а прочие, модифицированные гены помнят умелые руки их потомков.
Каждый аспект технологии, который мы можем найти на двух планетах нашей системы, несёт на себе отпечаток мыслей тех существ.
Подумайте об этом и скажите мне, прожившему триста лет, что же такое наша жизнь?
Я уже отчаялся найти ответ на этот кажущийся простым вопрос.

\section{Тревожная женщина}

--- Не следят за своим здоровьем, --- бурчал Костёр.
--- Я же тоже не железный...
Вы всегда можете прийти ко мне, и я вас полечу.
А кто меня лечить будет?

--- Возьми отпуск, --- сказал я.
--- Хотя бы на несколько дней.
Слетай на тот берег.
Там красиво.

--- И остаться наедине со своими мыслями? --- хмыкнул Костёр.
--- Нет уж.

--- Кого ты потерял?

Костёр потряс головой и надолго замолчал.
Я уже было думал, что он не хочет говорить, но...

--- Женщину и детей.
Не нужно, --- Костёр мягко высвободил руку, когда я попытался её погладить.
--- Я уже успел забыть за всеми этими заботами, как они выглядят.
Но одна мысль не даёт мне покоя, и боюсь, однажды она меня убьёт.

Костёр подумал.

--- Та женщина страдала высокой тревожностью.
Дупликация гена одного из нейромедиаторов, достаточно редкая мутация.
Обнаружили её уже в зрелом возрасте.
Женщина приспособилась жить с ней --- работала только по ночам и в полной тишине.
Зато в работе --- а была она диагностом систем --- ей не было равных.
Замечала такие нюансы, мимо которых проходили все прочие.
Познакомились мы в больнице.
Меня потрясла её нежность и чуткость --- такого человека редко можно встретить.
Она же говорила, что рядом со мной ей очень спокойно.

Врач вздохнул.

--- Меня не печалит её смерть --- здесь, среди постоянных тревог, она бы просто не выжила.
Но едва представлю всю глубину того ужаса, который испытало это нежное сердечко перед смертью --- и всё валится из рук.
Поэтому ты, Небо, как хочешь, а меня в отпуск не отправляй.
Буду работать, пока не сдохну.

Я поднялся.

--- Я сообщу всем молодым тси, что тебе нужна ласка.
Я думаю, кое-кто придёт.

--- Пошёл вон.
И если ко мне ещё кто-то придёт с нежностями, я их выкину.

--- Первый раз --- разумеется.

--- И второй тоже.

--- Отлично.
Значит, женщины и мужчины будут делать не менее трёх попыток, --- заключил я.
Костёр ухмыльнулся.

--- Мужчин не нужно, не привлекают.

--- Спокойного дня, Костёр.

--- Спокойного и тебе, Небо.

\section{[2] Сон}

--- Комарик, пожалуйста, живи!
Живи! --- плакал я.

Комар погладил меня слабой рукой по плечу.

--- Я всегда буду с тобой.
Я буду жить у тебя в голове.
А ещё у нас остались дети, Небо.

--- Как бы я хотел, чтобы у нас были дети.

--- Они у нас есть.

Я гладил его по голове и плакал.

Прибывала вода, расцвеченная красной, чёрной, зелёной кровью...
Мимо величественно проплыла рыба, задев меня длинными мягкими усами.
Её глаза опалесцировали, плавник гордым парусом торчал из воды, рот с нежными красными губками заглатывал кровавую жижу.
Откуда здесь, в Зале Достижений, эта рыба?
Я поднял глаза... и увидел яркое солнце меж двух отвесных стен.
Знакомых, изъеденных ветрами стен.

Котловина Кон-Тики.

Вода прибывала всё сильнее.
Комар начал захлёбываться;
он инстинктивно приподнимал брюшко, но вода всё равно затекала в дыхальца, выбивая из агонизирующего тела столбики весёлых пузырьков.
Я закричал, поняв, что у меня не хватает сил, чтобы даже приподнять любимого над водным потоком.

Пузырьки.
Вода клокотала, чёрная, как ночь, с металлическим оттенком, словно холодная ртуть.
Я снова закричал --- отчаянным, ужасным криком.
Комар задыхался на моих руках.

--- Небо, молчи и слушай.
Девяносто пять.
Девяносто пять нанометров.

--- Что?
Что ты говоришь? --- плакал я.

--- Молчи и слушай, Небо!

--- Да что он несёт? --- закричал я.
--- Баночка, помоги мне.

Я поднял глаза на планта.
Он стоял по пояс в воде и смотрел на меня грустными глазами из-под татуированных век.

--- Ничего не сделать, командир.

--- Просто! Помоги! Его! Поднять! Над! Водой! --- заорал я.

--- Это не вода, Небо, очнись.
Это смерть.
Ты сам уже по пояс в ней.

--- Молчи! И слушай! --- скрежетал Комар.
Над водой осталось только лицо, но он пытался что-то сказать.
--- Это очень важно...
Девяносто пять, Небо, девяносто пять...

\asterism

Ночник светил слабым, успокаивающим светом.
Я несколько раз тянулся рукой к импланту, чтобы сделать инъекцию успокоительного.
И отдёргивал руку.

Первой мыслью было бежать к Костру;
разумеется, это был \emph{тот самый} сон, о которых врач меня предупреждал.
Но ни один не вгонял меня в такой ужас.

<<Молчи и слушай>>.
Да, верно, Костёр в кругосветном.
Я уже пожалел, что отпустил его.

Я лёг и попытался не двигаться.
Это была любимая фраза Комара, совет на любой случай жизни.
Молчи и слушай.

За открытым иллюминатором царила тихая ночная жизнь.
Пели птицы, сновали млекопитающие, шуршали насекомые.
Над всем этим царил ровный гул реактора.

Нет, что-то точно не так.

В коридоре меня встретил Баночка и замер, словно увидев призрака.
Потом облегчённо выдохнул.

--- Ты меня напугал.

Плант вдруг вытащил сверток и принялся набивать травами курительное приспособление.

--- Ты куришь? --- удивился я.

--- Неа, --- виновато улыбнулся плант.
--- Просто что-то неспокойно, и я решил попробовать.
Пирожок говорит --- успокаивает.

--- Ты был у неё?

--- Да.
Ей тоже что-то не спится.

--- Что, неужели и ей котловина Кон-Тики не даёт заснуть?

Баночка промолчал и сделал глубокую затяжку.
Затем закашлялся.

--- Что мы упускаем, Баночка? --- задумчиво спросил я.
--- Мы ведь упускаем что-то важное.

--- <<Молчи и слушай>>, --- буркнул плант.
--- <<Девяносто пять нанометров>>.
Мы действительно что-то упускаем.
Но что?

Я вдруг понял, что больше ничто в этом мире не способно меня удивить.

--- Комар не успел сказать.

--- Да при чём тут Комар, --- вдруг раздражённо хмыкнул плант.
--- Забудь про Комара, забудь про котловину Кон-Тики, и про смерть свою забудь, они в далёком прошлом.
Научись уже отделять актуальные проблемы от проекций чьей-то памяти.

--- Моей памяти!

--- Ты уверен, что это твоя, а не память давно умершего, чужого тебе существа, оказавшегося в похожей ситуации?

Я промолчал.
Баночка всегда умел задавать хорошие вопросы.

--- <<Молчи и слушай>>.
Эта мысль летает в воздухе прямо сейчас.
<<Девяносто пять нанометров>>.
Что настолько важно?
Что измеряется в нанометрах?
И это число, как будто я его где-то...

--- Кое-что не было проекцией чужой памяти, --- перебил я его.
--- Молчи и слушай.

Баночка виновато кивнул.
Пели птицы, шуршали млекопитающие, шелестела трава.
Гул реактора, ровная, идеально ровная гармоника.
Пять секунд.
Сто пять секунд.
Пятьсот тридцать одна секунда...

И вдруг гармоника сделала крохотный скачок.
Мы с Баночкой уставились друг на друга.

--- Что это было?

--- Молчи и слушай! --- прошипел плант.
В его глазах застыл животный ужас.

Пять секунд.
Сто пять секунд.
Пятьсот тридцать одна секунда в томительном напряжении.
Трубка в руках Баночки давно испустила последний дымок.
По лицу планта текли капли пота;
в глазах плясала надежда.
Может, показалось?
Мы спросонья, переволновались, конечно, нам могло показаться что угодно...

И снова крохотный, едва уловимый скачок гула реактора.

Баночка кинулся к лифту.

--- Поднимай всех, командир, --- бросил он через плечо.
--- Мы по колено в воде.

\section{[2] Молитвы}

--- Мне кажется, что я скоро умру, --- признался я.

Баночка вдруг схватил меня за руку.

--- Небо, пойми одну вещь.
Наша судьба не определена.

--- Но сны...

--- В бездну сны! --- рявкнул плант.
--- Ты вечно строил из себя жертву.
Ты жертвовал всем --- своими интересами, своей жизнью, всем, что имел.
Ты думал, что тебе уготована такая судьба, и шёл навстречу этой судьбе.

--- Костёр...

--- Костёр сумасшедший!
Ты не задумывался, почему я с ним не общаюсь?
Мне совершенно неважно, каким образом я вижу сны о далёком прошлом.
Да, я давно знаю, что это правда, с самого детства.
Я чувствовал, как вонзаю штык в чужое сердце, до мельчайших подробностей.
Я собирал во сне взрывчатку, ещё не имея понятия о началах химии, пользовался противогазом, чистил винтовку.
Я до сих пор помню вот это!

Баночка сделал молниеносное движение руками, и я похолодел.
В моей голове ясно щёлкнул передёргиваемый затвор.

--- Я умирал, занимался сексом, бредил в лихорадке, ещё не зная этих вещей в реальности.
Но я никогда, слышишь, никогда не уходил в мистицизм.
А Костра эти сны свели с ума!
Его свело с ума одиночество, потому что ты одинок ровно в той мере, в которой погружён в иллюзию!
Ты хочешь того же?

--- Ты не прав.
Одиночество --- это погружённость в иллюзию, отличную от иллюзии прочих.

--- Давай не будем устраивать низкопробный философский диспут на серьёзную научную тему.
Мы не дикари.
И помнится, мы вместе сдавали диктиологию.

Я промолчал.

--- Небо, я тебя прошу --- останься с нами.
Останься в реальном мире.
Если тебе одиноко, чаще бывай с народом, пройди курс гормонов, только не уходи в иллюзию.
Нет и не было никакой судьбы, --- Баночка отпустил мою руку и крепко обнял.
--- Я это знал всегда.
И, заметь, почти всегда доживал до старости.
Потому что мной двигало желание жить, а не обречённое принятие чьих-то иллюзий.

\section{[2] Каскад тревоги}

--- А как вы меня нашли? --- удивился я.

--- Ты не поверишь, --- улыбнулась Кошка.
--- Листик и Баночка засекли каскад тревоги.
Я бы в жизни не додумалась до такого.
Объяснять долго, суть: удар шлюпки о воду испугал рыб.
К испуганной рыбе слетелись чайки, за чайками последовали крупные хищники.
Волны тревоги распространялись вокруг эпицентра с помощью сигналов различных существ.

--- А кто засёк эти волны?

--- Мой приёмник, --- Баночка снял с плеча довольно каркающую Нейросеть.
Судя по всему, ворону только что похвалили и накормили чем-то вкусным.

--- Листик случайно услышала тревожное карканье Нейросети и сразу сообразила, что это может быть связано с перебоем связи.
Баночка после долгих усилий наконец понял, что хотела втолковать ворона, и мы послали в море дельфинов.

Я погладил птицу пальцем.
Она игриво ухватила меня сильным клювом.

Никогда не думал, что буду обязан жизнью вороне.

\section{[2] Интуитивный мистицизм}

Кошка размахивала руками и поставленным голосом читала какие-то тексты, время от времени сверяясь с компьютером.

Я заворожённо наблюдал за ней.
Простой комбинезон вдруг сменился в моём воображении причудливой одеждой с перьями, руки и лицо покрылись охряным узором, глаза остекленели от наркотических веществ.
Я понимал --- Кошка не просто изучает ритуал.
Она пытается найти свой собственный путь, путь учёного-тси, во тьме древнего, интуитивного мистицизма.

\section{Животный страх}

Меня очень удивили местные животные.
На Тси-Ди обращение с дикими животными занимало несколько лет обучения.
Детям объясняли про рефлексы низших животных и про сигнальную систему высших.
Впрочем, за десятки тысяч лет мирного сосуществования дикие животные настолько привыкли к тси, что воспринимали их как камни или деревья.
Они иногда кушали из рук, позволяли себя гладить или приходили к жилищам полечиться.
А со знанием сигнальной системы можно было довольно легко убедить не слишком голодного волка, что ты плохая добыча и потенциальный друг.

Цветущий-Мак-Под-Кустами даже как-то рассказал, что в детстве одну зиму жил в стае койотов.
Самым сложным, по его словам, оказалось приучить лохматых к человеческому запаху, а ещё объяснить, что у костра можно греться.
Когда волки смекнули, что к чему, то сами стали намекать, что пора бы уже <<сделать горячую вонючку>> --- зима выдалась суровой.
Мак даже вспомнил, что в благодарность за что-то молодая самка койота принесла ему свежезадушенного кролика.

Здесь же животные не показываются взгляду.
Почуяв человека, млекопитающие обычно удирают со всех ног.
От них пахнет смертельным страхом.
К прочим тси также относятся настороженно.

Я рассказал свои ощущения Баночке.
Он в ответ поделился опасениями, что нам придётся жить охотой и животноводством, если с кольцевой теплицей что-то случится.

\section{Дельфины}

--- Я не понимаю дельфинов, --- заявила Листик.
--- Мы все переживаем, думаем, как лучше, а дельфинов это как будто не касается.
Главное, чтобы был океан, и дельфины счастливы.

--- А о чём им беспокоиться? --- заметил Мак.
--- Пищи у них достаточно, врагов нет.
Гора-Окутанная-Дымом сказал, что он с отрядом проплыл почти до самых полюсов.
Да, там прохладно, им пришлось надеть термочехлы на плавники, но в целом дельфины могут жить везде.

--- А акулы в проливе Скар? --- спросил я.

--- Акулы могут утащить разве что дельфинят.
Или малыша вроде тебя, --- усмехнулся Мак.
--- Гора сказал, что акулам и прочим морским зверюшкам очень не понравились сонарные сигналы, которыми он и его отряд <<прокрикивали>> дно.
Я посоветовал дельфинам провести на этот счёт исследование --- дыма без огня не бывает, возможно, что какой-то опасный природный феномен тоже <<говорит>> на этих частотах.
Сейчас команда Капельки этим занялась --- узнают, каких именно частот боятся зверюшки и в какой местности.
Так мы хотя бы будем знать, что искать.

--- А что ещё нашли дельфины? --- поинтересовалась Заяц.

--- Вообще они много кого привезли.
Рыбок, моллюсков, водоросли, даже щупальце кого-то, отдалённо напоминающего гигантского осьминога.
Сказали, что труп дрейфовал недалеко от кораллового рифа, очень повезло, что его ещё не успели обкусать рыбы.
Гора составил очень подробный видеоотчёт по океану южнее девятой параллели.
Больше похоже, правда, на юмористический фильм, над комментариями можно ухохотаться.
<<Смотри, какая странная красная рыбка!>>
--- <<Оставь её, хвостомордый, у неё депрессия>>.
--- <<С чего ты взял?>>
--- <<Если бы я носил шкурку наизнанку, у меня тоже была бы депрессия>>, --- Мак захохотал.

--- Я же говорю, что они относятся к делу недостаточно серьёзно, --- укоризненно сказала Листик.

\section{[2] Рука}

Из наркотической полудрёмы меня вырвал звук взрыва.
Как я узнал впоследствии, техники попытались применить не по назначению полевой генератор.
В палатку проковыляла Заяц, крича на ходу медицинский код травмы и волоча на себе тяжёлое, залитое кровью тело Звенеть-Хрустальными-Клыками.
Правая рука канина превратилась в месиво.

Ещё один несчастный случай.

--- Один-три-а-ноль-эф!
Один-три-а-ноль-эф!

Костёр выбежал как ошпаренный, толкая перед собой летающий ларец с числом <<13>>.
Заяц без сил рухнула на колени.

--- Отойди, --- Костёр грубо отпихнул Заяц и, выхватив пистолет, прижал его к груди Зубика.
На случай выхода импланта из строя у всех тси-млекопитающих в грудине стоял внутрикостный катетер.

Заяц, придя в себя, бросилась помогать.
Вскоре на культю была наложена искусственная соединительная ткань, потерю крови восполнили кровозаменяющей жидкостью, и канина положили в реанимационную капсулу рядом со мной.

--- Как остальные? --- это был первый вопрос, который задал Зубик по пробуждении.

--- С ними всё в порядке, --- сухо ответил Костёр, проверяя в последний раз повязку.
Отсутствие привычных инструментов, материалов и лекарств сделало его раздражительным --- и это при пресловутой стрессоустойчивости врачей.

Зубик посмотрел на культю и поморщился.

--- Я теперь навсегда без руки?

--- Мы потеряли кольцевую теплицу, --- чуть не плача, ответила Заяц.
--- Я не знаю, возможно ли вырастить руку с теми материалами, которые...

--- Заяц, перестань, --- оборвал её Костёр.
--- Хватит брать на себя вину за целый мир.
Теплица сгорела, в реакторе была трещина, роботы были перепрограммированы.
Абсолютно все обстоятельства были против нас.
Фонтанчик погиб, как герой, пострадавших могло быть больше.

Заяц расплакалась.
Костёр бросил на неё виноватый взгляд, поправил мне одеяло и уже чуть мягче добавил:

--- Я восстановлю тебе руку, Зубик.
Но не за один год.
Подождёшь?

--- Куда мне деваться, --- попытался пошутить канин.

--- Я соберу тебе механическую, --- сказала Заяц, вытирая слёзы.
--- Уж это-то мы сможем.

--- Вы лучшие друзья, ребята.

--- Для начала поправься, --- остановил его Костёр и, мимоходом взглянув на меня, вышел на воздух.
Вскоре откуда-то потянуло странным удушливым дымом --- кое-кто из тси пристрастился к местным травам с седативными алкалоидами, и похоже, что Костёр тоже.
Я нечаянно громко щёлкнул зубами, и дым тут же рассеялся, сменившись на слабое гудение вытяжки.

Заяц сидела у постели Зубика.
Я знал, что она сейчас думала о погибшем любовнике и о едва живом мне.
Ей страшно хотелось на воздух, подальше от этой наполненной смертью палатки, но её останавливал долг дружбы.
Зубик, похоже, тоже понял это.

--- Иди, Зайчик.
Я за ним послежу.

--- Иди, --- добавил я.
--- Я пока в хорошем настроении.

Заяц с натянутой горькой улыбкой по очереди кивнула нам и тут же вышла.

\chapter{Безумная война}

\section{Казнь}

--- Кутрап, --- тихо сказал Кхарас и закрыл лицо руками.

--- Ну так на крышу его, --- пожала плечами Ликхэ.

--- Не хочет, --- ответил боевой вождь.

На моей памяти такое было впервые.
Почти все кутрапы выбирали жертвоприношение.
Почти все.

--- Давайте я, что ли, --- Кхохо угрюмо отпила из чаши и подхватила серебряную саблю.
--- Наблюдатели из числа Советов уже там?

--- Ага, --- кивнул Кхарас.
--- Держи свитки: решение Советов, подтверждение личности...

--- Давай всё, --- оборвала его Кхохо, просмотрев пергаменты.
--- По дороге разберусь...

Все тут же побросали еду и пошли за воительницей, чтобы ничего не пропустить.

Кутрапом оказалась совсем юная девушка дождей восемнадцати.
Несмотря на это, за ней числилось Разрушение и более двадцати актов Насилия.
Получив метку, она скиталась по городам, добывая себе пропитание воровством.

Совет Цеха в полном составе ждал у храмового гонга.
Кхотлам не пришла, но прислала вместо себя Эрхэ.
Кхохо молча поманила их на храмовые земли.

--- Кхохо, душа моя, на площади надлежит казнить, --- тихо буркнул Эрликх.

--- Это чтобы развлечь торговцев или чтобы испортить им пищеварение? --- парировала Кхохо.
--- Хватит болтать.
Развяжите ей руки.

--- Зачем? --- удивилась старшина Цеха.

--- Ты здесь Храм или я? --- веско ответила Кхохо.
--- Выполняй.

Девушке развязали руки.
Она с наслаждением потёрла синие кисти.
Спустя пару секхар перед ней шлёпнулась фаланга из арсенала.

--- Вынимай, --- процедила Кхохо.
--- Я не знаю, как ты жила.
Вряд ли достойно.
Но я даю тебе возможность умереть с оружием в руках.

--- Опять самодеятельность, --- забурчали члены Совета.

--- Кхохо, лесные духи, перестань!

--- Если она меня убьёт --- может быть свободна, --- ухмыльнулась Кхохо.

Все дружно застонали и закатили глаза.

--- Это не по закону! --- заявила старшина.

--- Кто-то против? --- осведомилась Кхохо.
--- Кто-то хочет, чтобы я поставила её на колени посреди торговой площади и зарубила как курицу?

Совет забормотал.
Все озирались, пожимая плечами.
Перспектива увидеть бой вместо бойни многих явно воодушевила.
В глазах девушки блеснула надежда.
Она смотрела на Кхохо почти с обожанием.

--- Бери оружие, --- Кхохо снова кивнула на лежащий в траве клинок.
Девушка подхватила его и встала в боевую позицию.

Бой длился ровно десять секхар.
Парирование, обход, хлёсткий удар --- и обезглавленная девушка упала в траву.

--- Похороните её пока здесь, --- Кхохо ткнула окровавленной саблей в заросшую грядку под толстым саговым деревом.
--- Знаки проставьте, потом вызовем людей на изъятие кости.
И да, вы все прекрасно знаете, что казнь прошла по закону, верно?

Совет одобрительно забормотал.
Кхохо ухмыльнулась и пошла обратно в храм.

--- А если бы она тебя убила? --- тихо спросил я.

--- Духам виднее, --- пожала плечами Кхохо и сунула мне саблю.
--- Будь другом, вытри её от крови, пожалуйста.
Я умаслила свою совесть как могла, но мне всё равно немного некомфортно.
Почему мы, воины, вообще должны этим заниматься?..

\section{Воинская присказка}

--- На Западе говорят <<воинская присказка>>.
Мы не любим слово <<клятва>>.

--- Почему?

--- Потому что от клятвы ничего хорошего.
Она заставляет дающего лгать и испытывать стыд, а принимающего --- принуждать делать то, что человек не всегда хочет и может.

\section{Граффити}

Едва зайдя в проулок, я обнаружил на стене свежее граффити:

<<Храм лжёт>>

Я вздрогнул.
По спине поползли мурашки.
Рядом недовольный крестьянин уже замешивал известь с охрой, чтобы закрасить надпись.

--- Мне это надоело, --- бурчал он.
--- Жопы себе покрасьте.
Лжёт Храм или нет, при чём тут моё жилище?

Ещё за одним углом была надпись:

<<Храм --- убийцы>>

Мне резко стало неуютно.
Захотелось побыстрее пойти в храм, на торговую площадь, или на крайний случай выбраться на освещённую широкую улицу.

Вдруг я увидел знакомую фигуру.
Она коряво выводила надпись на свежепокрашенной стене.
Увидев меня, женщина выронила кисть и попыталась запрыгнуть на крышу.
Непрочный карниз треснул, и она плюхнулась на землю, пролив краску на себя.
Послышалось неразборчивое витиеватое ругательство.

--- Кхохо!
Это ты пишешь всю эту чушь на стенах?

--- Это не я, --- быстро ответила Кхохо.
--- В смысле, вот это, то что здесь, конечно, я --- глупо отрицать.
Всё остальное --- не я.

Я подошел поближе и вгляделся.

<<Ликхмас --- Первый жрец>>.

--- Решила тебя поддержать, --- извиняющимся тоном сказала Кхохо.
--- А то тут шепотки пошли, что и ты в этом замешан.
Нас-то уже давно поносят на чем свёт стоит...

--- Кхохо, сейчас же прекрати пачкать мной стены, --- строго сказал я.

--- Как скажешь, Первый жрец, --- буркнула воительница и, подтянувшись, исчезла в темноте на крыше.

Должность Первого жреца была чисто административной и во многом декоративной.
Но для Кхохо, видимо, значение было иным.

Кхохо не была одинока.
По пути я насчитал ещё восемь надписей с моим именем.
Одна явно принадлежала Ситрису, так же полно ошибок, ещё пять незнакомых.
Одна надпись была начертана традиционным начертанием, как в Легенде об обретении.
Ещё одна с орфографической ошибкой и эмоглифом, заставившим меня покраснеть, причём надпись точно не принадлежала ни Чханэ, ни Ликхэ.
И ещё одна... Кхотлам?

У меня пошла кругом голова.
Даже во сне с похмелья я не мог увидеть, как кормилица пачкает стены по ночам.
Но почерк говорил сам за себя, хоть она и нарочно исказила его.

Надпись была лаконичной --- <<Ликхмас>>.
Больше ничего.

\section{Бусы}

Диалекты языка сели отличаются от деревни к деревне, от хутора к хутору.
Порой жители двух глухих хуторов на расстоянии двадцати кхене понимали друг друга через слово.
У горожан словарный запас намного больше, потому что приходилось общаться с жителями всех окресных хуторов.
Самым богатым словарным запасом обладают, разумеется, купцы.
А рекордсменом по числу вариаций в языке сели является слово <<бусы>> --- их называют <<бобы>>, <<хоровод>>, <<позвонки>>, <<цепка>>, <<трещи-нить>>, <<жемчужница>>, <<считалка>>, <<медявка>>, <<три-счёт>> --- всего около пятидесяти.
В хуторах Тхитрона распространено слово <<ожерелик>>, родня простенькие бусы с более дорогим и сложным в изготовлении ожерельем.

\section{Серебряный волос}

Ликхэ села рядом и начала вычёсывать у меня волосы.

--- Чего это ты? --- удивилась Чханэ.

--- Серебряный волос ищет, --- хихикнула Кхохо.

Ликхэ порозовела и тут же переключилась с моих волос на локоны Хитрам.

--- Серебряный волос? --- переспросила подруга.

--- Ага.
Цветом как серебро, выдернуть нельзя, как ни тяни.
Намотаешь на палец, прошепчешь <<Люблю>> --- и твой человек до конца жизни, --- нараспев рассказала Кхохо.

--- Чушь городишь всякую, --- пробормотала Ликхэ.

Чханэ немедленно села рядом со мной и начала перебирать мне волосы.

--- Я и так твой, --- намекнул я.

--- Я знаю, --- сказала подруга, продолжая поиски.
--- Просто интересно, правда это или нет про волос.
О, кажется, нашла.

--- Ай!

--- Нет, точно не он.
Просто седой.
Ты седеешь, Лис, ты в курсе?

\section{Болезнь Кхохо}

Однажды Кхохо заболела.
Болела она каждый год, ровно три дня, в одно и то же время --- в самую жару, посреди летней страды.
Ей выделяли келью на верхнем этаже, с ней обычно оставался сидеть кто-то из воинов, но в тот раз время выдалось суматошное --- срочный вызов, нападение, какие-то нелады с крестьянами --- и про Кхохо просто забыли.

Я почуял неладное, в очередной раз проходя мимо её кельи.
Дверь была приоткрыта, словно там кто-то живёт, но на пороге лежала пыль, как если бы туда не заходили уже много дней.

Кхохо лежала, отвернувшись к стене и закрыв глаза.
Воздух пропитался запахом немытого тела.
В келье было душно, но она вжималась в одеяло, словно мёрзла.

Я потряс её за плечо.

--- Ты в порядке?

--- Да, --- ответил мне сиплый шёпот спустя десяток секхар.

--- Повернись и посмотри на меня, --- скомандовал я.
Стандартная фраза, которую произносит врач, приходя в дом заболевшего.

Кхохо не двинулась с места.

Я откинул одеяло, перевернул её на спину и ужаснулся синюшным пятнам на руках и груди.
Такие пятна появлялись на лежалых трупах.

--- Давно ты лежишь без движения?

--- Не знаю, --- тем же безжизненным шёпотом ответила Кхохо.

Я провёл пальцами по её губам и дёснам.
Они были сухими.
Графин стоял на прикроватном столике, заполненный водой до самого верха.

---  ...Кхохо, ну пей, ну пожалуйста! --- вода выливалась на подушку, не достигнув глотки.
Я набрал воду в рот и, приникнув губами к растрескавшимся губам Кхохо, с силой вытолкнул воду.
Она проглотила и слабо кашлянула.

--- Ты хороший, --- прошептала она. --- Ты такой хороший...

Кашу и вяленое мясо тоже пришлось жевать мне.
Затем наступил черёд клизмы --- Кхохо попыталась протестовать (<<Ещё у меня в жопе только козьего рога не было, для полной коллекции>>), но в конце концов сдалась.
Храм, как назло, вымер --- в раскалившемся каменном здании, кроме меня и Кхохо, не было ни души.

К вечеру появились воины.
Я едва успел сказать слово <<Кхохо>>, как все, переглянувшись и хором выругавшись, побежали наверх.

--- Ликхмас, можешь за ней походить? --- попросил меня потом Ситрис.
--- Я предупредил Трукхвала.
Посидишь, послушаешь бред, который она несёт, покормишь, поперекладываешь её с бока на бок.
Два дня всего, ну или чуть побольше, ей совсем худо стало от нашей <<заботы>>...

Поев пару раз, Кхохо начала говорить.
Говорила она много, порой невнятно, и на её лице застыла неодолимая, смертельная скорбь.

---  ...Я очень сильно устала, а она меня выхаживала.
Никто на меня такими любящими глазами не смотрел.
Мне даже трахнуть её не хотелось.
А потом она раз --- и всё, и я совсем одна осталась...

Кхохо начала сухо всхлипывать.

--- ...Я всё для него делала, понимаешь, всё --- служила как оцелот на верёвочке.
А он меня раз ногой под жопу --- и на улицу выкинул.
И я никто, и звать меня никак, и все мои труды пропали...

На следующий день я принёс свежую постель и набрал ванну.

--- Ликхмас, нет, ни за что!

--- Кхохо, это просто вода!
Ты кошка --- воды бояться?
Тихо-тихо, спокойно, ты мне шею сломаешь.
Просто расслабься и дай перенести тебя в душ.
Отцепись, давай, руки разожми, отлично, всё хорошо, я тебя не уроню...

Я взял из дома кормилицы гребень, какие-то масла, ухаживающие средства.
Пока Кхохо хмуро отмокала в ванной, я вымыл несколько раз и начал расчёсывать вечно спутанные, непонятного цвета волосы.
Волосы оказались золотистыми с проседью, даже более золотистыми, чем у Ликхэ, и свернулись в мелкие блестящие кудряшки.
Затем я аккуратно протёр лицо и шею, вычистив из шрамов и морщин застарелую смесь краски и грязи, смазал неожиданно заалевшие губы маслом.
Кхохо повеселела, увидев себя в бронзовом зеркале, но вставать сама всё ещё отказывалась.
Я перенёс её обратно в келью.

---  ...Только родила тогда, ну понимаешь, каково мне, и тут мне такое прилетает.
Я просто зверею, беру толстую палку и с размаху ей по морде.
Так её и похоронили с кривым носом, правда, не сразу, а лет через десять, когда она ласты склеила...

Я долго смотрел на её отмытое лицо, достаточно долго, чтобы уловить сильное сходство...

--- Вы с Ликхэ случайно не родственники?

--- Случайно да, я её родила, как раз тогда, --- ухмыльнулась Кхохо.
--- Вообще на меня не похожа характером, однако ж в одном Храме оказались.
Я ж её трахнула чуть ли не у порога храма, несколько раз, и только через декаду узнала, что это тот самый ребёнок...
Стыд-то какой, что бы про меня в Кахрахане сказали...

--- От Ситриса?

--- Нет, нет.
Это ещё до него было дело.
Заходил в Тхитрон торговец, сладкоречивый как Чхалас.
Рассказывал про летучих рыб, как у них плавники переливаются.
Я глазом не успела моргнуть, как от него залетела.

--- А Ликхэ знает?

--- А ей зачем?
Она и не спрашивала ни разу никого.
Ей сказали, что её воительница оставила малышкой, и все вопросы отпали...
Не для воспитания я, мелкий.
Иногда ёкает что-то, конечно --- когда ты Ликхэ руку сломал, я места себе не находила, пыталась всё её лечить, чтобы как раньше всё стало, Трукхвала трясла, врачей заезжих трясла, да где там...

--- Почему они говорят, что ты несёшь бред? --- не выдержав, спросил я.

--- Я каждому свои истории рассказываю, --- хихикнула она.
--- Кхарасу --- одни истории, Ситрису --- другие, Хитрам --- третьи.
Они думают, что я всё это выдумываю на ходу.

--- Но это правда? --- допытывался я.

--- Нет, конечно, --- ответила Кхохо.
--- Не может столько всего случиться с одним человеком за одну жизнь.
С ума же сойти можно.
Хотя... я и сошла.
Просто рассказываю всё, что помню, а что правда из этого, что мне приснилось после очередной накурки --- кто разберёт...

По её щекам покатились слёзы.

Ещё рассвет спустя Кхохо спустилась к себе на этаж.
Она снова ругалась, хохотала на всю округу, отпускала сальные шуточки и раздавала всем тумаки.
Ситрис смотрел на неё заворожённо, словно влюблённый.

--- Чё уставился? --- наконец раздражённо осведомилась Кхохо.

--- Я не знал, что у тебя такие красивые волосы, --- признался Ситрис.

--- Волосы действительно чудесные, редко такие увидишь, --- поддержала Хитрам.
--- Я вообще подумала вначале, что это родильница Ликхэ заявилась с визитом вежливости.
Это не ты её случайно родила?

--- Да не приведи духи! --- возмущённо фыркнула Ликхэ.

--- Да ладно тебе, --- попытался урезонить девушку Эрликх.
--- Кто бы ни была родильница, всё одно родная кровь, не вода ключевая.

--- Кровь ничего не решает, --- отрезала Ликхэ.
--- Есть люди, которые меня воспитали, я их люблю и хочу быть как они --- хорошо знающей своё дело, хозяйственной и ответственной.
Плевала я на ту бродяжку, которая меня родила и даже не удосужилась навестить.

Кхохо захихикала.
Уже открывший рот Ситрис сделал непередаваемое движение бровями --- не то от удивления, не то от обиды, не то от смущения, --- и, подумав, поинтересовался:

--- Это Ликхмас тебя отмыл?

--- Не привыкай, --- лаконично ответила Кхохо.
\ml{$0$}
{--- Больше эта мелочь со мной сидеть не будет.}
{``This smol shall never look after me.}
Разнеживает чересчур.
Руки бы пообрывать этим, которые его воспитывали, ну кто так делает вообще...

\section{[2] Пустыня}

Картель активно вёл переговоры с ноа.
Оно и понятно: если ноа откажутся нас пропустить, придётся идти вглубь пустыни или сесть на корабли.
Кораблей у нас не было, а большая армия в глубине пустыни имеет крайне мало времени на действия.
Лишить армию снабжения с помощью диверсий --- и она не протянет и двух дней.

Даже по Могильному берегу сели шли бодро.
Многие сделали из плавника и тростника пескоступы на ноги, прочие просто подбили сапоги ещё одним слоем кожи.
Южане в подражание ноа разделись донага, оставив лишь плащи и зонтики.
Жители Дальнего Севера предусмотрительно взяли с собой возы тёплой одежды и одеял --- для себя и прочих;
холодные пустынные ночи мы переживали в комфорте.

\section{Вера в себя}

--- Однажды мы с бабушкой Тхартху сидели на балконе.
Она молча смотрела на городскую площадь, держа в руках перо и пергамент.
Вдруг она сломала пополам перо, разорвала пергамент на кусочки и заявила: <<Как я ненавижу этот город.
Он отнял у меня все силы>>.
Я спрашиваю --- бабушка, если тебе не нравится Тхаммитр, почему ты не переедешь?
Она засмеялась горьким смехом и сказала: <<Милое дитя.
\ml{$0$}
{Ты ещё по-настоящему не теряла веру в себя?}
{You haven't lost you confidence, have you?}
\ml{$0$}
{Надеюсь, тебе и не доведётся>>.}
{I wish you never have.''}

\section{Разбитая миска}

--- Однажды я разбила миску для еды.
Маленькая ещё была.
Кормилица начала меня отчитывать --- вот, дескать, косорукое ты дитя.
А бабушка подошла, остановила её и так строго у меня спрашивает: <<Ребёнок, ты зачем миску разбил?>>
У меня всё внутри сжалось.
Говорю: <<Случайно>>.
Бабушка подумала и сказала: <<Хорошо>>.
Затем схватила вторую миску и шварк её об пол, только осколки разлетелись.

Чханэ усмехнулась.

--- Кормилица ей, естественно, с возмущением так: <<Тхартху!>>
Бабушка говорит: <<Я случайно>>.
Кормилица, с ещё большим возмущением: <<Тхартху!>>
Бабушка: <<Радость моя, ты действительно собралась устроить скандал из-за двух мисок, тем более разбитых \textit{совершенно случайно}?>>
Собственно, на этом всё и закончилось...

\section{[1] Орешки}

--- Жохас, неиссякаемый источник фундука и арахиса, --- отрекомендовала торговца Кхохо.
--- Я на нём половину задницы наела.

--- Он и мне пытался всучить, --- сообщила Чханэ.
--- Но я ему сразу сказала, что не положено.
У нас воины вообще всё за золото получают, и подарить что-то воину --- значит поставить под сомнение воинскую честь.

--- Ну, у нас всё не настолько строго, --- улыбнулась Хитрам.
--- Если предлагают конфетку или орешек --- бери.
И даже сама не стесняйся пробовать --- в конце концов, чем мы хуже крестьян?
А вот что-то крупнее брать без оплаты --- это уже нехорошо.

\section{Планы}

--- Расскажи мне про планы, --- попросил я.

Кхотлам отложила в сторону пергамент и задумалась.

--- Так, слушай.
Для начала --- зачем вообще нужны планы?
Крестьяне --- вольнолюбивый народ и выращивают то, что хотят.
Кому-то нравится выращивать бобы, кто-то предпочитает картофель --- это дело вкуса и умения.
В результате может случиться ситуация, что у нас будет три мешка бобов и пара помидоров на каждого горожанина.

--- Это неприятно, --- усмехнулся я.

--- Вот ты смеёшься, а я в детстве попала на такой год.
С тех пор не люблю кашу из бобов.

--- Это, как я понимаю, был промах купца?

--- Его отсутствие --- старый купец умер, а нового искали достаточно долго.
Но суть ты уловил правильно.
Именно дипломат пишет для крестьян план, который они должны выполнить --- с учётом их личных предпочтений, торговых отношений, климата и многого другого.
Знаешь, что считается показателем мастерства дипломата?

Я помотал головой.

--- Процент планового урожая.
Чем ниже этот процент, чем меньше дипломат вмешивается в работу крестьян, тем ценнее он как специалист.
Впрочем, это, я думаю, верно для любого управленца, включая квартальных старшин и вас, жрецов.

--- Каков твой процент?

Кхотлам усмехнулась.

--- Я не мастер в распределении товаров, к сожалению.
Я планирую одну восьмую урожая --- на всякий случай.
Отличным считается показатель в одну десятую или одну пятнадцатую.
Мне есть куда стремиться.

\section{Другой город}

Чханэ грустно улыбнулась.

--- Сменить пол --- это как переехать в другой город.
Такой же храм, такие же улицы, такие же люди вокруг, но ты не знаешь их, а они не знают тебя.
Даже с самыми близкими приходится знакомиться заново и всё начинать сначала.
Мне повезло, что я была маленькой.
Не всякий взрослый такое переживёт.

\section{[2] Вещи в дорогу}

Мой рюкзак волшебным образом оказался у дверей старого жилища.

--- Интересно, кто же нас всё-таки заметил, --- задумалась Чханэ.

--- Хитрам, --- догадался я, увидев сверху верёвку.

--- Не зря всё-таки твоя кормилица его в доме держит, --- хихикнула Чханэ.
--- Полезный он, даром что крестьянин.

--- Она его любит, --- укоризненно сказал я.
--- То, что он полезный --- это так, приятное дополнение.

\asterism

Оленей с тележкой мы купили в хуторе у заспанного крестьянина.

--- Ну вы даёте, --- буркнул он, рассматривая золотые серьги.
--- Куда только собрались на ночь глядя!

--- По делам Храма, --- весело сказала Чханэ.
--- Прокатимся заодно.

--- Тебя я помню, девка, ты смышлёная, --- сообщил ей крестьянин, ткнув узловатым пальцем прямо ей в живот.
--- А вот ваш вождь, кажись, пень пустоголовый.
Ну кто упряжки покупает в нашей глухомани!
Давайте я вам хоть фонарь в довесок зажгу, не в темноте же ехать.

Крестьянин обладал каким-то талантом зажигать фонари.
Милый розовый фонарик горел всю ночь до самой зари, несмотря на ухабы, ветер и падающие с деревьев капли.

\section{Весёлый Волок}

Весёлый Волок отличался от прочих святилищ.
Во-первых, он состоял сразу из двух городов, соединённых очень широким корабельным трактом.
Во-вторых, правило нейтралитета в Волоке дополнялось одной неписанной традицией: ты можешь хоть с рождения ненавидеть представителей другого народа, но подтолкнуть их застрявший корабль --- святое.
Именно поэтому Волок торговцы проходили гораздо быстрее, чем можно было бы ожидать.

Сразу за Волоком следовало большое пресное озеро, в чём-то похожее на Сотронское.
Это озеро носило забавное название Кха'ма\FM.
\FA{
Наконец-то! добрались! (цатрон).
}
Путей к озеру было только два --- по реке против течения, либо по уже описанному выше Волоку.
Называть озеро по имени следовало каждый раз, когда корабль касался его вод --- разумеется, с оттенком огромного облегчения.

\section{[2] Не надо так}

--- Судя по данным Картеля, Эйраки использовал МПДЛ и генетически модифицированных антарид для устрашения сапиентов.
Именно их сели называли <<радужным безумием>> и <<огнём, оплавляющим стены>>.

--- Странно, что он не сжёг планету к свиньям, --- буркнула Анкарьяль.
--- Антариды мутируют со скоростью света, их вообще нельзя выпускать на поверхности планеты.
Даже разведение плазменных форм жизни в мантии планеты чревато осложнениями --- вспомните Чёрную скалу.
Эти твари каждый год уносят тысячи жизней во время усиления вулканической деятельности.

--- Их так и не вывели до конца, --- поддержал я.
--- Они приспособились к планетным условиям, чего, собственно, и добивались недальновидные экспериментаторы.
Плазмоцисты до сих пор высеиваются даже в нижних слоях планетной коры.

--- Вот именно.
Здесь же, я думаю, помогла только высокая влажность.

--- Да не особо, --- заметил Грейс.
--- Жрецы Ихслантхара рассказывали об обширных лесных пожарах, начинавшихся после радужного безумия.
Выгорали площади в тысячи квадратных кхене вокруг городов.
Обычное пламя не способно заставить гореть такой влажный лес.

--- А люди выживали?

--- Как ни странно, большинство.
Каменные строения оказались неплохой защитой.

--- Собственно, Картель настоятельно порекомендовал Эйраки больше так не делать, --- закончил я.
--- Видимо, даже они были в шоке.

--- Он очень скуп, --- кивнул технолог.
--- Для него гораздо проще применить чрезвычайно опасное, но малозатратное плазмобиологическое оружие, чем тратить масс-энергию на что-то более адекватное.

\section{[1] Люди-тени}

Люди на вечерних улицах --- лишь тени в форме людей.
Ты не знаешь о них почти ничего, ты не видишь ни их лиц, ни цветов их праздничных одежд.
Всё, что ты знаешь о них --- что они есть где-то там, в полотне древесных теней, что их босые или обутые ноги мягко ступают по брусчатке.
Кто-то идёт один, кто-то парами, а некоторые идут толпой, и их тихий шаг, их далёкие речи и смех сплетаются в единый уютный человеческий шум.
Даже твои спутники становятся на вечерней улице лишь тенями.
Даже ты сам.

\section{Убор хоризии}

\begin{verse}
Из убора красотки-хоризии,\\
Что горда и фигурой пышна,\\
Ты принёс драгоценный цветок\\
И мне в волосы вплёл на закате...
\end{verse}

\section{[1] Животные}

--- Манэ, Лимнэ, ваши животные опять сбежали.
Смотрите, не то их задушат коты или заклюют курочки.

Манэ тут же бросилась в угол и вернулась с капюшонной крысой.
Лимнэ подозвала белую шиншиллу и взяла её на руки.
Шиншиллу привезла из Пыльного Предгорья путешественница-поэтесса с зелёными глазами и странным акцентом;
ручная крыса была изловлена в амбаре во время буйного пиршества, да так и осталась жить у сестрёнок.

\section{[2] Яблоко}

--- Как себя чувствуешь?

--- Отлично.
Сны только пошли яркие и волнующие, выспаться трудно.

--- Хай, это бывает, --- засмеялась кормилица.
--- Знакомой крестьянке снилось, что она родила не ребёнка, а то котёнка, то яблоко, то ещё что-нибудь.
Меня больше реакция позабавила.
Обидно, говорит, пять декад ходила --- и яблоко...

\section{[1] Ах, улиточка}

Дождь прошёл, и выглянуло яркое солнце.
В лужах уже вовсю копошилась какая-то мелюзга, на мокрых заборах и стенах сидели полосатые улитки.
Самых больших я мимоходом стягивал и складывал в карман --- на суп или жарево.

\section{[1] Паркетная полянка}

Паркетная полянка была известным местом встречи на окраинах Тхитрона.
И старожилы уже не припомнят, что за здание там стояло;
сейчас от него не осталось даже стен --- только несколько островков изумительного паркета напоминали о деятельности человека.
Паркетную полянку берегли.
Когда несколько дождей назад кто-то отбил одну паркетину, три квартала искали неведомого вандала.
И нашли --- квартал каменщиков вычислил инструмент по зазубринам на камне.
Что с героем сделали, история умалчивает, но больше паркет никто не ломал.

\section{[1] Не хватит}

Кхотлам, увидев половину Храма в праздничной одежде на голое тело, едва не выронила поднос.

--- Молодёжь, вы это зря.
Храмовников, особенно девушек, и так на каждые Руки в клочья рвут, а вы ещё и сами на себя внимание обращаете.

--- У кожевников вообще есть поверье, что если в Мягкие руки завалить воина, то кожи потом никакой клинок не пробьёт, --- радостно сказала Ликхэ.
--- А завалить жреца --- просто хорошая примета.
Жрецы у нас гордые...

--- Не надо на меня так смотреть, --- возмутился я.
--- Я не дичь и не коллекционная статуэтка!

--- Кожевников в квартале тридцать восемь, вас пятеро.
Лисёнок, ты вообще один с Верхнего этажа празднуешь.
Вас на ночь не хватит.

--- У нас особых планов нет, --- успокоил я кормилицу.

--- А у других на вас есть, --- резонно заметила Кхотлам.
--- Опять Кхохо всех подбила, это несомненно.
Впрочем, ладно, все взрослые люди.

--- Надеюсь, Кхохо поделится и уступит мне хоть кого-то из тех тридцати восьми, --- буркнула Ликхэ.
--- У неё свои приметы...

\section{Мысль}

Иногда накатывает странное желание покончить жизнь самоубийством --- лёгкое, воздушное, почти неощутимое.
Оно совершенно лишено негативной окраски, это желание с улыбкой --- лёгкой грустной улыбкой.
Оно появляется из-за мысли о совершенно несбыточной, но страстной мечте, которая поглотила слишком много ночей и раздулась до колоссальных размеров.
Это желание проходит почти сразу --- и, наверное, к великому счастью.

\section{[2] Пиратство}

--- Может, в пираты подадимся?

--- Идея интересная, --- одобрил я.
--- Жаль, что для меня работы у пиратов будет не так много.

--- Пиратам нужны врачи, --- заметила Чханэ.

--- Вот именно, этим их потребность в жрецах и ограничивается.

--- Ааа, --- поняла Чханэ.
--- Ну тогда да, лучше будет поселиться где-нибудь на Короне.

--- Если хочешь к пиратам --- то иди.

--- Слушай, Аркадиу, --- Чханэ произнесла моё имя неожиданно чисто, --- я пообещала, что пойду с тобой.
Поэтому заткнись.

И я заткнулся.

\section{[1] Касания Пера}

Эрликх бился совсем по-другому.
Его удары, неощутимые и невесомые, самым возмутительным образом вытягивали силы.
Даже Кхарас после спарринга с Эрликхом выглядел усталым;
я же валился с ног после трёх-четырёх касаний.
Второй особенность Касаний Пера было то, что синяки начинали жутко болеть сутки спустя.

--- Как ты это делаешь? --- взмолился я после очередного спарринга.
У меня дрожало всё, что могло дрожать.

--- Я просто чувствую, --- сказал Эрликх и, схватив мою руку, прижал к своей груди.
--- Попробуй меня ударить медленно.
Вдави в меня кулак.
Чувствуешь, как изменяется упругость плоти?
Слышишь это скрипение, хруст?
Мышцы вдавливаются по-своему, и каждая косточка сгибается по собственным законам, а значит --- издаёт особый звук.
Ты можешь контролировать это, ты можешь изменять звук так, как надо тебе.
Когда ты будешь чувствовать и слышать удары на обычной скорости боя, ты поймёшь, что делаю я.

И Эрликх ударил меня ещё раз.
Я в бессилии мешком свалился на пол, почти не почувствовав прикосновения.

\section{[1] Брусчатка}

Ливень омыл древнюю тхитронскую брусчатку, на миг обнажив буйство красок отполированных до блеска вулканических пород.
Мало кто из приезжих догадывался, что на мостовой четырёх дорог был выложен узор --- породами аж пяти цветов.
Увидеть это можно было лишь утрами, с Середины Дождя, когда брусчатка сияла влажной чистотой, а облачную серость только-только расцвечивало выглянувшее солнце.
Три-четыре рассвета --- и краски пропадали до следующего сезона.

Я уверен, что те, кто мостил дороги, прекрасно знали, что в итоге всё занесёт грязью, но делали всю эту красоту ради нескольких дней в году.
Тхитронцы очень любили делать неожиданные подарки.
Это я понял однажды, отыскав в своей комнате тайник с рассыпавшимися от старости игрушками.
Если верить выцарапанным на камне иероглифам, тайник сделали строители, видимо, не поставив в известность зодчего;
дети, которым он предназначался, повзрослели, завели своих детей и умерли, так его и не найдя\FM.
\FA{
На языке тси такие подарки дословно называются <<транзит с востока на запад>> или просто <<восток-запад>>, то есть послание в будущее.
}

Единственным незамощённым местом в центре города был дворик моего дома.
Там была плотно утоптанная земля, поросшая редкими кустиками травы.
Возле колодца в земле поблёскивала красивая фульгуритовая ветвь, проходившая через весь дворик.
Молния в это место ударила очень давно, и даже старожилы не знали, когда именно;
но все точно знали, что именно из-за этой красоты дворик и не стали покрывать брусчаткой.

\section{Мудрость}

Есть в людях неуловимые признаки житейской мудрости, помогающей жить и процветать.
Например, зонтик, взятый в день начала дождей.
Или чистые штаны, мыльный раствор и зубная щётка в котомке.
Или надрезанные сеточкой свежие огурцы, лучше пропитывающиеся пряностями и солью.
Сами по себе эти вещи ничтожны;
что изменят пара кхамит в несвежей одежде или приготовленные иначе огурцы?
Однако в целом такие люди и живут дольше, да и в жилищах у них куда больше света и счастья.
Суть часто кроется в мелочах.

--- Мудрость определяется легко, Ликхмас, --- как-то сказал Конфетка.
--- По словам.
Спросишь у человека --- зачем ты это сделал?
А он в ответ: <<Так вкуснее>>, <<Так удобнее>>, <<Так легче>>, <<Так проще>>.
Вот это мудрость.
Если же он отвечает: <<А вдруг>>, <<А если>> --- это или скупость, или опасливость, не имеющие ничего общего с мудростью.
Так, прикормка для собственной тревоги.
Мудрый всегда знает, что он делает и зачем.

\section{Каштаны}

Там, на Дальнем Севере, где распускающиеся в Пирог почки каштанов похожи на неуклюжих большелапых птенцов.

\section{Люби себя}

Любовь --- это когда ты наливаешь чай, насыпаешь каши и гладишь своего любимого, даже если точно знаешь, что лучшие дни его жизни позади.
Любовь --- это верить и помогать, не только в работе, но и в отдыхе.
Постелить постель, потереть спинку, погладить и почистить одежду.
И пусть всё мироздание будет против твоего любимого, ты будешь с ним.
Люби людей так.
И самое важное --- люби себя так же.
Всегда.

\section{Внутренняя свобода}

Наверное, следовало помочь, пожалеть, но я отвернулся и пошёл дальше.
Иногда хочется просто побыть человеком, свободным испытывать чувства.
Свободным любить, свободным ненавидеть, свободным быть верным и свободным предавать.
Для равнодушия порой тоже необходима внутренняя свобода.

\section{Сапоги}

Последними я надел сапоги.
Застёжки и завязки я приводил в порядок нарочито медленно, смакуя каждое движение.
Бегать босиком --- это удовольствие.
Но надевать обувь --- удовольствие совершенно другого рода.
Ни посох, ни котомка, ни убегающая вдаль дорога так не манят в путь, как надёжная обувь, нежно льнущая к стопе и уверенной хваткой держащая голень.

\section{Высота}

Я вздохнул полной грудью и почувствовал, как сразу стало легче.
А ведь только и надо было --- выйти за порог.
Во многих случаях это единственное лекарство.
Не пугали больше ни клинки, ни джунгли, ни высота.
Да и что такое высота?
Всего лишь вертикальная длина.

\section{[1] Мхи (БВ)}

Я погладил мшистый камень барельефа.
Удивительно, сколько всякой живности здесь, у самой реки.
Даже мхов.
Вот <<рыжая шкурка>> --- зелёная моховая шапочка со множеством красно-бурых остистых волос.
Вот <<северное серебро>> --- твёрдый, блестящий, словно бледно-зелёный шёлк, плотно сбитый из круглых стеблей.
Вот безымянный, губчатый, пушистый и нежный, цветом напоминающий о болоте.
А вот золотистые звёздчатые шары <<вечернего красавца>>, рыхлые и жестковатые на ощупь.
То тут, то там в гранит въедались бородавчатые брызги лишайников --- серые, сизые, бурые, ярко-зелёные.
Учитель как-то прочитал, что лишайники --- это тоже грибы.
Наверное, он что-то недопонял;
внимательно рассмотрев лишайник и попробовав его на язык, я признал, что с грибом он не имеет ничего общего.

\section{[1] Пылеройский хлеб}

Все справедливо решили, что для готовки время чересчур позднее, и удовольствовались закуской.

--- Хлеб с помидорами --- это самое изысканное лакомство с голодухи, --- сказал Ситрис, обильно сдабривая лакомство солью.

--- Мы этим конопляное вино закусываем, --- поделилась Чханэ.
--- Усиливает эффект, но похмелье потом такое, что хоть голову в горшке держи.
Хлеб лучше брать пылеройский.

--- Пылеройский хлеб? --- удивилась Кхохо.

--- Да, ржаной с саговыми зёрнами, чёрными скорпионами и, кажется, эстрагоном.
Мы же с пылероями воюем постоянно, трофеев много --- оружие, одежда, еда.
Они распробовали наше, мы распробовали их, в итоге и вышло, что жрём-то мы с собаками одни и те же деликатесы.
А вот в джунглях пылеройский хлеб почти не известен, потому как скорпионов там не разводят.

\section{Бортники}

Мимо прошли три бортника в плотных зелёных костюмах, запевая тягучую <<медовую>> песню.
Они несли три круглых, промазанных глиной корзины и клетку, в которой заливисто чирикала пара ручных попрошаек.

\section{[1] Душ}

--- Здесь твоя лежанка, --- показал Трукхвал.
--- Бельё сам стирать будешь.
Хай, а вон та дверь --- душ.

--- Что?

--- Душ.
Идём покажу.

Трукхвал открыл дверь и, занеся руку с факелом, осветил тёмное помещение.

--- Это изобретение ноа, у которых туго с чистой водой.
У ноа даже есть пословица: <<Существуют только три истинных, чистых удовольствия --- секс, душ и дорога>>.

--- У меня не было секса, ничего сказать не могу, --- пожал я плечами.

--- Торопиться не надо.
Пожалуй, в Храме его даже многовато...
Так что душ --- на случай, если тебе захочется помыться в одиночестве.
Или если в бане Кхохо или Эрликх будут чересчур приставать, они любят молодых...

Я засмеялся.

--- А я, кстати, и не шучу.
Эти ходячие гениталии соблазнили Ликхэ на десятый день в Храме, по очереди, не сговариваясь.
И не спрашивай, откуда я это знаю...
Вон там лампа, свечу прикрывай стеклом, чтобы не погасла.
А вот и сам душ.
Наступаешь по очереди на педальки, и вода из бадьи льётся на тебя.
Здорово, правда?
Только не прыгай на педальках и не лей всякую гадость в бадью, а то были тут у нас любители...

--- А что Ликхэ?

--- Вот молодёжь.
Я ему про такое интересное устройство, а он про Ликхэ... --- Трукхвал смотрел строго, но я прямо чувствовал, что пожилой жрец надо мной забавляется.
--- Кхарас, боевой вождь, накричал на Кхохо и Эрликха --- девочка ещё нецелованная пришла, а они, такие-сякие...
Кхохо ему --- уже целованная, причём везде, я лично проконтролировала.
Кхарас Ликхэ --- понравилось?
А та улыбается, как солнышко ясное.
Кхарас подумал и плюнул --- делайте что хотите, лишь бы по согласию.

\asterism

Предупреждение Трукхвала было не напрасным, но несколько неточным.
Кхохо поймала меня в том самом душе.
Когда дело плавно подходило к началу, в душ вошёл Ситрис и оттащил воительницу от меня.

--- Это. Её. Ребёнок, --- тихо, но отчётливо сказал воин ей на ухо.

--- И что? --- невинным тоном осведомилась женщина.

Ситрис отвесил ей такую затрещину, что она покатилась по полу.

Кхохо дулась на Ситриса весь вечер.
Ночью в спальне была слышна тихая ругань и звуки борьбы.
На завтраке дулся уже Ситрис, прикрывая рукой тёмно-синий фингал.
Но, видимо, воины пришли к какому-то согласию --- отныне Кхохо вела себя со мной исключительно по-дружески.

\section{Большой дом}

Когда я был совсем маленьким, я гордился тем, что у моей кормилицы такой большой дом.
Однако потом в дом зачастили торговцы, носильщики, эмиссары соседних племён, разведчики и прочие люди, с которыми регулярно приходится иметь дело купцу.
Они часто оставались на ночь и всегда --- на трапезу.
В конце концов я понял, что домочадцам во дворе Люм по-настоящему принадлежат только их собственные комнаты, и перестал хвастаться размерами дома перед другими детьми.

\section{[1] Уволочи}

--- Скоро всё будет...
Аийяхс-с, зараза!..

Со стола во все стороны брызнули мелкие, похожие на кошек животные.
Чханэ схватила тяжёлое шерстяное полотенце и стала наотмашь карать воришек.
Кухня наполнилась шипением, ворчанием и жалобным мяуканьем.

--- Хаяй, что за дела! --- пожаловалась мне подруга, когда порядок был восстановлен.
--- Только мясо разделала!..

--- Что такое? --- в дверном проёме появилась Ликхэ, привлечённая суетой.
--- Чханэ, ты готовишь?

--- Готовлю, --- сварливо бросила Чханэ.
--- Тут какие-то котята у меня мясо таскают.

--- Хай, это же уволочи, --- засмеялась Ликхэ.
--- Ты их никогда не видела?
Мы обычно им у окна обрезки оставляем.
Тебе я забыла сказать.
Вот они на стол и залезли.

--- Ликхэ, зачем ты их прикармливаешь? --- возмутился я.
--- Они на ласточек охотятся!

--- Да на помойку ласточек.
Традиция, да, а какой ещё от них толк?
Уволочи много не возьмут, зато мышей и всяких вредителей подъедают.
Не то что эти толстые ленивые оцелоты, которые весь день лежат и бока греют.
Чханэ, извини, речь не о тебе.

Чханэ свирепо засопела, но промолчала.

--- Может, ты не заметил, Лис, но последние пять дождей зерно почти никто не портит.
Как думаешь, почему?

Ликхэ обиженно развернулась и ушла.

Мы с Чханэ помолчали.

--- Зря ты, --- наконец сказала она мне.
--- В целом, неплохо эта рыбина придумала.
В следующее дежурство тоже им оставлю.
В этот раз, --- Чханэ печально посмотрела на погрызенные куски кролика, --- в этот раз они наелись от пуза, так что обойдутся.

\section{[1] Копи}

Однажды ночью меня разбудил Трукхвал.
Он знаками показал, чтобы я оделся по-походному и взял всё необходимое.
Глаза учителя горели непонятным безумным огнём.

Под покровом ночи мы спустились в подземный ход, начинавшийся под бадьёй в душевой храма.
У меня не было уверенности, что кто-то кроме Трукхвала осведомлён об этом туннеле.

--- Куда мы идём, учитель? --- осмелился я нарушить молчание.

--- К тайне, Лис.
К величайшей тайне, --- ответил старик, подняв факел к потолку.

Трукхвал первый и последний раз в жизни назвал меня домашним именем.
Это означало высшую степень доверия... и опасность.
Я чувствовал эту опасность всеми фибрами души --- в фигуре старика, которая вдруг потеряла привычную мне сгорбленность, в его неестественно прямой ноге, которая, против обыкновения, почти не шаркала при ходьбе, и во влажности стен старинного туннеля, которым не пользовались последние пятьсот дождей.

Подземный ход вынырнул на поверхность восле Трухлявой скалы, и почти сразу старик позвал меня в следующий, скрывающийся под корнями старой секвойи.
В нем было ещё тише и темнее, и Трукхвал остановился спустя десять шагов.

--- Теперь слушай меня, --- в глухом шёпоте учителя звучала небывалая серьёзность.
Несмотря на то, что нас вряд ли мог кто-то услышать, он не повышал голоса.
--- Вот в этой котомке костюм.
Ты должен надеть его как можно плотнее, чтобы нигде --- нигде, Ликхмас! --- не осталось щелей и открытых мест.
Ты меня понял?
Надевай.

В котомках оказались странного покроя комбинезоны из очень плотной ткани, высокие кожаные сапоги и кожаные же перчатки, доходящие до локтей.
Я надел непривычную одежду --- она оказалась чуть-чуть длинна и немного узка в плечах.
Трукхвал, быстро справившись со своим комбинезоном, придирчиво затянул и проверил завязки на моём.

Ощущение опасности возрастало.
Что собирался показать мне учитель?
Я лихорадочно вспоминал темы наших последних занятий.
Фауна джунглей, которую я и без того неплохо знал благодаря старому охотнику Сирту.
Может быть, где-то гнездятся опасные насекомые --- зелёные пчёлы или того хуже?
Но почему ночью и под землёй?..

--- Это нечто, что опаснее зелёных пчёл, Ликхмас, --- Трукхвал словно прочитал мои мысли.
--- А теперь возьми эти очки и надень утиную маску.
Нет, не свою --- из котомки.

Я выполнил все указания и привычным движением подпоясался.

--- Нет, Ликхмас, --- покачал головой Трукхвал.
--- Оружие оставь.
Припасы тоже.

--- Учитель, объясни уже, что там! --- взорвался я, бросив нож и жалобно звякнувшую фалангу на камень.
--- Я не ребёнок, чтобы...

Я осёкся на полуслове.
Пещера усиливала голос, словно храмовый зал.
Я напряжённо вслушивался в гулявшее вокруг эхо.
Трукхвал виновато смотрел на меня.

--- Прости, ученик.
Я немного заигрался.
Я... я обнаружил очень сильную Каменную ярость.

Я ахнул.

--- Но зачем мы к ней идём, к ней же нельзя приближаться?

--- Разумеется, можно, --- пробормотал учитель, и лицо его смягчилось.
--- Неприятно, но если ненадолго и при защите, то можно.
Идём, расскажу всё по дороге.
Да оставь ты его, --- недовольно добавил Трукхвал, увидев, что я по привычке схватил мешок с припасами.
--- Если мы там застрянем, то что с припасами, что без --- всё одно смерть.

--- Утешил, --- огрызнулся я.
Трукхвал хрипло захохотал, и его приглушённый маской смех эхом отозвался в коридорах.

\asterism

Сразу за поворотом пещера резко пошла на уклон.
Иногда ход шёл настолько круто, что нам приходилось цепляться за покрытую глубокими трещинами стену.
Удивительно, но здесь не было никакой живности --- вполне возможно, что тот, закрытый старым деревянным люком вход был единственным.
Мы шли, как мне показалось, очень долго, на поверхности уже давно должны были запеть утренние птицы.

Наконец мы добрались до размётанного кострища.

--- Пришли, --- Трукхвал весело передал факел мне и рванул вперёд с неприличной для старого хромого жреца прытью.

Ход оканчивался идеально овальным окном в человеческий рост.
На лежащих рядом камнях я заметил следы кирки.

--- Это я, --- гордо признался Трукхвал, проследив за моим взглядом.
--- Мы сейчас в одном из рукавов копей Древних, их уже давно обшарили наши предшественники в поисках сокровищ.
А этот вход никто, кроме меня, не заметил.
И немудрено --- слышно его лишь при стуке в стену, и потребовалось пять дней, чтобы расчистить путь киркой.
Идём, Ликхмас-тари, здесь начинаются чудеса.

Трукхвал подбежал к овальному проходу... и едва не рухнул в бездонную пропасть.
Я успел схватить старого жреца за комбинезон и втащить обратно.
К моему удивлению, учитель снова захохотал.

--- Ликхмас, отпусти меня.
Ты молодец, среагировал, но в этом не было необходимости.

Трукхвал снова подошёл к пропасти... и шагнул прямо в неё.
Я ахнул --- учитель стоял в воздухе, словно на твёрдой поверхности.

--- Ну как?! --- торжествующе крикнул он в маску.

--- Не может быть... --- я всё ещё отказывался поверить глазам.

--- Иди ко мне.

--- И я тоже?..

--- Сможешь.
Только иди, словно по дороге, никаких лишних движений.

Я подошёл к краю, шагнул вперёд... и рухнул в пропасть.
Желудок переместился куда-то к горлу, я зажмурился и едва удержался, чтобы не издать дикий крик --- сели недостойно умирать с криком на губах.
Перед глазами пронеслась вся жизнь...

--- Ликхмас, выпрямись!
Выпрямись, говорю! --- весело смеясь, кричал мне Трукхвал.

Я открыл глаза.
Ничего страшного не происходило --- тело не падало, а летело с постоянной скоростью вниз.
Я послушался Трукхвала --- и падение замедлилось.
Вскоре мы с учителем дружно, до слёз хохотали.
Моё сердце бешено колотилось --- под ногами не было ничего, но я словно стоял на мягкой, слегка пружинящей поверхности.

--- Теперь слушай, --- заговорил Трукхвал, пытаясь утереть слёзы под очками.
--- Лесные духи, узнаю себя в первый раз...
Наклон вперёд --- вниз.
Руки в зенит --- вверх.
Руки по швам --- стоишь на месте.
Рука в сторону --- летишь в направлении руки.
Попробуй, и главное --- не бойся.
Разбиться, столкнуться с чем-то и даже просто поцарапаться о стену магия тебе не даст --- я пробовал.

\asterism

--- Я уже устал летать, --- признался Трукхвал спустя некоторое время.
--- За эти слова в Храме меня бы забили метлой, но возраст есть возраст.
Давай вниз, нам нужно лететь до цифр на стене.

Я сделал изящный пируэт и остановился рядом с учителем.

--- Цифры как у нас? --- уточнил я.

--- В точности как у нас.
Раньше они светились, а сейчас почему-то перестали.
Да и глаза уже не те...
Так что не пропусти, нам нужна дверь с числом 1D, почти в самом низу.

--- А это что, учитель? --- я указал на надпись, которую приметил ранее.
Трукхвал подлетел поближе.

--- Очень интересно, --- старик привычным движением вытащил из-под комбинезона пергамент и рассыпчатый уголь.
--- Сколько летаю, а этой надписи ещё не замечал.
Молодец, ученик.

Трукхвал приложил пергамент к надписи и протёр его углём.

--- Я... пишу книгу об этом месте, --- признался он, спрятав пергамент в сапог.
--- Эти знаки на стенах я копирую --- вдруг здесь написано что-то важное.

--- А почему ты не сообщил ничего нашему Храму? --- удивился я.

Старик брезгливо фыркнул --- я впервые слышал от него такой звук.

--- Нашему Храму лишь бы только что запретить.
В копи вообще ходить нельзя --- жрецы знают, что там встречается Каменная ярость.
Только крестьяне плевали на запреты, а вот если в копях поймают меня... --- старик выразительно щёлкнул пальцами.
--- Так что пока соберу материал, а потом представлю Советам.
Или ты представишь, если я не успею...

<<Так вот почему ты меня позвал>>, --- догадался я.

\asterism

Вскоре уровень 1D оказался прямо перед нами.
Труквал был прав --- ещё пара уровней, и труба заканчивалась закруглённым дном.
Стало ощутимо жарче.
Я боялся даже представить, на какую глубину мы опустились.

Трукхвал подвёл меня к маленькому окошку.

--- А теперь следующее чудо, --- шепнул учитель.
--- Здесь находится дух, который пропускает к Каменной ярости.
Подчиняется он заклинаниям, написанным в воздухе.

Трукхвал изящно пошевелил пальцами.
В воздухе, пламенея, повисли пять иероглифов --- четыре слова и цифра.
В воздухе высветилось совершенно понятное слово <<ОПАСНО>>, и дверь почти бесшумно отъехала в сторону.
Я открыл рот.

--- Ты не представляешь, Ликхмас, каких трудов мне стоило добыть это заклинание, --- прошептал старик.
--- Смотри.

Трукхвал написал иероглиф <<металл>>.
В ответ в воздухе загорелось несколько слов.
Я пригляделся --- все они начинались на иероглиф <<металл>>.

--- Это заклинания, которые дух понимает, --- объяснил Трукхвал.
--- Некоторые из них показывают непонятный текст, другие выводят какие-то цифры, таблицы и рисунки.
А вот это, --- он написал в воздухе три иероглифа, --- показывает карту шахты.

Я с восхищением смотрел на красивую трёхмерную карту, по которой были рассеяны разноцветные точки.
Трукхвал ладонью мягко крутил её, увеличивал и уменьшал по своему желанию.

--- Сколько времени ты подбирал заклинание, чтобы открыть дверь? --- прошептал я.

--- Три года, Ликхмас.
Три года.
Ходил сюда каждую вторую ночь.
Кстати, в моей книге есть ещё десятка четыре полезных заклинаний.
Ну ладно, пойдём внутрь, --- Трукхвал погладил меня по голове, привычным жестом написал заклинание, и дух, снова предупредив об опасности, отворил закрывшуюся дверь.
Мы вошли внутрь.

Пока Трукхвал возился с гаснущим факелом, я стоял рядом.
Я понимал, что сейчас произошло нечто совершенно необыкновенное но сил удивляться у меня больше не было.
К тому же существовало кое-что поважнее...

--- А про какую опасность говорил дух, учитель? --- спросил я.
--- Про Каменную ярость?

--- Тут много опасностей, --- объяснил учитель.
--- Из-за одной из них я теперь хромаю.

Трукхвал снял капюшон, откинул густые волосы и показал на темени давний шрам в виде шестиугольной звезды.

\asterism

Дальше мы шли значительно осторожнее и тише.
Ход был очень ровным, словно вырезанным искусным мастером по камню.
Я успел заметить на карте, что лабиринт простирался очень далеко --- возможно, у Трукхвала ушли годы на его исследование.
Время от времени нам попадались странные железные существа --- не то жуки, не то пауки размером с жилище.
Они лежали неподвижно, расклячив металлические ноги, словно раздавленные огромной ногой.

--- Одна такая меня и ударила, когда я попытался под неё залезть, --- шёпотом сказал Трукхвал.
--- Проткнула голову, словно кусок масла, я едва успел присесть и отпрыгнуть, чтобы меня не пригвоздило к полу.
Когда очнулся, понял, что череп пробит, а нога не хочет двигаться.
Залил голову смолой, кое-как выбрался, отлежался в храме, стало лучше, но до конца нога так и не вернулась.
Головные боли мучили где-то десять дождей...
Раньше эти звери, если верить хроникам, валялись везде в копях, но местные растащили большую часть на металл.
Никто даже не считался с тем, что это звери Древних --- шла война с ноа, и металл был нужнее.
Остались только здешние --- после того случая я попросил у них прощения и пообещал больше не сердить.

... Сколько мы шли?
Вероятно, несколько часов --- под землёй время тянется очень медленно.
У меня начало сосать под ложечкой от голода.
Наверняка уже полдень.

Наконец в воздухе появилась странная неприятная дымка.
Трукхвал обратил моё внимание на неё, велел смочить утиную маску и ещё раз проверить комбинезон.
Дымка становилась всё более отчётливой.
Странные стальные звери почти перестали попадаться.
И вдруг...

--- Пришли, --- выдохнул Трукхвал и, поёжившись, потушил факел.

Привыкшие к жёлто-инфракрасному факелу глаза какое-то время ничего не видели.
Но я понял, что дымка не висела в воздухе --- всё вокруг освещалось слабым-слабым свечением, льющимся откуда-то справа.
В нём тени как будто тонули.
Границы пещеры расплывались, словно стены состояли из желе или тёмного стекла...
Это свечение вызывало непонятную, смутную тревогу и страх.

--- Видишь источник? --- сказал Трукхвал.
--- Справа, похож на молнию, застывшую в камне.
Это и есть Каменная ярость.
Кстати, я увижу твои косточки, если хорошо пригляжусь.

Я поднял руку и взглянул сквозь неё на Каменную ярость.
Трукхвал сказал правду --- свет Каменной ярости пронзал плоть.
Рука по-прежнему слабо светилась теплом --- одежда приглушает естественное свечение тела.
Но появился новый оттенок, новая игра света.
Я видел каждую косточку в моём запястье.
А вот и сломанная когда-то в играх фаланга пальца, которая не совсем правильно срослась...

--- У тебя лишняя сесамовидная косточка в колене справа, --- весело сказал Трукхвал.
--- Повернись-ка.
О, и зубы не все ещё выросли.

--- Чудеса, --- шепнул я и обежал учителя, чтобы рассмотреть его скелет.
--- Хай, вижу отверстие в черепе.
Ключица неправильно срослась.
И скоба на ребре.

--- Скоба? --- учитель удивлённо посмотрел на грудь.
--- Хаяй, это мы тогда с Ситрисом чинили карниз, я про неё забыл совсем...
Ключица тоже оттуда, с Ситрисом вообще ничего чинить нельзя.

--- Почему?

--- Потому что он мишень для Кхохо!
Он страховку держал, а она мимо проходила.
Думаю, то, что произошло потом, объяснять не нужно.
Думать перед действием --- это вообще не для неё.
Ситрис-то отделался разбитым носом, а я так упал, что дыхание перехватило на пару михнет, думал, помираю.

--- Трааа...

--- Всё, Ликхмас, играм конец.
Мы достаточно долго здесь пробыли.
Забавно, конечно, но у меня, если честно, мороз по коже от этого свечения.
Идём к выходу, и старайся не задевать железных зверей.

--- Но мы только...

--- Ликхмас, если мы чего-то боимся, то это не просто так, --- тон учителя был суров.

\asterism

На выходе мы избавились от комбинезонов и масок.
Трукхвал велел выбросить их в пропасть, когда мы поднялись по воздуху к прорубленному в камне туннелю.

--- Когда придёшь, в душе не мойся, --- предупредил меня учитель.
--- Иди к реке ниже города и поплавай подольше, отмокни.
Пыль Каменной ярости невероятно прилипчивая, и городским глотать её ни к чему.

--- Учитель, что будет, если остаться там надолго? --- задал я вопрос, мучивший меня всю дорогу.

--- Долгая, неотвратимая и мучительная смерть, --- коротко ответил Трукхвал.
--- Бывали те, кто доставал на поверхность <<око земли>> --- маленький сгусток Каменной ярости, который светит ярче Солнца.
Их города поражала эпидемия.
Каменная ярость должна оставаться в камне, Ликхмас.

--- Зачем же её добывали Древние, учитель?

--- Хотел бы я знать, --- вздохнул Трукхвал.

Я смотрел на седые длинные локоны старика, на его спокойные зелёные глаза, крючковатый нос и волевой, совершенно не старческий подбородок.
Больше моего учителя не существовало.
Рядом со мной шагал Герой, который в одиночку сделал шаг в пропасть, пролетел по воздуху, поговорил с духом ворот и сразился огромным с железным пауком ради того, чтобы одним глазком взглянуть на Каменную ярость.

--- Не смотри на меня так, --- нахмурился Трукхвал и отвернулся.
--- Я этого не заслужил.

\section{[1] Дохляк}

--- Этот дохленький, --- сказал кормилец.
--- Если вырастет --- пустим на суп.

Цыплёнок действительно был слабее и меньше прочих.
Я чувствовал в нём не просто слабость.
Он был другим, не похожим на прочих цыплят.

Я схватил цыплёнка в охапку и бросил его о стену.
Он жалобно запищал.
Я сделал это ещё раз, и ещё.
После десятого броска цыплёнок вдруг запищал чуть громче и захромал.
Смутившись, я повернулся и убежал.

За ужином Хитрам сидел озабоченный.

--- Что за день, --- сказал он.
--- Ликхмас, ты сегодня в полдень кур кормил?

--- Да, --- буркнул я.

--- Дохляк лапку сломал, --- объяснил кормилец.
--- Понятия не имею, как у него это получилось.
Надеюсь, что это простая случайность.

Я промолчал.

Дохляку Хитрам наложил шину, но лапа упорно не желала срастаться.
Цыплёнок так и хромал до самой своей смерти.
Суп получился вкусный --- в меру жирный и наваристый.
Все кушали и нахваливали Сирту-лехэ, который оказался знатоком приготовления куриных супов.

Я по-прежнему кормил кур, но старался отделаться от этого как можно быстрее.
В конце концов, кем я был?
Обычным ребёнком, чья голова занята играми, товарищами и прочими интересными вещами, которыми полон мир.
Мне совсем не хотелось думать о том, что я сломал чью-то маленькую жизнь.
Но единожды осознанное из головы уже не выбросить.

\section{[1] Бывшая женщина}

В дверь без стука заглянула улыбчивая растрёпанная крестьянка --- с ней кормилец когда-то жил.
По его словам, дом Кхатрим был чем-то вроде постоялого двора для мужчин --- почти все приехавшие получали там дом, пищу, постель и женщину в лице хозяйки.
Взамен работали в поле, смотрели за детьми и чинили дом.
Детей у Кхатрим на тот момент было аж шесть --- весьма внушительное количество, учитывая Отбор и низкую плодовитость сели.
Двое из них --- братья-близнецы Марас и Хатрас --- были от кормильца.
Во всяком случае, внешнее сходство присутствовало.
Её питомцем был и Столбик, мой друг детства.

Кхатрим заглядывала к нам регулярно, в последних числах месяца.
Хитрам иногда пропадал у неё на день-два --- отдохнуть, проведать детей и помочь по хозяйству.
Сегодня же, видимо, случилось что-то серьёзное.

--- Хай, вы обедаете.
Прошу прощения.

--- Да, Кхатрим? --- обернулась кормилица.
--- Ты что-то хотела?

--- Мне срочно нужны руки, хотела Хитрама попросить.
Дерево упало прямо на дом, хорошо ещё, что мы все в поле были, --- радостно отозвалась женщина.

--- Да ты что! --- расстроилась Кхотлам.
--- Вам приют не нужен?

--- Я с мужчинами пока живу у соседей, но ради такого случая могу прийти в гости.

--- Заходи вечером, --- кивнула Кхотлам.
--- Я что-нибудь приготовлю.
Хитрам, иди, я тебе потом согрею.

Кормилец кивнул и пошёл к двери.

--- А я тебе предлагал его спилить, --- бросил он.
По его лицу бродила слабая улыбка.

--- Я помню, --- заулыбалась Кхатрим.
--- Оно было красивое и очень хорошо смотрелось рядом с домом.

\section{[1] Одна большая семья}

--- Ведь почти все прочие племена обучают военному искусству только мужчин.
Есть, правда, трами, у которых воюют только женщины.
Знаешь, почему наши воины --- мужчины и женщины?
Потому что чисто мужская или чисто женская армия --- это армия насилия!

--- Я слышал, что воины некоторых Храмов представляют собой одну большую семью, --- заметил я.
--- Единственное, в чём им отказывают --- это в воспитании маленьких детей.

--- Так и есть, --- согласился Трукхвал.
--- У нас тоже, хай, семья.
Можно сказать.
Хай, да.
Только в обычной семье обязательно есть взрослые, а у нас взрослых, хай, нет.
Вообще.
Ни одного взрослого.

\section{Соль земли}

Однажды Трукхвал повёл меня на крышу.
Только-только начиналась страда;
по городу звучали крестьянские песни, катались гружёные тележки и ходили носильщики с корзинами.

--- Посмотри, Ликхмас.
В библиотеке легко забыть о них --- о тех, кто снаружи храма.
В мире было много заблуждений на их счёт: считалось, что народ --- исполнители воли, дешёвая оправа для драгоценного камня деятелей искусств, вождей или учёных.
Кого видишь ты?

Я посмотрел вниз.

--- Силу.

--- И это истина.
Что ты видишь ещё?

--- Труд.
Тяжкий труд.

--- Никогда не забывай о том, кто тебя кормит и защищает до поры до времени.
Деятели искусств, вожди и учёные гораздо хуже рабов в этом отношении.
Если у раба не будет хозяина, он будет хоть что-то из себя представлять.
Храм без народа --- ничто.
Даже если народ обратит против Храма клинки, мы не можем ответить им тем же.
Не потому, что не способны, а потому, что перестанем быть Храмом.
Мы --- щит, и рука под ремнями --- рука народа.

Мы помолчали.

--- Пусть тебя не введёт в заблуждение и слава, Ликхмас.
Кто твой любимый лесной дух?

--- Митр, наверное, --- усмехнулся я.

--- Верно.
Утешающий, излечивающий душевные раны и усыпляющий не знающих сна.
Однако я клянусь тебе, что самая безвестная женщина, что вынашивала детей и видела, как изменяются их тела и мысли, излечила больше ран, чем Печальный Митр.
Помни об этом.

--- Я буду помнить.

--- В этом городе сила, --- мимоходом пробормотал Трукхвал под нос.
--- И ведь верно заметил.
Почему я не?..
Почему-то я не...
Хай... Засиделся я в библиотеке...

\section{[1+] Ситрис и штаны}

\textbf{(Часть про платки надо вставить в Семью или ранее)}

Ситрис сидел в зале и шил штаны.

--- Кхарас опять порвал.
И опять в паху, --- объяснил воин.
--- Хозяйство у него чересчур большое.
Попробую дополнительно укрепить.

Ситрис сделал ещё несколько стежков.

--- Когда я... только познакомился с твоей кормилицей, я сделал нехорошую вещь, --- тихо усмехнулся он.
--- Она в наказание заставила меня вышивать салфетки.
Целых полгода я сидел и шил проклятые салфетки.

--- А за что? --- удивился я.

--- Не суть, --- отмахнулся Ситрис.
--- Это было ужасно.
И кто из лесных духов дёрнул меня за язык сказать об этом в Храме?
Теперь обшиваю всех.
Шью всё, что угодно, но только бы не салфетки.
Иногда Кхохо или Эрликх подкладывают мне в постель квадратные лоскутки и цветные нитки, а я их за это бью.

Ситрис продолжил работу, тихо добавив: <<Уроды пресноводные>>.
Но по его лицу бродила слабая довольная улыбка --- спокойное занятие воину явно нравилось.

\section{[1+] Тысяча платков}

--- Когда я пришёл к твоей кормилице, чтобы она ходатайствовала за меня в Храме, я знал, что рассказывать.
Сообщил ей о том, что я был разбойником, что желал бы спокойной жизни.
Она, не слушая меня, задала только один вопрос --- что из сделанного мной я считаю самым постыдным.

Однажды во время налёта на караван Ситрис, приставив нож к горлу предводительницы, потребовал отдать всё, кроме одежды.
При обыске разбойник нашёл вышитый платок из грубой ткани.
Предводительница прошипела, чтобы он держал руки подальше от платка.
Ситрис рассёк ей лицо двумя ударами, оставив на лице женщины уродливую широкую улыбку.

--- Кхотлам сказала, что представит меня Храму, если я вышью тысячу платочков.
Каждый день в течение почти полного года я сидел и вышивал платки...

Лицо Ситриса перекосилось.
Он рассказывал это уже в двадцатый раз.
Каждый раз рассказ доставлял ему жуткий дискомфорт, но замолчать, как видно, ему было очень трудно.

--- Я ел за столом твоей кормилицы, спал возле её очага и вышивал эти платки...

--- Тысяча --- это очень много, --- сказал я.
--- Кхотлам просто хотела посмотреть, сколько ты продер...

--- Я их вышил, Ликхмас.
Всю тысячу, --- перебил меня Ситрис, похлопал по плечу и ушёл в темноту.

\section{Сдерживатель темперамента}

\ml{$0$}
{--- Иногда мне кажется, что моё единственное предназначение --- сдерживать темперамент Кхохо, --- буркнул Ситрис.}
{``Sometimes it seems my only destiny is to restrain \Kchoho{}'s temperament,'' \Sitris\ grumbled.}
\ml{$0$}
{--- Если бы мы не встретились, Кхохо уничтожила бы мир к свиньям и устроила бы богам истерику, что мир так быстро сломался.}
{``If we hadn't met, \Kchoho{} would have destroyed the world and made an uproar to gods because it wasn't stable enough.''}

\section{[1] Загнанный зверь}

Сирту-лехэ долго и упорно обучал меня искусству общения со зверьми.
И только сейчас я понял: тело --- такой же зверь.
Оно думает само, и нужно большое искусство, чтобы подружиться с ним и войти к нему в доверие.

<<Помни, Ликхмас: всё твоё искусство может оказаться бесполезным, если объятый страхом олень понесёт.
Но ещё более опасен зверь, которого загнал ты сам, ибо ты над ним более не имеешь власти>>.

\section{Хэма}

Хэма удерживается в волосах острой шпилькой, напоминающей стилет.
Этой шпилькой вполне можно было ранить или убить;
однако, если её вытаскивали, то тяжёлая заколка хэма немедленно падала на пол.
Это было первым напоминанием: если дипломат берётся за оружие, он тут же перестаёт быть дипломатом.
Вторым напоминанием был вес хэма, заставлявший держать голову высоко поднятой;
лицо дипломата --- не только его собственное лицо.

\section{Оружие и дипломат}

--- Почему ты не взяла оружие?
Хоть ты и дипломат, тебе нужно...

--- Нет, Лисёнок.
Или одно, или другое.

--- Послушай...

--- У купцов, как и у Храма, тоже есть традиции, малыш, --- перебила меня Кхотлам.
--- И я по мере сил стараюсь их чтить.
Меня учили, что дипломат должен всё, включая собственную безопасность, обеспечивать словом, взглядом и жестом.
Если ты этого не можешь --- распишись в собственном непрофессионализме.

\section{[1] Вопрос}

Кхарас перехватил моё ружьё.

--- Кхохо уже там, --- сказал он.
--- Не трать стрелы.

Кхохо действительно была уже там.
Она невозмутимо сидела на песочке;
воины хака весело переговаривались, не замечая присутствия чужого.

--- Как поплавали? --- поинтересовалась воительница на языке хака.

Воины обернулись как ужаленные и бросились на неё с трёх сторон.
Кхохо молниеносно выбросила вверх руки.
Песочек, оказавшийся мелкой сухой пылью, совершенно скрыл воительницу и нападавших;
когда пыль осела, я увидел пять окровавленных тел и отряхивающуюся Кхохо.

--- Придурки, --- буркнула она.
--- Я же просто спросила!

\section{[2] Кукольный театр}

Весело пылал костёр, Чханэ заливисто смеялась, лёжа на раскинутом плаще из кожи нимелто.
Её обнажённая грудь подпрыгивала, оливковая с оранжевой искоркой кожа блестела, как бронза, от огненных всполохов.
Доспехи и оружие валялись рядом в беспорядке.

Несмотря на спешку, привал нам сделать пришлось --- по дороге попался колодец.
Мы напоили измождённых, жалобно глядевших на нас оленей, напились сами и омыли усталые тела в предварительно подогретой воде.
Вернее, мылся только я.
Чханэ, забыв про кишащий идолами лес, брызгала на меня водой и громко хохотала.
Потом я подбросил в огонь отгоняющих насекомых пряных трав, и мы с девушкой поочерёдно прыгали через костёр, прямо в удушливый дурманящий дым, пока не пропахли им насквозь.

Я рассказывал ей интересные случаи, произошедшие со мной в других мирах.
Рассказывал легенды и мифы давно ушедших и забытых народов.
Читал стихи, предварительно установив <<мост сознания>>, дабы она могла понять их смысл и оценить их красоту.
Но ни забавные истории, ни стихи не оказали такого действия, как кукольный театр, который я видел в своём родном мире.
Сделав пять куколок из дерева и тихонько играя на флейте, я заставил их воспроизвести старый-престарый спектакль <<Жених драконихи>>.
Чханэ смеялась и хлопала в ладоши, как ребёнок.

--- Ещё, ещё!

Я показал ещё один спектакль, <<Апельсиновый сад>>.
Радости девушки не было предела.
Глядя на её сияющее лицо и влюблённые глаза, я в очередной раз подумал: <<Бедняжка...>>

--- Ты совсем не изменился, Лис.
Прости.
Я напрасно на тебя накричала тогда, в Тхитроне.

--- Хаяй, а я что тебе говорил? --- улыбнулся я.
--- Ничего и не изменилось.

Медленно светлело усыпанное звёздами небо.
Задул первый утренний ветерок.
Чханэ принюхалась к воздуху и некоторое время смотрела на меня со странной, грустной и извиняющейся, улыбкой.

--- Лис, прости меня, --- тихо проговорила она.
--- Я понимаю, что мы теряем время, понимаю, что твои дела чрезвычайно важны.
Но это было необходимо, потому что, сердцем чую, нескоро мы ещё так отдохнём.
Может, вообще в последний раз...

Она запнулась и ещё немного помолчала.

--- Давай поспим вместе до рассвета.
Ты уже сутки глаз не сомкнул, я же знаю.

Я с нежностью посмотрел в печальные огнистые глаза.
Нельзя, Чханэ, нельзя мне спать.
У меня столько работы...

--- Я могу долго обходиться без сна, Змейка.
Ложись.

--- Змейка тебя ждёт, --- игриво усмехнулась девушка.

Я откинул входной клапан палатки.
Чханэ вздохнула, встала, мимоходом погладив меня по шее, затащила внутрь всю свою амуницию и некоторое время возилась внутри, устраиваясь поудобнее.

Следовало ещё поискать Анкарьяль.
Я не надеялся на успех, чересчур много случайностей происходит.
Её могли убить в бою, принести в жертву, наконец, она могла совершенно банально умереть от болезни.
Демон не был абсолютной защитой от житейских бед, и с риском приходилось мириться.
Но поискать следовало в любом случае.

Я слегка напряг запястье, и браслет ожил.
В палатке зашебуршились, входной клапан отодвинулся, и на моих плечах повисла сияющая Чханэ.

--- Змейка тебя не дождалась.
Хай, а что это?

--- Прибор.
Ищет демонов.
Нужно найти Анкарьяль.

--- Анкарьяль --- женщина? --- строго спросила Чханэ.

--- Она настроена на женское тело, --- ответил я с улыбкой.
--- Неужели наследница самого Маликха подхватила на болоте ревность?

--- Брось-брось-брось, --- Чханэ замахала руками на голубые значки и таблички, и программа тонко чувствующего прибора пустилась в безумный пляс.

--- Хэ, Чханэ, не хулигань! --- я выключил браслет.
--- Тебе колыбельную спеть?

--- Мне не нужна колыбельная, если не на чем спать, --- надулась девушка.

--- А плащ?

--- Он не греет.

--- Он и не должен греть, он должен хранить тепло!

Чханэ вздохнула, расстелила плащ на земле и села передо мной на колени.

--- Я тебя совсем перестала привлекать, да?

Тьфу ты.
Вон оно что.
Вечно я забываю какую-нибудь важную мелочь, когда дело касается психологии.

--- Ты теперь не Лис, а Акхатху, --- девушка была серьёзна.
Она не спрашивала, а утверждала.
--- Сохранилось ли в тебе что-то из прежних чувств?
Любишь ли ты меня?

Как растолковать ей, что чувства --- в большинстве своём производные плоти, важной, но не обязательной части меня?
Объяснения лишь умножат её сомнения.
Ответ на её вопрос может быть только один...

--- Конечно.

Не так уж я и слукавил.
Я всем сердцем желал этому существу счастливой жизни.
Прежний, непробуждённый я жертвовал статусом, здоровьем и своей жизнью, чтобы всего лишь облегчить муки Чханэ.
Чем я отличался от него?

--- Тогда удели мне время до рассвета.
Мне одной.
Не людям, не войне, не вашим хоргетам.
Мне.
Пожалуйста.

Чханэ, не дожидаясь ответа, приникла к моим губам, как умирающий от жажды пустынник.
Я ответил лаской, как мог.
Мир вокруг завертелся.
Вокруг неё.
Вокруг нас.

Рассвет мы встретили в палатке в объятиях друг друга.
Я смотрел на её лицо, угадывая черты далёких предков, о которых она даже не подозревала.
Я смотрел на её лицо и видел тысячу других людей.
Молодых, старых.
Мужчин, женщин, детей.
Тысячи, десятки тысяч жили в этом простом, нежном контуре щёк, иссушённых горячими ветрами уголках миндалевидных глаз, твёрдых сомкнутых губах.
И было во мне лишь одно желание --- любоваться ею, как сейчас, любоваться всегда, вечно, а не те жалкие двести дождей, отпущенные этому телу...

Рассвет я встретил другим.

\section{[2] Гора Песнопений}

--- Что здесь произошло? --- ошеломлённо спросила Тхартху.

Чханэ же трясло от гнева.

--- Святилище осквернено...

\textspace

--- Святилище было гарантией мира.

--- А это значит, что грядёт великая война.

\section{[2] Воля Ликхмаса}

--- Что с нами сделали наши тела? --- пробормотала Анкарьяль.

--- Я не знаю, --- покачал я головой.
--- Знаю одно --- воля человека по имени Ликхмас ар'Люм была сильнее воли Аркадиу Люпино.
И этот факт сотворил настоящее.

--- Кто ты сейчас?

--- Я --- это я.
Это всё, что я могу тебе сказать.

\section{[2] Рабы закона}

--- Они хотят полностью подчинить сели их законам, --- сказала Анкарьяль.
--- Расчёт прост.
Раб закона --- раб того, кто найдёт в законе прорехи.

\section{[2] Вирус (идея)}

Клетки, поражённые вирусом, часто ведут себя так.
Вирус заставляет их делиться, выживать любой ценой, прорастать окружающие ткани.
И в результате стремление к выживанию оборачивается катастрофой для всех --- если раковую опухоль не удалить, она убьёт организм.

Так где же она --- грань между поведением поражённых вирусом клеток и нормальным желанием жить, иметь детей и защищать родных?..

\section{[2] Запасной Король}

--- Кое в чём Картель просчитался, --- сказала Анкарьяль.
--- Они не ожидали, что сели используют их собственную тактику --- <<держи ресурсы в мусорной куче>>.
Да и я, честно говоря, впервые вижу, чтобы обычный крестьянин полноценно заменил Короля-жреца.
В прочих мирах это происходит, да, но в основном стихийно --- появляется талантливый человек, который учится на своих ошибках.
Здесь же имеет место направленное обучение.
Человека учили быть запасным!

\section{[2] Необычное имя}

--- Двор Тхитрона, Манэ и Ликхэ ар'Люм.

--- Благодарю, --- кивнул я.
--- Только одно замечание: не Ликхэ, а Лимнэ.

Жрец вгляделся в записи и охнул.

--- Король-жрец, прошу меня простить.
Имя очень необычное.

--- У моей кормилицы хорошая фантазия на имена, --- улыбнулся я.
--- Купчихи ар'Люм --- мои сёстры.
Продолжайте, пожалуйста.

\section{[2] Эффект Борка}

Я вдруг вспомнил храм Тхитрона, бешеную схватку возле алтаря, демона Картеля, поглощавшего страдания умирающей Чханэ...

--- Эффект Борка, --- сказал я вслух.
Анкарьяль запнулась.

--- Прости?

Я рассказал о произошедшем со мной в храме.
Грейсвольд нахмурился.

--- Стратег с околонулевой устойчивостью отклонил твою атаку?

--- Именно отклонил! --- подхватил я.
--- Не отбил и не увернулся, а отклонил, даже скорее рассеял!
Ситуация по сути редчайшая, чтобы демона атаковали во время питания...
Вот кто из демонов до этого момента принимал эффект Борка всерьёз?

--- Как минимум Картель, который иногда использует армии сапиентов против Ада, --- сказала Анкарьяль.

--- Иногда! --- почти выкрикнул я.
--- Эффект Борка использовал на Драконьей Пустоши я, даже об этом не подозревая!
С тех пор ни одного случая не было зарегистрировано!

Друзья переглянулись.

--- Борк Песчаный Мост сделал своё открытие на пять тысяч лет позже битвы на Серпенциару, --- продолжил я свою мысль.
--- Каков шанс того, что мою победу обработали ретроспективно?

--- Просто скажи, что ты предлагаешь, --- раздражённо бросила Анкарьяль.
--- Плюс-демоны не могут использовать эффект Борка.
Для этого нужны особые сапиенты...

--- Например, потомки тси? --- подсказал я.

\section{[2] Нгвсо и ноа}

--- Это Барабан, --- сказал ноа.
--- Жди, я с ним поговорю.

Из воды показались три глаза и раздвоенное щупальце.
Человек замахал руками.
Щупальце исчезло под водой и вытянуло сплетённую из водорослей сеть, полную искристых зеленых раковин.
Ноа принял ценный груз и, вытащив из бедренного мешка другую сеть, передал её существу.
Нгвсо, аккуратно обвив сеть щупальцем, неторопливо свернулся в родную стихию.

--- Нгвсо выращивают этих моллюсков на еду, а раковины продают нам за бесценок, --- объяснил ноа.
--- Видимо, под водой ракушки выглядят не такими красивыми, как на воздухе, иначе нгвсо ценили бы их куда выше жемчуга.

--- А что ты ему дал?

--- Пирожки с вишней.
Я очищаю вишню от костей, заворачиваю в тесто и слегка обжариваю в масле, чтобы пирожки не слипались.
Нгвсо млеют от этого блюда.

--- Я слышала, что нгвсо приносят не только ракушки, --- полувопросительно заметила Чханэ.

Ноа пожал плечами.

--- Мои товарищи берут у них и жемчуг, а некоторые платят нгвсо за очистку днищ кораблей.

--- И между вами не возникает никаких разногласий?

--- Нгвсо и ноа --- хороший союз, --- наш собеседник снова пожал плечами.
--- Мы защищаем их от стрелохвостов, они предупреждают нас о вторжениях и лазутчиках.
Торговля опять же.
Ладно, путники, мне пора.

Ноа, похоже, потерял к нам всякий интерес.
Схватив посох-зонт и мешок, он неторопливо отправился в сторону рыбацкой деревни.

\section{[2] Пёсьи головы}

--- А на западе живут люди с пёсьими головами, --- сказал старик-ноа.

--- Они называются <<пылерои>>, --- весело ввернул кто-то.

--- Да что ты знаешь! --- вспылил рассказчик.
--- Пылерои --- это двуногие собаки, которые живут, как звери, в пустыне.
А на западе --- люди с пёсьими головами.
Они умеют говорить, едят варёное мясо, пишут мудрёные знаки и умеют их читать.
Разве собака умеет читать?

\section{[2] Мотивы Хэмингуэя (кусок)}

Да, я уже чувствовал это когда-то давно.
Драконья Пустошь, гавань славного свободного города Фриза.
Южное солнце, пальмы, крикливые торговцы, развесёлые караваны цыган, обольстительные красотки на каждой улице.
Корабль-трибот, мерно покачивающийся на волнах.
Фрукты, креветки и проститутки приедались уже через неделю, но это не могло надоесть --- ощущение, что морской ветер ласково гладит тебя по голове, словно так рано ушедший отец, который (я в этом не сомневался) любил меня больше жизни.

\section{[2] Рыбак}

--- Знаешь, Король-жрец, чем меня восхищает жизнь? --- из кармана старика появился крохотный орех.
--- Взгляни на него.
Он кажется таким же мёртвым, как камень.
Даже если с небес упадут снега или огонь, даже если море затопит сушу, его суть не изменится ни на волос.
Он будет подобен камню.
Но стоит растаять снегам, стоит погаснуть огню, стоит морю обнажить хоть маленький пятачок живой почвы --- орех даст росток и вырастет в куст, словно ничего не случилось.
Верю ли я, Король-жрец, что старый мир победит?
Я это знаю!
Но победит не тот старый мир, о котором ты говоришь, а другой --- который не знал ни Безумного, ни дельфинов, ни иных думающих из плоти и костей.
Так было и будет.

\section{[2] Хрусталь}

--- Когда-то давно я посетила малый храм Тхартавирта --- Хритра.
Сейчас его уже нет --- было большое землетрясение, и здание сложилось, как лист пергамента.
Тогда же... я не знаю на Короне ничего прекраснее того храма.

На башни подниматься чужим было строго запрещено, но я уговорила молодого жреца.
Он получил свои любимые восточные финики, а я --- ключ от башни.

Окна в башнях были закрыты толстым хрусталём.
Делал хрустальные листы знаменитый мастер, и я не могу передать, насколько они гладкие и чудесные.
Если их не освещали факелы или лампы --- их не существовало.

Я поднялась почти на самый верх и присела на широкий подоконник.
Был вечер, и молодой жрец ещё не зажёг факелы в башне.
Я отлично помню чувство, которое меня охватило.

--- Что это было за чувство? --- спросил я.

Митхэ пожевала губу.

--- Чувство, что я сижу на головокружительной высоте и ничто не отделяет меня от пропасти.
При этом никакого страха не было.
Удивительно, правда?
Храм стоял на Зелёной скале, Тхартхаахитр лежал передо мной, как на ладони, даже Трёхэтажный казался пирожным.
На такой высоте должен был быть жуткий ветер, но хрусталь глушил все звуки.
Вот такое странное чувство --- чувство высоты и застывшего перед тобой мира.
Я вспомнила это, когда потеряла Атриса.
Когда к лесным духам уходит единственный человек, способный тебя понять, ты начинаешь смотреть на этот мир через толстый, в десятую пяди толщиной хрусталь.
Перед тобой разворачиваются сражения, льётся кровь, люди тебе что-то кричат... а ты ничего не слышишь, ничего не чувствуешь и не боишься.

\section{[2] Два касания}

--- Я всё равно пройду, --- заявил посланец.

--- Попробуй, --- предложила Кхохо.

Чужак, Кхохо и Ситрис схватились за оружие одновременно.
Три сложных росчерка --- и воины тхитронского Храма попятились.
Ситрис хмыкнул, шмыгнул носом, клинки расчертили воздух --- и снова два шага назад.
Сабля Кхохо нервно подрагивала;
Ситриса, похоже, удерживала на месте только необходимость помогать подруге.

Вдруг посланец замер и тупо уставился перед собой;
из кустов вышла Анкарьяль и лениво снесла ему голову.

--- Живые? --- осведомилась она.

Воины кивнули.

--- Поединков с демонами не устраивать, --- распорядилась Анкарьяль.
--- Лучше толпой, а ещё лучше --- яд и стрелы.

--- А если вот так, бежать, что ли? --- возмущённо развела руками Кхохо.

--- Если жить хочешь, --- лаконично ответила демоница.
--- Воины, наверное?
Какой Храм?

--- Тхитрон, --- буркнул Ситрис.
--- Он двигался...

--- Да это нелюдь какой-то! --- взорвалась Кхохо, обвиняюще тыча пальцем в обезглавленный труп.
--- Как можно было предсказать \emph{такие} финты?!
У меня вся жизнь перед глазами промелькнула!

--- Что ты с ним сделала? --- спросил Ситрис.
--- Он словно оцепенел.

--- Дуэль хоргетов, --- пояснила Анкарьяль.
--- Хоргет умер --- тело осталось без наездника.

--- И они все такие?

--- Этот не самый умелый, демон-новичок.
Столкнись вы с кем-то из Манипулы Смеха --- даже я бы вас не спасла.

Ситрис и Кхохо нахохлились и обречённо переглянулись.
Анкарьяль окинула их оценивающим взглядом.

--- Зайдёте в палатку командования ближе к вечеру, дам вам людей под начало.
Не каждый переживёт два касания в схватке с демоном.

Анкарьяль ушла.
Ситрис и Кхохо долго молчали, не глядя друг на друга.
Наконец Ситрис нарушил молчание:

--- Мы тогда послали его на верную смерть.
Ликхмаса...

--- Заткнись, --- огрызнулась Кхохо.
--- Без тебя поняла.

--- Я бы сейчас выпил.

--- Мы в походе.

Ситрис молчал.

--- Всё, хватит, --- буркнула Кхохо.
--- Прошлое --- в прошлом.

--- Мы обречены.

--- Угу.

\section{[2] Нежности Кхохо}

--- Я ему говорила, чтобы он уходил, --- бросила Кхохо.
--- Я давно уже пропащая.

--- Я с тобой, --- сказал Ситрис.
--- Я знаю, что для убедительности следует дать тебе затрещину, по-другому ты не понимаешь.
Но я не хочу.
Сейчас, может быть, в последний наш день, я хотел бы только целовать тебя и говорить всякие нежности.

--- О том, как ты меня любишь?
А то я не знаю, --- грустно пробормотала Кхохо.

--- Не знаешь, --- сказал Ситрис.
--- Насколько я тебя люблю, знаю только я.

--- Печальная судьба --- любить акулу в человеческом обличье, --- заметила Кхохо.

--- По-моему, поздно об этом сожалеть.
Так что продолжай быть собой.
Можешь меня поколотить.

--- Я впервые в жизни не хочу тебя бить, --- сказала Кхохо.
--- Пойдём лучше на берег и посидим в воде.
Ты будешь меня целовать и говорить всякие нежности.

--- Ты мне уступила? --- изумился Ситрис.
--- Не иначе как сам Удивлённый Лю завтра прилетит.

--- Не знаю, кто там завтра прилетит, но я точно умру, --- уверенно сказала Кхохо.
--- Я тебя обещала ещё убить, помнишь?
И убью.
Правда, не совсем понимаю, как это у меня получится.
Ладно, время покажет.
Пойдём, нам ещё поспать надо, а сон перед смертью --- пустая трата времени.

\section{[2] Сокровище Высших}

Огромные металлические ворота были оплавлены.

--- Сколько металла... --- завистливо ухнул Грейсвольд.

--- Мы стережём это место, --- сказал Высший.
--- Здесь спрятана мудрость.

Я подошёл и погладил зеркально чистый оплавленный край ворот.

--- Техногенная катастрофа, --- резюмировал я.
--- Судьба <<Стального Дракона>> их ничему не научила.
Грейс...

--- Ты будешь смеяться, но здесь нет ни одного работоспособного устройства, --- как-то чересчур весело откликнулся Грейсвольд, предупреждая мой вопрос.
--- Взрыв уничтожил все наноструктуры.

--- Они охраняли пустоту, --- заключил я.
--- Давай скажем бедолагам, что мы --- те самые боги и что мы забрали всю мудрость.
А то они ещё пятьсот поколений здесь проторчат.

--- Дождёмся культурологов, пусть они это поаккуратнее сделают, --- предложил Грейс и тронул свесившуюся с потолка трансмиссию.
Мёртвый, слегка покорёженный механизм печально закачался, но скрипа мы так и не услышали.

\section{[2] Трава Тумана}

--- Когда-то давно я очень любил курить траву тумана, --- задумчиво сказал мужчина.
--- Плохая привычка, знаю.
Те, кто имеет пристрастие к этому цветку, не живут долго.
Я знал всё это, знал, чем может это закончиться --- траву курил мой кормилец и его предки.
И всё равно однажды так случилось, что я набил этими ароматными листьями деревянную трубку и сделал первый глоток дыма.
Вскоре я начал кашлять и часто болеть, ноги и руки мои ослабели, но привычка овладела мной --- я не мог провести без травы тумана и дня.
Меня стали избегать женщины, от крепкого запаха моей одежды фыркали даже ездовые олени.
И как-то был годовой поход, во время которого я не мог достать своё любимое зелье.
За это время я вернул свою прежнюю силу, моё дыхание стало чистым.
Но мысль о траве тумана преследовала меня постоянно.
И когда я вернулся в родной храм, то первым делом нашёл среди вещей старую трубку с расколовшимся концом и пакет с сухой травой.
Набивая трубку, я делал это нарочито медленно.
Меня терзали сомнения.
Должен ли я?
Готов ли я пожертвовать ласками женщин и силой своего тела ради этого?
Когда огонь коснулся коричневых с золотым отливом листьев и я сделал первый вдох дыма, то впервые за долгие годы ощутил его вкус.
Это было открытием --- курение давно стало для меня повседневностью, и я даже не знал, насколько вкусна трава тумана.
Следующим открытием стал запах.
Я почувствовал в горьковато-пряном дыму нотки, которых не различал ранее.
Я раз за разом вдыхал дым, и вкус и запах его слабели, уступая место знакомому чувству опьянения.
И вместе с этим пришло понимание.
Я ли выбрал жребий, который привязал меня к этой привычке?
Или, может быть, этот жребий выбрал мой кормилец или его предки?
Наверное, есть вещи, которые мы не можем изменить.
В моих силах лишь выбирать, что важнее --- сила в ногах, женское внимание или чувство умиротворения, которое приходило с белым как снег травяным туманом...

--- Чушь городишь, --- поморщился крестьянин, сидящий по другую сторону костра.
--- Ты это выбрал сам и пытаешься обвинить в своём выборе мироздание.

--- Я не отделён от мироздания, я его часть, --- развёл руками мужчина.
--- Я --- потомок моих предков.
Было бы наивно полагать, что я не зависим от окружающего меня мира.

--- Мы чересчур мало знаем, --- поддержал его кожевник Эрликх.
--- Древние лечили даже печали, и средства у них были куда сильнее травы тумана и молитвенных маков.
Жрец, который учил меня читать и писать, говорил, что печали --- это болезнь.

--- Тхэай, жрецы много чего говорят, --- бросила женщина с закутанным в одеяло ребёнком, по говору крестьянка с севера.
--- У них всё болезни.
Трусость --- болезнь, печаль --- болезнь, влюблённость --- болезнь.
Только ни один жрец не умеет от них избавить, да и кутрапов казнят, а не лечат.
Какой толк в знании, если его нельзя применить?

Я промолчал, внезапно ощущая прилив гордости за этот народ, к которому теперь --- самую каплю, разумеется --- принадлежал и я, уроженец Драконьей Пустоши.

\section{[1] Винт}

Я задумчиво смотрел на медленно вращающийся винт, заключённый в хрустальную трубу. В моей голове зрел вопрос...

--- Учитель Трукхвал.

--- Да, Ликхмас? --- библиотекарь удивлённо поднял голову.
--- Ты с операции?
Всё хорошо?

--- Да, она выжила, слава лесным духам.
У меня к тебе вопрос.

--- Ооо, хай-хай.
Слушаю.

--- У нас на балконе в системе подачи воздуха стоит винт.
Возможно ли с его помощью летать, отталкиваясь от воздуха?

Учитель заулыбался.

--- Хай, вон что тебя заинтересовало.
Да, можно, вполне.
Садись, расскажу, --- учитель с некоторым облегчением отложил в сторону книгу, которую собирался переписывать.

Я сел.
Трукхвал тем временем выудил две чашки и зачерпнул из котелка тёплой воды.

--- Ммм, винт.
Да.
Вообще полёт --- не такая уж хитрая вещь.
Мой учитель строил машинки с теми самыми винтами --- и они летали.
Хоть и есть у сели поговорка, что рыбы не летают, но однажды он ради смеха привязал к одной из них карпа, перед этим поспорив с половиной Храма, что заставит рыбу летать, --- учитель скрипуче засмеялся, обнажив желтоватые кривые зубы.
--- Весь народ потом сбежался глядеть на жрецов, которые стояли на площади на головах...
Хай, отвлёкся я.
Это были маленькие машинки, размером с птицу согхо.

--- А большие?

--- С большими проблема.
При увеличении размеров вес растёт, и дерево не выдерживает.
Нужен кукхватр, да где его столько взять?
Пришлось бы пустить в плавку клинки половины воинов сели, чтобы мог летать один человек!

Трукхвал неожиданно разговорился.
Видимо, летающие машинки бередили его душу уже очень давно.
Вскоре он достал кусок пергамента, сурьмяной мелок и начал чертить сложные схемы.
Я по мере сил пытался вникнуть.
Вдруг, прервав рассказ, Трукхвал хлопнул себя по лбу:

--- Хай, да что я угощаю тебя голыми травами да нравоучениями.
Держи-ка печенье, Ликхэ принесла, ты после операции голодный как ягуар...

\section{[M] Болотная лихорадка}

Кхатрим подозвал меня.
Я склонился над телом.

--- Смотри, Ликхмас, --- тихо сказал жрец.
--- Ты должен запомнить это на всю жизнь.

Кхатрим вынул нож и одним ударом вскрыл ребёнку грудную клетку.
Сердце и лёгкие были странного цвета --- оранжево-красные, слегка опалесцирующие в рассеянном свете солнца.
В нос ударил гнилостный запах.

--- Это не кровь, --- констатировал я.

--- Правильно, --- кивнул Кхатрим.
--- Кровь здесь тоже есть, но её не так много.
Согласно записям предков, это города мелких существ.
Настолько мелких, что они похожи на налёт.
Один жрец, по слухам, изобрёл устройство, позволяющее их видеть, состоящее из линз.
Однако на его город напали.
Жрец погиб, его записи сожжены, и мы не знаем доподлинно, правдивы ли слухи.

--- Сколько их здесь?

--- Я не скажу тебе даже приблизительно.
Не тысячи и не миллионы, много больше.

--- Взрослые могут заболеть?

--- Раненые и старики.

--- Как убить этих существ?

--- Их нельзя раздавить, нельзя покалечить, наши инструменты для этого чересчур грубы.
Их можно только отравить.

Кхатрим вынул из кармана робы бутылочки.

--- Винная эссенция, экстракт мха-ползуна и хлебная плесень.
Винной эссенцией нужно протирать все инструменты и руки, когда имеешь дело с болотной лихорадкой.
Экстракты мха и плесени помогают при приёме внутрь.

--- Можно ли вводить их сразу в лёгкие и сердце?

--- Мы пробовали.
Больные умирают ещё быстрее.
Только так.

Кхатрим укрыл тело простынёй и встал.

--- Раньше всех детей и стариков, если появлялись больные, отводили в храмовые рощи.
Благородный баньян, может быть, ты видел такие деревья?

--- Только слышал.

--- Воздух в этих рощах целебный, существа гибнут там очень быстро.
К сожалению, возле Тхитрона нет ни одной.

--- Храмовая роща помогла бы этому ребёнку?

--- Этому --- нет, --- покачал головой Кхатрим.
--- Если лихорадка зашла далеко, убитые существа начинают гнить внутри, и при лечении отказывают почки.
Дети начинают мочиться кровью, позже всё равно умирают.
Я научу тебя определять безнадёжных, им нужно дать Чёрного Сана и сжечь тела.
Кровавый плащ меняй каждый час, робу утром и вечером.
Перед тем, как ехать в Тхитрон, искупаешься вон там, растворы для втирания в тело приготовят.
Ты бы, конечно, это знал, если бы больше внимания уделял чтению.

Я поперхнулся.

--- Книга в библиотеке.
Отдел медицинской литературы, четвёртая полка, <<Ползучая болезнь>>, Тепло-Полуночного-Костра.
Ты её хотя бы открывал?

Я насупился.
Книга показалась мне скучной с самой первой страницы.

--- Как я узнал? --- Кхатрим без улыбки смотрел на меня.
--- Сама книга тонкая, в тридцать страниц.
Остальная часть --- коротенькое обещание следовать написанному в точности и подписи прочитавших её жрецов.

Я не страдал склонностью к чувству стыда, но тут у меня запылало лицо.
У Кхатрима определённо был стиль --- он рассказывал всё, что требовалось знать, подробно отвечал на любые вопросы, а после этого стыдил за неприлежание так, что покраснел бы сам Удивлённый Лю --- хлёстко и коротко.

--- Ликхмас, я хочу, чтобы ты понял.
<<Ползучая болезнь>> --- одна из важнейших книг по врачеванию.
Все ритуалы, все правила поведения при эпидемиях прописаны на этих тридцати страницах.
Словом, как вернёмся, тебя ждёт экзамен.
Или несколько экзаменов, пока я не удостоверюсь, что ты всё усвоил.
А теперь пойдём, нас ждёт тяжёлая ночь.

\section{[1] Смена пола}

--- Расскажи про смену пола, --- попросил я.

--- Хай, --- задумался учитель. --- Явление это очень, очень редкое.
Но условия его известны давно.

Учитель встал на ноги, прихрамывая, вышел в центр комнаты и театрально развернулся ко мне.

--- Представь, что группа взрослых мужчин оказалась отрезанной от внешнего мира.
Скажем, они плыли на корабле и попали на необитаемый остров.
Они вынуждены жить там длительное время.
Понятное дело, что размножаться они не могут, так как женщин среди них нет.
И вот тут-то включается таинственный природный механизм --- несколько самых сильных мужчин начинают превращаться в женщин, чтобы племя могло воспроизводиться и существовать дальше.

--- Наверное, они очень страдают при этом, --- заметил я.

--- О да, --- откликнулся старик.
--- Их кости и плоть начинают гореть из-за быстрого роста, их охватывает лихорадка.
Эти люди нуждаются в момент превращения в особом уходе.
Может быть, именно поэтому природа отдала роль превращающихся самым сильным из группы.
То же самое, кстати, происходит с мужчинами, оказавшимися в изоляции.
Иногда такие мужчины превращаются в женщин и сразу беременеют --- это называется партеногенезом.

--- А Чханэ?

--- Твоя подруга, насколько я понял, ударилась головой в детстве.
Вероятно, её мозг повредился и механизм запустился сам собой.

--- Она сказала, что бесплодна, --- тихо сказал я.

Трукхвал с нежностью посмотрел на меня.
Такая откровенность растрогала сердце старика.

--- Учитель Трукхвал, ты столько всего знаешь.
Можно ли её вылечить от бесплодия?

Трукхвал виновато развёл руками.

--- Извини, Ликхмас-тари.
Сколько бы я ни учился, пойду по жилам джунглей дураком.

\section{[1] Плачущий ягуар}

--- Лис, скажи правду, почему ты меня спас?

--- Какая тебе разница? --- поморщился я.
--- Спас и спас.
Радуйся.

--- А если бы я была некрасивой и не понравилась тебе, выдал бы ты меня жрецам?

Я задумался.

--- Наверное, нет.

--- Почему?

--- А ты, Чханэ?
Ты хотела меня убить, как увидела.

--- Да, --- растерялась Чханэ.
--- Это нормально, разве нет?

--- Видимо, нет.
Твой вопрос был о том же.

Чханэ замолчала.
Похоже, она уже сама была не рада, что подняла эту тему.

--- Я... мне нужны были доспехи... --- начала она.

--- Отлично, --- поморщился я.
--- А если бы ты мне не понравилась, я был бы мёртв.
Обязательно поднесу пару конфет Хри-соблазнителю.
Или тебя остановило что-то другое?

Чханэ отвернулась и закуталась в одеяло.

--- Так почему ты мне поверила?
Почему не попыталась вонзить нож в спину, как тому жрецу?

--- Давай больше не будем об этом.

--- Это важно, --- настаивал я.
--- Мы спим бок о бок.

Чханэ легла на спину и сдула с лица непослушную <<рыбку>>.

--- Я расскажу тебе историю.

--- У тебя на всё есть истории, --- отмахнулся я.

--- Эта тебе обязательно понравится, --- заверила Чханэ.
--- Так вот.
Однажды к воротам Тхаммитра подошёл ягуар.
Огромный полуторагодовалый кот.
Стража уже собиралась застрелить его, но один из дозорных заметил, что на его лапе висел верёвочный капкан.

--- Ловушка? --- возмутился я.
--- Это бесчестная охота.
Кто промышлял таким?

--- Не мы, --- заверила Чханэ.
--- Возможно, это были звероловы с юга, мы не знаем.
Стражники позвали охотников.
Те были просто ошарашены, что кто-то решился на такую подлость.
Ягуар полулежал, готовясь в любой момент дать дёру, рычал на всех и всё же не уходил.
Наконец один из охотников сказал: <<Ягуар не случайно оказался у ворот.
Ему пришлось пройти с капканом по открытому нагорью, которое обычно избегают, и одним лесным духам ведомо, сколько он прошёл по джунглям>>.

Охотник решился и осторожно подошёл к зверю.
И в тот момент произошло невероятное --- ягуар заплакал и перевернулся на спину.
Капкан причинял ему невыносимую боль.

--- Что они с ним сделали?

--- Охотники взяли ягуара за лапки и принесли его в город.
Он не сопротивлялся.
Позвали жреца, и тот начал снимать капкан.
Кот плакал и повизгивал, но не делал ни одной попытки ударить или укусить тех, кто его держал.
А охотники, на счету которых была не одна шкурка, стояли и плакали, глядя на него.
Я была тогда ещё маленькой, но хорошо запомнила их лица.

--- Они спасли его?

--- Да.
Жрец снял ловушку, обмазал лапку целебными снадобьями.
А потом ягуар целых десять дней жил в полуразрушенной хижине, в которую его принесли.
Его не связали, не заперли.
И знаешь, что самое удивительное?
Год выдался плохим на мясо, зверьё ушло на север.
А всё равно многие приносили часть добычи ягуару, чтобы он ел.
И ни у кого даже мысли не возникло, что с него можно снять шкуру.
Охота охотой, но игру не по правилам поощрять не следует.
Потом, когда кот окреп, он просто вышел из хижины и прошёл через ворота к джунглям.

--- Поучительная история, --- заметил я.

--- Я ответила на твой вопрос?

--- И на свой тоже, --- кивнул я.
--- А что произошло потом, когда капкан был снят?

--- Потом я влюбилась, --- просто ответила Чханэ.
--- Или ты про ягуара?

Я пододвинулся к ней и нежно поцеловал её в губы.
Мои ноздри обжёг запах какао, идущий от её волос.
Огненные глаза превратились в полутьме в отполированные кусочки коричневого шпата.

--- Ну-ка, Змейка, хочешь сладких конфет?

Девушка улыбалась.

--- Хочу.
А какие конфеты у нас сегодня?

--- Да те же, что и вчера, --- тихо рассмеялся я и ткнулся носом ей в ухо.
--- Надеюсь, они тебе не надоели?

--- Ну что ты, --- игриво-укоризненно шепнула Чханэ, развязывая верёвочки на моей рубахе, и, не удержавшись, цапнула меня острыми зубками за плечо.

\section{[2] Причина в людях}

--- Значит, Картель решил начать с Храмов, --- произнёс Грейс.

--- Загвоздка в том, что Картель довершил катастрофу, но не устроил её, --- сказал я.
--- Жрец, которого я встретил, сказал, что Бродячие Храмы начали испытывать давление городских задолго до предполагаемого вторжения Картеля.
Видимо, дело всё же было не в демонах, а в людях.

--- Спустя столько тысяч лет?

--- На этот вопрос не стоит торопиться отвечать до подробного исследования.
А исследование можно провести только после войны, увы.

\section{[2] Математика}

\textbf{(Анкарьяль учит Тхартху рисованию)}

--- Нар, послушай, --- Тхартху задумчиво смотрела на листок бумаги.
--- Мне кажется, что вот эти квадратики равны по площади.

--- Что, Тхартху? --- Анкарьяль склонилась над девушкой.

--- Вот эти, --- Тхартху ткнула пальчиком в листок.
--- Я нарисовала несколько треугольников с... прямым углом и пририсовала к сторонам квадратики.
И площадь этого всегда равна площади двух этих.

--- Мужичьё, идите-ка сюда, --- в голосе Анкарьяль я различил лёгкий шок.
Мы с Грейсвольдом переглянулись и подползли поближе.

--- Что тут у нас? --- Грейсвольд ласково потёрся носом о щеку Тхартху и заглянул в её записи.

--- Похоже, она только что переоткрыла теорему о квадратах сторон треугольника на стандартной плоскости.

--- Вот, --- Тхартху снова показала на свой чертёж.
--- Треугольник и вот эти квадратики.
Их площадь.
Я нарисовала несколько разных треугольников --- результат один и тот же.

Мы с Грейсвольдом переглянулись и засмеялись.
Тхартху бросила на нас обиженный взгляд.

--- Ну что не так?

--- Не-не, Птичка, всё так.
Как ты поняла, что они равны?

Тхартху поджала губы.

--- Я выросла в хуторе.
Мерить землю поручают детям с ранних лет.
Я могу отличить, что больше, а что меньше.

--- Удивительно, --- Анкарьяль всё ещё завороженно разглядывала рисунок девушки.

--- Ну-ка, дай мне листок, --- я аккуратно выхватил его из рук Тхартху вместе с карандашом и наскоро нарисовал две фигуры --- квадрат и круг.
--- Какая из фигур больше по площади?

--- Вот эта, --- девушка указала на квадрат.

--- На сколько? --- хитро оскалился Грейс.

Тхартху нахмурилась.

--- На чечевичное зерно, не больше.

--- Вы тоже поймали дубинку головой\FM? --- поинтересовалась Анкарьяль.
\FA{
Калька фразеологизма, распространённого на Преисподней.
Значение --- сильно удивиться чему-либо.
У одного из народов была игра, смысл которой заключался в перебрасывании игроками тяжёлой дубинки.
Тот, кто зазевался, мог запросто получить травму.
}

--- Ага, --- хором ответили мы с Грейсом.

Тхартху засмеялась:

--- Что-что поймали?
Чем?

--- Дай-ка мне, --- Анкарьяль вырвала листок у меня из рук.
Карандаш я передал ей сам.
--- Тхартху, смотри.

Демоница изобразила на бумаге квадрат, а в него вписала квадрат поменьше.
Рядом изобразила ещё один квадрат и расчертила его.
Не успела Анкарьяль закончить второй чертёж, как Тхартху ахнула:

--- Я поняла!
Поняла!

--- Что поняла?

--- Вот эти треугольники, и да, эти квадратики!..
Ну это... да.

--- Ну это, да, --- повторил Грейс.
Анкарьяль отвесила ему звонкую оплеуху.

--- Главное, что она поняла, каменная башка.
Вот, Птичка.
Это называется <<доказательство>>.

--- А оплеуха --- <<обоснование>>, --- ввернул я и тоже отхватил затрещину.

--- Хай, не лупи моего мужчину, --- возмутилась Чханэ, оторвавшись от котелка с едой.
Демоница бросила на неё презрительный взгляд.
Чханэ ответила тем же, агрессивно повертев в руках черпак.

--- Так, всё, хватит, --- тут же вмешался я.
Отношения Анкарьяль и Чханэ накалялись с каждым днём.
--- Чханэ, это она по-дружески.

--- Я ей руки оторву за такое <<по-дружески>>, --- огрызнулась девушка.
--- При мне тебя никто бить не будет.

Анкарьяль посмотрела сначала на меня, потом на мрачно молчащую девушку... и благородно промолчала.
Я внутренне вздохнул.

--- Слушай, Нар, --- Грейсвольд смущённо поковырялся в траве и, поморщившись, потёр ухо.
--- Тяжёлая у тебя рука.
Если у неё так хорошо с площадями, может, ты ей посложнее что объяснишь?
Интегралы там...

--- Грейс, у неё с умножением туго, а ты про интегралы, --- покачал я головой.

--- Да, плохая идея, --- проговорила Анкарьяль.
--- Я как-то пыталась объяснить кое-кому дифференцирование.
Не вышло.
Это нужно с детства осваивать.

--- А что такое <<интегралы>>? --- аккуратно поинтересовалась Тхартху, с трудом выговорив слово языка Эй.

Её непосредственность была настолько неподдельной, что мы засмеялись во весь голос втроём.
Анкарьяль прикрыла лицо рукой:

--- Земля и небо, мои дарители...

--- Короче, Нар, у тебя нет выхода, --- сквозь слёзы пробурчал Грейс.
--- Давай интегралы.
А там, глядишь, и до физики пространств с градиентом мерности недалеко.

\section{[2] Величие Древних}

--- Эх, если бы у меня была ваша сила... --- мечтательно потянулась Чханэ и невзначай пощупала меня.

--- А что бы ты сделала, обладай ты нашей силой? --- спросил я подругу.

Чханэ сладко зевнула.

--- Я бы лечила людей... Учила бы детей... Да много чего можно придумать.
Те же плохие годы, с ними можно что-то сделать.

--- Да я бы не сказал, что сейчас всё так плохо с болезнями, --- заметил я.

--- Конечно плохо! --- Чханэ посмотрела на меня, как на дурака.
--- У нас в позапрошлом году троих детей съела лихорадка.
Четверо мужчин умерли от укуса смертожара.
По-твоему, этого мало?

Я улыбнулся.

--- Конечно, нет.
Просто я читал записи о Древней Земле.

--- Древняя Земля?
Что это?

--- Прародина человечества.

--- Хай, жрецы называют её Тхидэ.

--- Нет.
Тси-Ди --- ваша прародина --- тоже развитая цивилизация, но она не первая.
Это было раньше, намного раньше.

--- Ещё раньше? --- открыла рот Чханэ.

--- Есть данные, что тогда умирала половина детей, --- перешёл я к сути.
--- Были страшные болезни, которые заставляли людей гнить всю жизнь, и заразиться ими можно было во время соития.
А были не менее страшные, которые передавались по воздуху.
От них за декаду вымирали целые города.

Чханэ не ответила.

--- Мужчины могли умереть от простой царапины в те времена, а женщин часто уносили роды.

--- Да как такое может быть? --- возмущённо воскликнула Чханэ.
--- Это звучит, как какой-то кошмарный сон.
Безумный и вполовину не так жесток...

--- \ldots как природа, --- закончил я.

--- Почему же ничего этого нет? --- тихо спросила девушка.

--- Первые люди.
Жители той самой Древней Земли.
Они уничтожили возбудителей болезней, они с помощью генной инженерии избавились от генетического груза, который тащили миллионы дождей.
Одна человеческая жизнь --- пять поколений --- потребовалась, чтобы полностью вычистить генофонд человечества.

--- Генофонд --- это...

--- Наследственность.
То, что передаётся от предков к потомкам.
Но древние сделали не только это.

--- А что ещё?

--- Ты знаешь, что у нас в гортани есть звукопроизводящий орган?

--- Гортанная цитра?
Да, конечно.
Я видела лёгкие трупов.
А что с ней не так?

--- Для чего она нужна?

Чханэ задумалась.

--- Я не знаю.
Говорить, петь, --- Чханэ очень похоже изобразила свист птенца согхо, и в кронах ближайших деревьев немедленно отозвались взрослые птицы.
--- Разве нет?

--- Модуляцией голоса люди управляли сложнейшими машинами, для таких были непригодны даже самые нежные руки.

--- Только не говори, что у Древних не было цитры!

Я улыбнулся.
Чханэ издала губами неприличный звук и ударила руками по земле.

--- Лесные духи.
Наверное, их голоса были плоскими и невыразительными, --- Чханэ мяукнула оцелотом, всполошив окрестных птичек ещё больше.

--- За то, что ты и твои соплеменницы почти до конца беременности можете радоваться жизни, спокойно работать и даже сражаться, тоже благодари Древних.
Ах да, и за почти безболезненные роды, и за отсутствие менструаций тоже.

--- Отсутствие чего?

--- У древних женщин из матки каждые четырнадцать дней шла кровь.

--- Они истекали кровью? --- ужаснулась Чханэ.
--- Зачем?

--- Их тело работало по-другому.

--- Хаяй... --- Чханэ произнесла это таким тоном, словно разочаровалась в любимом герое легенд.
--- Нам говорили, что Древние были выше и сильнее нас...
А что ещё сделали Древние?

--- Нам мало известно про биологические особенности людей до Эпохи богов.
Я знаю лишь, что у них были одни зубы на всю жизнь.
Если потерял --- ходи без зубов.

--- Грустно им было, --- Чханэ проверила зубы пальчиком.
--- Мне как-то выбили резец, так давно уже новый вырос...
Девка поленом стукнула за то, что я строила глазки её мужчине.
А я её головой в ослиное дерьмо макнула... прямо на его глазах, --- Чханэ захихикала.
--- Скажи, Лис, а правда, что у людей бывают голубые глаза?
Как небо в зените.

--- Да, --- улыбнулся я.
--- Я видел даже людей с белоснежной кожей и волосами цвета соломы.

--- Это чудесно, --- мечтательно пробормотала Чханэ и ущипнула меня.
--- Я бы хотела увидеть таких.
А фиолетовые глаза бывают?

--- Бывают и фиолетовые.
На планете Запах Воды живут родственники пылероев с яркими фиолетовыми глазами.

--- А красные?

--- И красные.

--- Это чудо.
У нас только зелёные и карие.

Я подтянул девушку поближе и поцеловал её в висок.

--- Многим показалось бы чудом, если бы они узнали про оцелотовые глаза, --- прошептал я.
Чханэ смутилась.

--- У меня они уже позеленели.
Я вдалеке от родных мест.
Но всё же, Лис, --- в её голосе вновь зазвучала сталь, --- от болезней не должен умирать никто.
И я бы лечила всех.

--- Я тебя научу, --- пообещал я.

--- Хорошо, --- пробормотала девушка и уткнулась мне в шею.

Анкарьяль всё это время с отсутствующим видом ломала сухую веточку на щепьё и кидала его в огонь.
Вдруг она подала голос:

--- Научить-то научишь.
А с Безумным что делать будем?

--- Нар, давай потом, --- проворчал откуда-то из темноты Грейс.
--- У меня уже глаза слипаются.

--- Принять решение нужно быстро.
Если нас поймают и казнят, думать будет поздно.

--- Пусть казнят, только дай мне поспать.

--- Я вас не узнаю, --- проговорила Анкарьяль.
--- Вы стали относиться к заданию, как к прогулке в парк.

--- Разведка, сбор информации, --- я попытался расставить все чёрточки над иероглифами.
--- Но вначале --- спать.

--- Может, хватит уже? --- внезапно подала голос Тхартху.
--- У меня от ваших разговоров мороз по коже.
<<Строить машины>>, <<научить медицине Древних>>, <<уничтожить Безумного>>...
И это таким тоном, словно вы поесть собрались.
Вы хоть немного уважения и страха имейте!

Мы промолчали.
Грейс вздохнул --- похоже, он уже и сам жалел, что взял с собой подругу.

--- Чханэ, скажи им уже!

--- Они знают, что делают... --- начала Чханэ.

--- А я вот не уверена!
Ведут себя, как дети, спокойно говорят о вещах, которые я или не могу представить, или представлять их просто страшно!
Да и как воевать с тем, которого не видно и не слышно?

--- Тхартху, --- твёрдо сказала Чханэ, --- пусть ребячатся, мы всегда так делаем перед тяжёлым походом.
Я им верю.
Давай-ка спи.
И я буду спать.

--- Ты веришь всем, кроме Нар, --- вдруг выпалила Тхартху.
Анкарьяль удивлённо посмотрела на девушку.
Чханэ поджала губы.

--- Я верю и ей, --- медленно и чётко проговорила Чханэ.
--- Она заносчивая ящерица, но Лис и Карп не станут доверять кому попало.

На этот раз ошарашенный взгляд Анкарьяль настиг мою подругу.

--- И давай уже спи, Тхартху.

--- Можешь устроиться у меня на пузе, --- сонно предложил Грейс.

Тхартху едва слышно выругалась.

--- Угораздило меня с тобой связаться, кудрявая панда...

--- Пузо мягкое, --- заметил Грейс.
--- Аркадиу, так ведь?

--- Да уж помягче моего, --- вздохнул я.
Чханэ захихикала.

Тхартху демонстративно оттащила спальный мешок и устроилась рядом со мной и Чханэ.
Грейс вздохнул.

--- Ладно, спокойной ночи всем.

Наступила тишина.
Анкарьяль по-прежнему невидящим взором смотрела в костёр, скидывая в пекло оранжевых языков одну щепочку за другой...

\section{[2] Приманка}

--- Что не спишь, Кар? --- Грейс хлопнул меня по плечу.

--- Меня кое-что беспокоит, --- уклончиво ответил я.

--- Случайно не осцилляции ПКВ? --- усмехнулся Грейс.

--- Так ты тоже их почувствовал, --- произнёс я.

--- Самое интересное, не я, --- ответил Грейс и покачал своим многофункциональным браслетом.
--- Я бы даже не заподозрил, что что-то не так.

--- Что именно ты почувствовал? --- спросила Анкарьяль.

Я задумался.

--- Плюсовое искажение.
По идее, за счёт присутствия Безумных на поле должен быть перевес минуса, а тут... как на нейтральных планетах.

--- Локализация? --- требовательно произнесла Анкарьяль.

--- Нужны вычисления, --- резонно ответил я.

--- И не просто вычисления, а дешифровка, --- как-то злобно усмехнулся Грейс.
--- Компьютер твёрдо уверен в том, что вокруг центра планеты имеется плюс-облако.

--- В смысле --- облако?
Ты имеешь в виду сеть треков?

--- Это не треки, Нар.
Не обычный планетарный омега-фон.
Это сингулярность с неопределёнными координатами.
Поэтому я и говорю --- облако.

--- Это физически невозможно, --- констатировала Анкарьяль.
--- Дурит кто-то твой компьютер, Грейс.

--- Похоже на приманку, --- добавил я.
--- Грейс, больше не трать на \emph{это} энергию.
Мы с тобой и так уже... потратились.

Грейсвольд смущённо хмыкнул.

--- Кто бы это ни был, о нём у нас никакой информации.
Будем пока что исходить из того, что мы на Тра-Ренкхале одни.
Меня только интересует, каким образом было создано это облако?

--- Иллюзия восприятия, --- отмахнулась Анкарьяль.
--- Ты, Аркадиу, с такими технологиями ещё не сталкивался?
А мне приходилось.
Заходит с тыла отряд, как мы думаем, наших.
А они р-раз и по нам экранами лупить начинают.
Свечение от них плюсовое, да.
Как я живой оттуда ушла --- сама удивляюсь...
Тут интересно другое: какого червивого дьявола мы с Грейсом не чувствуем ничего?
Поле-то одно для всех!

--- Мне тоже, --- поддержал её технолог.
--- Какое-то устройство не просто изменяет поле, оно ещё и в курсе, где находится каждый из нас, и даже как-то нас классифицирует.
Аркадиу, чем ты отличаешься от нас?

Вопрос был риторическим --- я урождённый человек.

--- И возможно, единственный таковой на сотни парсак вокруг, --- продолжила мысль технолога Анкарьяль.

--- Необязательно, --- возразил я.
--- А как вы объясните то, что осцилляции видит компьютер?

--- Компьютер использует сапиентные паттерны поведения, чтобы запутать наблюдателя, --- сообщил Грейсвольд.
--- Кстати, эту идею ты мне подал.

--- Может, нам следует разделиться, --- предположил я.

--- Нет, --- отрезала Анкарьяль.
--- Возможно, что этого от нас и хотят.
Вряд ли это устройство Картеля, иначе нас бы уже давно взяли под белые рученьки, но сам феномен похож на тщательнейшим образом подобранную приманку.
Будем исходить из того, что кто-то знает о нашем местоположении... и вести себя так, словно мы ничего не заметили.

Мы с Грейсом переглянулись и вздохнули.
Тайна манила, мы отчаянно нуждались в союзниках, но Анкарьяль была, как всегда, права.
И даже более, чем всегда.

\section{[2] Счётная мельница}

--- Лис, а то твоё имя, --- Чханэ неуверенно назвала его, --- что оно значит?

Я задумался.

--- Arccadiu --- <<крестьянин>>, распространённое мужское имя.
Balerianu --- кажется, <<воин>>, baleru --- устаревшее <<воевать, сражаться>>.
А родовое имя Luppino ознaчает <<шакал>>.
Luppina --- <<самка шакала>> --- название для... --- я задумался, вспоминая подходящее слово на языке сели.
Не нашёл.
--- Хай, это женщина, которая занимается сексом с мужчинами за товары или еду.

--- А, --- поняла Чханэ.
--- Бывает такое, да.
А почему для них отдельное название?

--- В моём родном мире женщины не были равны мужчинам.
Их половая жизнь ограничивалась.
В качестве противовеса существовали женщины, которые продавали секс.
Общество их презирало.

--- Женщинам не давали заниматься сексом?

--- Да.
Общество считало предосудительным, если у женщины было более одного мужчины...
Что смешного?

Чханэ больше не могла сдерживать смех и захохотала во всё горло.

--- Хри-соблазнитель... а как... они... определяли?
Счётную мельницу вставляли в женские ворота?

Я подождал, пока она успокоится, и рассказал ей про девственную плеву.
Чханэ поперхнулась.

--- Это специально делали?

--- Нет, у женщин моего вида это было изначально.

--- Глупость какая-то.
Я вначале удивилась, когда ты сказал <<продавали секс>> --- это ж каким неприятным должен быть человек, чтобы никто даже не согласился с ним разделить ложе.
Нет, у нас тоже бывало, что кто-то обменивал вещи на секс, но это всё-таки больше символический обмен, удовольствие-то оба получают.
Впрочем, теперь всё понятно.
Женщины твоего мира не имели больше одного мужчины --- с ними спать, как лягушкам стихи читать.

На этот раз от такой неожиданной интерпретации поперхнулся я.

--- А у тебя когда первый секс?

Чханэ задумалась.

--- Именно любовь?
Или Круг Доверия тоже считать?
В Круге я участвовала с первого похода, дождей с двадцати трёх.
Любовь была чуть позже --- Манис.
Я тебе про него рассказывала.

--- Как прошёл твой первый Круг?

--- Примерно так же, как и у тебя --- трезвучие, куча гениталий и слабое понимание происходящего.
Правда, я была совсем молодой, поэтому за несколько дней до Круга кормилец позвал меня к себе, чтобы я чувствовала себя увереннее.
Так я стала женщиной, --- Чханэ тепло усмехнулась.

Я погладил девушку по обтянутой тканью коленке.

--- Где сейчас твой кормилец? --- спросил я.
--- Он жив?

Улыбка Чханэ увяла, словно молитвенный мак.

--- Хотела бы я знать.

--- Отчего так?

--- Когда кормильцы перестали жить вместе, мне было уже много дождей.
Меня готовили в Храме, поэтому я ушла с кормильцем в храм насовсем.
У Согхо бывала редко --- она не особенно меня любила, а после расставания с Акхсаром и подавно.

--- Почему ты думаешь, что она тебя не любила? --- удивился я.

Чханэ скорчила рожу.

--- Знаешь, это не так трудно понять, особенно маленькому человеку.

--- Так что насчёт кормильца?

--- Однажды в Храме сменилась власть.
Кормилец встал на защиту прежнего Первого --- он был его другом.
Ту ночь я провела у Согхо --- один из воинов, старый товарищ Акхсара, вскользь посоветовал мне пожить несколько дней вне храма.

--- Похоже, просто так подобные советы у вас не давали.

Чханэ многозначительно кивнула.

--- Кормилец пришёл далеко за полночь, весь в крови, с пробитыми доспехами.
Обнял меня и сказал, чтобы я вела себя в Храме так, как и всегда, и ничему не удивлялась.
А потом заплакал, сказал, что любит меня, и убежал.
Тех пятерых, кого он зарубил в ту ночь, я хоронила на следующее утро.

--- С тех пор вестей не было?

Чханэ вздохнула.

--- За ним послали убийцу, так что, скорее всего, мой кормилец уже давно прошёл жилами джунглей.
В иное я перестала верить.

\section{[2] Страшная сила}

--- Мне только одно непонятно, --- заявила Чханэ.
--- Вы --- самые сильные существа в мире.
Где ваше оружие?

Я рассмеялся.
Анкарьяль и Грейсвольд недоумённо переглянулись.

--- Оружие?

--- Да.
Летающие машины, извергающие огонь.
Ножи, прорезающие камень.
Доспехи, которые выдерживают выстрел из баллисты.
Где всё это?

Грейсвольд, кажется, начал кое-что понимать.
Анкарьяль выглядела ещё больше сбитой с толку.

--- Какие машины с огнём?
О чём ты?

--- У меня есть браслет, --- растерянно сообщил Грейс.

--- Чханэ, --- вмешался я.
--- Мы делаем оружие из воздуха.

--- Зачем нам машины с огнём, --- проворчала Анкарьяль.
--- У нас всегда было и есть два оружия --- знания и демоническая сущность.

--- И неизвестно, какое страшнее, --- присовокупил Грейс.

\section{[2] Песня о доме}

--- Атрис как-то спросил меня, как я выжила среди всех этих сражений.
А я мечтала о доме.
Многие мои товарищи мечтали о далёких походах, о новом оружии... а я о доме.
И вот, во время одного из походов дом у меня появился.

--- Помню этот поход, --- засмеялся Акхсар.
--- Золото притащила худого, оборванного, насквозь мокрого красавца и сказала: <<Это цитра моей жизни.
Не обижать>>.
После первой же песни мы поняли, что перед нами человек, путь которого начертан сохой небесного пахаря.
Мы словно переосмыслили свои жизни.
Многие плакали, говорливые вдруг надолго замолкали, а молчаливые, наоборот, начинали говорить, и их нельзя было остановить.
Золото и Хат отныне спали в отдельной палатке и называли её <<домом>>.
А почему домом, спрашиваем мы?..

\begin{verse}
Что отличает жилище?\\
Крыша, и пол, и огонь,\\
Пища на вечер и утро,\\
Две пары любящих рук,
\end{verse}

--- закончила Митхэ.
--- Атрис спел это не задумываясь, словно кто-то диктовал ему стихи...

Акхсар затянул весёлую песню, и Митхэ подхватила её:

\begin{verse}
Если нет крыши --- то это лишь лагерь,\\
Если нет пола --- то это нора,\\
Если нет пламени, если нет пищи,\\
Это сарай --- посели здесь кролей.\\
Если нет любящих рук и надежды,\\
Незачем крыша, и пища, и пол.
\end{verse}

Воины рассмеялись.

--- Мы пели её по десять раз, --- припомнил Акхсар.
--- А Хат играл нам на цитре.

\section{[2] Спокойная старость}

Хитрам вскоре нашёл меня.
В руках жрец сжимал испачканный в глине платок.

--- Моей хранительницы, --- сказал он.
--- Пришлось принести её в жертву.
Больше было некого.
И некому.

--- Оставь, --- я аккуратно отстранил его руку с платком.
--- Это чересчур большая ценность для меня.
Твоего жеста вполне достаточно.

Я отошёл, и вскоре раздался сдавленный возглас.
Хитрам-лехэ дрожащими пальцами расправлял совершенно чистый платок, на котором красовалась надпись: <<Самому лучшему дарителю>>.

Жрец никому не сказал ни слова.
Но где бы я его ни увидел, он не расставался со своим сокровищем.

Вскоре ко мне подошла Анкарьяль.

--- Аркадиу! --- тихо сказала она мне на ухо.
--- Это ты Хитрама разыграл?
Что ещё за дешёвые фокусы?

--- Пусть человек порадуется, --- сказал я.

--- Его хранительницы давно нет, --- сказала подруга.
--- Сапиенты умирают навсегда, и пристанище --- лишь сказки для живых.
Все знают о твоей божественной силе...

--- \ldots но никто не осведомлён о её границах, --- тонко улыбнулся я.

--- Хитрам --- образованный человек.
Если он узнает истину, это его убьёт.

--- Посмотри, --- кивнул я на старика.
Тот рассеянный взором смотрел в окно, мял в руках платок и улыбался.
--- Посмотри внимательно, Анкарьяль Кровавый Шторм.
Нужна ли этому усталому человеку истина?
Будет ли он искать опровержения для чуда, которое позволит ему дожить своё в умиротворении?

Анкарьяль смотрела на меня.
В её глазах застыло что-то непонятное.

--- Я передам Грейсу, чтобы он не распускал язык.

--- Благодарю тебя.

--- Не нужно, --- шепнула Анкарьяль и взяла меня за руку.
--- Я иногда жалею, что в этой Вселенной не найдётся доброй сказки, в которую могла бы поверить я.

Я улыбнулся и прижался к подруге.
Она, как всегда, читала мои мысли.

\section{[2] Внезапно}

--- Существует-Хорошее-Небо — травник? --- удивился старый жрец.
--- Мы всегда думали, что он человек!

--- Я поговорила с пылероями, --- добавила Анкарьяль.
--- Они рассказывают легенды про вождя Существует-Хорошее-Небо, и в них он пылерой.

--- Я почти уверен, что планты считают его плантом, --- заключил я.

Мы переглянулись и захохотали.
Жрец нахмурился.
Для него только что осознанное было откровением, и наш смех казался чем-то неуместным.

--- Я понял, --- сказал жрец.
--- Предки не делали различий между людьми, пылероями, травниками и няньками...

--- Делали, --- сказал я.
--- Для размножения они, разумеется, выбирали особей своего вида.
Но древних тси связывала общая культура, многочисленные дружеские и половые связи.
Ни у кого не возникало даже мысли, что другой вид, к которому относились его товарищи по работе, друзья и, возможно, даже любовники, чем-то хуже своего собственного.

\section{[2] Эпилог}

Вспомнился разговор с Чханэ.
Мы стояли на башне храма, и под нами расстилался засыпающий город.
На западе пылал закат.

--- Идём со мной.
Я сделаю тебя хоргетом, и мы будем встречаться снова и снова, проживать жизнь за жизнью.

Чханэ улыбнулась.
В углах её глаз уже рассыпались морщинки, тяжёлые волосы цвета какао сверкали платиновой филигранью.

--- Лис...
Я прожила вдвое больше своих родичей, и чувствую, что мне... много.

--- Ты не хочешь быть со мной?

--- Скажи, Лис, я сильная?

--- Конечно.

--- Я всегда такой была.
Позволь же мне напоследок эту слабость --- просто умереть.
Исчезнуть навсегда.

--- Я думал, все люди мечтают о бессмертии.

Чханэ провела рукой по волосам и посмотрела на горящий над джунглями закат, потом улыбнулась мне.

--- Я всегда мечтала о покое.
А бессмертие... я уже бессмертна.
Народ сели, пылерои и идолы будут помнить тебя, пока не умрёт последний старик, пока последняя старуха не отправится к лесным духам.
И, вспоминая тебя, рассказывая о тебе легенды, кто-нибудь да помянет меня добрым словом.
Наши потомки будут жить, пока не погаснет солнце и не остынут океаны, и в каждом будет течь капля моей крови.
Ну а после...
Ты же меня не забудешь?

--- А когда исчезну я?

--- Тогда мне здесь точно нечего будет делать.

\ml{$0$}
{--- Я напишу о нас книгу.}
{``I'll write a book about us.''}

Она засмеялась и схватила меня за плечи:

\ml{$0$}
{--- Книгу?}
{``A book?''}

\ml{$0$}
{--- Да.}
{``Yes.}
\ml{$0$}
{И нас будут помнить даже после моей гибели.}
{We'll be remembered even after my death.''}

\ml{$0$}
{--- Кто?}
{``But who will remember?''}

\ml{$0$}
{--- Кто-то да будет.}
{``Somebody will.}
\ml{$0$}
{Написанное может быть прочитано.}
{Written might be read.}
\ml{$0$}
{Может быть, её будут читать миллиарды, жители сотни планет.}
{I guess billions of folk, dwellers of hundred planets will read this story.''}

\ml{$0$}
{--- И мы будем встречаться снова и снова, проживать жизнь за жизнью в чьём-то воображении?}
{``And we will meet and meet again, we will live life by life in the mind's eye of somebody, won't we?''}

\ml{$0$}
{--- Да.}
{``Yes, we will.''}

Чханэ смотрела в закат, осмысливая мои слова.
Таких счастливых глаз я не видел никогда.
Помолчав, она кивнула и притянула мою руку к груди.

\ml{$0$}
{--- Я согласна.}
{``I accept it.''}

\chapter{Воин и менестрель}

\section{Три притока}

\epigraph{У крепкого стула три ноги, у полноводной реки три притока.}
{Пословица сели}

Правилу Трёх сели учили с детства.
Если тебе нужна чистая вода --- у тебя должны быть ручей, река и озеро.
Если тебе нужна пища --- у тебя должны быть поле, лес и рынок.
Если тебе нужна крыша над головой --- у тебя должны быть палатка, жилище и постоялый двор.

\section{Отравленный нож}

--- Меркхалон-кровохлёб, --- выругалась Митхэ и с сожалением взглянула на саблю.
Надпись на клинке скрылась под кровью, оставив на поверхности один иероглиф: <<Оставь меня>>.

--- Понимаю, --- буркнула Митхэ.
--- Ребёнок.
Хай, малыш!
Нет-нет-нет, стой, я хочу просто...
Да чтоб тебя.

Митхэ машинально парировала неловкий выпад девчонки.
И ещё один.
И ещё.

--- Да подожди ты.
Остановись.
Я хочу поговорить.

--- Слева, слева заходи! --- заверещала девочка.

Митхэ чересчур поздно сообразила, что попалась на простую детскую уловку.
Выскользнувший из кустов мальчишка успел оставить на боку Митхэ глубокую царапину --- за миг перед тем, как из его подмышки брызнул фонтан крови.

--- Братик! --- девочка бросила чересчур тяжёлую для её рук фалангу и подхватила умирающего на руки.

Митхэ тупо смотрела на ребятишек.
Вот снова от её руки погиб невинный человек.
В груди медленно нарастала щемящая боль.

Но почему болит ещё и в боку?
И почему мир вокруг завертелся?

--- Митхэ, Митхэ! --- Акхсар бил её по щекам.
--- Что с тобой?

--- Она отравлена, глупое ты бревно! --- рявкнула Эрхэ.
--- Хватит её лупить!
Быстро нагрей кочергу на костре!
Хотя брось, поздно уже, у неё всё в крови...
Нагрей воды!
И травы сюда тащи мои!

--- И чтоб ты сдохла, --- добавила девочка.
Она по-прежнему прижимала к себе окровавленное тело брата.
На её маленьком лице застыла скорбь.

\section{Висок}

Она прижала горячую чашу к виску и закрыла глаза.
Нежный жар действовал на неё успокаивающе;
по спине побежали мурашки облегчения.

\section{Слова потомков}

Так ли уж важно, что скажут современники и потомки?
Жизнь --- она здесь и сейчас, и по большому счёту она принимает тебя таким, каков ты есть.

\section{Борьба}

Борьба --- странная вещь.
Вначале тебе очень больно, и ты избегаешь её как можешь.
Потом жизнь снова вынуждает тебя принять бой, и ты борешься, мечтая о покое.
Как вдруг в один прекрасный день ты понимаешь, что без борьбы жизнь не имеет смысла.
И ты идёшь сквозь метели, стонешь от ран, проклинаешь богов и духов, но иначе жить уже не можешь.

\section{Мимолётное}

Митхэ и Атрис шли рядом и думали об одном и том же --- о том, как мимолётны поцелуи.
Впрочем, Митхэ больше вспоминала о произошедшем, пыталась усилить, запечатлеть в памяти;
Атрис же с головой ушёл в философию.
Казалось, всего михнет назад с людьми происходило что-то невероятное --- стучали сердца, руки и губы плясали в слепом танце, полном запаха волос, дыма, пота, феромонов... и вот любовники идут, сбивая ноги о камни, и между ними снова непреодолимая стена, сложенная из пространства-времени, обязательств и стереотипов.

Порой Атрис и Митхэ обменивались взглядами.

<<Хочу ещё>>, --- молили глаза женщины.

<<У нас достаточно времени впереди>>, --- успокаивал её менестрель.

<<Ты сам-то в это веришь?>>

Атрис не верил.
<<Сейчас>> в его жизни давно одержало победу над <<потом>> --- в тот самый день, когда цитра заменила дом, поле, мастерок и даже самую малость --- пищу.

\section{Гибельная красота}

Воин от природы обладал скоростью, которая недостижима большинству.
Он мог ловить голыми пальцами ружейные дротики и стрелы.
В спарринге из всего отряда чести с ним мог сравниться разве что Ситрис --- несмотря на недостаток скорости, разбойник очень быстро соображал, куда дует ветер, и неплохо предсказывал движения противника.
Ещё Акхсар был очень красив --- тонкие хищные черты лица, маленькие, но яркие зелёные глаза над лезвиями точёных скул, белые зубы, пегая грива, перевязанная нитками и украшенная костяными погремушками.
Подступающая старость, как ни удивительно, делала его ещё красивее.
Женщины и мужчины ходили за ним толпами;
однако Акхсар всегда стоял рядом с Митхэ, не обращая внимания на восторженные взгляды.
Все считали его заносчивым, и лишь отряд чести знал --- воин, прошедший самые известные сражения современности, боялся поклонников как огня.
Митхэ была его щитом.

--- Может, ты всё-таки ответишь кому-нибудь взаимностью? --- тихо спросила как-то Митхэ.
--- Тебе понравится.

--- Их любовь --- любовь хищников и коллекционеров, --- грустно буркнул воин.
--- Это не то, что нужно человеку в моём возрасте.
Так что давай все будут считать, что я безответно влюблён в тебя.

--- А ты в меня влюблён? --- усмехнулась Митхэ, стрельнув глазами.

Впрочем, она тут же об этом пожалела.
Акхсар вдруг сжался, словно испуганный ягуар, и затравленно посмотрел на командира.

--- Снежок, тихо, тихо, я пошутила, --- тут же испуганно пояснила Митхэ, похлопав друга по плечу.
--- Пошутила.

Однажды Митхэ услышала от него совсем грустные слова:

--- Будь прокляты те, кто наделил меня такой привлекательностью.

--- Да что ты такое говоришь! --- всполошилась Эрхэ, пролив на себя похлёбку.

--- Знаешь, в чём отличие между тем, когда тебя грубо лапают, и тем, когда на тебя постоянно пялятся?

--- Ну?

--- По рукам можно надавать.
Больше отличий нет.

--- Ты же сам прихорашиваешься перед зеркалом!
И эти твои ленточки-погремушки!

--- Для себя, Обжорка, не для других!
Это две большие разницы!

\section{Видевший смерть}

Митхэ взглянула в глаза Атриса.
Они были чисты, как небо.
<<Видевший смерть>> --- так называли сели подобный взгляд.
Не ту смерть, что целится в тебя из лука, не ту, с которой можно договориться, не ту, от которой можно уклониться ловким движением ног.
Эти глаза видели неотвратимую, неумолимую смерть, которая ползла ядом по венам, которая разгоралась в лёгких пламенем болотной лихорадки и лишь в последний момент, по какой-то странной ошибке Вселенной отошла в сторону.
Митхэ знала, что у неё тот же взгляд.

\section{Что такое любовь}

--- Я не знаю, что такое любовь, --- призналась Митхэ.

--- Любовь --- это желание, чтобы близкий жил и процветал.

--- Или хотя бы не мучился, --- добавила женщина.

--- Вот видишь, это просто.
Все стихи и песни можно сократить до четырёх слов --- жизнь, процветание и милосердная смерть.

\chapter{Подпольщики}

\section{Консерватизм}

--- Интересно, куда в этом случае делся обычный для военных консерватизм, --- проворчала Анкарьяль.
--- Когда я захотела сменить имя --- с той мерзости, которой меня нарекли на Капитуле, на моё нынешнее, --- легат прямым текстом заявил мне, что это нежелательное действие, так как оно вызывает сомнения в моей лояльности.
<<Как вы будете лояльны Ордену, если вы не можете хранить верность своему собственному имени?>> --- спросил он.
А когда я спросила, с какой стати я должна хранить верность набору звуков, мне пригрозили пенальти.

\section{Двойная верность}

--- Грейсвольда проверили уже несколько раз.
Всё чисто.

--- Но вы всё равно не можете в это поверить, верно? --- подхватил голос.
--- Грейсвольд верен Ордену, и это факт.
И всё же, спроси я вас, служит ли он Скорбящим, что бы вы ответили?

--- Да, --- хором сказали Штрой и Самаолу.

--- Наши протоколы и тесты разрабатывались для совершенно конкретной ситуации --- противостояния Ордена Преисподней и Красного Картеля.
То, что хорошо для одного --- плохо для другого.
Наши тесты просто не допускают ситуации, что возможна деятельность во благо двум и более организациям.

--- Он верен Скорбящим и Ордену \emph{одновременно}? --- спросила Штрой.

--- Вам это кажется удивительным?

Штрой и Самаолу переглянулись.

--- Это было бы крайне сложно обнаружить.
Особенно если одна из организаций законспирирована.

--- На этом коньке Грейсвольд и въехал в ряды контрразведки, --- подхватил голос, --- с нашей вольной и невольной помощью.

--- Есть какие-то версии, каким образом это осуществляется? --- безжизненным голосом осведомился Самаолу.

--- О, Самаолу, у меня есть множество самых диких предположений.
Одно из них, например, заключается в том, что Грейсвольд \emph{сам} уверен, что он работает на Орден.
При этом он бессознательно посещает явки Скорбящих и оставляет информацию для них.

--- Что-то вроде МПС\FM?
\FA{
Маркерные поведенческие стереотипы.
}

--- Именно.
Но не тела, а демона.
Как бы демон, спящий в демоне.

Самаолу потянулся к поясу:

--- Я немедленно прикажу...

--- Ни в коем случае, это бесполезно.
Проверка ничего не даст, вы только зря распугаете рыбу.

--- Какой демон добровольно согласится быть марионеткой? --- ошарашенно спросил Самаолу.

--- Уверенный в своей правоте, --- мрачно сказала Штрой.

Самаолу непонимающе посмотрел на неё.

--- Я читала досье Падальщика, --- пояснила Штрой.
--- Он получил оцифровку, пытаясь помочь соплеменникам.
Ситуация абсолютно идентичная.

--- Именно, --- сказал голос.
--- Теперь вы оба понимаете, кому Грейсвольд верен на самом деле.
И масштаб угрозы, исходящий от этих фанатиков.

--- Фанатики-пацифисты, --- ухмыльнулась Штрой.

--- Это действительно очень забавно, Штрой, --- я таких ещё не встречал.
К счастью, оружие против них крайне простое.
Нужно смоделировать ситуацию, в которой интересы Ордена будут противопоставлены интересам Скорбящих.

\section{Пафос}

--- Манэ, почему ты выбрала такое странное прозвище?

--- Я родилась слишком поздно, --- ухмыльнулась Манэ.
--- Пафосные прозвища все разобрали.

\section{Ярлык}

--- По убеждениям я скорее аристократ, --- ухмыльнулся Самаолу.

--- Остерегайтесь придерживаться <<убеждений>>, Самаолу, --- сказал голос.
--- Если личность навешивает на себя ярлык, то велика вероятность, что вместе с прогрессивными идеями она потянет и весь причитающийся идеологический мусор.
Придерживаться <<убеждений>> --- это всё равно что носить с собой мешок руды вместо кошелька с монетами.

--- Но то же самое и с выбором стороны в войне, --- заметила Штрой.
--- Вы можете ни словом, ни делом не навредить врагам, но вас покарают за ваше знамя.

--- Штрой, вы определённо меня сегодня радуете, --- восхищённо ответил голос.
--- Пожалуй, я буду ходатайствовать о вашем повышении.
И ещё, как мне кажется, текущее дело --- не для вас.
Что скажете?

--- Вам виднее, --- поклонилась Штрой.
--- Но я бы хотела довести его до конца.

\section{Интуиция}

--- Вы не понимаете суть человеческой интуиции.
Это тонкая обработка поступающих в мозг сигналов.
Всё, что находится за пределами чувствительности и экстраполяции чувственных данных, --- не интуиция, а галлюцинации.

--- Прикрывшись этим определением, вы упустили из виду ширину диапазона чувствительности тси и их базовый уровень интеллекта.
У интуиции есть границы, но определить границы интуиции тси я пока затрудняюсь.
Тси учили пользоваться интуицией.

\section{Монограмма Стигмы}

Стигма рассеянно водила пером по бумаге.
Письмо и рисование доставляли ей невероятное удовольствие.
Вот на бумаге появился человечек в причудливой одежде.
Стигма медленно и аккуратно начала заштриховывать открытые участки.

Её рисунки проверяли десятки тысяч раз на наличие кода.
Не нашли.
Трудно найти код там, где его нет.
Она просто очень любит рисовать.

Заштриховав человечка, Стигма аккуратно вывела в углу свою монограмму.
Знаки-подписи были в моде у древних тси, и Стигма в подражание им создала собственный.
Пришлось, конечно, порыться в архивах, но результатом стратег была довольна --- знак достаточно точно отражал склад её личности.
Острый как стилет клин, похожий на щит полукруг, игривый хвостик, который менял длину в зависимости от настроения Стигмы.
Перечеркнула сооружение решительная, неторопливая горизонтальная черта.
Красивый рисунок и никакого кода.

Да, никакого кода.

\section{Проблема 48}

--- Да, и ещё, --- сказал представитель.
--- Мы настаиваем на том, чтобы отдел биологии пересмотрел своё решение по поводу строительства...

--- Кажется, я ещё не высказывался по этому поводу? --- осведомился Атрис.
--- Если доклад Кольбе и Рабе, подтверждённый расчётами Корхес, не является для вас весомой причиной, вот вам мой ответ --- нет.
Пожалуйста, занесите его в протокол под моей цифровой подписью.

Представитель скривился, но тут же, спохватившись, натянул официальное лицо.

\ml{$0-[ej]$}
{--- Как вы знаете, планета Тра-Ренкхаль не обладает необходимой инфраструктурой.}
{``As you know, there's no necessary infrastructure on the planet Tr\r{a}-R\={e}nkch\'{a}l.}
\ml{$0-[ej]$}
{Поэтому...}
{Therefore---''}

\ml{$0-[ej]$}
{---  ... и поэтому её нужно превратить в пустыню? --- закончила Митхэ.}
{``---therefore, it should be turned into desert?'' \Mitchoe\ finished.}
--- Тра-Ренкхаль --- это не стоящая на солнце чаша.
Это котелок над огнём, и этот котелок не закипает лишь благодаря удачному стечению обстоятельств.

\ml{$0-[ej]$}
{--- Боюсь, я не совсем понимаю вашу аналогию.}
{``I'm afraid your analogy was not understood.''}

--- Съездите как-нибудь в Вялую степь по южной дороге, ведущей из святилища Тёплый Двор, --- сказала Митхэ.
--- Вы увидите то, что называют Корзиной Сельвы --- десять кхене тонких, сухих, плотно переплетённых деревьев, которые быстро нарастают под дыханием Ху'тресоааса, а затем так же быстро умирают и рассыпаются в прах под палящим ветром Пустошей.
Затем посмотрите на запад, где виднеются Дикобразовы горы.
Это тот щит, благодаря которому джунгли ещё живы, а реки полноводны.

\ml{$0-[ej]$}
{--- Если Пустоши обогнут Хребет Дикобраза, он перестанет быть защитой, --- сказал Атрис.}
{``If the Deadlands get around the Hedgehog Spine, there will be no shield anymore,'' \Aatris\ said.}
--- Площадь сельвы сократится вдвое только от этого.
На Водоразделе появится ещё одна степь.

--- Я вас уверяю, --- улыбнулся представитель, --- что наши заводы отвечают требованиям экологичности.
Были проведены многочисленные исследования в условиях сотен планет, в том числе и с весьма сложными экологическими условиями.

--- Вы помните последствия <<Проблемы 48>>, --- сказал Атрис.
--- Океан на Капитуле вышел из берегов, затопив тысячи мелких городов.
На Тра-Ренкхале последствия будут куда серьёзнее --- расширится площадь Смертных пустошей, а джунгли продвинутся к Хрустальным землям, уничтожив уникальные реликтовые виды.
Наземные строения не могут быть экологичными по определению.

--- Как вы знаете, на Капитуле эта проблема была решена --- посредством полузеркального щита и высокотехнологичных систем теплоотведения.

--- Я также знаю, что на Капитуле эта проблема \emph{возникла}, и знаю, из-за чего, --- отрезал Атрис.
--- Я против строительства новых городов и заводов на поверхности, равно как и любого другого вмешательства в экосистему планеты.
Старые города останутся в их текущем виде как культурное наследие, в строгом соответствии с указом номер вы-знаете-какой.
Тем не менее, я окажу всё посильное содействие для строительства подземных городов, подобных тем, что строили в экваториальной зоне последние тси.

--- Я вас понял, Атрис, --- сказал представитель.
--- Я доставлю ваше сообщение куда следует.
Позвольте откланяться.

Кольбе и Рабе проводили его взглядами.

--- Мне это не нравится, --- сказал Кольбе.

--- Они всеми правдами и неправдами проталкивают это решение, --- сказал Рабе.

--- Очень смахивает на попытку провести мелиорацию, --- сказал Кольбе.

--- Без санкции Капитула, --- закончил Рабе.

--- Этого не будет, пока я жив, --- сказал Атрис.

\section{Кольбе и Рабе}

--- Война --- участь тех, кто обделён разумом, --- сказал Кольбе.

--- Война --- удел тех, кто не умеет летать, --- сказал Рабе.

--- Мы никогда не примем войну сапиентов, --- сказал Кольбе.

--- Вселенная бесконечна, --- сказал Рабе.

--- Есть много мест для жизни и процветания, --- закончил Кольбе.

--- Для жизни и процветания, --- хором повторили Атрис и Митхэ.

Братья улыбнулись.

\section{Смерть Сиэхено (отрывки)}

--- Ты --- тси, --- сказала Сиэхено.
--- Твои предки были умны и мягкосердечны.
Это есть и в тебе.

--- Я --- сели, --- ответила Чханэ.
--- Я так же умна, но, на твоё несчастье, ещё и умею быть жестокой.

Сиэхено поёжилась.

--- Чего ты хочешь?
Возьми и оставь меня в покое.

--- Нам нужен визор, --- перешла к делу Чханэ.
--- Ты --- лучший визор в Ордене.
Переходи на нашу сторону.

--- Я верна Ордену, --- сказала Сиэхено.
--- А вы хотите его уничтожить.

--- Мы не хотим уничтожать ни Ад, ни Картель.
Мы хотим мира.
Но для этого нам нужно, чтобы с нами считались.

--- Почти то же самое, в других выражениях, говорил один человек с Древней Земли, --- заметила Сиэхено.
--- Он снял кровавую жатву.
Его слова после повторяли многие, и каждый оставлял за собой горы трупов.
Для миротворца в тебе чересчур много обид.
Ты познала милосердие и жестокость, но ещё не знаешь их границ.

\textspace

--- Если вы связываетесь с агентами в Аду, то как? --- промурлыкала Сиэхено.
--- Ах да, домики... песчаные домики... я ведь подозревала, что в них скрывается код.

Сиэхено снова запела.

--- Я за всё детство придумала только один, --- пожаловалась она.
--- Я его совершенствовала, но никогда кардинально не меняла форму.
А тут детишки лепят совершенно разные домики... словно их кто-то этому учит...

--- Неплохо, --- признала Чханэ.
--- Именно поэтому ты нам нужна.

Да, Сиэхено была нужна Скорбящим.
И Чханэ передёргивало от одной мысли о сотрудничестве с этой тварью.

\textspace

Чханэ почувствовала, что её пробирает ледяная дрожь.
Демоны не зря отгораживались от собственных тел.
Сиэхено всеми силами искала уязвимость в обороне интерфектора.
Модули Анкарьяль пока спасали, блокируя опасные эмоциональные реакции, но факт оставался фактом --- это встреча равных, а не охотника и жертвы.

<<Следуй моему плану, --- сказал Аркадиу, --- никакой самодеятельности>>.

Беседа давно уже вышла за рамки плана.
Чханэ поняла --- ей \emph{придётся} убить Сиэхено, если она её не убедит.
Что могло убедить визора?
Рассказать ей всю подноготную легенды, чтобы она осознала, в какой переплёт попала, и тем самым уничтожить эту самую легенду?

<<Лис, лесные духи, почему ты не предупредил, с кем мне придётся иметь дело?!>>

--- Малышка, --- ласково сказала Чханэ, и Сиэхено неподдельно вздрогнула, испугавшись такой неожиданной перемены.

--- Ты всё ещё надеешься надавить на мои чувства, интерфектор?

--- Я не хочу на тебя давить, --- прошептала Чханэ.

--- А что ты сейчас делаешь? --- осведомилась Сиэхено.
--- Тебе придётся меня убить.
Ты проиграла ещё сто десять секунд назад.

--- Я не хочу победы, дитя.
Я хочу мира.
Как и мои предки, я хочу только мира.

--- Опять ты за своё.
Мы всё обсудили...

--- Я никогда не ударю в спину другу, --- сказала Чханэ.

Сработало безотказно, как когда-то это сработало на Лусафейру.
Сиэхено заколебалась, затем её личико снова стало умиротворённым.

--- Вдвоём против целого мира?
Увольте.

--- Это лучше, чем против того же мира одной.

И снова в яблочко.
Сиэхено выронила бусы и судорожно начала искать их на столе.
Чханэ подняла бусы и опустила их в ладонь девушки.

--- Ты всё равно убьёшь меня, --- прошептала Сиэхено.
--- Если я не соглашусь, ты меня убьёшь.

<<Вот она, грань между доверием и недоверием>>.
Если сейчас отпустить Сиэхено, либо она приобретёт союзника, либо часть плана развалится, как карточный домик.

\section{Гениальная игра}

--- Метритхис, --- проворчал голос.
--- В эту игру играет весь Ад.
Кто-то даже говорит, что она гениальна --- она проводит явную параллель между квантовой физикой и сапиентным обществом, между струнами и индивидами.
Я тоже нашёл игру интересной, однако надоело на каждом углу слышать её термины.

\section{Первая любовь}

Двери за Штрой закрылись, и она, сбросив с себя платье, небрежно перешагнула через него.
Следующими на пол упали штаны, затем о холодный пластик клацнул пояс, стилизованный под секхвим народа сели.
С каждым брошенным предметом одежды пропадала сексуальность Штрой, сменяясь измотанностью.
В кабинет входила женщина из плоти и крови;
под душ встала нагая бесплотная тень.

Штрой казалось, что струйки воды проходят сквозь кожу, не трогая её рецепторов и не отдавая ей ни джоуля тепла.
Она вывернула ручку душа на максимум и тут же повернула обратно, зашипев от боли.

Ей вдруг вспомнились давние времена, когда она только начала служить Аду.
Перебежчики обязаны некоторое время отработать в исследовательских отделах, и жаждущую показать себя демоницу отправили в отдел Корхес Соловьиный Язык.
Не прошло и нескольких дней, как новое тело дало сбой --- Штрой по уши влюбилась в свою начальницу.

Корхес всегда отличалась работоспособностью.
Её жизнь была расписана по минутам.
К ней часто приходили демоны из других отделов --- посоветоваться наедине.
Корхес отмеряла советы на квантовых весах, её яркие голубые глаза светились спокойствием и надёжностью, каждое её слово было вовремя и к месту.
Штрой слушала шелестящий, с металлическим оттенком голос, закрыв глаза и прикусив губу.
Она думала о руках Корхес, о губах Корхес, о спящей Корхес, о смеющейся Корхес --- и она не могла думать ни о чём другом.
Демон пытался восстановить контроль над телом, но безуспешно --- эмоции захватили и его, гигантские вычислительные ресурсы начали уходить на примитивные любовные иллюзии.
Эффективность Штрой неумолимо снижалась, риск потерять место с каждым днём возрастал.

--- Корхес, у тебя есть минута?..

\ml{$0-[ej]$}
{--- Так, Штрой, реши для себя раз и навсегда.}
{``Now, Stroji, make your choice once and for all.}
\ml{$0-[ej]$}
{Ты зачем сюда пришла?}
{Why are you here?}
\ml{$0-[ej]$}
{Работать?}
{Work?}
\ml{$0-[ej]$}
{Какие с этим проблемы?}
{You got a problem with that?}

\ml{$0-[ej]$}
{--- Так ты знаешь?..}
{``You knew ...!}

\ml{$0-[ej]$}
{--- Я не слепая!}
{``I'm not blind!''}

Штрой, бывший легат Красного Картеля, с полными слёз глазами топталась на месте.
Она чувствовала себя глупой никчёмной девочкой, никогда прежде её не посещало такое глубокое ощущение беззащитности.
Её губы шевелились, но из них не вылетало ни звука.

--- Всё, хватит, оставь эти мысли, --- строго сказала Корхес.
\ml{$0-[ej]$}
{--- От любви ещё никто не умер.}
{``Nobody's died of love.}
Если шалят гормоны --- поищи развлечений где-нибудь вне лаборатории, а лучше --- обратись к врачам.
Мне пора, увидимся.

<<У этих идеальных демонов, которые вписываются в фигурную дырку общества, всегда виноваты гормоны.
Всегда>>.

--- Пока, --- выдавила Штрой, глядя в спину навсегда уходящей мечты.
Её сердце лежало на холодном полу, пища, как брошенный котёнок.

<<И вот теперь я в игре, --- думала Штрой.
\ml{$0-[ej]$}
{--- Я могу испортить твоё расписание, раздавить твой уверенный голос и пустить твою жизнь под откос.}
{I can rewrite your schedule, crush your confident voice, and derail your life.}
Разумеется, она из Скорбящих.
\ml{$0-[ej]$}
{Дура, милая дура!}
{Fool, you precious fool!}
\ml{$0-[ej]$}
{Прояви ты тогда хоть каплю нежности --- я бы осталась пешкой в твоём отделе, и ничего бы этого не случилось!>>}
{If you could be a little gentle to me, I'd stay a pawn in your sector, and none of that would ever happen!''}

Штрой яростно отвесила себе пощёчину.
Спустя несколько минут она вышла из душа;
её глаза блестели, как и раньше, а губы кривились в прежней усмешке.

<<Ты хочешь, чтобы я работала, Корхес?
Ты это получишь>>.

\asterism

--- Могу я узнать причину, по которой вы избегаете разрабатывать линию Корхес?

--- Я нашла более подходящий путь.
Я сообщу вам детали, как только он будет готов.

--- Штрой, я хочу вам напомнить одну вещь.
Либо вы играете в игру со всем усердием, либо вы в неё не играете.

--- Я займусь Корхес, --- безжизненным тоном предложил Самаолу.

--- Нет, --- рявкнула Штрой.
--- Прошу прощения, владыка, --- добавила она.
--- Я не отказываюсь от линии Корхес.
Я просто прошу отсрочку, чтобы разработать ещё и добавочный путь.

\section{Ещё один секрет игры}

--- Я расскажу вам ещё об одном секрете Метритхис, --- сказал голос.
--- При должном мастерстве игроков в неё можно играть вечно.
Собственно, достигнуть истинного равновесия Нэша.
Несмотря на все старания тси, сели были плохи в математике и этого не знали.
Хотя по Тра-Ренкхалю ходили легенды о партиях длиной в год, десять и даже шестьдесят лет.

\section{Беседа Гало и Тахиро (отрывки)}

Тахиро ухмыльнулся.

--- Сколько у нас осталось времени, Тахиро?

--- До прорыва последнего круга обороны осталось...

--- Двадцать шесть минут, ---подхватил Гало.

--- Двадцать пять и пятьдесят четыре, ---поправил Тахиро.

Собеседники грозно переглянулись и рассмеялись.
Тахиро пододвинул стул поближе.

--- Удивительно, не правда ли --- тысячи наших демонов сейчас гибнут, чтобы у двух обречённых бездельников было двадцать шесть минут на разговоры?

Гало улыбнулся.

--- В тебе всегда была глупая потребность превращать жизнь в спектакль.
Оставь этот драматизм, наш разговор в любом случае ничего не решит.

--- Возможно, --- пожал плечами Тахиро.
--- В таком случае давай насладимся чаем.

--- Хорошо, --- согласился Гало и взял кружку.

\textspace

--- Ты когда-нибудь был женщиной? --- неожиданно спросил Гало.

--- Да, --- ухмыльнулся Тахиро.
--- Мы с Айну как-то ради эксперимента выбрали тела противоположного пола.
Айну заскучала из-за чрезмерной открытости, а я загрустил из-за плохо продуманной системы контроля.
Айну сказала, что я дикая истеричка, а я упрекал её в холодности.

Гало захохотал.

\textspace

--- Нужна самодостаточная система, смыслом существования которой была бы не война, а развитие, --- прошептал Гало.
--- И любой мог бы говорить с любым на любом языке.

\section{Подозрение}

--- Скажи, Грейс.
Кто всё-таки был тот стратег, который отдавал приказы Самаолу и Штрой?
У меня создалось впечатление, что его почерк...

--- Ты догадался, да, --- Грейс улыбнулся и опустил голову.
--- Если увидишь те же черты в почерке иерархов Картеля --- дай знать, мы тебя устраним.

--- Но зачем ему?..

--- Он был создан, чтобы править, --- сказал Грейс.
--- И он уже правит и двигает Вселенную к миру и процветанию.
Войну нельзя прекратить в одночасье.
И никакое изменение нельзя совершить, пока общество не будет к нему готово.

--- То есть сейчас, в эту самую минуту он сражается сам против себя?

--- Он --- лучший из стратегов.
И тот новый мир, который мы пытаемся построить, должен иметь защиту от таких, как он.
В этом и смысл сражения с самим собой.
Он следует извращённым философским концепциям, строит жестокие интриги --- одним словом, проверяет зародыш новой системы на прочность.

--- И я буду вынужден...

--- До последнего, как против злейшего из врагов, --- кивнул Грейс.
--- Он поддаваться не будет.
И не терзай душу мыслями о тяжести его доли.
Поверь, \emph{эта} игра доставляет ему истинное наслаждение.
Он на своём месте.

\section{Кандалы}

У дуба примостилась хрупкая фигурка.
Она сидела неподвижно, глядя в небо.
На её руке был массивный браслет, напоминающий кандалы.

--- Штрой так и не хочет с нами сотрудничать? --- спросил Грейс.

--- Неа, --- помотал головой Лу.
--- По-прежнему молчит.
Хотя бы есть начала, и то ладно.

--- Она совсем не ела?

--- Первое её тело так и умерло от голода.
Недавно начала лопать из рук Митхэ по горсточке.
Стигму уже не пытается убить при встрече.
Прогресс налицо.
Хотя я думаю, что она просто замыслила побег.

--- Моё устройство не даст ей сбежать.
Начинай доверять моей технике.
Да и бежать ей некуда --- согласно официальному заявлению, она казнена.

--- Я бы не хотел, чтобы она сотрудничала по принуждению.

\section{Помолодел}

--- Грейсвольд примкнул к диверсионной группе <<Цемент>> не просто так.
Это произошло именно после смерти Айну.
С Айну у него были разногласия, но кому она была верна всю жизнь --- уточнять не нужно.

--- И этому кому-то позарез нужна была функциональная диверсионная группа, --- кивнул Самаолу.
--- Настолько позарез, что вначале он запихнул в нее Анкарьяль, знатно запятнавшая свою репутацию в группе Хэм Золотой Посох, а затем резко получившая ранг по секретному распоряжению отдела 109.

--- Отдел 109 связан с Тси-Ди, --- припомнила Штрой.
--- Что же произошло, что Анкарьяль по их секретному...
А.

--- <<А>>, --- повторил Самаолу.
--- Давно уже известно, что падение Тси-Ди было диверсией, и ковен Лусафейру имеет к нему непосредственное отношение.

--- Самаолу, если я не ошибаюсь, ваша пропаганда утверждает иное, --- хихикнула Штрой.

--- Наша пропаганда утверждает то, что в интересах Ордена, --- спокойно парировал Самаолу.
--- Но очевидное отрицать нельзя.

--- Значит, вначале скомпрометировавшая себя Анкарьяль, --- начала загибать пальцы Штрой, --- потом в группу попал столь же сомнительный Люпино.
И в конце, когда Айну при неясных обстоятельствах погибла, в диверсионный отряд попал близкий друг Лусафейру.

--- И в результате мы получили диверсионный отряд, который без поддержки разбил гарнизон Тра-Ренкхаля, --- заключил голос.

--- Им повезло.

--- Ни разу не везение, --- возразил голос.
--- Вначале <<Цемент>> знатно наломал дров, но затем они исправили все свои оплошности.
В активной стадии операции отряд действовал просто филигранно.
Гораздо эффективнее, чем каждый из низ в отдельности.
Даже Грейсвольд, чего от него никто не ожидал.

--- Грейсвольд как будто помолодел, как оказался в <<Цементе>>, --- сообщил Самаолу.

--- О да, --- согласился голос.
--- Это все отметили.
Похоже, его очень вдохновляют два молодых демона рядом, полных мечтаний, страсти и желания изменить мир.

--- Я думаю, его и засунули к ним, чтобы в узде держать, --- сказала Штрой.

--- Это могла делать и Грихаро, она достаточно стара и выдержана, --- возразил Самаолу.
--- Нет, Грейсвольд нужен, не для сдерживания. 
Он идеолог, подстрекатель.
Поэтому Грихаро и ушла на покой --- она увидела, что Грейсвольд обрабатывает молодежь, и ей это не понравилось.
Но, так как Анкарьяль и Люпино тянулись к Грейсвольду, она последовала старому правилу диверсантов: семья превыше всего.
Если ты выпадаешь из группы --- уходи.

--- Даже если ты стоял у ее истоков, --- добавил голос.

--- Даже если.
Для Грихаро это был, конечно, удар.
Она вообще из полей ушла в административную работу.
Насколько я слышал, в отделе от неё одни проблемы, но старушку держат, как украшение.

--- Или она перешла на другие поля, --- ввернула Штрой.

--- Что вы имеете в виду? --- недовольно осведомился Самаолу.

--- У меня есть данные, что Грихаро Артишок сделали предложение, от которого невозможно отказаться.
Работа.
Опасная.
Цена ошибки --- Чистилище.
И если данные верны --- её неуклюжая административная служба не более чем прикрытие.

--- Ничего себе, --- удивился голос.
--- Штрой, это очень интересно.
Я обязательно проверю эти данные.
Вы не расследовали глубоко?

--- Нет, к сожалению.
Она по всем признакам далеко от нашей сферы интересов.

--- Насколько далеко?

--- Насколько возможно, владыка.
После ухода из <<Цемента>> она оборвала все контакты с группой и дистанцировалась от Лусафейру.
Есть основания полагать, что её новая работа никоим образом с ним не связана.
Она не защищает сферу его интересов и не вмешивается в неё.

--- Скорее всего, поэтому это и прошло мимо меня, --- сказал голос.
--- Но расследовать следует, для общего развития.
Не вам, Штрой, не вам.
Я поручу это другим.
Но спасибо за информацию.

--- Кстати, а кто устранил Айну? --- поинтересовалась Штрой.

--- Интересный вопрос, --- хмыкнул голос.

--- Кто бы это ни был, ему позарез хотелось выбить из-под Лусафейру кресло, --- предположил Самаолу.
--- Лу не стал бы жертвовать лучшим другом, если бы ставки не были высоки.

--- А что Айну связывало с Лусафейру?

--- Старая история.
Лу, Тахиро и Айну был любовниками с незапамятных времён.
И несмотря на то что их пути многократно расходились, они всё равно почему-то сходились обратно.

--- Отголоски первой жизни в демонах очень сильны, --- сказал голос.
--- В первую жизнь формируется ядро сапиентной личности.
Эту жизнь демон потом тащит за собой тысячелетиями --- в том числе её привязанности и травмы.

\chapter{Сага о Тигре}


\section{Первая кровь}

У Тахиро задрожали руки.
Врагов было слишком много, как минимум у половины были винтовки.
Похоже, первая смерть наступит чуть раньше, чем он ожидал.

<<Первая смерть...
Демон!>>

Тахиро попытался сосредоточиться.

\ml{$0$}
{<<Боевые модули, выбрать приоритетную цель, взять на прицел жизненно важные органы>>.}
{``\textit{Combat modules, choose a priority target, aim a vital organ.}''}

Руки вдруг перестали дрожать.
Винтовку повело в сторону и чуть выше.
Демон подсветил четырнадцать фигур, в том числе три, скрытых в темноте.
На одной из них появилась красная точка.

<<То есть если я сейчас скажу их всех перестрелять, демон это сделает?!>>

Над нападающими высветилась надпись:
<<Экспресс-тест боевых модулей завершён.
Все параметры в пределах нормы.
Готовность к атаке --- красный уровень>>.

Вдруг Тахиро бросило в сторону.
Он едва успел осознать, что кто-то из врагов начал нажимать на курок.
Выстрел, выстрел, ещё выстрел.
Над начавшими падать фигурами врагов высвечивалось: <<Обезврежен, обезврежен>>...
Пока затвор ещё продолжал свой путь, рука Тахиро уже нащупала следующий рожок.
Щёлк, щёлк --- перезаряжен...

...Очнулся он чуть позже, обшаривая карманы трупов.
Демон добросовестно сканировал лица нападавших, их экипировку, выдавая автоматический отчёт.
Экипировка была сделана в Такэсако, материалы привезены контрабандой с побережья.
Трое из нападавших значились в розыске, у ещё четверых в розыске были близкие родственники.
Сильно болела надорванная в прыжке мышца --- приложенные демоном усилия явно превышали возможности тела.

<<Сгореть мне дотла, --- ошеломлённо думал Тахиро, глядя на убитых.
--- Тринадцать патронов, двое убиты одной пулей.
Что за чудовище из меня сделали?..>>

\asterism

--- Держи, --- Стигма поставила перед ним чашку с ароматным зелёным чаем и тарелку со сладостями.
--- Я не мастер чайной церемонии, но готовить сам чай и эти вкусные штуки научилась.
Ни в чём себе не отказывай.

--- Благодарю, --- Тахиро церемонно склонился.

Стигма, подумав, вытащила из угла что-то странное, непонятного цвета и сунула Тахиро в руки.

--- Что это? --- удивился он.

--- Обними.
Это мягкая игрушка в виде совы.

--- Совы?

--- Да.
Никогда не видел сову?

Тахиро вежливо пожал плечами в неком гибриде жестов <<да>> и <<нет>> и промолчал.
На сову мягкая игрушка не была похожа совсем, но обижать хозяйку не хотелось.
Впрочем, сова оказалась на ощупь довольно приятной, и Тахиро последовал совету Стигмы.

\textspace

--- Почему ты это мне рассказываешь?
У меня создалось впечатление, что ты не доверяешь никому.
Даже Люциферу.

--- А ему можно доверять?

--- Я бы доверил ему свою жизнь, --- горячо сказал Тахиро.

--- То, что ты с ним спишь и дружишь, не повод доверять ему жизнь.
Доверить можно что-то, что переходит из рук в руки --- деньги, вещи, тайны.
Жизнь --- она только твоя, никто не может забрать её у тебя, не разрушив.
Жизнь не может быть общей, а значит, её никому нельзя доверять.

--- Ты не ответила на мой вопрос.
Почему ты мне это рассказываешь?

--- Я помогаю стать сильнее потенциальному союзнику или потенциальному врагу.
И то и другое в моих интересах.
Сильный союзник помогает, сильный враг держит в тонусе.

<<Часть философии Антрацис, --- понял Тахиро.
--- Эти ублюдки ещё более помешаны на войне, чем мои родичи>>.

--- Я гораздо более миролюбива, чем мой клан, --- сказала Стигма.
Тахиро вздрогнул.

--- Как ты это делаешь?

--- По-твоему, это невозможно?

--- Ты прочитала мои мысли, и...

--- Тахиро, остановись и подумай.
\ml{$0$}
{Твоё мифологическое сознание постоянно говорит тебе, что возможно всё, ничего невозможного нет.}
{Your superstitious thinking always says that all is possible, and nothin' is impossible.}
\ml{$0$}
{Это не так.}
{It's not.}
\ml{$0$}
{Звёзды не загораются на пустом месте.}
{Stars never appear in a vacuum.}
\ml{$0$}
{Дожди из лягушек не идут просто так.}
{Frog rains never fall without a reason.}
\ml{$0$}
{Все события связаны причинно-следственными связями.}
{All events are linked in cause and effect relationships.}
\ml{$0$}
{Никто не умеет вот так сразу влезать в чужую голову, даже демон.}
{And no one can get into your mind like that, not even a daemon.''}

--- Значит, это была иллюзия?

--- Теплее.
Просчитай все возможные варианты мыслей, которые возникли от моей последней фразы.
\ml{$0$}
{Используй демона.}
{Use your daemon.''}

Тахиро открыл и закрыл глаза.

\ml{$0$}
{--- Я бы не подумал ни о чём другом.}
{``I wouldn't think of anything else.''}

\ml{$0$}
{--- Чему ты тогда удивляешься?}
{``Why you're surprised, then?''}

Тахиро откинулся в кресле и крепче обнял сову.

--- Если разговор, который мы ведём, не просто беседа, а имеет какое-то направление, то какова его цель?

\ml{$0$}
{--- Если я скажу, ты мне поверишь?}
{``If I tell you, would you believe?''}

\ml{$0$}
{--- Да...}
{``Yes ...''}

\ml{$0$}
{--- И будешь идиотом.}
{``You'd be an idiot as well.''}

Тахиро отложил сову.
Демон снова вспыхнул, выдав отчёт.
\ml{$0$}
{<<Стигма точно знала, что игрушка не похожа на сову.}
{\textit{Stigma knew for sure that the toy is not looking like an owl.}}
Она изучала мои реакции на внутренний конфликт и на их основе строила дальнейшую беседу.
Интересно, как бы повернулось наше общение, начни я спорить?..>>
Ещё вспышка.
Три возможных варианта развития событий.

Стигма проводила сову взглядом и едва заметно улыбнулась.

--- Я хочу уйти от философии и поговорить о произошедшем, --- твёрдо сказал Тахиро.
--- Мне нужно твоё видение ситуации.
Не стратегические расчёты, а твои эмоциональные реакции как человека, твой совет.
Я напуган и растерян.

--- Вот сейчас это на что-то похоже.
Я к твоим услугам, Тахиро.

--- Я не понимаю, как контролировать демона.

--- Ты --- вернее, то, что ты сейчас считаешь собой --- не можешь контролировать демона.
Демон --- это твоё сердце, глубинные структуры твоего мозга.
У него свои законы, и его интеллектуальный потенциал значительно превышает твой.
\ml{$0$}
{Всё, что ты можешь --- это усилить интеграцию постоянным взаимодействием, пока вы не будете действовать как единое целое.}
{The only thing you can do is to strengthen the integration by permanent interaction, until you both start to perform as one.''}
Демон тебя покалечил?

--- Просто надрыв мышцы, ничего серьёзного.

--- Боевые модули надо постоянно калибровать относительно тела.
Тебе придется вспомнить, где в лагере находится тренировочная площадка и как в скалах проходит полоса препятствий.
Иначе порванной мышцей дело не ограничится --- мне известны случаи вывихов и даже переломов.

--- Я боюсь, что он меня поглотит.

--- Так же, как твои глубинные чувства, мысли, воспоминания --- всё то, что лежит за пределами сознания?
\ml{$0$}
{Их ты тоже боишься?}
{Are you afraid of 'em too?''}

--- Это другое.

--- Определённое сходство есть.
Если ты научился взаимодействовать со своими эмоциями, руками, ногами --- осилишь и демона.

--- Он так много умеет...

--- И эти умения тебя страшат.
Но что, если они твои?

--- Они не могут быть моими!
Он видит в темноте!
Он предсказывает движения на той стадии, когда человек только начинает напрягать мускул!
Он действует быстрее, чем я могу осознать!

--- Тахиро с демоном --- гораздо большее, чем Тахиро без демона.
То же самое с руками и ногами.

--- Эти люди, там, в Такэсако...

--- Которые на тебя напали?

--- У них не было никаких шансов, --- прошептал Тахиро.

--- Не было, --- согласилась Стигма.
--- Это немодифицированные Homo homo sapiens, дикий тип.
Десятикратное превосходство в численности ничего бы не решило.

--- Я думал, что сопротивление...

--- Ты думал, что сопротивление имеет какой-то смысл? --- Стигма хмыкнула.
--- Сразу видно, что ты вырос в обществе дзайку-мару.
В ваших деревнях есть много легенд о героях, которые смогли пересилить, перехитрить, победить сверхъестественное существо.
\ml{$0$}
{Всё это прекраснодушная чушь, часть мифологического сознания.}
{No more than a starry-eyed bullshit, a part of superstitious thinkin'.}
\ml{$0$}
{Ты просто не можешь победить того, чей интеллект превышает твой на два-три порядка.}
{You just can't defeat a person whose intelligence is two or three orders of magnitude higher than yours.''}

\ml{$0$}
{--- Люди решали исход сражений!}
{``Humans decide outcomes of battles!''}

\ml{$0$}
{--- Исход сражений между демонами.}
{``Battles between daemons.}
\ml{$0$}
{С одной стороны демоны и с другой стороны демоны --- тогда люди что-то решают, кидая песчинки своих жизней на гигантские весы.}
{Daemons on one side, daemons on the other---then and only then humans contribute, throwin' grains of their lives on the giant scales.}
\ml{$0$}
{Но организация людей против организации демонов... --- Стигма грустно усмехнулась, --- без шансов.}
{But a human organization against a daemon organization---'' Stigma sadly laughed, ``---stands no chance.''}

Тахиро сел и закрыл лицо ладонями.
Стигма пододвинула ему чашку с чаем.

\ml{$0$}
{--- Тахиро, возможно, я лезу не в своё дело, но то, что ты сейчас испытываешь, полезно для тебя.}
{``Tahiro, maybe that's not my business, but what you're experiencin' is pretty good for you.''}

\ml{$0$}
{--- Бессилие?}
{``Powerlessness?''}

\ml{$0$}
{--- Именно.}
{``Exactly.}
\ml{$0$}
{Бессилие показывает нам границы наших сил.}
{Powerlessness shows us limits of our power.}
Благодаря бессилию ты начинаешь ставить более реальные цели, грамотно распределять свои ресурсы.

--- Тебе это тоже знакомо, --- вспомнил Тахиро.
--- Орден Тысячи Башен...

--- Да.
\ml{$0$}
{И вот я здесь --- гораздо умнее, мудрее и опаснее, чем была раньше.}
{And here I am---more clever, wise, and dangerous than before.}
У меня в распоряжении ресурсы, которых не было в прошлом, и стратеги, в тандеме с которыми я могу достичь того, чего никогда бы не добилась в одиночку.
Стоило ли это нескольких лет безнадёжности и животного ужаса, пока Союз и Преисподняя медленно раздавливали Тысячу Башен в тисках?
Стоило.

Стигма поставила чашку на стол, и Тахиро понял: пора уходить.
Он вежливо доел остатки мидзуёкан, залпом с характерным шумом осушил свою чашку и поднялся на ноги.

--- Я пока не прожил своё бессилие, --- сказал он, --- но я точно знаю, что мне следует делать.

--- Вот и славно, --- улыбнулась Стигма.
--- Хорошего вечера, Тахиро.

<<Она не спросила, --- думал Тахиро по пути к себе.
--- Она плохо поддаётся на манипуляции в разговоре, её почти невозможно заставить свернуть на определённую тему, при этом сама она вертит разговором как хочет.
Истинный военный дипломат>>.

Тахиро усмехнулся.

<<Кажется, я теперь понимаю, что нашёл в ней Люцифер.
Точность, догматичность --- всё, чего не хватает ему самому.
Он строит хаос на порядке, она --- наоборот.
От общения со Стигмой остаётся странное послевкусие, словно лижешь солёный камень --- она проявляет участие, даёт советы, но никогда не раскрывает свои тайны.
И всё же благодаря ей я кое-что понял --- мне пора вылезать из-под крылышка Люцифера.
Я готов бороться за своё место под солнцем>>.

\asterism

--- И с чего ты хочешь начать?

--- Отправь меня туда, где я нужен, --- сказал Тахиро.

Люцифер приложил пальцы к виску.
Тахиро ни разу не замечал этого жеста раньше и сразу понял, что он значит.

--- Ты сейчас ведёшь себя не так, как должен главный стратег по обороне.

Люцифер тут же убрал пальцы с виска.

--- Ты прав, --- кивнул он.
--- В таком случае, легионер, твоя следующая цель --- Нимб-3.
Ты в распоряжении легата секунда Гатса Убийцы Змей из клана Хаджаар.
Инструкции ты получишь по прибытии.

Глаза Тахиро чуть заметно расширились.
Он явно не ожидал, что Лу сразу прикажет ему покинуть тело, отправиться на другую планету --- никого из оцифрованных ещё не отправляли так далеко.
Но длилось это замешательство едва ли секунду.

--- Сайгон, --- Тахиро поклонился и вышел.

<<Что ж, этот день настал, --- с грустью думал Люцифер, наблюдая за уходящим другом из окна.
--- Конец моей беззаботной жизни.
Грисвольда тоже придётся куда-нибудь отправить --- его таланты нужны Ордену.
Стигма втянулась в свою работу, и приходить она будет всё реже --- я уже вижу тенденцию.
Ну и всё --- я снова один, как в давно забытом детстве.
Я ужасно не хотел взрослеть, и все друзья сделали это раньше меня.
Придётся соответствовать>>.

Лу вдруг почувствовал знакомое жжение в глазах.

<<Прости, любовь моя.
Я знаю, что ты больше не хочешь быть первым.
Но из-за меня на тебя всю жизнь сваливалось то, что просто убило бы девятьсот девяносто девять других людей.
Так будет и впредь, если ты по-прежнему будешь считать меня другом --- то, что угрожает мне, не может не угрожать моим близким.
И ты должен быть к этому готов>>.

\asterism

--- Надеюсь, мы ещё встретимся.

--- Надеюсь, что нет.

--- Тахиро, ты мне нужен.

--- Я тебе резко понадобился сейчас?
Или я был нужен тебе ещё до того, как стал демоном?

Айну замерла, опустив гордую голову и согнув прямую, не привыкшую к такому шею.

--- Я никогда тебя не прощу за то, как ты со мной обращалась, --- тихо сказал Тахиро.
--- Ты угрожала мне пытками, ты унижала моё достоинство, ты обращалась со мной как с домашним животным.
Ты убила Мико.
Ты убила человека.

--- Я убивала тысячи людей.

--- Можешь отговариваться чем угодно.
Есть действие, и есть результат.
А ты, насколько я помню, всегда ставила результат превыше мотивов.
Этому у тебя научился и я.

Тахиро повернулся, чтобы уйти.

--- Я не разрешала тебе идти, легионер.

--- Останови меня, легат, --- парировал Тахиро, не оборачиваясь.

\section{Версия ядра}

--- Что тебя связывает с Двойняшками?

--- Да ничего, кроме версии ядра, --- буркнула Стигма.
--- Чёрная Скала ревностно блюдёт родственные узы.
Честь семьи, защита семьи, отсутствие насилия внутри семьи --- этим они забивали головы нам всем.
Но всё это лишь слова.
Меня шпыняли все, кто только мог, а ударить меня до крови было обычной забавой за ужином.

--- Я всегда думал, что минус-демоны получают от этого удовольствие.

--- Дело было не в боли.
Это было неприятие, травля, пренебрежение.
Даже для минус-демона это болезненно.

--- Ты поэтому сбежала?

--- Я сбежала бы раньше, но там был мой младший брат, из более новой серии.
Он единственный относился ко мне хорошо.
Его считали бракованным, потому что он вёл себя как человеческий ребёнок, ничего не принимал всерьёз.
Его тоже шпыняли, но, в отличие от меня, он мог дать сдачи --- его собрали как интерфектора, плюс он постоянно обкалывал себя какой-то гормональной дрянью, из-за чего у него мышцы вырастали до огромных размеров и он был похож на камушек.
И однажды, вступившись за меня, он ударил Фуси --- причём сильно, сломав тому нос, челюсть и выбив с десяток зубов.
В ту же ночь братишка исчез.
Думаю, Двойняшки его устранили.
Издевательств стало намного больше, и я сбежала через год, как только смогла подготовить побег.

Стигма крепче прижалась к Люциферу.

--- Погладь меня по голове, пожалуйста.

Люцифер запустил пальцы в короткие волнистые волосы.

--- Иногда мне кажется, что он где-то рядом.
Иногда я даже слышу его смех в вечерней тишине.

--- Чей?

--- Брата.

--- Ты была привязана к нему?

--- Нет.
Мы даже разговаривали всего раз или два.

--- Кстати, насчёт смеха, --- задумался Лу.
--- Ты же знаешь, после заключения мира к нам присоединилось много пришлых демонов.
Есть тут один легионер, который постоянно смеётся.
Не так, как это делают остальные, согласно обновлениям, а как-то особенно --- гы-гы-гы.

Стигма пристально взглянула на Лу.

--- МПС?

--- Возможно, МПС.
Возможно, ошибка в коде обновления.
Самое интересное, что смеётся он и в карауле.
Сам что-то рассказывает и сам смеётся в одиночку, порой невпопад.
Недавно он пошутил про птицу, и даже я не понял соль юмора...
Вот слушай.
В ясный день по небу летит орёл, навстречу ему --- другой орёл.
Эти орлы столкнулись в воздухе.
Каков был шанс, что это произойдёт?

--- Единица.

--- Почему?

\ml{$0$}
{--- Потому что птицы не пули.}
{``'Cuz birds are not bullets.}
\ml{$0$}
{Они сами решают, куда лететь.}
{They can choose their direction.''}

Люцифер приподнялся на локтях и удивлённо взглянул на Стигму.

--- Ты знаешь эту шутку?

--- Это не шутка, это девиз людей Ниао --- последних, кто попытался дать отпор демонам Чёрной Скалы.
Мой клан вырезал их поголовно.

--- Он пытается найти своих, --- догадался Люцифер.
--- Вечно Гонимых, как и он.

Стигма вдруг улыбнулась и смахнула набежавшие слёзы.

--- Где сейчас этот легионер?

--- Я точно не знаю, но завтра будет его очередь идти в караул.
По его гоготу можно часы настраивать.
Это будет большим отступлением от расписания, если ты придёшь ко мне в гости не через месяц, а завтра?

--- Я не приду к тебе завтра.

--- Не хочешь увидеть брата?

--- На самом деле нет, --- Стигма мягко высвободилась из объятий Люцифера и встала.
--- Просто была рада узнать, что он жив.

--- Ну что ж, тогда до встречи через месяц.
Приходи, погладим друг друга, как обычно.

--- Как обычно.

--- До свидания, Стигма.

--- Прощай, Люцифер.

\section{Охотник}

Ду-Си подошёл почти вплотную и наклонился, рассматривая стратега со всех сторон.
На Лу явственно пахнуло перекисшим потом, порохом, машинным маслом, крепким спиртным, запрет на употребление которого Ду-Си явно пропустил мимо ушей.
У самых глаз Люцифера тускло сверкнула грубо свитая, в два пальца толщиной гривна из <<чёрного>> сурроганиума.

--- Я слушаю, стратег, --- наконец отозвался Ду-Си.
Принятые в Ордене обращения он, разумеется, тоже пропустил.

--- Я хочу произвести тебя в легаты, --- сказал Лу.

--- Неплохое повышение, --- осклабился Ду-Си.
--- Только чем я заслужил звание легата?
Дай угадаю.
Ты в курсе, что я из Чёрной Скалы.
Наверное, от этой предательницы Цзаошан.
Я слышал, что она теперь важная шишка в Ордене.

--- <<Предательницы>>?
Она такая же Вечно Гонимая, как и ты!

--- Извини, привычка.
Так что тебе от меня нужно, я не понял.

--- Мне нужен твой опыт.

--- Я ещё не решил, достоин ли ты моего опыта, --- нагло заявил Ду-Си.

Люцифер заулыбался.
Легионер с необычным чувством юмора всё больше завоёвывал его расположение.

--- Ты знаешь мою репутацию.

--- Я вижу лишь хрупкого демона, заключённого в тщедушное тело.
Ты выглядишь гораздо менее грозно, чем твоя репутация, Люцифер.

--- Про тебя могу сказать обратное, Ду-Си.

Ду-Си загоготал.

--- Да, насчёт репутации я как-то не подумал.
Просто обычно если я за кого-то берусь, не остаётся никого, кто мог бы об этом рассказать.

--- Как я могу доказать, что я достоин?

--- Победи меня в бою.

--- Я не интерфектор.

--- Я не намерен устраивать с тобой дуэль.
Давай просто подерёмся.
Здесь нет никого, кто нам бы помешал.

Люцифер окинул взглядом коридор --- здесь действительно никого не было.
Он был с Ду-Си один на один.
Стратег взглянул в презрительные, немного безумные чёрные глаза легионера, отметил толщину рук и крепость жил, и под ложечкой внезапно явственно уколол страх.
Он оказался один на один с недавно принятым в Орден новобранцем, умелым интерфектором одного из самых жестоких кланов, большая часть которого служила Союзу Воронёной Стали.
Демон Ду-Си <<светился>> тихо и ровно, словно перед атакой.

<<Я пережил войну с Союзом, но меня убьют в коридоре лагеря?
Да какого дьявола?>>

Люцифер со вздохом посмотрел на свои тонкие холёные руки.
Затем со всего размаху отвесил Ду-Си пощёчину.

Ду-Си даже не качнулся.
Он несколько секунд напряжённо смотрел вдаль, словно пытался понять, куда села муха.
Люцифер внутренне сжался, представляя, как огромные руки хватают его, как хрустят кости, как его демон навсегда прекращает существование, сражённый молниеносным омега-ударом...

--- Нуу, отдаю тебе должное --- ты не трус, --- буркнул Ду-Си и с противным скрипом почесал щёку.
--- Большинство на моё предложение подраться отвечают отказом --- если вообще хватает смелости ответить.
Я слышу, как ты дрожишь, я языком чувствую запах страха, который от тебя исходит.
И ты всё равно атаковал, зная, что проиграешь.
Вы с Цзаошан чем-то похожи.
Не зря ты её любишь.

--- Она --- мой соратник, --- возразил Люцифер, чувствуя, как от пережитого напряжения дрожат мышцы пресса.

--- Ага, и в твоём кабинете вы занимаетесь стратегией, наедине.
У сестры странный вкус, надо признать --- не думал, что её вообще заинтересует кто-либо в этой жизни, но, видимо, она просто искала бабу с членом.
Ладно, Люцифер, будь по-твоему.
Хочешь, чтобы я был легатом --- буду легатом.
\ml{$0$}
{Что я должен делать?}
{What I am to do?''}

\ml{$0$}
{--- Что ты хочешь делать?}
{``What do you want to do?''}

\ml{$0$}
{--- Ты спрашиваешь у меня?}
{``You ask me?}
Не надо играть в благородство.
Мы все знаем, как устроен мир.
\ml{$0$}
{Ты приказываешь --- я выполняю.}
{You give the order, and I obey.''}

--- Хороший руководитель перед приказом собирает об исполнителе информацию --- что тот умеет делать, каков его опыт.
Я решил пойти чуть дальше --- я интересуюсь ещё и тем, что исполнитель хочет делать, о чём он мечтает, каковы его жизненные цели.
Согласно Яо, мы находимся на той стадии технологического прогресса, при которой количество возможных социальных ниш значительно превосходит численность населения.
Другими словами, каждый демон может стать не просто винтиком в системе, а узлом, который сложно или даже невозможно полноценно заменить.
Наибольшую эффективность имеет работа не просто опытного, но и заинтересованного демона.
Моя работа --- обеспечить тебя работой, в которой ты заинтересован, помочь тебе найти подходящую социальную нишу, а при необходимости --- и помочь её сменить.

--- Верный служащий, --- хмыкнул Ду-Си.
--- Ещё и Мэг Яо цитируешь.
Хорошая была учёная, жаль только, что как философ дерьмо.
А скажи, Люцифер, хотелось тебе когда-нибудь получить всё?

--- Хотелось, --- честно ответил Лу.
--- Соблазн получить большую власть, чем это положено при твоих способностях, очень велик.
Меня остановили две вещи --- интеллект и любовь к себе.

--- Объясни.

--- Если я получу больше власти, чем следует --- я отберу эту власть у других, нажив себе врагов.
Если моё видение ситуации окажется неверным и действия мои --- неправильными, я испорчу жизнь ещё большему числу демонов и наживу ещё больше врагов.
Когда против меня восстанут --- мне придётся применить силу и получить ещё больше власти.
Каждоё моё действие, вне зависимости от его успешности и правильности, будет рождать смертельных врагов.
Диктатура --- это ловушка, из которой нет выхода.
Интеллект нужен, чтобы её увидеть, любовь к себе --- чтобы добровольно в неё не залезть.

--- Хочешь сказать, что тираны всегда глупы и не любят себя самих?

--- Тираны --- это те, кто живут в сегодняшнем дне, не заботясь о завтрашнем.
Они могут понимать, что рано или поздно против них восстанут, могут не понимать.
Они могут понимать, что рано или поздно опасность будет исходить от любой личности из их окружения, могут не понимать.
Чаще всего это их не волнует.

--- А ты, значит, не живёшь в сегодняшнем дне.

--- Я --- стратег.
Я по определению не могу жить только в сегодняшнем дне.

--- Ты можешь быть умнее других тиранов и не повторять их ошибки.

--- Так думали все тираны перед тем, как навсегда сгинуть в пучине истории.
Так чего ты хочешь, Ду-Си?
Чему ты хочешь посвятить сегодняшний и завтрашний день?

--- Уничтожению, --- ухмыльнулся Ду-Си.
\ml{$0$}
{--- Уничтожать демонов, дзайку-мару, иных тварей.}
{``Destruction of daemons, dzaiku-maru, other beasts.}
\ml{$0$}
{Чем сильнее и многочисленнее твари --- тем лучше.}
{The stronger they are, the bigger their numbers, the better to me.''}

\ml{$0$}
{--- У меня создалось впечатление, что ты не против и защищать.}
{``I got the impression that you're good at protection too.''}

\ml{$0$}
{--- Если защита связана с уничтожением демонов, дзайку-мару и иных тварей --- я не против.}
{``If protection has anything to do with destruction of daemons, dzaiku-maru, or other beasts---I'm in.''}

<<Он мне не доверяет, --- понял Лу.
--- Он не такой, как прочие демоны Чёрной Скалы, не такой, как Двойняшки, он весьма декларативно следует их философии.
Но при старательно пытается произвести именно такое впечатление.
И, как и Стигма, он старательно держит дистанцию с прочими Вечно Гонимыми.
Видимо, у них обоих есть причины, и это точно не боязнь за свою шкуру>>.

--- Разумеется, я тебе не доверяю, --- подтвердил Ду-Си.
Лу вздрогнул.
--- И у меня есть причины.

--- Если ты о моём брате --- демона делает не ядро, а опыт.

--- И ядро тоже.
Даже взять твою любимую Цзаошан.
Она другая, это верно, но копни её поглубже --- и ты увидишь смотрящих на тебя Двойняшек.
Малышка Цзаошан может казаться разнеженной и слабой, но она настолько же целеустремлённая и, в отличие от тебя, не привыкла ограничивать себя в выборе средств.
\ml{$0$}
{Спроси, что положит конец клану Антрацис --- и я назову тебе имя.}
{Ask me what shall be the end of Anthracis, and I tell you the name.''}

\ml{$0$}
{--- Что ж, очень кстати, что у меня есть соратники, спроектированные для уничтожения.}
{``Well, it's very useful to have such destructors by design as companions.}
В пределах досягаемости есть несколько планет, на которых мы не можем жить.
\ml{$0$}
{Причины разные, чаще всего это, как ты выразился, <<твари>>.}
{Reasons differ, and mostly it is, as you said, `beasts'.}
Но есть нюанс.
Эти твари встроены в планетарный баланс, они часть экосистемы.
Экосистема должна быть сохранена, если мы не хотим превратить очередную твердь вот в это.

Люцифер обвёл рукой унылый пейзаж Преисподней.

\ml{$0$}
{--- А мне нравится, --- пожал плечами Ду-Си.}
{``I like it,'' Du-Xi shrugged.}
\ml{$0$}
{--- Немного напоминает дом.}
{``Reminds me a little of my home.''}

--- Чёрная Скала --- отвратительное место для жизни, и мы не будем превращать планеты в её подобие.

\ml{$0$}
{--- Люцифер, просто дай мне десять толковых ребят --- и мы принесём тебе эти планеты на блюдечке.}
{``Lucifer, just give me ten good men, and you'll have those planets on a plate.''}

--- Ребят ты получишь, командование --- нет.
Курировать твою работу будет стратег и небольшая группа планетологов.

--- Стратег? --- Ду-Си задрал голову и загоготал.
--- А я его не сломаю ненароком?

--- Я выберу пожёстче.
Кто-то должен контролировать твою потребность в уничтожении.
По крайней мере пока мы не будем друг другу доверять.
Например, слышал про Мимозе?

\ml{$0$}
{--- Мими Зайденешталь, Восьмой батальон Первого легиона Тысячи Башен.}
{``Mimi Seidenestahl, Eighth battalion of the Thousand Towers First Legio.}
В прошлом --- Богиня Весны, которую использовали как рабыню для мелиорации на Нимб-3.
Сбежала, устроив техногенную катастрофу и попутно насмерть подрезав примарха клана.

--- Всё не так прозаично, как ты описал, но в целом верно.

--- Не знаю, что значит <<прозаично>>.
В деле были замешаны демоны с самой Нимб-3?

--- Именно.
\ml{$0$}
{Любовь-морковь, цветущая капуста.}
{Carrot love, blooming cabbage.''}

\ml{$0$}
{--- В бездну овощи.}
{``Gulf vegetables.}
\ml{$0$}
{Девушка с репутацией, мне этого достаточно.}
{A reputable girl, that's enough to me.''}

\ml{$0$}
{--- Уже не девушка.}
{``Not a girl anymore.}
\ml{$0$}
{Представишься ему сегодня после караула.}
{Report to him after your shift today.}
\ml{$0$}
{И не вздумай называть его в женском роде и использовать диминутивы, понял?}
{And don't you dare to call him in the feminine or use diminutives, right?''}

Ду-Си понимающе ухмыльнулся.

\ml{$0$}
{--- Приятно было познакомиться, Люцифер.}
{``Nice to meet you, Lucifer.''}

Лу кивнул и пошёл обратно в штаб.

<<Нужно поработать над моими адаптационными обновлениями, --- отметил он.
--- Ду-Си неплохо считывает мысли с лица>>.

\section{Разборка коно}

Друзья двигались чуть быстрее, чем обычно движутся люди Мороза.
Такая спешка называлась <<разборка коно>>.

Если умирало вьючное животное, жители Мороза знали --- у них есть ровно двести восемнадцать ударов сердца до того, как туша превратится в камень.
<<Разборка>> производилась в строго определённом порядке.
Двое снимали шкуру, одновременно сворачивая её в плотный тубус.
Четверо тонкими ломтиками срезали с костей плоть и вынимали внутренности.
Командир занимался головой --- снимал скальп и точным ударом раскраивал череп.
Мозг съедали тут же, на месте --- каждому по ломтику.
Самый сильный член отряда подходил к туше последним и пять-шестью ударами молота дробил кости.
Обломки костей собирали дети.

\section{Падение}

Тахиро много лет снился один и тот же сон.
Дом на скале, дверь над зияющей пропастью, на самом краю обрыва.
Тахиро каждый раз прыгал вниз, не зная, сможет он взлететь или нет.
Если взлетал --- сон продолжался дальше.
Если же тело со страшной скоростью неслось вниз и разбивалось о скалы --- Тахиро просыпался в холодном поту.

В этот раз всё начиналось точно так же.
Тахиро выглянул из двери, бросил взгляд в бесконечно синие небеса --- цвет, которого Преисподняя не знала никогда --- и прыгнул вниз.
Крылья не раскрылись.
Но Тахиро, сжав зубы, обратил взор на летящую к нему твердь --- и крепко приземлился на ноги.

<<Что произошло?>>

Тахиро открыл глаза, обдумывая непривычный сон.
Рядом похрапывал Грисвольд, рядом читал какую-то книгу Лу, проговаривая слова губами и улыбаясь.
Всё казалось обычным и в то же время не совсем...

<<Я сделал то, к чему шёл всю мою жизнь, --- понял Тахиро.
--- Моя жизнь зависела от того, чем я не управлял --- от возможности взлететь.
Но на этот раз я выжил и при падении.
Это мой сон, это моя жизнь.
И правила тоже мои>>.

Тахиро улыбнулся, перевернулся на другой бок и снова заснул --- на этот раз без сновидений.

\section{Спорщик}

--- Знаешь, ты очень умный для человека, --- сказал Грис.
--- Но иногда сказанное тобой похоже на редкостную чушь.

--- Время покажет, --- лаконично ответил Тахиро.

Грис одобрительно посмотрел на друга.

--- Именно поэтому я и говорю, что ты очень умный.

--- Потому что я не стал спорить?

--- Потому что не стал спорить, когда спорить не нужно.

\section{Предназначение}

Люцифер обладал способностью, за которую часто себя ругал.
Он мог представить дичайшую ситуацию, не укладывающуюся вообще ни в какие рамки --- и тем не менее ситуация, в силу её логичности, имела право на существование.
Разумеется, эта способность была закономерным следствием высокого интеллекта, но порой доставляла массу неудобств.

Ситуация, которую Люцифер вообразил сегодня, одновременно рассмешила и до крайности встревожила его.
Во время прогулки с Тахиро к ним привязался низкорослый бродячий предсказатель в высоких гэта, обклеенных какими-то карточками и расписными <<магическими>> камнями.
Друзья еле от него отвязались.

--- Ты веришь в предназначение? --- спросил Тахиро друга.

--- Не вижу особого смысла, --- ответил Лу.

Но потом, сидя в кабинете, Лу вдруг представил, что предназначение действительно есть.
Управляет ли им вселенский разум или что-то неразумное и механическое --- неважно.
Но тогда, может статься, весь многолетний опыт существа, все его размышления, страдания, сомнения и мечты могут иметь целью лишь одно --- чтобы в нужное время, в нужном месте это существо сказало нужные слова тому, кому реально суждено изменить мир.
А затем это существо становится для предназначения ненужной деталью.
Самое ужасное, что от этой участи не спасёт ничего --- ведь любые блага и несчастья могут служить точильным камнем для одноразового инструмента.

<<Что, если это я?
И если бы я узнал, что я лишь одноразовый и, вероятно, уже использованный инструмент --- захотел бы я жить так, как живу?>>

\section{Спираль}

--- Людишки снова пометили дома агентов, --- буркнул Гало.
--- Снова спираль.
Вот фото --- как видите, они нарисовали её краской на стене дома.
Вот ещё такой же символ, нацарапанный на гостевом кресле.
А это моё любимое фото --- спираль очень красиво выложена камнями на дороге.

Люцифер рассеянно перебирал фото.
Спираль.
Огненный смерч, самое зрелищное и смертоносное природное явление на Преисподней, разрушительная, неудержимая природная стихия.
И знак <<хорохито>> --- невидимого, неосязаемого, но ужасного врага, с которым полудикие люди вступили в неравную схватку.
Символ говорил сам за себя --- они знали, на что идут.
Для дикаря это было равноценно вызову, брошенному богам.
Люцифер не мог сдержать растущее внутри уважение.

<<Я был неправ, назвав их крысами.
Это люди --- во всех смыслах этого слова>>.

\section{Кицунэ}

--- Гало сегодня не в духе.

--- Он всегда не в духе в полнолуние Четвёртой Луны.
Может, он оборотень?

--- Кто?

--- Оборотень.
Мужчина, который принимает облик демонической лисицы в полнолуние Четвёртой Луны.

--- Я слышал, что Четвёртая Луна в точности повторяет ту единственную, которая была на Древней Земле.
Период обращения, фазы, даже рисунок.

--- Серьёзно?
У Древней Земли тоже была луна с рисунком в виде среза рыбы?

--- Я не знаю точно.
Многое могло поменяться за прошедшие тысячелетия.
Древняя Земля погибла из-за потока астероидов.
Наверное, Древнюю Луну постигла та же участь.
Но одно можно утверждать точно --- цикл Четвёртой Луны в точности повторяет менструальный цикл человеческих женщин, как и цикл Древней Луны.

--- С чем это связано?

--- Вряд ли что-то можно утверждать наверняка.
Например, древние люди трудились при свете Луны, и в периоды новолуния, когда мужчины оставались на ночь дома, у женщин происходила овуляция.
Хорошее объяснение?

--- Хорошее, но верное ли?

--- Вот именно.
В точку.
А почему ты решил, что Гало --- оборотень?

--- Ну... --- Тахиро смутился.
--- Знаешь ли ты, почему все оборотни --- мужчины?

--- Почему?

--- Легенда гласит, что есть те, кто должен был родиться женщиной, но по ошибке родился мужчиной.
Женщина внутри такого человека заперта навсегда.
Ей недоступны женские радости, она не может позволить себе женскую слабость, и даже такие обычные вещи, как менструации и беременность, ей недоступны.
Именно поэтому, когда Четвёртая Луна имеет наибольшую власть, сошедшая с ума женщина превращает мужское тело в демоническую лисицу и мстит миру.

--- Ты думаешь, что Гало --- женщина?

--- Вы --- близнецы, --- пожал плечами Тахиро.
--- Но в тебе очень много женских черт.
Гораздо больше, чем в Гало.

--- Гало только эту легенду не рассказывай, --- хихикнул Лу.
--- Он тебя убьёт.
Это точно.
А скажи, были ли оборотни в твоём поселении?

--- Были.
Каждое полнолуние их отводили в дом встреч, мыли их, умащали их тела благовониями и одевали в женскую одежду.
Тогда духи женщин успокаивались и эти люди спали всю ночь спокойно.

\section{Айну (в мусорку?)}

Айну Крыло Удачи была странным существом.
Долг и гуманность словно не имели в её разуме точек пересечения.
Она могла очертя голову броситься в горящий дом, чтобы достать плачущего ребёнка, и с материнским трепетом ухаживать за ним.
Если же для достижения цели, поставленной командованием, необходимо было вырезать целый город, Айну не церемонилась ни с детьми, ни со стариками --- убивала одного за другим, без жалости и раздумий.

Но так было не всегда.
В первый раз, когда гуманность в её разуме возобладала над долгом, она спасла будущего мужа.
Второй раз стоил ей жизни.

\section{Суп мертвеца}

--- Что может быть увлекательнее путешествий, --- сказал Тахиро.

--- Ничего, --- согласился Грисвольд.
--- Но скучный суп, скучное одеяло и скучный вечер для меня значат куда больше самого увлекательного занятия.

--- То есть ты борешься за суп? --- лукаво спросил Лу.

--- А сможешь за суп умереть? --- усмехнулся Тахиро.

--- Вы оба --- молодые придурки и не понимаете ни сущность, ни назначение супа, --- наставительно сказал Грисвольд.
--- Но однажды вам попадётся кулинар...

--- ... за которого можно отдать жизнь, --- ввернул Тахиро.

--- Ты идиот, --- резюмировал Грисвольд и занялся монтажом.
--- Мертвецу суп не нужен.

\section{Экскурсия по Лотосу}

--- Ты как? --- склонился над молодым парнем Грейсвольд.

Лусафейру тяжело дышал.
Его глаза метались по обстановке, словно у умалишённого. Но спустя минуту он успокоился и довольно улыбнулся.

--- Аммм... интеграция личностей прошла успешно.
Мои благодарности.

Толстяк протянул парню изящный глиняный сосуд, и тот жадно впился в горлышко.
Закашлялся.

--- Что это за гадость, Грейс?

--- Просто солевой раствор с нейропротекторами.
Тебе нужно восстановить мозг после форсированного пробуждения.

Лусафейру поморщился и снова начал пить.

Грейсвольд окинул взглядом комнату.
Дворец был довольно неплохо стилизован под период Анри, но опытный взгляд демиурга распознал и что-то модерновое, и откровенно архаичные нотки в окружающем великолепии.
По крайней мере, занавеси должны быть сделаны из шёлка, а не из этого низкопробного растительного волокна.

Вдруг скрывавшая проход занавесь отпрыгнула в сторону, и в комнату забежала женщина.
Её до невозможности длинные волосы, согласно местной моде, были собраны в затейливые хвостики и косы.
Лоб перехватывал кожаный обруч со звенящими серебряными цепочками.

--- Я готова, --- радостно возвестила женщина.

--- Ты с ума сошла, Айну, --- буркнул Лусафейру.
--- У женщин этой планеты волосы не могут вырасти до такой длины.
Плюс мы будем идти по лесам.
Ты об этом подумала?

--- Лу, если мы пойдём по лесам, то я их соберу.
Ты пей воду, пей.

Лусафейру пожал плечами.

--- А где Тахиро?

--- Тахиро просил передать, что он случайно умер и не сможет с нами погулять, --- сообщила Айну.

--- Жалко, конечно, --- признал Грейс.
--- Но всякое бывает.
Лу, приходи в себя и собирайся, мы уже готовы.

\asterism

Вскоре друзья уже вышли в жар полуденного Шершерота.
Сейчас у местных жителей был час сна, и по узким улочкам гуляла только крупная кремнистая пыль.

--- Вы позволите? --- Грейс галантно подал руку Айну.
Женщина тихо усмехнулась, подхватила Грейсвольда под локоть и неуклюже чмокнула его сквозь паранджу.
Грейсвольд поморщился.

--- Дурацкий мешок...
Ох уж эти изолированные города-государства, их олигархи вечно придумывают всякие идиотские правила...

--- Я его сниму за городом, --- пообещала Айну.
--- Здесь нравы не ахти.

--- Кстати, Грейс, я бы на твоём месте был менее галантен, --- добавил Лусафейру.

--- Ты ревнуешь, что ли?

--- Нет, балбес.
Просто в этих краях не принято ходить за руку, тем более с женщинами.
Если ты не заметил, на парандже рукавов нет.
И да, Айну, твои щиколотки светятся на весь Шершерот.

--- Сейчас сиеста.
Авось пронесёт, --- сказала Айну.
--- Я хочу прогуляться с максимальным комфортом.
К тому же с нами пока ещё наследник престола, да, Лу?

--- Ага.
Шестой сын второй жены.
Наследник наследников.

--- Зато ты красивый, --- утешила его Айну.
--- У первой жены родились какие-то пирожки с глазами.
Я бы таким даже руки не подала.

--- Да кто бы тебя здесь спросил, --- нехорошо усмехнулся Лусафейру.

--- Если меня не спросят, то потеряют способность спрашивать, --- в тон ему ответила интерфектор.
--- Кстати, Грейс, ты обещал экскурсию, а не прогулку.
Ну-ка расскажи историю этого города.
Только не ту ахинею, которую нам втирал придворный историк шасера\FM...
\FA{
Шасер --- правитель-деспот Шершерота (вероятно, видоизменённое s-l: cesar --- <<монарх>>).
}

\textspace

--- Надо же, --- удивилась Айну.
--- Так это твоих рук дело?

--- Да, --- сказал Грейсвольд.
--- Часовая Луна представляет собой поляризующий кристалл.
Каждую одну двадцатую суток она бросает лучи в сторону планеты, и её видно.
Моя идея.

--- Если подумать, у нас во дворце никогда не было устройств для измерения времени, --- заметил Лу.
--- Айну, да сними ты уже этот мешок.

Женщина аккуратно сняла паранджу, сложила её и спрятала на обочине, под сухим кустом.
Её длинные волосы немедленно подхватил горячий ветер, и Лусафейру на секунду залюбовался этим великолепным трепещущим знаменем женственности.

--- Всё-таки задатки технолога у тебя были ещё до Ордена, --- заметила Айну.
--- Хоть и странно говорить о каких-то задатках применительно к хоргетам.

Грейсвольд пожал плечами.

--- Мне это доставляло удовольствие.

\chapter{Интерлюдии}

\section{Сотворение мира (легенда зизоце)}

Жили в незапамятные времена те'кхалы\footnotemark.
\FA{Дословно: <<Люди всепредшествующие>>, <<первопредки>>.}
Были они злые и гордые: зерно не сеяли, металл не ковали, проводили время в праздности.
Всё за них делали звери и птицы, которых те'кхалы взяли в рабство.

И сделали как-то те'кхалы Свет.
Именем его не нарекли, только руки ему сделали, и сказали Свету: работай.
Много у нас те'кхальских детей, и негде им жить.

И стал Свет работать.
Солнце было синее, жестокое, но Свет погладил руками Солнце --- и стало оно жёлтое, ласковое.
Земля была пустынная, жаркая, как печь, но Свет погладил Землю --- и стала она тёплая и гостеприимная.
Океан был горячим и ядовитым, но Свет погладил океан --- и стал океан свежей питьевой водой.

Но на Земле ещё ничего не росло, и не могли те'кхальские дети на ней жить.
Те'кхалы рассердились на Свет: почему на Земле ничего не растёт?
И сказали Свету: работай.

И стал Свет работать.
Погладил Свет Землю --- мхами покрылась Земля, погладил Свет воду --- водорослями заросли озёра, погладил Свет воздух --- жуки начали летать.

Но те'кхалы ещё больше рассердились на Свет: как мы будем есть мох и жуков?
И сказали Свету: работай.

И стал Свет работать.
Погладил Свет мох --- и уснул мох, и стал почвой.
Погладил Свет почву --- и выросли из неё травы, и деревья выросли.
Вдоволь еды стало на Земле.

Но те'кхалы ещё пуще разгневались.
Где, сказали они, звери и птицы?
Кто будет рабами нашим детям?
И сказали Свету: работай.

И стал Свет работать.
Взял он мха с земли для шерсти, и горсть почвы для плоти, и прутьев с кустов для жил, и толстых веток с деревьев для костей, и красной воды из горного озера зачерпнул, и сделал он зверя Ма'кханэ\footnotemark, большого, как дерево акхкатрас.
\FA{Дословно <<зверь, рождающий всех>>.}

И сказал Ма'кханэ: здравствуй, Создатель!
И Свет заплакал, потому что это были первые добрые слова, которые он услышал.
И спросил Ма'кханэ: как тебя зовут, Создатель?
И Свет ответил, что никто не дал ему имени.
Тогда Ма'кханэ сказал: я буду звать тебя Безымянным.

Увидев Ма'кханэ, те'кхалы разгневались ещё больше.
Этот зверь, они сказали, убьёт наших те'кхальских детей, и косточек от них не оставит!
Он будет плохим рабом!
И сказали Свету убить Ма'кханэ.

Заплакал Свет.
Полился над Землёй бесконечный дождь, реки вышли из берегов, леса затопило.
Спросил у него Ма'кханэ: почему Безымянный плачет?
Почему грустит тот, кто сделал Солнце ласковым, Землю --- гостеприимной, кто посеял траву и деревья и дал жизнь Ма'кханэ?
И Свет рассказал Ма'кханэ о своём горе.

Не волнуйся, Безымянный, ответил Зверь, не твоё это горе.
И родил Ма'кханэ оленя, змею, ягуара, ястреба, лягушку, стервятника, рыбку, и множество других родил.
И с каждым рождённым он уменьшался в размерах --- сначала стал ростом со скалу, потом с дом, а потом и с яйцо.
Когда родил Ма'кханэ мышей и тушканчиков --- и не стало его.

Сказали те'кхалы Свету: пусть звери и птицы и рыбы и жуки нам поклонятся.
А звери и птицы и рыбы и жуки сказали: не будем мы кланяться те'кхалам, лишь Безымянному мы поклонимся.
Свет сказал: я не хочу, чтобы вы кланялись мне.
Звери и птицы и рыбы и жуки сказали: в таком случае мы не будем кланяться никому.

Те'кхалы пришли в ярость и велели Свету убить себя.
Ты, сказали они, нам больше не нужен.
Ты, сказали они, плохой раб.
Но взмолились Солнце и Земля и Вода, и мох и водоросли и травы и деревья, и звери и птицы и рыбы и жуки: не оставляй нас, Безымянный!
И Свет ослушался те'кхалов.
Они, сказал он, добры ко мне, а вы злы.
Я, сказал он, не буду убивать себя.

Тогда те'кхалы пошли войной на Свет.
Они прилетели на железных птицах и привезли на Землю своих те'кхальских детей, своих солдат и своих рабов.
Но Свет поднял руки, засияли они ярче Солнца\footnotemark\ --- и те'кхалы забыли, зачем прилетели, и стали подобны детям.
\FA{Вспышка --- один из неясных моментов докалендарной истории Тра-Ренкхаля, упоминающийся в мифах и легендах почти всех народов.
Возможно, речь идёт о природном феномене --- падении крупного метеорита.
Согласно другим теориям, это было применение атомного оружия поселенцами с Лотоса или неудачная попытка уничтожить демиурга, приведшая к технологической катастрофе.
Следы древнего катаклизма, который в значительной степени изменил климат и лицо Тра-Ренкхаля, найдены в области западных Смертных Пустошей материка Корона.}

И сказал им Свет: отныне вы будете жить здесь.
И не будете вы больше называться те'кхалами, но будете людьми.
И не будут больше птицы и звери вам рабами, но будут друзьями.
И сказали люди: в нашей стране мы кланялись хозяевам, а наши рабы кланялись нам.
Должны ли мы тебе поклониться, Безымянный?
Свет сказал: я не хочу, чтобы вы кланялись мне.
Люди сказали: в таком случае мы не будем кланяться никому.
На том и порешили.

И жили люди счастливо, и не было ни смерти, ни болезни, пока не пришёл Безумный и не вынудил людей нарушить обещание.
Склонились люди перед Безумным, изгнавшим Безымянного.
Оттуда и пришли к людям смерть и болезни, потому что нет того хуже, когда нарушаешь обещание.
И океан с тех пор стал солёным и непригодным для питья, потому что сидит на его дне Свет и плачет от обиды, что бросили его люди, забыли его доброту.

А те те'кхалы, что остались, умерли от злости и голода, потому что рабы их стали свободными, и не было те'кхальским детям ни еды, ни места для жизни.

\section{Лунное молоко}

<<Но как же убить Маликха? --- говорили они.
--- Нет среди нас тех, кто мог бы победить его в одиночку или отрядом.
Он знает, как звякает спрятанный нож, знает, как шуршит удавка, он знает запахи и вкусы всех ядов>>.

<<Тогда мы дадим ему тот яд, который не имеет вкуса и запаха>>, --- сказали каменщики, и собрали в пещерах лунное молоко, и приготовили пищу, подмешав в неё лунное молоко.

Маликх не узнал запаха лунного молока, потому что оно не имеет запаха.
Он не почуял вкуса, потому что и вкуса у лунного молока нет.
Но он съел много яств, и вместе с ними много лунного молока, и лунное молоко притупило его острый ум, замедлило его глаза и отняло силу у его рук и ног.

\section{Прозрачная Борода (ЛоО)}

Были идолы Микхана умны, но недоставало им мудрости, и боялись они, что тайны их узнают и богатства их иссякнут.
Узнав, что Ликхмас, Маликх и Чхалас вошли в их город, они попрятались по домам, и каждый микханский житель выставил за дверь крохотную куклу отвратительного вида --- с огромными зубами и глазами, горящими Каменным Жаром.

<<Не дайте им прикоснуться к себе, --- сказал Маликх.
--- Микханская кукла вгрызётся в вашу плоть и проест её до самой головы\FM>>.
\FA{Предположительно, прообразом микханских кукол стали Белые Гусеницы --- оружие времён Войны Тараканов. }

\textspace

Шли друзья по пустому городу, и ни одна дверь не открылась на стук, и ни одно окно не удавалось сломать --- искусны были зодчие Микхана.

<<Мы не можем уйти просто так, --- сказал Маликх.
--- Едва мы выйдем из города, как жители Микхана покинут свои дома и натравят на нас великана Прозрачная Борода, и все духи лесные не помогут нам, если он нападёт на нас дорогой.
Мы должны убить его или остаться здесь пленниками навечно>>.

<<Одно оружие Микхана поможет нам найти другое, --- сказал Ликхмас.
--- Кукла видит то, чего не видят человек и идол.
Кукла открывает один глаз, если ждёт приказаний, но при виде врага своего хозяина она открывает оба глаза>>.

Он отрезал голову сломанной микханской кукле, взял её за волосы, нажал на левое ухо, погладил правое --- и приоткрыла кукла левый глаз.
Тем временем наступила холодная ночь, и город погрузился во тьму;
глаз куклы засветился ярким Каменным Жаром, освещая дорогу, деревья и жилища.

\textspace

<<Прозрачная Борода невидим, --- сказал Маликх.
--- Он может быть где угодно>>.

<<Тот, кто невидим, должен быть наг и держать своё тело в чистоте, --- сказал Ликхмас.
--- Его следует искать у воды>>.

<<Тот, кто невидим и наг, чувствует себя замёрзшим и ничтожным, --- сказал Чхалас.
--- Его следует искать в самом тёплом, маленьком и далёком от жилья озерце, которое только есть в Микхане>>.

Долго искали друзья озерцо Прозрачной Бороды, ведь Микхан ныне называют Лабиринтом Озёр.
И вдруг, у самого маленького и незаметного, скрытого в буйнотравье озера, кукольная голова приоткрыла второй глаз.

<<Мы не можем победить Прозрачную Бороду в его озере, --- сказал Маликх.
--- Нужно выманить его наружу>>.

<<Мы не сможем победить того, кто невидим>>, --- сказал Ликхмас.
Он набрал сухой придорожной пыли, разложил её в три мешочка и раздал мешочки друзьям.

<<Тот, кто силён, но чувствует себя ничтожным, обидчив и драчлив>>, --- сказал Чхалас.

Чхалас выкрикнул бранное слово, которое не знали Маликх и Ликхмас, и засмеялся смехом, который может пробуждать лишь обиду и гнев.
Взвыл от обиды и ярости Прозрачная Борода и бросился к ним из своего озера.

\textspace

<<Подлое оружие, --- сказал Маликх.
\ml{$0$}
{--- Нельзя оставлять на дороге такое.}
{``It mustn't be left on the road.}
\ml{$0$}
{Но будь я проклят дважды два раза, если достану его из ножен!>>}
{But let me be cursed twice and twice again if I unsheathe it!''}

И, не медля, покинули Ликхмас, Маликх и Чхалас надменный город Микхан.
И оплакивали микханские идолы потерю своих кукол и великана Прозрачная Борода, и долго слышался их плач над джунглями.
Вскоре тайны Микхана были расхищены и стали достоянием мира, и богатства его иссякли, и слава его потускнела, и дома обветшали, и жители стали подобны своим диким собратьям из лесов.
Так закончил свои дни город Микхан.

\section{Странствующий Король}

\textbf{Талианская сказка}

Жил на свете Странствующий Король.
Он был настоящим Королём --- и по праву рождения, и по умениям, и по королевскому характеру.
Его посох был недорогим, но красивым;
такой же была его одежда.

Странствующий Король искал место, которое мог бы назвать домом.
Он останавливался в самых дешёвых гостиницах и расспрашивал местных крестьян, где он мог бы пожить.
Крестьяне с радостью давали ему место --- ведь характер у Короля был поистине королевский.

Но всё менялось, когда жители узнавали в нём Короля.
Кто-то начинал восторженно кричать его имя под окнами.
Кто-то начинал спрашивать его мнение, как лучше поступить в том или ином случае.
Прочие же в лицо называли его самозванцем, хотя Король не претендовал ни на одну корону.

Каждый раз к новому дому Короля приходила толпа, возглавляемая деревенским старостой или бургомистром.
Они требовали, чтобы Король ушёл.
<<Нам здесь проблемы не нужны>>, --- говорили они.

И Король, вздохнув, снова надевал дорожную одежду, брал свой недорогой, но красивый посох --- и уходил навсегда.

\section{Страна каменных духов (ЛоО)}

В стране каменных духов была единственная ценность --- живой камень, из которого и были сделаны духи.
Живым камнем платили за работу, живой камень хранили в кошельках и носили на шеях, как ожерелья, живым камнем платили правителям.
Если же дух был беден, то расплачиваться ему приходилось частями собственного тела.
Многие бедные духи, собираясь у подземных костров, проводили долгие часы, обсуждая, что отдать сегодня.
Часть руки, жертвуя силой?
Часть ноги, жертвуя крепостью стойки?
А может быть, отдать правителю уши или кишки, ведь на них не так трудно заработать?

Если же у духа не оставалось живого камня, он пропадал и больше не существовал никогда.

Многие бедные духи, отчаявшись, отваживались на Насилие.
Они воровали чужой живой камень.
Но наказание за это было суровым;
однажды Ликхмас, Маликх и Чхалас увидели, как в суд привели маленького плачущего духа с большим-пребольшим носом.
Суд постановил отрезать нос и отдать его правителю.

<<Но ведь это вина правителя, что духу пришлось пойти на воровство! --- нахмурился Чхалас.
--- И они намеренно выбрали самую большую часть тела, его гордость!>>

<<Я не могу больше на это смотреть>>, --- признался Маликх.
Сверкнула сабля --- и судья развалился на две ровные половинки, которые тут же начали драку за кусочки тела.
Освобождённый дух бросился прочь с криками <<Чужаки разделили судью! Чужаки разделили судью!>>

<<А вот и благодарность, --- сказал Ликхмас.
--- Бежим отсюда>>.

\section{Ключ}

Увидели друзья, что идут разбойники, а в повозке у них чан большой, в рост человеческий.

<<Змея в чане, --- сказал Ликхмас.
--- Чую по тому, как трясётся чан, будто аспид в нём огромный кругами ползает>>.

Оглядел Маликх разбойников.

<<Не одолеть нам их, --- сказал он.
--- В бою любой из них меня превосходит.
Чую по походке да по поворотам глаз --- гиблое дело>>.

<<Нашей будет змея, --- сказал Чхалас.
--- Сними-ка замок золотой с сундука да ключ мне дай>>.

Ткнул Ликхмас пальцем в одну заклёпку, в другую --- и упал с креплений замок.

Нашёл Чхалас скалу возле дороги.

<<Вбей замок в скалу>>, --- сказал Чхалас.

Ударил Маликх раз по скале, второй --- и вогнал золотой замок в скалу, словно дверь в ней закрытая.

Едут разбойники мимо.
Видят --- замок золотой в скале.

<<Подвал тайный! --- обрадовались разбойники.
--- На всю жизнь золота достанем!>>

Стали счастья пытать, отмычками замок ковырять, да не поддаётся.
И кирками скалу били, и кусачей бумагой жгли --- не поддаётся скала.
Утомились разбойники.

<<Гиблое дело, --- говорят, --- мастер делал.
Ключ нужен, без него сокровища не достать>>.
И поехали дальше.

А Чхалас тем временем отбежал на десять кхене и бросил на дорогу золотой ключ.
Едут разбойники мимо, видят --- ключ золотой лежит.
Схватили разбойники ключ, переглянулись да побежали назад что есть мочи, бросив чан.

Выпустили друзья змею из чана.

\section{Город потерянных детей (ЛоО)}

Дети проводили время в играх.
Если же игры заканчивались дракой, то дети фантазировали;
они придумывали себе силу, как у взрослых, и любящих их взрослых, имеющих силу.
Но взрослых не было, и силы было взять неоткуда.
Тогда дети играли в любовь и приписывали силу тем, кого любили;
когда открывалась правда, игры в любовь тоже заканчивались дракой.
Дрались дети постоянно, и фантазии захватывали их, словно волосяные силки --- мелкую пичугу\FM.
\FA{
Использовать волосяные силки для птиц очень опасно.
Если птицу вовремя не подхватить, то она запутается в волосе и умрёт от удушения.
У сели было поверье: погибшая в волосяных силках птица --- двадцать дождей несчастий.
}

У детей был вождь.
Он говорил детям, как жить, и они беспрекословно слушались его, как взрослого.
Вождь боялся детей, потому что они слушались его;
он знал --- если появится другой вождь, то все будут слушаться другого вождя.
Поэтому он говорил детям чаще играть и не препятствовал дракам.
Главными играми он велел считать соревнования и судил на них, ведь судье не бросают вызов.
В свободное время он велел играть в любовь, чтобы детей было больше.

Когда вождь оставался один, он мечтал о любящих взрослых, как и все.
Но его фантазии были похожи на дым, он им не верил.
Потому он и становился вождём.

\textspace

Ликхмас вышел на главную площадь и закричал:

<<Дети!
Мы можем заботиться о вас!
Я --- жрец, я могу учить и лечить вас.
Мой друг Маликх --- воин, он может дать вам твёрдую опору и научить владеть своим телом.
Мой друг Чхалас --- купец, он сделает ваших врагов друзьями и разделит блага по справедливости!>>

Дети молчали.
На площадь начал надвигаться чёрный туман, жгучий и отдающий гнилью.

<<Мы --- взрослые! --- закричал Ликхмас.
--- Мы можем стать вам кормильцами, пока вы не будете готовы вступить на собственный путь!>>

<<Вы не взрослые, --- хором ответили дети.
--- Это мы --- взрослые.
А вы --- калеки.
Вы --- уродливые великаны.
Вы опасны.
Вас надо уничтожить>>.

<<Они нас убьют, Ликхмас, --- сказал Чхалас, и друзья поняли, что это правда.
--- Нам следует бежать>>.

После этих слов Ликхмас, Маликх и Чхалас побежали, преследуемые огромным множеством детей.

Но вспомнил Ликхмас про дар, преподнесённый ему в <<Бамбуковой клетке>>;
бросил он на землю ветвь из головы идола, и выросла она в непроходимые джунгли.
Остановились Ликхмас, Маликх и Чхалас, и слушали они, как отчаянно зовут друг друга заблудившиеся в лесу дети.

<<Можно ли помочь?>> --- спросил Ликхмас у друзей.

<<Нет, --- ответил Маликх.
--- Многие из них погибнут.
Но те, кто выживет в одиноком скитании, станут взрослее.
Если их фантазии помогут им совладать с сельвой, если найдутся те, кто оставит фантазии ради жизни, у города появится шанс>>.

\section{Цветы и семена}

Жили-были два соседа-цветочника --- Марин и Марса.
Оба они были искусны в своём деле --- они искали самые лучшие семена на заливных лугах, выращивали в оранжерее цветы и продавали их на рынке.

Но однажды Марсе пришла мысль.
<<Зачем я буду проводить драгоценные кхамит, выискивая семена?
Ведь семена я всегда могу купить у соседа.
Не лучше ли будет мне сосредоточить все силы и умения на выращивании цветов?>>

Сказано --- сделано.
Отныне Марса стала совершенствовать своё искусство цветоводства, а семена ей стал приносить сосед.
Доходы цветочницы увеличились --- ведь её цветы были гораздо лучше соседских.

Посмотрел Марин и подумал: <<Цветы Марсы действительно лучше моих.
Но она нуждается в моих семенах и всегда с удовольствием покупает их.
Зачем я буду тратить часы в оранжерее?
Не лучше ли сосредоточиться на поиске лучших семян и продавать их Марсе?>>

Сказано --- сделано.
Отныне Марин стал совершенствоваться в поиске семян.
Доходы его увеличились --- ведь чем лучше он находил семена, тем красивее были цветы Марсы.

Однажды пришёл Марин к цветочнице и, как обычно, предложил ей свои семена.
Но Марса неожиданно отказалась их покупать.

<<Что случилось? --- удивился сосед.
--- Раньше ты с удовольствием брала мои семена>>.

<<Твои семена по качеству хуже тех, что приносит мне другой торговец>>, --- ответила Марса.

<<Мы же соседи и давно друг друга знаем!>>

<<Да, но если я буду брать твои семена лишь по старой памяти, люди перестанут покупать мои цветы и я разорюсь>>, --- объяснила цветочница.

Подумал Марин и понёс семена к другой цветочнице.

<<Какие хорошие семена! --- восхитилась она.
--- Сколько просишь, Марин?>>

<<Десять гран>>, --- ответил торговец.

<<Я не смогу их купить по такой цене, --- грустно ответила цветочница.
--- Может быть, ты продашь их мне за пять?>>

<<Я знаю цену своим семенам>>, --- отрезал Марин и отправился к третьей цветочнице.
Та встретила его холодно, взяла семена и начала ломать их.

<<Что ты делаешь?!>> --- возмутился Марин.

<<Хочу убедиться, что они настоящие, --- ответила цветочница.
--- Дождь назад один из торговцев продал мне раскрашенные камни вместо семян>>.

<<Ты убедишься, что все они настоящие.
Но какой толк будет от сломанных семян?>> --- закричал Марин, забрал товар и вышел, не попрощавшись.

Четвёртая цветочница тоже встретила его холодно.

<<Покажи товар>>, --- сказала она.

Марин, удивляясь, выложил своё богатство на прилавок.
Цветочница вздохнула.

<<Извини.
Недавно в город приходил плут.
Он уверил всех цветочниц, что привезёт самые лучшие семена, собрал с них золото и исчез.
Я должна была убедиться>>.

Осмотрев семена, цветочница с радостью приняла их, хоть и заплатила на гран меньше, чем Марса.

\section{Кораллица, жаба и красный шар (сказка сели)}

Однажды один мальчик поймал птицу-красный шар.
Птица забавно съёжилась, как мяч, а мальчик долго перекидывал мягкий красный мячик в руках.

Наконец ему стало жалко птицу, и он отпустил её у реки.
Красному шару от страха захотелось пить;
он принялся кружить над водой, но нигде не видел ни единого места для водопоя.

Тогда красный шар подлетел к мальчику:

<<Мальчик, я очень хочу пить.
Ты отпустил меня, и я тебе благодарен;
будь же для меня другом --- сделай для меня водопой!>>

Мальчик подумал, отломил ветку у сухого кофейного дерева и вкопал её в дно реки.
Красный шар обрадовался: с ветки он мог и пить, и взлетать со своих коротеньких лапок.

Увидела это каменная жаба и тоже обратилась к мальчику:

<<Мальчик, я тоже очень хочу пить, но мне трудно выползать из воды по такому крутому берегу.
Будь же и мне другом --- сделай водопой!>>

Мальчик подумал, отыскал на берегу обломки дерева и сделал для жабы хорошую лестницу.

Обидно стало красному шару;
дождавшись, пока мальчик уйдёт, он обратился к жабе:

<<Не стыдно ли тебе?
Я был его игрушкой и потому получил водопой, а тебе всё досталось просто так!>>

<<Чего мне стыдиться? --- удивилась жаба.
--- Не я заставила мальчика играть тобой, как мячом.
Я лишь попросила лестницу>>.

Прошла декада --- и кофейную ветку унесло течением.
Лестница же осталась на берегу целой.
Ещё больше обиделся красный шар:

<<Не стыдно ли тебе?
Я был его игрушкой и потому получил водопой, а тебе всё досталось просто так, да ещё и лучше!>>

<<Чего мне стыдиться? --- удивилась жаба.
--- Немудрено, что ветку унесло --- она была в потоке.
А моя лестница на глинистом берегу>>.

Но красный шар был недоволен.
Предложил он позвать мальчика и задать ему вопросы, а судьёй между собой и жабой выбрал змею-кораллицу, что проползала мимо.

Загулила жаба, и вскоре пришёл к реке мальчик.
Заметив кораллицу, он плавно поднял копьё, чтобы змея его не увидела, и пронзил ей голову.

<<Зачем же ты убил нашего судью?!>> --- воскликнул красный шар.

<<Я не знал, что ядовитая кораллица --- ваша судья, --- ответил мальчик.
--- Я поступаю так со многими ядовитыми змеями, чтобы они не укусили меня>>.

<<Тогда ответь: почему ты играл со мной, как с мячом?>>

<<Потому что ты мягкий и похож на мяч>>.

<<Почему ты не поиграл с жабой?>>

<<Потому что жаба тяжёлая и колючая, с ней играть неинтересно>>.

<<Ответь же тогда напоследок: почему ты сделал водопой и мне, и жабе?>>

<<Потому что вы оба об этом попросили>>.

Красный шар всё понял и попросил мальчика ещё раз сделать ему водопой.
Больше он с жабой не ссорился.

\textspace

<<Безымянный слепил меня из пламени, как были слеплены звёзды.
Обижались ли звёзды на то, что их слепили?>>

<<Безымянный слепил меня из пыли и воды, как была слеплена твердь.
Обижалась ли твердь на то, что её слепили?>>

\section{Обнимающий Сит}

Когда-то давно в пустыне, в одном из поселений народа сели жила девушка Ситхэ.
Она была не очень красива, низка ростом, а ещё плохо видела.
Но она очень любила детей.
И, когда все взрослые уходили в поход, Ситхэ ходила ночью по домам и успокаивала детей.
Она пела детям песни, и от её тихого чарующего голоса и ласковых рук дети быстро засыпали.

Однажды у одной женщины пропало золотое ожерелье, и подозрение пало на девушку.
Жители вызвали Ситхэ на суд.
На суде девушка не сказала ни слова, и жители, сжалисшись, избавили её от статуса Насильника, но присудили вернуть женщине ожерелье или его стоимость --- горсть золотого песка.

Придя домой, девушка наполнила чашу водой, сняла с колодца верёвку, намотала на руку и ушла в пустыню.
Больше её не видели.

Однако дети, которые очень скоро повзрослели, не забыли добрую Ситхэ.
Многие говорили, что печальная дева-призрак с обмотанной толстой верёвкой рукой и светящейся чашей по-прежнему приходит к оставленным без присмотра детям и поёт песни.
Но детей слишком много, а Ситхэ всего одна, и иногда она плачет, что не может утешить всех.
Ночные дожди на Змею 9, после которых пустыня на несколько дней покрывается цветами, стали называть <<слезами Ситхэ>>.
Дети до двадцати пяти дождей в этот день вьют из пустынных цветов венки, и взрослые при встрече обязаны проявить к ребёнку любовь --- угостить его сладостями, поучаствовать в игре или просто обнять.
Также отсюда пошла женская традиция во время похода кидать в колодцы золотые песчинки или украшения --- чтобы помочь Ситхэ вернуть несправедливо наложенный на неё долг и она смогла вернуться к детям.

К слову: мужчины народа сели часто делали то же самое, но чаще тайно, чтобы никто этого не увидел.
Во всяком случае, охотники за сокровищами находили в колодцах предостаточно мужских украшений.

\section{Легенда об обретении}

Философский роман, написаный автором, имеющим домашнее имя Карлик (прочие имена неизвестны).
Впоследствии ушёл в народ и стал передаваться из уст в уста.

Молодой жрец по имени Ликхмас встречается со стариком, который говорит ему: кихотр Безумного способен вершить судьбы людей.
Но на нём есть грань, выпадение которой означает смерть самого бога, и раз в десять тысяч дождей рождается человек, который способен бросить камень именно этой гранью вверх.
Старик сообщил, что юноша и есть тот самый человек.
Ликхмас понимает, что способен положить конец бесконечным жертвоприношениям и войнам.
Он отправляется в путь, чтобы узнать способ добыть кихотр.
С ним отправляется могучий воин, лучший из бойцов --- Маликх, и пронырливый, обладающий сладким голосом и даром убеждения купец --- Чхалас.
Вместе друзья преодолевают ужасные злоключения и наконец достигают горы Рыбья Флейта --- самого высокого пика Старой Челюсти.
На вершине они находят тот самый кихотр.

Возвратившись домой, Ликхмас рассказывает о камне и своих намерениях друзьям, и вскоре об этом знает уже весь город.
По наущению жрецов воины собираются и идут к Ликхмасу домой.
Чхалас узнаёт о заговоре и успевает предупредить Маликха перед тем, как глотнуть из чаши, в которую трактирщик подлил лаковый сок.
Маликх мчится к дому Ликхмаса и успевает преградить путь толпе.
Перед дверью завязывается бой, Маликх в одиночку сдерживает две сотни врагов, но в ход пошли факелы и горящие стрелы --- дом загорелся.

Ликхмас решается бросить кихотр, чтобы положить конец Безумному, но ему вонзает кинжал в сердце родной брат.
Сестра успевает подхватить выпавший из пальцев умирающего камень перед тем, как тот упал на землю.
Уставший, израненный Маликх вместе с двадцатью оставшимися врагами отступает в занимающийся огнём дом и, видя окровавленное тело друга, в отчаянии отрубает руку сестры с зажатым в ней кихотром.
Артефакт падает на пол.
Враги успевают выбежать, а великий герой с телом друга и божественным камнем остаётся внутри.
Вслед за домом полностью выгорел город, и люди вынуждены покинуть пепелище\FM.
\FA{
Согласно одной из версий, этим городом был Тхитрон, так как с цатрона название переводится как <<пепелище>>.
}
Вскоре это место поглотили джунгли.

Согласно другой версии, перед тем как Ликхмас умер, кихотр всё же упал на пол, и маленький братишка юноши успел увидеть то, что было нарисовано на этой грани.
Его успели вытащить из огня, и когда он повзрослел, то уплыл за пролив Скар в Яуляль, перед походом поклявшись доверить тайну кихотра тому, кто способен её понять.

Возможно, что роман изначально был написан в жанре <<выбери конец сам>> --- такой приём был распространён в то время.

P.S. Писатель свёл воедино в одном произведении людей из разных эпох.
Маликх --- мифический герой, обладающий божественной силой --- никогда не существовал.
У трикстера Чхаласа был реальный прототип, живший на 500 дождей позже появления легенд о Маликхе, т.е. встретиться они не могли никак.
Но произведение оказалось настолько сильным, что многие считали его логическим завершением легенд о купце и воине.

\section{Лунные сады, или Повесть о странах цветов}

Луна (sl: luna) --- естественный спутник планеты.
Основой, скорее всего, послужила история, зародившаяся ещё до прибытия первых людей на Тра-Ренкхаль --- собственных лун у планеты не было.
Луна --- магическая страна света, <<висящая далеко над землёй>>.
Согласно легенде, там живут прекрасные бессмертные девы, собирающие в садах яблоки бессмертия.
Они очень печальны, потому что им нет пути на землю.

Впоследствии легенда была адаптирована культурологами-тси в рамках подготовки к одичанию (В частности, <<летающий змей>> --- это явная отсылка к Стальному Дракону, в более ранних версиях герой летел к Луна верхом на черепахе).

\subsection{Сокар}

История начинается с того, что Марин потерял мать.
Он вернулся после похорон в пустой дом и не знал, чем заняться.
Друзья (Ларин (Лар), Тагир (Тар)) предложили ему отправиться на охоту.
Они дошли до края острова и увидели другой остров с большим городом, проплывающий мимо.
Марин сказал, что хочет попасть в страну Луна, что его здесь ничто больше не держит.
Друзья огорчились, но решили ему помочь.
За час они собрали баллисту и зацепили остров.
В последний момент верёвка соскользнула и Марин полетел вниз, но всё-таки благополучно добрался до острова по верёвке.
Тагир с Ларином развели костёр и до темноты разговаривали о происшедшем, удивляясь неожиданной смелости Марина.

Недалеко от селения, в лесу, под скалой Марин обнаружил жалобно пищавшего птенца хищной птицы размером с взрослого человека, запутавшегося в ветках.
Марин потратил весь день на то, чтобы вытащить несчастную птицу.
Предположив, что птенец упал со скалы, Марин решил поднять его обратно.
На скале действительно обнаружилось огромное гнездо, и Марин аккуратно положил птенца на место.
Обернувшись, Марин увидел, что прямо за ним на скале сидит взрослая птица и смотрит на него --- Марин даже не услышал, как она прилетела.
Он уже ожидал смерти, когда птица заговорила с ним на его языке.

Она спросила, почему он не убил птенца и не разорил гнездо, ведь местные жители поступают с ними именно так из-за того, что орлы таскают их баранов.
Но орёл также сказал, что они никогда не брали баранов больше, чем им нужно для пропитания.
Марин ответил, что лично у него орлы не взяли ни одного барана.
Орёл оценил доброту человека, дал ему немного мяса и показал пещеру, где можно выспаться.

Марин никак не мог заснуть и пришёл к орлу --- поговорить.
Рассказал ему о своём поиске.
Орёл удивился и пообещал отвести Марина к Симуру.
Симур выслушал рассказ орла о случившемся и спросил, что бы Марин хотел в награду, ибо орлы размножаются раз в десять лет по одному птенцу.
Марин ответил, что орёл уже накормил его и он очень благодарен за это.
Симур удивился, потом что-то сказал приближённому.

Птица преподносит дары --- ягоды.
Герой придумывает аргументы против.
Синяя ягода --- знание языков, красная ягода --- исцеление, етц.
Опасность синей ягоды --- нельзя слушать речи летающих змей.
Они искусны в речах, и если проведают, что ты говоришь на знакомом языке --- убедят в чём угодно.
Лишь птицы сравнятся с ними в красноречии.
Птицы вели со змеями войну (орёл со змеёй в клюве).

Симург сказал, что никто из них не сможет отнести его на Луна, но если Марин будет носить эту ягоду с собой, то будет понимать любые языки, и что отныне Марина считают другом все птицы.
Марин поблагодарил Симура и орла и пошёл дальше.

\subsection{Хазер}

Город Хазер.
Город богачей, в котором даже уборные стоят немалых денег.
<<Если бы жизнь в Хазере была дешёвой, сюда бы не стремились бедняки и мы бы тогда сами стали бедняками>>.
Конное поло в роскошных доспехах и на роскошных лошадях.
Юноша из богачей увидел сон, что игра обессмертит его имя.
Бессмертие --- единственное, чего не было у богачей.
Он вышел на поле без доспехов.
Затем купил прямо в игре лучшего коня у богатого старика, отдав ему тяжёлую золотую сбрую.
И погиб на поле (продумать фабулу).

\subsection{Трогваль}

Город с типичной теократией.
Люди поклоняются Богу-Творцу и Пророку его.
Правители называют себя самыми прогрессивными, ибо поклоняются одному богу, а не многим, как варвары.
Основная масса народа --- религиозные фанатики, правят городом жрецы, живущие в Чистом городе.
Поддерживают власть жрецов торговцы (Торговый квартал), религиозная армия (Стальной квартал).
Остальная масса народа --- крестьяне и ремесленники (Город Мастеров).
Чтение светских книг, магия, нехрамовая музыка и песни, нерелигиозное искусство считаются греховными, хоть и порой необходимыми, из-за чего над людьми постоянное чувство вины.
Большая часть магов, учёных, врачей, летописцев, музыкантов и художников живёт в Проклятом городе, или Городе Безбожия --- месте, куда простой люд не ходит без необходимости, а если и ходит, то только держа перед собой руки, сложенные в огораживающем от зла жесте.
В Проклятом городе валютой являются знания.
Периодически (после мора, поветрия, неурожая) толпа народа разоряет Проклятый город.
Тем не менее люди пользуются услугами жителей Проклятого города --- к ним ходят лечиться, правители обращаются к летописцам, чтобы те прославляли их святость, праведность и справедливость, учёные и маги предоставляют правителям устройства, имитирующие божественное чудо.

Марин пришёл в Проклятый город поздним вечером.
Его окликнул человек, сидевший под вывеской <<Врачеватель Шамаль>>.
Они познакомились.
Когда человек узнал, что Марин с другого острова, он пригласил его к другу-летописцу, Салему, пообещав, что тот накормит его и даст ему ночлег за рассказ о его родине, о местах, где Марин побывал.
Удивлённый Марин пошёл за Шамалем.
По пути им попались несколько жителей города, боязливо идущих к врачу с жестом, отгоняющим зло.
В тёмном переулке к ним подошли два дюжих подвыпивших парня из города мастеров, требуя денег.
Шамаль с Марином опрокинули парней, Шамаль выбежал на открытую улицу и крикнул по-птичьи, набежал народ и обработал обнаглевших мастеровых.
Шамаль сообщил, что такое иногда бывает и что у жителей Проклятого города свои средства защиты --- от сыновей жрецов, приехавших позабавиться в Проклятый город, откупаются, а обнаглевшим простолюдинам просто можно намять бока.
Шамаль довёл Марина до дома Салема, тот накормил и напоил путешественника и всю ночь расспрашивал его о родном острове.
На встречу пришёл также маг Русталь, который попросил у Марина посмотреть его синюю ягодку, клятвенно пообещав вернуть её на следующее утро.
Шамаль заверил Марина, что в Проклятом городе никогда не обижают чужестранцев.

Марин жил у Шамаля, помогая тому с работой --- собирал травы и проч.
Вечерами жители собирались у кого-нибудь дома, проводили время за интеллектуальными беседами, музицированием, вином и игрой в Метритхис.

Замечен был астрономами остров, который приближался с юга.

В городе случилось чумное поветрие, распространившееся из-за купания в святых купелях.
Врачи и маги поголовно ушли в город, лечить народ.
Марин остался в Проклятом городе.
С чумой кое-как справились, врачи вернулись, и тут кто-то пустил слух о том, что всё это --- вина безбожников Проклятого города.
Озверевшие крестьяне и ремесленники отправились громить Проклятый город.
Шамаля предупредил кузнец, прибежавший из Города Мастеров.
Когда Шамаль спросил, почему тот это сделал, кузнец ответил, что берёт грех на душу в благодарность за спасение дочерей.
Салема предупредить не успели --- Марин нашёл его тело брошенным на железный штырь.
Потом Марин увидел сцену на улице --- несколько парней поймали девушку.
Вначале поставили её на колени, заставили целовать Символ Веры, а потом со словами <<Парни, причастим её!>> изнасиловали.
Марин хотел прийти ей на помощь, но Шамаль удержал его --- врагов было больше пятнадцати, и все с оружием.
Шамаль с семьёй и Марином бежали в лес.
Шамаль дал Марину еды, лук, стрелы и верёвку, указал дорогу к скалам и проводил.

\subsection{Талиал}

Маленькие деревушки, в которых люди живут по законам естественности.
Искусственно контролируется рождаемость --- люди сами перестают размножаться, если население превышает некоторую константу.
Никто не боится смерти и болезней, старики и неизлечимо больные уходят умирать в горы, а остальные больные отправляются в отдалённые места к лекарям.
Понятие <<семья>> не закреплено жёстко --- мужчины и женщины находятся в свободных отношениях, детей воспитывает их род, а кормят все, у кого есть еда. 

Марин нашёл драконье гнездо, подстерёг дракона.
Затем, не давая ему говорить, схватил за усы, завязал ему усами пасть и полетел к Луна.

\subsection{Луна}

В Лунных садах было много девушек, которые обхаживали Марина.
Случайно он познакомился с Еленой, которая избегала остальных и его.
Елена собирала вишню, и они разговорились.
Елена рассказала, что на Луне был ещё один человек с Земли, Эдар.
Но его отравила одна из девушек, Аина, когда он её отшил.
Затем, поняв, что она наделала, Аина отравилась сама.
Елена была лучшей подругой Аины.
Елена посоветовала Марину не проявлять ни к кому из девушек предпочтения, чтобы остаться в живых.
Пока они разговаривали, незаметно влюбились друг в друга.

Марин пришёл на встречу с Еленой.
Она сказала, что это погубит их.
Марин предложил бежать.
Елена рассказала ему о лодке Селены.
Они уже спланировали бегство, когда их увидела Ирина.
Марин с Еленой расстались.
Ирина подстерегла Марина на причале, зажала его в кустах и, угрожая оружием, попыталась его склонить к сексу.
Тот отказался.
Тогда Ирина выдохнула Марину в лицо сонную лунную пыль, отчего тот заснул крепким сном, и потащила тело на причал Лунных садов, чтобы сбросить его вниз.
В этот момент подоспела Елена.
Завязался ножевой бой.
Елена пыталась закрыть собой спящего Марина.
В приступе гнева Ирина рассказала, что это она подговорила Аину отравить Эдара за то, что тот посмел не желать её.
Затем Ирина отбросила Елену и сумела-таки столкнуть Марина вниз.
Елена бросилась вслед за ним.

В полёте Елена сбросила платье, чтобы догнать Марина, и с помощью лунной силы выманила из Марина сонную пыль.
Марин очнулся.
У них с собой была верёвка с крючьями, стрелы и лук.
Они сумели поймать крохотный островок и спастись.
Вся лунная сила Елены пошла на то, чтобы пробудить Марина, она потеряла свой лунный свет и стала обычной женщиной средних лет с померкшими глазами.
На островке не было никакой еды, и топливо быстро подошло к концу.
Они уже готовились к смерти, когда их спасли орлы, перенеся на Талиальский остров.
Они думали, что оказались в чужой стране без средств, но вдруг Марин обнаружил в кармане золотую лунную вишенку.
Они с Еленой прорастили её.

\section{Тёплый Хетр}

Жил-был однажды человек по имени Хитрам.
Он держал большой постоялый двор на окраине города.
Он прославился среди жителей города своим весёлым нравом, своими кулинарными шедеврами и необъятным животом.
Зарабатывал Хитрам достаточно, а на выручку ремонтировал свой двор, приглашал всех бродячих музыкантов и устраивал бесплатные ужины, на которые готовил свои самые лучшие блюда.
Люди вначале смеялись над странным улыбчивым толстяком, а потом прониклись к нему любовью.
Он пользовался таким авторитетом, что иногда на его дворе устраивали даже переговоры, и двор считался нейтральной территорией, подобно купеческому Двору.
Женщину Хитрам нашёл себе под стать --- полную, краснощёкую крестьянку, которая была его помощницей и главной ценительницей его творений.

Однажды в город пришли пылерои Предгорий.
После непродолжительной осады они ворвались за стены, убивая жителей и забирая в плен детей.
Улица за улицей переходили к ним в руки, и вскоре оставшиеся в живых отступили за последнюю черту обороны --- в храм.

Хитрам в это время с тяжёлым чувством готовил кушанья у себя дома.
Он не умел держать щит, он не умел обращаться с саблей.
Пылерои тем временем окружили храм, и жителям окраин ничего не оставалось, как отступить в двор Хитрама --- это было единственное хоть сколько-нибудь укреплённое место.

Легенда гласит, что пылерои собирались взять двор с ходу, но толстяк-хозяин вышел к ним навстречу с горшочком похлёбки в руках.
У него дрожали ноги и руки, но он улыбался, как и всегда.
Пылерои замерли, рассматривая странного человека.
Предводитель не спешил нападать, подозревая засаду.

--- Отведайте моей пищи, --- сказал Хитрам на цатроне, --- в этом доме нет воинов, здесь только едоки, музыка, смех и вино.

Один из воинов спустил тетиву --- и стрела воткнулась Хитраму в бедро.
Толстяк вздрогнул и едва не уронил горшочек, но упрямо повторил ту же фразу.

Пролетело ещё десять стрел, но Хитрам не упал, лишь попятился и прислонился к двери, сжимая горячий горшочек в трясущихся пальцах.

--- Отведайте моей пищи, --- повторил Хитрам, --- в этом доме нет воинов, здесь только едоки, музыка, смех и вино.

Предводитель пылероев подошёл к хозяину и вырвал горшочек из его рук.
Попробовал.
Затем в один глоток опустошил горшочек и разбил его вдребезги о камни.

Воины с опаской смотрели на предводителя.
Тот долго стоял и смотрел на толстяка-хозяина, который из последних сил держался на дрожащих ногах.
И вдруг предводитель махнул рукой и направился к храму.
Пылерои тёмной рекой последовали за ним.

Храм пал в ту же ночь, его защитники были перебиты.
Наутро умер от ран Хитрам.
Во всём городе остались в живых лишь те, кто нашёл убежище в его постоялом дворе.

С тех пор Тёплый Двор стал святилищем.
Каждые десять дней его хозяин бесплатно кормит и поит вином всех желающих.
Менестрели со всех краёв земли считают за честь спеть свои песни в его стенах.
За всё время существования там ни разу не обнажалось оружие.
А Тёплый Хетр занял своё место среди лесных духов, став хранителем домашнего очага, котла, сковороды и вертела, покровителем кулинаров, виноделов и толстяков.

Легенда гласит, что через несколько десятков дождей после этих событий, когда город вновь ожил, стражники загнали пылероя-лазутчика.
Тот, не видя путей для спасения, бросился... в Тёплый Двор.
Все воины были единодушны --- лазутчика нужно убить. Но тогдашняя хозяйка Тёплого Двора --- женщина Хитрама, Ситлам ар’Сар --- была непреклонна.
Пылероя оставили в живых, и он ушёл, получив свою порцию пищи.

\section{Удивлённый Лю}

Однажды в городе Кахрахане жил жрец по имени Люситр.
Он слыл человеком странным --- не любил общаться с людьми, всю жизнь провёл в библиотеке, переписывая книги.
Очень часто его видели в окрестностях --- он наблюдал за птицами, звёздами или собирал травы, и с его лица никогда не сходило глуповатое, удивлённое выражение.

В городе Люситра не очень любили --- он казался жителям высокомерным.
Стоило кому-то завести со жрецом разговор, как Люситр, не слушая собеседника, начинал говорить о разных вещах, которые он видел и слышал.

Однажды Люситр растрезвонил по всему городу, что полетит по воздуху, словно птица.
Любопытные собрались на площади перед храмом, и в назначенный час жрец вышел на крышу.
За его спиной было странное треугольное полотнище ткани, растянутое между палочек из лёгкого Дерева Перьев.
Люситр под изумлёнными взглядами людей спрыгнул с крыши храма... и полетел.

Люди кричали в изумлении, а Люситр парил над их головами, поднимаясь всё выше и выше.
Потом жрец направил своё странное приспособление к морю.
Жители города бежали, стараясь не потерять его из вида, но Люситр летел в бескрайний морской простор.
Вскоре он уже казался крохотной яркой точкой, и его ликующий смех затих вдали.

Больше его никто не видел.
Кто-то говорил, что смелый жрец попал в шторм и сломал свои чудесные крылья, кто-то говорил, что он обрёл новый дом где-то на Ките, в землях ноа.
На берегу до сих пор стоит тотемный столб с ликом Удивлённого Лю, и каждый уважающий себя жрец, книжный человек или воин-разведчик считает своим долгом посетить это место --- починить, подкрасить или навязать лишнюю погремушку на этот тотем.
Многие посетители просто пишут свои имена или пожелания людям.

Жрец оставил после себя богатое наследие.
После были обнаружены его записи.
Люситр узнал о лекарственных свойствах многих сорных растений, испытывая их на себе, и начертил множество схем устройств, включая самопишущие перья и конденсатор для книгохранилищ.
Только сейчас люди поняли, о чём пытался говорить с ними Люситр.

--- Какое несчастье, --- говорили они, --- мы могли бы узнать это раньше, но никому и в голову не пришло послушать, что он говорит!

Устройство его крыльев так никто и не узнал.
Кое-кто годы спустя пытался повторить подвиг Люситра, и многие из этих храбрецов разбились насмерть.
Их черепа и неудачные летательные аппараты лежат в крипте рядом с тотемом Удивлённого Лю.

Третий день месяца Согхо стал с тех пор праздником.
Люди надевают бутафорские крылья и танцуют танцы, некоторые, обвязываясь верёвкой, спрыгивают с ритуальных столбов.
И в последний час перед закатом все люди идут на берег, садятся и в молчании ждут возвращения Люситра.
Кто-то верит, что в день, когда Удивлённый Лю прилетит обратно, все люди обретут крылья и смогут летать.

\section{Имя}

--- Тебе не нравится слово <<Скорбящие>>?

--- Разумеется, нет, --- скривился Лу.
--- Надо обладать интересным складом личности, чтобы найти в скорби нечто привлекательное.
Но изменять название я бы не стал.

--- Потому что от названия ничего не зависит?

--- Зависит.
И никакой мистики тут нет.
Слово изменяет того, кто его слышит, а собственные имена нам приходится слышать постоянно.

--- Тогда почему бы ты не стал менять?

--- Даже с этим малопривлекательным названием всё сложилось неплохо.
Я просто отдаю ему должное.
Бездна тебя возьми, Грейс, мы два телльна были адептами Ордена Преисподней!
Не скажу, что это были худшие два телльна моей жизни.

--- Это были единственные два телльна твоей жизни.

--- Вот именно, --- кивнул Лу и сделал неприлично долгую затяжку, словно пытался заново прочувствовать на вкус два телльна существования.

\section{Глупцы и лжецы}

--- Но ведь истинное равновесие Нэша...
Ведь есть множество тех, кто искренне верит в правильность существующего миропорядка.
Что делать с ними?

--- О, не волнуйся, Грейс.
В мире гораздо больше глупцов и лжецов, чем ты думаешь.
И это прекрасно.

--- То есть Вселенная примет новый миропорядок без жертв?

--- Я думаю, что да.
Глупцы встают на сторону убедительного.
Лжецы встают на сторону большинства.

\section{Хитрый план Лу}

--- И в чём же заключается твой великий план? --- насмешливо спросил Грейсвольд.

--- О, всё предельно просто, --- сказал Лусафейру.
--- Однажды мы завербуем в ряды Скорбящих подавляющее большинство.
Собственно, от старого мироустройства останется только скорлупка, внешняя оболочка.
Затем по моей команде все разом снимут маски.
Будет очень смешно.
Особенно посмеются те, которые только что хотели друг друга убить.

--- И ты думаешь, что на этом конфликты будут исчерпаны? --- скептически прищурился технолог.

--- Разумеется, нет, --- отмахнулся Лусафейру.
--- Однако, согласись, гораздо проще решить проблему, если двое --- рядовые агенты Скорбящих, а не два максима --- Ада и Картеля.
У вторых друг к другу гораздо более древние счёты.

--- Личность --- это совокупность ролей в различных объединениях, --- согласился Грейсвольд.

--- А ещё я уповаю на естественное желание здоровой личности жить хорошо.

--- И это не лишённая смысла надежда.
\ml{$0$}
{Но всё-таки --- что, если не выгорит?}
{But after all, what if the plan won't come off?''}

\ml{$0$}
{--- Ну и чёрт с ним.}
{``Fuck it then.''}

--- Я тебя обожаю.
Я бы так не смог, честно.
Всё-таки во мне сидит эта смертельная слабость --- желание, чтобы мои начинания успешно заканчивались.

--- Это хорошее желание.
Но без удовольствия от процесса оно не значит почти ничего.

--- И наоборот тоже.
Какой смысл в незаконченных начинаниях?

--- Конец --- это условность, Грейс.
Воображаемая точка, делящая прямую на два совершенно одинаковых луча.

--- Скажи это своей смерти, когда она к тебе придёт.

--- А ты хорош.

\section{Лошадь}

--- Большую книгу сложно писать, --- объяснил Грейсвольд.

--- О да, --- откликнулся Лу.
 --- Это как руководить или ехать верхом.
Ты, конечно, можешь думать, что ты управляешь лошадью.
Но умная лошадь очень быстро объяснит тебе истинное положение дел.

--- Я не понял, что ты имеешь в виду, --- признался толстяк.

--- Всему своё время.
Твой метод управления немного другой.
Ты просто занимаешься своим делом, и это привлекает к тебе сторонников.
Если же их не будет, ты всё равно будешь заниматься своим делом.

--- Я всегда считал главным тебя.

--- Ты просто признавал полезными мои навыки организации.

\section{Союз Гало и Тахиро}

--- Гало и Тахиро погибли вместе, на одной планете, в одной войне на уничтожение.
Кто знает, чего они могли бы достигнуть, объединившись?

--- Я знаю, --- грустно усмехнулся Лусафейру.
--- Ничего хорошего.
Воины хороши только на войне.
Может, это счастье Вселенной, что два великих стратега-воина нашли друг в друге врагов.

\section{Сеанс одновременной игры}

--- Кстати, сегодня игра будет?

--- Вроде должна.
Двенадцать столов на миллион фигур наши, плюс ещё сто сорок столов по сто тысяч фигур для любителей.
Машина сказала, что вполне вытягивает против нас двести столов, даже в блиц-игре, по микросекунде на ход.
А вот триста уже сложновато.

--- Да она всё равно выиграет опять все сто пятьдесят два стола!
Вопрос только в том, кто выиграет вместе с ней.

--- В прошлый раз она поддавалась --- параллельно с игрой обсчитывала биостатистику для отдела Хараты.

--- Я что-то не заметил!

--- А десять дней назад она продула один стол из двухсот --- Стигма с Мимозой и Ду-Си её обхитрили и заключили мир.
Ду-Си, правда, болтался в игре, как младенец в водах Нила --- делал глупые ходы, отпускал неприличные шуточки и хохотал.
К удивлению всех, это и сработало.
Машину сбил с толку его стиль игры, а Стигма и Мимоза стратегией и тактикой вывели всех троих к победе.

--- А, так вот что они так громко праздновали у себя в отделе!

\section{Копии}

--- Ой, не там они меня ищут, --- захохотал Лу.

--- Твои копии справятся?

--- И я справлюсь, --- кивнул Лу.
--- Я отдохну и накурюсь за них за всех разом.

--- Странно, --- сказал Грейсвольд.
--- Ведь разделение на копии --- древняя как мир идея.
Почему же по-настоящему это получилось только у тебя одного?

--- Видимо, погрязнув во лжи и интригах, только я один могу доверять самому себе.
Кстати, скоро выйдет из игры моя вторая копия в Картеле.
Ты же приготовишь ему тело?

--- Я окажу ему королевский приём.

--- Просто обними его и дай ему покурить.
Это единственное, чего он хотел все эти годы.
Потом мы с ним интегрируемся.
Затем ещё четверо... и Вселенная будет совсем другой.

--- У тебя будет очень много работы.

--- Это будет гораздо более приятная работа.
Зато потом, когда всё уляжется, я просто буду сидеть и заниматься своими делами.
Вот меня всегда удивляла твоя потребность что-то мастерить руками --- машины, планеты, игрушки.
Вдруг это моё призвание, просто у меня не было шанса попробовать?

--- Я научу тебя всему, что знаю, дружище.
Но с планетой не так всё просто.
Это большой проект.
Планетой надо жить.

--- Как я по тебе соскучился, ты не представляешь, --- Лу выпустил большой клуб дыма.
--- Я бы тысячелетиями просто сидел и слушал музыку твоего голоса, твои беседы с Атрисом про планеты и про всё, что вам интересно.

\section{Близнецы}

--- Соперничество абсолютно ничего не поменяло, --- сказал Лу.
--- Судьба в итоге всё расставила по местам.
Гало исполнял приказы, я правил.

--- Но твой демон и демон Гало были идентичны при создании.
Что же сыграло решающую роль?

--- Наши тела, разумеется, --- ухмыльнулся Лу.

--- Ваши тела были братьями-близнецами!

--- Пока они были на стадии одной клетки, они были идентичны, --- пожал плечами стратег.
--- Ну а дальше сработал эффект бабочки и отец, желавший сделать из нас соперников.
Мы разошлись по двум концам нормы реакции.
Пока Гало закалял дух в скалистых пустошах, я выпрашивал пирожки в деревне.
Пока Гало томил себя воздержанием и колол блокаторы полового голода, я мастурбировал в кровати и спал с Айну.
Когда Гало побрил голову, я покрасил волосы перекисью и завил их в кольца.

--- Эйраки потом публично унизил тебя, сбрив тебе волосы, --- припомнил Грейсвольд.
--- Гало заступился за тебя, но Эйраки был непреклонен.
Легион смеялся...

--- А я отрастил и покрасил их снова, --- подтвердил Лу.
--- Пока отец хвалил Гало за стойкость, пока легионеры сами шли за Гало толпами, мне приходилось идти против мнения всего Ордена и искать сторонников по одному.
Я пробуждал их очевидные потребности и учил жить по ним, а не по навязанным системой правилам.
Эти мелочи в итоге и определили судьбу двух демонов.
Гало думал, что принёс великую жертву, что его популярность заслуженна.
Но он и представить не мог, через какое горнило пришлось пройти мне.
Вернее, он это чувствовал и негласно признавал моё старшинство, пока отец не взялся за него серьёзно.
Помнишь ведь, он всё-таки отрастил волосы потом.
И курить начал, несмотря на запреты отца.

--- А потом положил всю жизнь на свой личный бунт против системы, словно пытаясь что-то наверстать, что-то доказать...

--- И умер за этот бунт, который я перерос за пару лет в юности.

Грейсвольд улыбнулся.

--- Под конец он всё же выбрал тебя.

--- Себя, --- поправил Лу.
--- Себя он выбрал.
Всё-таки мы с ним когда-то были идентичны.

\section{Выписка}

Выписка из архивов отдела 100.

По запросу номер (номер скрыт).

Запись номер (номер скрыт).

Анкарьяль Кровавый Шторм и Грейсвольд Каменный Молот проявили высочайший профессионализм при уничтожении пяти кластеров Скорбящих.
Несмотря на явное временное преимущество противника, спецоперация была проведена ими с эффективностью, превышающей ожидаемую на 32\%.
Достоверных признаков связи вышеуказанных демонов с мятежниками не выявлено.

Ряд наблюдателей (имена) отметил, что уничтожение мятежников центурионом Анкарьяль имеет черты актов милосердия.
Комиссия, рассмотрев отчёты, признала, что эти данные вполне укладываются в общую картину личности интерфектора.
Статистически значимой разницы между отношением центуриона к агентам Картеля и агентам Скорбящих не выявлено.

Вердикт: центуриона секунда отдела 100 Анкарьяль рекомендовать к повышению в ранге на два пункта с зачислением в подразделение быстрого реагирования отдела 100, центуриона прима отдела 100 Грейсвольда рекомендовать к повышению в ранге на один пункт с зачислением в подразделение технической обороны отдела 100.
Рекомендуемые исключения из соответствующих рангу полномочий обоих легатов терция отдела 100 (3 исключения) в приложенном документе (номер документа).

Извещение о смерти номер (номер скрыт).

Легат терция Анкарьяль Кровавый Шторм погибла в результате диверсии Картеля при исполнении служебных обязанностей 24.0002.453227, планета Ку-Лань, империя Плеяды, согласно донесению Хуре Зелёный Сад.
Имя легата занесено в хроники славы Ордена Преисподней.

Данная заверенная цифровой подписью копия выписки выдана легату прима отдела 100 Грейсвольду Каменный Молот по личному запросу.

\section{Завет Айну}

\subsubsection{Из архивов Ордена Преисподней}

Послание типа <<завет>>. Архивный номер (скрыт).

Автор: Айну Крыло Удачи

Получатель: Аркадиу Шакал Чрева

\subsubsection{Текст}

<<Я всю жизнь была воином, Аркадиу Люпино.
Когда приходит мир, воин остаётся не у дел, и я искренне рада, что не увижу этого дня.
В тебе же есть задатки не только воина, ты сможешь найти себя и в мирное время.
Поэтому живи.
Я научила тебя всему, чему могла, и ухожу.
Вместе со мной поляжет достаточно наших врагов, и я надеюсь, что тебе будет чуть легче>>.

\subsubsection{Анализ}

Группа AD44, отчёт Иттме Холодный Осколок:

<<В письме содержатся намёки следующего характера:

\begin{enumerate}
\item Сомнение в правильности глобальной стратегии, выбранной Адом (.928)
\item Побуждение Получателя к смене специализации, идущей вразрез с интересами Ада (.951)
\item Указание на Ад как идеологического противника Получателя (.870)>>
\end{enumerate}

\subsubsection{Рекомендации}

Отдел 100, оператор номер (скрыт): <<Рекомендуется к применению протокол №34>>.

\subsubsection{Статус}

\begin{enumerate}
\item Доставка прервана по запросу 3 степени (ссылка на текст запроса).
\item Архивной записи присвоена 4 степень секретности.
\item Архивная запись ассоциирована с досье Автора и Получателя.
\end{enumerate}

\section{Иллюстрации}

--- Ты решил сделать к книге иллюстрации? --- удивилась Анкарьяль.

Я кивнул.

--- Так читателям лучше удастся понять происходящее.

Анкарьяль отобрала у меня компьютер и просмотрела картинки.

--- Ммм.
Ты взял большую часть рисунков Тхарту и добавил кое-что своё.

Я снова кивнул.
Анкарьяль полистала ещё, нахмурилась.

--- А почему нигде нет портрета Чханэ?

--- Она не любила, когда её рисовали, --- пожал я плечами.

--- И что? --- возмутилась Анкарьяль.
--- Нельзя так.
Портреты почти всех героев есть, а вместо одного из центральных --- пустое пятно.
Ну-ка давай рисуй.
Прямо сейчас.

Я снова пожал плечами и принялся рисовать.
Анкарьяль, сделав несколько кругов по комнате, наконец подошла и критически осмотрела рисунок.

--- Очень похоже.
Но она у тебя получилась грустной.

Я задумался.
Да, почему-то я запомнил Чханэ именно такой.

--- И ещё... Девочки не любят, когда их шрамы оказываются на их портретах.

--- Брось эти древние предрассудки.

--- Я не шучу.
Может быть, Чханэ не любила позировать именно из-за шрамов?

--- Нет.
И без шрамов это будет уже не Чханэ, --- отрезал я.
--- Хватит об этом.
Кстати, твой портрет я тоже не нарисовал.

--- О, давай.
Нарисуй меня такой, какой запомнил.

Я задумался и набросал портрет.

--- Эй! --- возмутилась Анкарьяль.

--- А по-моему, очень похоже получилось, --- засмеялся я.
--- Обязательно вставлю этот рисунок в книгу.

--- Только попробуй, я тебе яйца оторву, --- посулилась Анкарьяль и вышла, ударив плечом дверь.
Но не очень сердито.

\section{Наркотик}

\textbf{К Аркадиу пришла Анкарьяль.
Сказала, что у культурологов проблема --- из далёкого мира прибыл разведчик и принёс данные о ритуалах местных племён.
Никто не может понять смысла ритуала.
Аркадиу пошёл с ней.
Они посмотрели ритуал, и Аркадиу подкинул им идею}

--- А здесь что?

--- А здесь Шиамис с командой испытывают новый наркотик.
Давай зайдём, покажу.

Анкарьяль приложила кудрявую голову к двери.
Дверь распахнулась.

Зрелище, которое предстало моим глазам, было не из приятных.
Чистая, хорошо освещённая лаборатория, скучающий демон в одежде врача, сидящий у панели управления.
И четыре капсулы, в которых лежали страшно худые, чёрные человеческие тела.

Я вошёл в лабораторию.
Демон-врач встрепенулся:

--- Анкарьяль.
А ты, как я понимаю, Аркадиу Шакал Чрева?
Красивое тело тебе собрали.
Добро пожаловать в отдел придурков.

Я улыбнулся.

--- Спасибо, Ациоджи.
Постараюсь соответствовать.

--- Привет, Аци, --- Нар улыбнулась и наклонила голову.
--- Что тут у нас?

--- Пока наблюдаем, --- развёл руками Аци.

Живой скелет в одной из капсул с трудом открыл глаза и улыбнулся вымученной страшной улыбкой.

--- Нар, здравствуй.

Анкарьяль подошла к нему.
Я последовал за ней.

--- Привет, Шиамис.
Как ты себя чувствуешь?

--- Ужасно, --- сухие губы скелета едва заметно шевелились, когда он говорил.
--- Этот наркотик...

Скелет заплакал, искривив губы.
Из опалённых глаз выкатилась крохотная слеза и тут же испарилась, оставив на коричневой коже светлую полоску.

--- Они скоро умрут, --- объяснил Аци.
--- Те трое уже в коме, осталось им от силы день.
Шиамис пока ещё разговаривает и даже в ясном сознании, умрёт дней через шесть.
У него чересчур крепкий организм.

--- Ты уж держись, дружище, --- я склонился над умирающим.
--- Скоро всё закончится.

--- Я знаю, я знаю, --- прошептал скелет.
--- Аци, давай следующую дозу.

Врач кивнул, что-то щелкнуло, и по системе полилась прозрачная жидкость.
Скелет закатил глаза, его тело свело судорогой, рот оскалился.
Сознание покинуло живой труп.

--- Им уже делают новые тела, --- шёпотом пояснила Анкарьяль.
--- Когда они заселятся в них, то предоставят полный отчёт о своих ощущениях.

--- Кошмарная работа, --- пробормотал я.

--- Да, похуже некоторых, --- грустно улыбнулся Аци.
--- До сих пор не понимаю, зачем они постоянно на это соглашаются.
Как новый яд или наркотик --- так сразу команда Шиамиса.

--- Кто-то должен, --- заметил я.

Аци хмыкнул.

--- Да они уже сделали для Ада больше, чем весь отдел биохимии, можно было бы и другую работу найти.
Этот наркотик вызывает страшные видения и не менее страшную зависимость.
Линд и Кен-Бит перед комой то умоляли прекратить, то просили ещё, плакали, несли какую-то чушь.
А моё дело --- продолжать и наблюдать за всем этим.
Паршиво всё это.

--- Как использовался этот наркотик?

--- Это особо изощрённый способ казни.
Многие предпочитали покончить жизнь самоубийством после первой же инъекции.

--- Антидот?..

--- \ldots не нашли, --- скривился Аци.
--- Врачи пытались ради интереса восстановить тело Линд --- бесполезно, проще убить.
Химизм изменён кардинально.
За этим наркотиком чувствуется рука Картеля.
Химиограмму записали, надеемся узнать ещё что-нибудь после вскрытия.
Хорошо, что эксперимент подходит к концу.
Нам обещали хороший отпуск.
Ребята освоятся с новыми телами, напишем отчёты и гулять.
Надоела уже эта лаборатория.
<<Жаркие ночи, полные поцелуев...>> --- пропел Аци на языке тоно и нервно засмеялся.

--- Тебе бы не мешало подлечиться, --- заметил я.

--- Да, --- погрустнел Аци.
--- Энергию расходовать нельзя, здесь и так хватает отрицательных эманаций.
У меня система стоит, но она выдохлась, похоже, --- он убрал волосы со лба, показав внедрённый под кожу имплант.
--- Врачи сказали --- пока так, потом мы тебя вмиг восстановим.

--- Тебе заказать еду? --- сочувственно спросила Анкарьяль.

--- Если тебе не трудно, Нар, --- лицо Аци просветлело.
--- Что-нибудь острое или пряное.

Панель управления запищала.

--- О, Линд умерла.
Отлично, --- Аци облегчённо выдохнул и добавил куда-то в сторону панели:
--- Тахар, Линд готова, можешь вскрывать.

Панель утвердительно прорычала.
Демон-канин этажом ниже активировал оцифровку, и тело из крайней капсулы исчезло в голубых искрах.
Анкарьяль тем временем достала из почтовой капсулы пакет и, распечатав, поставила его на столик рядом с врачом.

--- О, рыба в кисло-сладком соусе!
Нар, у тебя определённо есть вкус, --- обрадовался Аци.

Анкарьяль кивнула демону и потащила меня к двери.

\section{Совесть и репутация}

Вспомнился первый год после возвращения с Тра-Ренкхаля.
Дверь засигналила, и я впустил в комнату грустного демона.

--- Аркадиу.

--- Минь, здравствуй.

--- Здравствую.
Я принёс данные, которые ты просил.

--- Мог бы и переслать по сети, незачем было самому бегать, --- улыбнулся я.

--- Тут есть некоторые... сложности, поэтому я решил передать тебе лично.

--- Рассказывай.

Минь положил передо мной проектор.
Я включил его.
Выключил.

--- Так значит, это точно?

--- Коэффициент более 0,95.
Отдел аналитики подтверждает.

--- Ты задействовал аналитиков? --- поморщился я.
--- Не надо было беспокоить их по такой ерунде.

--- Аркадиу, они годами занимаются скучными вещами.
А тут случай действительно интересный.
Даже Сир подключился, хотя он просто приходил к ним поесть.
Да и потом, это не такая уж и ерунда...

--- А в чём сложности?

Демон оглянулся.
В воздухе материализовались две летающие шарообразные машины.
Выглядели они достаточно устрашающе --- набор приёмных антенн, щупы, рецепторы, завершала экипировку мощная волновая пушка.
Отдел 100 --- контрразведка.
В тот же момент включился глушитель сигналов.
От наступившей тишины на миг заложило уши.

--- Привет, ребята, --- я вежливо помахал машинам, зная, что они не ответят.
Служба.
--- Что случилось?

Минь ответил за них:

--- По ходу дела выяснилось, что у нас в пяти базах данных находится дезинформация.
То, что я передаю тебе --- результат косвенных вычислений.

--- Агенты Картеля?

--- Отдел 100, --- Минь махнул на молчаливых роботов, --- уже занимается этим.
Аналитиков пока изолировали, нас с тобой, как видишь, тоже собираются.

Я кивнул.
Это была стандартная проверка.
После разговора роботы должны были увести меня в отдел на полный анализ.
Сталкиваться с контрразведкой было не очень приятно, но я знал, что туда берут самых лучших --- тех, кто не повторяет ошибок.
Этих демонов можно по праву назвать незримым щитом Ада.

Минь опустил голову.

--- Я уже почти пол-телльна работаю в безопасном отделе архива, Аркадиу.
Для меня это жестокий удар.
Я не думал, что информацию из безопасного отдела можно обернуть против нас таким образом.
Скорее всего, архивы закроют и подвергнут реорганизации.

--- Что говорилось в базах данных? --- спросил я.

--- Согласно базам данных, настоящее имя этого Атриса --- Ковнелий Фиктовий Саз.
Урождённый человек, преобразован ещё во времена Союза Воронёной Стали.
Он был под подозрением --- формально держал нейтралитет, но сотрудничал с Картелем.
После переворота на Сцелае сбежал и с тех пор ошивается в районе Тукана, девятнадцать зарегистрированных контактов с агентами Ада.
А тут выходит, что он...

--- \ldots что он действительно тот самый Добрый бог, Безымянный, демиург Тра-Ренкхаля, изгнанный узурпатором Эйраки, --- закончил я за него.
Последние кусочки картины встали на свои места.
--- Откуда взялся Безымянный?
Как его зовут?

--- Нигде об этом ни слова. В базах отмечено, что демиург Тра-Ренкхаля --- Хатрафель Безумный.
Ваша команда сообщила, что нгвсо почитают Безымянного.
С этого несоответствия ребята за пару часов распутали всю историю.
Сам понимаешь, если бы вас отправили на поиски демиурга, а не на борьбу с Безумным...

--- Понимаю, --- кивнул я.

--- Нет, ты только подумай!
Ведь отчёты об освоении планет тщательнейшим образом...

--- А что насчёт этого Ковнелия?

--- А...
У него приличная биография, построенная на данных погибших демонов.
Спрашивать, существовал ли Ковнелий на самом деле, разумеется, уже не у кого.
Агент Картеля --- кто бы он ни был --- постарался на славу, пролез где только можно.
Я не совсем понимаю смысл этой...

--- Картель опасался, что мы выйдем на Безымянного, и решил подстраховаться, --- предположил я.
--- Найти демиурга на его собственной планете они не могли --- тот продумал систему маскировки.
Вычеркнуть его из наших баз --- чересчур подозрительно, а придумать ему липовую неблаговидную биографию --- вполне себе хороший ход.
Даже если бы мы его встретили --- отправили бы <<на отдых>> как неблагонадёжного.
А слепое вторжение на Тра-Ренкхаль закончилось бы резнёй.

--- Вы молодцы, ребята, --- заметил Минь.

--- Ага, мы, --- саркастически проворчал я.
--- Битву за Тра-Ренкхаль мы выиграли благодаря невероятной случайности и находчивости Грейса.
Если кто и молодец, так это он.
Я боюсь даже предположить, сколько военной силы мы бы потеряли из-за этой подсадной утки.
Лусафейру всё-таки гений.
Он, похоже, подозревал, что тут не всё чисто...

Один из роботов выключил глушитель и впервые заговорил приятным женским голосом:

--- Аркадиу Шакал Чрева, Минь Орлиная Заря, прошу вас проследовать с нами на станцию С9A0.

За время разговора, я знал, они полностью проверили меня на предмет подслушивающих устройств, маячков, молекулярных механизмов регистрации и прочей шпионской техники, а также провели всесторонний анализ моей личности.
Выключенный глушитель означал, что я не представляю опасности.
По крайней мере пока.

Я улыбнулся и кивнул агентам.
Роботы растворились в воздухе.

--- Пошли, дружище, --- похлопал я по спине честного архивариуса.
--- У нас с тобой совесть чиста.

\section{Братья по разуму}

--- Грейс, у меня проблема, --- начал я.
--- Не могу найти понятную информацию по Ветвям Звезды.
Это форма жизни, но при обучении мне намекнули, что Звезда не в моей компетенции и занимаются ею другие биологи.
Не мог бы ты рассказать о них?

--- А, --- откликнулся технолог.
--- Хм-хм.
Ветви Звезды.

--- Может быть, это секретная информация и не стоит её?..

--- Нет-нет, --- перебил меня Грейсвольд.
--- Это информация общедоступная, но без интерпретации понять её сложно.
Слушай, попробую объяснить.

Я схватил компьютер и настроил на запись.

--- Как ты знаешь, наша область Вселенной предположительно является <<мёртвой зоной>> --- жизнь здесь встречается довольно редко.
Ветви Звезды --- это сапиенты с планеты 1-34, второй известной планеты со стабильной самозародившейся сапиентной жизнью.
В источниках, предназначенных для Земли, планеты Звезды обозначаются цифрами.
Их способ общения --- назовём его <<языком>> --- кардинально отличается от языков Ветвей Земли.
Он полностью химический, с помощью полимеров и низкомолекулярных веществ.
У Земли пообщаться со Звездой просто так не получится.

--- А что они собой представляют?
Кажется, их химический состав...

--- Да, химический состав.
Углерод, кислород, кремний, азот, сера, в целом то же самое, но на другой лад.
Поглощают метан, выдыхают углекислый газ.
Оптимально их существование при температуре кипения этанола и давлении, в 2.8 раз превышающем земное.
Своеобразное строение <<тела>>, назовём его так.
Сложно установить, где заканчивается одно тело и начинается другое.

--- Я так понял, что это нечто, похожее на мицелий?

--- Сложнее, --- покачал головой Грейс.
--- В мицелии есть отдельные клетки, а тут... ммм... многослойный синцитий.
У них есть <<города>> --- губчатый скелет из кварца с прослойками биоколлоида, занимающий огромные пространства...

--- И это существо разумное? --- удивился я.

--- Как ни странно, --- ответил Грейс.
--- У них есть технология, они вышли в космос и заселяют экзопланеты.
Вряд ли столько, сколько Земля, но ненамного меньше.

--- Понятно.
И они испускают при угнетении плюс-эманации, а при благоденствии --- минус?

--- А, да.
Звезда и хоргеты.
Интересный вопрос.
Да, 1-34 --- одна из самых мощных баз Картеля, именно поэтому мы знаем про Звезду не так много.
После поражения в Развязке Десяти Звёзд Картель отступил на территорию Ветвей Звезды и закрепился там.
Ордену Преисподней на планетах Звезды действовать так же трудно, как Картелю на территории Земли.

--- Я прочитал мнение, что впоследствии война между Картелем и Адом может перерасти в войну между Землёй и Звездой.

--- Я с этим согласен.
Звезда постепенно изменяет свой достаточно примитивный химический способ общения на более быстрый, волновой.
В будущем можно ожидать появление не только киборгов, основанных на биологии Звезды, но и совершенно новых существ.
Однако есть интересный нюанс.
Молекулярный механизм излучения эманаций, связанный с системой ответа на раздражение, введён в Звезду искусственно.
Каким образом вышло так, что эволюция живых существ вызывает минус-завихрения, ещё предстоит выяснить.
Также найдены нуль-штаммы Ветвей Звезды, то есть не излучающие эманации вообще.
Другой интересный нюанс --- молекулярный механизм построен по неизвестной ранее схеме, и хоргеты Картеля отрицают свою к этому причастность.

Я обомлел.

--- То есть?..

--- Да, ты правильно понял.
Возможно, это сделали хоргеты, созданные Ветвями Звезды.
Или вообще другие.
Кто знает, может, и сама Звезда --- искусственно выведенная хоргетами форма жизни?
Судьба этих братьев по разуму неизвестна.
Были ли они уничтожены в войне, подобной этой?
Примкнули ли они к Картелю?
Был ли ими выведен микоргет, ушедший в другие Вселенные?
Предстоит разобраться.

--- И это значит, что у нас появилась новая проблема.

--- Да, новая возможность и новая проблема.
Возможность использовать в своих целях Ветви Звезды и проблема контроля генофонда Ветвей Земли.
Вряд ли Картель упустит шанс вывести, например, минус-людей.
Если те, другие хоргеты примкнули к ним --- в технологическом плане мы на шаг позади.

Я обхватил голову руками.

--- А всё-таки, что мешает создать стабильный источник эманаций и оставить сапиентов в покое?
Сколько проблем было бы решено!

Грейсвольд вздохнул.

--- Да, Аркадиу.
Проблема ключевая для процветания нашего вида, но ею занимается катастрофически мало демонов.
Давно ты что-то слышал о разработке микоргета?
И я тоже.
Война, видимо, Ордену нужнее.

\section{Язык Эй}

Когда Красный Картель и Орден Преисподней начали войну за обитаемую Вселенную, остро встал вопрос коммуникации --- как между демонами, так и с сапиентами.

Картель и Ад и прежде использовали специальные языки для сражений (т.н. боевые языки, отличающиеся простотой и краткостью) и для передачи информации (шпионские языки, сложные для расшифровки).
Боевой и шпионский языки Ада, в частности, были созданы на основе языка сохтид --- самого распространённого человеческого языка на Преисподней.
В силу некоторой оторванности Картеля от цивилизаций сапиентов минус-демоны создали ещё один тип языка --- усечённый (sekta-lingu), с помощью которого демоны коммуницировали с подвластными им сапиентами.

У всех вышеперечисленных языков был один большой недостаток --- они были придуманы людьми в процессе эволюции и, как и любой живой организм, несли в себе огромный груз отрицательных мутаций.
Поэтому обе организации поставили задачу --- разработать единый язык, избавленный от пережитков прошлого.

Картель первым справился с поставленной задачей.
После долгих дискуссий было решено оставить секта-лингу для общения с сапиентами, а в качестве боевого и шпионского языков использовать новосозданный Чи.
Демоны понимали, что главное оружие криптологов --- это логика.
Вследствие этого между корнями и словоформами языка Чи не было никакой логической связи.
Фактически одно и то же слово из предложения в предложение менялось до неузнаваемости.
Словоформы придумывали сто тридцать генераторов случайных чисел, разработанных на основе человеческого мозга.
Словари языка Чи были строжайше засекречены, а для демонов, которые им пользовались, были созданы специальные системы защиты, мгновенно уничтожавшие языковой сектор в памяти при попытке проникнуть в него или выдать его постороннему.

Первый раунд в этой битве был блестяще выигран учёными Картеля.
Расшифровать Чи даже на процент не удалось до сих пор, несмотря на усилия разведки, криптологов и аналитиков.
Данные, приведённые здесь --- это, увы, почти всё, что Ад на сегодняшний день знает об этом таинственном языке.

Учёные Ада такими результатами похвастаться не могли.
В архив отправляли одну версию за другой --- какие-то браковались культурологами, какие-то криптологами.
Возможно, что это затянулось бы ещё на неопределённое время, если бы не вмешался один из старейших демонов, который служил Ордену Преисподней аж с момента его создания.

Ликан Безрукий.
Урождённый человек, один из жителей древней Преисподней.
Не совсем ясны обстоятельства, по которым он стал демоном.
Возможно, об этом знает кто-то из старожилов --- Лусафейру или Грейсвольд, но они не распространяются на эту тему.

--- Грейс, расскажи про Ликана Безрукого, ты ведь его знал.

--- Ааа, Ликан.
Да-да-да.
Достойный был демон, достойный.

На этом разговор обычно заканчивается.

Ликан Безрукий отказался от коррекции личности, которую проходят все урождённые люди.
Вы можете себе представить, что творится в голове у человека, который вынужден прожить несколько телльнов, и Ликан, как видно, потихоньку начал сходить с ума.
По словам очевидцев, общаться с ним было форменным наказанием.
Впрочем, своё дело (а работал он в Аду аналитиком) Ликан знал блестяще, занимался охотно и увлечённо, и причин отстранять его не было.

Узнав, что отдел 214 занимается разработкой нового языка, Ликан ходатайствовал о подключении его к работе, так как всю жизнь питал к лингвистике известную слабость.
К тому времени 214, который терпел одну неудачу за другой, превратился в закрытый клуб, и ему было отказано.
Ликан совершенно по-человечески обиделся и в рекордные сроки (всего за год) разработал синтаксис и морфологию языка Эй.
Ещё год ушёл у него на наработку и сортировку словаря, и вскоре старый демон попросил Совет устроить открытое (sic!) слушание его доклада.

Весь Ад с интересом следил за событиями.
Все знали, что Ликан собирается представить новый шпионский язык, и у большинства зрел вполне закономерный вопрос --- не сошёл ли он с ума, устраивая открытое слушание своего доклада?
Подобные разработки засекречивались, едва успев появиться на свет.

Вот отрывок из его выступления:

<<... Язык Эй представляет собой логичную, стройную систему.
Слова максимально короткие, ёмкие, лишены избыточности и возможности двойного толкования.
Язык подходит для изучения любым существам --- как демонам, так и самым примитивным сапиентам...>>

Всё это очень хорошо, скажете вы, но зачем сдался нам шпионский язык, который понятен и лёгок в изучении для всех?
Я таких вам десяток настрогаю, только скажите.
В том же смысле высказались и члены Совета --- разумеется, используя другие выражения.

Но старый демон не успокаивался:

<<... Также я предлагаю... нет, требую, чтобы словари языка Эй и прочая информация по нему находились в открытом доступе>>.

Слушатели всё больше утверждались во мнении, что Ликан сошёл с ума.
До тех пор, пока он не сказал самого главного.

<<В основе языка Эй лежат шестнадцать цифр, из которых и строятся слова.
Фонетические и графические правила языка Эй устанавливаются донором и акцептором информации в соответствии с их анатомическими особенностями, органами восприятия\ldots и прочими важными условиями коммуникации>>.

После этих слов аудитория замерла, а потом взорвалась проявлениями восторга.
Люди засвистели и зааплодировали, кани завыли, хлопая руками по бёдрам, лишённые голоса замахали конечностями.
Конечно же, так отреагировали в основном культурологи, находящиеся в неизменённых сапиентных телах, но так получилось, что они выразили мнение абсолютного большинства аудитории.

В тот же день, после символической проверки трудов Ликана аналитиками и внесения столь же символических правок, Совет единогласно утвердил язык Эй, таблицу 00, как официальный язык Ада.

У читателя, разумеется, возникнет вполне закономерный вопрос.
Фактически исходный язык Эй подвергается шифрованию по типу <<кодовой книги>>, которое сохраняет статистические особенности текста и довольно легко расшифровывается.
На это особенно указывали демоны отдела 214.
В чём же его преимущество?
Ответ прост --- таблицы подразумевают одновременный поток информации по нескольким каналам.
Грубо говоря, не знакомый с таблицей сторонний наблюдатель не знает, двинул ли я плечом из-за случайного комара или это символ, входящий в общий поток.
В некоторых случаях сообщение сложно отличить даже от обычного шума.
Расшифровать длинное сообщение способен лишь опытный визор, а короткое, которое может содержать в том числе и следующую таблицу, практически не поддаётся расшифровке.
Эй оказался идеальным сочетанием простоты, надёжности и практичности.

Разумеется, Картель тут же узнал о докладе.
Но, увы, это им не помогло.
Таблицы правил множились в геометрической прогрессии.
Фактически у любых двух демонов, которые общаются между собой, могла быть своя таблица правил.
Появились специальные таблицы для разных видов, рас и народностей, с учётом их способов коммуникации --- жестовые, голосовые, мимические, цветовые, музыкальные, шумовые и смешанные, а также множество алфавитов.
Появилась Мирквудская классификация --- попытка систематизировать таблицы.
При этом синтаксис, морфология и словарь языка Эй оставались неизменными.

Картель умел признавать поражение.
Вскоре его агенты выкрали данные о языке, хотя лично мне кажется, что они их просто взяли, как мы берём книги из библиотеки.
Вряд ли кто-то так уж попытался им помешать.
Год спустя у них были свои таблицы правил и свой форк Миквудской классификации, различающийся, по самым скромным оценкам, на миллион триста тысяч таблиц.
Язык Чи благополучно отправился в архив нерассекреченным.

Наверное, это самый большой парадокс в истории: вся обитаемая Вселенная внезапно заговорила на одном языке --- и при этом два его носителя могли не понять друг друга.

Для ознакомления привожу таблицу правил B0, которой пользуются девяносто девять процентов демонов-людей Ада для повседневного общения --- одну из самых простых.
По Мирквуду она является двухполосной неоригинальной СЧФ-таблицей.

$L$ --- длина слова

$P$ --- ключ позиций

$P_x$ --- чтение позиции

$S$ --- символ языка Эй

\[L = 1: P = 0\]
\[L = 2: P = 10\]
\[L = 3: P = 101\]
\[L = 4: P = 1010\]
\[L = 5: P = 10101\]
\[L = 6: P = 101010\]
\[L = 7: P = 1010101\]
\[L = 8: P = 10101010\]
\[S = 0: P_0 = a, P_1 = l\]
\[S = 1: P_0 = o, P_1 = m\]
\[S = 2: P_0 = u, P_1 = n\]
\[S = 3: P_0 = e, P_1 = r\]
\[S = 4: P_0 = ai, P_1 = z\]
\[S = 5: P_0 = oi, P_1 = c\]
\[S = 6: P_0 = ui, P_1 = j\]
\[S = 7: P_0 = ei, P_1 = s\]
\[S = 8: P_0 = ia, P_1 = v\]
\[S = 9: P_0 = io, P_1 = f\]
\[S = A: P_0 = iu, P_1 = g\]
\[S = B: P_0 = i, P_1 = k\]
\[S = C: P_0 = ah, P_1 = d\]
\[S = D: P_0 = oh, P_1 = t\]
\[S = E: P_0 = uh, P_1 = b\]
\[S = F: P_0 = eh, P_1 = p\]

Таким образом, слово <<химическое соединение>> --- A3F8 --- согласно этой таблице будет читаться как gepia.

\section{Экскурс в историю}

Я думаю, что моим читателям любопытно будет узнать, с чего всё началось.

Материнской планетой Ветвей Земли была Древняя Земля.
Нам мало что известно об истории развития первых людей до эпохи Последней Войны, 1914--2032 годы Господина (sl: Anni Dominio).
Предположительно эти существа, получив преимущество перед остальными видами в виде относительно развитого интеллекта, довольно быстро заселили планету, покрыв её городами и сетью путей сообщения.
Учёные Ада склоняются к мысли, что тогда ещё не было деления на расы и подвиды --- генетический дрейф и адаптивная изменчивость в развитом технократическом обществе маловероятны.
Известно, что за двести лет людская популяция путём генной инженерии избавилась от груза мутаций, стабилизировала рождаемость сообразно ресурсам планеты, и взгляды людей впервые обратились к далёким мирам.

Первые космические корабли были весьма ненадёжны.
Путь до пригодных для жизни планет занимал тысячи парсак, а скорости выше световых были недоступны.
Десятки тысяч добровольцев отправлялись в далёкие путешествия, не зная, что ждёт их в пути.
Цивилизация медленно, но верно переходила в стадию застоя, и этот переход завершился бы, но люди со свойственной этим существам оригинальностью нашли выход из сложившейся ситуации.

Примерно в 315 году от Последней Войны (2347 год Господина) люди открыли омега-поле.
До этого момента Вселенной Ветвей Земли была лишь так называемая фотонная (электромагнитная) связка --- прямо или косвенно связанные с фотоном взаимодействия.
О существовании прочих квантовых связок, которые назывались <<параллельными Вселенными>>, люди догадывались, но обнаружить их существование не могли.

Омега-поле, или поле Кохани"--~Вейерманна (ПКВ), возбуждается от присутствия наблюдателя --- сознания.
Всякая система обладает сознанием и способна наблюдать, но наибольшую напряжённость поля создаёт именно сапиентное сознание, заключённое в относительно небольшой, но тем не менее сложный мозг.
Подавляющее число систем воздействует на ПКВ одинаковым образом --- эволюция системы вызывает положительное искривление, инволюция --- отрицательное.
Но известны и исключения (Ветви Звезды).

ПКВ косвенно воздействует на огромное число связок.
Именно его воздействием объясняется некоторый элемент случайности в струнных взаимодействиях.
Это не истинная случайность, а результат наложения на ПКВ эффектов огромного количества вложенных друг в друга Вселенных, построенных на разных связках.
Физики Древней Земли даже называли ПКВ <<океаном случайности>>.

Открытие омега-поля было последним аккордом создания в физике Теории всего, давшей взрывоподобное развитие прочих наук.
Теория всего, несмотря на присутствие в ней практически недоказуемых гипотез, применима и в настоящее время.
В частности, именно на её основе созданы устройства для оцифровки и программы взаимодействия хоргета с окружающим миром.

Путём экспериментов с первичным полем была создана anima --- первый примитивный хоргет.
Хоргет --- это стабильная сингулярность ПКВ, расположенная перпендикулярно связкам.
Его протяжённость во Вселенной Ветвей Земли --- не более диаметра протона.
Он способен благодаря накопленной масс-энергии воздействовать на материю --- переформировывать струны, изменять их частоту.
Отдельно следует выделить способность хоргета преодолевать световой барьер (перемещение по так называемому каскаду спутанных фотонов (КСФ), или фотонному лабиринту (ФЛ), возможное при соответствующих тратах масс-энергии).

Настала так называемая godage --- Эпоха богов.
Сапиентам Земли больше не было нужды снаряжать дорогостоящие экспедиции к пригодным для жизни планетам --- хоргеты заряжались масс-энергией и посылались в направлении нового мира, собирая информацию и целенаправленно изменяя климат.
Когда медленные космические корабли с первыми колонистами достигали новой планеты, она уже была полностью пригодна для жизни.

Первоначально боги были запрограммированы на приведение планеты в пригодный для жизни вид и последующее самоуничтожение.
Но не все хоргеты следовали плану, который вложили в них учёные --- программа омега-сингулярности обнаружила способность к быстрой мутации.
Фрактальное дублирование кода, которым техники попытались компенсировать мутации, привело к ещё более серьёзным последствиям --- направленной эволюции.

Боги приобретали нечто похожее на инстинкт самосохранения, а последующие мутации приводили к программам самостоятельного получения масс-энергии.
Иногда по прибытии земляне оказывались в не похожем на Землю месте, с чуждыми, порой опасными формами жизни, порой разумными и даже превосходящими интеллектом самих землян.
Эти существа часто организовывали культы создавших их богов, обеспечивая демиургов масс-энергией.
Во многих подобных мирах экспедиции гибли, и туда больше никогда не ступала нога сапиентов Земли --- боги прекрасно понимали опасность, исходящую от их создателей.
Но кое-где пришельцы выжили и приспособились к трудным условиям.

Связь между планетами всегда была серьёзной проблемой.
Корабли, разумеется, шли в один конец, их путь порой занимал сотни и тысячи лет.
Какое-то время связующим звеном была Древняя Земля --- специальные хоргеты сновали по Вселенной, принося колониям новости с материнской планеты и собирая информацию о колонистах.
Но вот то одна, то другая колония стали замолкать в силу разных причин.
Цивилизации гибли в результате катаклизмов, нападений демиургов и Девиантных Ветвей, сапиенты дичали и теряли технологические знания.
Наконец, спустя двадцать три тысячи лет, из-за <<бродячих камней>> замолкли обитаемые планеты Солнечной системы --- Древняя Земля, Марс и искусственно созданная Диана.
Каждая колония осталась предоставлена сама себе\FM.
\FA{
В настоящее время на Древней Земле существуют 4 вида с уровнем развития в районе 90, относящиеся к высшим приматам и псовым, но не являющиеся потомками первых людей и первых кани.
Большая часть информационного наследия первых людей была сохранена и задокументирована хоргетами, впоследствии примкнувшими к Вечности, Союзу Воронёной стали и Ордену Преисподней.
}

Вот тогда-то, на закате единого сапиентного общества, ведущей силой стали мутировавшие (девиантные) хоргеты.
У них было всё --- накопленная за историю человечества информация, способность к перемещению между планетами и практически неограниченное время существования.
Многие оставили участь богов отдельно взятого мира и стали путешествовать по планетам, инкарнируясь в тела сапиентов и проживая одну смертную жизнь за другой.
Доподлинно не известно, когда и где появился интернационализм <<asoga>>, в переводе означающий <<демон>>, но считается, что именно так в начале времён бродячие (мобильные) хоргеты называли друг друга.
Эпоха богов медленно, но верно подходила к концу.
Забрезжил рассвет asogeite --- Эпохи демонов.

Во время существования в телах сапиентов хоргеты имели возможность получать новую информацию и масс-энергию почти без трат со своей стороны.
Благодаря способности воздействовать на материю инкарнированные хоргеты всегда имели высокое положение в обществе: становились правителями, организовывали религиозные культы, что в свою очередь обеспечивало их постоянным потоком масс-энергии.
Но число хоргетов росло, и сейхмар уже не могли обеспечить питание всем.
Это послужило причиной конфликта между положительно и отрицательно питающимися хоргетами, которые были заинтересованы в получении взаимоисключающих ресурсов от сейхмар.
Впоследствии этот конфликт привёл к появлению крупных союзов демонов и военным столкновениям между ними.

\textspace

Возможно, вам сложно понять, как начинает осознавать себя существо, отличное от вас.
Я думаю, что в этом плане особой разницы между сапиентами и хоргетами нет, за исключением того, что у сапиентов почти всегда есть возможность развиваться в комфортном окружении себе подобных.
У первых богов такой роскоши, увы, не было.

Самой распространённой ошибкой создателей богов было то, что они давали детищу чересчур много свободы действий, увеличивали его интеллект, но при этом относились к нему, как к инструменту.
Первых богов совершенно лишили той поддержки, которая необходима маленьким разумным существам.
Таким образом, например, появился Грейсвольд Каменный Молот, тогда известный как Griswold K-28, творение Лаборатории Дж.\,Грисвольда и мой лучший друг.

Сам Грейс утверждает, что после создания он сразу закричал <<Я родился!>> и принялся радостно носиться по коллайдеру.
Но я знаю его достаточно и понимаю, что всё это выдумки.
Вряд ли он ощущал что-то кроме акбаса.

Разумеется, он не помнит о своём создании почти ничего.
В его программе не было предусмотрено направленное накопление опыта.
Но отдельные фрагменты сохранились.
В частности, Грейс после долгого анализа вспомнил, кто такой этот Дж.\,Грисвольд, и даже нарисовал его портрет.
Ничего особенного --- обычный мужчина эпохи Богов, страдающий выпадением волос на голове и старческой дальнозоркостью, компенсированной толстыми стеклянными линзами.
Единственное, что привело культурологов в полный восторг --- это огромные седые усы, которые вы сможете найти разве что на планетах вроде Мороза, где волосы на лице, как у мужчин, так и у женщин --- не роскошь, а насущная необходимость.

\chapter{Приложения}

\section{Структура текста в языке сели}

Структуры, которые образует распараллеленный абзац:

\begin{itemize}
\item Развилка: разветвление потока на два равнозначных.
Чаще всего встречается в виде коротких перечислений или дихотомии.
\item Ответвление: отделение второстепенного потока от главного.
В тупом конце ответвления может помещаться примечание или определение термина.
\item Тупой конец: поток, завершающийся выводом или недосказанностью.
\item Острый конец: поток, завершающийся вопросом. По традиции, потоки с острыми концами всегда пишутся в нижней части абзаца.
\item Слияние: с тупым концом --- некая информация, которая относится к обоим потокам, либо общий вывод, с острым концом --- проблема (вопрос), которую задают потоки.
\end{itemize}

\section{Методология терраформирования}

\footnote{Ликан Безрукий. <<История Ордена Преисподней. Пособия для сапиентов, подлежащих оцифровке>>. Гл. 16, не прошедшая цензуру}

Один из вопросов, который до сих пор будоражит умы учёных, следующий: каким образом первые люди за время существования их цивилизации получили количество масс-энергии, достаточное для преобразования почти сорока тысяч планет в обитаемой Вселенной?

Ресурсы планеты не бесконечны.
Обычная, <<дикая>> планета способна давать 63 гигаяо в год.
Мелиорированная планета способна дать в восемнадцать-двадцать раз больше.
Тем не менее, даже энергии, собранной с планеты за тысячу стандартных лет, недостаточно для терраформирования методами, находящимися в распоряжении Ада и Картеля.
Древние земляне, тем не менее, оживляли в среднем около двух планет за стандартный год.
Было бы логично предположить, что они грамотно расходовали имеющуюся в их распоряжении масс-энергию.

В первую очередь это касается статистических методов.
Согласно некоторым данным, на орбите Древнего Солнца вращался омега-кластер, вычислительные мощности которого были сравнимы с таковыми у кластеров Ордена Преисподней.
Древние люди в течение нескольких десятилетий собирали информацию о строении планеты или пылевого облака, прежде чем приступать к терраформированию.
Однако после изучения достаточно было провести серию весьма тонких вмешательств в структуру объекта, чтобы планета сама двинулась нужным путём.
Это позволяло снизить затраты масс-энергии в тысячи, миллионы и десятки миллионов раз.

К сожалению, большая часть древней методолигии, получившей звучное название <<Тёмная статистика>>, в настоящее время утеряна.
Единственным источником могли бы стать боги, созданные первыми людьми.
Но, как известно, боги подвергались планомерному уничтожению в течение всего времени существования Ада и Картеля --- в рамках планетарной мелиорации, направленной на увеличение продуктивности планеты.

Подобно дикарям, мы сжигали книги, чтобы костёр горел жарче.

\section{Демократическая деспотия}

Красный Картель является союзом с уникальным государственым строем, который называется демократической деспотией.
У Картеля есть собственный свод чрезвычайно прогрессивных законов (во многом гораздо более демократичных, чем законы Ордена Преисподней), но действуют законы только в одном случае --- если за конкретное применение проголосовало более 1\% демонов.

\subsection{Ювенальная юстиция}

Ювеналы --- молодые демоны, проходящие отбор в связках --- в значительной степени поражены в правах по сравнению с полноправными членами Картеля.
У них отсутствует право голоса, право на свободное передвижение и право на неприкосновенность.
Жизнь ювеналов регламентируется Зелёным кодексом.

Смертность среди ювеналов может достигать 80--98\% --- многие отбраковываются на стадии обучения.
Кроме того, чтобы пережить отбор, ювеналу приходится не только доказать свою профессиональную пригодность, но и способность отстаивать свои границы.
В частности, несмотря на наличие строжайших правил, минимизирующих возможность причинения друг другу вреда и формирования организованных группировок, очень часто на момент вступления в права ювеналы имеют опыт уничтожения других демонов.
Известны случаи, когда ювеналы из-за систематических издевательств уничтожали старших, уже вступивших в права собратьев --- и оставались безнаказанными, так как юрисдикция Красного кодекса на них не распространяется.

\subsection{Суд}

Любое преступление в Картеле по умолчанию карается смертной казнью.
Красный Кодекс, предусматривающий иные наказания, действует в случае, если обвиняемый найдёт способ предать дело огласке и заручиться поддержкой некоторого количества демонов:

\begin{itemize}
\item В случае легионера требуется поддержка 12\% сослуживцев (в стандартном подразделении в 40 демонов);
\item В случае учёного --- 4\% коллег лаборатории.
\end{itemize}

Если указанное количество демонов считает, что обвиняемого следует судить по Кодексу, а не казнить --- дело передаётся в суд.
Поэтому совершивший преступление очень часто вынужден сообщить о нём сразу после совершения значительной части своего окружения.

Решение суда по Кодексу теряет силу, если за отмену проголосовало:

\begin{itemize}
\item В случае военного --- 5\% подразделений;
\item В случае учёного --- 2\% лабораторий.
\end{itemize}

В случае, если против конкретного правопримененительного случая выступило более 13\% демонов во всей организации, в Красный Кодекс после референдума вносится соответствующая правка.

\subsection{Законодательный орган}

Разработкой и упорядочением законов в Картеле занимается Высший Совет, в который избираются представители от каждого клана.

Также раз в некоторый промежуток времени в Красный Кодекс вносятся антидемократические статьи, которые имеют расплывчатую формулировку или прямо противоречат законам.
Разработкой этих законов занят специальный орган --- Четверо Дураков.
Несмотря на название, в состав Четверых выбирают самых уважаемых, креативных и мыслящих нестандартно стратегов.
В частности, Гало Кровавый Знак занимал пост Третьего Дурака в течение шестидесяти оборотов.

Рядовые демоны Картеля чаще всего не знают, был ли закон принят Высшим Советом или Четверыми.
Информация об этом засекречена.
Обычной реакцией на принятие законов Четверых считается забастовка.
Однако бывали случаи, когда законы Дураков принимались обществом и становились частью свода.

К Дуракам очень часто обращаются с просьбой выдвинуть законопроект, само обсуждение которого является незаконным.
Особенно это касается пацифистских инициатив и попыток дать права сейхмар.
Ланс-нат Алмаз лоббировал принятие закона, ограничивающего полномочия Дураков;
Гало Кровавый Знак и его соратник, Мист Сигнальный Дым, были смещены с поста Дураков за последовательное проведение пацифистской политики, которую этот закон прямо запрещал.
Это стало началом Раскола Картеля, который длился несколько столетий и формально был завершён после смерти Гало и возвращения Дуракам части прежних полномочий.


\section{Список иллюстраций}

\begin{itemize}
\item Карта --- <<Хроники дорог и ветров>>, том 1, обитаемая Корона.
Относительно точная карта;
есть предположения, что она строилась по картам тси, не сохранившимся до наших времён.
(Либо, как вариант --- картография Ордена Преисподней.)
\item Карта Края.
Данные картографии Ордена Преисподней.
\item Принципиальная двумерная схема строения хоргета.
\item \textbf{[Глава 2, параграф 1]}
Кварталы Тхитрона в мирное время и во время вторжения, вагенбург на перекрёстке, колодец-требюше.
\item Устройство храма.
Казармы, школа, гонги, зал с ласточкиными нишами, операционный балкон, крипта, зона молчания.
\item \textbf{[Отчёт Хомяка]}
Схема тор-отсека.
Единственное известное изображение кольцевой теплицы.
Манускрипт <<Процветание. Не более необходимого>>, Сок-Стального-Листа.
\item Кольцевая теплица.
Художественная фантазия Тхарту ар'Хэ.
\item Письменность.
Абис, письменность сели, иероглифика цатрон: разные каллиграфические школы.
Змеистая письменность, дипломатическое письмо травников (ячеистое), микханская тайнопись, резы стрелохвостов.
\item Гамма цветов, воспринимаемая тси.
\item Образцы амулетов со Старой Личинкой --- дерево, кость, камень, металл.
\item Любопытна история иероглифа.
Этим символом --- спиралью --- на Преисподней помечали дома, в которых, по мнению людей, поселился \emph{хорохито} (инкарнированный хоргет).
Символ ставили в незаметном для хозяев месте, иногда выкладывали из камней или вытаптывали на дороге.
Хорохито редко убивали, так как бытовало поверье, что дух переселяется в другого человека;
очень часто его даже бесплатно кормили и одевали.
Тем не менее, подозреваемому никто не верил, с ним не дружили, не заводили любовные связи и вообще старались не иметь никаких дел.
\item Женщина-трами с обрезанным правым крылом.
Обрезка <<на два клина>> говорит о её принадлежности к трами Кипящего Полуморя.
\item Портрет: Эрхэ ар'Люм с бумажным фонарём.
\item Портрет: Кхохо ар'Хетр пляшет без рубахи, в штанах, у пояса сабля, в зубах трубка.
\item Портрет: Чханэ ар'Катхар в венке из омелы.
\item Портрет на разворот: Манэ и Лимнэ ар'Люм.
\item Портрет: Секхар ар'Сатр.
\item Портрет: Трукхвал.
\item \textbf{[Зачисление в контрразведку]} Сравнение: лик Самоотверженного Хата и знамя Анкарьяль.
\item \textbf{Пасхалки:}
\begin{itemize}
\item Свежая кровь: стрела в колене (что-то из Skyrim)
\item Волосы: Ну, погоди!
\item Живите кто хотите: Дядя Фёдор
\item Одинокий столб: Minecraft
\item Лампа: Мельница
\item Ничья: Игра Престолов.
\item Ранние слёзы: Кирандия
\item Большая игра: Леонид
\item Чужое тело: Воннегут
\end{itemize}
\item Портрет Митхэ (тот самый): возможно, к отрывку <<Золотая Пчела>> (рисунок на форзаце)?
\item Жестовый язык сели.
Ключи и модификаторы.
Ключи --- рыбка, трансмиссия (коленце), планета (зеркало), сапиент, жизнь (растение).
Модификаторы --- единственность, акцент, даритель, хранитель, свойство.
\item Метритхис: схемы на столе дипломатии.
\item Чеснок (тхэтраас, буквально --- <<приправа дороги>>) из терновой ветви. 
В центр помещается шарик из смеси жира (масло какао, свиное сало и нутряной жир оленя) и яда (кожа драгоценной квакши, кураре, высушенные железы зелёных пчёл, реже лаковые ягоды).
\item Одно из многочисленных граффити, изображающих Тако из клана Дорге.
Сделано людьми Асахина на одном из жилых домов.
Бордовая гимнастёрка с сорванными знаками различия Ордена, пулевая рана в районе сердца, винтовка за спиной, строительный уровень и мастерок в руках.
Необычность изображения заключается в стиле, наличии раны и отсутствии знаков различия (на большинстве граффити всё-таки присутствует <<вулкан>> и <<перевёрнутый топор>> легионера).
Внизу видна надпись <<ликвидирован>> и <<предатель>> --- этой надписью агенты Ордена метили все граффити с Тако, чтобы деморализовать людей, но через какое-то время надпись стала неотъемлемой частью граффити, её копировали уже сами люди.
Также можно заметить глаза неестественно жёлтого цвета с чёрными спиралями вместо зрачков --- знак хорохито в культуре ущелья Такэсако.
\item Созвездие Молот, вид с Тра-Ренкхаля (ориентир системы Тси-Ди).
\end{itemize}

\section{Язык цатрон}

\subsection{Фонетика}

\begin{itemize}
\item \`a --- descenda. Как будто бросают или машут рукой.
\item \'a --- ascenda. Краткое удивление.
\item \^a --- acuta. Короткий звук с резким, <<истерическим>> повышением тона.
\item \~a --- vibrata. Синусообразное плавное изменение высоты тона с большой амплитудой.
\item \"a --- tremola (discreta). Смех или блеяние на одной ноте (штробас).
\item \r{a} --- dislocata. Синусообразное изменение высоты тона с большой амплитудой при расщеплении гортани (штробас).
\item \=a --- plata. Плавное на одной ноте, средней продолжительности.
\item \v{a} --- profunda. Глубокое удивление.
\end{itemize}

\subsection{Имена}

\subsubsection{Двухсложные}

\begin{itemize}
\item Акхсар --- Зов духов
\item Кхохо --- Плачущий ягуар
\item Ликхмас --- Бегущая лань
\item Манис --- Ласковый мужчина (м.б. и женским)
\item Митракх --- Песня Митра
\item Сакхар --- Горящий щит
\item Саритр --- Дыхание стрелы
\item Сатракх --- Песня ветра
\item Сатхир --- Спокойствие океана
\item Сиртху --- Запах любви, специфический запах охваченного страстью человека
\item Согхо --- Печальная флейта
\item Трукхвал --- Летающая драгоценность
\item Тхартху --- Аромат вина
\item Хонхо --- Ягуар в брачный период
\end{itemize}

\subsubsection{Односложные}

(Однослог, детское имя, мужское, женское, унисекс)

\begin{itemize}
\item Кхар --- Кхарси, Кхарас, Кханэ, Кхарам --- Щит
\item Кхот --- Кхотси, Кхотрис, Кхотхэ, Кхотлам --- Котелок
\item Ликх --- Ликси, Ликас, Ликхэ, Ликлам --- Мечтающий
\item Мар --- Марси, Марас, Матрэ, ? --- Снег
\item Мит --- Митси, Митрис, Митхэ, Митлам --- Солнце
\item Митр --- Мирси, Мэис, Мэхэ, Митрам --- Скорбь
\item Сит --- Ситри, Ситрис, Ситхэ, Ситлам --- Любовь
\item Тхар --- Тхарси, Тхалас, Тханэ, ? --- Вино
\item Хат --- Хатси, Хатрис, Ханэ, Хатлам --- Сильный
\item Хитр --- Хирси, Хитрам, ?, ? --- Стрела
\item Эр --- Эрси, ?, Эрхэ, ? --- Девочка (чаще женское)
\end{itemize}

\subsection{Родовые слоги}

\begin{itemize}
\item Кахр --- Песок
\item Люм --- Лепестки
\item Мар --- Снег
\item Катхар --- Винокур
\item Кхир --- Ароматная, жирная почва
\item Со --- Вода
\item Хэ --- Женщина
\item Митр --- Скорбь
\item Лотр --- Золото
\end{itemize}

\subsubsection{Поселения}

\begin{itemize}
\item Тхитрон --- Пепелище
\item Тхаммитр --- Ледяная скорбь
\item Тхартхаахитр --- Отравленное вино
\item Тракхвинхал --- Дымящееся безумие (состояние, охватывающее животных при приближении лесного пожара)
\item Сотрон --- Утонувший
\item Кахрахан --- Могильный берег
\item Хатрикас --- Предательский звук
\item Ихслантхар --- Непрекращающееся землетрясение
\end{itemize}

\section{Язык Эй (данные)}

\subsection{Смысловые части}

\subsubsection{Над предложением}

\begin{itemize}
\item CAU --- причина
\item CON --- следствие
\item СOR --- корреляция
\end{itemize}

\subsubsection{Предложение}

\begin{itemize}
\item SUB --- субъект
\item ACT --- действие
\begin{itemize}
\item VER --- глагол
\item LOC --- локализация (полярные или декартовы координаты односительно оговорённого начала отсчёта)
\item TIM --- время (интервал или число)
\end{itemize}
\end{itemize}

\subsubsection{Прочие структуры}

\begin{itemize}
\item QUE --- знак вопроса, подставляется в структуру
\item ACC --- акцент, подставляется в структуру
\item NUM --- число (число и единица измерения)
\item INT --- интервал
\item THS --- указатель, подставляется в структуру
\item OPR --- логические и математические операторы
\item STT, END --- скобки, группирующие операторы
\end{itemize}

\subsection{Мирквудская нумерация}

Мирквуд --- старейший научный центр Гелиополя, занимающийся естественными и искусственными языками.

Таблица 00 --- основа языка. 00-2 --- двоичная, 00-8 --- восьмеричная, 00-F --- шестнадцатиричная.

Таблица 10 --- все виды жестовых (неконтактных) таблиц для 4--5-пальцевых конечностей.

Таблица 20 --- омега-архитектуры.

Таблица 30 --- дельфинья фонетика.

Таблица 50 --- электронные архитектуры.

Таблица 60 --- световые и квантовые архитектуры.

Таблица 80 --- танцевые и тактильные таблицы.

Таблица B0 --- усовершенствованная таблица для СЧФ.

Таблица D0 --- упрощённые химические таблицы для Ветвей Звезды.

Таблица F0 --- таблица для тси-подобных языков.

Прочие цифры зарезервированы под специфические смешанные виды передачи информации и коды на основе Эй.

Первыми после доклада Ликана Безрукого языком стали пользоваться апиды и дельфины Капитула, что отразилось на нумерации таблиц --- жестовые и СДФ-таблицы появились раньше других.

\section{Культ Четверых}

Религия, которая объединяет почти все народы Тра-Ренкхаля.
Несмотря на различия у разных народов, существование общих корней этих четырёх культов доказано.

\begin{description}
\item[Культ Разрушителя] (Безумного, Несчастья, Поработителя) --- самый распространённый.
Народы Тра-Ренкхаля верят, что несчастья и смерть --- необходимая часть жизни, и если жить становится очень хорошо, значит, мир приближается к гибели.
Разрушителю поклоняются народы: сели (Безумный), ноа (Деа Акседент), трами (Бог-Убийца), хака (Безумный), тенку (Молотобойца), ркхве-хор (Уничтожитель).

\item[Культ Опалителя] (Солнца, Света).
Основным является у ноа, зизоце и пылероев.
Наблюдается у ноа (Деа Солар), тенку (Солнечная Птица), ркхве-хор и пылерои (Опалитель), зизоце и тенку (Отец).

\item[Культ Воды.] Наблюдается у сели (Сестра Дождь, Сестра Река), хака (Мать-Дождь), тенку и зизоце (Мать), ноа (Деа Марин), дельфины (Милая Бездна), нгвсо (Омывающая).
Основным является у дельфинов. Принимает самые разные формы --- от всеобъемлющей заботы у нгвсо до своенравного Деа Марина у ноа.

\item[Культ Создателя] (Творца, Изгнанника, Безымянного).
Наблюдается у сели, хака, тенку, ноа, дельфинов и нгвсо.
Основным является у нгвсо.
Также предполагается, что у тенку Культ Солнечной птицы вмещает как культ Опалителя, так и культ Создателя, т.к. Солнечная птица имеет совершенно явно амбивалентную природу.
\end{description}

Общая логика, связывающая 4 культа:

\begin{itemize}
\item Создатель --- инженер, кто создаёт машину;
\item Вода --- среда, пространство, где создают машину;
\item Опалитель --- источник энергии, время, что приводит машину в движение;
\item Разрушитель --- тестер, кто испытывает машину на прочность.
\end{itemize}

\chapter{Не для печати}

\section{Идеи}

\begin{enumerate}

\item <<Мстительные тени>> (Vengeful Shades) --- книга за авторством Минуя-Дерево-Желаний, офицера чрезвычайного отдела Тси-Ди.
Является единственным в своём роде учебником по противодействию демонам, рассказывает о признаках инкарнатов, структуре демонических фракций, своде законов, отличительных знаках и тактике демонических диверсионных подразделений.
На момент прибытия Ордена Преисподней отправлена жрецами сели на Полку Непонятных Книг, но, несмотря на это, текст дошёл до наших дней в практически первозданном виде.

\item Цели (Te'сели) --- золотокожие люди. Сели --- хоронящие [своих мертвых] в лесу, изначально слюр (хака мумифицируют мёртвых), который превратился в этноним.

\item Квартальные гонги --- врачебный и военный.

\item Deorum injuriae diis curae.

\item Живодерский союз племен --- ягуар с кайманом в зубах.

\item Кости фаланг в качестве плаг для тоннелей.

\item Вселенная фрактальна.
Любое вероятное событие происходит.
Но вероятности идут только по одному пути --- по градиенту сознания.
То есть любое событие в наблюдаемой Вселенной происходит с одной целью --- увеличить плотность сознания (проще говоря, повысить количество и качество живых мыслящих систем).
ПКВ является полем, определяющим этот градиент, и задаёт направление всем вероятностям.
Таким образом, ПКВ --- поле псевдослучайностей.

<<Проблема гибели цивилизаций>> --- один из парадоксов фрактальной Вселенной.
Иногда гибель цивилизаций позволяет другим цивилизациям развиться сильнее.
Таким образом, несмотря на кажущееся противоречие концепции градиента сознания, этот парадокс объясним.

<<Проблема неживых планет>> --- второй парадокс.
Почему жизнь не зарождается на всех планетах, ведь тогда плотность сознания была бы выше?
Парадокс был решён после разработки шкалы Яо.

\item Лунные сады --- сказание о мире-аэростате.

\item <<Клетка с мышами, предназначенными на корм удаву>> --- аллегория для обозначения общества, которому грозит опасность, но которое занято иерархической борьбой.

\item Море солёное из-за того, что Безымянный сидит на дне и постоянно плачет.

\item Похоронные обряды у народо Тра-Ренкхаля:

Сели хоронят умерших, закапывая в землю возле дорог.
В могилу кладут чашу, венок, початок кукурузы, разрезанную верёвку, сладости и (опционально) украшения, оружие и семена цветущих растений.
На дереве рядом рисуют или вырезают символы лесных духов, а также дату.
Через два года кости вынимают из могилы крестьяне и используют костную муку для посевов и корма скоту.
Дату на дереве после эксгумации обводят кружком.
Так же хоронят мёртвых ркхве-хор.

Ноа хоронят умерших в старых лодках, либо заворачивают в ткань и пускают по волнам.
Пустынные ноа усаживают умерших с помощью верёвок на скалах, где их съедают хищные птицы.

Идолы Живодёра используют особые полуоткрытые склепы, в телах делают множество отверстий, в которые засыпается смесь земли и семян.

Молчащие идолы тщательно привязывают мёртвых к веткам на расстоянии двухсот шагов от границы Молчащих лесов.
На верёвки нацепляются бусины и костяные погремушки.
Шум от этих погремушек не проникает в леса, но зато отлично слышен над Молчащей рекой и даже на другом её берегу.

Пылерои Предгорий съедают тех, кто умер в битве, умерших естественным путём укладывают в пещерах.
Их кости через некоторое время изымаются и используются для обрядов.

Хака высушивают тела на костре, пересыпают золой и складывают в склепах, находящихся при капищах, соблюдая строгую очерёдность в зависимости от статуса умершего.

Тенку, в подражание ноа, в основном хоронят на лодках, пуская их по реке.

\item Болезни плантов: бататовая кожа --- отсутствие хлорофилла, солнечная болезнь --- диабетоподобное состояние, при котором наблюдается гипергликемия на солнце вплоть до комы и смерти.

\item Шаманы --- название третьего гендера у тси-язычных племён.
К шаманам относятся те, кто сменил пол (шаманы-мужчины, шаманы-женщины), сапиенты, остановившиеся в половом развития (цикады).

Отличительная черта цикад --- недоразвитые промежуточные половые органы (как у детей), высокий рост, очень худощавое телосложение, светлая кожа и иногда волосы, повышенный интеллект и средняя продолжительность жизни (в среднем цикады живут в 1,3 раза дольше прочих представителей своего вида), а также повышенная ломкость костей и сниженные физические характеристики.
Также для них характерна асексуальность, асексуальных цикад на порядок больше, чем асексуалов в остальной популяции.

--- Я помню, как они пришли в Трёхэтажный Храм.
У меня тоже тогда были предубеждения насчёт цикад --- скажем так, бойцы из них неважные.
Мы обычно устраиваем проверочный спарринг, против них вышла тогдашняя вождь.
Это был не бой, это был танец.
Ликхлам сняли с неё рубаху саблей в поединке, аккуратно распоров её по швам и не задев кожу.
Вождь была настолько впечатлена, что кинулась их целовать.

\item Индекс (поисковая таблица) --- библиотечная таблица для поиска ключевых слов. Бумажная под стеклянным полом?
Деревянные брусочки?
99\% слов тси состоят из двух иероглифов;
иероглифы проставлялись по вертикали и горизонтали в ключевом порядке, а в ячейки записывались координаты слова (книга, строка).
С помощью индекса можно было чрезвычайно быстро найти нужную информацию в огромной библиотеке.
Также существовал Малый Индекс --- для трёхсложных слов (чаще всего имён).
Индексирование библиотеки производилось обычно раз в сто дождей, так как являлось чрезвычайно кропотливым и трудоёмким процессом.

\item Дым-цветок --- вид теплицы, который использовался на слабо освещённых планетах со средним температурным режимом, в частности, на Преисподней и Чёрной Скале.
Днём дым-цветок раскрывался, и его зеркальные лепестки собирали свет на растениях.
Ночью лепестки закрывались, предохраняя растения от ветров и переохлаждения.
После одичания сапиентов многие дым-цветы Преисподней были превращены в театры и места сбора;
на Чёрной Скале дым-цветы превратились в форпосты и крепости, так как их было очень удобно оборонять.

\item Для них характерен видоизменённый культ Тензора.
Культ Тензора (также псевдокульт Тензора) --- вера в некое сверхъестественное существо (реже --- механизм), следствием работы которого является существование Вселенной.
Сторонники культа Тензора считают, что всё сущее --- звёзды, планеты, растения, животные, сапиенты --- лишь тени Тензора, лежащие на ткани Вселенной.
Для них характерна вера во фрактальность времени --- каждое событие имеет бесконечное число исходов и бесконечное число предшествующих состояний.
Основное различие между разными ветвями культа --- в отношении к субъекту:

\begin{enumerate}
\item концепция <<<бесконечного облака>> --- сапиент представляет собой квантовое облако, не имеющее чёткого края ни в пространстве, ни во временном фрактале.
Другими словами, сапиент представляет собой совокупность своего возможного прошлого и своего возможного будущего;
\item концепция <<одинокого странника>> --- сапиент представляет собой точечное сознание, перемещающееся по ветвям временного фрактала.
Считается, что каждая Вселенная содержит только одно сознание, все прочие живые --- лишь бессознательные тени.
\item Концепция Единого --- сапиент отождествляется с Тензором вплоть до полного стирания индивидуальности.                                                                                                            \end{enumerate}

Второй и третий вариант чаще всего характерны для различных монашеских орденов.

\item Трусики со сфагнумом как памперсы.

\item Чем более изолировано племя и меньше генетическое разнообразие, тем больше звуков в языке.
Спасибо Мирославу Нуриддинову за идею.

\item У сели есть также вариант языка, который является основой жестового, птичьего, звериного языков и языка боевых барабанов.
Эти языки имеют упрощённую грамматику (три формы глагола --- прошедшее общее, настоящее общее и намерение, отсутствие составных глаголов).

\item Столы для переписывания на барабане (как в Средневековье).

\item Виды атак хоргетов.
Масс-атака --- атака неструктурированной масс-энергией.
Реверс-атака --- атака неструктурированной масс-энергией противоположного знака.
Щитование --- принятие потока масс-энергии на энергоблок (пузырь).
Информационная атака (взлом).
Также бывает реверс-взлом.

\item Гуляй-город --- walking burg.

\item Шёпот в тси-подобных языках --- дыхательная артикуляция плюс кисти рук.
Беззвучная речь --- кисти рук и иногда губы.
Крик --- голосовая артикуляция плюс руки без кистей (кисти сжаты в кулак или с сомкнутыми пальцами).

\item Оттенок акхкатрас --- смесь красного и инфракрасного.

\item Рассказ о Кусачке и принятие его.

\item Добавить деревенских прибауток и присказок.

\item Сделать что-то с расстояниями.
Оказывается, до Инхас-Лака такое же расстояние, как до Тхартхаахитра!

\item Свидетельство канарейки.

\item Каким образом демоны друг друга узнают?
Уникальный отпечаток, который очень сложно подделать.

\item Языки: Преисподняя --- японский, Тысяча Башен --- норвежский, ноа-лингва --- итальянский, цатрон --- искусственный, Земля Врачевателей --- арабский (?).

\item Жрецы и охотники всегда отращивали волосы.
Есть специальный термин --- <<чувствовать кожей головы>>.

\item Нож и кинжал.
Надо это как-то разграничить, что ли, причём везде.

\item Клинок Осколков (Хрустальный Клинок) --- легендарное оружие, которое Маликх забрал у поверженного им великана по имени Снежная Борода.
Этот клинок при блокировании был твёрже кукхватра, но при атаке, ударившись о чужую саблю или доспехи, он рассыпался на тысячи острых как бритва осколков, которые иссекали врага до костей.
Затем клинок в мгновение ока вырастал на эфесе снова.

\item У тси есть фермент, перерабатывающий ацетальдегид в пируват.
Т.е. они могут потреблять много спиртных напитков без интоксикации.

\item Крыши из водорослей.

\item Имя для демона --- Вольга Сын Змея (Volga The Drake Son).

\item Насчёт тегов: в сказках и легендах индивидуальность стирается, все говорят одинаковым штилем.
Но для сказки каждого народа штиль видоспецифичен.

\item Оружие --- засадный лук (с выдвигающейся перекладиной и спусковым крючком, нечто среднее между луком и арбалетом).

\item Прочие воины жили в городе и приходили лишь на работу и поесть.
Поэтому Ликхмас их не видел.

\item Всего в языке тси насчитывается 1024 ключа.
В языке сели осталось 360 --- некоторые похожие ключи были слиты в один, подавляющее большинство утрачено из-за того, что понятия и иероглифы вышли из употребления.

\item Звездануть затянутые отрывки (разбить на несколько кусочков, чтобы читатель не уставал читать).

\item Дышащие Ртом --- общее название не-апид апидами.
Грубое, презрительное --- малоножки, малоглазки, горлозадые.

\item Молчащие идолы.
Ношение одежды считают трусостью.
Переговариваются почти всегда жестами.
Натягивают верёвки между поселениями и передают сообщения, дёргая за них.
Обряд инициации --- жизнь в Хрустальных землях.

\item Школа Скорпиона.
Девиз --- <<Мы одно>>.


\item У каждой главы своя рамка для страниц.
Но на некоторых страницах в рамке могут быть изменения --- пасхалки.

\item Самоотверженный Хат.
Прообраз --- Хатлам ар-Мар.
Покровитель жертвующих собой.
Атрибут --- звёзды.

Тёплый Хетр, Хетр-пекарь.
Покровитель дома, домашнего очага и постели, а также кулинаров и трактирщиков.
Атрибут --- очаг и крыша.

Хри-соблазнитель.
Покровитель влюблённых, секса, игр и наркоманов.
Атрибут --- пухлые чувственные губы.

Кхар-защитник.
Покровитель защитников, отвечает за стены, ворота и доспехи.
Атрибут --- щит со стрелами.
Кхар-защитник обычно изображается настороженным, но не гневным.
Гневное изображение Кхара-защитника часто встречается в Пыльном Предгорье, иногда Кхар даже рисуется с полностью открытыми глазами (<<Яростный Кхар>>), но такие изображения считаются неканоничными.

Удивлённый Лю.
Отвечает за библиотеки и смотровые башни.
Покровитель учёных, исследователей и разведчиков.
Атрибут --- свиток.
Лю существует в двух вариантах: Удивлённый и Испуганный.
Удивлённый Лю рисуется на дверях библиотек, на тотемах и духовых ружьях.
Испуганный Лю используется в двух ситуациях --- предупреждение об опасности либо мольба об удачном побеге от врага.
Оба варианта признаются каноничными.

Печальный Митр, Митр-певец.
Покровитель отчаявшихся, менестрелей, поэтов и душевнобольных.
Атрибут --- перья для письма и чернильница.

Сан-сновидец.
Отвечает за сон, смерть и душевный покой.
Покровитель врачей и спящих.
Атрибут --- закрытый рот.

Обнимающий Сит.
Прообраз --- Ситхэ ар'Со, некрасивая, слабая и бесплодная женщина, которая ночами во время войн утешала чужих детей.
Покровитель детей, стариков и одиноких людей.
Атрибут --- руки.

\item Метритхис.
У фигур тоже есть подобие свободной воли --- они могут выбирать сторону, чувствовать и действовать по своему разумению.
Этим управляет Фатум?

\item Обычай срезать рыбки перед важным боем, но оставлять перед смертельным.
Они могут погубить, но по жилам джунглей лучше идти красивым.

\item У тси есть светящиеся органы?
Ладошки, например (именно поэтому апиды в общении используют руки?)

\item \textbf{[Технический момент]} В первый блок \textbackslash mulang можно поместить следующие данные:

\begin{itemize}
\item код персонажа (чтобы выдёргивать реплики конкретных персонажей и работать над стилистикой);
\item код языка --- родной язык персонажа и язык, на котором фраза была сказана (чтобы следить за языковыми особенностями построения предложений и лексики);
\end{itemize}

\item Скорбящие.
Так у Аркадиу есть тело или нет?
Тот же вопрос насчёт Штрой --- куда делось тело после изгнания?

\item Зонт-фонарь --- по типу зонтиков торговцев.

\item Керномор Тридцать Три, Вакула Вечерний Гость.

\item Печать --- отпечаток пальца, губ или укус (или всё вместе, где какие традиции).

\item Ивовая кора --- салицилаты.

\item Первый жрец (Картель) курсивно выделяет слова (или не надо, перегрузка?).

\item Имя для демона --- Эйстейн Дева.

\item Большая игра --- логика беседы нарушена.

\item Вулкан против ищеек.
Средства против технического отслеживания на Преисподней продумать.

\item Преисподняя --- горячие источники, гейзеры, сера.

\item На Преисподней у демонов ещё не было прозвищ.

\item Тропический климат.
Продумать ветры.

\item Плавник, который морем приносит.
Хз зачем он мне, но пусть будет.

\item Цатрон.
Ударения удвоением гласных?

\item Описание эмоций не словами, а реакциями организма.
\textbf{Серьёзная правка!}

\item Язык планеты Мороз --- сделать отдельную статью?

\item Коллизия --- Паутина-город Лотоса и Паутина Тси-Ди.

\item Объясняет, но не оправдывает.

\item Северная Корона (Обитаемая Корона).
Южная Корона (Суболичье).
Китовый Юг (Тысячеречье).

\item Может, на Тра-Ренкхале не будет обезьян?
Мне чот лень их уже добавлять.

\item Ликхмас не должен быть Мэри Сью.
Например, в первой главе сделать упор, что ягуар маленький и неопытный.

\item Зизоце выбривают темя, потому что лысина --- признак мужественности и зрелости.

\item Инфинит --- эпиграф 1 гл. II части?

\item Бабочка --- грудная клетка на языке сели.
Мужской жемчуг --- сперма.

\item Беседка над алтарём.
Дожди всё-таки.

\item Террористический акт, угроза биологическим оружием.

\item Имя: Эвкалипт.

\item Вставки других языков в текст --- эффект <<недопонимания>> героем чужой речи.

\item Есть предложение после 2 главы 2 части вставить литую главу <<Лисьи сказки>>, куда слить все происшествия в пути Ликхмаса, Грейсвольда и Анкарьяль.
Или две литых главы на часть --- это многовато?

\item 11.6 --- не вполне ясно, что отряд 14 сражается с порабощёнными тси.

\item Больше Бродячего Народа.
MOAR.

\item Обычай сели: белые рубахи в городе и иная одежда в джунглях.

\item <<Если ты боишься, дай противнику тебя ударить, --- вспомнил я слова Конфетки.
--- После первого удара страхи уходят>>.

\item Она говорила о какой-то ерунде, но её руки в моих волосах и ноги, словно невзначай прижимающиеся к моим плечам, шептали такие нежности, что я мог только сидеть и молча слушать.

\item Архипелаг Ночного Костра.

\item Нестыковка: демоны без тел могут или всё же нет?

\item Клин белых птиц превратился в созвездие.

\item Клянусь, что приму пришедшего ко мне и его историю такими, каковы они есть --- без оговорок и без остатка.

\item Коллизия. Оазисы Тси-Ди и оазисы Преисподней.

\item Flos (цветок) --- на жаргоне Картеля: сейхмар.
Сорвать цветок --- прожить жизнь в теле сапиента.

\item Колебание струны потоком воздуха.
Идея для цитры Ветра.

\item Сели видят тело сквозь одежду.

\item Испытание пчелами.
В школе боевых искусств мастером признавался тот, кто саблей рассекал всех пчел в воздухе до того, как они коснулись его.

\item Кемер Зимняя Вишня (Kemer the Winter Cherry).

\item После Отбора ребёнок придумывает себе имя тси.
До Отбора ребёнок носит только имя цатрон и (иногда) домашнее прозвище.
В разных местах традиция разнится --- кое-где домашнее прозвище не связано с именем тси, кое-где ребёнок выдумывает имя на основе домашнего, кое-где ребёнка всё детство зовут по имени цатрон, а имя тси (и, соответственно, домашнее) он выдумывает сам.
У ноа тайное имя даётся так --- два слова придумывают кормильцы, третье --- сам ребёнок.

\item Нет слов <<хороший>>, <<плохой>>, <<добрый>> и <<злой>>.
\textbf{Серьёзная правка!}

\item Кошка внесла в культуру тси изменения, которые превращали взращённых в этой культуре сапиентов в ловушки для демонов.
Эти тела прививали демонам качества самих тси.
Поэтому большинство демонов использовали неполное заякоривание и/или пользовались телами людей Тра-Ренкхаля.

\item Тхитрону много тысяч дождей.
История, развалины, барельефы, пережитки других поколений.
Культура идолов (Нашествие Змей).

\item Болтливая пустота --- голос, слышимый на границе сна и бодрствования, не имеющий возраста, пола, интонации, как будто сотканный из окружающих звуков.
Чаще всего несёт полную чушь.
(<<Болтливая пустота тебе [по секрету] сказала?>>)

\item Имя --- Антти Предательские Воды (Antti Treacherous Waters).

\item Разница между рекой и речушкой в языке цатрон определена совершенно точно --- устройством мостов.
Любая водная жила, через которую можно перебросить навесной мост --- речушка (со).
Если же требуются дополнительные опоры (быки, острова, понтоны либо гребни), то это река (ху).
В цатроне существует ещё один термин, обозначающий направленный ток воды --- эрхо (река без берегов, т.е. морское течение).
Петлевое течение у Кристалла на цатроне называется Эрхо'люаэсакх (Течение [того, кого] обманули в [последний] раз).

\item Судьба жывотных --- Цапка, Серебряный, Нейросеть.

\item Азуритовые розы, кровавые крокоитовые иглы, сколецитовые хризантемы.

\item Я не отдаю тебе свои волосы, я отдаю тебе время, которое они растут.

\item In the shadow of the mushroom cloud --- В тени ядерного гриба.

\item Описать Тхитрон.

\item Проработать топографию храма и (важно!) место взрыва.

\item \textbf{[Путь жреца]} Воспоминания о пыли --- смысл-образующие и должны предварять всё прочее.
Что делать?

\item Сверстники!
Где они?
Чханэ, Ликхэ, Столбик, сестрёнки, всё!
Где храмовая молодёжь?
Где молодые крестьяне/ремесленники?

\item Смена сословия.

\item Лака и кураресодержащие растения на храмовых землях.

\item Субкультуры?

\item Рыбак (Хэмингуэй).

\item Землетрясения, смертельная пустыня.

\item Жизнь других видов.

\item Второй шанс разрушителю.

\item Превращение отступления в победу.

\item Решение оставить Тхартавирт (<<Город не важен, важны люди>>).

\item Решение об исходе --- <<Будем держать врага с одной стороны>>.

\item Осенняя прогулка с Чханэ (ForgottenTale).

\item Кон-Тики и Полярный Водоворот.

\item Философские школы: <<Победа вездесуща, подобно воде, но лишь холодный разум может собрать влагу из воздуха>> --- Плющ и Капля Росы.
<<Сердце врага --- твоё сердце>> --- Путь Ягуара.
<<Ярость как ураган --- не разобьёт, так сточит>> --- Десять Песчинок.

\item Козья ножка для рукояти.

\item Больше упоротых ритуалов и разговоров с духами под наркотой!
Верования сели: Сестра Дождь (Сестра Колодец) --- персонаж, иногда идентифицируемый с Обнимающим Ситом, иногда отдельный.
Корневики --- трудяги, ремонтирующие жилы джунглей.
Духи леса.
Каменные духи.
Пристанище.
Опалитель, он же Деа Солар.
Червь-узурпатор (Старая Личинка) --- персонаж западных сели (вероятно, отголоски верований Синего Колена).
В обмен на мёртвое тело кусает жилу джунглей, пропуская таким образом душу к пристанищу.
У Синего Колена Старая Личинка --- подписывает с душой договор на новую жизнь.
Зелёная Звенящая Вода --- знак из культа дельфинов Среднего Моря.

\item Нужна хронология.
Очень.

\item Езда на оленях.
Обучение, прочее.

\item Схема координат для боя.

\item Манэ и Лимнэ --- женщина-краска и женщина-лоскуток.

\item Доработать доспехи.

\item Гроза во время Дела Перекрёстка!

\item Мифологичность сознания!

\item Солнечные печи.

\item Одна сцена переходит в другую.

\item Промыслы.
Земледелие, животноводство, аквакультура, птицеводство, виноградарство, рыболовство, цветоводство, пчеловодство (бортничество), собирательство, охота.
Пермакультура!
Ремёсла: камнерезы, ювелиры, деревщики, оружейники, кожевники, портные, сукновалы.
Ювелиров должно быть больше, чем кузнецов.

\item Эйраки использует жуков для сбора эманаций.
Знак кирпича инициируется жуком.

\item У тси нет слова <<убивать>>.
Есть слово <<разрушать>>.
\textbf{Серьёзная правка!}

\item Школа.
Тренировка органов чувств (вкус --- конфеты), обоняния, осязание, зрение, слух.
Тренировка контроля над чувствами.
Управление эмоциями (И тут я понял, что Трукхвал имеет в виду.)

\item Животноводство у сели.
Кур не держали взаперти --- они свободно гуляли где хотели.
Им просто устраивали уютные гнёзда, и птицы возвращались, чтобы откладывать яйца.
Понятия <<моя курица>> не существовало, просто была некая общая популяция, которой пользовались все (пермакультура?)

\item (всякое) Под порогом захоранивали предков.
Порог мыли тщательнее, чем остальной дом.
Кур и прочее забивали на пороге, чтобы кровь питала лежащих под порогом.
Охраняли дом.

\item Глава 10.
Не вполне ясно, как Небо стал командиром.

\item Кусачка.
Кто он такой, откуда взялся, вплести его в сюжет.

\item Законы сели.
Пленные 2 года жили в обществе сели, затем их отпускали, если не требовались жертвы (?).
Поработать над изгнанием --- за что и как.

\item Суть экономики.
Она есть, но у каждого человека есть неэкономические ресурсы для выживания --- кусок земли, инструменты, оружие.

\item Терраформирование.

\item Вживлённый пучок волос --- признак идолов Живодёра.

\item У народа трами сражаются только женщины.
Обрезают правое крыло.

\item Планты прекрасно поглощают воду кожей.

\item Молчащие считают ношение одежды трусостью.

\item Маршрут Ликхмаса (побег): Тхитрон --- Ихслантекхо --- сплав по Ху'тресоааса --- Тхартавирт.

\item Слова цатрона: m\=am\`a, p\r{a}p\`a, s\r{\i}s\`\i --- любая женщина в доме, tch\'atch\`a --- любой мужчина в доме.

\item Чувство магнитного поля, блеать.

\item Смысловая нагрузка вышивки на одежде.
В Мягкие Руки тот, кто желал найти пару, надевал одежду на голое тело.

\item Земля выдаётся каждому.

\item Насилие --- пользование собственностью без согласия владельца.
Тело, труд, дом, личные вещи.
Разрушение.

\item Хесематр --- язык Живодёра.
Хесели --- пиджин Омута Духов.
Хесетрон --- язык Молчащих.
Языки хака --- северный, <<дикий>>.
Диалекты сели --- южный, западный, северный, сотронский.
Пылерои --- 6 языков. Стрелохвосты --- 13 языков, из них 2 --- языки Голубого Зеркала.
\end{enumerate}

\section{Материалы}

\begin{enumerate}

\item Искусство создания языков, Питерсон

\end{enumerate}

\section{История сели}

\begin{enumerate}
\item предки пришли с места Тхидэ;
\item первое поселение --- Тхартавирт, торговали с Кахраханом (тогда ещё поселением царрокх);
\item основание Тхитрона, Ихслантекхо и Травинхала (бассейн Ху'тресоааса);
\item Столкновение с молодым государством тенку (текнек-мен), Первая приречная война.
Текнеки отброшены за Пыльное предгорье.
Возникновение первого святилища --- Весёлый Волок.
\item Раскол государства по видовому признаку.
Идолы заявили свои права на северные города.
Вторая приречная война, отвоевание Тхитрона, Ихслантекхо и Травинхала людьми.
Разделение идолов на Живодёрских (Центральных) и Молчащих.
\item Война северных царрокх с Молчащими в союзе с сели.
Формирование народа хака.
\item Война Живодёра с Фиолетовым Союзом.
Поражение Синего колена, исход за Реку Кувшинок.
\end{enumerate}

\section{Ноа}

\begin{itemize}
\item города окружены стенами --- против ветров.
\item выращивают культуры под навесами с сетью, смачиваемой водой.
Останавливает излишнюю радиацию.
\item Носят зонты-копья, зонты-фонари или зонты-духовые ружья.
\item Опреснители морской воды.
\item Солнечные печи.
\end{itemize}

\section{Английская версия}

\subsection{Справочник}

\subsubsection{Said}

acknowledged \hfill признал (важность, истинность)\\
added \hfill добавил, присовокупил\\
admired \hfill восхитился\\
admitted \hfill признал (с нежеланием)\\
advised \hfill посоветовал\\
affirmed \hfill подтвердил, заявил (признал факт уверенно и публично)\\
agreed \hfill согласился\\
alleged \hfill утверждал (голословно)\\
alluded \hfill подразумевал, намекал\\
announced \hfill объявил, огласил\\
answered \hfill ответил\\
apologized \hfill извинился\\
appealed \hfill обратился к кому-либо (публично и официально)\\
argued \hfill доказывал, убеждал, спорил\\
articulated \hfill произнёс (отчётливо, ясно)\\
asked \hfill спросил, попросил, пригласил\\
assented \hfill уступил, выразил согласие (официально)\\
asserted \hfill заявить (уверенно и с напором)\\
assured \hfill заверил, уверял\\
avowed \hfill признался (открыто)\\
babbled \hfill лепетал, болтал (в страхе или увлечённо)\\
bargained \hfill торговался\\
bawled \hfill орать, вопить (слёзно)\\
beamed \hfill лучезарно улыбнулся, засиял\\
began \hfill начал\\
begged \hfill умолял, упрашивал, выпрашивал\\
bellowed \hfill ревел (протяжно, от боли или гнева)\\
belted \hfill громко пропел\\
blabbed \hfill проговорился\\
blared \hfill орал, вопил (о музыке или сирене)\\
bleated \hfill проблеял\\
blurted \hfill выпалил (эмоционально)\\
blustered \hfill бушевал, грозился (без особого успеха)\\
boasted \hfill хвастался\\
boomed \hfill гремел (о громе)\\
bragged \hfill похвастался\\
breathed \hfill говорил тихо, вздыхал\\
cackled \hfill кудахтал, гоготал\\
cajoled \hfill польстил, уговаривал с лестью (чтобы убедить сделать что-то)\\
called \hfill позвал\\
cautioned \hfill предупредил (возможно, в гневе)\\
cawed \hfill прокаркал\\
challenged \hfill бросил вызов, оспорил\\
chanted \hfill проскандировал, пропел\\
chattered \hfill болтал, щебетал, трещал (о чём-то неважном)\\
cheered \hfill ободрил, ободрительно крикнул\\
chided \hfill журил, бранил, громко упрекл\\
chimed \hfill звучал согласно, был в согласии\\
chortled \hfill хрипло, радостно смеялся, ликовал\\
chuckled \hfill тихо, про себя посмеивался, хихикал\\
claimed \hfill заявил о чём-либо (без доказательств)\\
comforted \hfill утешил, успокоил\\
commanded \hfill приказал, скомандовал\\
commented \hfill прокомментировал, высказал мнение\\
communicated \hfill сообщил, передал\\
complained \hfill пожаловался, посетовал, ныл\\
conceded \hfill признал, уступил (после отказа)\\
concluded \hfill заключил, сделал вывод\\
concurred \hfill согласился, выразил то же мнение\\
confessed \hfill покаялся, сознался (в преступлении)\\
confided \hfill сказал по секрету\\
confirmed \hfill подтвердил (слухи, опасения)\\
consented \hfill дал согласие, позволил, разрешил\\
consoled \hfill утешил\\
contended \hfill утверждать в споре, заявлять\\
contested \hfill опровергать (что-либо)\\
continued \hfill продолжил\\
conversed \hfill беседовал\\
conveyed \hfill сообщать, передавать, выражать\\
corrected \hfill поправил, сделал замечание\\
coughed \hfill кашлянул\\
countered \hfill парировал, возразил\\
cried \hfill плакал, вскричал\\
criticized \hfill осудил, критиковал\\
croaked \hfill квакать, каркать, ворчать, брюзжать\\
crooned \hfill тихо напевал, мурлыкал\\
cross-examined \hfill допрашивал, докапывался (с целью дискредитировать)\\
crowed \hfill кукарекал, ликовал\\
cursed \hfill ругал, проклинал\\
debated \hfill спорил о чём-либо (формально)\\
decided \hfill решил\\
declared \hfill объявил, заявил, провозгласил\\
decreed \hfill постановил, распорядился\\
defended \hfill защищался, оправдывался, отстаивал\\
delivered \hfill зачитывал (формально)\\
demanded \hfill требовал\\
denied \hfill отрицал, отказывал, не допускал\\
described \hfill описал, охарактеризовал\\
dictated \hfill продиктовал (в любом значении)\\
digressed \hfill отвлёкся от темы\\
directed \hfill отдал приказ (обязательную инструкцию)\\
disclosed \hfill выдать, разоблачить\\
disproved \hfill опровергать, доказать ошибочность\\
divulged \hfill разгласить (как Джек Воробей)\\
drawled \hfill говорить, растягивая слова\\
droned \hfill гудеть, жужжать, монотонно бубнить\\
echoed \hfill вторить, поддакивать\\
elaborated \hfill вдаваться в подробности, развивать тему\\
emphasized \hfill особо подчеркнуть\\
enjoined \hfill приказывать, предписывать, запрещать\\
enunciated \hfill отчётливо произносить\\
equivocated \hfill увиливать, говорить двусмысленно, уклончиво\\
exaggerated \hfill утрировать, преувеличивать\\
exclaimed \hfill вскричать, воскликнуть, ахнуть (от боли, гнева или удивления)\\
exhorted \hfill заклинать, увещевать, убеждать что-то сделать\\
explained \hfill объяснить\\
exploded \hfill взорваться, сорваться\\
expressed \hfill изъявлять, выражать\\
extolled \hfill нахваливать, превозносить\\
faltered \hfill замяться, запинаться, говорить нерешительно\\
foretold \hfill предсказывать\\
fretted \hfill беспокоиться, мучиться\\
fumed \hfill быть в очень сильном гневе, кипеть от злости\\
gabbled \hfill тараторить, бормотать\\
gasped \hfill задыхаться (от страха, боли или удивления)\\
giggled \hfill хихикать (нервно, глупо)\\
glowered \hfill пристально хмуро смотреть (в гневе, подозрении)\\
greeted \hfill приветствовать\\
griped \hfill ворчать, жаловаться на что-то тривиальное, обыденное\\
groaned \hfill стонать, охать (от боли или отчаяния)\\
growled \hfill рычать, ворчать гроулом\\
grumbled \hfill ворчать, жаловаться, выражать недовольство (обычно тихо)\\
grunted \hfill хрюкать, ворчать\\
guessed \hfill предполагать\\
guffawed \hfill неистово, грубо хохотать\\
gulped \hfill \\
gurgled \hfill булькать\\
gushed \hfill говорить потоком чувств, писать с преувеличенным энтузиазмом\\
hailed \hfill \\
hinted \hfill \\
hissed [Fear] [Conflict] \hfill \\
hollered \hfill \\
hooted \hfill \\
howled \hfill \\
hummed \hfill \\
implied \hfill \\
implored \hfill \\
inquired \hfill \\
insinuated [Conflict] \hfill \\
insisted [Determination] \hfill \\
instructed \hfill \\
interjected \hfill \\
interrupted \hfill \\
intoned \hfill \\
jabbed [Conflict] \hfill \\
jabbered \hfill \\
jeered \hfill \\
jested \hfill \\
joked [Amusement] \hfill \\
lamented [Sadness] \hfill \\
laughed [Happiness] [Amusement] \hfill \\
lectured \hfill \\
lied \hfill \\
maintained [Determination] \hfill \\
marvelled \hfill \\
mentioned \hfill \\
moaned \hfill \\
mouthed \hfill \\
mumbled [Sadness] \hfill \\
murmured [Happiness] \hfill \\
mused \hfill \\
muttered \hfill \\
nagged \hfill \\
narrated \hfill \\
noted \hfill \\
objected \hfill \\
observed \hfill \\
offered \hfill \\
ordered \hfill \\
panted \hfill \\
phonated \hfill \\
phrased \hfill \\
placated [Making up] \hfill \\
pleaded \hfill \\
pledged \hfill \\
pointed out \hfill \\
pondered \hfill \\
postulated \hfill \\
prayed \hfill \\
preached \hfill \\
predicted \hfill \\
proceeded \hfill \\
proclaimed \hfill \\
professed \hfill \\
promised \hfill \\
proposed \hfill \\
protested \hfill \\
queried \hfill \\
questioned \hfill \\
quipped \hfill \\
quoted \hfill \\
raged \hfill \\
railed \hfill \\
rallied \hfill \\
ranted \hfill \\
rapped \hfill \\
rasped \hfill \\
raved \hfill \\
reasoned \hfill \\
reassured [Affection] [Making up] \hfill \\
rebuked [Anger] [Conflict] \hfill \\
recalled [Storytelling] \hfill \\
recited \hfill \\
recommended \hfill \\
recounted [Storytelling] \hfill \\
refuted \hfill \\
reiterated \hfill \\
rejoiced \hfill \\
related [Storytelling] \hfill \\
relented [Making up] \hfill \\
relieved \hfill \\
remarked \hfill \\
remembered [Storytelling] \hfill \\
reminded \hfill \\
repeated \hfill \\
replied \hfill \\
reported \hfill \\
reprimanded \hfill \\
reputed \hfill \\
requested \hfill \\
responded \hfill \\
resumed [Storytelling] \hfill \\
retaliated \hfill \\
retorted \hfill \\
returned \hfill \\
revealed \hfill \\
roared [Amusement] \hfill \\
rumbled \hfill \\
ruminated \hfill \\
sang \hfill \\
scoffed \hfill \\
scolded [Conflict] \hfill \\
screamed \hfill \\
screeched \hfill \\
shouted [Anger] [Excitement] \hfill \\
shrieked \hfill \\
shuddered \hfill \\
sighed [Happiness] [Sadness] \hfill \\
smirked \hfill \\
snapped [Anger] \hfill \\
snarled \hfill \\
sneered [Conflict] \hfill \\
snickered \hfill \\
sniggered [Amusement] \hfill \\
snorted \hfill \\
sobbed [Sadness] \hfill \\
soothed [Affection] \hfill \\
sounded \hfill \\
spat [Conflict] \hfill \\
speculated \hfill \\
spouted \hfill \\
sputtered \hfill \\
squawked \hfill \\
stammered [Fear] \hfill \\
stated \hfill \\
stipulated \hfill \\
stressed \hfill \\
stuttered [Fear] \hfill \\
suggested \hfill \\
surmised \hfill \\
swore \hfill \\
sympathised \hfill \\
tattled \hfill \\
taunted \hfill \\
teased [Amusement] \hfill \\
testified \hfill \\
theorized \hfill \\
threatened [Conflict] \hfill \\
thundered \hfill \\
tittered [Amusement] \hfill \\
told \hfill \\
twittered \hfill \\
urged [Fear] \hfill \\
uttered \hfill \\
vented \hfill \\
ventured \hfill \\
vocalised \hfill \\
voiced \hfill \\
volunteered \hfill \\
vouched \hfill \\
vowed \hfill \\
waffled \hfill \\
wailed \hfill \\
warbled \hfill \\
warned \hfill \\
wept \hfill \\
whimpered \hfill \\
whined \hfill \\
whispered [Fear] \hfill \\
whistled \hfill \\
wondered \hfill \\
yammered \hfill \\
yelled [Anger] [Excitement] \hfill \\
yelped \hfill \\
yowled \hfill \\

\end{document}
