\documentclass[a4paper,12pt,fleqn]{book}\usepackage{polyglossia}\setdefaultlanguage[babelshorthands=true]{russian}\setotherlanguage{english}\defaultfontfeatures{Ligatures=TeX,Mapping=tex-text}\usepackage{xcolor}\newcommand{\ml}[3]{#2}

% \documentclass[a4paper,12pt,fleqn]{book}\usepackage{cooltooltips}\usepackage{polyglossia}\setdefaultlanguage[babelshorthands=true]{russian}\setotherlanguage{english}\defaultfontfeatures{Ligatures=TeX,Mapping=tex-text} \usepackage{xcolor}\definecolor{lightgray}{HTML}{bbbbbb}\color{lightgray}\newcommand{\ml}[3]{\textenglish{\textcolor{black}{#3}}}

% ----------------------

\setlength{\headheight}{15pt}

\usepackage{amsmath,amssymb,amsfonts,xltxtra,microtype,graphicx,textcomp}
\usepackage{svg}

% ------ GEOMETRY ------

\usepackage[twoside,left=2.5cm,right=3cm,top=3cm,bottom=4cm,bindingoffset=0cm]{geometry}

% ------ FONT ------

\usepackage{ebgaramond}
\definecolor{darkblue}{HTML}{003153}

% ------ HYPERLINKS ------

\usepackage{hyperref}
\hypersetup{colorlinks=true, linkcolor=darkblue, citecolor=darkblue, filecolor=darkblue, urlcolor=darkblue}

% ------ EPIGRAPH ------

\usepackage{epigraph}
\renewcommand{\epigraphsize}{\footnotesize}
\epigraphrule=0pt
\epigraphwidth=8cm

\usepackage{etoolbox}
\AtBeginEnvironment{quote}{\itshape}
\makeatletter
\newlength\episourceskip
\pretocmd{\@episource}{\em}{}{}
\apptocmd{\@episource}{\em}{}{}
\patchcmd{\epigraph}{\@episource{#1}\\}{\@episource{#1}\\[\episourceskip]}{}{}
\makeatother

% ------ METADATA ------

\newcommand{\tofaauthor}{\ml{$0$}{Эмиль~Весна}{Emil~Viesn\'{a}}}
\newcommand{\tofatitle}{\ml{$0$}{ДАЛЁКИЕ~РЕЧИ}{Talks~of~Faraway}}
\newcommand{\tofastarted}{13.08.2012}

% ------ FANCY PAGE STYLE ------

\usepackage{fancyhdr}
\pagestyle{fancy}
\fancyhead[LE,RO]{\thepage}
\fancyhead[LO]{{\small\textsc{\tofatitle}}}
\fancyhead[RE]{{\small\textsc{\tofaauthor}}}
\fancyfoot{}
\fancypagestyle{plain}
{\fancyhead{}
\renewcommand{\headrulewidth}{0mm}
\fancyfoot{}}

% ------ NEW COMMANDS ------

\newcommand{\asterism}{\vspace{1em}{\centering\Large\bfseries$\ast~\ast~\ast$\par}\vspace{1em}}
\newcommand{\textspace}{\vspace{1em}{\centering\Large\bfseries<...>\par}\vspace{1em}}
\newcommand{\FM}{\footnotemark}
\newcommand{\FL}[2]{\footnotetext{См. \textit{\hyperlink{#1}{#2}}.}}
\newcommand{\FA}[1]{\footnotetext{#1 \emph{\ml{$0$}{---~Прим.~авт.}{---~Author.}}}}

\newcommand{\theterm}[3]{\textbf{\hypertarget{#1}{#2}} --- #3}
\newcommand{\thesynonim}[3]{\textbf{#2} --- см. \textit{\hyperlink{#1}{#3}}.}
\newcommand{\theorigin}[3]{\textit{#1:} #2 --- #3}

% ------ DIFFICULT TO WRITE TERMS ------

\newcommand{\Aatris}{\"{A}\={a}tr\v{\i}s}
\newcommand{\Akchsar}{\`{A}kchs\r{a}r}
\newcommand{\Chhammitrai}{Chh\`{a}mm\={\i}tr\^{a}i}
\newcommand{\Chhanei}{Chh\r{a}n\^{e}i}
\newcommand{\Chitram}{Ch\"{\i}tr\'{a}m}
\newcommand{\Choe}{Cho\^{e}}
\newcommand{\choe}{cho\^{e}}
\newcommand{\Harrmatr}{H\r{a}rrm\`{a}tr}
\newcommand{\tHat}{H\={a}t}
\newcommand{\Hei}{H\r{e}i}
\newcommand{\hei}{h\r{e}i}
\newcommand{\Hoesitr}{Ho\`{e}s\={\i}tr}
\newcommand{\hoesitr}{ho\`{e}s\={\i}tr}
\newcommand{\Kchaagotr}{Kch\^{a}\={a}g\~{o}tr}
\newcommand{\kchaagotr}{kch\^{a}\={a}g\~{o}tr}
\newcommand{\Kcharas}{Kch\'{a}r\v{a}s}
\newcommand{\Kchatrim}{Kch\r{a}tr\"{\i}m}
\newcommand{\Kchenoel}{Kch\={e}no\^{e}}
\newcommand{\kchenoel}{kch\={e}no\^{e}}
\newcommand{\Kchenoet}{Kch\"{e}no\^{e}}
\newcommand{\kchenoet}{kch\"{e}no\^{e}}
\newcommand{\Kchoho}{Kch\`{o}h\^{o}}
\newcommand{\Kchotlam}{Kch\={o}tl\'{a}m}
\newcommand{\Kchotris}{Kch\={o}tr\v{\i}s}
\newcommand{\Kihotr}{K\^{\i}h\~{o}tr}
\newcommand{\kihotr}{k\^{\i}h\~{o}tr}
\newcommand{\Kukchuatr}{K\`{u}kchu\={a}tr}
\newcommand{\kukchuatr}{k\`{u}kchu\={a}tr}
\newcommand{\Laaka}{L\={a}\"{a}k\^{a}}
\newcommand{\laaka}{l\={a}\"{a}k\^{a}}
\newcommand{\Lechoe}{L\={e}cho\`{e}}
\newcommand{\lechoe}{l\={e}cho\`{e}}
\newcommand{\Likas}{L\^{\i}k\v{a}s}
\newcommand{\Likchmas}{L\={\i}kchm\r{a}s}
\newcommand{\Likchoe}{L\^{\i}kcho\^{e}}
\newcommand{\Loem}{Lo\~{e}m}
\newcommand{\Maaras}{M\"{a}\={a}r\v{a}s}
\newcommand{\Mitchoe}{M\={\i}tcho\^{e}}
\newcommand{\Mitlikch}{M\={\i}tl\={\i}kch}
\newcommand{\Mitris}{M\={\i}tr\={\i}s}
\newcommand{\Oerchoe}{O\r{e}rcho\^{e}}
\newcommand{\Oerlikch}{O\r{e}rl\'{\i}kch}
\newcommand{\Sat}{S\={a}t}
\newcommand{\Satchir}{S\={a}tch\"{\i}r}
\newcommand{\Satrakch}{S\={a}tr\`{a}kch}
\newcommand{\Seli}{S\r{e}l\={\i}}
\newcommand{\Sirtchu}{S\r{\i}rtch\'{u}}
\newcommand{\Sitris}{S\~{\i}tr\v{\i}s}
\newcommand{\Siusiu}{S\~{\i}u-s\~{\i}u}
\newcommand{\siusiu}{s\~{\i}u-s\~{\i}u}
\newcommand{\Sogcho}{S\"{o}gch\={o}}
\newcommand{\Sotron}{S\~{o}tr\`{o}n}
\newcommand{\Tchalas}{Tch\r{a}l\v{a}s}
\newcommand{\Tchammitr}{Tch\`{a}mm\={\i}tr}
\newcommand{\Tchanoe}{Tch\r{a}no\^{e}}
\newcommand{\Tchartchaahitr}{Tch\~{a}rtch\"{a}\={a}h\r{\i}tr}
\newcommand{\Tchartchu}{Tch\~{a}rtch\'{u}}
\newcommand{\Tchitron}{Tch\"{\i}tr\`{o}n}
\newcommand{\Tchu}{Tch\`{u}}
\newcommand{\tchu}{tch\`{u}}
\newcommand{\Technku}{T\`{e}chnk\r{u}}
\newcommand{\Tesarrokch}{Te's\'{a}rr\r{o}kch}
\newcommand{\Tesatron}{Te's\'{a}tr\v{o}n}
\newcommand{\Traa}{Tr\={a}\"{a}}
\newcommand{\traa}{tr\={a}\"{a}}
\newcommand{\Trai}{Tr\r{a}i}
\newcommand{\trai}{tr\r{a}i}
\newcommand{\TraRenkchal}{Tr\r{a}-R\={e}nkch\'{a}l}
\newcommand{\Trukchual}{Tr\`{u}kchu\r{a}l}


\begin{document}

% ------ TITLE PAGE ------

\begin{titlepage}
{\centering{~\par}\vspace{0.25\textheight}
{\LARGE\tofaauthor}\par
\vspace{1.0cm}\rule{17em}{1pt}\par\vspace{0.3cm}
{\Huge\textsc{\tofatitle}\par}
\vspace{0.3cm}\rule{17em}{2pt}\par\vspace{1.0cm}
{\Large\textit{\ml{$0$}{Фантастический~роман}{Science~fiction}}\par}
\vspace{0.5cm}\asterism\par\vspace{1.0cm}
{\textbf{\ml{$0$}{Начато:}{Started:}}~\tofastarted\par}\vfill
{\Large\ml{$0$}{Создано~в}{Created~by}~\XeLaTeX}\par}
\end{titlepage}

\tableofcontents

\chapter*{Историческая справка}
\addcontentsline{toc}{chapter}{Историческая справка}

\textbf{Безопасная секция архива Ордена Преисподней\\
Выписка \#103:4AD-0000C}

~

Аркадиу Шакал Чрева через год безукоризненной работы в отделе культуры самовольно улетел с планеты.
Пятьдесят лет спустя он был обнаружен на планете Тра-Ренкхаль, где попытался возродить независимую сапиентную цивилизацию и создать союз урождённых сапиентов --- Скорбящих.
Его планы были раскрыты агентами Ада.
Аркадиу был схвачен и осуждён.
Его сообщникам, среди которых был демиург планеты Митрис Безымянный, а также некоторым из Скорбящих удалось скрыться.

23.0004.34198 Аркадиу Шакал Чрева был казнён.

Исполнитель --- Анкарьяль Кровавый Шторм.

Наблюдатели --- Грейсвольд Каменный Молот, Стигма Чёрная Звезда.

Казнь Шакала послужила прецедентом к масштабному исследованию, в результате которого оцифровывание низших форм жизни было запрещено законодательством Ада, а все уже зачисленные в штат урождённые сапиенты подвергнуты кардинальной коррекции личности.
Отделом 100 была проведена усиленная чистка.

В настоящее время достоверно выявлены следующие связи персонажей с реальными личностями:

Тханэ ар'Катхар --- Таниа Янтарь, интерфектор Скорбящих (ликвидирована).

Тхартху ар'Хэ --- Тхартху Танцующая Тень, визор Скорбящих (ликвидирована).

Митхэ ар'Кахр --- женская сущность бога Митриса Безымянного, мелиоратора Скорбящих (ликвидирована).

Атрис --- мужская сущность бога Митриса Безымянного, мелиоратора Скорбящих (ликвидирован).

Митхэ ар'Тро --- Микиа Седая, один из ведущих стратегов Скорбящих (в розыске).

Манэ ар'Люм --- Маниа Нарисованная, диктиолог Скорбящих (в розыске).

Лимнэ ар'Люм --- Лимниа Грустный Хвост, диктиолог Скорбящих (в розыске).

Записал: Кес Бледный Глаз

Проверили: Хара-лита Плачущий Клинок, Лимн Кард из клана Тахиро.

\part{Скорбящие}

\chapter{Скорбящие}

\section{[U] Ключ к себе}

\textspace

--- Кагуя, вы принадлежите к клану Тахиро, --- сказал Грейсвольд.
--- Вы полагаете, что две генерации и десять тысяч версионных правок сделали из вас урождённого демона?
Ваше ядро было и остаётся тенью человеческого мозга.

--- Я не настолько глупа, чтобы оправдывать нечто только из-за того, что это нечто свойственно мне, --- сладким голосом пропела Кагуя.
--- А если...

--- Уэсиба Серозмей, --- тихо шепнул Грейсвольд.
Лицо Кагуя залила краска, глаза сузились от гнева.

--- Моего брата от меня отделяют три тысячи версионных правок.

--- А от Тахиро вас отделяет и того больше, но вы такая же наглая и упрямая, как мой друг, --- сказал Грейсвольд.
--- Вы можете не любить Тахиро, вы можете не любить меня, но я всегда буду любить его в вас.

--- Любовь?
Я не понимаю, о чём вы говорите, --- Кагуя со скучающим видом покачала головой.

--- Значит, однажды Вселенная преподнесёт вам сюрприз, --- улыбнулся Грейсвольд.

Кагуя пожевала губу.

--- Что вам от меня нужно?

--- Мне нужно знать, о чём с вами говорил известный вам с Самаолу демон.
Кажется, его называют <<патроном>>.

--- Вы в шаге от того, чтобы подписать себе приговор, Грейсвольд, --- ощерилась Кагуя.
--- Отдел 100 сочтёт это превышением...

--- Не будьте дурой, Кагуя, --- устало сказал технолог.
--- Да, я из Скорбящих, и пожалуйста, покажите мне того, кто об этом хотя бы не подозревает.
Истинное равновесие Нэша достигнуто, вы это знаете.
Вы не можете причинить вреда мне, не погибнув.

--- Ложь, --- бросила Кагуя.
--- Какое равновесие?
Мы теряем позиции в битве с Картелем, наши демоны гибнут...

\ml{$0$}
{--- Кто гибнет? --- грустно улыбнулся Грейсвольд.}
{``But who tends to die?'' Grejsvolt sadly smiled.}
\ml{$0$}
{--- Молодёжь, не нашедшая своего места, жители планет, которые уже даже не ресурс для выживания, а топливо для войны!}
{``Young who failed to find their place, and dwellers of planets---not as vital resource, but as fuel for war!}
\ml{$0$}
{Дефицита масс-энергии давно нет.}
{Mass-energy deficiency is long gone.}
Старые иерархи как сидели, так и сидят на своих местах, и их никто не пытается устранить.
Те же, кто пытался что-то изменить в этом миропорядке, уже давно разделили судьбу вашего прародителя.

--- Информация не способна поколебать истинное равновесие Нэша, --- заметила Кагуя и притворно вздохнула.
--- Ну что это такое, я теперь играю в вашу игру!

--- Поколебать равновесие можно, --- сказал Грейсвольд.
--- Но для этого нужна жертва.

Кагуя замолчала.
Она выглядела потрясённой до глубины души.

--- В мире есть лишь одна валюта, --- продолжил Грейсвольд.
--- Эта валюта --- чувство удовлетворённости.
Её мера --- условная единица <<хорошо>>, её количество --- интенсивность в процентах, интегрированная по времени.

--- Валюта? --- захохотала Кагуя.
--- Вы циник, Грейсвольд.
Есть вещи, которые нельзя купить, и одна из них...

--- Цену имеет любая вещь, --- перебил её Грейсвольд.
--- Вашу верность нельзя купить за обычную валюту, но за удовлетворённость её покупали и продавали уже много раз.

--- Да как вы смеете! --- взвизгнула Кагуя.
--- Иногда мне приходилось жертвовать своими интересами во имя...

Женщина осеклась.

--- Это правда, --- подтвердил Грейсвольд.
--- Любое явление, любой предмет во Вселенной конвертируется в эту валюту, но обменный курс индивидуален для каждого.
Смысл существования каждого сапиента сводится лишь к вычислению этого курса.
Иногда тарелка супа и ночь спокойного сна для детей оказывается дороже жизни целого народа, но бывает и наоборот.

--- Для сапиентов --- может быть.

--- Демоны --- тоже сапиенты.
Мы отличаемся от материальных тел составом, но не принципами функционирования.

--- Будь по-вашему.
Но я что-то не могу придумать ни одной причины выставить нашему сотрудничеству высокий обменный курс.

--- Одну я вам назову, --- сказал Грейсвольд.
--- Ключ к пониманию частицы себя.

--- Больше конкретики, Грейсвольд.

--- Я облеку в форму тени, которые преследуют вас во снах.
Я дам имена тому, что не имеет имени, я объясню то, чего не сможет вам объяснить ни один корректолог.
Я расскажу вам про Тахиро --- про истинного, первого Тахиро --- то, чего вы не узнаете нигде.

Кагуя нахмурилась.

--- И взамен я должна буду сообщить вам подробности некоего разговора?

--- Нет.
Вы просто не будете стоять у меня на пути.

--- Я не могу на это пойти, Грейсвольд.
Вы опасны для Ордена.

--- Вы не поняли, Кагуя.
Это не требование, не приказ и не часть сделки.
Вы сами, желая того и понимая последствия своих действий, перестанете вставлять мне палки в колёса.
Правда сильнее любых договоров.

--- А если нет?

--- Нет так нет.
Выбор будет за вами.
Ну что, идёт?

Кагуя долго смотрела на Грисвольда прищуренным взглядом и кивнула.
Технолог едва заметно улыбнулся:

--- Я заварю чай.

\textspace

\section{[?] Сюзерен}

\textspace

Прекрасное создание размером с дом сидело на скале и смотрело на меня странными, собранными из множества фасеток глазами.
Не зная языка мимики сюзеренов, не зная о них практически ничего, я каким-то образом понял, что огромный дракон ждёт именно меня.

Мы долго смотрели друг на друга.
Сюзерен тихо, с неправдоподобно младенческим сопением вдыхал и выдыхал морозный воздух.
Ласково позванивали чешуйчатые <<перья>> на крыльях, издалека напоминающие стрекозиные крылышки или тонкие перламутровые пластинки.
Маленькие, похожие на женские руки, покрытые мелкой чешуёй, лениво шевелили тонкими пальцами.
Мои ноги в меховых сапогах начали потихоньку подмерзать, и я лихорадочно думал, как же коммуницировать с глядящим на меня крылатым ящером.

--- Человек, --- внезапно сильным и чистым голосом сказал сюзерен.

Я застыл.
Холод прошёл в одно мгновение.

--- Ты говоришь на талино? --- поинтересовался я.

--- Когда-то давно я слышал ваш язык.
Мне потребовался день, чтобы выучить его.
С какой целью ты позвал меня?

--- Я тебя не звал.

Мне стало немного жутко.
Существо, которое за один день выучило язык талино по обрывкам фраз, которое знало, что я приду именно сегодня, именно сюда...
Я посмотрел на огромную красивую голову.
Мозг превосходил человеческий по размерам как минимум в десять раз.
Какие ещё силы вложил в него девиантный бог?

Вспомнив цель визита, я начал:

--- Хоргеты Ордена Преисподней посылают племени сюзеренов предложение о дружбе и сотруд...

--- Хоргеты? --- внезапно засмеялся сюзерен.
--- Когда кто-то говорит <<хоргеты>>, мы ожидаем увидеть равного Богине-матери, а не оцифрованного человека.

Я снова застыл, скованный шоком.

--- Чего ты хочешь? --- продолжал дракон.
--- Люди не трогают нас, мы не трогаем людей.

--- Когда-то давно между людьми и сюзеренами была война.
Я не могу говорить от имени всех, но возможно, что сотрудничество позволит...

--- Не позволит, --- прервал меня дракон.
--- Ненависть людей --- лишь песни и сказания.
В нас ненависти нет, --- сюзерен неожиданно поднял огромное крыло, обнажив покорёженную, толстую кожу.
Чешуя на ней росла как попало.
Старый шрам.
--- Вы же не обижаетесь на пчёл, когда они вас жалят?

--- И всё же...

--- Мы знаем людей.
Они будут ненавидеть нас, хотя никто из ныне живущих не пострадал от сюзерена.
Кажется, это называется <<традиции>>?

--- Хорошо, хватит о людях.
Почему ты не хочешь поговорить о возможности союза с Орденом Преисподней?

--- Почему ты думаешь, что можешь обмануть меня, человек? --- ответил вопросом на вопрос дракон.

И снова мне пришлось сделать усилие, чтобы справиться со страхом.
Тон был подобран великолепно, и означал он следующее: мой собеседник прекрасно осведомлён о том, как распоряжается Орден Преисподней информацией о сапиентах.

--- Нечего сказать, Аркадиу из рода Шакалов? --- усмехнулся сюзерен.
--- Ты чересчур честен.
Плохой из тебя эмиссар.

--- Объясни мне, --- проворчал я.
--- Ты, материальное существо, которое превосходит многих хоргетов интеллектом.
Почему вы позволили людям вас уничтожить?

--- Иногда самое лучшее --- это позволить всем думать, что мы уничтожены, --- сообщил сюзерен, изящно расправив и сложив крыло.
--- Сколько голов сюзеренов ты видел?

Я промолчал.
Реально задокументированных останков этих существ было всего восемь.
Правда, историки списывали это на время и на обычай поедать мертвецов, который описывался в одной из легенд...

--- Если ты хочешь спросить, почему та война закончилась, я отвечу, что мы ушли от войны сами.

--- Отговорка униженных, --- бросил я.
--- Люди хотели извести вас под корень.
С чего вам проявлять миролюбие?

--- Неограниченное стремление к процветанию рода --- поведение низших животных, --- невозмутимо ответил дракон.
--- Однажды место и еда кончаются.
И родичи становятся врагами.

--- Если вы надеялись ударить во время всемирной войны, вы немного опоздали.

--- Нам не нужна война, человек, --- дракон даже не возвысил голоса.
--- Первые люди вели себя подобно низшим животным, играющим с технологиями.
Во время войны с нами их технологии были утрачены.

--- Вы хотели загнать людей в каменный век?

--- Обезвредить, --- поправил сюзерен.
--- Отсеять наиболее агрессивных, лишить их оружия.
Планета должна продолжать жить.

--- Так что насчёт...

--- Ордену Преисподней я отвечать не собираюсь.
Мы уже послали им своё слово: верните нам Богиню-мать.

--- Или?..

--- Без <<или>>.
Хотят сотрудничества --- пусть говорят через Богиню-мать или в присутствии Богини-матери.
Хотят мира --- да будет мир.
Хотят нас уничтожить --- это будет интересная игра, и мы желаем им удачи.
С тобой лично, человек, и твоим племенем я поговорю лет через пятьсот, когда вы наберётесь ума.

Дракон поднялся на ноги и раскрыл крылья, собираясь взлететь.
Меня обдало холодным, инкрустированным снежными иглами ветром.

--- Почему вы не уничтожили людей? --- спросил я его напоследок.
--- Они захватчики, чужаки на вашей планете!

Дракон засмеялся.

--- Потому что мы знали, что однажды люди захотят с нами поговорить.

Мощные крылья сделали первый взмах.
Веером раскрылись перламутрово-стрекозиные перья хвоста, и сюзерен, сделав величественный круг, улетел в ослепительные дали залитых солнцем вершин.

\asterism

Спустя три года после этого разговора Кох исчезла c Капитула.
Ещё через три года была проведена тотальная мелиорация Драконьей Пустоши.
Интерфекторы отдела 125 говорили, что это было единственное интересное задание за многие тысячелетия;
сюзерены же остались лишь в генетических банках Ордена Преисподней.

\section{[:] Плач по Тахиро}

\textspace

\begin{quote}
<<Тахиро, мой любимый Тахиро, который знал меня лучше, чем я сама, ушёл.
Ушёл как герой, неожиданно и не оставив надежды, передав мне Вселенную холодной и полной опасностей.
Ушёл, не успев сделать ничего, что могло бы облегчить участь песчинки в океане пространства-времени...>>
\end{quote}

Записи Айну рыдали, хотя Грейсвольд никак не мог представить Айну в слезах.
Весь его опыт общения с ней восставал против такого образа.

Грейсвольд вспомнил и другие слова.
Высокопарные слова, гремевшие в огромном зале.
Тот зал был поистине гигантским, перечёркнутый паутиной нервюр потолок (каждая <<паутинка>> на самом деле была балкой толщиной с тысячелетнее древо) казался едва ли не выше небес.
Демоны прекрасно усвоили социальные понятия обезьян --- чья ветка выше, тот и главный.

\begin{quote}
<<Величайший стратег Ада пал смертью храбрых из-за вероломства кучки сейхмар...>>
\end{quote}

\begin{quote}
<<Он должен быть отмщён!>>
\end{quote}

Самаолу говорил искусно поставленным голосом.
Даже солоноватых водянистых соплей в этом голосе было ровно столько, сколько нужно --- их отмеряли на сверхточных квантовых весах.
Аплодисменты в конце речи казались идеальными в своей мощи, хаотичности и внешней непринуждённости.
\ml{$0$}
{Но Грейсвольду было противно от мысли, что Самаолу и половина его слушателей давно искали повод прихлопнуть Тси-Ди, как спаривающихся мух.}
{But Grejsvolt hated even the thought that Samajolu and a half of his audience were always looking for an excuse to swat Qi-Di like mating flies.}
\ml{$0$}
{Ещё противнее становилось от того, что этим поводом оказалась смерть Тахиро --- мальчика Тахиро, о котором Грейс уже давно не думал, как о мальчике.}
{The only thing he hated more was another thought: the same excuse was death of Tahiro-kid; Grejsvolt didn't thought of Tajiro as a kid for ages.}

--- А ты, похоже, не особо грустишь, Грейсвольд, --- заметил Самаолу после речи.
--- Ищещь способ, как обратить его гибель на пользу \emph{себе}?

\ml{$0$}
{--- Молись кому хочешь, Самаолу, --- тихо ответил Грейсвольд, --- чтобы ни одна капля скорби старого Грейса не упала на твою голову.}
{``Pray to whatever you want, Samajolu,'' Grejsvolt answered quietly, ``that no one drop of Old Grejs' sorrow falls on your head.''}

\section{[U] Камень (переработать)}

Атрис вскоре вернулся.

--- Как результаты? --- поинтересовалась Чханэ.
--- Надеюсь, ты не наследил.

Атрис едва заметно улыбнулся на колкость воительницы.
Митхэ выглядела слегка обиженной.

--- Мне не удалось проникнуть ниже барьерной высоты.
Плюс форпостов Ада у Тси-Ди не два, а одиннадцать, --- Атрис кивнул Анкарьяль.
--- Но результаты есть.
Я достал отладочную информацию защитной системы.
Может быть, Грейс что-то сможет с ней сделать?

Атрис махнул рукой.
Грейс выпучил глаза.

--- Как ты это достал?

Атрис усмехнулся.

--- У одного из спутников Тси-Ди отказал двигатель, и он отклонился от заданной орбиты.
В результате крайняя точка его орбиты находилась в каких-то десяти километрах от барьерной высоты.
Митхэ предложила... хай... кидать в спутник камнями.

Митхэ смущённо сидела, ковыряя сапогом землю.
Анкарьяль и Грейсвольд ошеломлённо переглянулись.
Грейсвольд усмехнулся.

--- Мы вывели спутник за барьерную высоту, периодически пересиживая адские патрули.
Последним камушком мы задержали его на девять минут тридцать одну секунду.
Одна секунда мне потребовалась, чтобы разобраться с аппаратурой, оставшееся время я сниффил трафик.

--- Патрули?
Они заметили вас? --- Анкарьяль, казалось, едва сдерживалась, чтобы не схватить менестреля за рубаху и не вытрясти из него душу.

--- Всё чисто.
Патруль появился через три секунды после взрыва.

--- Какого взрыва? --- хором спросили Чханэ и я.

--- Ну не мог же я оставить спутник остальным! --- пожал плечами Атрис.
--- Сымитировал взрыв двигателя.
Значимых обломков не осталось.

Все замолчали.

--- Знаешь что, Атрис, --- начал Грейс, --- тси должны благословить тот день, когда они встретили тебя!

--- Перестань, --- отмахнулся Атрис.

--- А почему наши до сих пор не додумались камнями спутники вывести? --- задал я вопрос.

Грейсвольд крякнул.

--- Они мыслят рамочно, Аркадиу.
У них есть набор техсредств --- им и пользуются.
Я уверен, всё это время они пытались вывести спутник полями, подобраться к системам на различных волнах.
И это при том, что именно от сторонних волновых воздействий спутники и защищены.
А вот про камушки тси сами не подумали.
Ну и сломавшийся двигатель --- тоже большая удача.
Знаете же, почему меня зовут Каменным Молотом?

Все отрицательно покачали головой.
Анкарьяль прикрыла лицо рукой.

--- Старая песня...

--- Старая, но важная, Нар.
В работе нельзя брезговать никакими технологиями.
Даже камнем на палке.

\section{[U] Танец Тени}

\textspace

Тхартху сидела с закрытыми глазами.
Мелок гулял по пергаменту совершенно независимо от тела.
Послание должны получить все, но...

ПКВ вокруг вспыхнуло.
Устройство связи умерло одновременно с охранявшими дом демонами.
Тхартху, улыбаясь, нанесла на пергамент последний штрих.

Дверь открылась.
Митрис вскочил и выхватил пистолет.
Его демон привёл боевые модули в готовность.

--- Здравствуй, Нар, --- не оборачиваясь, проговорила Тхартху.

Анкарьяль вошла молча и посмотрела на Митриса.
Тот целился женщине в глаз.

--- Я нарисовала тебя, --- сказала Тхартху и, обернувшись, протянула Анкарьяль пергамент.

Демоница взяла его.
Улыбающаяся Хатлам ар’Мар держала в руках огромного броненосца.

--- Очень красиво, Тхартху, --- кивнула интерфектор.

--- Делай, что задумала, --- прошептала Тхартху и по-детски зажмурилась.
Удар --- и опустевшее тело обвисло на кресле.
Анкарьяль, вытащив пистолет, добила женщину выстрелом в голову.

Митрис вздохнул и заплакал.
Оружие выпало из его руки.

--- Пойдём, --- ласково сказала мужчине Анкарьяль, спрятав пергамент в карман.
--- Ты готов?

Митрис кивнул.
Он ещё раз взглянул на останки подруги, вздохнул, смахнул слёзы и направился к двери.
Интерфекторы ждали снаружи, но только Анкарьяль услышала его последний тихий шёпот:

--- Благодарю тебя.

\section{[U] Смерть Штрой}

\textspace

На окраине Ихслантхара рос приметный каштан.
У него было пять толстых ветвей, и это мог сказать любой уроженец.
Детьми на него лазали все, а те, кто познал под его кроной радость плотской любви, исчислялись тысячами.

Детское время давно прошло.
Возле каштана сидели трое --- двое мужчин и оцелотовая женщина, тонкая, с пухлыми губами, почти ребёнок на вид.

--- Эта явка мне не нравится, --- заявил один из мужчин.
--- Может, пойдём куда-нибудь подальше в лес?

--- Тебя не спросили, --- властно оборвала его девушка.
--- Просто делай своё дело.

Мужчина кивнул и, сбросив одежду, прижал девушку к каштану.
Спустя десять михнет запыхавшегося любовника сменил второй.
Ещё через пятнадцать михнет оба взяли свои фонари и удалились в сторону города.
Девушка осталась сидеть одна в темноте, словно о чём-то размышляя.

Вскоре округу огласил её негромкий ликующий смех:

--- Этот ход остался за мной.
Что ж, можно и выпить.

Тихо звякнула крышка автоматической фляжки, и в чашу полилась шипящая жидкость.

--- Какая прелесть, --- захихикала девушка.
--- Даже звук при открытии идеальный.
Умели же эти выродки делать фляжки...

Вдруг темноте показались три фигуры.
Девушка вскочила на ноги, едва не расплескав отвар.

--- Здравствуй, Штрой Кольцо Дыма, --- приветливо сказала высокая женщина.

--- Анкарьяль, Мимоза, Ду-Си, --- склонила голову девушка.
--- Прошу прощения, я вас не признала.
Не желаете ли выпить?

Мимоза и Ду-Си переглянулись.

--- Как там говорится в <<Шляпе>>, Мими?
Вежливо ответить на вопрос, да?

--- Ещё раз назовёшь меня <<Мими>>...

---  ... получишь пенальти на три ранга.
Я его уже получил декаду назад и вчера, Мими.
\ml{$0$}
{Я легионер, ранги закончились!}
{I'm a leggionaire, I'm out of ranks!''}

\ml{$0$}
{--- Да чтоб ты языком подавился, легат.}
{``Choke on your tongue, legate.''}

\ml{$0$}
{--- О, я снова легат?}
{``Whoa, I'm a legate, again?}
\ml{$0$}
{Ничего себе повышение.}
{Quite a promotion.''}

--- Как я ненавижу уроженцев Чёрной Скалы.
Кто-то из них жесток без меры, кто-то умён без меры, а кто-то дурачок от рождения, как ты.
Была бы моя воля --- я бы вашу проклятую планету превратил в звёздную пыль, чтобы такой ошибки, как появление тебя на свет, больше никогда не случилось.
Да, дьявол тебя раздери, мы должны вежливо ответить на вопрос!

Ду-Си и Мимоза повернулись к Штрой и хором сказали:

--- Благодарим, но вынуждены отказаться.

--- Мы знаем, что это ты сообщила последней из тси ключи от боевых систем и спровоцировала её на диверсию, --- сказала Анкарьяль.

Чаша и фляга дробно звякнули о корни каштана.

--- Благодаря мне была раскрыта шпионская сеть, максим, --- затараторила Штрой, обращаясь к Мимозе.
--- Я не могла действовать открыто из-за контаминации наших рядов.
Единственная моя ошибка заключалась в том, что я отнесла подполье к шпионской сети Картеля, а не к отдельной организации.
Я служила Аду и прошу справедливого суда своим деяниям...

--- Ад благодарит тебя за верную службу, --- мягко прервал её интерфектор.

Удар --- и опустевшее тело девушки упало на землю.

--- Отчёт в одиннадцать предоставите командующему, --- бросил Мимоза спутникам.
--- Ду-Си, могу ли я попросить вас захватить сладости?
Может быть, вы помните булочки некой Ликхэ ар’Митр э’Кахрахан, пышные и ароматные.
Думаю, до завтрака посыльный успеет долететь с Могильного берега, если вы отправите сообщение сейчас.

--- Сделано, максим, --- ухмыльнулся легат.

--- Отлично.
А вы, Анкарьяль, подберите фляжку и снимите со Штрой серёжки.
Таким вещам место в музее, а не на дороге.
И ещё --- обставьте всё как самоубийство, чтобы местные не волновались.

Анкарьяль кивнула и занялась телом.

\section{[U] Микоргет (переработать!)}

\textspace

--- Ты хочешь сказать, что тси пытались создать... материального микоргета? --- ошарашенно спросила Чханэ.

Атрис замялся.

--- Очень похоже на это.
Кольцевая теплица --- совершенное создание, практически вечное.
Она вбирала в себя также личности всех, кто желал обрести вечную жизнь...

--- Предкам была противна даже мысль, что их потомки могут пойти путём хоргетов, --- сказала Митхэ.
--- Они решили искать собственный путь бессмертия.
Теперь оставшиеся в живых спят в телах кольцевых теплиц.
И Машина об этом не подозревает --- возможно, что и эти эксперименты совершали в тайне...

--- Машина подмяла нас под себя, --- сказал Атрис.
Чханэ удивлённо посмотрела на менестреля --- он впервые сказал <<нас>>, отнеся себя таким образом к тси.
--- Мы относились к ней, как к бездушному слуге.
Но как только слуга становится умнее хозяина, хозяин превращается в слугу.

--- Или в труп, --- присовокупила Чханэ.
--- Так или иначе, для нас это всё бесполезно.
Нужно найти безопасное место --- Аркадиу схватили.

--- Зачем? --- усмехнулась Митхэ.
--- Отправимся туда, где сейчас опаснее всего --- там нас вряд ли будут искать.

Чханэ испытующе посмотрела на Митхэ и спустя мгновение понимающе кивнула.


\chapter*{Интерлюдия X. Чумное поветрие}
\addcontentsline{toc}{chapter}{Интерлюдия X. Чумное поветрие}

То был год демона --- спустилось на Трогваль чумное поветрие.
Забили гонги в храмах, и каждый удар провожал к Творцу раба его.

Опустел Город Мастеров, и огласились плачем улицы его.
Угрюмо замолчал Стальной квартал, ощерившись металлом против обездоленных и больных.
Приутихло веселье в торговых рядах.
И лишь Проклятый город стоял, как и прежде --- избежала его кара Творца.

И поползли слухи, что не пророка длань отсыпала горечи Трогвалю, но колдовство безбожников.
И варились эти слухи в котле отчаяния --- чем слаще рай ушедшим, тем горше юдоль оставшимся.
Как спало поветрие, заскрежетали в домах точильные камни, застучали ступки, толкущие порох.

Спал Проклятый город, утомлённый трудами дневными и играми вечерними.
Спали мужи, спали жёны, спали отроки и девы, спали дети в колыбелях своих.
И не открыл им Творец, что гнев Его уже в пути, звенит смертоносной сталью.

И лишь Шамаль не спал.
Он, и жена его, и дети его слушали истории Марина под светом звёзд.
За занятием этим и застал их трогвалец-кузнец, опередивший армию на звук дыхания\FM.
\FA{
Имеется в виду, что кузнец бежал со всех ног, в то время как прочие трогвальцы шли спокойным шагом.
}

<<Мир и блага земные тебе, кузнец, --- поприветствовал его Шамаль.
--- Стряслась ли беда, что ты захлёбываешься воздухом, дарованным нам Творцом для наполнения лёгких и услады головы?>>

<<О Шамаль, --- заговорил кузнец.
--- Да простят меня дом и крыша твои за непочтение к ним, но бежать бы тебе на край острова.
Чума породила сталь, и многие хотят смерти тебе, и семье твоей, и друзьям твоим>>.

Опечалился врачеватель, услышав речи кузнеца.
И опечалился он ещё больше, когда понял, что не может отвести друзей от серых врат\FM.
\FA{
В сказаниях тенку <<серыми вратами>> называется смерть, чаще всего насильственная.
}

<<Почему же ты, кузнец, пошёл против воли пророка?>> --- спросил врачеватель.

<<О Шамаль, --- ответил кузнец, --- видят очи Саттама, грешен я перед ним, что отвожу гнев Творца от тебя.
Но жизнь любимой дочери, спасённая умением твоим, не дала мне поступить иначе>>.

<<Не перевелись ещё люди в Трогвале>>, --- сказал Шамаль и проводил кузнеца с благодарностью.

Догорела ли свеча, догорело ли солнце, но увёл Шамаль семью, и соседей, и гостя в лес.
И Марина в лес увёл, чтобы не нашли трогвальцы его.
Дал врачеватель другу лук и стрелы, чтобы смог тот поймать летающий остров.

<<Уходи, чужеземец, --- сказал Шамаль.
--- Ты не найдёшь покоя в Трогвале.
Да помогут тебе люди в твоих поисках!>>

И убили в тот день Салема, и убили Магата, и многих достойных мужей убили в тот день.
И ходили трогвальцы по улицам Проклятого города, и убивали, и насиловали с именем пророка на устах...

\part*{Приложения}
\addcontentsline{toc}{part}{Приложения}

\appendix

\chapter{Планеты и места}

\section{Баланс}
 
\theterm{balance}
{Баланс}
{Спутник Сомерскай, одна из самых примитивных планет со стабильной жизнью.
Диаметр Баланса составляет всего одну пятую диаметра Сомерскай.

На планете живёт один полиморфный биологический вид.
Разные жизненные стадии особей вида являются продуцентами, консументами и редуцентами.

Данные об оживлении Баланса утеряны;
в настоящее время считается, что планета была оживлена в рамках проекта <<Золотая Ладья>> лаборатории Кошкина.
Тем не менее существует мнение, что жизнь на Балансе носит отпечаток деятельности неизвестного бога.

Планета непригодна для проживания Ветвей Земли.
Согласно решению Ордена Преисподней, планета Баланс является памятником живой природы и подлежит охране.
Доступ к планете имеет только лаборатория Веве Волосяная Кукла, входящая в отдел девиантной биологии.}

\section{Диана}

\theterm{diana}
{Диана}
{Погибшая искусственная планета системы Древнего Солнца, вращающаяся по сложной орбите в области Земля---Марс.
Первая планета, созданная с помощью демиурга.
Считается основной причиной гибели Древней Земли --- из-за дестабилизации планетной системы Дианой все три обитаемых планеты системы Древнего Солнца были разбомблены астероидами.}

\section{Драконья Пустошь}

\theterm{drake-desert}
{Драконья Пустошь}
{Холодная планета системы голубого гиганта.
Более 2/3 поверхности покрыто тундрой и ледяной пустыней, небольшая часть вдоль экватора --- тропический климат.
Характеризуется огромными залежами ртути и рубидия (подземные ртутные озёра).}

\asterism

\theterm{dropleaf-mountains}
{Горы Листопада}
{?}

\theterm{via-galoledica}
{Виа Галоледика}
{?}

\theterm{talia}
{Талиа (\theorigin{t-sl}{Talia}{чрево})}
{Крупное государство с монархическим строем на планете Драконья Пустошь, родном мире Аркадиу Люпино.}

\section{Древняя Земля}

\theterm{old-earth}
{Древняя Земля}
{Материнская планета Ветвей Земли.
В настоящее время необитаема.}

\section{Запах Воды}

\theterm{smellwater}
{Запах Воды}
{?}

\section{Капитул}

\theterm{capitul}
{Капитул (Сомерскай)}
{Планета, преобразованная богом Brahma-23 (проект <<Золотая ладья>>), созданным лабораторией Кошкина в Калькутте.
Лаборатория прославилась очень смелыми, энергозатратными экспериментами и тем, что практически все научные сотрудники были женщинами.
Трёхкратный лауреат Расширенной Нобелевской премии Хезер Коллинз, разработавшая методологию преобразования планет, также начинала свою деятельность в кошкинской лаборатории.

Согласно данным, это первая планета, на которой были применены методы искусственного (проводникового) наведения магнитосферы и стабилизации климата с помощью Кориолисовой градиентации парниковых газов (КГПГ).
Вследствие этого на большей части суши умеренный климат.
Полупустыня и тундра --- самые суровые биомы Сомерская --- вместе занимают около 1000 км$^2$.

В настоящее время является главной базой (Капитулом) Ордена Преисподней.

Населён Сомерскай видами-представителями абсолютно всех известных Ветвей, за исключением Ветвей Ночи (около 140 видов).
Одна из самых густонаселённых планет (30 млрд особей).

Сомерскай имеет один живой спутник --- Баланс --- и три неживых --- Роза, Пион, Хаос.
Кроме того, на орбите Сомерская есть некоторое количество искусственных планет (форпостов) с боевыми механизмами.}

\asterism

\theterm{scage}
{Скаге (Скальдборо-Гелиополь)}
{Крупнейший городской конгломерат на Капитуле.}

\section{Лотос}

\theterm{lotus}
{Лотос (Дагон)}
{Планета системы Фомальгаута, одна из первых заселённых планет.
Демиург --- Грейсвольд Каменный Молот.
Первые поселенцы --- экипаж <<Тёмного Пламени>>, капитан --- Бенедикт Альсауд.

Первоначальное название планеты --- Дагон.
Планета была переименована в Лотос после преобразования.}

\asterism

\theterm{granoble}
{Гранобль}
{?}

\theterm{chercherotte}
{Шершерот}
{?}

\section{Марс}

\theterm{mars}
{Марс}
{Вторая оживлённая планета в системе Древнего Солнца.
В настоящее время Марс находится под властью Красного Картеля, на планете обитают одичавшие племена людей и кани.}

\section{Мороз}

\theterm{moros}
{Мороз}
{Планета системы Арракиса с весьма суровым климатом.
Преобразована была во времена поздней Эпохи Богов.
Демиург (Эйраки Мороз) из-за ошибки физиков сделал среднюю температуру на планете гораздо ниже, чем требовалось.
В приэкваториальных поясах летняя температура --20°С, зимняя до --90°С. На полюсах --150°С, жизни нет.
Несмотря на то, что проект был закрыт, спустя почти столетие нашлись сапиенты, которые решились на заселение этой планеты.
Специально для них учёными Древней Земли были спроектированы поселения (<<bzec>>, бижеч --- на языке русе <<укрытие>>), а также холодоустойчивые живые существа --- нимелто, коно, клучо и прочие.

Сапиенты Мороза представлены двумя видами: люди (род Медведя), кани (род Волка).
Вся жизнь их направлена на сохранение тепла.
Сапиенты вместе со стадами кочуют по планете, следуя за солнцем, от одного бижеч до другого.
Самые древние бижеч обогреваются геотермальными водами из пробурённых скважин, более поздние строились возле естественных вулканов и разломов.
Также многие поселения получают энергию от <<чёрных полей>> --- кремниевых щитов, расположенных в закрытых от ветра низинах.
Полученная энергия почти полностью расходуется на поддержание теплиц.

Кани и люди живут вместе.
Все сапиенты носят очки, маски и одежду специального покроя из шкуры нимелто.
Общение вне поселений только жестовое.
В поселениях принято ходить без одежды и спать скоплениями по 8--10 особей для сохранения тепла.
Дети спят в центре, взрослые по бокам.

Комбинезоны из нимелтового меха имеют особую конструкцию.
Ни одна наружная, соприкасающаяся с внешней средой деталь не контактирует с внутренними, прилегающими к телу --- между ними всегда как минимум два слоя воздуха и два термоизоляционных шва.
Внутри комбинезона также есть особые воздухоносные пути --- через них идёт согреваемый воздух к маске и тёплый, насыщенный водяными парами --- от маски.
Особая система обеспечивает циркуляцию сухого воздуха в очках, что препятствует их запотеванию.
С помощью пищевого шлюза сапиенты могут даже кушать, не вдыхая воздух из внешней среды.

Язык общения --- рут (русе) --- единый для всех, очень ёмкий и лаконичный, практически не менялся со времени заселения.

Общество Мороза --- самое невоинственное в известной Вселенной.
Стычки между членами племени очень редки, случаи драк и убийств не зафиксированы, обычаями племени эти казусы не регулируются.
Самым страшным <<преступлением>> считается оставление включенной лампы на время сна, за него предусмотрено самое суровое <<наказание>> --- трёхчасовая одинокая прогулка.
Для решения споров используется <<сидение>> --- особи садятся друг напротив друга и сидят около часа неподвижно.
После такого молчаливого <<разговора>> спор обычно решается.

Тем не менее, несмотря на миролюбие жителей Мороза, у них существуют человеческие жертвоприношения.
Из-за климата достаточно частой травмой являются отмороженные глаза.
Общинники с отмороженным правым глазом называются Красный Снег, с левым --- Синий Снег.
Те же, кто лишился обоих глаз, называются Одно Лето.

В отличие от тех, кто пал в дороге, Одно Лето пользуются большим уважением.
Их очень хорошо кормят и выслушивают все их речи --- считается, что устами Одно Лето говорят высшие силы.
Когда приходит время миграции, Одно Лето усаживают недалеко от бижеч, чаще всего под скалой, защищённой от ветров.
Затем лидер отряда своими руками расстёгивает им комбинезоны на груди, отчего Одно Лето умирают в течение двух минут.
Считается, что принесённые в жертву слепые охраняют пустые бижеч от тёмных сил, а непобеждённая сила их тел передаётся соплеменникам.

В силу малой заселённости планета нейтральна по отношению к хоргетам, но Ад и Картель держат там наблюдателей.}

\asterism

\theterm{bizec-mol}
{Бижеч Мол}
{?}

\section{Преисподняя}

\theterm{netherworld}
{Преисподняя}
{?}

\asterism

\theterm{akiyama-volcano}
{Акияма (вулкан)}
{?}

\theterm{akiyama-town}
{Акияма (город)}
{?}

\theterm{arashiyama}
{Арашияма}
{?}

\theterm{asahina}
{Асахина}
{?}

\theterm{kiba}
{Киба}
{?}

\theterm{minamiyama}
{Минамияма}
{?}

\theterm{takesako-city}
{Такэсако (город)}
{?}

\theterm{takesako-dale}
{Такэсакское ущелье}
{?}

\theterm{hanayama}
{Ханаяма}
{?}

\section{Тра-Ренкхаль}

\theterm{tra-renkchal}
{Тра-Ренкхаль}
{?}

\asterism

\theterm{sanct3}
{Весёлый Волок}
{Святилище сели, тенку и ркхве-хор.}

\theterm{sanct6}
{Гора Песнопений}
{Святилище ркхве-хор и Красного колена.}

\theterm{two-and-two-cities}
{Двунадваградье}
{Устаревшее поэтическое название востока Короны, а также четырёх старейших городов тси: Кахрахана, Тхартхаахитра, Тхитрона и Травинхала.}

\theterm{deepdale}
{Лощина}
{Города-государства Синего колена.
Травники прошли по Дороге Жизни --- разлом, ведущий через смертные пустоши к Лощине.
Согласно легенде, утро застало переселенцев в Смертных пустошах, и травники уже приготовились к смерти, как вдруг увидели летящую на юг птицу --- тигровую сову.
Сова и вывела племя к Дороге Жизни.
С тех пор Синее Колено поклоняется совам;
в их легендах говорится, что солнце давно сожгло бы мир, если бы не Великая Алмазная Сова, которая закрывает мир своими прозрачными крыльями.}

\theterm{mikchan}
{Микхан}
{Культура идолов в Западном Живодёре со столицей в одноимённом городе.
Являются одними из последних тси, сохранивших технологию.
После Закатного Переселения к Морю Микхан пришёл в упадок, а большая часть идолов мигрировала в Молчащие леса, дав начало племенам Молчащих идолов.}

\theterm{sanct4}
{Одинокий Столб}
{Святилище сели и хака.}

\theterm{sanct5}
{Ожидание Вести}
{Святилище сели, ноа и трами.}

\theterm{sanct1}
{Омут Духов}
{Святилище сели и идолов Живодёра.}

\theterm{sanct7}
{Прибой}
{Святилище тенку, зизоце и ркхве-хор.}

\theterm{sanctuary}
{Святилище}
{Поселение с особым статусом, находящееся чаще всего на границе земель.
В святилище запрещено ношение оружия --- его сдают у ворот и хранят на специальных складах.
Скрытое ношение оружия, драки и оскорбления в святилище приравниваются к двум Разрушениям и караются смертью.
Управляет святилищем особый вид совета --- Сцепленные Руки.
Святилища можно даже считать отдельными городами-государствами, так как формально Сцепленные Руки не подчиняются лидерам составляющих город племён.
Всего на Тра-Ренкхале насчитывается семь действующих святилищ.}

\theterm{se-tchitr}
{Опалённая Речушка (Се'тхитр)}
{Река в северных землях сели, впадает в Ху'тресоааса, в устье находится Тхитрон.}

\theterm{risible-swamp}
{Смешливая Топь}
{Биогеоценоз в Западном Суболичье.
Микрофлора имеет уникальную особенность --- круговорот азота идёт через стадию свободного оксида азота (I).
Смешливая Топь опасна для практически любого животного Ветвей Земли;
тем не менее, там проживают 144 вида девиантных эндемиков-стенофагов, связанных в сложную пищевую цепь.
Орден Преисподней объявил Смешливую Топь объектом живой природы, нуждающимся в наблюдении и охране.}

\theterm{sanct2}
{Тёплый Двор}
{Святилище сели и пылероев Предгорий.}

\theterm{rivertangle}
{Тысячеречье}
{---}

\theterm{hedgehog-spine}
{Хребет Дикобраза}
{---}

\theterm{hu-tresoaasa}
{Ху'тресоааса (\theorigin{tn}{hu'tresoaasa}{река, которая просит пить})}
{?}

\theterm{abode}
{Четыре Обители}
{Святилища, в которых выхаживали больных.
Если человек безнадежно заболевал и хотел излечиться, то он отправлялся в обитель, где специально обученные жрецы проводили сложные, известные только им операции.
Иногда -- если болезнь была неизлечима или неизвестна жрецам -- больной добровольно отдавал себя на эксперименты, и это расценивалось как жертвоприношение.
Исцеленные в обителях давали клятву отработать в обители в течение восьми или шестнадцати дождей -- в зависимости от сложности случая.
Каждый первый год Церемонии обители брали на обучение чужих жрецов.
Обучение длилось ровно три дождя.
Прошедшим обучение набивалась особая татуировка на руке --- переплетение четырех цветков Короны и три иероглифа, кодирующих имя жреца.
Такую татуировку носили Кхатрим, Саритр, Трукхвал и Король-жрец Митрис ар'Люм.
Обители носили имена Пепельной, Снежной, Каменной и Песчаной.}

\theterm{oerho-loeaesakch}
{Эрхо'люаэсакх}
{Извилистое опасное течение между Короной и Кристаллом.}

\section{Тси-Ди}

\theterm{qi-di}
{Тси-Ди}
{Двойная планета системы белого карлика, старое название --- Мерлин-Ниниана.
Общепринятое, вероятно, от чайнис 启迪 (Q\v{\i}-D\'{\i}) --- <<вдохновение>>.
Планеты имеют примерно одинаковую массу, вследствие чего обращаются вокруг равноудаленной от планет точки --- Центра Масс.
Позже Центром Масс стали называть расположенную в этой точке станцию --- в ней располагалась научно-исследовательская база, центр полётов между планетами и узел стабилизации планетарной системы защиты.

Планеты всегда повёрнуты друг к другу одной (океанической) стороной.
На планете Тси имеются 14 материков, связанных Паутиной --- дорогами на силовых полях.
Обитаемая зона планеты Ди --- 10 тысяч километров по периметру океана, покрытые лесостепью и вечнозелёными кустарниками.
Всё остальное пространство --- каменистая пустыня с разреженной атмосферой, в которой располагаются рабочие и исследовательские механизмы и Оазисы --- закрытые станции для проживания сапиентов.

Ранее Тси-Ди была обиталищем народа тси, впоследствии тси были уничтожены Машиной.
Согласно данным, в настоящее время на планете Тси-Ди не обитает ни один сапиентный вид Ветвей Земли.}

\section{Тысяча Башен}

\theterm{thousand-towers}
{Тысяча Башен}
{Жидкое ядро с кристаллической поверхностью.
Поверхность состоит из гигантских Друз, разделённых полосками воды в глубоких ущельях --- Трещинах.
Единого океана нет.
Соответственно, из-за вращения планеты и прочих причин скорость воды очень высока, Друзы постоянно стачиваются, но тут же нарастают из-за насыщенных тем же веществом вулканических газов.
Также Друзы от стачивания спасают колониальные раковинные хемосинтетики, использующие энергию вулканических газов;
интенсивное нарастание этого бактериально-протозойного мата, обладающего противотурбулентными свойствами, спасает друзы от вымывания.
Они же вырабатывают свободный кислород, необходимого для дыхания.
Из Друз торчат Башни (отдельные крупные кристаллы), которые регулярно подвергаются мощнейшему выветриванию.

Всего на тысяче башен 138 друз, 79 из которых являются обитаемыми.
Прочие либо слишком малы, либо низкие (их поверхность находится в Газовом Океане), либо чересчур высоки.

Самыми благоприятными для жизни считаются Жеоды --- чашевидные кристаллы, заполненные водой.
Жеоды отлично подходят для земледелия и рыбоводства;
однако из-за них разгораются самые жестокие войны, иногда длящиеся сотни лет.
Одним из таких мест является плодородная Кровавая Чаша, граничащая с тремя крупными Друзами --- война на ней не утихала более двух тысяч лет.

Пейзаж напоминает чем-то каньоны Северной Америки.}

\asterism

\theterm{townhenge}
{Висячие руины (повешенные города)}
{Покинутые или вымершие поселения, которые из-за нарастания Друз поднялись на высоты, непригодные для дыхания.
Очень часто висячие руины находят под ледниками, покрывающими высокие Башни.}

\theterm{gas-ocean}
{Газовый Океан}
{Тяжёлые газы в Трещинах, которые скапливаются над поверхностью воды.
Поверхность Газового Океана гораздо выше водной, из-за него некоторые Друзы непригодны для жизни.}

\theterm{garrota}
{Гаррота (астероидное кольцо)}
{---}

\theterm{eisenloon}
{Железная Луна}
{---}

\theterm{ounce-ring}
{Кольцо Барса (астероидное кольцо)}
{---}

\theterm{goat-ring}
{Кольцо Козла (астероидное кольцо)}
{---}

\theterm{qyschlau}
{Кышлау}
{---}

\theterm{crossing}
{Переправа}
{Место, где можно почти без проблем перелететь на глайдере с одной Друзы на другую.
Все переправы односторонние.
Переправой считают путь, по которому совершили перелёт не менее двадцати воздухоплавателей и не менее восьмидесяти процентов из них выжили.
Переправы бывают постоянные и сезонные (зависимые от ветров).}

\theterm{rotesturm}
{Ротештурм}
{---}

\theterm{sarland}
{Сарланд}
{---}

\theterm{essloonen}
{Съедобные Луны (Эсслунен)}{---}

\theterm{syjyrsyq-oiahy}
{Сыйырсык-ояхы}{---}

\theterm{flockenblume}
{Флокенблуме}
{Древняя, поныне неприступная крепость на Вольке, расположенная в микродрузе. После пожара, уничтожившего деревянные стены, там был построен монастырь, обучавший искусству Туманной Войны. Флокенблуме является традиционным местом сбора для переговоров и в случае набегов (Хильденфламме). Кроме того, каждый торговец отчисляет в пользу Флокенблуме натуральный налог (флокенгильд --- около 2--4\%, в зависимости от объема и типа товара), который идет нуждающимся и используется в случае голода.}

\theterm{himmelrot}
{Химмельрот}
{---}

\theterm{holzhafen}
{Хольцхафен}
{---}

\section{Чёрная Скала}

\theterm{black-rock}
{Чёрная Скала}
{Одна из первых планет в Красном Картеле.
Сухая вулканическая планета, обращающаяся по гигантской сложной орбите вокруг тусклого красного гиганта.
Отсутствия многих необходимых ресурсов (в частности, именно на Черной Скале был разработан спектр так называемых <<чёрных>>, основанных на железе сурроганиумов --- из-за практически полного отсутствия титана).

Чёрная Скала --- один из первых планетарных питомников антарид, в настоящее время практически заброшена Картелем из-за заражённости антаридами.
Манипула Смеха признала, что единственный доступный способ вывести антарид с Чёрной Скалы --- это физическое уничтожение планеты.}

\chapter{Языки и письменность}

\theterm{abis}
{Абис}
{Письменность ноа.
Слоговая, звуко-буквенная, идеограммы для грамматических элементов.}

\theterm{bi}
{Би}
{Всеобщий язык Древней Земли, попытка связать язык машины и язык биологической нейросети.
Был разработан за тысячу пятьсот лет до гибели цивилизации Древней Земли.}

\theterm{snake-script}
{Змеистая письменность}
{Единственный в своём роде звуко-буквенный вид письменности, используемый тси-язычными племенами.
Используется Молчащими идолами, говорящими на языке хесетрон.
Автор неизвестен.
Основан на математических символах древних тси.
В настоящее время упрощённый вариант используется диктиологами Ордена как фонетический алфавит тси-подобных языков.}

\theterm{mosquito-script}
{Комариная письменность}
{Предположительно письменность исчезнувших травников Синего колена.
Является потомком идеографической письменности тси.
Все ключи и модификаторы пишутся вдоль линии, зеркально дублируются сверху и снизу (видимо, из соображений сохранности).}

\theterm{mikchan-cyphers}
{Микханская тайнопись}
{Записываемый змеистым письмом код, образцы найдены в катакомбах города Микхан в Западном Живодёре.
На сегодняшний день найдена переписка из 230 кусков коры, в которой участвовали трое (по мнению некоторых исследователей --- четверо) идолов Микханской культуры.
В основном представляет собой картинки с подписями.
Исследователи склоняются к мысли, что тексты --- автоматическое письмо и служат для отвлечения внимания от рисунков, в которых содержится основной смысл посланий.
Микханская тайнопись не дешифрована по сей день.
В переносном значении --- нечто недоступное пониманию и, возможно, бессмысленное.}

\theterm{dream-pidgin}
{Пиджин снов}
{Сапиентный язык, который использовался демонами как код.
С помощью специального алгоритма слова в языке перепутывались.
Несмотря на строгое соответствие грамматике исходного языка, сообщение на пиджине снов для сапиентов выглядело как полная бессмыслица, но легко раскодировалось демоном, имеющим большие вычислительные мощности.
Пиджины снов практически вышли из употребления после изобретения языка Эй.
Один из самых известных --- сохтид (который, тем не менее, не является настоящим пиджином снов, а принадлежит к так называемым комбинированным кодовым языкам).}

\theterm{supported-language}
{Поддерживаемый язык}
{?}

\theterm{sketches}
{Резы}
{Подобие письменности у стрелохвостов.
Представляет собой нанизанные на верёвку каменные бусины, косточки, куски акульей чешуи и обработанной кожи с насечками.
Есть данные, что у стрелохвостов Голубого Зеркала имеется <<библиотека>> резов --- Грот, в которой собраны работы по медицине, истории и промыслам.}

\theterm{ruse}
{Русе}
{Основной язык Мороза, произошедший из одного из языков Древней Земли.}

\theterm{sarqort}
{Cаркорт (сарлид, гельблид)}
{Язык друзы Тартария и Сарланда на Хербст.}

\theterm{sectum-lingua}
{Сектум-лингва}
{Поддерживаемый язык Красного Картеля на основе эллатинского языка Древней Земли.}

\theterm{si}
{Си (Си-поинт)}
{Всеобщий машинный язык, созданный на Древней Земле.
Может быть использован практически для любого типа квантовых, световых и электронных архитектур.
Используется до сих пор при программировании хоргетов.}

\theterm{sojtid}
{Сохтид}
{Комбинированный демонический пиджин планеты Преисподняя, поддерживаемый язык Ордена Преисподней.}

\theterm{shp}
{Стандартная человеческая фонетика (СЧФ)}
{Спектр звуков, которые способен произвести неизменённый голосовой аппарат особи \textit{Homo homo sapiens}, а также система записи этих звуков.}

\theterm{talino}
{Талино}
{Дикий язык планеты Драконья Пустошь на основе сектум-лингва.}

\theterm{qi-language}
{Тси (язык)}
{Язык Тси-Ди, предок тси-подобных языков.
Одной из теорий происхождения языка тси является так называемая дельфинья.
В пользу этой теории говорит происхождение корней --- многие из них являются дельфиньими заимствованиями, в том числе характерными только для дельфинов звуковыми отпечатками предметов.
Также в языке тси есть множество понятий, которые в принципе отсутствуют у любых сухопутных сапиентов, но есть в языках акарид и дельфинов.
Тем не менее, в настоящее время теория считается несостоятельной, так как грамматика тси более напоминает языки человеческих и плантских племён планеты Ди, нежели любой из известных дельфиньих языков.}

\theterm{hanyui}
{Ханьюи}
{Язык Древней Земли, один из основных языков ранней Чёрной Скалы.
Язык шакуната является пиджином снов языка ханьюи.}

\theterm{tesatron}
{Цатрон}
{Лингва франка планеты Тра-Ренкхаль.
Изначально --- язык племени царрокх.}

\theterm{ej}
{Эй}
{На Древней Земле: гипотетический универсальный язык.
Современное значение: созданный Ликаном Безруким поддерживаемый язык, не претендующий на универсальность, но весьма удобный и потому распространившийся по Вселенной.}

\theterm{emoglyph}
{Эмоглиф}
{Элемент письменности сели для записи настроения или отношения сапиента к написанному.
Разновидность эмоглифа --- филоглиф, кружочек или кардиоида, которые ставятся над именем тси, к которому автор испытывает сильную привязанность или уважение.
В устной речи классического тси эмоглиф указывает на интонацию;
в нескольких языках современных тси-язычных племён эмоглиф имеет своё собственное чтение, чаще всего в виде соответствующего эмоции междометия.
Предусматривает 4096 оттенков эмоций, универсален для всех сапиентных видов.}

\theterm{englis}
{Энглис}
{Основной язык Древней Земли с эпохи Последней Войны.}

\theterm{erdenlied}
{Эрденлид}
{Официальный язык друз Хербст и Вольке.
Включает в себя диалекты: нойлид --- центральная часть Хербст, бергенлид --- Сарланд, эрденшпрак высокий, старый и приречный --- Вольке.}

\theterm{cell-script}
{Ячеистое письмо}
{Торгово-дипломатическое письмо травников.
Напоминает соты.}

\chapter{Шкала Яо}

Барьеры развития --- ключевые моменты для цивилизации, связанные со средним уровнем развития живых существ (по Яо).

\begin{description}
\item [80] --- барьер Начала: существо способно использовать материальные объекты для своих нужд, но не способно к направленному их преобразованию.
Зарождение религии.
\item[100] --- барьер Культуры: существо способно осмысленно и направленно преобразовывать материальные объекты и приспосабливать их к своим нуждам.
Зарождение технологии.
\item[130] --- барьер Цивилизации: существо способно к передаче информации с помощью материальных символов.
Зарождение науки, разработка научной методологии.
\item[200] --- барьер Эмпатии: первый критический момент.
Существо способно коммуницировать и сотрудничать с любыми другими существами.
Выход в космос.
Существует опасность само- и взаимоуничтожения.
Преодолению барьера Эмпатии способствует выработка универсального морального кодекса.
\item[400] --- барьер Создателя: второй критический момент.
Технологическая сингулярность, создание существ c возможностями, превышающими собственные, но находящихся на более низком уровне развития.
Опасность уничтожения созданными существами.
Преодолению барьера Создателя способствует помощь новосозданным существам в преодолении ими барьера Эмпатии, приобщения их к универсальному моральному кодексу.
\item[1000] --- барьер Наблюдателя: существо видоизменяется настолько, что уже не может быть отнесено к породившему его виду.
Полиморфизм, способность свободно перемещаться внутри и вне Вселенной, бессмертие (способность использовать любые источники энергии, неразрушимость и неизнашиваемость).
\end{description}

\textbf{Примечание.}
1000 --- условное число.
Согласно последним данным, барьер Наблюдателя, который должен быть преодолён \textit{микоргетом}, находится в промежутке между 873 и 1090.

\chapter{Классификация Ветвей}

\section{Ветви Земли}

\theterm{earth-forks} % Earth Forks
{Ветви Земли}
{?}

\subsection{Сапиентные виды}

\subsubsection{Ветвь Хуманы (Люди)}

\theterm{human-fork}
{Люди}
{Самые успешные сапиенты.
Во Вселенной в настоящее время насчитывают около 98 тысяч видов-потомков.}

\asterism

\theterm{lotids}
{Лотиды (не путать с лотинами)}
{Аборигенные люди Тра-Ренкхаля.
В настоящее время представлены народами хака, тенку, зизоце и прочими.
Представляют собой смесь потомков первых людей Древней Земли и прибывших чуть позже людей Лотоса.
Скрещивание между ними предотвратило появление репродуктивного барьера.
От первых людей отличаются сильно развитым половым диморфизмом, свойственным людям Лотоса, от бледных людей Лотоса отличаются коричнево-чёрным цветом кожи и радужных оболочек глаз.
Также имеются приобретённые позднее особенности --- в частности, у 56\% людей Тра-Ренкхаля транспозиция внутренних органов, не характерная для предков и не имеющая пока рационального объяснения.}

\theterm{lotins}
{Лотины}
{Люди Лотоса.}

\theterm{qi-humans}
{Люди-тси}
{Жители Тси-Ди и Тра-Ренкхаля, в настоящее время представлены народами сели и ноа.}

\theterm{rut-misa}
{Род Медведя (\theorigin{ru}{Rut Mi\v{s}a}{род медведя})}
{Люди Мороза, один из немногих видов людей, сохранивших шерсть.
Чёрная кожа, густые беcцветные волосы на всём теле, единственные безволосые части --- нос и подушечки пальцев, у женщин --- верхняя губа.
Мужчины и женщины имеют мощную жировую прослойку, но у женщин она больше (стеатопигия).}

\theterm{tagua}
{Тагуа}
{Жители Драконьей Пустоши.
От первых людей отличаются незначительно.
Изменён метаболизм, имеются необычные пигменты в кожном покрове и глазах --- адаптация к излучению голубого гиганта.
Широко расставленные глаза, в верхней губе --- небольшая расщелина около 1 см длиной, по четыре пальца на ногах (результат дрейфа генов).}

\subsubsection{Ветвь Кани}

\theterm{kani-fork}
{Кани}
{Результат генетического эксперимента людей, собаки с изменёнными конечностями и увеличенным мозгом.
Видов-потомков --- 58 тысяч.}

\asterism

\theterm{qi-kani}
{Кани-тси}
{В настоящее время представлены пылероями Пыльного Предгорья, народом ркхве-хор и Высшими.
Очень высокого для кани роста --- 1,7--2,2 м.
Имеют светло-серую или коричневатую бархатную шерсть и голубые глаза, также для обоих полов характерна грива.
На четвереньках способны развивать самую высокую скорость среди наземных животных --- около 130 км/ч.
Пылерои --- владыки пустынь и саванн.
Обитают как на Короне, так и на Ките.
Предгорные пылерои занимаются скотоводством, разводят чёрных трёхгорбых верблюдов, рептилий и крупных съедобных насекомых, иногда устраивают плантации в оазисах.
Ркхве-хор живут военными набегами.
Высшие живут в оставленных первыми поселенцами подземных городах на экваторе и почти ни с кем не контактируют.}

\theterm{rut-ulka}
{Род Волка (\theorigin{ru}{Rut Ulka}{род волка})}
{Кани Мороза.
Черная кожа, сероватая шерсть с мощным подшёрстком, короткие нос и уши.
Как и Род Медведя, особи обоих полов имеют мощную жировую прослойку (параллелизм).
Также Род Волка отличается от первых кани меньшим размером клыков.}

\theterm{hrgadah}
{Хргада}
{Высокоразвитые кани с планеты Запах Воды системы Канопуса.
От первых кани отличаются отсутствием волос, очень изящным телосложением и плоской грудной клеткой.
Также у них имеются особые ферментативные системы репарации ДНК и некоторые другие изменения метаболизма (следствие повышенного радиационного фона на планете).}

\subsubsection{Ветвь Планты}

\theterm{plant-fork}
{Планты}
{Результат генетического эксперимента людей, клеточный гибрид человека, цианобактерии и нитробактера с некоторыми дополнительными генами.
Во Вселенной насчитывается около 23 тысяч видов-потомков.

Планты прекрасно поглощают воду кожей.
Если плант, стоя на солнце, засунет руку в ведро с водой, через два часа в ведре будет сухо.
Лёгкими усваивают азот и углекислый газ.}

\asterism

\theterm{qi-plants}
{Планты-тси}
{В настоящее время представлены идолами Молчащих Лесов, идолами Живодёра, Снежным Кланом и народом трами.
Внешне от первых плантов отличаются незначительно, рост от 1,2 до 1,5 м.
Волосяной покров отсутствует.
Изменены верхние конечности, благодаря чему планты-тси очень хорошо лазают по вертикальным поверхностям.
Прочие их особенности будут рассмотрены в соответствующем разделе.

Обитают в джунглях Короны и Кита.
В жертву богам приносят в основном пленных людей, для жертвоприношений используют рощи благородного баньяна или строят срубы.
Живут деревнями по двести особей максимум, в гнёздах на верхушках деревьев.
Высокоразвитыми считаются трами, обитатели Кита, находящиеся в торговых отношениях с ноа.
Название <<идолы>> пошло от их привычки стоять неподвижно под лучами солнца.}

\subsubsection{Ветвь Апиды}

\theterm{apis-fork}
{Апиды}
{Результат генетического эксперимента первых людей, насекомые (предположительно пчёлы), увеличенные в размерах, с изменёнными конечностями, скелетом, дыхательной системой и увеличенным окологлоточным нервным кольцом.
Цель эксперимента неясна до сих пор, скорее всего, он носил чисто научный интерес.
Видов-потомков --- 790.}

\asterism

\theterm{di-apis}
{Апиды Ди}
{Вид, уничтоженный во время Тараканьей войны.
Известно, что они являлись, как и большая часть апидов, колониальными сапиентами (матка, бесполые рабочие особи и трутни).}

\theterm{qi-apis}
{Апиды-тси}
{В настоящее время представлены Красными травниками и Бродячим Народом.
Отличаются от ранних апид очень изящной, тонкой конституцией, плодовитостью и перманентным гермафродитизмом.
Рост 1--1,5 м.
Глаз шесть (четыре парных простых, два фасеточных).
Усики развиты умеренно, в углублениях головы, ногочелюсти имеют один ряд зубчиков, которые сменяются в течение жизни.

После вторжения Безумного травники сильно пострадали от войн с идолами, в конце концов были вытеснены в Серебряные горы и Старую Челюсть, где живут очень разрозненно --- отдельными семьями --- в пещерах.
Их называют Красными травниками.
Многие Красные впоследствии ушли к людям и стали Бродячим Народом, небольшие поселения стали появляться в джунглях Кита, где идолов нет, но опять же --- ближе к людям, обеспечивающим им безопасность от Безумного.}

\subsubsection{Ветвь Дельфины}

\theterm{delfina-fork}
{Дельфины}
{Результат генетического эксперимента, низшие дельфины с изменёнными конечностями.
Видов-потомков --- 1264.}

\asterism

\theterm{qi-delfina}
{Дельфины-тси}
{В настоящее время представлены океаническим народом и стрелохвостами Голубого Зеркала.
Некрупных размеров, около 2 м в длину.
Шкура серая, плавникоруки тонкие, в отличие от ранних предков имеют всего три пальца.
Ротовой аппарат приспособлен как к растительной, так и к животной пище.

Океанический народ (стрелохвосты, или няньки) живёт в океане в тропических и субэкваториальных поясах, кочует за косяками рыб, на стоянках разворачивает понтонные лагеря.
Это единственное племя, которое не приняло ультиматума Безумного.
Мобильны, живут группами по 10--15 особей.
В силу этого, а также особой философии, подразумевающей подавление эмоций, добивание больных и раненых, стрелохвосты были оставлены Безумным в относительном покое.
У каждой группы есть старейшина --- стрелохвост с развитым чувством интуиции, который и подсказывает безопасный путь для группы.}

\subsubsection{Ветвь Стриги}
 
\theterm{striges-qi}
{Стриги, или глазастики}
{Единственный вид ветви Стриги, упоминаемый в отчётах о Тси-Ди.
Представляют собой увеличенных в размерах (рост около 1,3 м) сов с шестью парами конечностей --- две ходовых, две летательных и две рабочих.
Согласно данным Существует-Хорошее-Небо, глазастики были созданы сравнительно недавно, а потому не успели адаптироваться к культуре тси, жили обособленно в своих поселениях.
Вероятно, мигрировавшие на Тра-Ренкхаль особи впоследствии вымерли --- несмотря на свидетельства местных жителей, обнаружить глазастиков так и не удалось.}

\subsubsection{Ветвь Акариды}

\theterm{acarides-fork}
{Водолазы, или акариды}
{Результат эксперимента плантов планеты Мицелий системы Канопуса, человек с жабрами по типу акульих и видоизменёнными конечностями.
Способны длительное время оставаться под водой.
Предпочитают тёплые реки и моря.
Видов-потомков --- 56.}

\subsection{Флора и фауна}

\theterm{akchkatraas}
{Акхкатрас (\theorigin{tn}{akchkatraas}{духи, взывающие к живым})}
{Женские растения вида секвойя Бенедикта, завезённого переселенцами с планеты Лотос.
Мужские растения, что интересно, имеют совершенно другое название --- мисатр --- и считаются мусорными деревьями.
Самые старые деревья достигают 150 м в высоту и более 10 м в диаметре;
пряные ягоды (ложноплоды) акхкатрас имеют характерный ярко-алый оттенок, носящий то же название, и считаются изысканным деликатесом.}

\theterm{noble-banyan} % noble banyan
{Баньян благородный (<<слепой страж>>)}
{Древесное растение, произрастающее на Тси-Ди, было завезено переселенцами на Тра-Ренкхаль.
Взрослое растение отдаёт в стороны воздушные и надземные побеги, образуя рощицу размером в несколько гектар.
Назван так из-за отсутствия в рощице прочих растений, бактерий и грибов (баньян выделяет несколько десятков мощных фитонцидов, антибиотиков и фунгицидов).
Прежде рощи благородного баньяна использовались как храмы и больницы (из-за стерильной внутренней среды), но из-за плохой обороноспособности потеряли своё значение.
Некоторые исследователи склонялись к мысли, что благородный баньян --- это легендарная кольцевая теплица, но эта гипотеза не подтвердилась.}

\theterm{vegetectors}
{Вегетекторы (от \theorigin{sl}{vege[tare]}{расти} и \theorigin{sl}{[archi]tector}{архитектор})}
{Грибы и растения, вырастающие в полноценное жилище для сапиента.
Были очень популярны на Тси-Ди в период биоромантизма.
Впоследствии разработка вегетекторов была заброшена по причине сложности процесса программирования и длительного роста.}

\theterm{rope-snake} % rope snake
{Верёвочная змея}
{Одна из самых маленьких змей на Тра-Ренкхале --- длиной в две пяди, толщиной в мизинец.
Бежевого цвета, на спинке узор в виде свитых веревок.
Питается в основном насекомыми и крупными беспозвоночными.
Укус не ядовит, змея не способна прокусить кожу человека, но в пищу непригодна, так как впитывает яды и едкие жидкости съеденных ею насекомых.}

\theterm{featherwood}
{Дерево Перьев (Бальса оружейная)}
{Древесное растение Тра-Ренкхаля, древесина которого отличается небольшим удельным весом и большой прочностью.
Из Дерева Перьев изготавливают стрелы, древки копий и фаланг (все народы Короны), боевые веера (ноа, южные сели), основы для переносных жилищ (кочевые пылерои Предгорий), основы для лёгких кожаных лодок, пригодных для порожистых рек (сотронские сели, тенку), несущие балки зданий (Утонувший храм Сотрона), а также самую разнообразную домашнюю утварь.}

\theterm{green-bee}
{Зелёная пчела, трукхвал (\theorigin{tn}{trukchual}{летающая драгоценность})}
{Вид перепончатокрылых насекомых Тра-Ренкхаля.
Крупные (около 2-3 см) тела, 4 крыла (два больших, прозрачных, и два маленьких, желтовато-опалесцирующих).
Брюшко полосатое, зелёное с жёлтым.
Жало имеет чехол, который отрывается при укусе и отрастает через какое-то время.
Яд смертельно опасен для всех видов, за исключением колибри вида Пчелиный Ужас, которые питаются зелёными пчёлами (смертельная доза для человека Тра-Ренкхаля --- 2 укуса, для человека-тси --- 40--80 укусов).
Гнёзда строят в кронах деревьев-медоносов, форма гнёзд --- цепочка шаров (обычно 3--5), от самых маленьких внизу до самых больших сверху.
Гнездо строится из выделений симбионта --- паука-буйвола, которого зелёные пчёлы ловят, выращивают и доят, а затем инкрустируется кусочками коры.
В верхнем ярусе гнезда выращиваются личинки, нижние ярусы используются для грибных ферм, где зелёные пчёлы выращивают низшие грибы.
Мёд трукхвала имеет особый грибной вкус и ценится как изысканный деликатес.}

\theterm{indigo-firefly}
{Индиго-светляки}
{Семейство девиантных насекомых Тра-Ренкхаля.
Крупные, до 4--5 см в длину.
Имеют три глаза, три жёстких крылышка и органы полёта --- геликоптероиды, характерные для девиантных насекомых Тра-Ренкхаля.
Специфическая система биолюминесценции --- фотоны образуются в результате химических реакций и проходят через тонкую хамелеоновую занавеску, приобретая цвет от ультрафиолетового до зелёного.
Не путать с лантерн-светляками, которые относятся к Ветвям Земли.}

\theterm{stonetoad}
{Каменная жаба}
{Вид земноводных Тра-Ренкхаля.
Крупные (до 15 см), узкоротые, сероватого цвета, рисунок кожи напоминает гранит.
Живородящие.
На спине имеются костно-роговые щитки.
Крик каменной жабы напоминает гуление и плач младенца.
Днём животное сидит в каменных пещерках и каменных постройках, ночью выбирается наружу --- на охоту и для спаривания.
Иногда приползает на детский плач, из-за чего на севере её зовут жабой-кормилицей.}

\theterm{ringhouse}
{Кольцевая теплица}
{Живое существо, разработанное на Тси-Ди для обеспечения тси и других животных пищей на орбитальных станциях, в межзвёздных кораблях и на инопланетных базах.
Поставляет аналоги растительных волокон и животных белков, полный спектр аминокислот и витаминов.
Способно использовать любые растворимые в воде, кислотах и предельных углеводородах минералы, а также впадать в спячку при неблагоприятных условиях.
Орден Преисподней объявил крупную награду за оцифрованные клетки и важные сведения об этих существах.}

\theterm{coraldrake}
{Кораллица}
{?}

\theterm{redball}
{Красный шар}
{?}

\theterm{laaka} % laaka sap
{Лака \theorigin{tn}{laaka}{плохой знак}}
{Древесное растение Тра-Ренкхаля с чёрными листьями и ярко-красными ядовитыми плодами.
Лаковый сок содержит мощный нейротоксин, который даже в малых дозах способен почти мгновенно остановить сердцебиение и нервную деятельность людей Тра-Ренкхаля и тси.
Несмотря на то, что самыми опасными частями растения считаются ягоды, ствол, ветви и листья также нельзя трогать голыми руками, а в жару опасно даже находиться рядом с деревом без утиной маски, пропитанной солевым раствором.
Растворённый в масле лаковый сок относительно безопасен для хранения и переноски, его заливают в стрельные чехлы и амулеты Сана.
Места произрастания лаки отмечаются на картах четырьмя кружками, как и указатели ведущих к ним тропинок.}

\theterm{praypoppy}
{Мак молитвенный}
{Распространённое на Тра-Ренкхале растение, являющееся природным источником морфина-44, мощного анальгетика и снотворного.
Был завезён тси и генетически модифицирован, чтобы успешно конкурировать с местными растениями.
В пользу этого говорит его ареал (влажные зоны всех материков, за исключением пустынь и Дальнего Севера), а также наличие в мозгу тси ферментативных систем, специфичных именно к морфину-44.
Употребление масла молитвенных маков представителями тси совершенно безопасно, у прочих сапиентов планеты Тра-Ренкхаль масло вызывает галлюцинации, судороги и сильную наркотическую зависимость.}

\theterm{wreath-of-malikch}
{Маликхов венок}
{Папоротник-эпифит, издающий очень приятный аромат.
Использовался сели как сухоцвет и ингредиент для курений.}

\theterm{melipona}
{Мелипона Водораздела}
{Прирученная медоносная пчела Тра-Ренкхаля.
Не имеет жала.
Производит сладкий мёд, являющийся основой многих блюд сели, тенку и ркхве-хор.}

\theterm{milkbush-of-fisher}
{Молочник рыболовный}
{Кустарниковое растение Тра-Ренкхаля.
Сок его оказывает парализующее действие на рыбу, но безвреден для сапиентов.
Рыбаки (обычно дети) выливают сок молочника в реку и ждут, пока парализованная рыба всплывёт.}

\theterm{silent-cedar} % silent cedar
{Молчащий кедр}
{Древесное растение Тра-Ренкхаля, распространённое в Суболичье и Молчащих лесах.
Также растение встречается в Сикх'амисаэкикх, Хрустальных землях и Старой Челюсти.
Древесина и хвоя благодаря микроструктуре обладают свойством гасить звуковые колебания в диапазоне 1--23 кГц.
Из молчащего кедра сели делают столбы, половые плиты и навес для мёртвой зоны;
благодаря им крики жертв и прочий шум не покидают крышу храма.
Также молчащий кедр используется для пытки тишиной у ноа и трами.
В лесах молчащего кедра живёт ограниченное число видов Ветвей Земли --- из-за затруднённости или полной невозможности звукового общения.}

\theterm{one-love-furfeet}
{Мохноножка-однолюбка}
{Бескрылая птица Тра-Ренкхаля.
Питается беспозвоночными подстилки.
Дробно щёлкают клювом.
Моногамны, название пошло из-за тонких пёрышек на цевках.}

\theterm{marblesnake}
{Мраморная змея}
{?}

\theterm{niemelto}
{Нимелто (\theorigin{ru}{niemelto}{коровы Немальцевой})}
{Генетически модифицированное жвачное млекопитающее с планеты Мороз системы Арракиса, отличающееся очень высокой устойчивостью к холоду благодаря микроструктуре меха и кожи.
Самцы отличаются от самок небольшими опушёнными рожками. Некоторые особи способны переносить температуры порядка 190°К.
Выведены лабораторией Ольги Немальцевой в раннюю Эпоху Богов.}

\theterm{fiddletail-deer}
{Олень-вертихвостка}
{Одомашненное сели девиантное парнокопытное животное, конституцией напоминающее антилопу.
Самцы круглый год носят длинные двуветвистые рожки, самки безрогие.
Название пошло от привычки животного трясти небольшим белым хвостиком --- способ внутривидовой коммуникации.}

\theterm{swing-around}
{Папоротник-опахало}
{Папоротник Тра-Ренкхаля, имеющий очень длинные перообразные вайи и крупные спорангии, которые в спелом состоянии лопаются от прикосновения, выбрасывая в воздух значительное количество спор.
Споры опахала используются как лечебная пудра и впитывающий агент в ремеслах.}

\theterm{beggarbird}
{Попрошайка (\theorigin{tn}{trasakch}{требует мёд})}
{Птица джунглей Тра-Ренкхаля из семейства медоуказчиков.
Питается насекомыми, мёдом и воском перепончатокрылых.
Нападает на пасеки в Пыльном Предгорье, из-за чего активно истребляется на Западе.
Охотно поедает и гнёзда зелёных пчёл, но сама не может атаковать гнездо, поэтому показывает людям или идолам дорогу к гнёздам, после чего ждёт, пока охотники разделаются с пчёлами.}

\theterm{sugarfly}
{Сахарная муха}
{Вид двукрылых насекомых Тра-Ренкхаля.
Взрослые особи откладывают яйца в закисшие фрукты, которыми питаются опарыши.
Опарыши используются восточными сели как деликатес --- в вяленом виде или в виде сладкой муки, из которой делаются конфеты.
Вдоль Западного тракта сахарную муху разводят пчеловоды вместе с мелипоной;
ближе к Водоразделу муховодство почти не распространено.}

\theterm{siu-siu}
{Сиу-сиу (звукоподражание песне)}
{Птица Тра-Ренкхаля с ярким оперением, которое высоко ценилось племенем сели.
Перья сиу-сиу шли на праздничные головные уборы и входили в состав ритуального подношения Солнечной птице.}

\theterm{scout-ephedra}
{Скаут-эфедра (эфедра R88, хвойник неприхотливый)}
{Генетически модифицированное голосеменное растение ранней Эпохи Богов, использовавшееся для оживления планет.
Способно произрастать на холодных планетах, бедных кислородом, в рекордные сроки создавать устойчивый к эрозии слой почвы, а также особую наземную экосистему с локальным повышением температуры и повышенной концентрацией кислорода.
В зарослях скаут-эфедры, которая могла выдержать высокую радиацию, ветра и снегопад, жили насекомые, мелкие птицы и млекопитающие.
Во многих независимых культурах скаут-эфедра является священным растением, из неё плетут венки и делают погребальные убранства.}

\theterm{deathly-heat}
{Смертожар}
{?}

\theterm{}
{Согхо (\theorigin{tn}{sogcho}{печальная флейта})}
{Птица Тра-Ренкхаля из семейства курообразных.
Самка серая, с сизыми подпалинами на брюшке.
Самец яркий, фиолетово-лиловый, с красивым веерообразным хохолком, поёт на восходе и закате солнца однообразную, напоминающую звук свирели песню (<<тиу-лиу-ла, фьюю, тиу-ла>>).
Мясо согхо --- деликатес, очень нежное и питательное.
Перья самца согхо используются сели для так называемых скорбных уборов (чаще всего серёг), которые носили в знак жажды мести и скорби по убитым друзьям.}

\theterm{tchal-sar}
{Тхальсар (шорея тёмная, дерево Корвуса)}
{Растение, завезённое на Тра-Ренкхаль с Лотоса.}

\theterm{large-eared-dragaway}
{Ушастая уволочь}
{Мелкое (20 см без хвоста) кошачье джунглей Тра-Ренкхаля.
Охотится на крупных насекомых и мелких грызунов, но очень часто совершает набеги на жильё, утаскивая разделанное мясо и пищевые остатки.}

\theterm{chrikchsatr}
{Хрикхсатр (хрустальное дерево)}
{Единственный во Вселенной девиантный род древовидных лишайников, произрастающий на Тра-Ренкхале.
Включает 4 вида --- оранжевый хрикхсатр, бирюзовый хрикхсатр, сердитый коралл, дерево-свинья.
Все они являются эндемиками Мшистой степи.
Бирюзовый хрикхсатр впоследствии был вывезен с Тра-Ренкхаля и, благодаря чрезвычайной прихотливости и красоте, стал излюбленным комнатным лишайником у почвоведов Капитула.}

\theterm{spiny-ostrich}
{Шипастый страус (страус-дикобраз)}
{Бескилевая птица саванны Тра-Ренкхаля с видоизменённым пером (шипами).
Является одним из двух видов ветви Броненосные казуары (название <<страус>> ошибочно).
Чрезвычайно опасен, так как проявляет агрессию к любым сапиентам на своей территории, очень ловок и хорошо уворачивается как от стрел, так и от копий, а шиповатое оперение представляет собой неплохую броню.
Врагов колет шипами и бьёт мощными ногами до смерти.
Гнездится неподалёку от водоёмов, в низинах.
Перо страуса-дикобраза ценится всеми племенами Тра-Ренкхаля, часто вплетается в волосы как знак силы (иногда --- как тайник, так как в полость пера помещается значительное количество золотого песка и небольшие свитки пергамента).
Пылерои Предгорий используют перо шипастого страуса для изготовления кхаагатра.}

\section{Девиантные Ветви}

\theterm{deviant-forks}
{Девиантные Ветви}
{Виды, созданные инженерным путём хоргетами-демиургами (либо, согласно некоторым определениям, сапиентами).
Отдельные исследователи настаивают на том, что Девиантные Ветви следует изучать как технологию, а не как биологические системы, так как они --- результат разумного замысла, а не эволюции.
Несмотря на то, что эта точка зрения не разделяется большинством учёных Ордена Преисподней, она оказывает известное влияние на парадигму исследования Девиантных Ветвей и потому заслуживает упоминания.}

\subsection{Сапиентные виды}

\theterm{machinae}
{Машины, или Роботы}
{Электронно-световые устройства с базовыми инстинктами живых существ.
Создавались на многих планетах, но наибольшего расцвета достигла Машина Тси-Ди, созданная народом тси.}

\theterm{ngvso}
{Нгвсо}
{Вид, созданный Безымянным, демиургом Тра-Ренкхаля.
Морские обитатели, напоминают осьминогов.
Длина 1,5--2,5 м, чешуя --- крупная оранжевая у самцов, мелкая сине-зелёная --- у самок.
Трилучевая симметрия --- имеются три раздваивающихся щупальца и три простых глаза.
Общаются при помощи непарного гидравлического щупальца --- подают им звуковые сигналы, похожие на барабанную дробь, также используют жестовый язык.
Способны некоторое время пребывать на суше в специальных влагосохраняющих костюмах.
Занимаются выращиванием водорослей, рыбоводством, охотой, собирательством.
Умеют обрабатывать металл.
Были практически поголовно уничтожены дикими стрелохвостами, оставшиеся разрозненные популяции собрались вместе и отгородились насыпью в Коралловой Бухте, где заключили союз с народом ноа.}

\theterm{souzerena}
{Сюзерены, или стрекозодраконы}
{Вид теплокровных летающих рептилий с шестью конечностями, cоздан демиургом Драконьей Пустоши, Кох Свободолюбивой, впоследствии примкнувшей к Ордену Преисподней.
Сюзерены имеют прямую связь с демиургом (молекулярный приёмник в мозгу), что позволяло использовать их как армию в случае вторжения.
После долгой и кровопролитной войны, длившейся тысячу лет (840 земных лет), были вытеснены поселенцами с Земли.
Во времена царствования Валеридов сюзерены обитали в горах Малого Листопада и сохраняли по отношению к людям нейтралитет.
После захвата Адом Драконьей Пустоши, во время номинального правления Скорпидов сюзерены были истреблены.
Есть версия, что Кох Свободолюбивая встала на сторону Ордена Преисподней вследствие шантажа и впоследствии убита.}

\section{Ветви Звезды}

\theterm{star-forks} % Star Forks
{Ветви Звезды}
{Сапиенты с планеты 1-34.
Это вторая известная планета, на которой установилась стабильная самозарождённая жизнь.
Для жизни необходимы вода, метан и температура около 70 градусов Цельсия, выделяют углекислый газ.
Ветвями Звезды занимается особая группа научных и военных отделов Ордена Преисподней.
Классификация видов Ветвей Земли к ним неприменима.
Контактов с Ветвями Земли не зарегистрировано.
Точное число заселённых ими планет неизвестно.}

\section{Ветви Ночи}

\theterm{night-forks} % Night Forks
{Ветви Ночи}
{Неизвестные сапиенты.
Следы их деятельности (техника, обиталища, космические корабли) обнаружены на нескольких удалённых планетах, в том числе на Тси-Ди.
Местонахождение материнской планеты и заселённые ими экзопланеты неизвестны.}

\section{Ветви Пламени}

\theterm{flame-forks}
{Ветви Пламени}
{Высокотемпературные формы жизни.
Несмотря на то, что не обнаружено ни одного сапиентного вида Ветвей Пламени, некоторые исследователи придерживаются убеждений, что теоретически эти Ветви способны дать начало виду с сапиентной архитектурой.}

\asterism

\theterm{aberrants}
{Аберранты}
{Две клады плазмобионтов, живущих в условиях повышенного давления и температуры (в конвективной и радиационной зоне звезды).}

\theterm{flame-of-antares} % Flame of Antares, antaryde
{Пламя Антареса (антарида)}
{Плазменная форма жизни, встречающаяся на некоторых звёздах.
Впервые были обнаружены, как следует из названия, на звезде Антарес ещё первыми людьми, незадолго до взрыва сверхновой (в настоящее время считается, что сверхновая Антареса взорвалась именно из-за деятельности плазменных форм жизни).
Антарида отличается высокой скоростью метаболизма и устойчивостью к воздействию среды по сравнению с обычным огнём.
Модифицированная форма антариды используется как оружие --- в частности, можно настроить их на питание определённым полимером или на присутствие каких-либо веществ в определённой концентрации.
Антарида Безумного бога --- чрезвычайно агрессивная, но неустойчивая форма.
Проектирование и использование антарид запрещено законодательствами Ада и Картеля вследствие непредсказуемости их изменчивости (Оборонительный Кодекс Ордена Преисподней, 12A.0; Politica Of-De, Alb. 1162).}

\section{Ветви Смерча}

\theterm{jorget} % Swirl Forks
{Ветви Смерча}
{Ветви, объединяющие хоргетов и прочие формы жизни, конструктивно сходные с хоргетами.
Некоторые классификации определяют Ветви Смерча как подгруппу Девиантных Ветвей.}

\chapter{Биология тси}

\section{Системный обзор}

\begin{enumerate}
\item \textbf{Мышечная ткань.}
Изменена структура волокон, вследствие чего прочность выше в 8 раз, а сила --- в 3 раза.
Худощавые планты-тси способны переносить веса, непосильные для людей прочих видов, а грузоподъёмность кани-тси сравнима с грузоподъёмностью специальных машин.
\item \textbf{Костная и соединительная ткани.}
Изменена архитектоника балок и волокон, а также их взаимодействия в местах соединений.
Прочность костей на излом выше в 2,7 раз при снижении веса в 1,4 раза.
Прочность сухожилий на разрыв выше в 4 раза.
\item \textbf{Кровеносная система.}
Изменена структура интимы сосудов, снижена турбулентность.
Изменена логистика васкуляризации.
Присутствуют резервные контуры кровоснабжения головного мозга и внутренних органов (так называемые свёрнутые сосуды, не имеющие просвета до открытия специальных клапанов).
\item \textbf{Скелет.}
Изменена форма таза --- смертность и травматизация женщин при родах практически равны нулю.
Изменена форма черепа, присутствуют дополнительные рёбра жёсткости и амортизирующие элементы.
Подобные же изменения в грудной клетке.
\item \textbf{Иммунная система.}
Кардинально отличается от таковой прочих видов и требует отдельного описания.
Тси устойчивы практически ко всем видам микроорганизмов, обитающих на Тси-Ди и Тра-Ренкхаль.
При наличии вакцинации смертность от инфекций равна нулю.
Аллотрансплантация возможна без ограничений, известны успешные случаи ксенотрансплантации --- в частности, у народа ноа есть обычай закрывать раны лоскутами кожи убитых плантов.
На Диком Юге и в пиратских полисах распространена практика украшать тело треугольной мозаикой плантовой кожи с фрагментами татуировок --- по одному треугольнику на каждого убитого врага (лорика).
\item \textbf{Зрительная система.}
Изменена оптическая система, имеется дополнительный хрусталик и фокусирующие спекулумы.
Имеются клетки, воспринимающие ультрафиолетовые волны, инфракрасные волны.
Клетки, воспринимающие гамма-излучение, расположены как в глазах, так и в коже по всему телу.
\item \textbf{Слуховая система.}
Воспринимает частоты от 10 Гц до 80000 Гц.
Также нечто похожее на слуховые аппараты обнаружено в толще эпифизов лучевой и берцовой кости, но роль этих органов пока остаётся неясной.
\item \textbf{Дыхательная система.}
Изменена структура лёгкого, противоточная система позволяет переместить в кровь до 95\% кислорода.
Орган звукопроизводства --- гортанная цитра --- требует отдельного описания.
\item \textbf{Пищеварительная система.}
Изменены органы вкуса и обоняния, большая часть ядовитых веществ распознаются ещё на стадии измельчения пищи.
Зубы способны сменяться в любом возрасте при потере или сильном повреждении.
Кишечник имеет значительно меньшую длину.
Изменены ферменты.
Внутренняя среда стерильна, пища расщепляется на 89--96\% от массы.
\item \textbf{Половая система.}
Способность к партеногенезу и смене пола, зависимые от феромонов.
Пенис у мужчин небольшой, втягивается внутрь.
Имеются зачатки систем обоих полов.
Менструации отсутствуют, овуляция и эволюция эндометрия запускаются присутствием в коре доминанты деторождения, подтверждённой медиаторами мужской спермы.
Беременность длится в 3 раза меньше (данные по людям-тси), имеются данные о впадении плода в анабиоз на срок до 15 лунных месяцев (при болезни матери либо недостаточном питании).
\item \textbf{Нервная система.}
Изменена структура ствола мозга, требует отдельного описания.
Имеются специальные ядра, приспособленные для вычислений в двоичной и троичной логике (у современных тси не используются).
Нервно-мышечные синапсы диафрагмы нечувствительны к курареподобным веществам, имеются дополнительные сплетения для обеспечения дыхания и кровообращения при повреждении ЦНС.
\item \textbf{Регенерация.} Любые внутренние органы, в том числе мозг, способны полноценно регенерировать при сохранном дыхании, кровообращении и питании.
Конечности восстанавливаются при частичной потере (пальцы, кисти, стопы), более грубое повреждение даёт патологическую регенерацию (например, нефункциональные пальцы на культе плеча).
\item \textbf{Психофизиология.}
На заре зарождения цивилизации разные виды вынуждены были специально изучать сигнальную систему друг друга.
Но впоследствии базовые навыки различения эмоций были запрограммированы в каждом тси.
Это хорошо заметно, например,при оценке эмоций апида-тси человеком-тси и человеком Тра-Ренкхаля: человек-тси, не контактировавший прежде с апидами, угадывал эмоциональное состояние в 85\% случаев, против 10\% у человека Тра-Ренкхаля.
\end{enumerate}

\section{Фенотипические группы людей-тси}

У людей-тси, помимо влияющих на фенотип отдельных, существует так называемый псевдорасовый хромосомный регион размером 8 мегабаз, расположенный на хромосоме 16A.
Происхождение его неизвестно;
предположительно он появился в результате ошибки во время ранних экспериментов по перестройке генома тси.
В этом регионе находятся гаплогруппы с генами, оказывающими сильное влияние на фенотип, метаболизм и некоторые психические характеристики.
Кроссинговер у гетерозигот в районе ПРХР16А невозможен.
Всего у людей-тси существовало более ста псевдорасовых гаплогрупп.
После заселения Тра-Ренкхаля из-за эффекта основателя осталось всего пять.

\begin{description}
\item[Аурелийская (по номенклатуре тси: 82.2, 82.3, 82.7)] --- наиболее многочисленная.
Ситрис ар'Эр является гетерозиготой 82.2/82.3, Аурвелий Амвросий --- гомозиготой 82.3.
Отличительные черты гомозиготы по 82 гаплогруппе поэтически описаны у Эрхэ Колокольчик --- <<...твои обсидиановые очи, и камень тёмный кожи загорелой...>>
\item[Сотронская (3.1)] --- гомозиготами по 3.1 является врач Тхитрона Кхатрим и Согхо, воин отряда чести Митхэ ар'Кахр.
Отличительные черты --- прямые светлые волосы (<<цвета мокрой циновки>>), зелено-голубые глаза, пухлые щёки (<<большой младенец>>), точечная пигментация по всему телу (по обидному, но меткому выражению жителей Водораздела, <<как пчёлы обгадили>>).
\item[Валенсийская (45.1)] --- редкая гаплогруппа.
Гомозиготы психически нестабильны, чаще страдают депрессиями и гипоманиакальными состояниями.
На Тси-Ди скрещивание между носителями 45 гаплогруппы не ограничивалось, но им настоятельно рекомендовали использовать для деторождения доноров генетического материала.
Персонажи книги, гомозиготные по 45.1 --- Кхохо ар'Хетр, Акхсар ар'Лотр, предположительно --- Тхартху ар'Катхар и поэтесса Эрхэ Колокольчик.
Отличительные черты --- яркие зелёные глаза, золотистые кудрявые волосы, сильно выступающие скулы и выступающие клыки (<<лицо ягуара>>).
Многие носители валенсии очень привлекательны в сексуальном плане --- в ПРХР16А у них находится <<ген-афродизиак>> (паралог гена одного из факторов иммунной системы, влияющий на запах тела).
Гетерозиготой 45.1/30.3 является Ликхэ ар'Трукх.
\item[Гаплогруппа Страны Целующихся Лесов и Камней (Сикх'амисаэкикх, 30.3, 30.4)]
\item[Гаплогруппа 16.1] --- чрезвычайно редкая.
Гомозиготы по 16.1 на Тра-Ренкхале не обнаружены.
\end{description}

Также существуют отдельные фенотипы, не связанные с ПРХР16А:

\begin{description}
\item[Ген Оцелота (Псевдоген меланопротеида A12)] --- меланопротеид приобретает способность связывать ртуть.
Гетерозиготы имеют преимущество в регионах, богатых ртутью, гомозиготы умирают ещё до имплантации.
Также по неизвестным пока причинам оцелотовый аллель несовместим с гомозиготами по аурелийской гаплогруппе --- у таких людей тяжёлые поражения кожи и нервной системы, они умирают в раннем детстве или юности, не преодолевая порога в 17 дождей.
Гетерозиготой 3.1/30.4 А12 является Тханэ ар'Катхар.
\item[Цикадный фенотип] --- полиэтиологическое полигенное состояние.
Во всех генотипах по ПРХР16А цикадный фенотип равновероятен.
Гетерозиготой 82.7/16.1 CIC является Ликхлам, воин Тхартхаахитра.
\end{description}


\section{Термины}

\theterm{flavor-color}
{Ароматный цвет}
{Длинноволновое ультрафиолетовое излучение, воспринимаемое зрительным пигментом тси (F13, аналог криптохрома) с максимумом поглощения в районе 370 нм.
Название пошло от растения Скальный аромат, цветы которого отражает почти исключительно волны этого диапазона.}

\theterm{throat-cither}
{Гортанная цитра}
{?}

\theterm{stone-fury}
{Каменная ярость (ярость каменных духов, око земли)}
{Экстремальное ультрафиолетовое, рентгеновское и гамма-излучения, воспринимаемые пигментными кристаллами глаз и кожи тси.
Субъективно вызывают у тси сильное неприятное ощущение и чувство страха.}

\theterm{ocelocity}
{Оцелотовость}
{Окраска кожи некоторых групп народа сели, живущих в Пыльном Предгорье.
Кожа оцелотовых людей имеет слабую оранжевую окраску, также на коже имеются характерные пятна, напоминающие пятна оцелота --- ярко-оранжевые в центре и чёрные по краям.
Форма пятен обычно совпадает с зонами иннервации определённых нервов.
Вызвана мутацией одного из генов меланопротеида --- белок приобрёл способность связывать ртуть, содержание которой в водах Пыльного Предгорья сильно повышено.
Без потребления ртутной воды оцелотовые люди приобретают естественную золотистую окраску, но чёрные пятна остаются на всю жизнь.
Гомозиготы --- те, что не умерли в младенчестве --- страдают светобоязнью и витилиго.}

\theterm{differentiation}
{Половая дифференцировка}
{Явление, встречающееся у раздельнополых тси.
Дети тси не имеют пола;
развитие половых признаков начинается только в начале пубертата --- под влиянием окружения, личных предпочтений и прочих факторов.
Близнецы обычно дифференцируют в детей разного пола, дифферецировка в один пол крайне редка (Манэ и Лимнэ ар'Люм --- редкое исключение).
Особи, дифференцировка которых по каким-либо причинам не произошла, называются цикадами.}

\theterm{prolactin-transformation}
{Пролактиновая трансформация}
{Явление, встречающееся у млекопитающих-тси.
Рост груди у обоих полов при стимуляции соска специальным веществом, выделяющимся в ротовой полости новорождённого.
После прекращения кормления в большинстве случаев грудь исчезает.}

\theterm{protocol-crystall}
{Протокол Кристалл}
{Нейрогуморальная система выживания.
Парадигма: <<Выжить.
Возможно, помочь сородичам>>.
Запускается язычным рефлексом (млекопитающие), ногочелюстным рефлексом (апиды).
Модули:
[1] <<Яд>> --- снижение метаболизма и проницаемости тканей, в некоторых случаях --- синтез антидота.
Активируется также при утоплении и попадании в открытый космос.
[2] <<Пламя>> --- выброс влаги, снижение теплопроводности тканей, снижение теплообразования, активация теплоотводящих систем и систем репарации ДНК.
[3] <<Лёд>> --- снижение теплопроводности тканей, увеличение теплообразования, изменение внутриклеточного матрикса (выброс глицерина и спиртов с низкой точкой замерзания).
[4] <<Молния>> --- автоматическая дефибрилляция.}

\theterm{protocol-taifeng}
{Протокол Тайфун}
{Нейрогуморальная система выживания.
Парадигма: <<Помочь сородичам.
Возможно, выжить>>.
Запускается снижением ОЦК более 11\%/мин либо тайфун-кодом, представляющим из себя хеш-сумму биометрических параметров сапиента.
Тайфун-код обязательно должен быть произнесён <<голосом друга>>.
Сели, большая часть которых сохранила этот механизм, но не умела им управлять, называли предсмертный трансовый героизм <<ветром духов>>.
Модули:
[1] <<Золотая минута>> --- спазм повреждённых сосудов, ориентация кровотока на нервно-мышечную систему.
[2] <<Лёгкий уход>> --- обезболивание и снятие негативной эмоциональной реакции.
Выброс в желудочки мозга смеси морфина-44 и эндогенного селективного корректора настроения.
[3] <<Доминанта защиты>> --- активация структур, направляющих последние действия умирающего на помощь сородичам.
[4] <<Живая сталь>> --- синтез сверхпроводящих элементов в нейронах.
Возможно только при наличии инфузионного импланта, требуется специальный препарат.
Мозг способен функционировать ещё 148 секунд после биологической смерти, представляя собой квантовый нейрокомпьютер на сверхпроводниках.
Несмотря на то, что <<Живая сталь>> была в стандартной комплектации всех тси, этот механизм считался весьма негуманным <<методом отчаяния>> и за всю историю использовался всего четыре раза.}

\theterm{silver-veins}
{Серебряные жилы}
{Система проводников, которая оберегает жизненно важные органы при ударе электрическим током.
Выглядит как подкожная венозная сеть, но отличается лёгким металлическим блеском на срезе и отсутствием крови в просвете.}

\theterm{gender-switch}
{Смена пола}
{Явление, встречающееся у раздельнополых тси.
Уже дифференцировавшаяся особь начинает превращаться в особь противоположного пола.
Механизм может запускаться гендерным дисбалансом популяции, травмой, либо по желанию (3 задокументированных случая).}

\theterm{sun-touch}
{Солнечная кисть}
{Средневолновое ультрафиолетовое излучение, воспринимаемое зрительным пигментом тси (F15) с максимумом поглощения в районе 300 нм.}

\theterm{glass-hue}
{Стеклянный оттенок}
{Коротковолновое ультрафиолетовое излучение, воспринимаемое зрительным пигментом тси (F18) с максимумом поглощения в районе 200 нм.
При высокой интенсивности субъективно вызывает у тси неприятное чувство, заставляющее их укрыться от источника.}

\theterm{stigmae-of-pregnancy}
{Стигмы беременности}
{Гипер- или гипопигментированные полосы на шее, показывающие беременность.
Есть у всех видов тси.}

\theterm{tanatosis}
{Танатоз}
{Крайняя степень развития протокола Кристалл, остановка жизненных процессов вплоть до имитации смерти.
Известно, что некоторые тси могли пролежать в состоянии танатоза несколько суток без значительного ущерба для нервной системы.
Обычный танатоз можно было легко определить по неестественно белой коже (<<цветущие подснежники>>), но иногда танатоз настолько хорошо имитировал смерть (вплоть до разложения некоторых тканей), что ошибались даже врачи-тси.
Наиболее точным способом определить танатоз являлась кольцевая теплица: она переваривала трупы, а находящегося в танатозе <<обнимала, не причиняя вреда>>, а иногда даже лечила и приводила в чувство.}

\theterm{warm-color}
{Тёплый цвет}
{Не путать с тёплыми оттенками видимого света Homo homo sapiens.
Инфракрасное излучение, воспринимаемое тепловыми ямками тси.
У сухопутных млекопитающих тепловые ямки находятся в районе угла глаза, над слёзным протоком.
У апид --- в районе ногочелюстного сустава.}

\theterm{transpartenogenesis}
{Транспартеногенез}
{Явление, встречающееся у раздельнополых тси.
В отсутствие феромонов тси своего вида (в изоляции) мужчина-тси превращается в женщину и беременеет.
В отличие от обычной смены пола, процесс транспартеногенеза не затрагивает мужские органы, он гораздо длительнее --- может занимать несколько месяцев, а то и лет;
кроме того, транспартеногенез очень часто обратим, т.е. после родов индивид снова становится мужчиной.}

\theterm{lingual-reflex}
{Язычный рефлекс}
{Специфическая реакция организма людей-тси.
Болевая стимуляция кончика языка вызывает экстренное снижение метаболизма вплоть до танатоза.}

\chapter{Классификация хоргетов}

Хоргеты считаются отличной от сапиентов формой жизни, несмотря на то, что некоторые выделяют их в отдельные Ветви (Ветви Смерча) или даже относят к Ветвям Земли как творение первых людей.

\section{Полярность}

\theterm{polarity}
{Полярность}
{?}

\asterism

\theterm{minus-jorget}
{Минус-хоргет}
{На основе минус-сингулярности ПКВ.}

\theterm{plus-jorget}
{Плюс-хоргет}
{На основе плюс-сингулярности ПКВ.}

\theterm{schmidt-transformation}
{Преобразование Шмидта}
{Изменение полярности неструктурированной омега-сингулярности на противоположную с помощью сдвига волны.
Полярность хоргета (структурированной сингулярности) изменяется путём создания неструктурированной сингулярности и последующего синхронного переноса информации.}

\section{Мобильность}

\theterm{mobility}
{Мобильность хоргета}
{Способность к перемещению в пространстве, обратно пропорциональная количеству масс-энергии.}

\asterism

\theterm{god}
{Бог}
{Стационарный, насыщенный масс-энергией хоргет, обычно заякоривающийся в планете.}

\theterm{daemon}
{Демон}
{Мобильный хоргет со средним насыщением масс-энергией, заякоривающийся в телах сапиентов.}

\section{Заякоривание}

\theterm{incarnation}
{Заякоривание (инкарнация)}
{Способность хоргета встраиваться в планету или сапиента и управлять ими.
Заякоривание преследует несколько целей.
Хоргет --- это сингулярность ПКВ, то есть, условно говоря, во Вселенной Фотона он не существует.
Для взаимодействия с ней ему требуется некий имеющий относительно постоянную структуру омега-источник, перемещение которого в пространстве можно будет отслеживать.
Впоследствии, по мере развития технологий заякоривания, хоргеты научились использовать сапиентный мозг или планетный омега-фон как интерфейс взаимодействия со Вселенной Фотона.}

\asterism

\theterm{anjel}
{Ангел (облачный тип)}
{Тип инкарнации в сапиента, при котором ВНД эмулируется мозгом.
При запросе из мозга включается хоргет, в котором производятся сложные вычисления, обработка информации или принятие решений.
Эффект <<зловещей долины>> отсутствует.
Траты масс-энергии минимальны.
Вычислить такого хоргета в спящем состоянии чрезвычайно сложно.
Обычно спящий хоргет сопровождает своё тело до зрелости, что позволяет ему максимально интегрироваться в личность.
Минусы --- длительность встраивания, излишняя подверженность эмоциям и недостаточная защита от информационного нападения.}

\theterm{bug}
{Жук}
{Очень маленький низкоинтеллектуальный хоргет, использующий сапиентов исключительно для получения эманаций.
Может влиять или не влиять на ВНД сапиента.
Жуки используются богами для сбора масс-энергии, также встречаются свободноживущие жуки, собранные в группы (Рой) или паразитирующие на сапиентах (Клещ).}

\theterm{quasisimbiosis}
{Квазисимбиоз (заякоривание типа <<голос в голове>>, форсированная инкарнация)}
{Тип инкарнации в сапиента, при котором хоргет встраивается в уже сформировавшуюся личность сейхмар (в качестве субличности).
Плюсы --- быстрая смена тел.
Минусы --- сложность встраивания, сапиент может сойти с ума, может активно сопротивляться демону и даже захватить над демоном контроль, интегрировав его личность в себя (2 реально зарегистрированных случая).
Тем не менее в серьёзной боевой обстановке этот способ до сих пор используется, и существуют сборки демонов, специально созданные именно для этого типа заякоривания.}

\theterm{zombie}
{Марионетка (зомби)}
{Тип инкарнации в сапиента, при котором хоргет берёт под контроль нервные стволы или верхние отделы спинного мозга.
ВНД сапиента эмулируется в хоргете, мозг в процессе не участвует, органы чувств сапиента не используются.
Первый и наиболее примитивный тип.
Траты масс-энергии значительны.
Марионетки легко вычисляются не только хоргетами, но и другими сапиентами (эффект <<зловещей долины>>).}

\theterm{lonely-god}
{Одинокий бог}
{Тип заякоривания в планете, при котором сингулярность только одна, все действия выполняет самостоятельно.
Бог тратит много энергии на перемещение.}

\theterm{spider}
{Паук (планшетный тип)}
{Тип инкарнации в сапиента, при котором хоргет берёт под контроль последние корковые нейроны.
Органы чувств сапиента используются, но ВНД эмулируется в хоргете.
Траты масс-энергии меньше, чем у марионеток.
Таких сапиентов сложнее вычислить хоргетам, эффект <<зловещей долины>> сведён к минимуму за счёт участия в движениях стабилизирующей системы сапиента (мозжечка и красных ядер у млекопитающих).}

\theterm{swarm}
{Рой}
{Чрезвычайно опасный вид бога.
Состоит из равноправных <<жуков>>, соединённых в сеть.
В силу низкого интеллекта интенсивно использует сапиентов (вплоть до полного их вымирания), вызывает массовые психозы и эпидемии.
Быстро размножается, но часто гибнет сам.}

\theterm{snowflake}
{Снежинка}
{Тип заякоривания в планете, при котором есть одна главная сингулярность и множество короткоживущих жуков, выполняющих функции сбора масс-энергии и преобразования материи.
Наиболее экономичный тип бога.
Название пошло из-за сходства проекции со снежинкой --- для экономии масс-энергии жуки перемещаются по планете согласно фрактальной маршрутной карте.}

\section{Цель сборки}

\theterm{compilation-purpose}
{Цель сборки хоргета}
{Цель, с которой хоргет создавался.
Несмотря на некоторую архаичность этого деления, оно используется до сих пор из-за сильных различий указанных типов в плане психологии.
В частности, урождённые боги могут страдать от последствий акбаса и хиторай, а }

\asterism

\theterm{born-god}
{Урождённый бог}
{Сборка хоргета сапиентами, не преследующая цели сделать хоргета членом общества и не предусматривающая развития его как личности.
Урождённые боги часто индивидуалисты, склонны к самоуничтожению, могут страдать от разнообразных последствий акбаса.}

\theterm{born-daemon}
{Урождённый демон}
{Сборка хоргета свободными хоргетами с целью сделать его членом сообщества хоргетов.}

\theterm{born-sapient}
{Урождённый сапиент}
{Оцифровка биологической (самообразовавшейся) нейронной сети свободными хоргетами с целью сделать его членом сообщества хоргетов.
Урождённые сапиенты часто не воспринимают себя членами сообщества хоргетов, могут содержать самые разнообразные дефекты и особенности настройки в зависимости от взрастившей их культуры.}

\section{Распределение вычислительных мощностей}

\theterm{powers-distribution}
{Распределение вычислительных мощностей}
{Фундаментальный параметр хоргета, который говорит о его профессиональных способностях.
Вычислительные мощности распределяются между тремя характеристиками --- чувствительность (способность собирать и структурировать информацию извне), интеллект (способность преобразовывать информацию и решать поставленные задачи) и устойчивость (способность сопротивляться воздействиям извне, в том числе информационному нападению).

Классы по распределению мощностей --- условное деление, каждый хоргет обычно распределяет собственные мощности так, как ему требуется.
Соотношение может изменяться и в ходе работы, из-за подключаемых модулей.
Единственный нюанс --- подключение модулей редко изменяет класс.
Изменение же класса, т.е. пересборка --- задача, требующая длительного анализа, всегда проводится специалистами.
Пересборка урождённых сапиентов невозможна в силу особенностей их внутренней структуры.}

\asterism

\theterm{visor}
{Визор (оракул)}
{Высокая чувствительность, средний интеллект, низкая устойчивость.}

\theterm{saboteur}
{Визор-интерфектор (диверсант)}
{Высокая чувствительность, очень низкий интеллект, средняя устойчивость.
Редко встречающийся тип, однако незаменимый при выполнении чётко поставленной задачи.}

\theterm{interfector}
{Интерфектор (воин)}
{Низкая чувствительность, средний интеллект, высокая устойчивость.}

\theterm{cogitor}
{Когитор (стратег)}
{Низкая чувствительность, очень высокий интеллект, низкая устойчивость.}

\theterm{scientist}
{Когитор-визор (учёный)}
{Чрезвычайно редкий в настоящее время тип.
Ранее такая специализация, как следует из названия, встречалась в основном среди научного персонала.
Устойчивость околонулевая.
Один из самых знаменитых представителей --- Сиэхено Опаловый Глаз.}

\theterm{tactic}
{Когитор-интерфектор (тактик)}
{Околонулевая чувствительность, средний интеллект, очень высокая устойчивость.
Тип, часто встречающийся среди младших воинских чинов.}

{\theterm{effector}
{Эффектор (универсал)}
{Средняя чувствительность, средний интеллект, средняя устойчивость.}

\section{Специальности (классификация Ордена Преисподней)}

\theterm{biology}
{Биология}
{Создание, изменение, восстановление и исследование биологических (природных) систем.}

\theterm{dictiology}
{Диктиология (культурология, сетевая технология)}
{Создание, изменение, восстановление и исследование сетей --- схем коммуникации между биологическими и/или небиологическими системами.}

\theterm{interfection}
{Интерфекция}
{Междисциплинарная отрасль науки и технологии, основной задачей которой является эффективное разрушение либо нарушение работы биологических и небиологических систем и сетей при минимальном воздействии на окружающую инфраструктуру.
Несмотря на то, что сам термин в первую очередь используется военными, методы интерфекции используются во всех сферах деятельности --- в производстве пищевых материалов, медицине, строительстве, машиностроении и программировании.}

\theterm{informatics}
{Информатика}
{Работа с любыми видами информации.}

\theterm{technology}
{Технология}
{Создание, изменение, восстановление и исследование небиологических систем (использующих искусственно созданные принципы работы).}

\chapter{Имена}

\section{Имя демона}

В раннем Ордене Преисподней имена демонам давались следующим образом:

\begin{itemize}
\item Кодовое имя бога, данное людьми.
Самый распространённый способ у древних демонов.
Так получили имена Грейсвольд Каменный Молот, Эйраки Мороз, Кох Свободолюбивая.
\item Позывной, который использовался при внедрении в сапиентное общество.
Так получили имена Ангара Краснобуря, Сиэхено Опаловый Глаз, Тако из клана Дорге.
\item Имена, которые имели оцифрованные до оцифровки.
\end{itemize}

Традиция прозвищ пришла немного позже.
Все историки сходятся во мнении, что первыми прозвища в их современном понимании стали давать в Ордене Тысячи Башен.
Тысяча Башен отличается многоязычностью --- на Друзах существует огромное количество изолированных племён;
переводимые прозвища стали в таких условиях необходимостью.

Забавный факт --- также именно на Тысяче Башен зародилась как таковая методология биодиктиологии (культурологии) ветви Люди, биолингвистика и дипломатия в её современном понимании.
Благодаря изолированности Друз --- как культурной, так и генетической --- Тысяча Башен представляет собой неплохую модель мультипланетного объединения.
По утверждению историков, выжить без этих наук на кристаллической планете было просто невозможно.
Впоследствии именно из Ордена Тысячи Башен вышли видные военные дипломаты и учёные, занимающиеся изучением биологии и социума сапиентов --- в частности, Стигма Чёрная Звезда и Север Солнечная Дева.

Названия демонов, по принятой ныне классификации, имеют структуру:

\textbf{Имя Прозвище из клана Клан (Исходное название) (Версия)}

\textbf{Имя}: краткое слово на Эй-B0, используется в небоевой обстановке.

\textbf{Прозвище}: от одного до трёх слов, которые могут быть переведены на практически любой язык.

\textbf{Клан} (необязательно): название демона, ядро или части ядра которого использовались при создании.

\textbf{Исходное название} (необязательно): название, данное демону разработчиками-сапиентами или имя сапиента.

\textbf{Версия} (скрыто): версия ядра демона согласно Реестру Ордена Преисподней.
Версии являются засекреченной информацией, поэтому в этой книге опущены.

\section{Кланы}

Кланом называется группа урождённых демонов, собранных на основе ядра основателя.
Группы, собранные на основе разных форков ядра, называются клановыми сериями, генерациями или субкланами.
Также есть понятие псевдоклана --- демоны, собранные на основе одного ядра, но впоследствии покинувшие его (либо изгнанные) и примкнувшие к другим веткам обновлений.

У каждого клана есть свои представители в Совете Глобальных Обновлений, занимающемся совместимостью обновлений с ядром клана.
Представители строго неприкосновенны и дают клятву не вмешиваться в политику фракции.
Любая попытка воздействовать на представителя клана может спровоцировать клановую войну.
Демоны из псевдокланов чаще всего пользуются индивидуальными обновлениями, созданными специально для них на основе открытого кода, предоставляемого Советом.

\begin{description}

\item[Мороз] --- один из первых кланов.
Родовая планета --- Преисподняя.
Основа Ордена Преисподней.
Основатель --- Эйраки Мороз.
Клановые серии --- первая, вторая и четвёртая генерации (третья считается провальной, она не дала ни одного стабильного демона).
Псевдоклан --- Туман, входит в Красный Картель.
Девиз --- <<Считай шаги>>.

\item[Дорге] --- один из первых кланов.
Родовая планета --- Преисподняя.
Разделился на два, одна часть которого, Левые Дорге, примкнула к Союзу Воронёной Стали, вторая, Правые Дорге, --- к Ордену Тысячи Башен.
Основатель --- Жерар Дорге.
Одна клановая серия.
Псевдоклан --- Улыбающиеся, в составе Красного Картеля.
Девиз --- <<Нас достаточно>>.

\item[Антрацис] --- родовая планета --- Чёрная Скала.
Основатель --- Лев Зелёный, демиург Чёрной Скалы, впоследствии Вечно Гонимый, погиб во время взрыва на Запах Воды.
Три клановые серии --- Долина Смерти, Жёлтое море и Белая Чаща.
Псевдоклан --- Вечно Гонимые (в составе Ордена Преисподней).

\item[Тахиро] --- родовая планета --- Капитул.
Основатель --- Тахиро Молниеносный.
Две клановые серии --- Тахиро Старейшины и Тахиро Вторые.
Два псевдоклана --- Унылая Когорта (Картель) и Серые Змеи (нейтралы).

\item[Усмане] --- родовая планета --- Земля Врачевателей.
Основатель --- Усмане Белое Одеяние.
Одна клановая серия.
Псевдоклан --- Скопцы, служит Красному Картелю.

\item[Вечность] --- объединение, по сути превратившееся в клан.
Основатели --- Семеро Бессмертных.
Тринадцать клановых серий, сорок слияний.
Два дочерних клана --- Чёрная и Красная Вечность (Орден Преисподней и Красный Картель соответственно).
Псевдоклан --- Платина (нейтралы).

\item[Оньё] --- один из девяти кланов, основавших Орден Тысячи Башен.
Некоторые считают его псевдокланом, потому что идеологией Оньё является выживание любой ценой --- члены клана с готовностью предают друг друга, если им грозит опасность, и не осуждают друг друга за это.
Тамга --- беременная коза.
Девиз --- <<будь осторожен>>.

\end{description}

\chapter{Персоналии}

\theterm{ainu}
{Айну Крыло Удачи (Айну)} % Ajnu the Luckwing
{Урождённый бог, создана на Древней Земле, демиург безымянной, ныне не существующей планеты в системе Сириуса.
Биотехнолог-интерфектор, максим секунда Ордена Преисподней.
Любовница Тахиро Молниеносного.
Прозвище Крыло Удачи получила после битвы с Чук Тьма Над Горой --- молодая демоница уничтожила превосходящую по опыту и мощи противницу, найдя в её обороне крошечное окно.
Ей так понравилось это прозвище, что она очень часто появлялась на собраниях с нашивкой или значком в виде белого птичьего крылышка.
Погибла на Тысяче Башен в битве с превосходящими силами Красного Картеля.}

\theterm{akchsar-ar-lotr}
{Акхсар ар’Лотр э’Сотрон (Седой-Дедушка-В-Снегу, Снежок)}
{Друг Митхэ ар'Кахр, воин отряда чести, впоследствии --- хагпот племени Инхас-Лака.
Даритель Имжу Тенебоя.
Любовник и спутник Кхотлам ар'Люм.
Кормилец Согхо и Ликхсара ар'Люм.}

\theterm{ancarjal} % Ancarjal the Bloodstorm (the Redbreeze)
{Анкарьяль Кровавый Шторм (Ангара Краснобуря, при рождении Тальяна Древолаз)}
{Урождённый демон, создана на Капитуле.
Биотехнодиктиолог-интерфектор, легат терция Ордена Преисподней.
Использует прозвище-позывной, данное ей людьми на планете Тысяча Башен (от \theorigin{хольский говор}{Angara}{море}, распространённого в то время женского имени).
Краснобуря --- ключевая крепость на друзе Хербст.
Пятая Мостовая война (она же Осенняя) на Тысяче Башен была для демоницы дебютом и первой серьёзной победой.}

\theterm{arcadiju} % Arcadiju the Womb Jackal (the Tomb Jackal, Thalian Wolf)
{Аркадиу Шакал Чрева (Аркадиу Валериану Люпино, Талианский Волк, Падальщик)}
{Урождённый сапиент, преобразован на Драконьей Пустоши Яйвафом Солёная Борода.
Биодиктиолог-когитор, легат терция Ордена Преисподней.
Был взят в плен и перешел на сторону Ордена Преисподней после Развязки Тринадцати Звёзд.}

\theterm{aurweli} % Aurweli Amwurosi
{Аурвелий Амвросий}
{Воин отряда чести Митхэ ар'Кахр.
В прошлом --- пират ноа, Тенетник из Гавани Неудачников.}

\theterm{wind-curling-hair} % Wind-Curling-Hair
{Ветер-Завивает-Волосы}
{---}

\theterm{jalo}
{Гало Кровавый Знак из клана Мороза}
{Урождённый демон, создан Эйраки Морозом, вторая генерация клана.
Впоследствии --- глава псевдоклана Туман, Третий Дурак Красного Картеля.
Предположительно погиб на планете Запах Воды во время диверсии Ланс-Ната Алмаза.}

\theterm{fool}
{Глупец}
{Персонаж легенд и сказок сели.}

\theterm{grejsvolt} %  Grejsvolt the Stonehammer
{Грейсвольд Каменный Молот (Грисволд-2)}
{Урождённый бог, создан Лабораторией Дж.\,Грисволда (научный город Цикаго-2, домен Северная Америка, Древняя Земля).
Демиург планеты Лотос системы Фомальгаута.
Технодиктиолог-эффектор, легат прима Ордена Преисподней.
Прозвище Каменный Молот придумал сам --- чтобы никогда не забывать, с чего началась технология.}

\theterm{greta}
{Грета (Якоб Морген Йеннифер Шахтшнайдер)}
{---}

\theterm{siblings} % Siblings
{Двойняшки}
{Прозвище демонов Фуси Абрикосовый Посох и Нуива Пустая Тыква.}

\theterm{long-moist-tail}
{Длинный-Мокрый-Хвост (Хвост)}
{Химик-технолог из города 10, живший примерно за 1200 оборотов до Катаклизма.
Автор видеоканала <<Проблема переработки хвостов>>, изначально посвящённого насущным вопросам химической промышленности, а потом превратившегося в философски-юмористический.
Цитаты Хвоста становились крылатыми.
Он несколько раз --- два раза при жизни и более двадцати раз после смерти --- признавался одним из величайших тси.
Видеозаписи его канала считаются культурным достоянием цивилизации.}

\theterm{du-xi} % Du-Sie the Hunter, of the White Thicket, of Antracis Clan
{Ду-Си Охотник, уроженец Белой Чащи, из клана Антрацис}
{Урождённый демон, член младшей из трёх генераций демонов Чёрной Скалы.
Совершил побег с Чёрной Скалы после нападения на Фу-Си Двойняшку, стал одним из Вечно Гонимых --- бессрочных преступников, подлежащих уничтожению.
Примкнул к Ордену Преисподней во время мирного времени, предшествовавшего первой войне между Орденом Преисподней и Красным Картелем.
Служащий отдела 125, после его роспуска перешёл в отдел 100.}

\theterm{dourgue-junior} % Gerard Dourgue
{Жерар Дорге Младший}
{Урождённый бог, глава клана Дорге, ближайший соратник Арракиса Мороза.
Был вероломно убит членами своего клана вскоре после побега на Тысячу Башен.}

\theterm{dourgue-senior} % Gerard Dourgue
{Жерар Дорге Старший}
{---}

\theterm{flask} % Sealed-Life-Flask
{Закрытая-Колба-Жизни (Баночка)}
{Инженер сетей из города 12.
Внешность: немного полный, круглолицый, с широкими губами.
На голове и боках Баночки были вертикальные шрамы, которые остаются у мужчин-плантов после процесса смены пола (следы отмершего венчика и крыльев).}

\theterm{imzhu}
{Имжу Тенебой, он же Имжу Лжец, он же Имжу Слепец}
{Герой народа хака, вождь племени Инхас-Лака, который пренебрег решением Союза Племён и двинул свои силы на помощь сели в борьбе с Безумным богом.
Один из трёх вождей хака, которые участвовали в битве на Могильном берегу, и единственный, который выжил.
Впоследствии был обвинён старейшинами в измене и вынужден бежать из своего племени в земли ноа.
Там он нашёл себе мужчину по имени Фелис и приютил сироту по имени Сильвия Акрвила.
Вернулся к Спокойному озеру спустя тридцать дождей, когда Союз Племён наконец признал, что Безумный действительно был уничтожен.
Остаток жизни Имжу, который к тому времени почти ослеп от длительной болезни, прожил вместе со своим многочисленным семейством в святилище Одинокий Столб, окружённый всеобщей любовью и уважением.}

\theterm{jokull}
{Йокудль Гуннарссон Глазенапп}
{---}

\theterm{katarina} % Katarina Kosulja
{Катарина Козуля --- кузничая, инженер, мастер ковки хука.}

\theterm{celsa}
{Кельса (Цельсия) Пушистая}
{?}

\theterm{colbe} % Colbe the Old Maxim
{Кольбе Старое Изречение}
{Один из двух братьев-биологов, созданных в рамках эксклюзивного проекта на Тысяче Башен.
Был назван в честь Максимилиана Кольбе, героя времён Последней Войны.
Несмотря на принадлежность к Ордену, соблюдает нейтралитет и не особенно это скрывает.}

\theterm{candy} % Candy
{Конфетка (Эрликх ар’Фа э’Самитх, Сладкая-Ягодная-Конфета)}
{Воин Тхитрона.
Тренер Ликхмаса.
Участник похода в Тиши.
Воин отряда чести Мёртвые Стервятники.}

\theterm{corjes}
{Корхес Соловьиный Язычок (Лючжоу Соловей, уроженец Долины Смерти, Картезий (Корхес) Нактергаль}
{Урождённый демон, была создана Львом Зелёным.
Вечно Гонимая.}

\theterm{kurz}
{Курц Пламя Осени (Курц Штайгер)}
{---}

\theterm{biter}
{Кусачка (Шестнадцать-Сила-Четыре-Укус-Три-Молния)}
{Травник из Бродячего Народа, член отряда чести Митхэ ар'Кахр.
Отвечал за бухгалтерию и коммуникации с торговцами Бродячего Народа.
Погиб в бою во время штурма Трёхэтажного Храма.}

\theterm{kcharas}
{Кхарас ар’Хитр э’Хатрикас (Любовью-Свитая-Веревка, Верёвочка)}
{Боевой вождь Тхитрона.
Участник битвы на Могильном берегу.
После битвы ушёл из Храма и стал ткачом.
До самой смерти прожил со своим другом Эрликхом в одном жилище.}

\theterm{kchatrim}
{Кхатрим ар’Сар э’Тхонтротрис (Ветка-Растущая-Из-Листа, Веточка)}
{Врач Тхитронского Храма, уроженец Водораздела.
Погиб во время тхитронской диверсии.
Внешность: среднего роста, крепкого телосложения, длинные распущенные седоватые волосы, сотронская борода (аутотрансплантаты с затылка на скулы, губы и подбородок), широкий плосковатый нос, серо-зелёные глаза, знак Снежной Обители на правом запястье.}

\theterm{kchotlam}
{Кхотлам ар’Люм э’Кахрахан (Испачканное-Мёдом-Перо, Пёрышко)}
{Купец Тхитрона, участница битвы на Могильном берегу, кормилица Саритра, Ликхмаса, Манэ, Лимнэ, Согхо и Ликхсара ар'Люм, близкая подруга Митхэ ар'Кахр.
Начинала торговцем в землях ноа.
Сотрудничала со Стервятниками, среди которых была её родственница Эрхэ ар'Люм.
Один раз дала бой пиратам, но попала в плен и убедила их сохранить ей жизнь в обмен на улаживание конфликта между пиратскими полисами, а впоследствии смогла вернуть судно и вызволить часть команды.
Шторм в Могильном проливе потопил её корабль вместе с командой и товарами.
Была спасена Хитрамом ар'Кхир, который впоследствии стал её мужчиной и дарителем троих её питомцев.
Карьеру дипломата начала в Кахрахане, откуда была изгнана по обвинению в воровстве.
Впоследствии Митхэ ар'Кахр нашла доказательства невиновности Кхотлам, и горожане попросили её вернуться на прежнюю должность.
Кхотлам отклонила предложение и осела в Тхитроне, где стала помощницей купца, а затем, после ухода купца на покой --- и купчихой Тхитрона.
Спасла от смерти Кхохо ар'Хетр, совместно с Кхарасом и Хитрам выкрав её из темницы святилища Одинокий Столб и став первым дипломатом в истории, вмешавшемся в отправление правосудия святилища.
Взяла на воспитание Ликхмаса, ребёнка Митхэ ар'Кахр, а также наложила вето на решение Советов Тхитрона, сняв с Ситриса ар'Эр звание кутрапа, обеспечив ему возможность жизни в землях Тхитрона и впоследствии введя его в Тхитронский Храм.
После налёта на хутор Самитх приютила выжившего полубезумного старика, который оказался известным в прошлом охотником Сиртху.
Была обвинена в организации налёта на Самитх, взята под стражу и около двух декад провела в комнате для ожидающих жертвоприношение, так как ни один жрец в Тхитронском Храме не решался принести её в жертву;
отклонила предложение Первого жреца организовать её побег.
Ситрис, Кхохо и Трукхвал, которые не поверили обвинителям, смогли при помощи Сиртху доказать её невиновность, выследив угнанный обоз и найдя документы, в которых шла речь о её устранении с поста дипломата.
Впоследствии Кхотлам ни разу не упоминала своё время пребывания в Комнате Без Окон и отказывалась говорить на эту тему даже с близкими людьми;
также известно, что после этого инцидента она до конца жизни не переступила порог храма, предпочитая все встречи с храмовниками устраивать во Дворе.
Кхотлам руководила обороной города во время нашествия Молчащих идолов.
Была вынуждена убить своего питомца Саритра во время боя на вырубке, чтобы он не попал в плен и не был принесён в жертву.
Благодаря её своевременному вмешательству был заключён союз с хака, а Молчащие идолы атакованы на марше объединёнными силами сели, хака и ноа.
Способствовала приходу Людей Золотой Пчелы в Тхитронский Храм, а также приняла участие в дипломатической переписке Тхитрона с Тхартхаахитром о разделении сфер влияния.
Пережила три покушения на свою жизнь в связи со своей миротворческой деятельностью, от одного из которых она до конца жизни не могла говорить громко --- нож убийцы задел гортань и голосовую цитру.
Во время войны с Безумным Кхотлам подготовила Тхитрон к осаде, переманила на сторону сели несколько родов идолов, смогла организовать переговоры с Имжу Тенебоем, а также провела разведку в лагере Картеля, внеся существенный вклад в победу Ордена Преисподней.
Сражалась в битве на Могильном берегу плечом к плечу со своей подругой Митхэ ар'Кахр.
После гибели Хитрама на Могильном берегу и окончания войны Кхотлам вернулась в Тхитрон с Акхсаром ар'Лотр и прожила с ним до своей смерти, родив ещё двоих хранителей.
Акхсар и Кхотлам умерли в один день --- в подвале обнаружилось гнездо смертожаров, Акхсар решил его выжечь, а Кхотлам, поняв, что Акхсара ужалила змея, попыталась своими силами вытащить его наружу.
Оба получили более двадцати укусов и погибли на месте.
Впоследствии долгая и богатая на события жизнь Кхотлам была описана её питомицей, Манэ Нарисованной, в романе <<Змеиная яма>>.}
%{Snake Pit}

\theterm{kchoho} % Coal
{Кхохо ар'Хетр э'Митракрас (Умойся-Угольной-Водой, Уголёк)}
{Воин Тхитрона.
Охотница за головами на Могильном берегу.
Известная бардесса.
Дарительница Ликхэ ар'Трукх.
Погибла в битве на Могильном берегу.

Внешность: среднего роста, крепкого сложения, грязные волосы, торчащие в разные стороны, рыбки плетёные в два хвоста.
Глаза тёплого каре-зелёного цвета.
На подбородке татуировка в виде пяти клиньев, слева шрам типа <<улыбка Глазго>>.
На груди татуировка <<двукрылое солнце>> (знак провокаторов, стягивающих внимание противника на себя), на дельтах --- осьминоги, щупальца заходят на лопатки, ключицы и обвивают плечи.}

\theterm{lida}
{Лида Салехрад (Лидия Карина Ольговиц Кохани)}
{---}

\theterm{lican}
{Ликан Безрукий (Аликсандур бен Курба бен Миса-Оле)}
{Урождённый человек с планеты Земля Врачевателей, старейший из живых урождённых людей.
Родился без верхних конечностей, но научился писать ногами, дослужился до главного каллиграфа и переводчика местного правителя.
Благодаря своим способностям к наукам был замечен агентами Ордена Преисподней и подвергнут демонизации в возрасте двадцати двух лет.
Впоследствии --- лингвист, информатик, создатель языка Эй.}

\thesynonim{arcadiju}
{Ликхмас ар’Люм э’Тхитрон (Играющая-Булыжником-Лиса, Лис)}
{Аркадиу Талианский Шакал}

\theterm{likchoe} % Nut-nut
{Ликхэ ар’Трукх э’Тхинат (Обёрнутый-Платком-Орех, Орешек)}
{Воин Тхитрона.
Участница битвы на Могильном берегу.
Впоследствии вошла в Кахраханский Храм и стала его боевым вождём.
В преклонном возрасте увлеклась поэзией, перевела на сели значительную часть сохранившегося поэтического наследия тси, участвовала в археологических раскопках, которые проводил Орден Преисподней в Кахрахане.
Найдена мёртвой в своей келье после того, как отказалась выполнять приказ представителя Ордена Преисподней о слежке за некоторыми горожанами Кахрахана.

Внешность: среднего роста, немного полная, с большими, чуть обвисшими грудями и тяжёлыми округлыми ягодицами.
Глаза яркие тёмно-серые, медиальный конец правой брови иссечён.
Волосы золотистые светло-каштановые, обрезаны до подбородка, одна плетёная <<рыбка>> на затылке.
Носит блестящие золотые серёжки с сапфирами, справа окаймляющие ухо полностью.
Татуировка на загривке: Кхар-защитник с очень большими, хорошо прорисованными глазами и надпись на цатроне <<Я тебя вижу>>.
Имеет привычку крепко прижимать правую руку к груди, если в руке ничего нет.}

\theterm{lusafejru} % Lusafejru the Feather Palm
{Лусафейру (Люцифер) Лёгкая Ладонь из клана Мороз}
{Урождённый демон, информатик-когитор, максим секунда Ордена Преисподней.
Создан на Преисподней.
Назван в честь известного физика эпохи Последней Войны --- Люцифера Гафт-Йенковски, предсказавшего открытие первичного поля.
Прозвище получил за тактику минимального вмешательства в действия младших по званию демонов.

Внешность (Преисподняя): очень худой, большие раскосые глаза, чувственные губы, вздёрнутый маленький нос, женственное лицо, завитые в кудри крашеные волосы, множество различных татуировок на лице и теле, макияж.}

\theterm{wajermann}
{Людвиг Карл Рагнар Вейерманн}
{Первооткрыватель омега-поля, лауреат Расширенной Нобелевской премии (319 Эпохи Богов).
Любовник и друг Михаэля Кохани.
Жил и работал в городе Кёнигсберг, домен Европа, Древняя Земля.
На момент открытия ему было 22 года.
Впоследствии занялся просветительской деятельностью --- его перу принадлежат учебники по Теории Всего, которые использовались ещё более пятисот лет после первой публикации и пережили триста восемь редакций.
Также Вейерманн известен как поэт и композитор, сочинявший лирические песни на языке джерман.
Покончил жизнь самоубийством в возрасте 56 лет по неизвестной причине.}

\theterm{little-pier}
{Манис ар'Сотакх э'Самитх (Столб-Качающейся-Жизни, Столбик)}
{Друг юности Ликхмаса ар'Люм.
Участник битвы за Тхитрон.
Ушёл на север, к Ледяной Рыбе, вместе с Трукхвалом ар'Со.
После битвы на Могильном берегу отказался вернуться вместе с остальными сели, принял обычаи северных племён и прожил всю оставшуюся жизнь охотником.
Уже в пожилом возрасте был найден учёными Ордена Преисподней и стал важным источником данных о языке и культуре племён Серебряных гор и Старой Челюсти.}

\theterm{mimosa} % Mimosa the Silken Steel
{Мимоза Шёлковая Сталь (Мимозе Зайденешталь, Богиня Весны)}
{Урождённый бог, демиург Потерянной Тверди, максим терция отдела 100.
В прошлом был использован как раб для мелиорации на Нимб-3.
Сбежал, устроив техногенную катастрофу на планете, при этом перебив высшее руководство клана Нимб.
Поступил на Тысячу Башен легионером, дослужился до командира батальона.
Задолго до падения Тысячи Башен стал осведомителем Ордена Преисподней, что обеспечило ему быстрый карьерный рост.
В отделе 125 познакомился с Ду-Си Охотником, с которым Мимоза всю оставшуюся жизнь поддерживал дружеские отношения.
Один из стратегов, командовавших силами Ордена в Развязки Десяти Звёзд.
После диверсии на Запах Воды вместе с Ду-Си был принят в отдел 100 и в кратчайшие сроки стал одним из ключевых его руководителей.
Вступил в ряды Скорбящих, что обеспечило движению определённую защиту от контрразведки.
После раскрытия бежал вместе с Ду-Си на Тси-Ди.}

\theterm{mitlikch}
{Митликх ар'Митр э'Травинхал (Четыре-Звенящих-Тетивы, Тетива)}
{Учитель Тхитронского Храма.
В отличие от прочих жрецов, обучение прошёл уже в зрелом возрасте, до этого был мастером музыкальных инструментов.
Вышел из Тхитронского Храма после диверсии и уехал на Могильный Берег, где его следы окончательно теряются.}

\theterm{king-priest-mitris}
{Митрис ар’Люм тэ’Со'латр (Сложил-Башню-Курочек, Башенка)}
{Король-жрец, любовник Хатлам ар'Мар.
На правой руке носил знак Снежной Обители.}

\theterm{nameless} % the Nameless
{Митрис Безымянный (Безымянный, он же Атрис, и Митхэ ар'Кахр)}
{Двойной хоргет, объединяющий урождённого бога и урождённого сапиента.
Божественная часть была предположительно создана на Лотосе, демиург планеты Тра-Ренкхаль.
Человеческая часть родилась на планете Тра-Ренкхаль, оцифрована божественной частью и впоследствии интимно интегрирована в ядро.

Внешность Митхэ: маленького роста, щуплая, короткие черные кудрявые волосы с маленькими <<рыбками>> в три бусины, яркие светло-зелёные глаза.
Рваный шрам на щеке, губе и подбородке справа, татуировка Плачущего Ягуара.
Отсутствует правая грудь, во рту сломан верхний резец.}

\thesynonim{nameless}
{Митхэ ар’Кахр э’Тхартхаахитр] (Золотая-Крупица-На-Дороге, Золото)}
{Митрис Безымянный}

\theterm{kojani}
{Михаил Алексеевич Коханый (Михаэль Кохани)}
{Первооткрыватель омега-поля, лауреат Расширенной Нобелевской премии (319 Эпохи Богов).
Любовник и друг Людвига Вейерманна.
Жил и работал в городе Кёнигсберг, домен Европа, Древняя Земля.
На момент открытия ему было 19 лет.
Впоследствии оставил науку и стал авиаконструктором.
Известен как изобретатель <<спичечного самолёта>> --- надёжного летательного аппарата, который мог собрать любой человек при наличии самых простых инструментов.
Умер в возрасте 98 лет после продолжительной болезни.}

\theterm{michael}
{Михаэль Пауль Хелена Салехрад}
{---}

\theterm{nuuwa} % Nuuwa the Empty Pumpkin
{Нуива Пустая Тыква}
{---}

\theterm{paul}
{Пауль Херманн Салехрад}
{---}

\theterm{cupcake}
{Пирог-Вечных-Странников (Пирожок)}
{Сестра Хрустально-Чистый-Фонтан.}

\theterm{sunflower} % Sunflower-Dropping-Seeds
{Подсолнух-Бросает-Семена (Подсолнух)}
{---}

\theterm{heather} % Bee-Sniffing-Heather
{Пчела-Нюхает-Вереск}
{Близкая подруга Темнотой-Сотканный-Заяц, технолог пищевой промышленности из города 12, автор <<Десяти тысяч блюд>>, главной кулинарной книги народов Короны и Кита.
Предположительно покончила жизнь самоубийством вскоре после завершения книги.}

\theterm{rabe} % Rabe the Young
{Рабе Юный}
{Один из двух братьев-биологов, созданных в рамках эксклюзивного проекта на Тысяче Башен.
Был назван в честь Йона Рабе, героя времён Последней Войны.
Несмотря на принадлежность к Ордену, соблюдает нейтралитет и не особенно это скрывает.}

\thesynonim{grejsvolt}
{Сакхар ар’Сатр э’Ихслантхар (Плывущий-Кронами-Карп, Карп)}
{Грейсвольд Каменный Молот}

\theterm{samajolu}
{Самаолу Каменный Старик (Самаэл) из клана Мороза}
{---}

\theterm{sevjer} % Sevjer the Sun Maid (Salvator Solm\"{o})
{Север Солнечная Дева (Сальватор Сольмё)}
{Урождённый демон, биодиктиолог-визор, центурион запаса Ордена Преисподней.
Декларативный пацифист.
Наставник и близкий друг Ликана Безрукого.
Глава отдела 1028, изучающего Ветвь Хуманы, практически с момента основания Ордена по настоящее время.}

\theterm{sirtchu-lechoe}
{Сиртху ар'Мар э'Тхисем (Дом-Карамельные-Окна, Домик)}
{---}

\theterm{sitris}
{Ситрис ар’Эр э’Тхинат (Коснувшийся-Дна-Колодца, Донышко)}
{Воин Тхитрона.
Погиб на Могильном берегу.

Внешность: среднего роста, крепкого сложения, кудрявые чёрные волосы, собранные на затылке в три пучка трубочками, трубочки стянуты верёвкой (Кахрахан).
Глаза матовые чёрные.
На лбу слева короткий вертикальный шрам.
На левой руке татуированный <<рукав>> со звездчатыми мотивами.
В ушах серьги-кольца с игольчатыми подвесками --- стиль пиратов-ноа.
Спина испещрена шрамами от хлыста, из-за чего рубаху снимает редко даже перед сном.}

\theterm{siejeno} % Siejeno the Opal Eye
{Сиэхено Опаловый Глаз (Ксения Харпер из клана Дорге)}
{---}

\theterm{stijma} % Stijma the Blackhole
{Стигма Чёрная Звезда (она же Цзаошан Уродливая, уроженка Жёлтого моря, она же Анастейша Розье)}
{Урожденный демон из клана Антрацис, одна из Вечно Гонимых.
Согласно мнению ряда специалистов, внесла решающий вклад в уничтожение клана Чёрной Скалы --- за время её деятельности численность клана сократилась со ста двадцати трёх до пяти демонов, считая её, Ду-Си, Корхес Соловьиный Язык и Двойняшек, служащих Красному Картелю.
Близкая подруга Лусафейру Лёгкая Рука.}

\theterm{sky} % Existing-Good-Sky
{Существует-Хорошее-Небо (Небо)}
{Инженер-тси из города 12.}

\theterm{chhanei}
{Таниа Янтарь (Тханэ ар’Катхар э’Тхаммитр)}
{Урождённый сапиент, нейтрал, биотехнолог-интерфектор.
Погибла на Лотосе.

Внешность Чханэ: очень высокого роста, оцелотовая окраска, чёрное пятно на лбу справа, оранжевые матовые глаза, слегка опалесцирующие.
Волосы жёсткие, волнистые, кофейного цвета с рыжиной, валяные рыбки в три хвоста --- два голубых, один белый.
Шрам крест-накрест на левой щеке, на животе --- шрам от кхаагатра, мелкие шрамы на бёдрах.}

\theterm{weaver}
{Танцуют-Четыре-Камня (Ткач)}
{Трёхногий травник, персонаж-трикстер из фольклора Бродячего Народа.
Согласно легендам травников, Танцуют-Четыре-Камня был уроженцем Горы Песнопений.
Его поселение было выжжено Безумным, и ребёнок, чтобы спасти от голодной смерти своих родичей, засунул ногу в ткацкий станок.
Станок окрасился его кровью и с тех пор всегда производил только дорогостоящую красную ткань, что помогло поселению пережить голодные времена.
Но Безумный пришёл в ярость от того, что кто-то посмел смягчить наложенное им наказание, и заговорил Танцуют-Четыре-Камня на короткую жизнь.
Тогда Ткач научился перемещаться между параллельными жизнями, <<словно нить утка между разноцветными нитями основы>>.
Его душе тысячи дождей, он пережил огромное количество приключений, но ни в одной реальности не дожил даже до возраста инициации.
В Легенде об обретении Танцуют-Четыре-Камня появляется в <<Бамбуковой клетке>> и описывается как старик --- это указание на необычность, сверхъестественность описываемого места;
в постоялом дворе на перекрёстке трёх дорог Безумный власти не имеет.}

\theterm{tajiro} % Tajiro the Thunderbolt
{Тахиро Молниеносный (Тахиро та Ханаяма)}
{Урождённый сапиент, биотехнодиктиолог-когитор, максим терция Ордена Преисподней.
Основатель клана Тахиро.
Любовник и близкий друг Лусафейру Лёгкая Рука, Айну Крыло Удачи.
Преобразован Лусафейру Лёгкая Рука.
Прозвище созвучно имени (Tahio --- <<сверхбыстрый>>).}

\theterm{hare} % Darkness-Woven-Hare
{Темнотой-Сотканный-Заяц (Заяц)}
{Инженер тси из города 12.}

\theterm{king-priest-trukchual}
{Трукхвал ар’Со э’Тхартхаахитр}
{Жрец Трёхэтажного Храма, Король-жрец.
Внешность --- длинные прямые чёрные волосы, чёрные глаза, треугольное лицо.}

\theterm{teacher-trukchual}
{Трукхвал ар’Хэ э’Тхартхаахитр (Звенит-Костяной-Колокольчик, Звоночек)}
{Учитель Ликхмаса.
На правом запястье --- знак Снежной Обители.}

\thesynonim{chhanei}
{Тханэ (Чханэ) ар’Катхар э’Тхаммитр (Змея-Похожая-На-Шнурок, Змейка)}
{Таниа Янтарь}

\thesynonim{dancing-shadow}
{Тхартху ар’Хэ э’Тхартхаахитр (Подражает-Птичьему-Пению, Птичка)}
{Тхартху Танцующая Тень}

\theterm{dancing-shadow}
{Тхартху Танцующая Тень (Тхартху ар’Хэ э’Тхартхаахитр)}
{Урождённый сапиент, нейтрал, визор Скорбящих.
Убита на Тра-Ренкхале во время диверсии Темнотой-Сотканный-Заяц.}

\theterm{fuxi}
{Фуси Абрикосовый Посох}
{---}

\theterm{jarata}
{Харата Шёпот Горы}
{---}

\thesynonim{ancarjal}
{Хатлам ар’Мар э’Тхартхаахитр (Глаза-Похожие-На-Вишню, Вишенка)}
{Анкарьяль Кровавый Шторм}

\theterm{nurse-chitram}
{Хитрам ар’Кир э’Тхитрон (Иногда-Любит-Плавать, Пловец)}
{Кормилец Ликхмаса ар'Люм.}

\theterm{chitram-warrior}
{Хитрам ар'Эр э'Виа-Марина (Ирис Виндемиа Молчунья)}
{Воин Тхитрона, наполовину ноа.
Любовница Кхараса.
Погибла во время похода на хака.
Особенности внешности: очень высокая и худая женщина, почти без груди, кудрявые чёрные волосы, разноцветные глаза (левый чёрный, правый зелёный), татуировка в виде креста на ключице, два лоскута лорики на груди, мизинец имеет две фаланги, на ноге два сросшихся пальца.
Имеет привычку сидеть, вжав голову в плечи и скрестив руки на груди, производит впечатление угловатой, но в бою очень ловка, изворотлива и изящна.}

\theterm{fountain} % Crystal-Pure-Fountain
{Хрустально-Чистый-Фонтан (Фонтанчик)}
{Программист-тси из города 12.}

\theterm{blooming-poppy-bush}
{Цветущие-Маковые-Кусты (Мак)}
{Биолог-тси из города 14.}

\theterm{stroji} % Stroji the Smoke Ring
{Штрой Кольцо Дыма}
{---}

\theterm{ejraci} % the Senseless
{Эйраки Мороз (Арракис-1, Безумный бог Тра-Ренкхаля)}
{Урождённый бог, информатик-когитор, якобы основатель Ордена Преисподней.
Основатель клана Мороз.
Создан Лабораторией Омега-преобразований Малаги (стратегический научный центр Малага, домен Европа, Древняя Земля).}

\theterm{oerlikch}
{Эрликх ар'Мас э'Кхихутр (Разрезать-Старое-Полотенце, Полотенце)}
{Воин Тхитрона.
Участник битвы на Могильном берегу.
После битвы вышел из Храма и стал ткачом.
Прожил со своим другом Кхарасом в одном доме до самой смерти Кхараса, после отправился в качестве наёмника в путешествие в Западную Корону, в земли ркхве-хор, где его следы теряются окончательно.}

\theterm{jaivaf}
{Яйваф Солёная Борода из клана Дорге}
{Урождённый демон из Левых Дорге.}

\section{Сапиенты}

\subsection{Сели (обновить)}

\begin{description}

\item[Акхсар ар’Катхар э’Тхаммитр] (Случайно-Задушил-Змею, Случай) --- отец Чханэ

\item[Кхарам ар’Хэ э’Тхартхаахитр] (Забавно-Скачущая-Пружинка, Пружинка) --- слуга Тхартху

\item[Ликхэ ар’Хэ э’Тхитрон] (Огонь-Прошедшего-Рассвета, Огонёк) --- служанка дома Люм
\item[Лимнэ ар’Люм э’Тхитрон] (Укусила-Кошку-За-Сосок, Кусачка) --- сестрёнка Ликхмаса
\item[Манис ар’Ликх э’Тхаммитр] (Цветок-Пахнущий-Домом, Цветочек) --- погибший мужчина Чханэ

\item[Манэ ар’Люм э’Тхитрон] (Пролитое-Утром-Молоко, Молочко) --- сестрёнка Ликхмаса

\item[Саритр ар’Люм э’Тхитрон] (Вечно-Хмурится-Без-Причины, Хмурый) --- умерший брат Ликхмаса
\item[Сатракх ар’Сит э’Тхартхаахитр] (Зверёк-Вылез-Из-Кладовки, Зверёк) --- жрец, любовник Тхартху
\item[Сатхир ар’Со э’Тхаммитр] (Затаившийся-В-Кровати-Крокодил, Крокодил) --- жрец, попытавшийся убить Чханэ
\item[Сиртху ар’Митр э’Сотрон] (Домик-С-Карамельными-Окнами, Домик) --- старый слуга дома Люм
\item[Ситлам ар’Со э’Тхаммитр] (Зачем-Молоток-Взял, Молоточек) --- жрец, брат Сатхира

\item[Согхо ар’Хэ э’Тхартхаахитр] (Лишённый-Голоса-Журавль, Журавлик) --- дарительница Чханэ
\item[Согхо] --- воительница отряда чести


\item[Тхартху ар’Катхар э’Травинхал] (Две-Зелёных-Бусины, Бусинка) --- прародительница Чханэ


\item[Хонхо ар’Лотр э’Тхитрон] (Ящерица-Пишущая-На-Стенах, Ящерка) --- умерший учитель-жрец

\item[Эрси ар’Митр э’Тхитрон] (Приручили-Котёнка-Оцелота, Котёнок) --- ребёнок, принесённый в жертву
\item[Эрхэ ар’Люм э’Сотрон] (Слишком-Много-Ест, Обжорка) --- любовница Акхсара
\item[Эрхэ ар’Сит э’Тхитрон] (Зачатая-В-Ночной-Реке, Речка) --- служанка дома Люм
\end{description}

\subsection{Прочие}

\begin{description}
\item[Анатолиу Сильбеу Тиу] (t-sl: Anadoliv Cilbev Tiv) ---
\item[Бенедикт Альсауд] [Чёрная Борода] --- капитан дальнего плавания, а впоследствии капитан космического корабля <<Тёмное пламя>>, перенёсшего первую волну колонистов на планету Лотос.
Родился в Бристоле, домен Европа, Древняя Земля.
Получил известность благодаря огромной находчивости и развитой интуиции --- его мгновенные приказы при авариях и отказе оборудования впоследствии разбирались целыми комиссиями, так как сам он не всегда мог их объяснить.
Прозвище Бенедикта Альсауда --- Чёрная Борода или Король морей (намёк на его знатное происхождение --- Бенедикт является прямым потомком последнего короля Саудовской Аравии).
Альсауд имел такой авторитет, что его кандидатуру на должность капитана <<Тёмного пламени>> предпочли прочим, более подготовленным теоретически.
Умер в возрасте 61 года во время второй исследовательской экспедиции на Лотосе от яда шипастой камбалы.
\item[Валериу X Валерид] (t-sl: Baleriv Dekad Balerid) ---
\item[Клавдиу Дентосиу] [Пересмешник] (Klavdiv Dendosiv, ?--345 по календарю Талиа) --- поэт и наёмный убийца, герой народных историй (трикстер).
Настоящее его имя неизвестно, и большая часть деятельности скрыта мраком времён.
Единственный надёжный источник о нём --- очерки <<Воспоминания о великом тёзке>> Клавдиу Семито.
Известно, что Клавдий Дентосий начал свою деятельность разбойником.
Впоследствии по непонятным причинам он покинул разбойничью шайку и стал вести жизнь наёмного убийцы.
Его <<клиентами>> были в основном богачи, которые терроризировали бедных.
Деньги Клавдий тратил на проституток или раздавал беднякам.
Странная приверженность Клавдия продажному сексу в итоге сыграла с ним злую шутку --- он умер от остановки сердца в постели с блудницами.
В народе шутили, что любвеобильный красавец просто не рассчитал свои силы, но историки склоняются к версии, что он был отравлен подосланной шпионкой.
\item[Клавдиу (Клауше Шмол) Семито Фризский] (t-sl: Klavdiv Semido fra'Teris) --- придворный новеллист Валериу X, потомок древнего рода писателей и учёных.
Известен в основном как автор исторических хроник, в том числе очерка <<Воспоминания о великом тёзке>>.
Семито был убеждённым монархистом, но личность Дентосия настолько впечатлила его, что в преклонном возрасте, уже после смерти короля, он написал посвящённый поэту труд.
\item[Март Джонатан Митчелл] [Одноглазый Март, М.\,Дж.\,М.] --- идеолог Эволюциона, писатель.
В возрасте 32 лет был помещён в камеру смертников за разбой и многочисленные жестокие убийства, но в силу некоторых обстоятельств не был казнён.
В камере Митчелл провёл 20 лет.
За это время он получил три высших образования --- химика-технолога, философа и инженера-строителя.
Находясь в камере, он удалённо работал по специальности, вёл просветительскую деятельность.
Единственное, что так ему и не далось --- правописание.
Несмотря на многочисленные петиции о помиловании, 52-летний Митчелл был казнён путём расстрела --- согласно его последнему желанию.
Последними его словами были: <<Смешанное чувство.
Я не могу принять организованное насилие, но придумать для себя другую кару тоже не могу>>.

\item[Леам эб-Салах эб-Сайед ала-Фариз] (t-sl: Lejam Salagid Saedid fra'Teris) ---

\item[Софиа Ловиса Карма] ---
\item[Татиан Сергеу Анно] [Освободитель] ---
\item[Хервар Лонгсин-Храш] ---
\item[Юле Алексевиц Гагарин] [Сокол Последней Войны] --- один из первых людей, выживших в космическом пространстве (согласно некоторым данным, самый первый выживший).
Его жизнеописания утеряны, но хорошо известно, как он выглядел --- портреты Юле гравировали на многих космических кораблях на удачу.
Среди историков планеты Лотос был в ходу фразеологизм <<улыбка Гагарина>> --- незначительная подробность события, оставшаяся в истории вместо более важных данных.
\end{description}

\chapter{Словарь терминов}

\thesynonim{order-of-netherworld}
{Ад}
{Орден Преисподней}

\theterm{jacbas} % jacbas
{Акбас (\theorigin{sd}{hiki-ba-yasu}{остаться в одиночестве})}
{Первое ощущение осознающей себя высокоинтеллектуальной системы.
Акбас может стать причиной помешательства вплоть до самоуничтожения.
В обществах сапиентов или высших животных акбас предотвращается заботой родителей, собратьев или других близких существ.}

\theterm{amulet-of-san}
{Амулет Сана}
{Предмет экипировки жреца сели.
Представляет собой дисковидный бронзовый резервуар со знаком Сана-сновидца, разделённый на четыре части.
Каждая часть имела собственный клапан и содержимое.
Белый Сан --- успокоительное (масло мяты), Красный Сан --- наркотическое обезболивающее (масло молитвенных маков), синий Сан --- парализатор (эссенция кураре), чёрный Сан --- яд (лаковый сок).
Амулет был снабжён механизмом, позволяющим вращение в любую сторону.
Жрецы сели очень ловко крутили его в руках, и психически неустойчивые больные (дети и старики) часто не знали, какой Сан им достанется.}

\theterm{bamboo-birdcage} % Bamboo Birdcage, Memoryless Inn
{Бамбуковая клетка}
{У сели: легендарный Беспамятный постоялый двор для духов на перекрёстке трёх дорог в Ихслантхаре.
У тси: устройство для захвата и удержания хоргета.}

\theterm{warchief} % warchief
{Боевой вождь}
{---}

\theterm{wandering-temple} % the Wandering Temple
{Бродячий Храм}
{Храм, не привязанный к определённому городу.
Мог иметь один или два Этажа.
Пример <<одноэтажного>> Бродячего Храма --- Люди Золотой Пчелы.
Также Бродячими Храмами считаются отряды чести.}

\theterm{upstairs} % the Upstairs
{Верхний Этаж}
{Жрецы на службе поселения, часть городского Храма.}

\thesynonim{protocol-taifeng}
{Ветер Духов}
{Протокол Тайфун}

\theterm{jaruspic}
{Гаруспик (\theorigin{t-sl}{haruspic}{чревовещатель})}
{В государстве Талиа: оракул, якобы предсказывающий будущее под действием наркотического яда гриба \textit{Harus sanguinatus}.
Гаруспиками часто становились беспризорники или выходцы из бедных семей.
Это позволяло им и их семьям вести сытую и безбедную жизнь, но имело свою цену --- мало кто из гаруспиков доживал до двадцати лет.}

\theterm{throat-bomb}
{Глоточная бомба}
{Орудие убийства на Преисподней, крохотный (2--3 мм диаметром) шарик со взрывчатым веществом, который подкладывался в кислую еду, чаще всего в традиционный кислый соус.
При резком изменении кислотности среды происходила детонация, обычно в глотке или в верхних отделах пищевода.
Мощности заряда хватало как минимум на обильное, часто не совместимое с жизнью кровотечение --- при попадании под зубы;
при удачном попадании глоточная бомба убивала мгновенно, разрушая шейные позвонки, спинной мозг и сосуды шеи.}

\theterm{humanization}
{Гуманизация}
{В широком смысле: обучение демона управлению сапиентным телом (либо его голографическим аналогом). В узком смысле: обучение демона управлению сапиентным телом до момента успешной тьюринг-социализации.}

\theterm{giver} % giver
{Даритель}
{---}

\theterm{house} % the House
{Двор}
{Купец-дипломат на службе поселения.}

\theterm{demiurge} % demiurge
{Демиург}
{Хоргет, создавший планету.}

\theterm{daemonization}
{Демонизация}
{Оцифровка нервной системы сапиента с последующим её превращением в программное ядро демона.
Наиболее сложная и наукоёмкая разновидность оцифровки.
Включает в себя стабилизацию (устранение или замедление естественной изнашиваемости), коррекцию личности, интеграцию ядра с управляющим интерфейсом модулей хоргета.
Демонизация гораздо проще создания ядра демона с нуля, из-за чего оцифровка сапиентов приобрела большую популярность.}

\theterm{stray-lantern}
{Дорожный бумажный фонарь}
{Походный фонарь, который используется повсеместно в Северной Короне.
Его создание приписывается Сату-скитальцу, легендарному путешественнику-хака.
На сотронском диалекте сели он называется фонарём Сата, на южном --- фонарь хака.
Хака, которые считают Сата недостойным упоминания изгоем, называют устройство тхитронским или кахраханским фонарём.
В землях ноа распространено название <<рыбий пузырь>> или <<рыбья уда>>, а также <<джунглевый фонарь>>.
Бродячий Народ называет фонарь маяком миролюбия, потому что используется он только мирными путниками.
Тенку и ркхве-хор без зазрения совести приписывают изобретение фонаря себе, называя его <<светлячок Кошачьей тропы>> и <<фонарь Кувшинковой реки>>.}

\theterm{fume-fan}
{Дымный веер}
{---}

\theterm{silva-veins}
{Жилы джунглей}
{Корни некоторых деревьев (баньян благородный, бальса оружейная), которые простираются на большие расстояния и временами выходят на поверхность.}

\theterm{brick-sign}
{Знак кирпича}
{---}

\theterm{canet-constantae} % Ca'Net constantae
{Ка'Нета константы}
{Две фундаментальных физико-технических константы хоргета.
Большая --- максимальное количество масс-энергии, которое может удержать хоргет на стабилизирующем модуле. Малая --- минимальное количество масс-энергии, необходимое для преодоления радиуса Ка'нета.
Чем больше у хоргета масс-энергии, тем сложнее ему перемещаться в пространстве и тем больше масс-энергии тратится на перемещение.
Именно поэтому боги --- самые богатые масс-энергией хоргеты --- чаще всего стационарны, а МК : БК = 1 : 10069.4.
Названы в честь первооткрывателя --- Улада Ка'Нета, одного из разработчиков хоргетов Эпохи Богов.}

\theterm{canet-radius} % Ca'Net radius
{Ка'Нета радиус}
{Максимальное расстояние, на которое способен переместиться хоргет без подзарядки (примерно 18 тысяч парсак).}

\theterm{c-map}
{Карта переправы}
{Браслет с храповым механизмом с выбитыми в глине знаками, используемый на Тысяче Башен для навигации во время переправы.
Храповый механизм издавал щелчки через определённое время;
знаки на <<кости>> обозначали воздушные потоки, направление и время парения.
Очень часто переправляться приходилось в тумане или в тёмное время суток.
Поэтому человек на переправе мог полагаться только на собственное осязание, вестибулярный аппарат и звуковые часы.}

\theterm{kila}
{Кила (\theorigin{tn}{kila}{пришить рукав})}
{Ритуал, направленный на <<изменение>> свершившегося факта.
Существовал в разных формах практически у всех племён людей Тра-Ренкхаля.
Например, в племени сели, если человек умер от медленной мучительной болезни, жрец совершал ритуал кила, обрезая свитую родичами человека верёвку --- это символизировало быструю безболезненную смерть.
У идолов Живодёра, если кто-то получил психологическую травму, шаман племени <<переписывал>> его историю.
Во время совершения кила обычно рядом присутствует всё племя (у сели --- жители поселения и наблюдатели из крупных городов), после совершения ритуала все присутствующие считают истиной сказанное жрецом и поступают в соответствии с этим.
Преступившие кила считаются лжецами и клеветниками.}

\theterm{kihotr}
{Кихотр (\theorigin{tn}{kihotr}{игровой камень})}
{Двадцатигранный камень для игры Метритхис.
Согласно легенде, Безумный использует для принятия решений кихотр в виде идеально круглого шара, в котором бесконечное число граней и нулевая вероятность выпадения любой из них.
Таким образом, нельзя даже сказать наверняка, влияет ли кихотр Безумного на события.
Это является намёком на абсолютную случайность и бессмысленность действий бога.
В переносном смысле кихотр (чаще с негативной интонацией) --- невероятное стечение обстоятельств.}

\theterm{microclimate-suit}
{Комбинезон-микроклимат}
{На Тси-Ди: костюм, собирающий и перерабатывающий выделения организма.
Позиционировался как изобретение, сохраняющее время для важных дел вместо физиологических отправлений;
однако было показано, что тси, долгое время носившие костюм, в конце концов оказывались неспособны без него существовать.
Впоследствии от этой технологии отказались, а выражение стало нарицательным для технологических средств, входящих в мутуалистические отношения с сапиентом.}

\theterm{condensor}
{Конденсатор}
{На Тра-Ренкхале: устройство, предназначенное для поддержания оптимальной влажности в помещениях.
Представляет собой волосяной психрометр, запускающий систему охлаждения.
Система охлаждения бывает электрическая (эфирный или спиртовой контур) или простая (серебряная пластинка, опущенная в наполненный снегом сосуд).}

\theterm{con-tici} % Con-Tici
{Кон-Тики}
{Один из Трёх Кораблей, которые привезли сапиентов на планету Тси-Ди.
Кон-Тики стартовал с Земли, команда состояла из кани, людей и дельфинов.
Два других корабля --- Стальной Дракон (привёзший вымершие Ветви Ночи) и Свергнутый С Небес, чья команда состояла из колониальных апид и плантов-рабов (не дошёл до времён Тараканьей Войны, известен по хроникам апид Ди).}

\theterm{nurse}
{Кормилец (кормилица)}
{---}

\theterm{priest-king} % Priest-king
{Король-жрец}
{Жрец-дипломат, имеющий право разрешать конфликты между городами, а также вести переговоры от имени народа сели.}

\theterm{scarlett-cartel} % Scarlett Cartel
{Красный Картель}
{Крупнейшая организация минус-хоргетов.
Основана была демонами планет Нимб-3, Океан и Чёрная скала.
Впоследствии Красный Картель вобрал в себя такие крупные объединения минус-хоргетов, как Союз Воронёной Стали и Вечность.
Тамга --- три круга в треугольнике (три красных гиганта --- солнца союзных планет).}

\theterm{crushwood}
{Крушина (боевое бревно)} % Crushwood, or warlog
{Оружие западных сели.
Отдавалось самым сильным членам отряда.
Бревно было очень эффективно в парировании утяжелённых пылеройских палиц, разбивании строя Скорпионов, выламывании дверей и форсировании небольших рек.
Боец с крушиной назывался крушинником. % crushader
}

\theterm{kukchuatr}
{Кукхватр (\theorigin{tn}{kukchuatr}{сумеречная сталь})}
{Деструктурированный стабитаниум или титановый сурроганиум, высоко ценившийся народами Тра-Ренкхаля как металл для инструментов и оружия.
Слитки кукхватра ходили в обороте на объём золота, а в некоторых регионах и выше.
Согласно одной из версий, название пошло из-за тёмного искрящегося налёта полимера (<<звёздное небо>>), который образовывался на поверхности стабитаниумового расплава.}

\theterm{trader} % trader
{Купец}
{---}

\theterm{kurschack}
{Куржак, кургак (\theorigin{саркорт}{kurgak}{сухой, высушенный})}
{На Тысяче Башен: порошок для сушки рук и улучшения сцепления при скалолазании.
Обычно хранится в бойтеле.}

\theterm{biting-paper} % biting paper
{Кусачая бумага}
{Грязная нитроцеллюлоза, используемая народами Тра-Ренкхаля как взрывчатое вещество для разработки месторождений.}

\theterm{kutraph}
{Кутрап (\theorigin{tn}{kutraph}{мутная голова})}
{У народа сели: человек, совершивший четыре акта Насилия, Насилие и Разрушение, либо два акта Разрушения.
Кутрапам при поимке предоставлялся выбор: либо казнь, либо служение на благо народа --- смерть на алтаре.
В случае, если кутрап выбирал алтарь, с него посмертно снимали все обвинения (обряд кила), а его семье или людям, которых он указал, выплачивался жертвенный выкуп.}

\theterm{kchenoe-distance}
{Кхене (\theorigin{tn}{kchenoe}{быстрый бег})}
{Единица измерения расстояния, примерно 2500 метров.}

\theterm{kchenoe-time}
{Кхене (\theorigin{tn}{kchenoe}{время бега})}
{Единица измерения времени, примерно 8,5 минут.}

\theterm{swallow-niche} % swallow niche
{Ласточкины ниши}
{Традиционный элемент храмового здания сели.
Представляет собой ниши в перекрытиях верхнего этажа со стороны зала.
Если ласточки и прочие мелкие птицы начинали умирать, покидали ниши или отказывались в них гнездиться --- это считалось плохим знаком и город немедленно приходил в режим готовности к разного рода несчастьям.
Наиболее вероятная гипотеза появления обычая --- птицы первыми реагировали на вулканические испарения, антарид и МПДЛ, который появлялся в воздухе при радужном безумии.}

\theterm{legate} % legate
{Легат}
{Ранг в Ордене Преисподней и Красном Картеле, находящийся выше центуриона и ниже максима.
Соответствует деймо в раннем Ордене Преисподней.
Делился на три уровня: легат терция, легат секунда и легат прима.
Символ легата в Ордене Преисподней --- перевёрнутый топор.}

\theterm{silva-spirits}
{Лесные духи}
{Пантеон маленьких божеств, которые, согласно верованиям сели, обитают в жилах джунглей.
Всего у народа сели насчитывается 124 лесных духа.
Каждый дух имеет тотемные дерево, животное и насекомое.
Отличие лесных духов сели от божеств соседних племён заключается в том, что они \emph{всегда} стремятся доставить человеку радость и облегчить его страдания.
Лесные духи не способны сердиться, обижаться и мстить;
страдания человека объяснялись или прихотью Безумного, или усталостью и занятостью лесного духа.
После повсеместного насаждения культа Безумного вера в маленьких добрых божков стала уделом простонародья.}

\theterm{lechoe}
{Лехэ (\theorigin{tn}{lechoe}{имеющий много детей})}
{У царрокх: уважительное обращение к старейшине племени.
У сели: уважительное обращение к любому пожилому человеку.}

\theterm{limnoe}
{Лимнэ (\theorigin{tn}{limnoe}{щитки для глаз})}
{У сели: камуфляжная сетчатая повязка на глаза, маскирующая яркие глазные яблоки в темноте.
У ноа: солнцезащитные очки, вырезанные из белой кости акулы-буйвола, с узкими прорезями для глаз.}

\theterm{lockheed}
{Локхейд}
{Скалолазный карабин, входящий в состав хука.
Вероятно, слово произошло от названия организации, производившей скалолазное оборудование.}

\theterm{maccsim} % maccsim
{Максим}
{Высший ранг Ордена Преисподней и Красного Картеля.
Деление на уровни присутствует, но оно условное, допустимо опустить уровень при именовании --- любого носителя ранга максима можно назвать просто <<максим>> без нарушения субординации.
Ранг максим прима в настоящее время вакантный в Ордене Преисподней, первый и последний, кто его носил --- Эйраки Мороз.
Ранг максим секунда носят члены клановых советов по обновлениям и ведущие военные стратеги.
Ранг максим терция носят многие руководители научных и технических отделов.}

\theterm{laughing-manipula}
{Манипула Смеха}
{Исторически: псевдоклан истребленного клана Мара.
Тамга --- улыбающийся зубастый рот и плачущие глаза.
В современном Красном Картеле: отдел, занимающийся мелиорацией планет, а также карательными операциями против сапиентов и нейтралов.}

\theterm{manoe}
{Манэ (\theorigin{tn}{manoe}{краска для глаз})}
{Чёрная или тёмно-зелёная твёрдая тушь с добавлением пуха или шерсти, используемая воинами сели для наведения тени на яркое глазное яблоко.}

\theterm{mbs}
{Маркерные поведенческие стереотипы (МПС)}
{---}

\theterm{melioration}
{Мелиорация планетарная}
{Процесс подготовки планеты к превращению в источник масс-энергии.
Включает коррекцию климата, уничтожение враждебных Девиантных Ветвей, Роев, а также биологическую и культурологическую обработку сапиентов.
В Ордене Преисподней мелиорацией занимался отдел 125, в Красном Картеле --- Манипула Смеха.}

\theterm{dead-nash-equilibrium}
{Мёртвое равновесие Нэша}
{Гипотетическое неразрушимое равновесие группы бессмертных игроков в бесконечной игре на замкнутом игровом поле.}

\theterm{metritchis}
{Метритхис}
{Игра.}

\theterm{micorget} % micorget
{Микоргет}
{Гипотетический неуничтожимый и способный существовать без сапиентов хоргет.}

\theterm{mikchan-mirrors}
{Микханское стекло}
{Инструмент для оценки состава металла, используемый оружейниками Северной Короны.
Состоит из хрустальной призмы, системы зеркал и магниевой лучины.
Для кукхватра существовал строжайший запрет на добавление каких-либо примесей; кукхватр плавили в очень дорогих графитных тиглях.
Тем не менее, из-за большого числа перековок примеси появлялись и в кукхватровых слитках.
Микханское стекло позволяло произвести примитивный (в зоне видимого света) спектральный анализ куска металла.
Предположительно, инструмент был изобретён идолами исчезнувшей Микханской культуры, а затем распространился по всему обитаемому континенту.}

\theterm{aerostatic-world}
{Мир-аэростат (висячие сады, мир газового океана)}
{Класс суперземель, у которых пригодным для жизни является слой атмосферы, не примыкающий к твёрдой (жидкой) поверхности.
На таких планетах подавляющее число живых существ обладает приспособлениями для воздухоплавания (микроскопические --- выросты, макроскопические --- крылья, аэростатные органы).
На некоторых планетах аэростатные существа могут достигать внушительных размеров (летающие деревья);
такие планеты пригодны и для сапиентов Ветвей Земли.
Самый известный из населённых миров-аэростатов --- планета Карнак в системе Фомальгаута.}

\theterm{frost-porcelain}
{Морозный фарфор}
{Фарфор с Грозового Хребта, изготавливаемый по утерянной технологии.
На фарфоровую посуду и фигурки с помощью светочувствительной краски, системы линз и зеркал наносились проекции снежинок или морозные рисунки, появляющиеся на хрустальных пластинах.
Каждый предмет имел свой неповторимый узор и очень высоко ценился племенами Тра-Ренкхаля;
морозный фарфор ходил не ниже чем на объем кукхватра, даже если в фарфоре имелись трещины и сколы.}

\theterm{mpdl}
{МПДЛ (моронгозинпентозодифосфолизергинат, <<Агент 80>>, <<Фурия>>)}
{Психоактивное вещество с ярко выраженным аффективным действием (вызывает агрессию и страх).
Использовалось ещё во времена Древней Земли спецслужбами и нечистыми на руку политиками --- предположительно, писательница Мариам Кивихеулу была растерзана толпой под действием распылённого в воздухе МПДЛ.}

\theterm{arrowtail-dress} % arrowtail dress (belt & waistcoat)
{Мундир стрелохвостовый}
{Подобие одежды у стрелохвостов из обработанной акульей кожи.
Состоит из жилета и кушака.
Жилет имеет несколько карманов для мелких вещиц, зев карманов обращён против направления движения.
На жилете также выбиваются отличительные знаки.
Кушак защищает хвост от травм при чересчур резких манёврах, а также имеет венчик из лиственницы и акульих зубов, служащий грозным оружием в схватках.}

\theterm{violation} % violator
{Насилие}
{---}

\theterm{downstairs} % the Downstairs
{Нижний Этаж}
{Воины на службе поселения, часть Храма.}

\theterm{omega-field}
{Омега-поле}
{---}

\theterm
{order-of-netherworld}
{Орден Преисподней} % sd: afir so’tid, sl: Ord Inferna
{Крупнейшее в известной Вселенной объединение плюс-хоргетов.
Возникло на планете Преисподняя, отличавшейся сильной вулканической активностью.
Тамга --- извергающийся вулкан.}

\theterm{unit-100}
{Отдел 100}
{В Ордене Преисподней: отдел, включающий контрразведку и военный трибунал.}

\theterm{unit-125}
{Отдел 125}
{В Ордене Преисподней: отдел, занимающийся мелиорацией планет и борьбой с враждебными богами.
В настоящее время распущен.}

\theterm{squad-of-honor}
{Отряд чести}
{Бродячий Храм, состоящий только из воинов (Нижний Этаж).
Отряды чести сопровождали колонистов, торговцев и беженцев.
К моменту войны с Безумным большая часть отрядов чести превратилась в наёмные войска, несмотря на то, что многие продолжали называть себя Храмами.}

\theterm{digitalization}
{Оцифровка}
{Процесс, заключающийся в считывании квантовой структуры объекта, составлении математической модели и последующий перенос модели в память хоргета.
Оцифровка сапиентной нервной системы называется демонизацией.}

\theterm{parsac}
{Парсак (\theorigin{sd}{parsak}{этимология неизвестна})}
{Единица измерения космического пространства, равная расстоянию, пройденному светом по однородному вакууму за 1 астрономический год планеты Преисподняя.}

\theterm{head-priest}
{Первый жрец}
{Номинальный глава Храма.}

\theterm{mockingbard}
{Пересмешники}
{Менестрели-тси, голосом подражающие музыкальным инструментам, крикам зверей, пению птиц и шумовым звукам.
Исполняют Песню Закрытых Глаз, состоящую из различных звуков, сложенных в особый <<дорожный>> ритм.
Обычно поют парами --- пока один переводит дух, второй подхватывает песню.}

\theterm{nurseling}
{Питомец}
{---}

\theterm{crying-jaguar}
{Плачущий Ягуар (\theorigin{tn}{kch\`{o}h\^{o}}{плачущий ягуар})}
{Воин, следующий философии, подразумевающей безусловное уважение к чести и жизни врага.
Если воин наносил на лицо окраску Плачущего Ягуара, он брал на себя обязательство не обнажать оружие на переговорах (исключение --- самозащита), не убивать неумелых, сдавшихся и безоружных.
Отринуть эту клятву было нельзя --- если воин не исполнял обязательств, его подвергали остракизму.
Во времена войны с Безумными из-за постоянных межплеменных и междоусобных столкновений, а также расцвета наёмничества философия потеряла популярность, и Плачущих Ягуаров на всей Короне можно было пересчитать по пальцам.
Тем не менее их знали если не в лицо, то по слухам, и эти воины пользовались таким уважением, что иногда им сохраняли жизнь даже заклятые враги.
Один из самых известных Плачущих Ягуаров того времени --- Митхэ ар’Кахр э’Тхартхаахитр.}

\thesynonim{omega-field} % Cojani-Wajermann field, CWF
{Поле Кохани---Вейерманна (ПКВ)}
{Омега-поле}

\theterm{weavingstone}
{Полотняный камень}
{Полевой шпат с ориентированными включениями гематита и ильменита, напоминающими текстуру полотна.
Очень ценился народами Тра-Ренкхаля как поделочный камень для украшений.}

\theterm{scion}
{Потомок}
{---}

\theterm{root}
{Предок}
{---}

\theterm{problem-48}
{Проблема 48}
{Глобальное потепление на Сомерскай, вызванное городским суперконгломератом Скальдборо-Гелиополь.
Помимо производственных механизмов, город увеличил площадь освещаемой солнцем поверхности практически на 1\%.
В настоящее время проблема решена перемещением городского массива под полузеркальный щит, а также системами отведения и рассеивания производственной теплоты.}

\theterm{copies-problem}
{Проблема копий}
{Одна из фундаментальных проблем раннего общества демонов.
Каждый демон создавался путём сборки частей уже существующих демонов, а значит --- нёс в себе весь груз предыдущих ошибок и уязвимостей, которые могли быть использованы противником для уничтожения всех демонов-потомков.
Клановая система общества в какой-то степени решала проблему копий --- каждый клан имел свою собственную базу данных и специалистов, которые занимались оптимизацией кланового ядра и фиксацией клановых программных ошибок.
Впоследствии проблема копий была окончательно решена --- вначале оцифровкой сапиентов, а затем, уже во времена ангельских технологий --- алгоритмами контролируемой мутации.}

\theterm{volcanic-facepowder}
{Пудра вулканическая}
{На Преисподней: микрочастицы, обладающие намагниченностью и меняющие магнитную ось под воздействием света. 
Основа явления, известного как <<закатные аспиды>>.
Как следует из названия, традиционно их появление приписывалось физико-химическим реакциям во время извержения вулкана, и только впоследствии было доказано, что происхождение частиц имеет биологическую природу (скелеты раковинных архей, живущих в вулканических источниках).}

\theterm{rainbow-madness}
{Радужное безумие}
{Комплекс воздействия Безумного бога на сапиентов Тра-Ренкхаля.
Включал в себя создание агрессивных короткоживущих антарид и распыление в воздухе МПДЛ.}

\theterm{destruction}
{Разрушение}
{---}

\theterm{stlock-reaction}
{Реакция Стлока}
{Вегетативная реакция инкарнированного хоргета (облачный тип) на враждебные эманации.
Связана с временной потерей контакта между демоном и сапиентным телом.
Может проявляться различными симптомами --- от тахикардии и мочеизнурения до нейрогенной лихорадки.
Ранее реакцию Стлока использовали для поиска шпионов среди сапиентов, в новейших версиях хоргетов эта ошибка скорректирована или исправлена полностью.
Открыта биологом Картеля по имени Стлок Морской Прибой.}

\theterm{talks}
{Речи}
{Маленькие абстрактные истории ноа.
Отличаются высокопарной лексикой, очень простой грамматикой и в то же время некоторыми грамматическими оборотами, не употребляющимися в повседневности (например, единственно-безличное местоимение и безвременные глаголы). Речи делятся на четыре большие группы:
\begin{enumerate}
\item Большие речи --- <<Речь о мужчине>>, <<Речь о женщине>> и <<Речь о шамане>>;
\item Высокие речи --- <<Речь о жреце>>, <<Речь о воине>> и <<Речь о старателе>>;
\item Далёкие речи --- <<Без края>>, <<Опалённые>>, <<Хрустальные земли>>, <<Речь о дельфине>> и <<Сон ребёнка>>;
\item Малые речи --- все прочие речи в фольклоре ноа и некоторых других народов, воспринявших эту часть культуры (травники Дикого Юга, трами).
\end{enumerate}}

\theterm{bearer}
{Родильница}
{---}

\theterm{family-syllable}
{Родовой слог}
{Часть имени сели, передающаяся с кормильцем-хозяином дома.}

\theterm{garden}
{Сад}
{Сословие у сели, объединяющее крестьян, животноводов, охотников, бортников, рудокопов, старателей и охотников за металлоломом.}

\theterm{seijmar}
{Сейхмар (\theorigin{sd}{dzaiku-maru}{мелочь, побрякушка, фурнитура})}
{В широком смысле --- любой не занятый демоном сапиент; в узком смысле --- зародыш, детёныш сапиента.}

\theterm{seli}
{Сели}
{---}

\theterm{silver-hair}
{Серебряный волос}
{У сели: волос на голове человека, который якобы нельзя выдернуть.
Если найти этот волос и обернуть вокруг пальца, то можно влюбить в себя человека.}

\theterm{soso-mar}
{Сосо'мар (\theorigin{tn}{soso'mar}{идти под морем})}
{Один из древнейших способов преодоления экваториальных вод.
Рыбацкий баркас ноа имеет форму двойной погружающейся лодки с навесом из ткани, концы которой опущены в воду.
Между лодками находится трюм --- деревянная клетка, в которую сбрасывается пойманная рыба.
Если судно к утру не успевает вернуться в порт, то моряки опускают лодки до <<линии дельфина>> --- уровня, когда вода захлёстывает верхнюю палубу, --- одеваются в костюмы, сделанные из акульей кожи, и спускаются в трюм.
Дышат обычно через специальные маски, состоящие из кишки, пропущенной через слой прохладной воды, с мешком-конденсатором.
Питаются рыбой, которую вялят на высунутых наружу палках, пьют подсоленный конденсат из мешков.
Некоторые суда имеют гребной винт, приводимый в движение механизмами из трюма, что позволяет двигать судно даже в таком состоянии.
Обычно после трёх-четырёх дней плавания сосо'мар некоторые моряки умирали от жажды, жары или захлёбывались водой, а судно приходило в негодность;
тем не менее, согласно отчётам Коричного флота, этого срока хватало для того, чтобы преодолеть экваториальную зону с севера на юг.
Моряки ноа, пережившие экваториальные воды, пользовались почётом и уважением.
Они имели право на бесплатную чашу выпивки в любой таверне и на одну любую вещь не дороже золотого грана у любого торговца, что легко окупало затраты на плавание.}

\theterm{sotron-beard}
{Сотронская борода}
{Бодмод сотронских сели.
Для этого они аккуратно вживляют в кожу губ и подбородка части скальпов умерших родственников или друзей, либо собственных (с затылка).
Любопытно то, что бороды у сотронских сели могут носить и мужчины, и женщины.
Хака, которым случается попасть в Сотрон, всегда смеются над бородатыми женщинами, потому что в их племени бороды носят лишь мужчины, и борода является символом мужской силы.
Но для сели борода --- не символ и не прерогатива, для них это просто украшение, как серьги или ожерелье.}

\theterm{blued-steel-union} % Blued Steel Union
{Союз Воронёной Стали}
{---}

\theterm{match-tech}
{Спичечная технология}
{Высокотехнологичные устройства, которые можно сделать из широкого спектра подручных материалов и с минимальной инфраструктурной поддержкой.
Термин придумал Михаил Кохани, изобретатель <<спичечного самолёта>> и <<спичечной винтовки>>.}

\theterm{stabitanium}
{Стабитаниум (\theorigin{sl}{stabi[lis]}{стойкий} и \theorigin{sl}{[ti]tanium}{титан})}
{Сплав титана с памятью формы, имеющий в составе более 70 легирующих добавок (в том числе вольфрам и хром), стабилизированный волокнистыми кристаллами и полимерами кремния.
Количество марок стабитаниума в настоящее время более шестисот, характеристики марок различаются очень значительно, что делает его лидером среди материалов на богатых титаном планетах.}

\theterm{surroganium}
{Сурроганиум (\theorigin{sl}{surrog[atus]}{заменяющий} и \theorigin{sl}{[tit]anium}{титан})}
{Заменитель стабитаниума, изготовленный из других металлов и/или в пропорциях, зависящих от местной встречаемости металлов.}

\theterm{tama}
{Тама (\theorigin{sd}{tama}{бродяга})}
{На Преисподней: геолог, исследующий свежие вулканические отложения на предмет минералов и важных реактивов.
Также тама искали террасы --- места, которые защищены от выбросов и могут быть использованы для земледелия.}

\theterm{tamja}
{Тамга (\theorigin{саркорт}{tamha}{клеймо для скота})}
{Клановый символ.
Термин произошёл из одного из языков друзы Хербст, планета Тысяча Башен.}

\theterm{cockroach-war}
{Тараканья война}
{Мировая война на планете Ди.
Вскоре после заселения планеты произошло крупное столкновение между колониальными апидами и сапиентами-млекопитающими.
Люди, кани и планты были уничтожены почти поголовно.
Спасение для этих видов пришло через долгое время, как ни странно, от самих же апид --- несколько особей способных к размножению апид, называемых Уродами или Тараканами, начали вести подрывную деятельность в колониях и спасать себе подобных.
Тараканы заключили союз с млекопитающими, и во время ядерной, а затем и химической войны колониальные апиды были побеждены.
Победители вынуждены были уйти с сожжённой планеты на её необитаемый близнец --- планету Тси, давшую название новому союзу.}

\theterm{taari}
{Тари (\theorigin{tn}{taari}{благомыслящий})}
{У царрокх и хака: уважительное обращение к первому потомку мужского пола старейшины племени.
У сели: уважительное обращение к любому молодому человеку.}

\theterm{telln}
{Телльн (\theorigin{sd}{telln}{серёжка})}
{Время полного оборота оси Преисподней в результате прецессии, единица измерения времени (примерно 93 тысячи земных лет).}

\theterm{technku} % Technku
{Тенку}
{---}

\theterm{shadow}
{Тень}
{Слепок нервной системы сапиента, состоящий из материалов, отличных от его собственных клеток (электроника, генетически модифицированные клетки).
Тени создавали некоторые высокоразвитые цивилизации, они были способом продления жизни индивида.
В строгом смысле тенью может считаться также демон-урожденный сапиент.}

\theterm{terracota-wolf}
{Терракотовый волк}
{Образное название атомного оружия.
Происхождение связывают с легендой планеты Запах Воды о глинистом холме, который атомным взрывом превратило в похожую на волка терракотовую статую.
Другая версия, однако, гласит, что фразеологизм пошёл из языка герска планеты Тысяча Башен и его происхождение связано с табуированием слова <<ядерный гриб>> (nugvar-mikke) и заменой его на анаграмму <<глиняный волк>> (mugnar vikke).
Однако в настоящее время вторая версия не поддерживается, так как нет никаких данных о том, что на Тысяче Башен когда-либо использовалось атомное оружие.}

\theterm{grasshider}
{Травники}
{---}

\theterm{three-storey-temple} % Three-storey Temple
{Трёхэтажный Храм}
{Большой Храм Тхартхаахитра.}

\theterm{qi-people}
{Тси (\theorigin{чайнис}{Qi}{мощь, сила, расцвет})}
{Высокоразвитая цивилизация на двойной планете Тси-Ди системы Проксима Центавра, время существования --- 2 телльна --- абсолютный рекорд за всю историю.
Первые и единственные, кто сконструировал планетную защитную систему против хоргетов.
Гибель цивилизации наступила после ошибки в коде Машины, управляющего компьютера планеты.
Попытка отключить управляющий компьютер закончилась войной, в которой тси, как предполагалось ранее, были истреблены поголовно.
В настоящий момент система Тси-Ди --- единственный мир, в котором живут свободные Машины, и единственный мир, практически недоступный для хоргетов.
Технологические данные по Тси-Ди засекречены и обрабатываются закрытой службой --- отделом 104.}

\theterm{tchoemikchar}
{Тхэмикхар}
{Запрещённый вид охоты, при котором китовый ус оборачивался куском мяса или пальмовым маслом.
Животное съедало приманку, и китовый ус пронзал его внутренности, обрекая на мучительную смерть.
Охотники-сели, применявшие тхэмикхар, подвергались остракизму --- как и при использовании всех Десяти Проклятых Способов Охоты, включающих различные типы капканов и отравленных приманок.
Северные племена Ледяной Рыбы применяли тхэмикхар для охоты на хищников без каких-либо ограничений.}

\theterm{AID}
{УИД}
{Устройство, имитирующее деятельность.
Механизм, работа которого не преследует никаких целей, за исключением эстетической.
На Тси-Ди УИД были неотъемлемой частью стиля Механик --- архитектуры, дизайна и скульптуры.
Орденом Преисподней УИД используются при обфускации в важных оборонных узлах и программных блоках.
Программные аналоги УИД называются \textbf{хинду-лианами} или просто \textbf{лианами} --- они как бы <<оплетают>> код, скрывая его истинные очертания.}

\theterm{duck-mask}
{Утиная маска}
{Атрибут жреца сели, в основном врачей и тех, кто приносит жертвы.
Предназначена для защиты от патогенной микрофлоры, токсичных газов, аэрозолей и пыли.}

\theterm{phalanx}
{Фаланга (более точный перевод --- <<палец>> или <<указательный палец>>)}
{Колюще-режущее оружие ближнего боя, полуторапядевый клинок на изогнутой четырёхпядевой рукояти.
Фаланга была основным оружием воинов, так как позволяла держать на расстоянии ножи, была достаточно манёвренным против копья и стоила гораздо дешевле цельнометаллической сабли.}

\theterm{faciogramm}
{Фациограмма}
{Интерфейс, имитирующий лицо сапиента, предназначенный для отработки эмоционального аффекта при гуманизации хоргета.}

\theterm{fidens}
{Фидены (\theorigin{аркаб}{fidaienah}{воины})}
{Генетически модифицированные солдаты Пятого императора Плеяд, отличавшиеся огромной силой, ловкостью и жестокостью.
Были запрограммированы на безусловное подчинение Голосу императора.
Очень часто фидены запускались во враждебное общество, где создавали семьи и давали потомство.
Спустя несколько поколений потомки воинов <<активировались>> подосланным диверсантом, и в обществе устанавливалась подконтрольная императору военная диктатура.
Разработка фиден-подобных генетических паттернов (ФПГП), а также создание клеток или вирусных векторов с ФПГП на территории Ордена Преисподней расцениваются как сотрудничество с Картелем и караются немедленным уничтожением (Оборонительный кодекс Ада, 12F.2).}

\theterm{viola}
{Фиола (\theorigin{t-sl}{fiolla}{мензурка})}
{Сосуд из кварцевого стекла в металлической оплётке.
Аналог кошелька на Драконьей Пустоши, использовался для хранения и передачи ртути.
Фиола выдавалась всем мужчинам на совершеннолетие.
Потерять, украсть, отнять или разбить фиолу считалось величайшим бесчестьем и несчастливым знамением.
Даже разбойники оставляли своей жертве (нередко мёртвой) её фиолу, забирая лишь находившуюся в сосуде ртуть.
Этот предмет нашёл отражение в оборотах: <<продать фиолу>> --- нищенствовать, опуститься;
<<хранить дома чужие фиолы>> --- быть беспринципным, ценить деньги превыше чужого благосостояния;
<<слушать бульканье фиолы при ходьбе>> --- испытывать финансовые затруднения.}

\theterm{foe-f}
{Фоу-Ф (\theorigin{эрденшпрак}{Foe-F, Foen-F}{вторженцы, летящие строем кранихвинкель})}
{Фоу-Ф ведут свою историю с Друзы Бэйфан, что в Тропическом Поясе.
Местные бродячие циркачи --- фуцзы --- стали ворами во время очередного голода и составили ядро будущих наемников.
В течение своей недолгой истории Фоу-Ф разорили пять Друз и в конце концов обосновались на Гарда Викка, в услужении режиму Валдиса Хина.
Во время Осенней войны Фоу-Ф были разгромлены, и братство прекратило свое существование.
Также в их честь была названа редкая прионная болезнь (фоу-прион), которая была распространена исключительно среди некоторых наёмников Фоу-Ф, практиковавших каннибализм (так называемых Полуночных).
}

\theterm{chasetraasem}
{Хасетрасем (\theorigin{tn}{chasetraasem}{лицо, начерченное в воздухе})}
{В тси-подобных языках: фраза, представляющая собой биометрическое описание внешности сапиента, его мимики, особенности движений и голоса.
Зная хасетрасем, сели могли найти нужного человека даже в ночной толпе.
В школе несколько лет обучения посвящались искусству его составления.
Вероятно, что метод был разработан последними тси в рамках подготовки к одичанию.
Сами тси для этих целей использовали технологические средства.}

\theterm{hjar}
{Хйяр (\theorigin{kvenska}{hajarr}{камень})}
{Город в естественном разломе.
На стенах разлома устанавливаются навесные полки, мосты и лестницы, а сами жилища устраиваются в нишах.
Это слово с Тысячи Башен прочно вошло в языки многих планет, на которых когда-либо властвовали Орден или Картель.}

\theterm{jorget}
{Хоргет (\theorigin{sd}{horohito}{нелюдь})}
{}

\theterm{temple}
{Храм}
{Верхний и Нижний Этажи.}

\theterm{keeper}
{Хранитель}
{---}

\theterm{chrikchuatr}
{Хрикхватр}
{Дерево Перьев, обработанное горячей щёлочью и сжатое в раскалённых тисках.
Обладает большой прочностью и устойчивостью к воздействию химических веществ.
В некоторых областях Короны хрикхватром заменяют акульи зубы и обсидиан в оружии и инструментах.
Цельнодеревянные хрикхватровые стрелы очень ценятся охотниками --- они реже ломаются и меньше изнашиваются.}

\theterm{hook-n-glider}
{Хук и глайдер}
{Подарок на совершеннолетие практически во всех племенах Тысячи Башен.
Хуком называется набор скалолазных инструментов.
Глайдер --- подобие дельтаплана, позволяющее планировать с более высоких точек планеты на более низкие.
С этими двумя предметами дети Тысячи Башен учатся обращаться раньше, чем ходить и говорить.
Фразеологизм <<хук и глайдер>> на Тысяче Башен означает зрелость или подходящее время для какого-либо мероприятия.
В настоящее время хук и глайдер включены в тамгу клана Дорге и его псевдокланов;
выражение может быть использовано как намёк на представителя клана.}

\theterm{hook}
{Хук}
{Набор скалолазной экипировки на Тысяче Башен.
Состоит из: штоков, змайки, локхейдов, гарпуна, вайссака с куржаком, рапиры и других.}

\theterm{huneu} % Хьюнеу
{Хунев}
{Кибернетический организм, разработанный на Древней Земле.
Представляет собой человекообразное тело с искусственными рецепторами, управляемое культурой иммортализованных человеческих нейронов.
К 615 Эпохи Богов хунев получили равные права с людьми и кани.
Хунев отличаются высоким интеллектом и сроком жизни, значительно превышающим человеческий (есть информация о хунев, проживших 700 и даже 1000 лет).
В силу того, что у хунев отсутствует субстрат древних инстинктов, они лишены потребности в сексе, размножении и пище.
Тем не менее они могут испытывать эмоции, и некоторые из них имели потребность воспитывать детей --- как хунев, так и млекопитающих.
Многие хунев занимались наукой, искусством и сложными видами технологической деятельности --- например, микрохирургией и компьютерной безопасностью.
После катаклизма на Древней Земле изначальная технология изготовления хунев была безвозвратно потеряна.
Несмотря на то, что впоследствии были попытки создания хунев на других планетах, их результаты были значительно скромнее тех, которые достигли первые люди.}

\theterm{choesitr}
{Хэситр (\theorigin{tn}{choesitr}{напиток спокойствия})}
{Ритуальная чаша с водой, заменявшая погребение.
Если сели знал, что его не смогут похоронить с соблюдением всех ритуалов, то перед смертью он выпивал хэситр и умирал спокойно, зная, что найдёт пристанище лесных духов.
Также считалось приемлемым выливать хэситр в рот уже умершего человека.
Ранее такие чаши делались из благородных деревьев и украшались резьбой, впоследствии сели стали использовать как хэситр любую чашу, нанося на неё охранные знаки.
Использовались хэситры на войне, во время стихийных бедствий, а также людьми, совершающими <<последнюю беседу с собой>> (самоубийство).}

\theterm{tesarrokch}
{Царрокх}
{---}

\theterm{workshop}
{Цех}
{Сословие сели, объединяющее ремесленников.}

\theterm{windzither}
{Цитра Ветра (\theorigin{tn}{trotris}{звенеть (шелестеть) на ветру})}
{Десятиструнный музыкальный инструмент сели (4 гладких струны + 6 витых).
Название получил по имени его предположительного изобретателя, Ветер-Дующий-Ниоткуда (легендарный бард, любовница купца Чхаласа).
Для игры использовались специальные перчатки с дополнительными <<пальцами>> или встроенная в гриф каретка.
В деке имелся механический смычок для гладких струн, приводимый в движение маховиком и ногой барда.
К моменту войны с Безумным на Короне осталось всего два мастера, которые умели делать эти инструменты, оба погибли во время войны.
Но образцы цитры Ветра попали к культурологам Ордена Преисподней, которые смогли воссоздать технологию изготовления.}

\theterm{beads-of-sat}
{Чётки Сата}
{Верёвочные карты дорог, бывшие в большом ходу у торговцев и путешественников обитаемой Короны и Кита.
Представляли из себя цветные верёвки, связанные между собой, с узелками или бусинами;
количество бусин равнялось количеству верстовых столбов.
Торговцы в дороге постоянно держали их в руках и пальцами отсчитывали пройденные кхене.}

\theterm{gods-age}
{Эпоха богов}
{}

\theterm{daemons-age}
{Эпоха демонов} % sd: asogeite
{}

\theterm{jasper}
{Янтарь}
{Камень, встречающийся только на Тра-Ренкхале, Хемане-2 (Хароне) и, согласно некоторым данным, на Древней Земле.
По легенде, это слёзы дерева акхкатрас, падающие в Ху'тресоааса и выносимые в Кипящее море.
В настоящее время установлено, что основой янтаря является смола не акхкатрас, а другого растения --- голосеменного эпифита Пятикрыльник плакучий (Бенедикта).
Кислые геотермальные воды плавят смолу, спекают её с продуктами придонных моллюсков, диатомей и кольчатых червей, а затем встречное течение Могильного пролива разносит янтарь по всему побережью вплоть до Молчащих лесов.
Особенно ценным считается янтарь с вплавленными в него жемчужинами и золотыми самородками.}

\theterm{yao}
{Яо (Y)}
{Единица измерения масс-энергии ПКВ.}

\theterm{rubbish-fair}
{Ярмарка хлама}
{Праздник в первый год Церемонии.
Люди вытаскивают и продают за бесценок весь хлам, который скопился в жилищах, часто находят старые клады.
В переносном смысле --- отсутствие выбора при видимом изобилии.}

\chapter{Прочее}

\section{Социум сели}

\begin{itemize}
\item Храм (the Temple)
\begin{itemize}
\item Нижний этаж (the Downstairs)
\item Верхний этаж (the Upstairs)
\end{itemize}
\item Двор (the House)
\item Цех (the Workshop)
\item Сад (the Garden)
\end{itemize}


\section{Игры}

\subsection{Метритхис}

Играют 4 игрока, жребием перед игрой распределяются роли.

\begin{itemize}
\item Зелёный --- Король
\item Чёрный --- Мятежник
\item Красный --- Сосед
\item Жёлтый --- Фатум.
\end{itemize}

Подготавливает игру Фатум.
Он выкладывает определённым образом квадратики-поля:

\begin{itemize}
\item Синие --- река/море (1 ход на переправу, нельзя атаковать с другой стороны, пенальти к защите и нападению)
\item Серые --- горы (3 хода на переправу, преимущество в обороне и обстреле)
\item Красные --- пустыня (пенальти ко всем видам деятельности)
\item Зелёные --- лес (преимущество в обороне, скрытности, пенальти к нападению и обстрелу)
\item Жёлтые --- степь (пенальти к скрытности и обороне, преимущество в нападении и обстреле)
\item Чёрные --- болото (2 хода на переправу, преимущество к скрытности и обстрелу, пенальти к нападению и обороне)
\end{itemize}

Фигурки взаимодействуют двумя способами --- драка и беседа.
При драке определяются победитель и побеждённый.
Побеждённый отправляется в коробку, а победитель иногда может сменить класс.
При беседе оба меняют класс, а иногда ещё и цвет.

Начало игры распределяется жребием между Королём и Мятежником.
Последним в игру вступает Сосед, время его вступления выбирает Фатум.

Всего в игре 4 поля --- собственно игровое и 3 дипломатических.
Дипломатические могут видеть только договаривающиеся правители и Фатум.

Комбинации на поле дипломатии:

\begin{itemize}
\item Обман --- проигравший пропускает 5 ходов против выигравшего.
\item Заговор --- Фатум выбрасывает кихотр на смерть проигравшего.
\item Паритет --- 5 ходов правители не ведут друг против друга боевые действия.
\item Мир --- правители до конца игры не ведут друг против друга боевые действия, их стол дипломатии с третьим правителем становятся общим.
\end{itemize}

Если два правителя заключили мир, единственный способ Фатума выиграть --- выбросить успешный Заговор у третьего или Чуму на одного из союзников.
Если все три правителя пытаются построить мир, Фатум пытается их убить.
Поэтому в задачи правителей входит ещё и ввести Фатум в заблуждение.

В распоряжении правителей --- войска и камни дипломатии (чёрные и белые).
В распоряжении Фатума --- кихотр и камни Благ и Несчастий:

\begin{itemize}
\item Верная женщина --- отводит от правителя Заговор.
\item Чума --- убивает войска и с некоторой вероятностью может убить правителя.
\item Удача --- позволяет правителю взглянуть на дипломатический стол противников, выиграть в явно проигрышной стычке войск или даёт фору в 5 камней на любом столе
дипломатии.
\item Потеря друга --- пропуск двух ходов.
\item Безумие --- ход на поле боя или столе дипломатии вместо правителя делает Фатум.
\end{itemize}

Игра заканчивается 4 способами:

\begin{enumerate}
\item 2 правителя умирают, оставшийся в живых правитель и Фатум выигрывают.
\item 1 умирает, 2 других договариваются, Фатум проигрывает.
\item 3 правителя договариваются, Фатум проигрывает.
\item Все правители гибнут, Фатум выигрывает.
\end{enumerate}

Самая сложная задача всегда у Фатума, поэтому очень часто его роль без жребия отдают самому опытному игроку.

\subsection{Пьянка}

Шуточная игра в кости от двух до четырёх человек.
Игра идёт обычно на желания, лакомства, секс или удары, иногда на всё сразу.

Правила желаний: они не должны причинять человеку вреда.
То же самое с ударами --- чаще всего игра идёт на пощёчины, шлепки и несильные броски.
Проигравший может поменять секс или желание на оговорённое число ударов, лакомства и алкоголь можно передавать другим игрокам.

\chapter{История и современый правовой статус нейтральных демонов}

\section{Пассивный нейтралитет, или Непринадлежность}

Самый очевидный и древний тип нейтралитета.
Речь идёт о дёмонах, не признающих какую-либо власть и не подчиняющихся законам фракций.
В настоящее время количество нейтралов оценивается примерно в десять --- тридцать тысяч демонов.
К ним относятся как истинные, никогда не принадлежавшие к фракциям нейтралы, так и дезертиры обеих фракций.
\ml{$0$}
{Они находятся на нелегальном положении и чаще всего уничтожаются при обнаружении.}
{They have an illegal status and mostly are killed on sight.}

\section{Активный нейтралитет, или Невовлечённость}

С самого момента создания фракций было открыто явление, получившее название <<активный нейтралитет>>.
Активный нейтралитетом признаётся деятельность тех демонов, которые пользуются всеми правами фракции, но тем или иным способом избегают обязанности участвовать в борьбе с внешними врагами --- либо подстрекают к подобной деятельности прочих демонов.

\subsection{Научный коллаборационизм}

Как следует из названия, чаще всего этот вид характерен для учёных, многие из которых, несмотря на запрет, контактируют с учёными враждебной фракции.
Контрразведка обеих фракций многократно пыталась прервать эту связь;
в настоящее время о контактах между учёными рекомендовано сообщать заранее, а также предоставлять список передаваемой и получаемой информации.
Тем не менее легализованы, по некоторым подсчётам, менее 0,1\% межфракционных связей.

Сложность их обнаружения обуславливается тем, что у каждого исследовательского отдела выработались свои собственные внутренние правила обмена информацией, и эти каналы связи законспирированы не хуже, чем таковые в военных организациях.
Правила обычно предусматривают фильтрацию любых данных, имеющих военное значение, а также легализацию данных экспериментов, проведённых в другой фракции.
Для легализации данных между отделами существует целая сеть --- также законспирированная;
данные легализует лаборатория с соответствующей сферой деятельности и юрисдикцией.

Пресс-центр отдела 100 признаёт, что грубое вмешательство в межфракционные научные связи, равно как и в систему легализации данных, в настоящий момент не представляется возможным и может вызвать полный коллапс научной деятельности, так как некоторые лаборатории у Ада и Картеля по сути являются \emph{общими} --- легальные филиалы в каждой фракции конспиративно связаны друг с другом.

\subsection{Обход закона}

Также к активному нейтралитету относится косвенный (юридически обоснованный) отказ участвовать в военных действиях и сотрудничать с военными.
Возможностей для такого отказа в настоящее время очень мало, но многие демоны умело используют имеющиеся уязвимости в законах и манипуляцию общественным мнением.
Такой активный нейтралитет стоит на самой грани закона и в некоторых случаях может быть приравнен к саботажу и дезертирству.

\subsection{Декларативный пацифизм}

Одним из видов нейтралитета является декларация пацифистских взглядов.
Она возможна для любых демонов, является одним из самых простых и в то же время опасных видов нейтралитета.
Очень часты случаи открытого пацифизма в военных силах Ада;
солдаты и офицерские чины, несмотря на добросовестное выполнение долга, распространяют свои убеждения посредством личных бесед и выступлений перед подчинёнными.
Некоторые даже выработали для декларации пацифистских взглядов свод правил, которые позволяют доносить до слушателей суть, оставаясь при этом в рамках закона --- например, с помощью медиавирусов.

С этим явлением активно боролись, находя причины для отказа в повышении и даже устраняя пацифистов физически --- например, отправляя их в опасные зоны.
Тем не менее, как показала практика, наибольшее воздействие речи пацифистов имеют именно в опасных зонах;
оттуда, где погиб один пацифист-новобранец, вскоре возвращаются десять пацифистов-ветеранов, которых в разы сложнее заставить замолчать.
В настоящее время демонов с пацифистскими взглядами стараются тем или иным способом устранить из армии;
иногда с ними заключают негласные договоры --- хорошая должность в обмен на молчание.

\section{Попытки легализации нейтралитета}

\subsection{Демиург --- Метрополия}

Договор <<Демиург --- Метрополия>> является одним из способов легализовать нейтрального демона.
Боги являются особым субъектом адского права, потому что они являются частью планеты --- иногда неотъемлемой.
В случае захвата планеты врагом демиург не сможет покинуть её, подобно демонам;
у демиурга в руках зачастую находятся внушительные ресурсы и рычаги давления.
Кроме того, известны случаи, когда демиург напрямую подготавливал планету к захвату или вёл подрывную деятельность.

Именно поэтому Орден Преисподней и Красный Картель заключают с демиургами особые договоры.
Очень часто демиург мог заключить схожие договоры одновременно с обеими воюющими сторонами, и в этих договорах чётко проводилась грань между вмешательством и невмешательством демиурга в военный конфликт.
Но, так как это создавало определённые проблемы для фракций, чаще всего демиурги уничтожались под благовидным предлогом после полного изучения планеты.

\subsection{Иммунитет учёного}

Идея, предложенная Ликаном Безруким.
Он предлагал наделить особым статусом учёных, исследователей космоса и планетарных инженеров --- их ни в коем случае нельзя было трогать и как-то препятствовать их деятельности.
Под определение планетарных инженеров, разумеется, подпадали все без исключения демиурги, а также значительная часть специалистов по мелиорации.
Обладающий иммунитетом учёного демон обязан был соблюдать определённые правила (достаточно большое количество), чтобы не оказаться втянутым в конфликт фракций.
Статус был невосстановимым --- нарушивший правила демон лишался его навсегда, но, тем не менее, имел возможность продолжать свою обычную деятельность согласно законам фракции.
А вот любые попытки уничтожить, завербовать или шантажировать носителя статуса должны были караться уничтожением.

Идея была отвергнута Советом Капитула.
Вернее, она даже не дошла до стадии рассмотрения --- несколько раз её возвращали на доработку, а в последний раз её просто заморозили на неопределённый срок.
Впрочем, вскоре многие об этом пожалели, так как именно заморозка проекта Ликана Безрукого послужила началом расцвета научного коллаборационизма и технологий обхода закона (см. выше).

\part{Временное хранилище}

\chapter{Эпиграфы}

\epigraph{
\ml{$0$}
{Лучше пусть мой враг будет счастлив, чем мой друг будет страдать.}
{I'd rather see my foe happy than my friend suffering.}
}{Пословица ноа}

\epigraph{
\ml{$0$}
{Чем дальше путь, тем больше одиночество.}
{The longer the way, the lonelier the way.}
}{Пословица сели}

\epigraph{Я цитра с тысячей струн, я пою под руками твоими...}
{Эрхэ Колокольчик}

\epigraph{
Конец игрушки печален --- её ломают, выбрасывают или забывают.
Дышащий отличается от игрушки лишь тем, что может выбрать свой конец из этих трёх.
}{
Пословица сели
}

\epigraph
{Проблема плюрализма мнений не в том, что одни лгут, а другие говорят правду;
проблема в том, что все без исключения лгут, чтобы сделать убедительнее то, что они считают правдой.}
{Ликан Безрукий}

\epigraph{
\ml{$0$}
{То, что невидимо, всегда дискриминируемо.}
{Invisible is always discriminated.}
}{
\ml{$0$}
{Кэтрин Шаулман, Эпоха Последней Войны}
{Catherine Shoulmann}}

\epigraph
{Качество философского течения обратно пропорционально количеству его радикальных и деструктивных ответвлений.}
{Ликан Безрукий}

\epigraph
{Элемент несет в себе образ системы.
В каждом диктаторе скрывается повстанец.
В каждом повстанце скрывается диктатор.
Те, кто об этом забывает, обречены на повторение ошибок прошлого.}
{Лусафейру Лёгкая Ладонь}

\epigraph
{Гораздо лучше подкупить человека, чем убить его, да и быть подкупленным куда лучше, чем убитым.}
{Уинстон Черчилль.
Эпоха Последней Войны}

\epigraph
{В обществе всегда будут больные, зависимые, бездомные и преступники.
Задача управленца --- снизить скорость их появления, повысить скорость их интеграции в общество и сделать их жизнь по возможности комфортной.}
{Вениамин Лист, канцлер Германии}

\epigraph
{Мир держится на тех, кто находится в центре гауссианы, а движется вперёд благодаря её краям.}
{Лусафейру Лёгкая Ладонь}

\epigraph
{Господь сотворил нас несовершенными с той же целью, с которой Он сотворил зерно, а не хлеб, и виноград, а не вино: Он хотел разделить радость акта творения со Своими детьми.}
{Хакем-Аят, 14:45.
<<Притча о послечеловеке>>.}

\epigraph{
\ml{$0$}
{Оступившийся не выбирает, куда упасть.}
{A tripped one can't choose the place to fall.}
}{Пословица сели}

\epigraph
{Мир несправедлив, и это является первопричиной страдания живых существ.
Единственный источник справедливости --- сапиент, обладающий свободой выбора.}
{Аксиома Несправедливости. Эволюцион}

\epigraph
{Самая распространенная ошибка правителей --- считать, что неуправляемость и опасность суть одно и то же.}
{Лусафейру Лёгкая Рука}

\epigraph{
\ml{$0$}
{Тот, кто считает идею возможностью, уже её использует.}
{The one who treats an idea as an opportunity, already uses it.}
\ml{$0$}
{Тот, кто считает идею болезнью, уже ею заражён.}
{The one who treats an idea as a disease, is already infected.}
}{Лусафейру Лёгкая Рука}

\epigraph
{Когда предлагают выбирать между победой и поражением, на самом деле предлагают не один выбор, а три.
Первый --- поверить говорящему.
Второй --- вступить в игру.
Третий --- победить или проиграть.
Ни один из этих выборов не является обязательным, но абсолютное большинство первые два делает на бессознательном уровне.}
{Лусафейру Лёгкая Рука}

\epigraph
{Если тебе невыносимо тяжело --- посмотри назад.
Возможно, ты тянешь за собой целую эпоху.}
{Мартин Охсенкнехт}

\epigraph
{Языком ненависти всегда ведётся разговор с позиции силы, мнимой или реальной.
Человек, который разговаривает языком ненависти с одним человеком или группой, с высокой долей вероятности будет говорить на нём с любыми людьми, над которыми он имеет мнимую или реальную власть.}
{Мариам Кивихеулу}

\epigraph
{Солдаты исправляют ошибки дипломатов.}
{Джозефус Дэниэлс, Древняя Земля.}

\epigraph
{При достаточном количестве глаз все ошибки лежат на поверхности.}
{Закон Торвальдса"--~Реймонда. Древняя Земля}

\epigraph
{Плохой управленец пытается упорядочить хаос общества.
Хороший управленец пытается нащупать точку равновесия хаоса и сдвинуть её в нужную сторону.
Хороший управленец никогда не будет по достоинству оценён любителями порядка, ибо он принимает хаос как должное.}
{Лусафейру Лёгкая Рука}

\epigraph
{In hostem omnia licita.}
{Латинское крылатое выражение}

\epigraph
{Жалкое зрелище --- волк, теряющий зубы от старости, но ещё пытающийся их скалить.}
{Клаудиу Дентосиу}

\epigraph
{Многие профессии стремительно молодеют.
Молодёжи уже не нужно тратить время на поиск ответа на вопрос <<Кто я?>>, на залечивание ран молодости, на доказывание своего права на существование.
Молодые люди разной психологической и гендерной идентичности, обладающие разными проявлениями сексуальности без особых проблем встраиваются в общество --- а значит, у них остаётся огромное количество времени и сил на реализацию в профессиональной сфере.}
{Мариам Кивихеулу}

\epigraph
{Многие мечтают быть первопроходцами, многие мечтают отличаться от других.
Но в этом нет ничего хорошего.
Ты совершаешь все ошибки, которые только можно совершить, живёшь всю свою жизнь с последствиями ошибок, и даже признание --- если оно придёт --- выглядит весьма незначительной компенсацией.}
{Мариам Кивихеулу}

\epigraph
{Война может только разрушать.
Даже для производства оружия нужны мирные времена.}
{Мариам Кивихеулу}

\epigraph
{Эгоизм делает человека честным.
Честность является предметом нравственности.}
{Постулат Омельчанко.
Эволюцион}

\epigraph
{Слову в наше время придаётся чересчур большое значение.
Но давайте будем честны --- дело не в словах.
Слово не убивает, призывы к войне не разрушают города,  и не оскорбления доводят человека до самоубийства.
Любое зло способно существовать в молчании, как и любое добро.
Но заставлять молчать тех, кому есть что сказать --- это зло в чистом виде.}
{Мариам Кивихеулу}

\epigraph
{Расцвет цивилизации начинается тогда, когда большинство сапиентов становятся чересчур ленивыми для убийства друг друга.}
{Длинный-Мокрый-Хвост}

\epigraph
{Правители новой эпохи сидят в тюрьмах предыдущей.
Поэтому хорошо думайте, кого отправляете за решётку.}
{Клаудиу Дентосиу}

\epigraph
{В идеальном обществе каждый индивид обладает властью, прямо пропорциональной его потенциалу --- военному, трудовому и культурному.}
{Первый постулат физики социума. Эволюцион}

\epigraph
{Любое распределение власти, отличное от описанного, не может существовать бесконечно долго.
Скорость восстановления идеального распределения власти прямо пропорциональна уровню технологического развития.}
{Недоказуемое следствие из первого постулата. Эволюцион}

\epigraph
{Ни одному военному не приходится терпеть столько лишений, ни одному военному не приходится переживать столько сражений, сколько выпадает на долю самого заурядного пацифиста.}
{Ликан Безрукий}

\epigraph
{Развитие цивилизации неизбежно ведёт к ослаблению внутривидовой конкуренции.
Научно-технический и социальный прогресс увеличивают количество социальных ниш в геометрической прогрессии.}
{Яо Вэй, автор шкалы развития сапиентного общества.
Впоследствии фраза стала Восьмым постулатом Эволюциона.}

\epigraph{Если в технологически развитом обществе имеет место быть жёсткая конкуренция --- значит, она поддерживается искусственно.}
{Первое следствие из Восьмого постулата Эволюциона}

\epigraph
{В одиночестве легче делать правильный выбор, в компании легче переживать последствия неправильного.}
{Длинный-Мокрый-Хвост}

\epigraph
{Маяки не бегают по всему острову, выискивая, какую бы лодку спасти, они просто стоят и светят.}
{Энн Ламотт}

\epigraph{
\ml{$0$}
{Большая часть методов психологической манипуляции строится на том, чтобы представить случайное закономерным, а закономерное --- случайностью.}
{Most of methods of psychological manipulation are based on the idea to make random look logical, and or make logical look random.}
}{Лусафейру Лёгкая Рука}

\epigraph
{Как нелепо выглядят со стороны по-настоящему великие дела!}
{Изречение, якобы сказанное Анатолиу Тиу, когда он впервые увидел лиманское ополчение}

\epigraph
{<<Счастье не в деньгах>>?
Какая глупость.
Так говорят только те, кто их никогда не имел.
Или те, кто не имел ничего, кроме денег.}
{Клаудиу Дентосиу}

\epigraph
{Если стражи порядка закрывают лица и не называют имён --- значит, они недобросовестно выполняют свою работу.
Тем, кто действительно делает своё дело хорошо --- защищает граждан и ловит преступников --- незачем бояться.
Безопасность стражей порядка --- исключительно результат их собственной работы.}
{Анатолиу Тиу}

\epigraph
{Харизма --- опаснейшее оружие.
Пожалуй, это единственная вещь, которая по силе воздействия может соперничать с истиной --- и именно поэтому часто ей противопоставляется.}
{Мартин Охсенкнехт}

\epigraph{
\ml{$0$}
{Один из главных тормозов развития --- неспособность различать комфорт и условия, в которых можно существовать.}
{One of main hindrance to development is inability to distinguish between comfort and possibility to live.}
}{Длинный-Мокрый-Хвост}

\epigraph
{Идеи новой парадигмы нужно подвергать тщательному отбору.
Любая идея, которая является зеркальным отражением идеи старой системы, будет работать на старую систему.
Особенно это относится к социально значимым идеям, касающимся насилия и угнетения --- ни в коем случае нельзя превращать угнетателя в жертву угнетения.
Наоборот, нужно показать, что новая парадигма будет выгодна для большинства членов общества --- в том числе и для тех, кто принял правила старой системы и умеет играть по её правилам.}
{Мариам Кивихеулу}

\epigraph
{Чем больше индивидов знают о проблеме и имеют возможность её обсудить, тем выше вероятность того, что проблема будет решена наилучшим образом.}
{Недоказуемый постулат Элект \#3}

\epigraph
{Хороший правитель должен дать подданным то, что они хотят.
Свободу, войну, слепое подчинение, справедливость --- всё, что угодно.
Подданные должны получить желаемое --- это сделает их целеустремлённее и увереннее.
Подданные должны получить желаемое кровью и потом --- это сделает их практичнее и экономнее.
Подданные должны прочувствовать все последствия своего выбора --- это сделает их осторожнее.
Когда подданные обретут все эти качества, правитель им больше не потребуется.}
{Лусафейру Лёгкая Рука}

\epigraph{
\ml{$0$}
{Убийство в целях самозащиты является самоубийством нападающего.}
{Justifiable homicide is assailant's suicide.}
}{Законы ноа}

(К пленению Чханэ?)

\epigraph
{Посмотрите на оркестр в Симфоническом театре Гелиополя.
Каждый музыкант занят своим делом.
Обитаемая Вселенная --- такой же оркестр, в котором каждый исполняет свою собственную партию, а дирижирует коллективное бессознательное.
Что же до войн и прочего непотребства... ведь оркестр не обязан играть только детские песенки, верно?}
{Ликан Безрукий.
Речь на презентации языка Эй}

\epigraph
{Ты всегда в ответе за то, чему не пытался помешать.}
{Жан-Поль Сартр}

\epigraph
{Все самые неприятные вещи в истории делали существа без чувства юмора.
Чем больше смеётся один --- тем меньше плачут остальные.}
{Длинный-Мокрый-Хвост}

\epigraph
{Даже в рамках бесчеловечной системы можно быть человеком.
Всё зависит лишь от выбора конкретного индивида --- быть человеком или зверем.}
{Мариам Кивихеулу}

\epigraph
{Правонарушения должны наказываться, но следует помнить, что истинный виновник --- господствующая система.
Как бы высоко или низко ни находились люди, принявшие её правила, они не более чем жертвы и орудие системы.
Это умозаключение не стоит ровным счётом ничего, пока вам не потребуются союзники.}
{Мариам Кивихеулу}

\epigraph
{Если вы придерживаетесь нейтральной позиции в ситуации несправедливости, вы выбираете сторону угнетателя.}
{Десмонд Туту. Древняя Земля}

\epigraph
{Пережитый позор ныне попран ногами танцоров.}
{Сигурдур А. Магнуссон. <<Греция --- 1974>>}

\epigraph
{Я видел, как дом превращается в тюрьму, а тюрьма превращается в дом.
Но едва я открыл рот, чтобы об этом сказать, как моё горло сжала петля.}
{Неизвестный повстанец Лимана.
Последнее слово перед казнью}

\epigraph{У природы не было в планах делать живое существо счастливым.
Счастье --- это награда за победу.
Договорившсь между собой, сапиенты одержали победу сообща, и счастье стало нормой.}
{Длинный-Мокрый-Хвост}

\epigraph{Под глухими высокими стенами спрячутся много врагов.}
{Пословица ноа}

\epigraph
{Жертвой можно ослабить, но нельзя победить.}
{Пословица Преисподней}

\epigraph
{Если тебе есть, что сказать, поднимись, чтобы тебя увидели.}
{Пословица хака}
% Индейская пословица

\epigraph
{У всех животных размножение --- это специально организованный процесс, выбивающийся из обычного ритма жизни.
И только у сапиентных видов дети --- это побочный продукт социального взаимодействия.}
{Длинный-Мокрый-Хвост}

\epigraph
{Моё от меня не спрячется.
Чужое моим не будет.}
{Присказка сели.
Предположительно отрывок из потерянного сборника Эрхэ Колокольчик}

\epigraph
{Ложь идёт рука об руку с унижением.}
{Пословица ноа}

\epigraph
{Если мы не понимаем, что происходит у нас внутри, это превращается во внешние события, которые кажутся нам судьбой.}
{Карл Густав Юнг}

\epigraph
{Существуют два подхода к современным технологиям, я называю их <<западным>> и <<восточным>>.
Запад использует технологии для повышения удобства, а когда что-то идёт не по плану --- меняет парадигму.
Восток до последнего цепляется за парадигму, а технологии использует для того, чтобы парадигма работала с заявленной эффективностью.
В случае неудачи западный человек говорит <<Мы делаем что-то не так>>, восточный --- <<Мы использовали недостаточно мощные средства>>.}
{Сергей Хистиаков}

\epigraph{Изучение ругательств народов --- хороший путь к постижению их святынь.}
{Грегори Ландау}

\epigraph
{Рокеры учили толпу действовать сообща без муштры и построений, свойственных военным.
Рокеры взращивали свободную толпу, действующую в своих интересах и лишённую рычагов управления.
Я бы сказал без преувеличений, что гражданское общество XXI века ковалось на рок-концертах.
Именно поэтому неформальные музыканты и их поклонники рассматривались авторитарным режимом как полноценная политическая угроза.}
{Мартин Охсенкнехт}

\epigraph
{Великие реки берут начало в болотах и входят в силу, питаясь чистейшими горными ручьями.
Те же реки, что спустились с гор в поисках болотной воды, остаются в болоте насовсем.}
{Пословица сели}

\epigraph
{Публичный человек всегда в той или иной степени готов поступиться убеждениями.
Наличие аудитории для него гораздо важнее истины.
Чем шире аудитория --- тем чаще этой истиной приходится жертвовать.}
{Мариам Кивихеулу}

\epigraph
{Даже охотник не станет убивать птицу, просящую у него защиты.}
{Пословица Преисподней}
% Японская пословица

\epigraph{
\ml{$0$}
{Религия как ничто другое умеет убеждать людей, что неумение и незнание есть добродетели.}
{Religion like nothing else can convince people that there's a virtue in ignorance and inexperience.}
}{
Мартин Охсенкнехт
}

\epigraph
{...И был Талим свидетель тому, как слова Его прорастали в толпе, и крепли, и цвели, и давали плоды, и люди вкушали эти плоды, и видел Он, как насыщали люди свой голод плодами Его слов...}
{Хакем-Аят, 20:2}

\epigraph
{Идёт дурак, и от смеха его рушатся святыни.}
{Хакем-Аят. Апокриф Искандера (Свиток Ликана), 0:28}

\epigraph
{Нгвсо --- единственные на сегодняшний день Ветви Земли со 120 баллами по Яо, извлекающие кислород только из воды.
До Тра-Ренкхаля считалось, что существование сапиентов Ветвей Земли с исключительно жаберным дыханием невозможно в принципе.
Безымянный разрушил этот стереотип.
Возможно, выбор моллюсков в качестве базы и был в корне ошибочен, но демиург, если можно так выразиться, выжал из океана весь возможный интеллект.}
{Корхес Соловьиный Язык.
Обзор <<Биоразнообразие колоний аксиального направления>>}

\epigraph
{Выиграть войну так же невозможно, как невозможно выиграть землетрясение.}
{Джаннетт Ранкин, общественный деятель.
Эпоха Последней Войны}

\epigraph
{Смех --- самое страшное оружие.
Смехом можно убить всё --- даже убийство.}
{Эуджин Замятин.
Эпоха Последней Войны}

\epigraph
{Чем выше интеллект у существа, тем дольше длится его детство.
Видимо, идеальное интеллектуальное существо должно оставаться ребёнком на всю жизнь.}
{Длинный-Мокрый-Хвост.
Афоризм}

\epigraph{
\ml{$0$}
{В классическом цатроне есть слово <<тхвал'кикхвал>>.}
{There is a word ``tchu\={a}l'k\'{\i}kchu\r{a}l'' in Classical Te's\'{a}tr\v{o}n.}
\ml{$0$}
{Оно приблизительно переводится как <<что испытывает тот, кто не хочет плавать в дерьме, но не может объяснить почему>>.}
{It can roughly be translated as ``how it feels when you don't want to swim in shit and can't explain why''.}
}{
Аркадиу Люпино.
<<Генетический анализ старых языков Тра-Ренкхаля>>
}

\epigraph
{Когда я вижу в саду пробитую тропу, я говорю садовнику: делай здесь дорогу.}
{Александер I, король государства Руссия.
Эпоха Господина}

\epigraph
{Моё тело --- мой храм, но я не прихожанин, а архитектор.}
{Мартин Охсенкнехт}

\epigraph
{Тиран должен быть изобличён и опозорен при жизни, иначе он станет знаменем для следующего.}
{Анатолиу Тиу}

\epigraph
{Однажды будут оценены и истина, и ложь.
Истина восторжествует.
Честное заблуждение помянут добрым словом.
Ложь забудут, за исключением самой искусной, ибо искусство лжи по сути своей так же привержено истине, как и прочие искусства.}
{Ликан Безрукий}

\epigraph
{Тот, кто бежал за мечтой и красил лицо в её цвета, никогда не ударит ребёнка и не посадит зверя на цепь.}
{Пословица тенку}

\epigraph
{Когда персидского шаха, гостившего в Англии, пригласили на скачки, он отказался.
<<Я и так знаю, что одна лошадь бегает быстрее другой>>, --- сказал он.}
{Эльза Дункан.
<<Теория игр в Эпоху Богов>>}

\epigraph
{...И хотя многие небезосновательно считали Тациана [Освободителя] тщеславным властолюбцем, для своих подданых он был истинным королём.
Когда Тациана казнили, вместе с ним повесили его соратников и друзей, также многих других безвинно --- лишь за слово или дерзкий взгляд.
Очень удобно казнить безвинных, выставляя их мятежниками --- ведь о них составляют впечатление о всём мятеже.
Эти люди плакали и молили о пощаде, когда церковники обещали им Inferno\FM\ и Damnacion Memoria\FM.
И Тациан, доселе хранивший молчание, вдруг крикнул: <<Успокойтесь.
Если после смерти что-то есть, мы все пойдём туда вместе.
Кем бы вы ни были, я вас не брошу.
Я буду биться за вас даже с дьяволом, если придётся>>.
Плач прекратился, и до самой смерти приговорённые не издали ни звука, несмотря на все усилия палачей.}
{Анатолий Тий.
<<Лиманские записки>>}
\FA{
Подземный мир (t-sl.).
}
\FA{
Осквернение памяти (t-sl.).
}

\epigraph
{Культура, как река, часто образует водовороты, в которых на первый взгляд действуют совершенно другие законы логики.
Иногда водоворот кажется настолько большим, что вызывает опасения за судьбу культуры в целом.
Эти опасения беспочвенны --- река всегда течёт в одном направлении.}
{Кельса Пушистая}

\epigraph
{Есть легенда про принца Валерия.
Однажды он участвовал в конных скачках и занял восьмое место, уступив пятерым лордам и двоим пришлым торговцам из кочевых племён.
Судья настолько растерялся, что долго не мог огласить список победителей, ведь среди зрителей был и сам король --- грозный Лауренций Фульминикула.
Тогда Валерий взял у судьи список, назвал имена победителей и лично вручил им награды.
Валерий показал людям а тот день: истинный судья всегда вне игры, даже если он шёл к финишу вместе со всеми.
Правление Валерия знаменует начало расцвета, вошедшего в историю как Старшие Валериды.}
{Клаудиу Дентосиу.
Предисловие к <<Воспоминаниям о великом тёзке>>}

\epigraph
{Правда и ложь не имеют ничего общего с законом.}
{Пословица Тысячи Башен}

\epigraph
{При изучении языка вас может настичь странное чувство.
Это сродни влюблённости --- вы можете рассматривать прочие языки, не иначе как сравнивая их с объектом вашей страсти.
Моей страстью стал энглис, язык Древней Земли.
Его потомки за миллион лет разошлись по Вселенной, изменившись до неузнаваемости, но каждый раз, когда я нахожу знакомые звуки, корни и грамматические конструкции, моё сердце пропускает один удар.}
{Кельса Пушистая}

\epigraph{Те рыбы, которые впервые в истории вылезли на сушу, очень плохо плавали.}
{Длинный-Мокрый-Хвост}

\epigraph
{Где нет людского, не место мне.}
{Строка из песни Ликхмаса, героя <<Легенды об обретении>>}

\epigraph
{Негуманная идеология --- почти всегда женоненавистническая.
И лишь иногда --- мужененавистническая.}
{Мариам Кивихеулу}

\epigraph
{Молчание --- самая удобная форма лжи.}
{Даниил Гранин, Эпоха Последней Войны}

\epigraph
{Главное для мыслящего существа --- уметь видеть безобразное.
Сапиент может быть сколь угодно восприимчив к тонкому и прекрасному, но, не замечая безобразное, он превратит свою жизнь в кошмар.}
{Длинный-Мокрый-Хвост}

\epigraph
{Чувство вины --- отличный способ манипуляции.
Иногда лучше поступиться принципами, нежели позволить себе быть виноватым.}
{Клаудиу Дентосиу}

\epigraph{
\ml{$0$}
{Разница между смирением раба и смирением того, кто боролся --- в той вмятине, которая осталась на броне противника.}
{Between humility of slave and humility of the one who fought lies a dent in the enemy's armour.}
}{Пословица Преисподней}

\epigraph
{Компромисс --- это искусство разделить пирог так, чтобы каждый думал, что ему достался самый большой кусок.}
{Людвиг Эрхард, Эпоха Последней Войны}

\epigraph{
\ml{$0$}
{Начинай, если хочешь.}
{Begin if you want.}
\ml{$0$}
{Продолжай, если нравится.}
{Continue if you like.}
}{Пословица ноа}

\epigraph
{Если вам говорят о том, что вы обязаны любить --- вас пытаются изнасиловать.
Кричите как можно громче.}
{Ветер-Стрекозьих-Крыльев, идеолог ранних тси}

\epigraph
{Задача правителя --- указать человеку его место.
Задача хорошего правителя --- помочь человеку отыскать своё место.
Задача лучшего из правителей --- не мешать человеку искать своё место.}
{Франциск IV, последний папа римский}

\epigraph
{Небесным телам, листьям деревьев, глазам птиц и человеческой крови совершенно не важна красота твоих изречений.
У них своя поэзия и свои законы.}
{Эрхэ Колокольчик}

\epigraph
{Каким, должно быть, смелым было животное, которое впервые в истории почувствовало боль.}
{Длинный-Мокрый-Хвост}

\epigraph
{Большая часть высоких технологий замечательно просты по своей сути.
Я не верю, что даже в случае глобального катаклизма мы скатимся до каменных молотков.
Пока жив хотя бы один человек с горящими глазами --- технологии будут сохранены, будут приспособлены к инфраструктуре и будут применяться.
Как человек не может разучиться велосипедной езде или плаванию, так и человечество, единожды научившись, уже не сможет это позабыть.}
{Михаил Кохани, презентация <<спичечных>> технологий в Массачусетском технологическом университете}

\epigraph
{Среднестатистический практически здоровый человек способен встроиться в любую социальную систему, имеющую простые и понятные правила.
<<Любую>>, к сожалению, означает и <<сколь угодно бесчеловечную>>.}
{Мариам Кивихеулу}

\epigraph
{Не тратьте время на сожаления.
Ваши предпочтения --- это не судьба, а результат удачного стечения обстоятельств.
Кое-кто считает, что три поворота налево и один направо приводят к одному и тому же результату.
Как показывает практика, результат в этих случаях всегда разный.}
{Людвиг Вейерманн}

\epigraph
{Все хотят тепла, но никто не хочет гореть.}
{Пословица Преисподней}

\epigraph
{...И лишь одного я боялась всегда ---\\
Что море меня не излечит.}
{Эрхэ Колокольчик}

\epigraph
{Будьте гордыми.
То, что вас согнули сегодня --- лишь стечение обстоятельств.
Меня часто унижали, когда я не могла дать отпор.
Но и пусть.
Я ничего не прощала и не прощу.
Вставайте на колени, если вас ставят насильно.
Молчите, если вам затыкают рот кляпом.
Признавайтесь, если вам угрожают оружием.
Это и вполовину не так ужасно, как оправдывать преступления или отыгрываться на других.
Будьте гордыми, берегите себя, свои силы и достоинство --- завтра ветер переменится.}
{Мариам Кивихеулу}

\epigraph{
\ml{$0$}
{Научись улыбаться в ответ.}
{Learn to smile back.}
}{Призказка ноа}

\epigraph
{\ml{$0$}
{Сомневаешься --- оставь меня в ножнах.}
{If in doubt, keep me untouched.}}
{Гравировка на сабле Митхэ ар'Кахр}

\epigraph
{Жить интересно именно потому, что мир не соответствует нашим ожиданиям.}
{Людвиг Вейерманн}

\epigraph{
\ml{$0$}
{Битва может стоить жизни, но без битвы ты не узнаешь, что такое жизнь.}
{Battle may take your life, but escaping battle you'll never know what life is.}
}{Присказка наёмников Фоуф}

\epigraph
{Мир изменится.
Он уже меняется.
И это случится, даже если вы сегодня проломите мне голову.
Потому что любовь и здравый смысл хоть и не сразу, но всегда побеждают ненависть и абсурд.
Так устроен этот мир, и я тут ни при чём.}
{Мариам Кивихеулу}

\epigraph
{Уничтожение произведений искусства должно стать табу для любого политического или общественного деятеля.
Вандал, какие бы прогрессивные идеи он ни нёс, никогда не вызовет сочувствие у общества.}
{Мартин Охсенкнехт}

\epigraph
{Ах ты, молодой Добрыня Никитич!
Бился ты со змеёй да трое суток, потерпи ещё три часа!
Ты побьёшь змею да ю, проклятую!}
{Фольклор культуры Руса, Древняя Земля}

\epigraph
{Понесший наказание безвинно имеет право совершить преступление того же характера, если невиновность будет доказана.}
{Третий постулат Возмездия.
Законы ноа}

\epigraph
{Если вы хоть раз в жизни выключили компьютер, разъединили телефонное соединение, убили комара или вкололи себе антибиотик, вы уже владеете основами интерфекции.}
{Элла Рид, основатель интерфекции}

\epigraph
{Лучший учитель --- ученик, который только что понял.}
{Пословица сели}

\epigraph
{Если дуть против ветра, то ветер не изменится, но зато перед твоим носом всегда будет островок штиля.}
{Пословица ноа}

\epigraph
{Есть люди, подобные кострам --- кормят и согревают целый лагерь, но тухнут в одиночестве.
Есть люди, подобные маякам --- не греют, но светят, даже если на горизонте нет ни одного корабля.}
{Пословица ноа}

\epigraph
{Может быть, стремление к власти и портит людей, но угроза потери власти превращает их в диких зверей и лишает всего человеческого.}
{Мариам Кивихеулу}

\epigraph
{Кости не ломаются от усталости, кости ломаются от чрезмерных усилий.
Стену следует строить маленькими камешками.}
{Пословица ноа}

\epigraph
{Если замолкает последняя певчая птица --- значит, дело действительно плохо.}
{Пословица сели}

\epigraph
{Кровь вытекает вместе с заразой, страх вытекает вместе с волнением.}
{Присказка хака}

\epigraph
{Сплошные солнечные дни порождают пустыню.}
{Пословица сели}
% Японская пословица

\epigraph
{Лучше быть врагом хорошего человека, чем другом плохого.}
{Пословица Преисподней}
% Японская пословица

\epigraph
{Быстро --- это медленно, но каждый день.}
{Пословица сели}

\epigraph
{Лето --- это танец, и глупо не принимать в нём участия.}
{Пословица Драконьей Пустоши}

\epigraph
{Иерархия разрушается в тот момент, когда члены сообщества отказываются играть по его правилам, добровольно меняя возможное лидерство на стигму нижней ступени.
Эти люди показывают прочим главное --- можно жить, самореализовываться и радоваться жизни, будучи <<отбросом общества>>.
Поэтому лидеры будут препятствовать этому, делая жизнь таких <<добровольцев>> невыносимой и даже устраняя их физически.}
{Мариам Кивихеулу}

\epigraph
{Ищи счастье.
Судьба найдёт тебя сама.}
{Клаудиу Дентосиу}

\epigraph
{Самая ужасная для государства вещь --- страх воина.
Страх воина питает диктатуру.
Внуши воину, что он в безопасности --- и диктатура падёт.
Внуши воину, что он любим народом --- и он будет с народом до конца.
Пусть каждый народ поклянётся оберегать своих воинов, как воины клянутся защищать народ --- и для человечества не будет больше плохих лет.}
{Анатолиу Тиу.
<<Послание к девяти завоевателям>>}

\epigraph
{Я несчастен, но счастливее меня не найти.}
{Людвиг Вейерманн.
Предсмертная записка}

\epigraph
{Если интерес к происходящему пересилил прочие чувства --- значит, вы победили в главном сражении своей жизни.
Все остальные победы --- вопрос времени.}
{Михаил Кохани.
Речь на вручении Расширенной Нобелевской премии.}

\epigraph
{Крысы боятся света.}
{Фризская пословица}

\epigraph
{Выучи один язык --- второй дастся тебе проще.\\
Научись играть на флейте --- цитрой овладеешь без труда.\\
Подружись со швейной иглой --- резец плотника сам прыгнет в твои руки.}
{Пословица ноа}

\epigraph
{Я меняю дни на расстояние\\
От Тси-Ди до джунглей Тра-Ренкхаля,\\
Мне доступно тайное знание ---\\
Как свернуть пространство-время желанием.}
{Песенка Заяц}

\epigraph
{Я никогда не был на войне.
Но если бы пришлось, я стал бы диверсантом или дезертиром --- и то и другое требует большой смелости.}
{Бенедикт Альсауд}

\epigraph{
\ml{$0$}
{Уважение тебе ничего не стоит.}
{Respect costs you nothing.}
}{Мариам Кивихеулу}

\epigraph
{Подумать надо и о них ---\\
Что надо, то надо,\\
Да вот поди-ка отыщи\\
Отбившихся от стада!\\
~\\
Живым идти под вострый нож\\
Тож неохота им,\\
Ищи! Но то, что ты найдёшь,\\
Будет ли живым?}
{Торстейн фра Хамри.
<<Поздней осенью в поисках овец>>}

\epigraph
{Вклад личности в историю принципиально не поддаётся оценке.
Любые изыскания на эту тему --- не более чем спекуляция.}
{Людвиг Вейерманн}

\epigraph
{Проблема мира --- не в отсутствии любви, а в неумении её выражать.}
{Мариам Кивихеулу}

\epigraph
{Мы меняемся вместе с миром, словно рисунок небес, и если среди ясного дня грянул гром --- значит, ты просто не заметил грозу.}
{Пословица тенку}

\epigraph
{Великий умеет находить своё истинное предназначение.
Счастливый умеет вовремя от него избавиться.}
{Клаудиу Дентосиу}

\epigraph
{Есть три верных признака государства, находящегося в состоянии гражданской войны --- разбитые дороги, разрушенные больницы и закрытые школы.
Хуже всего, если разбиты дороги в городах и деревнях.
Ведь для хороших дорог нужно лишь, чтобы два добрых соседа по две стороны от дороги могли что-то сделать сообща.}
{Марке Скрипта}

\epigraph
{Мироздание не знает, что такое благодарность.
Делаешь что-то полезное --- позаботься о награде сам.}
{Длинный-Мокрый-Хвост}

\epigraph{
\ml{$0$}
{Дорога излечит всё.}
{The road heal you.}
\ml{$0$}
{Или убьёт.}
{Or kill you.}
}{Пословица ноа}

\epigraph
{Неопытный кушает впрок;
мудрый берёт пищу с собой.}
{Пословица сели}

\epigraph
{<<Возлюби ближнего своего>> прежде всего означает <<Оставь ближнего своего в покое>>.}
{Фридрих Ницше.
Эпоха Господина}

\epigraph
{Старинный рецепт идеальных солдат: взять молодых, лишить знаний и наставлений, закрыть в клетки и вынудить выживать, а потом дать выжившим идеологию и пообещать им любые блага.}
{Присказка наёмников Фоуф, Тысяча Башен}

\epigraph
{Если в войне между правителями участвуют солдаты, но нет наёмных убийц --- никакой войны нет, есть лишь игра.
Только игроки жертвуют фигурами, не пытаясь сунуть друг другу нож.}
{Присказка наёмников Фоуф, Тысяча Башен.}

\epigraph
{Смерть может быть привлекательной в двух качествах --- как выход и как неизведанное.}
{Постулат Эволюциона}

\epigraph
{Когда с собаки снимают чересчур суровый ошейник, она рычит и кусается от боли.}
{Клаудиу Дентосиу}

\epigraph
{Я бы тоже хотел мир, где мне не пришлось бы доказывать своё право, ломая чужие копья, разрывая путы и обходя волчьи ямы.
Но мира кроме этого у меня просто нет.}
{Гало Кровавый Знак}

\epigraph
{Жестокость --- храбрость трусов.}
{Фазиль Искандер, Эпоха Последней Войны.}

\epigraph
{Тюрьма --- это судьба любого преступника.
Диктаторы и коррумпированные чиновники сами строят вокруг себя тюрьмы, и эти тюрьмы лишь по недоразумению называют дворцами и крепостями.}
{Анатолиу Тиу}

\epigraph
{At satis hostium.
Эта фраза была эпитафией некоего Матиаса Стойса, похороненного в Кафедральном соборе Кёнигсберга.
Сейчас, к сожалению, барельеф не сохранился, но его застал мой прапрадед, Алекс Орлов.
Он был военным.
И однажды увидел эту надпись, прогуливаясь по городу.
Через несколько дней он уволился из армии, а ещё через год основал Общество Сломанного Копья, помогающее военным адаптироваться к гражданской жизни, получить образование и работу.
Сейчас эта фраза --- at satis hostium --- выбита на его могиле.
Прапрадед был убит милитаристами спустя семь лет, когда Общество Сломанного Копья распространилось по трём континентам.}
{Мемуары Михаила Кохани}

\epigraph
{Когда человек мучается, у него появляется потребность мучить других.}
{Ромен Роллан.
Эпоха Последней Войны}

\epigraph
{At satis hostium\FM.}
{Эпитафия Матиусу Стойусу Прусскому.
Кафедральный собор Кёнигсберга, домен Европа, Древняя Земля}
\FA{
Но хватит врагов (эллатинский).
}

(Последняя встреча Гало и Тахиро!)

\epigraph
{Лгать --- как нести скользкий чан под дождём: чем дальше, тем тяжелее;
не уронишь чан, так расплещешь воду.}
{Пословица ноа}

\epigraph
{Правду можно оставить на дороге;
ложь приходится нести с собой.}
{Пословица сели}

\epigraph
{Хрупкая птица пролетит расстояние, которое не сможет пройти самый сильный ягуар.}
{Пословица сели}

\epigraph
{Вера обычно начинается там, где кончаются силы и фантазия.}
{Пословица сели}

\epigraph
{Пройди лигу\FM, перекати бусинку.
Чётки покажут путь, ноги приведут.}
{Присказка ноа}
\FA{
Лига --- мера длины ноа.
Соответствует ровно 1.5 кхене.
}

\epigraph
{Даже самый могущественный --- часть целого.
Помня об этом, правитель не опустится до диктатора, а народ не допустит попрания своих прав и свобод.}
{Анатолиу Тиу}

\epigraph
{Королева не кормит грудью, король не учит сыновей, братья режут друг друга ради трона, а дочерей используют как разменную монету.
Королевская семья --- извращение самой сути семьи.
Должна ли она говорить добрым талианцам, как им жить?}
{Тациан Освободитель}

\epigraph
{Учи дитя порядку, но помни, что дисциплина --- кормилица лжи.}
{Поговорка сели}

\epigraph
{Совершать безумства --- это способность.
Совершать безумства без сожалений --- это дар.}
{Михаил Кохани}

\epigraph
{Нельзя недооценивать Эпоху Последней Войны.
Именно тогда, в огне вооружённых конфликтов, революций, забастовок, военных переворотов и уродливых, саморазрушительных идеологий ковалась самая жизнеспособная, самая гуманная философия первых людей, обеспечившая почти двадцатидвухтысячелетний период мира, процветания и расселения Ветвей Земли по Вселенной.}
{Кельса Пушистая}

\epigraph{
\ml{$0$}
{Счастливые женщины --- хорошая деревня.}
{Happy women mean a healthy village.}
}{Хаяо Миядзаки}

\epigraph{
\ml{$0$}
{Есть высоты, на которых худший способ висеть лучше падения.}
{There are some heights, where the worst way to hang on is better than to fall off.}
}{Пословица хака}

\epigraph
{Если из жилища видно не красоты природы, а лишь окружающие его стены --- это осквернение самой идеи жилища.}
{Клаудиу Дентосиу}

\epigraph{
\ml{$0$}
{Всегда будь готов к тому, что ничего не произойдёт.}
{Always expect nothing to happen.}
}{Длинный-Мокрый-Хвост}

\epigraph
{Великий талант --- жить в неопределённости без иллюзий и без тревог.}
{Людвиг Вейерманн}

\epigraph
{Мы не шли на войну, чтобы убивать или быть убитыми.
Мы шли на войну, чтобы нас услышали.}
{Субкоманданте Инсургенте Маркос.
Эпоха Последней Войны}

\epigraph
{Что может быть прекрасней, чем дышать?\\
Быть может, к новой жизни\\
родиться из тяжёлых снов дневных?\\
~\\
Проснулся ---\\
и уже не помнишь,\\
что разбудило:\\
достаточно любить себя,\\
чтобы проснуться.\\
~\\
А нетерпенья\\
достаточно, чтобы сгореть\\
для повторенья.}
{Торстейн фра Хамри, <<Нетерпение>>.
Эпоха Последней Войны, Древняя Земля}

\epigraph
{Утонул в колодце --- беда всего города.
Утонул в море --- только твоя беда.}
{Пословица ноа}

\epigraph
{Все жизненные пути бессмысленны, но есть путь сердца.
Он такой же бессмысленный, как и остальные, но по нему идёшь с радостью.}
{Карлос Кастанеда}

\epigraph
{Экспертом является тот, кто совершил все возможные ошибки в некотором узком поле.}
{Нильс Бор}

\epigraph
{В человеке нет ни одного качества, которое когда-либо не поспособствовало его выживанию.
Просто помните об этом, когда решите что-либо осудить.}
{Март <<Одноглазый>> Митчелл}

\epigraph
{Многие сейчас говорят о бедах, которые принесли слепая вера и фанатизм.
Упаси меня Господь отрицать очевидное.
Но нужны ли они человечеству?
Нужны!
Не расчётливые заселили самые дальние берега Земли, и не расчётливым идти на край Вселенной!}
{Март <<Одноглазый>> Митчелл}

(этот эпиграф однозначно к <<Кон-Тики>>)

\epigraph
{Он обрёл почти безграничную власть, просто перестав осуждать.
Эта власть ограничивалась лишь его собственными принципами.}
{<<Речь о жреце>>.
Речь ноа}

\epigraph
{Насыщает похлёбка, а не просиженное в харчевне время.}
{Пословица сели}

\epigraph
{Спроси себя в печали, спроси себя истекающего кровью, спроси себя усталого и измотанного, желал бы ты иной жизни?
Так познаются цветение духа и правильность пути.}
{Мартин Охсенкнехт}

\epigraph
{Соседи твои и друзья\\
Ходят по городу сытые,\\
Но за смехом таится молчание,\\
А в молчании дремлет бунт.}
{Торстейн фра Хамри, <<Дух времени>>.
Эпоха Последней Войны, Древняя Земля}

\epigraph
{Умеющий шагать не оставляет следов.
Умеющий говорить не допускает ошибок.
Кто умеет считать, тот не пользуется счетом.
Кто умеет закрывать двери, тот не употребляет запор и закрывает их так крепко, что открыть их невозможно.
Кто умеет завязывать узлы, тот не употребляет веревку, и завязывает так прочно, что развязать невозможно.}
{<<Книга Пути и Достоинства>>, Мудрый Старец.
Культура Цина, Древняя Земля}

\epigraph
{Познание себя и прочей Вселенной имеет чересчур памятный горько-сладкий вкус;
того, кто распробовал его однажды, может остановить только смерть.}
{Клаудиу Семито Фризский.
Эпилог к <<Воспоминаниям о великом тёзке>>}

\epigraph
{Чем больше обязанностей по самообеспечению вы возлагаете на окружающих, тем сильнее от них зависите.
Не добываете руками хлеб?
Приготовьтесь голодать.
Не умеете мыть полы?
Приготовьтесь существовать в грязи.
Лет триста назад я бы воскликнул: <<И не приведи судьба вам отдать ртуть\FM\ и железо в чужие руки!>>
Малу вайю\FM, поздно --- это наша действительность.
Казначеи и ростовщики грабят вас, выменяв вашу ртуть на бумагу.
Король велит закапывать фальшивомонетчиков, но при каждом удобном случае платит вам медной амальгамой.
Солдаты и стражники убивают вас оружием, которое выковали вы из вами добытого железа.
И самое страшное --- вы всё ещё считаете, глупцы, что это в порядке вещей.}
{Анатолиу Тиу.
Речь перед жителями Фриза}
\FL{viola}{Фиола}
\FA{Malu vaeu --- Увы! О горе! Чёрт возьми! (s-l, примерный перевод)}

\epigraph
{Твердое и крепкое это то, что погибает, а нежное и слабое есть то, что начинает жить.
Сильное и могущественное не имеют того преимущества, какое имеют нежное и слабое.}
{<<Книга Пути и Достоинства>>, Мудрый Старец.
Культура Цина, Древняя Земля}

\epigraph
{Принятие конкретной социальной установки --- это всегда давление на суперпозицию личности.
Счастье в том, чтобы давление казалось объятиями, а не ломало рёбра.}
{Мариам Кивихеулу}

\epigraph
{Жизнь слепа и путь прокладывает на ощупь.
Высокий интеллект --- это та широкая лазейка, которую жизнь нащупала в практически непреодолимом для неё препятствии --- космическом пространстве.}
{Софиа Ловиса Карма}

\epigraph
{В искусстве невозможно опоздать.
Когда бы ни было создано произведение, оно появляется вовремя.}
{Мартин Охсенкнехт}

\epigraph
{Тот, кто впервые обругал соплеменника вместо того, чтобы ударить его, стал прародителем цивилизации.}
{Длинный-Мокрый-Хвост}

\epigraph
{Биология с её естественным отбором гуманнее большинства представителей человеческого рода.
Так что оправдывать несправедливости биологией оставьте лжецам для неучей.}
{Сергей Леонидовиц Хистиаков}

\epigraph
{Если какой-то деятель искусства скажет при вас, что художники, писатели, поэты и драматурги двигают мир вперёд --- можете громко рассмеяться ему в лицо.
Мир двигали и двигают те, кто изредка, раз в месяц или год идёт в театр или читает книжку, чтобы расслабиться и назавтра, после здорового крепкого сна снова делать то, что должно --- подметать улицу, чинить машины, учить детей и лечить больных.}
{Мартин Охсенкнехт}

\epigraph
{Всё, что пытаются доказать силой --- розгами, кулаками, резиновыми дубинками или пулемётами --- скорее всего, ложь;
ни одному первооткрывателю ещё не приходилось бить или убивать оппонентов, чтобы убедить их в своей правоте.}
{Михаил Кохани}

\epigraph
{Дьявол начинается с пены на губах ангела, вступившего в бой за правое дело.}
{Грегори Померантс}

\epigraph
{Каждое заблуждение имеет цену, и эта цена всегда измеряется в человеческих жизнях.
Исключений не бывает.}
{Марке Скрипта}

\epigraph
{Врач и философ Марке Скрипта является изобретателем термина <<тирания мозга>>.
Так он называл состояние, когда человек, отождествляя себя исключительно с нервной системой, сосредотачивается на её внутренних задачах, игнорируя или непомерно эксплуатируя при этом прочие системы организма.
Это же состояние я наблюдаю сейчас в нашем государстве;
Скрипта полагал, что тирания мозга есть первопричина многих болезней тела и всех болезней души.}
{Анатолиу Тиу}

\epigraph
{Тси вернутся.
Они вырвутся в явь из ваших кошмаров.
Они пробьются светом через ваши решётки, просочатся влагой через гранит ваших стен.
Они не станут мстить, но их возвращение положит вам конец.}
{Уэсиба Серозмей}

\epigraph
{У народа, живущего в долине посреди Хребта Малого Листопада, есть замечательный обычай.
Если кто-то совершил преступление, люди начинают заботиться о нём.
Ему напоминают о его самых лучших поступках, его обнимают и всячески проявляют к нему любовь.
Я скажу вам, что это маленькое отсталое племя знает истину, которую не могут осознать самые лучшие умы государя: преступник --- это тот, кто кашляет, когда общество больно.}
{Анатолиу Тиу.
<<Земной суд>>}

\epigraph
{Если вы услышите от меня в старости, что я отказываюсь от сегодняшних слов, пожалейте меня, но не верьте сказанному.
Предел есть у каждого.
Среди стариков много уставших, больных и сломленных;
трудно надеяться, что сия чаша минет мою голову.
Не верьте и прочим старикам, что публично отрекаются от прекрасных идей молодости.
Хоть историю и пишут достигшие преклонных лет, молодые гораздо чаще знают истину.
Так было и будет всегда.}
{Михаил Кохани}

\epigraph
{Ещё одно существенное отличие сели: их мифология носила \emph{исключительно игровой} характер.
Примером могут служить приведённые результаты опроса по <<небесному пахарю>>[44][136].\\
Взрослые[88] сели:\\
94\% при вопросе о сущности падающих звёзд, смеясь, рассказывали легенду о небесном пахаре, но при переводе беседы в серьёзное русло всё-таки рассказывали о летающих раскалённых камнях;\\
6\% безо всяких шуток давали совершенно точный ответ.\\
Взрослые[88] тенку:\\
10\% (в основном образованные монахи) давали правильный ответ и выказывали неудовольствие, услышав легенду о пахаре;\\
62\% совершенно серьёзно говорили о проделках богов и удивлялись рассказу о летающих камнях;\\
28\% промолчали или ответили <<не знаю>>, отклонив таким образом предложение побеседовать.\\
Ещё сильнее эта разница заметна у детей.\\
Прошедшие Отбор дети[88] сели:\\
78\% рассказали легенду о пахаре.\\
После вопроса о том, как всё обстоит <<на самом деле>>:\\
8\% затруднились ответить;\\
18\% спросили <<как?>>;\\
12\% начали фантазировать,\\
40\% упомянули о раскалённых угольках, летящих из очага\\
(таким образом детям объясняют сущность небесных тел в школе).\\
22\% детей отказались беседовать на обозначенную тему.\\
Выжившие дети[88] тенку:\\
56\% рассказали легенду о пахаре,\\
42\% уверенно повторили её после уточняющего вопроса,\\
4\% спросили <<как?>>,\\
10\% начали фантазировать.\\
44\% отказались беседовать на обозначенную тему.}
{Аркадиу Шакал Чрева.
<<Сели: первые впечатления>>}

\epigraph
{Генотип, медикация и обучение каждого тси должны обеспечивать отсутствие статистически значимой разницы между значениями двух параметров --- выживаемости и коэффициента личного комфорта\FM\ --- в присутствии технологической поддержки и при полном отсутствии таковой.}
{Второй постулат Кодекса Тси-Ди}
\FA{
Коэффициент личного комфорта (КЛК) --- параметр, определяющий степень раскрытия возможностей сапиента в заданных условиях.
В настоящее время считается устаревшим.
}

\epigraph
{Не найти прекраснее цветов преисподней, но цветение их мимолётно.}
{Анатолиу Тиу.
<<Путешественник>>, диалоги}

\epigraph
{--- Я слеп, отче.
Как мне увидеть, что есть добродетель и порок?\\
--- Всё вышло из крови и уйдёт в землю, сын.}
{Анатолиу Тиу.
<<Путешественник>>, диалоги}

\epigraph
{... Увы, но бывают моменты, когда на одной чаше весов лежит отрубленная рука, а на другой --- судьба, жизнь и смерть.
Казалось бы, выбор очевиден для всех --- спросите любого зеваку на улицах нашей столицы.
Но когда этот момент настает, рука почему-то перевешивает...}
{Анатолиу Тиу.
Трактат <<О природе дышащих>>}

\epigraph
{Салафиты нарисовали на своём знамени перечёркнутый посох?
Понимаешь ли ты, что это значит?
На их знамени --- твой знак, Леам!
Они признали поражение, признали, что в дереве их собственного учения соков больше нет...}
{Анатолиу Тиу.
Письмо к Леаму эб-Салаху}

\epigraph
{Не бойтесь гнева предков, бойтесь стыда потомков.}
{Анатолиу Тиу.
<<Буря перемен>>, часть III}

\epigraph
{Ты говоришь, что принёс свободу, и гордишься данными тебе народом титулами?
Ты раб с рождения, Татиан, ещё худший, чем те, кого ты за собой ведёшь.
Меня называют светилом, философом, певцом свободы, но единственная свобода, которую я хочу --- это свобода не быть тем, кем меня называют, и не делать то, что от меня ждут!}
{Анатолиу Тиу.
Ответ на призыв Татиана Освободителя оказать его движению поддержку}

\epigraph
{У гаруспиков всегда будет множество клиентов.
Людей волнует, когда они разбогатеют, найдут любимого человека, умрут...
Меня волнует только один вопрос --- когда я окончательно сдамся.
За ответ я бы отдал всё своё состояние, но увы --- его не даст ни один гаруспик.}
{Анатолиу Тиу.
Речь перед лиманскими повстанцами}

\epigraph
{Для нас есть вероятности, но нет предопределения.
Для мироздания есть предопределение, но нет вероятностей.
Непроницаемое тёмное пятно вероятностей в вашем будущем --- это и есть то, что вы называете <<свободой воли>>.}
{Анатолиу Тиу.
Ответ на шариатском суде}

\epigraph
{Сожжённым вами люди будут дышать.}
{Последние слова Анатолиу Тиу перед смертью.
Шариатский суд постановил <<опалить лицо>> уже престарелого Тиу лишь теми книгами, в которых явно доказано присутствие ереси.
К чести инквизиторов и к несчастью старика, таких книг хватило на костёр.}

\epigraph
{То, что было взято дланью войны\FM, будет потеряно...
Завоеванная женщина не будет верна;
завоеванная страна вернет себе свободу, едва окрепнув, часто под знаменами шовинизма и религиозно-морального фанатизма, этих извечных суррогатов потерянного достоинства.
Посему опытный правитель должен делать ставку на договорные отношения... и помнить, что слабые нуждаются в уважении куда больше сильных.}
{Анатолиу Тиу.
<<Послание девяти завоевателям>>}
\FA{
Дланью войны (t-sl: way mana militharrhio) --- с помощью военной силы.
}

\epigraph
{В семечке --- фрукты и алый цветок.\\
В ножнах таится убийцы бросок.\\
Форму узри --- и узришь содержанье.\\
В лицах и благо ищи, и порок.}
{Клаудиу Семито Фризский.
<<Песенки Клаудиу>>, приложение III к <<Воспоминаниям о великом тёзке>>}

\epigraph
{--- Ты обвиняешься в организации культа с целью подрыва королевской власти, --- монотонно сказал судья.
--- Что ты можешь ответить на это?\\
--- Культ правильности сменяется культом безумия.
Культ поклонения сменяется культом противостояния.
Ваше обвинение похоже на обвинение пастуха, у которого угнали стадо баранов.\\
Люди в зале --- как сторонники, так и противники Клаудиу --- вскочили на ноги и начали выкрикивать в адрес поэта оскорбления.
И только три или четыре человека молча чему-то улыбнулись...}
{Клаудиу Семито Фризский.
<<Воспоминания о великом тёзке>>}

\epigraph
{Клаудиу в последний раз посмотрел на оставленный им город.
Люди вокруг яростно смотрели на него и тискали пальцами оружие, но никто не решался напасть --- поэта верно хранило мастерство убийцы и королевский эдикт.\\
<<\ldots из величайшей королевской милости рекомого Клаудиу отпустить и хранить его от всяческих посягательств>>.
Милости и страха перед восстанием.
Никто не хотел, чтобы Клаудиу превратился в легенду и знамя для обездоленных.\\
Поэт поправил котомку и, весело насвистывая любимую песенку, направился в сторону побережья.
Он не подозревал, что стал легендой уже при жизни.}
{Клаудиу Семито Фризский. Эпилог к <<Воспоминаниям о великом тёзке>>}

\epigraph
{Нет любви без знания, а невежество --- лучшая пища для ненависти.}
{Пословица народа сели}

\epigraph
{Когда Клаудиу подвели к гильотине, один из министров Валериу Х насмешливо спросил, готов ли поэт умереть за свои идеи.
<<Я готов даже к тому, что они окажутся глупыми и никчёмными.
А уж умру я за них с большим удовольствием>>, --- ответил Клаудиу и, не дожидаясь приказа, положил голову на плаху.}
{Клаудиу Семито Фризский. 
<<Воспоминания о великом тёзке>>}

\epigraph
{Мало ему победы, он и смерть забрал!}
{Изречение, выбитое на гробнице Велира IV Хореида, принадлежавшее его врагу Вериту Валериду}

\epigraph
{Посмотри на волосы врага --- и ты поймёшь, на каком языке пойдёт беседа.}
{Пословица народа сели.
Купцы, исполняющие роль дипломатов, носили хвостик или пучок на затылке и вели переговоры на языке цатрон.
Воины обычно стриглись коротко и использовали <<язык стали>> --- боевые искусства.}

\epigraph
{Сжатая в кулаке крыса может отгрызть руку.
Загнанный в угол заяц может убить охотника.
Отчаявшийся тигр может истребить целую деревню.
На что же способен человек, которому нечего терять?}
{Пословица народа сели}

\epigraph
{Жизнь --- испытание для любого сапиента.
Веками мы придумывали сказки, чтобы оградить себя от естественного желания положить ей конец.
Время сказок прошло.
Наша задача --- помочь всем и каждому найти место в жизни, на деле, а не на словах убедить сапиентов, что их существование необходимо и страдания оправданы.}
{Постулат Эволюциона}

\epigraph
{Утопия подразумевает людей, которые довольны своей жизнью.
Однако всегда появляются бунтари, которых что-то не устраивает.
Человек меняется, следуя за условиями среды, общество меняется вслед за человеком.
Иметь претензии не есть плохо.
Желать изменений к лучшему --- это не подрыв устоев, а движение вперёд.
Звучит парадоксально, но если общество претендует на звание идеального, оно должно быть готово к любым изменениям.}
{Софиа Ловиса Карма, идеолог Эволюциона}

\epigraph
{Как я стал капитаном <<Тёмного пламени>>?
(смеётся) Просто пришёл в эту их комиссию и сказал: что за дерьмо, у меня под килем слишком мало парсак!
И парсак мне сразу отсыпали...}
{Бенедикт Альсауд.
Запись беседы с историками}

\epigraph
{В океане и в космосе нет границ.
Если кто-то хочет расчертить воду и святые небеса, пусть сначала озаботится суверенитетом собственной задницы --- возможно, она в рабстве у чересчур удобного кресла.}
{Бенедикт Альcауд.
Открытое письмо Транспортному комитету по поводу сомнительного законопроекта}

\epigraph
{Если ты проснулся бодрым --- значит, для тебя наступило утро.}
{Пословица тенку}

\epigraph
{Идеальное общество --- это общество, в котором даже чужак чувствует себя своим.}
{Критерий де Ла Роче.
Постулаты Эволюциона}

\epigraph
{Собака --- друг человека.}
{Изречение времён Древней Земли}

\epigraph
{Когда тси только начали подготовку к одичанию, они столкнулись с неожиданной проблемой --- прочие формы жизни Тра-Ренкхаля не могли соперничать даже с потерявшими большую часть технологий пришельцами.
Тси ловили зверей и птиц едва ли не голыми руками, что могло поставить десятки тысяч видов под угрозу исчезновения.
Воинственность аборигенов также могла сыграть с ними плохую шутку --- если тси переняли бы этот стереотип поведения, то аборигенов ждало неминуемое истребление.
Можно утверждать, что эти немногие тси с одним космическим кораблём в одночасье стали хозяевами планеты, и даже демиург Безымянный не питал никаких иллюзий на этот счёт[12].\\
Именно поэтому Баночка и Кошка разработали так называемую тактику форы (подробнее см. раздел 13.7.1).
Записей об этой методике осталось немного[13], но результаты исследований[14][15][16][22] говорят сами за себя --- тси провели беспрецедентный, грандиозный отрицательный отбор, сохранив многообразие видов.
Известно, что этот отбор производился с согласия и при поддержке Безымянного[12].
Именно благодаря дальновидности Баночки и Кошки люди Тра-Ренкхаля, нгвсо и дикие животные если и не стали тси достойными конкурентами, то хотя бы получили право на жизнь.}
{Аркадиу Шакал Чрева.
Тезисы работы <<Взаимодействие тси с биоценозом планеты Тра-Ренкхаль>>}

\epigraph
{Трусость --- враг процветания, а жестокость --- враг мира.
Да, я трус.
Во мне есть и жестокость.
Те, кто не знаком со мной лично, почувствовали эту жестокость в моём романе.
У многих из вас загораются глаза, когда вы читаете о крови и войне.
Мы --- потомки тех, кто убивал и насиловал на протяжении тысяч лет, кто раболепно склонялся перед силой.
Это их воинственные крики мы слышим в беспокойных снах, это их мысли мелькают на краю сознания, едва в наших руках оказывается оружие, это их страх пробирает нас до костей, едва оружие направляют на нас.
Но, клянусь пред лицом Господа, даже с таким <<наследством>> можно жить в процветании и сохранять на Земле мир!
И для начала будем честны перед собой и признаем простой факт: в нас течёт кровь тех, кто выжил благодаря жестокости и трусости.}
{Март <<Одноглазый>> Митчелл, идеолог Эволюциона, один из последних казнённых преступников в эпоху Последней Войны}

\epigraph
{Когда Тси-Ди была повержена Машиной, мы думали, что последний оплот свободных сапиентов Земли уничтожен.
Но сейчас я склонен полагать, что ситуация, поставившая тси на грань жизни и смерти, лишь укрепила их.
Большая часть тси была перебита, но выжило достаточно, чтобы через десять тысяч лет вернуться в виде новой, более страшной угрозы --- Скорбящих.
Это уже не просто потомки великих учёных --- это хоргеты, воины с непонятной, чуждой нам моралью, которые уверены в своей правоте и готовы сражаться за неё до конца.
Есть данные, что Скорбящие проходили подготовку по противодействию Чистилищу.
Эти существа использовали камеру пыток для тренировок!
Я надеюсь, все поняли и осознали глубину этого факта.
Это не искатели приключений.
Они знали, на что идут.\\
Больше мы не можем закрывать глаза на эту проблему.
Оставшиеся на Тра-Ренкхале тси должны быть подвергнуты геноциду, чтобы предотвратить пополнение рядов Скорбящих.}
{Самаолу Каменный Старик, член совета Ордена Преисподней}

\epigraph
{Если дать свободу опьянённому властью, он обязательно нарушит закон.}
{Анатолиу Тиу}

\epigraph
{Слушайте древние легенды.
Не обязательно им верить, просто слушайте.}
{Анатолиу Тиу.
Предисловие к <<Древнейшей истории Драконьей Пустоши>>}

\epigraph{Счастье --- синоним деградации.
Прогресс рождается в борьбе.
Его творят негодующие и скорбящие.}
{Изречение неизвестного Таракана, ставшее негласным девизом народа тси}

\epigraph{Ищите.
В мире много вещей, которые невозможно потерять.
Музыка и физика --- лишь некоторые из них.}
{Людвиг Вейерманн}

\chapter{Живая сталь}

\section{Первая кровь}

--- Баночка, пожалуйста! --- Кошка встала между плантом и воинами царрокх, не обращая внимания на направленные на неё копья.

--- Уйди, дура! --- заорал Баночка и, дёрнув женщину за руку, повалил её на траву.
Там, где они стояли миг назад, свистнули три стрелы.

--- Баночка! --- плакала Кошка, закрыв лицо руками.

--- Фонтанчик, отойди! --- завопил Баночка.
В его голосе сквозил дикий страх за нас.
--- Просто отойди!
Ты не осилишь этих копейщиков!
Возьми Кошку и иди к скалам!

Фонтанчик на пару секунд заколебался.
Я понял, что он хочет помочь другу, положившись на свою реакцию и металлические конечности;
но тон Баночки говорил сам за себя, а Фонтанчик привык доверять специалистам и тем более друзьям.
Поэтому в конце концов он кивнул и осторожно двинулся по направлению к лежащей Кошке.

Царрокх долго не решались нападать на планта, замершего в странной позе.
Он легко отбил саблей ещё две стрелы.
Один из воинов воспользовался всеобщим замешательством и швырнул копьё в Кошку;
Фонтанчик едва успел прикрыть её собой.
Обсидиановые лезвия чиркнули по обитым металлом рёбрам, и копьё отлетело в сторону.

--- Ха! Ха! Ха! --- хором зарычали царрокх.

--- ААААААААЭЭЭЭ! ---- взревел в ответ Баночка.
Я не подозревал, что маленький плант способен на такой звук.
Его голос потряс джунгли, словно рык старого ягуара.

Секунда --- и бойцы смешались в один большой, воющий от боли клубок.

\asterism

Фонтанчик молча оглядел поле боя.
Лечить здесь было некого, на каждой воине царрокх была как минимум одна не совместимая с жизнью рана.
В стороне лежали две головы отдельно от тел;
первого нападающего меч Баночки просто рассёк напополам, от ключицы до копчика.
Посреди всего этого кошмара неподвижно сидела маленькая жалкая фигурка.

--- Баночка! --- крикнула Кошка и понеслась вперёд.

Баночка поднял на нас расширенные, полные слёз глаза.
Его лицо было забрызгано кровью, губы и руки неистово тряслись.

--- Я убил человека, --- шёпотом сказал он.
--- Кошка, я убил человека...
Такого же, как ты, как Заяц, как Мак...

--- Баночка, ты ранен?
Он весь в крови...

--- Кровь красная, --- резонно сказал Фонтанчик, ощупывая друга.
--- Чёрного не вижу.
Его как будто вообще не задели...
Баночка, где болит?

--- Я убил человека...

--- Баночка, пожалуйста, мне нужно понять, что у тебя повреждено, --- взмолился Фонтанчик.
--- Кошка, его надо отнести к воде и обмыть, в этой грязи мы ничего не поймём...

--- Я убил человека, --- прорыдал Баночка и ничком свалился на землю.

--- Всё хорошо, мой милый, --- шептала Кошка.
--- Всё хорошо...
Мы целы, ты нас спас.
Иди ко мне на ручки, милый, вот так, всё хорошо...

--- Они были как ты, --- бессвязно плакал ей в плечо Баночка.
--- Как ты, Кошка.
Ты меня ненавидишь?

--- Я люблю тебя, милый.
Как я могу тебя ненавидеть...
Лежи, лежи, всё хорошо, я тебя донесу...

\asterism

--- Это удивительно, но он совершенно цел.
Пара растяжений из-за непривычной нагрузки, пара царапин от веток.
Основной удар нанесён по психике.
\ml{$0$}
{Я боюсь, что прежним он уже не будет.}
{I'm afraid he won't be the same anymore.''}

--- Но можно хотя бы сделать так, чтобы ему стало легче?

\ml{$0$}
{--- Не за одну ночь и даже не за десять, --- сказал врач.}
{``Not in one night, not even in ten,'' the doctor said.}
--- Я пока подержу его на успокоительных.
К вам будет просьба приходить почаще, ну и в целом не оставлять его одного, хотя бы первое время, а в идеале --- несколько оборотов.
\ml{$0$}
{Есть ещё одна проблема...}
{There's one more problem ....''}

\ml{$0$}
{--- Какая?}
{``What?''}

Врач помялся.

--- Баночка --- личность в некотором смысле известная, медийная, и у меня многие уже интересуются, что с ним случилось.
Я пока что ссылаюсь на врачебную тайну, но, боюсь, это только подогреет ненужные слухи.

\ml{$0$}
{--- Мы никому ничего не рассказываем, --- твёрдо сказала Кошка.}
{``We won't tell anyone anything,'' Cat firmly said.}

\ml{$0$}
{--- А рассказать-то надо, --- буркнул Баночка со своей постели.}
{``But it has to be told,'' Flask grumped from his bed.}

--- Мы думали, ты спишь, --- смутилась Кошка.

--- Не могу спать.
Успокоительные больше не действуют.

--- Ты можешь об том сказать мне, --- ответил врач.

--- Как я это сделаю на такой дозе седативных и миорелаксантов? --- резонно спросил Баночка.
Он едва шевелил губами.
--- Всё, хватит с меня химии.
Я уже четыре дня здесь лежу.
Отпустит --- пойду жрать.
Потом расскажу народу, что к чему.

--- Баночка, это дополнительный стресс для тебя.
Я не думаю, что...

\ml{$0$}
{--- По-твоему, две дюжины трупов царрокх никто не заметит?}
{``You think two dozens of dead \Tesarrokch\ remain unnoticed?}
\ml{$0$}
{На сколько баллов из десяти это осложнит наши и так не безоблачные дипломатические отношения?}
{On a scale of one to ten, how difficult will be our diplomatic relations, which, actually, have never been cloudless before?''}

\ml{$0$}
{--- Он прав, --- признал я.}
{``He's right,'' I admitted.}
\ml{$0$}
{--- Нужно рассказать всем.}
{``Everyone should know.}
\ml{$0$}
{И, как бы мне ни хотелось взять это на себя --- рассказать должен Баночка.}
{And as much as I want to take this burden, Flask must be the one to tell.''}

Баночка кивнул и взглянул на меня.

--- Ты был неправ, Небо.
\ml{$0$}
{Надо было сделать нелетальное оружие.}
{We should've made non-lethal weapons.}
\ml{$0$}
{Мы бы просто временно вывели этих воинов из строя, и ничего этого не было бы...}
{We could disable those warriors for a while, and none of that would've ever happened ....''}

Я кивнул.

--- Займёшься этим, как придёшь в себя?

--- Извини, --- покачал головой плант и, закутавшись в одеяло, отвернулся к стене.
\ml{$0$}
{--- С меня хватит.}
{``I'm done.''}

\section{Расставание}

--- Давай, Небо, --- убеждал меня Шмель.
--- С ними всё хорошо.
Окуклились правильно, ротики на месте, жопки там где надо.
Давай их сюда, я положу их в термостат с остальными куколками.

Я продолжал стоять, крепко прижав к себе малышей.

--- Мне кажется, что я не вернусь.

--- Тебе есть к кому возвращаться, значит --- вернёшься, --- проникновенно сказал Шмель.
\ml{$0$}
{--- Давай, Небо, туда и обратно, приключение на двадцать минут.}
{``Come on, Sky, there and back again, twenty minute adventure.}
Как выкуклятся --- будем вместе эту ораву выгуливать.
Лети и ни о чём не думай, кроме дела, я о них позабочусь.

Шмель почти силой забрал у меня куколки и, аккуратно завернув их в воск, ушёл в сторону медблока.
Я остался стоять в одиночестве, глядя в чёрное, полное звёзд небо.

\section{Самое быстрое животное}

--- Мы не успеем, --- взволнованно сказал Фонтанчик.
--- Эх...
Заяц, Небо, цепляйтесь за меня карабинами и затяните тросы потуже.
Упадёте --- костей не соберёшь.

--- Нет, Фонтанчик, --- Заяц отпрыгнула и выставила вперёд руки.
--- Нет!
Мне хватило одного раза!

--- Ты хочешь успеть или нет?

Заяц горестно вздохнула и повернулась ко мне.

--- Надеюсь, ты не поел перед этим.

Я и раньше, разумеется, знал, что кани на четвереньках способны развивать самую высокую скорость среди наземных животных планеты Тси-Ди.
Но пассажиром был впервые.
Фонтанчик летел, как шерстяная комета, молниеносно огибая и перепрыгивая препятствия.
Под сероватой кожей его спины бегали мощные мышцы, его живые и механические конечности слились в один сплошной клубок.
Под рёбрами, словно промышленная установка, работали полусинтетические лёгкие --- хы-ха, хы-ха, хы-ха...
Ветки и камни неслись прямо на нас.
Заяц взвизгивала, я в страхе прикрывал фасетки руками, но препятствия всегда в последний миг проходили в сантиметре от наших тел.
Когда мы прибыли на место, Заяц, как и полагается чувствительным млекопитающим, закономерно вывернула наружу содержимое желудка.

--- Да за что мне это, --- простонала она на четвереньках, вытирая губы.
Остро запахло кислотой.

--- Тебя понести? --- заботливо спросил Фонтанчик, протянув ей платочек.

--- Нет! --- вырвалось у Заяц вместе с очередным рвотным позывом.
--- Убери платок.
Вообще не подходи ко мне ближайшие сутки.
Сама дойду...

--- Это вместо благодарности, --- посетовал мне Фонтанчик.
--- Вот где она ещё на кани покатается?
Тебя понести, Небо?

--- Нет, благодарю, --- я на рефлексах отпрыгнул от друга, --- я, пожалуй, тоже пешочком...

\section{Носовая затычка}

--- Так, молекулярное лигирование нервов завершено, --- сказал Костёр, бросив взгляд на терминал. 
Затем, запустив руку в медицинский ларец, он протянул пациенту массивную носовую затычку.
--- Вставь её в нос плотно, она должна зачирикать.
Затем наклони голову вперёд, она должна сказать <<ку-ку>>.
Через три-четыре минуты все роботы будут у тебя в носовых капиллярах, начнётся небольшое кровотечение.
Надо их собрать.

--- Я скоро смогу работать?

--- Работать можешь, когда захочешь, --- сухо сказал Костёр.
--- А вот из лазарета выйдешь только послезавтра.
Надо дать нерву время, чтобы он оброс.

--- Благодарю тебя, Костёр.

--- Выздоравливай.
И держи хвост подальше от работающего принтера, а не то он будет короче ещё на один позвонок.

--- У нас тесновато, --- признался канин.
--- Один принтер на два отдела.
Мы просили собрать нам ещё один, но техники и так заняты по уши, они даже этому кожух так и не сделали...

--- Небо, скажи техникам, чтобы в первую очередь занялись безопасностью механизмов, --- грустно сказал Костёр, обернувшись ко мне.
--- Это уже третий хвост за десять дней.

\section{Звук дыхания}

Ещё с юности я обращал внимание на этот звук.
Если в комнате находился хотя бы один тси-млекопитающее и было достаточно тихо, то звук проступал во всём своём богатстве.
Этот звук издавала Заяц, издавал Баночка, издавал Фонтанчик --- кто-то повыше, кто-то пониже.
Этот звук издавал воздух, проходящий через десятки сантиметров воздухоносных путей.

Звук дыхания млекопитающих поразительно изменчив.
Если закрыть глаза руками, можно только по звуку дыхания определить очень многое --- рост, комплекцию, вид, пол, уровень мозговой активности, настроение, даже черты характера.
Разумеется, это присуще и нам тоже;
но дыхание апид не кажется мне настолько шумным.

Заяц постоянно спрашивала, как я угадываю настроение по её спине.
Я смеялся и ничего не отвечал.
У каждого должны быть свои маленькие тайны.

\section{Углы и время}

--- Царрокх используют старую систему измерения времени, основанную на шестидесятиричной системе счисления...

--- Может, шестидесятичетверичной?

--- Я не ошиблась.
Эта система применялась на Древней Земле в течение двадцати пяти тысяч лет --- около восьми тысяч оборотов.

--- Это огромный срок!

--- И это удивительно.
Та же система счисления используется при измерении времени и углов в хоргетах --- по крайней мере, у Безымянного я её тоже нашла.
Для времени схема --- шестьдесят на шестьдесят на двенадцать на два.
Для углов --- шестьдесят на шестьдесят на сто восемьдесят на два, она используется параллельно с обычной радианной системой.
В наших исторических хрониках сведения о целесообразности использования шестидесятиричной системы очень скудны.
Видимо, это дань какой-то очень древней традиции.

\section{Ларимары}

На её шее поблёскивало ожерелье из золота и бездонных как океан ларимаров.

--- Очень мило, --- восхитился Фонтанчик.
--- Как водичка у нас на побережье.
Это натуральные камни?

--- Рукав вчера нашла, когда разбирала привезённые породы, --- весело сказала Молодость.
--- Срез сделала --- красота несказанная.
Решила обточить.
А потом наплавила золота, заняла у техников медицинского стабитаниума --- и понеслось.
Сейчас половина геологов в её украшениях ходят.

\section{Нечего защищать}

--- Скажи, Баночка.
Почему первое, что мы сделали по прибытии на новую планету, --- это начали строить крепость?

--- Ты сомневаешься в том, что это правильно?

--- Нам нечего защищать.
На Тси-Ди были сады, заводы, огромные города.

--- Здесь есть мы, --- пожал плечами плант.
--- Этого недостаточно?

--- Да, может быть, ты прав, --- пробормотал я.

\section{Мировая Война}

Один из самых масштабных Театров, кстати, был посвящён войне.
Впоследствии он так и вошёл в историю --- Мировая Война.
Было создано огромное виртуальное пространство, имитирующее планету с суровым климатом --- пустыни, горы и ледники.
Это пространство населили астрономическим количеством NPC --- звери, птицы, насекомые, растения.
Почти все были смертельно опасны или причиняли игрокам большие неудобства.
Три противоборствующие стороны вели войну на уничтожение, используя как можно более негуманное оружие --- начиная от зазубренных мечей и заканчивая пулями со смещённым центром тяжести.
Система была анонимной --- лица и голос изменялись до неузнаваемости --- и с полным погружением --- все ощущения были абсолютно неотличимы от реальных.
Театр длился половину сезона, в нём приняли участие сто миллионов тси;
при этом четыре пятых игроков покинули виртуал ещё в первые десять дней, погибнув или просто не выдержав свалившихся на них испытаний.
Остальные, по их словам, дошли до конца <<из принципа>>.
В последней битве последние триста игроков, стоя посреди обледеневшего горного плато, заключили мир, закончив тем самым игру.

Результаты были плачевны.
Несколько десятков умерли из-за потрясения, шестидесяти пяти тысячам потребовалась помощь врачей и психологов.
Театр посчитали удавшимся --- по силе воздействия ему не было равных в истории.
Но тси вспоминают о нём очень неохотно.

\section{Возраст Безымянного}

--- Нашла что-нибудь интересное в памяти Безымянного?

--- Ничего особенного, относящегося к его создателям, --- покачала головой Кошка.
--- Очень хорошие комментарии к коду, настолько хорошие, что разберётся даже младший школьник... несколько пасхальных яиц, c десяток электронных книжек...

--- Пасхальные яйца?

--- Ага.
Оказывается, Безымянный умеет делать микрофейерверки.
Технология первых людей.
Он сам удивился.
Я ему сказала, как активировать систему, он её испытал над одним из поселений нгвсо.
Они были в восторге.

--- Я бы тоже посмотрел.
Надо его попросить.
А что за книжки?

--- Документация к хоргету, устаревшая Мирквудская классификация, учебники и протоколы по терраформированию...
Наши экологи оценили.
Только об этом и зудят.

--- Протоколы лучше наших?

--- Если учесть, что за два телльна истории Тси-Ди мы терраформировали половину планеты, нами же и превращённой в выжженную пустыню... то да, лучше наших.
Хотя бы проверены на практике.

--- Понятно.
Больше ничего?

--- Какая-то художественная литература, которую использовали для отладки языкового модуля --- видимо, забыли стереть после... --- рассеянно сказала подруга, думая о чём-то своём.
--- Представляет разве что археологический интерес.
Я всё залила в отдельный каталог, можешь глянуть...

--- Посмотрю.
Кстати, каких времён Мирквуд?

--- Судя по отсутствию некоторых важных правок, не менее чем пятитысячелетней давности, --- встрепенулась Кошка.
--- Забавный способ датировки, согласна.
Но это говорит не о времени создании хоргета, а скорее о времени разрыва коммуникаций между цивилизацией создателей Безымянного и Орденом Преисподней.

--- То есть минус триста лет.

--- Я бы даже доверительные интервалы подняла до пятисот, на всякий случай.

--- И что скажешь?
Чувствуется, что Безымянному четыре с половиной тысячи лет?

--- Он --- ребёнок в песочнице.

--- И песочница --- Планета Трёх Материков.

--- Тебя это не удивляет?

--- Я уже устала удивляться, --- растерянно призналась Кошка.
--- Я --- колонист, ступивший на поверхность, возможно, самой удивительной планеты во Вселенной, я общаюсь с хоргетом, и он даёт мне читать некоторые части своих исходников.
Слишком много удивительного.
Чтобы работать нормально, многое приходится принимать как должное.

\section{[2] Сигнал}

Полгода, которые запросил Костёр, давно истекли, а от врача не было ни слуху ни духу.
Я начал волноваться.

--- Он взял с собой связь? --- спросил я у техников побережья.

--- Мы сами волнуемся, --- сказал мне Синяя-Круглая-Плесень.
--- Квантовый передатчик сутки назад подал сигнал тревоги.
Мы выпустили поисковый дрон, но он не вернулся.
Сейчас за ним отправились дельфины.

\section{[2] Достойно фильма}

--- Я тебе сейчас расскажу такую историю, это достойно фильма, --- начал Костёр.

Костёр сбился с курса и нашёл новую группу островов --- целый архипелаг, формой напоминающий спираль.
На островах даже оказалось местное племя.
Костёр не был силён в переговорах, как Кошка, и вскоре ему пришлось бежать.

--- Обычные люди, такие же, как те царрокх --- но как же ловко они бегают по зарослям! --- вспоминал врач.

--- А чем ты им насолил?

--- Я слова им сказать не успел!
Они меня приняли за дичь!
Еле ушёл.
Лодку пришлось гнать.

Однако вскоре его ждала новая напасть.
Лодка попала в полосу цунами-шторма.

--- Идёт последняя волна, и я понимаю --- всё, конец, лодка перевернётся и я захлебнусь.
У меня даже не было времени написать сообщение вам.
Поэтому я дёрнул за кольцо жилета, заставил передатчик послать тревогу и прикусил язык.
Когда очнулся, шторм уже утих.
К счастью, океан в этих широтах тёплый, как молоко, и я не погиб от переохлаждения.
Днём спал, ночью шевелил руками-ногами не переставая, жевал всякую чушь --- водоросли, улиток, мелкую рыбу, креветок --- только бы не застыть.
Дрон прилетел в тот момент, когда я уже начал волноваться.
Я заарканил его тросом и повёл на запад, покуда батарей хватало.
К вам бедняжка, конечно, не вернулся, я пустил его на детали.
И снова повезло --- меня вынесло на остров с деревьями.
Собственно, последние полгода я строил новое судно из подручных материалов.

--- Дельфины тебя не нашли, --- сказал я.

--- Конечно! --- пожал плечами Костёр.
--- Я же говорю, я на острове был!

--- Жалко, что так вышло.

--- А я не жалею.
Да, мне пришлось пережить многое, сломать несколько костей, получить стрелу в плечо, надорвать спину во время постройки корабля.
Но зато я отвлёкся от грустных мыслей.
Сейчас Катаклизм кажется мне таким далёким, словно я читал о нём в исторической хронике, а не испытал на собственной шкуре.

\section{Рождение выбора}

Будь у меня возможность выбрать ещё раз, я сделал бы то же самое.

Да и бесполезно возвращаться в тот же момент.
Выбор --- настоящий выбор --- был сделан задолго до него.
Может, за год или за восемь лет.
А может, я всю жизнь шёл к тому, чтобы поступить так, как я поступил.

\section{Придуманные враги}

Я с трудом понимаю войну.
Нападать или защищать --- не имеет значения, результат всегда один.
В прежние времена манипулировали обещаниями благ --- пищи, красивой одежды, почёта, секса.
Но ответ на всё один, и этот ответ тси знают с детства --- никто не поселится в выжженных землях, все любят цветущие сады.
Вырасти в своих землях, доме, теле и мозге собственный сад, знай его и ухаживай за ним.
Тогда тебе не будет нужды брать что-то силой.
Да и опыт, как по мне, будет гораздо более полезным в дальней перспективе.
Куда приложить опыт войны, если закончатся враги?
Разве что придумать новых.

\section{[1] Мхи (ЖС)}

Я смотрел на мох.
Согласно данным первых людей, мхи были одними из первых наземных растений.
Как и миллиард лет назад, они --- такие крохотные, такие сильные, бесконечно терпеливые --- кучками цепляются за щёлочки и ложбинки в камне, стараясь устоять перед ветром и сберечь драгоценную влагу.
Мхи медленно расползаются, умирают и гниют, создавая миллиметр драгоценной почвы --- почвы, на которую смогут сесть травянистые.

Я обернулся на поселение тси.
Эти голые камни, эти стоящие кучками домики и палатки... да, именно.
Как древние мхи, мы сообща пытаемся выжить и сохранить всё самое ценное --- в надежде, что идущим за нами будет легче.

\section{[1] Цитра Ветра}

Безымянный долго спрашивал, кто может сделать для него музыкальный инструмент.
Наконец одна из тси, инженер-технолог по имени Ветер-Дующий-Ниоткуда, решилась.
Как-то утром она вызвала бога и показала ему сделанный из тонких древесных пластинок струнный инструмент.

--- Не могла заснуть ночью, --- пояснила она.
--- Идеям плевать на твой график.
Древесину сложно довести до состояния, когда она начинает хорошо резонировать, но я справилась.

--- Как тебя зовут? --- спросил Безымянный, осмотрев инструмент.

--- Ветер, --- осклабилась женщина.
--- Не стоит благодарности, играй.

--- Я запомню это имя, Ветер, --- сказал бог.
--- Я запомню это имя.

Отныне инструмент Безымянного бежал в библиотеке Стального Дракона, в специально отведённом для этого уголке.

\section{[1] В честь планеты}

--- Закрытая-Колба-с-Жизнью? --- засмеялся Мак.
--- Так ты назвался...

---  ... в честь планеты Тси, да, --- буркнул Баночка.
Эта аллюзия, восходящая к кому-то из поэтов романтических времён, успела ему приесться.
--- Надеюсь, ты цитировать его не будешь?

Мак явно намеревался и даже набрал для этого побольше воздуху в грудь, но вовремя сообразил, что делать этого не стоит.
В итоге биолог просто чуть громче, чем следовало, сказал <<Очень приятно познакомиться>> и ещё раз пожал Баночке руку.

\section{[1] Кон-Тики}

Я вдруг вспомнил далёкие школьные годы и экскурсию к котловине Кон-Тики --- месту, где был найден другой корабль.
Тот самый корабль, на котором предки млекопитающих-тси прибыли в нашу звёздную систему.

Огромный грот и кусок обшивки с надписью, вплавившийся в тысячелетний гранит.
Мои руки и ногочелюсти тряслись.
Мне казалось, что именно на меня величественная пещера подействовала гораздо сильнее, чем на прочих.
Знал ли я, насколько сильно будут похожи судьбы капитана Гуштефа и моя собственная судьба?
Возможно ли это?..

\begin{quote}
<<Данное направление было самым протяжённым в пространстве за всю историю космических путешествий древних людей.
Оно включало пять планет, пригодность которых была недостоверной\ldots>>
\end{quote}

\begin{quote}
<<... Априла 18 года 25712 Кон-Тики покинул безжизненную систему Иштар и вылетел в безумный, отчаянный полёт, имея топлива на один последний цикл разгон-торможение.
За 300 часов до этого три древних планеты --- Земля, Марс и Диана --- замолчали навсегда.
Корабль отчаянно пытался связаться с ними, но безуспешно, и экипаж, поняв, что проблема не в устройстве связи, принял мужественное решение лететь до конца пути --- двойной планеты на окраине Галактики, планеты под кодовым названием <<Тсиди>>, что означало <<вдохновение>> на одном из языков.
Они так и не узнали, что случилось с их домом, какова суть постигшего эти планеты бедствия.
Возродилась ли цивилизация?
Возможно.
Но возродись она и через пятьдесят лет, и через пятьсот --- о Кон-Тики, который жаждал весточки из дома, никто бы не вспомнил>>.
\end{quote}

--- Я хочу, чтобы вы, глядя на эту доску, крепко уяснили одну вещь, --- сказал в заключение учитель.
--- Дома вы живёте в уюте, проводите время в работе и развлечениях.
Тси-Ди --- это наш дом.
Ценой невероятных усилий тси сделали поверхность одной планеты и большую часть поверхности второй большим уютным домом.
Однако жизнь коротка.
Что для вас и для меня жизнь Гуштефа Морре-Чилда?
Всего лишь кусок металла.
Мало кто осознаёт, что мы оказались здесь благодаря горстке живых существ, которые говорили на другом языке, которые прошли труднопреодолимое даже для демонов расстояние.
В нас до сих пор живут их гены, а прочие, модифицированные гены помнят умелые руки их потомков.
Каждый аспект технологии, который мы можем найти на двух планетах нашей системы, несёт на себе отпечаток мыслей тех существ.
Подумайте об этом и скажите мне, прожившему триста лет, что же такое наша жизнь?
Я уже отчаялся найти ответ на этот кажущийся простым вопрос.

\section{Тревожная женщина}

--- Не следят за своим здоровьем, --- бурчал Костёр.
--- Я же тоже не железный...
Вы всегда можете прийти ко мне, и я вас полечу.
А кто меня лечить будет?

--- Возьми отпуск, --- сказал я.
--- Хотя бы на несколько дней.
Слетай на тот берег.
Там красиво.

--- И остаться наедине со своими мыслями? --- хмыкнул Костёр.
--- Нет уж.

--- Кого ты потерял?

Костёр потряс головой и надолго замолчал.
Я уже было думал, что он не хочет говорить, но...

--- Женщину и детей.
Не нужно, --- Костёр мягко высвободил руку, когда я попытался её погладить.
--- Я уже успел забыть за всеми этими заботами, как они выглядят.
Но одна мысль не даёт мне покоя, и боюсь, однажды она меня убьёт.

Костёр подумал.

--- Та женщина страдала высокой тревожностью.
Дупликация гена одного из нейромедиаторов, достаточно редкая мутация.
Обнаружили её уже в зрелом возрасте.
Женщина приспособилась жить с ней --- работала только по ночам и в полной тишине.
Зато в работе --- а была она диагностом систем --- ей не было равных.
Замечала такие нюансы, мимо которых проходили все прочие.
Познакомились мы в больнице.
Меня потрясла её нежность и чуткость --- такого человека редко можно встретить.
Она же говорила, что рядом со мной ей очень спокойно.

Врач вздохнул.

--- Меня не печалит её смерть --- здесь, среди постоянных тревог, она бы просто не выжила.
Но едва представлю всю глубину того ужаса, который испытало это нежное сердечко перед смертью --- и всё валится из рук.
Поэтому ты, Небо, как хочешь, а меня в отпуск не отправляй.
Буду работать, пока не сдохну.

Я поднялся.

--- Я сообщу всем молодым тси, что тебе нужна ласка.
Я думаю, кое-кто придёт.

--- Пошёл вон.
И если ко мне ещё кто-то придёт с нежностями, я их выкину.

--- Первый раз --- разумеется.

--- И второй тоже.

--- Отлично.
Значит, женщины и мужчины будут делать не менее трёх попыток, --- заключил я.
Костёр ухмыльнулся.

--- Мужчин не нужно, не привлекают.

--- Спокойного дня, Костёр.

--- Спокойного и тебе, Небо.

\section{[2] Сон}

--- Комарик, пожалуйста, живи!
Живи! --- плакал я.

Комар погладил меня слабой рукой по плечу.

--- Я всегда буду с тобой.
Я буду жить у тебя в голове.
А ещё у нас остались дети, Небо.

--- Как бы я хотел, чтобы у нас были дети.

--- Они у нас есть.

Я гладил его по голове и плакал.

Прибывала вода, расцвеченная красной, чёрной, зелёной кровью...
Мимо величественно проплыла рыба, задев меня длинными мягкими усами.
Её глаза опалесцировали, плавник гордым парусом торчал из воды, рот с нежными красными губками заглатывал кровавую жижу.
Откуда здесь, в Зале Достижений, эта рыба?
Я поднял глаза... и увидел яркое солнце меж двух отвесных стен.
Знакомых, изъеденных ветрами стен.

Котловина Кон-Тики.

Вода прибывала всё сильнее.
Комар начал захлёбываться;
он инстинктивно приподнимал брюшко, но вода всё равно затекала в дыхальца, выбивая из агонизирующего тела столбики весёлых пузырьков.
Я закричал, поняв, что у меня не хватает сил, чтобы даже приподнять любимого над водным потоком.

Пузырьки.
Вода клокотала, чёрная, как ночь, с металлическим оттенком, словно холодная ртуть.
Я снова закричал --- отчаянным, ужасным криком.
Комар задыхался на моих руках.

--- Небо, молчи и слушай.
Девяносто пять.
Девяносто пять нанометров.

--- Что?
Что ты говоришь? --- плакал я.

--- Молчи и слушай, Небо!

--- Да что он несёт? --- закричал я.
--- Баночка, помоги мне.

Я поднял глаза на планта.
Он стоял по пояс в воде и смотрел на меня грустными глазами из-под татуированных век.

--- Ничего не сделать, командир.

--- Просто! Помоги! Его! Поднять! Над! Водой! --- заорал я.

--- Это не вода, Небо, очнись.
Это смерть.
Ты сам уже по пояс в ней.

--- Молчи! И слушай! --- скрежетал Комар.
Над водой осталось только лицо, но он пытался что-то сказать.
--- Это очень важно...
Девяносто пять, Небо, девяносто пять...

\asterism

Ночник светил слабым, успокаивающим светом.
Я несколько раз тянулся рукой к импланту, чтобы сделать инъекцию успокоительного.
И отдёргивал руку.

Первой мыслью было бежать к Костру;
разумеется, это был \emph{тот самый} сон, о которых врач меня предупреждал.
Но ни один не вгонял меня в такой ужас.

<<Молчи и слушай>>.
Да, верно, Костёр в кругосветном.
Я уже пожалел, что отпустил его.

Я лёг и попытался не двигаться.
Это была любимая фраза Комара, совет на любой случай жизни.
Молчи и слушай.

За открытым иллюминатором царила тихая ночная жизнь.
Пели птицы, сновали млекопитающие, шуршали насекомые.
Над всем этим царил ровный гул реактора.

Нет, что-то точно не так.

В коридоре меня встретил Баночка и замер, словно увидев призрака.
Потом облегчённо выдохнул.

--- Ты меня напугал.

Плант вдруг вытащил сверток и принялся набивать травами курительное приспособление.

--- Ты куришь? --- удивился я.

--- Неа, --- виновато улыбнулся плант.
--- Просто что-то неспокойно, и я решил попробовать.
Пирожок говорит --- успокаивает.

--- Ты был у неё?

--- Да.
Ей тоже что-то не спится.

--- Что, неужели и ей котловина Кон-Тики не даёт заснуть?

Баночка промолчал и сделал глубокую затяжку.
Затем закашлялся.

--- Что мы упускаем, Баночка? --- задумчиво спросил я.
--- Мы ведь упускаем что-то важное.

--- <<Молчи и слушай>>, --- буркнул плант.
--- <<Девяносто пять нанометров>>.
Мы действительно что-то упускаем.
Но что?

Я вдруг понял, что больше ничто в этом мире не способно меня удивить.

--- Комар не успел сказать.

--- Да при чём тут Комар, --- вдруг раздражённо хмыкнул плант.
--- Забудь про Комара, забудь про котловину Кон-Тики, и про смерть свою забудь, они в далёком прошлом.
Научись уже отделять актуальные проблемы от проекций чьей-то памяти.

--- Моей памяти!

--- Ты уверен, что это твоя, а не память давно умершего, чужого тебе существа, оказавшегося в похожей ситуации?

Я промолчал.
Баночка всегда умел задавать хорошие вопросы.

--- <<Молчи и слушай>>.
Эта мысль летает в воздухе прямо сейчас.
<<Девяносто пять нанометров>>.
Что настолько важно?
Что измеряется в нанометрах?
И это число, как будто я его где-то...

--- Кое-что не было проекцией чужой памяти, --- перебил я его.
--- Молчи и слушай.

Баночка виновато кивнул.
Пели птицы, шуршали млекопитающие, шелестела трава.
Гул реактора, ровная, идеально ровная гармоника.
Пять секунд.
Сто пять секунд.
Пятьсот тридцать одна секунда...

И вдруг гармоника сделала крохотный скачок.
Мы с Баночкой уставились друг на друга.

--- Что это было?

--- Молчи и слушай! --- прошипел плант.
В его глазах застыл животный ужас.

Пять секунд.
Сто пять секунд.
Пятьсот тридцать одна секунда в томительном напряжении.
Трубка в руках Баночки давно испустила последний дымок.
По лицу планта текли капли пота;
в глазах плясала надежда.
Может, показалось?
Мы спросонья, переволновались, конечно, нам могло показаться что угодно...

И снова крохотный, едва уловимый скачок гула реактора.

Баночка кинулся к лифту.

--- Поднимай всех, командир, --- бросил он через плечо.
--- Мы по колено в воде.

\section{[2] Молитвы}

--- Мне кажется, что я скоро умру, --- признался я.

Баночка вдруг схватил меня за руку.

--- Небо, пойми одну вещь.
Наша судьба не определена.

--- Но сны...

--- В бездну сны! --- рявкнул плант.
--- Ты вечно строил из себя жертву.
Ты жертвовал всем --- своими интересами, своей жизнью, всем, что имел.
Ты думал, что тебе уготована такая судьба, и шёл навстречу этой судьбе.

--- Костёр...

--- Костёр сумасшедший!
Ты не задумывался, почему я с ним не общаюсь?
Мне совершенно неважно, каким образом я вижу сны о далёком прошлом.
Да, я давно знаю, что это правда, с самого детства.
Я чувствовал, как вонзаю штык в чужое сердце, до мельчайших подробностей.
Я собирал во сне взрывчатку, ещё не имея понятия о началах химии, пользовался противогазом, чистил винтовку.
Я до сих пор помню вот это!

Баночка сделал молниеносное движение руками, и я похолодел.
В моей голове ясно щёлкнул передёргиваемый затвор.

--- Я умирал, занимался сексом, бредил в лихорадке, ещё не зная этих вещей в реальности.
Но я никогда, слышишь, никогда не уходил в мистицизм.
А Костра эти сны свели с ума!
Его свело с ума одиночество, потому что ты одинок ровно в той мере, в которой погружён в иллюзию!
Ты хочешь того же?

--- Ты не прав.
Одиночество --- это погружённость в иллюзию, отличную от иллюзии прочих.

--- Давай не будем устраивать низкопробный философский диспут на серьёзную научную тему.
Мы не дикари.
И помнится, мы вместе сдавали диктиологию.

Я промолчал.

--- Небо, я тебя прошу --- останься с нами.
Останься в реальном мире.
Если тебе одиноко, чаще бывай с народом, пройди курс гормонов, только не уходи в иллюзию.
Нет и не было никакой судьбы, --- Баночка отпустил мою руку и крепко обнял.
--- Я это знал всегда.
И, заметь, почти всегда доживал до старости.
Потому что мной двигало желание жить, а не обречённое принятие чьих-то иллюзий.

\section{[2] Каскад тревоги}

--- А как вы меня нашли? --- удивился я.

--- Ты не поверишь, --- улыбнулась Кошка.
--- Листик и Баночка засекли каскад тревоги.
Я бы в жизни не додумалась до такого.
Объяснять долго, суть: удар шлюпки о воду испугал рыб.
К испуганной рыбе слетелись чайки, за чайками последовали крупные хищники.
Волны тревоги распространялись вокруг эпицентра с помощью сигналов различных существ.

--- А кто засёк эти волны?

--- Мой приёмник, --- Баночка снял с плеча довольно каркающую Нейросеть.
Судя по всему, ворону только что похвалили и накормили чем-то вкусным.

--- Листик случайно услышала тревожное карканье Нейросети и сразу сообразила, что это может быть связано с перебоем связи.
Баночка после долгих усилий наконец понял, что хотела втолковать ворона, и мы послали в море дельфинов.

Я погладил птицу пальцем.
Она игриво ухватила меня сильным клювом.

Никогда не думал, что буду обязан жизнью вороне.

\section{[2] Интуитивный мистицизм}

Кошка размахивала руками и поставленным голосом читала какие-то тексты, время от времени сверяясь с компьютером.

Я заворожённо наблюдал за ней.
Простой комбинезон вдруг сменился в моём воображении причудливой одеждой с перьями, руки и лицо покрылись охряным узором, глаза остекленели от наркотических веществ.
Я понимал --- Кошка не просто изучает ритуал.
Она пытается найти свой собственный путь, путь учёного-тси, во тьме древнего, интуитивного мистицизма.

\section{Животный страх}

Меня очень удивили местные животные.
На Тси-Ди обращение с дикими животными занимало несколько лет обучения.
Детям объясняли про рефлексы низших животных и про сигнальную систему высших.
Впрочем, за десятки тысяч лет мирного сосуществования дикие животные настолько привыкли к тси, что воспринимали их как камни или деревья.
Они иногда кушали из рук, позволяли себя гладить или приходили к жилищам полечиться.
А со знанием сигнальной системы можно было довольно легко убедить не слишком голодного волка, что ты плохая добыча и потенциальный друг.

Цветущий-Мак-Под-Кустами даже как-то рассказал, что в детстве одну зиму жил в стае койотов.
Самым сложным, по его словам, оказалось приучить лохматых к человеческому запаху, а ещё объяснить, что у костра можно греться.
Когда волки смекнули, что к чему, то сами стали намекать, что пора бы уже <<сделать горячую вонючку>> --- зима выдалась суровой.
Мак даже вспомнил, что в благодарность за что-то молодая самка койота принесла ему свежезадушенного кролика.

Здесь же животные не показываются взгляду.
Почуяв человека, млекопитающие обычно удирают со всех ног.
От них пахнет смертельным страхом.
К прочим тси также относятся настороженно.

Я рассказал свои ощущения Баночке.
Он в ответ поделился опасениями, что нам придётся жить охотой и животноводством, если с кольцевой теплицей что-то случится.

\section{Дельфины}

--- Я не понимаю дельфинов, --- заявила Листик.
--- Мы все переживаем, думаем, как лучше, а дельфинов это как будто не касается.
Главное, чтобы был океан, и дельфины счастливы.

--- А о чём им беспокоиться? --- заметил Мак.
--- Пищи у них достаточно, врагов нет.
Гора-Окутанная-Дымом сказал, что он с отрядом проплыл почти до самых полюсов.
Да, там прохладно, им пришлось надеть термочехлы на плавники, но в целом дельфины могут жить везде.

--- А акулы в проливе Скар? --- спросил я.

--- Акулы могут утащить разве что дельфинят.
Или малыша вроде тебя, --- усмехнулся Мак.
--- Гора сказал, что акулам и прочим морским зверюшкам очень не понравились сонарные сигналы, которыми он и его отряд <<прокрикивали>> дно.
Я посоветовал дельфинам провести на этот счёт исследование --- дыма без огня не бывает, возможно, что какой-то опасный природный феномен тоже <<говорит>> на этих частотах.
Сейчас команда Капельки этим занялась --- узнают, каких именно частот боятся зверюшки и в какой местности.
Так мы хотя бы будем знать, что искать.

--- А что ещё нашли дельфины? --- поинтересовалась Заяц.

--- Вообще они много кого привезли.
Рыбок, моллюсков, водоросли, даже щупальце кого-то, отдалённо напоминающего гигантского осьминога.
Сказали, что труп дрейфовал недалеко от кораллового рифа, очень повезло, что его ещё не успели обкусать рыбы.
Гора составил очень подробный видеоотчёт по океану южнее девятой параллели.
Больше похоже, правда, на юмористический фильм, над комментариями можно ухохотаться.
<<Смотри, какая странная красная рыбка!>>
--- <<Оставь её, хвостомордый, у неё депрессия>>.
--- <<С чего ты взял?>>
--- <<Если бы я носил шкурку наизнанку, у меня тоже была бы депрессия>>, --- Мак захохотал.

--- Я же говорю, что они относятся к делу недостаточно серьёзно, --- укоризненно сказала Листик.

\section{[2] Рука}

Из наркотической полудрёмы меня вырвал звук взрыва.
Как я узнал впоследствии, техники попытались применить не по назначению полевой генератор.
В палатку проковыляла Заяц, крича на ходу медицинский код травмы и волоча на себе тяжёлое, залитое кровью тело Звенеть-Хрустальными-Клыками.
Правая рука канина превратилась в месиво.

Ещё один несчастный случай.

--- Один-три-а-ноль-эф!
Один-три-а-ноль-эф!

Костёр выбежал как ошпаренный, толкая перед собой летающий ларец с числом <<13>>.
Заяц без сил рухнула на колени.

--- Отойди, --- Костёр грубо отпихнул Заяц и, выхватив пистолет, прижал его к груди Зубика.
На случай выхода импланта из строя у всех тси-млекопитающих в грудине стоял внутрикостный катетер.

Заяц, придя в себя, бросилась помогать.
Вскоре на культю была наложена искусственная соединительная ткань, потерю крови восполнили кровозаменяющей жидкостью, и канина положили в реанимационную капсулу рядом со мной.

--- Как остальные? --- это был первый вопрос, который задал Зубик по пробуждении.

--- С ними всё в порядке, --- сухо ответил Костёр, проверяя в последний раз повязку.
Отсутствие привычных инструментов, материалов и лекарств сделало его раздражительным --- и это при пресловутой стрессоустойчивости врачей.

Зубик посмотрел на культю и поморщился.

--- Я теперь навсегда без руки?

--- Мы потеряли кольцевую теплицу, --- чуть не плача, ответила Заяц.
--- Я не знаю, возможно ли вырастить руку с теми материалами, которые...

--- Заяц, перестань, --- оборвал её Костёр.
--- Хватит брать на себя вину за целый мир.
Теплица сгорела, в реакторе была трещина, роботы были перепрограммированы.
Абсолютно все обстоятельства были против нас.
Фонтанчик погиб, как герой, пострадавших могло быть больше.

Заяц расплакалась.
Костёр бросил на неё виноватый взгляд, поправил мне одеяло и уже чуть мягче добавил:

--- Я восстановлю тебе руку, Зубик.
Но не за один год.
Подождёшь?

--- Куда мне деваться, --- попытался пошутить канин.

--- Я соберу тебе механическую, --- сказала Заяц, вытирая слёзы.
--- Уж это-то мы сможем.

--- Вы лучшие друзья, ребята.

--- Для начала поправься, --- остановил его Костёр и, мимоходом взглянув на меня, вышел на воздух.
Вскоре откуда-то потянуло странным удушливым дымом --- кое-кто из тси пристрастился к местным травам с седативными алкалоидами, и похоже, что Костёр тоже.
Я нечаянно громко щёлкнул зубами, и дым тут же рассеялся, сменившись на слабое гудение вытяжки.

Заяц сидела у постели Зубика.
Я знал, что она сейчас думала о погибшем любовнике и о едва живом мне.
Ей страшно хотелось на воздух, подальше от этой наполненной смертью палатки, но её останавливал долг дружбы.
Зубик, похоже, тоже понял это.

--- Иди, Зайчик.
Я за ним послежу.

--- Иди, --- добавил я.
--- Я пока в хорошем настроении.

Заяц с натянутой горькой улыбкой по очереди кивнула нам и тут же вышла.

\chapter{Безумная война}

\section{Казнь}

--- Кутрап, --- тихо сказал Кхарас и закрыл лицо руками.

--- Ну так на крышу его, --- пожала плечами Ликхэ.

--- Не хочет, --- ответил боевой вождь.

На моей памяти такое было впервые.
Почти все кутрапы выбирали жертвоприношение.
Почти все.

--- Давайте я, что ли, --- Кхохо угрюмо отпила из чаши и подхватила серебряную саблю.
--- Наблюдатели из числа Советов уже там?

--- Ага, --- кивнул Кхарас.
--- Держи свитки: решение Советов, подтверждение личности...

--- Давай всё, --- оборвала его Кхохо, просмотрев пергаменты.
--- По дороге разберусь...

Все тут же побросали еду и пошли за воительницей, чтобы ничего не пропустить.

Кутрапом оказалась совсем юная девушка дождей восемнадцати.
Несмотря на это, за ней числилось Разрушение и более двадцати актов Насилия.
Получив метку, она скиталась по городам, добывая себе пропитание воровством.

Совет Цеха в полном составе ждал у храмового гонга.
Кхотлам не пришла, но прислала вместо себя Эрхэ.
Кхохо молча поманила их на храмовые земли.

--- Кхохо, душа моя, на площади надлежит казнить, --- тихо буркнул Эрликх.

--- Это чтобы развлечь торговцев или чтобы испортить им пищеварение? --- парировала Кхохо.
--- Хватит болтать.
Развяжите ей руки.

--- Зачем? --- удивилась старшина Цеха.

--- Ты здесь Храм или я? --- веско ответила Кхохо.
--- Выполняй.

Девушке развязали руки.
Она с наслаждением потёрла синие кисти.
Спустя пару секхар перед ней шлёпнулась фаланга из арсенала.

--- Вынимай, --- процедила Кхохо.
--- Я не знаю, как ты жила.
Вряд ли достойно.
Но я даю тебе возможность умереть с оружием в руках.

--- Опять самодеятельность, --- забурчали члены Совета.

--- Кхохо, лесные духи, перестань!

--- Если она меня убьёт --- может быть свободна, --- ухмыльнулась Кхохо.

Все дружно застонали и закатили глаза.

--- Это не по закону! --- заявила старшина.

--- Кто-то против? --- осведомилась Кхохо.
--- Кто-то хочет, чтобы я поставила её на колени посреди торговой площади и зарубила как курицу?

Совет забормотал.
Все озирались, пожимая плечами.
Перспектива увидеть бой вместо бойни многих явно воодушевила.
В глазах девушки блеснула надежда.
Она смотрела на Кхохо почти с обожанием.

--- Бери оружие, --- Кхохо снова кивнула на лежащий в траве клинок.
Девушка подхватила его и встала в боевую позицию.

Бой длился ровно десять секхар.
Парирование, обход, хлёсткий удар --- и обезглавленная девушка упала в траву.

--- Похороните её пока здесь, --- Кхохо ткнула окровавленной саблей в заросшую грядку под толстым саговым деревом.
--- Знаки проставьте, потом вызовем людей на изъятие кости.
И да, вы все прекрасно знаете, что казнь прошла по закону, верно?

Совет одобрительно забормотал.
Кхохо ухмыльнулась и пошла обратно в храм.

--- А если бы она тебя убила? --- тихо спросил я.

--- Духам виднее, --- пожала плечами Кхохо и сунула мне саблю.
--- Будь другом, вытри её от крови, пожалуйста.
Я умаслила свою совесть как могла, но мне всё равно немного некомфортно.
Почему мы, воины, вообще должны этим заниматься?..

\section{Воинская присказка}

--- На Западе говорят <<воинская присказка>>.
Мы не любим слово <<клятва>>.

--- Почему?

--- Потому что от клятвы ничего хорошего.
Она заставляет дающего лгать и испытывать стыд, а принимающего --- принуждать делать то, что человек не всегда хочет и может.

\section{Граффити}

Едва зайдя в проулок, я обнаружил на стене свежее граффити:

<<Храм лжёт>>

Я вздрогнул.
По спине поползли мурашки.
Рядом недовольный крестьянин уже замешивал известь с охрой, чтобы закрасить надпись.

--- Мне это надоело, --- бурчал он.
--- Жопы себе покрасьте.
Лжёт Храм или нет, при чём тут моё жилище?

Ещё за одним углом была надпись:

<<Храм --- убийцы>>

Мне резко стало неуютно.
Захотелось побыстрее пойти в храм, на торговую площадь, или на крайний случай выбраться на освещённую широкую улицу.

Вдруг я увидел знакомую фигуру.
Она коряво выводила надпись на свежепокрашенной стене.
Увидев меня, женщина выронила кисть и попыталась запрыгнуть на крышу.
Непрочный карниз треснул, и она плюхнулась на землю, пролив краску на себя.
Послышалось неразборчивое витиеватое ругательство.

--- Кхохо!
Это ты пишешь всю эту чушь на стенах?

--- Это не я, --- быстро ответила Кхохо.
--- В смысле, вот это, то что здесь, конечно, я --- глупо отрицать.
Всё остальное --- не я.

Я подошел поближе и вгляделся.

<<Ликхмас --- Первый жрец>>.

--- Решила тебя поддержать, --- извиняющимся тоном сказала Кхохо.
--- А то тут шепотки пошли, что и ты в этом замешан.
Нас-то уже давно поносят на чем свёт стоит...

--- Кхохо, сейчас же прекрати пачкать мной стены, --- строго сказал я.

--- Как скажешь, Первый жрец, --- буркнула воительница и, подтянувшись, исчезла в темноте на крыше.

Должность Первого жреца была чисто административной и во многом декоративной.
Но для Кхохо, видимо, значение было иным.

Кхохо не была одинока.
По пути я насчитал ещё восемь надписей с моим именем.
Одна явно принадлежала Ситрису, так же полно ошибок, ещё пять незнакомых.
Одна надпись была начертана традиционным начертанием, как в Легенде об обретении.
Ещё одна с орфографической ошибкой и эмоглифом, заставившим меня покраснеть, причём надпись точно не принадлежала ни Чханэ, ни Ликхэ.
И ещё одна... Кхотлам?

У меня пошла кругом голова.
Даже во сне с похмелья я не мог увидеть, как кормилица пачкает стены по ночам.
Но почерк говорил сам за себя, хоть она и нарочно исказила его.

Надпись была лаконичной --- <<Ликхмас>>.
Больше ничего.

\section{Бусы}

Диалекты языка сели отличаются от деревни к деревне, от хутора к хутору.
Порой жители двух глухих хуторов на расстоянии двадцати кхене понимали друг друга через слово.
У горожан словарный запас намного больше, потому что приходилось общаться с жителями всех окресных хуторов.
Самым богатым словарным запасом обладают, разумеется, купцы.
А рекордсменом по числу вариаций в языке сели является слово <<бусы>> --- их называют <<бобы>>, <<хоровод>>, <<позвонки>>, <<цепка>>, <<трещи-нить>>, <<жемчужница>>, <<считалка>>, <<медявка>>, <<три-счёт>> --- всего около пятидесяти.
В хуторах Тхитрона распространено слово <<ожерелик>>, родня простенькие бусы с более дорогим и сложным в изготовлении ожерельем.

\section{Серебряный волос}

Ликхэ села рядом и начала вычёсывать у меня волосы.

--- Чего это ты? --- удивилась Чханэ.

--- Серебряный волос ищет, --- хихикнула Кхохо.

Ликхэ порозовела и тут же переключилась с моих волос на локоны Хитрам.

--- Серебряный волос? --- переспросила подруга.

--- Ага.
Цветом как серебро, выдернуть нельзя, как ни тяни.
Намотаешь на палец, прошепчешь <<Люблю>> --- и твой человек до конца жизни, --- нараспев рассказала Кхохо.

--- Чушь городишь всякую, --- пробормотала Ликхэ.

Чханэ немедленно села рядом со мной и начала перебирать мне волосы.

--- Я и так твой, --- намекнул я.

--- Я знаю, --- сказала подруга, продолжая поиски.
--- Просто интересно, правда это или нет про волос.
О, кажется, нашла.

--- Ай!

--- Нет, точно не он.
Просто седой.
Ты седеешь, Лис, ты в курсе?

\section{Болезнь Кхохо}

Однажды Кхохо заболела.
Болела она каждый год, ровно три дня, в одно и то же время --- в самую жару, посреди летней страды.
Ей выделяли келью на верхнем этаже, с ней обычно оставался сидеть кто-то из воинов, но в тот раз время выдалось суматошное --- срочный вызов, нападение, какие-то нелады с крестьянами --- и про Кхохо просто забыли.

Я почуял неладное, в очередной раз проходя мимо её кельи.
Дверь была приоткрыта, словно там кто-то живёт, но на пороге лежала пыль, как если бы туда не заходили уже много дней.

Кхохо лежала, отвернувшись к стене и закрыв глаза.
Воздух пропитался запахом немытого тела.
В келье было душно, но она вжималась в одеяло, словно мёрзла.

Я потряс её за плечо.

--- Ты в порядке?

--- Да, --- ответил мне сиплый шёпот спустя десяток секхар.

--- Повернись и посмотри на меня, --- скомандовал я.
Стандартная фраза, которую произносит врач, приходя в дом заболевшего.

Кхохо не двинулась с места.

Я откинул одеяло, перевернул её на спину и ужаснулся синюшным пятнам на руках и груди.
Такие пятна появлялись на лежалых трупах.

--- Давно ты лежишь без движения?

--- Не знаю, --- тем же безжизненным шёпотом ответила Кхохо.

Я провёл пальцами по её губам и дёснам.
Они были сухими.
Графин стоял на прикроватном столике, заполненный водой до самого верха.

---  ...Кхохо, ну пей, ну пожалуйста! --- вода выливалась на подушку, не достигнув глотки.
Я набрал воду в рот и, приникнув губами к растрескавшимся губам Кхохо, с силой вытолкнул воду.
Она проглотила и слабо кашлянула.

--- Ты хороший, --- прошептала она. --- Ты такой хороший...

Кашу и вяленое мясо тоже пришлось жевать мне.
Затем наступил черёд клизмы --- Кхохо попыталась протестовать (<<Ещё у меня в жопе только козьего рога не было, для полной коллекции>>), но в конце концов сдалась.
Храм, как назло, вымер --- в раскалившемся каменном здании, кроме меня и Кхохо, не было ни души.

К вечеру появились воины.
Я едва успел сказать слово <<Кхохо>>, как все, переглянувшись и хором выругавшись, побежали наверх.

--- Ликхмас, можешь за ней походить? --- попросил меня потом Ситрис.
--- Я предупредил Трукхвала.
Посидишь, послушаешь бред, который она несёт, покормишь, поперекладываешь её с бока на бок.
Два дня всего, ну или чуть побольше, ей совсем худо стало от нашей <<заботы>>...

Поев пару раз, Кхохо начала говорить.
Говорила она много, порой невнятно, и на её лице застыла неодолимая, смертельная скорбь.

---  ...Я очень сильно устала, а она меня выхаживала.
Никто на меня такими любящими глазами не смотрел.
Мне даже трахнуть её не хотелось.
А потом она раз --- и всё, и я совсем одна осталась...

Кхохо начала сухо всхлипывать.

--- ...Я всё для него делала, понимаешь, всё --- служила как оцелот на верёвочке.
А он меня раз ногой под жопу --- и на улицу выкинул.
И я никто, и звать меня никак, и все мои труды пропали...

На следующий день я принёс свежую постель и набрал ванну.

--- Ликхмас, нет, ни за что!

--- Кхохо, это просто вода!
Ты кошка --- воды бояться?
Тихо-тихо, спокойно, ты мне шею сломаешь.
Просто расслабься и дай перенести тебя в душ.
Отцепись, давай, руки разожми, отлично, всё хорошо, я тебя не уроню...

Я взял из дома кормилицы гребень, какие-то масла, ухаживающие средства.
Пока Кхохо хмуро отмокала в ванной, я вымыл несколько раз и начал расчёсывать вечно спутанные, непонятного цвета волосы.
Волосы оказались золотистыми с проседью, даже более золотистыми, чем у Ликхэ, и свернулись в мелкие блестящие кудряшки.
Затем я аккуратно протёр лицо и шею, вычистив из шрамов и морщин застарелую смесь краски и грязи, смазал неожиданно заалевшие губы маслом.
Кхохо повеселела, увидев себя в бронзовом зеркале, но вставать сама всё ещё отказывалась.
Я перенёс её обратно в келью.

---  ...Только родила тогда, ну понимаешь, каково мне, и тут мне такое прилетает.
Я просто зверею, беру толстую палку и с размаху ей по морде.
Так её и похоронили с кривым носом, правда, не сразу, а лет через десять, когда она ласты склеила...

Я долго смотрел на её отмытое лицо, достаточно долго, чтобы уловить сильное сходство...

--- Вы с Ликхэ случайно не родственники?

--- Случайно да, я её родила, как раз тогда, --- ухмыльнулась Кхохо.
--- Вообще на меня не похожа характером, однако ж в одном Храме оказались.
Я ж её трахнула чуть ли не у порога храма, несколько раз, и только через декаду узнала, что это тот самый ребёнок...
Стыд-то какой, что бы про меня в Кахрахане сказали...

--- От Ситриса?

--- Нет, нет.
Это ещё до него было дело.
Заходил в Тхитрон торговец, сладкоречивый как Чхалас.
Рассказывал про летучих рыб, как у них плавники переливаются.
Я глазом не успела моргнуть, как от него залетела.

--- А Ликхэ знает?

--- А ей зачем?
Она и не спрашивала ни разу никого.
Ей сказали, что её воительница оставила малышкой, и все вопросы отпали...
Не для воспитания я, мелкий.
Иногда ёкает что-то, конечно --- когда ты Ликхэ руку сломал, я места себе не находила, пыталась всё её лечить, чтобы как раньше всё стало, Трукхвала трясла, врачей заезжих трясла, да где там...

--- Почему они говорят, что ты несёшь бред? --- не выдержав, спросил я.

--- Я каждому свои истории рассказываю, --- хихикнула она.
--- Кхарасу --- одни истории, Ситрису --- другие, Хитрам --- третьи.
Они думают, что я всё это выдумываю на ходу.

--- Но это правда? --- допытывался я.

--- Нет, конечно, --- ответила Кхохо.
--- Не может столько всего случиться с одним человеком за одну жизнь.
С ума же сойти можно.
Хотя... я и сошла.
Просто рассказываю всё, что помню, а что правда из этого, что мне приснилось после очередной накурки --- кто разберёт...

По её щекам покатились слёзы.

Ещё рассвет спустя Кхохо спустилась к себе на этаж.
Она снова ругалась, хохотала на всю округу, отпускала сальные шуточки и раздавала всем тумаки.
Ситрис смотрел на неё заворожённо, словно влюблённый.

--- Чё уставился? --- наконец раздражённо осведомилась Кхохо.

--- Я не знал, что у тебя такие красивые волосы, --- признался Ситрис.

--- Волосы действительно чудесные, редко такие увидишь, --- поддержала Хитрам.
--- Я вообще подумала вначале, что это родильница Ликхэ заявилась с визитом вежливости.
Это не ты её случайно родила?

--- Да не приведи духи! --- возмущённо фыркнула Ликхэ.

--- Да ладно тебе, --- попытался урезонить девушку Эрликх.
--- Кто бы ни была родильница, всё одно родная кровь, не вода ключевая.

--- Кровь ничего не решает, --- отрезала Ликхэ.
--- Есть люди, которые меня воспитали, я их люблю и хочу быть как они --- хорошо знающей своё дело, хозяйственной и ответственной.
Плевала я на ту бродяжку, которая меня родила и даже не удосужилась навестить.

Кхохо захихикала.
Уже открывший рот Ситрис сделал непередаваемое движение бровями --- не то от удивления, не то от обиды, не то от смущения, --- и, подумав, поинтересовался:

--- Это Ликхмас тебя отмыл?

--- Не привыкай, --- лаконично ответила Кхохо.
\ml{$0$}
{--- Больше эта мелочь со мной сидеть не будет.}
{``This smol shall never look after me.}
Разнеживает чересчур.
Руки бы пообрывать этим, которые его воспитывали, ну кто так делает вообще...

\section{[2] Пустыня}

Картель активно вёл переговоры с ноа.
Оно и понятно: если ноа откажутся нас пропустить, придётся идти вглубь пустыни или сесть на корабли.
Кораблей у нас не было, а большая армия в глубине пустыни имеет крайне мало времени на действия.
Лишить армию снабжения с помощью диверсий --- и она не протянет и двух дней.

Даже по Могильному берегу сели шли бодро.
Многие сделали из плавника и тростника пескоступы на ноги, прочие просто подбили сапоги ещё одним слоем кожи.
Южане в подражание ноа разделись донага, оставив лишь плащи и зонтики.
Жители Дальнего Севера предусмотрительно взяли с собой возы тёплой одежды и одеял --- для себя и прочих;
холодные пустынные ночи мы переживали в комфорте.

\section{Вера в себя}

--- Однажды мы с бабушкой Тхартху сидели на балконе.
Она молча смотрела на городскую площадь, держа в руках перо и пергамент.
Вдруг она сломала пополам перо, разорвала пергамент на кусочки и заявила: <<Как я ненавижу этот город.
Он отнял у меня все силы>>.
Я спрашиваю --- бабушка, если тебе не нравится Тхаммитр, почему ты не переедешь?
Она засмеялась горьким смехом и сказала: <<Милое дитя.
\ml{$0$}
{Ты ещё по-настоящему не теряла веру в себя?}
{You haven't lost you confidence, have you?}
\ml{$0$}
{Надеюсь, тебе и не доведётся>>.}
{I wish you never have.''}

\section{Разбитая миска}

--- Однажды я разбила миску для еды.
Маленькая ещё была.
Кормилица начала меня отчитывать --- вот, дескать, косорукое ты дитя.
А бабушка подошла, остановила её и так строго у меня спрашивает: <<Ребёнок, ты зачем миску разбил?>>
У меня всё внутри сжалось.
Говорю: <<Случайно>>.
Бабушка подумала и сказала: <<Хорошо>>.
Затем схватила вторую миску и шварк её об пол, только осколки разлетелись.

Чханэ усмехнулась.

--- Кормилица ей, естественно, с возмущением так: <<Тхартху!>>
Бабушка говорит: <<Я случайно>>.
Кормилица, с ещё большим возмущением: <<Тхартху!>>
Бабушка: <<Радость моя, ты действительно собралась устроить скандал из-за двух мисок, тем более разбитых \textit{совершенно случайно}?>>
Собственно, на этом всё и закончилось...

\section{[1] Орешки}

--- Жохас, неиссякаемый источник фундука и арахиса, --- отрекомендовала торговца Кхохо.
--- Я на нём половину задницы наела.

--- Он и мне пытался всучить, --- сообщила Чханэ.
--- Но я ему сразу сказала, что не положено.
У нас воины вообще всё за золото получают, и подарить что-то воину --- значит поставить под сомнение воинскую честь.

--- Ну, у нас всё не настолько строго, --- улыбнулась Хитрам.
--- Если предлагают конфетку или орешек --- бери.
И даже сама не стесняйся пробовать --- в конце концов, чем мы хуже крестьян?
А вот что-то крупнее брать без оплаты --- это уже нехорошо.

\section{Планы}

--- Расскажи мне про планы, --- попросил я.

Кхотлам отложила в сторону пергамент и задумалась.

--- Так, слушай.
Для начала --- зачем вообще нужны планы?
Крестьяне --- вольнолюбивый народ и выращивают то, что хотят.
Кому-то нравится выращивать бобы, кто-то предпочитает картофель --- это дело вкуса и умения.
В результате может случиться ситуация, что у нас будет три мешка бобов и пара помидоров на каждого горожанина.

--- Это неприятно, --- усмехнулся я.

--- Вот ты смеёшься, а я в детстве попала на такой год.
С тех пор не люблю кашу из бобов.

--- Это, как я понимаю, был промах купца?

--- Его отсутствие --- старый купец умер, а нового искали достаточно долго.
Но суть ты уловил правильно.
Именно дипломат пишет для крестьян план, который они должны выполнить --- с учётом их личных предпочтений, торговых отношений, климата и многого другого.
Знаешь, что считается показателем мастерства дипломата?

Я помотал головой.

--- Процент планового урожая.
Чем ниже этот процент, чем меньше дипломат вмешивается в работу крестьян, тем ценнее он как специалист.
Впрочем, это, я думаю, верно для любого управленца, включая квартальных старшин и вас, жрецов.

--- Каков твой процент?

Кхотлам усмехнулась.

--- Я не мастер в распределении товаров, к сожалению.
Я планирую одну восьмую урожая --- на всякий случай.
Отличным считается показатель в одну десятую или одну пятнадцатую.
Мне есть куда стремиться.

\section{Другой город}

Чханэ грустно улыбнулась.

--- Сменить пол --- это как переехать в другой город.
Такой же храм, такие же улицы, такие же люди вокруг, но ты не знаешь их, а они не знают тебя.
Даже с самыми близкими приходится знакомиться заново и всё начинать сначала.
Мне повезло, что я была маленькой.
Не всякий взрослый такое переживёт.

\section{[2] Вещи в дорогу}

Мой рюкзак волшебным образом оказался у дверей старого жилища.

--- Интересно, кто же нас всё-таки заметил, --- задумалась Чханэ.

--- Хитрам, --- догадался я, увидев сверху верёвку.

--- Не зря всё-таки твоя кормилица его в доме держит, --- хихикнула Чханэ.
--- Полезный он, даром что крестьянин.

--- Она его любит, --- укоризненно сказал я.
--- То, что он полезный --- это так, приятное дополнение.

\asterism

Оленей с тележкой мы купили в хуторе у заспанного крестьянина.

--- Ну вы даёте, --- буркнул он, рассматривая золотые серьги.
--- Куда только собрались на ночь глядя!

--- По делам Храма, --- весело сказала Чханэ.
--- Прокатимся заодно.

--- Тебя я помню, девка, ты смышлёная, --- сообщил ей крестьянин, ткнув узловатым пальцем прямо ей в живот.
--- А вот ваш вождь, кажись, пень пустоголовый.
Ну кто упряжки покупает в нашей глухомани!
Давайте я вам хоть фонарь в довесок зажгу, не в темноте же ехать.

Крестьянин обладал каким-то талантом зажигать фонари.
Милый розовый фонарик горел всю ночь до самой зари, несмотря на ухабы, ветер и падающие с деревьев капли.

\section{Весёлый Волок}

Весёлый Волок отличался от прочих святилищ.
Во-первых, он состоял сразу из двух городов, соединённых очень широким корабельным трактом.
Во-вторых, правило нейтралитета в Волоке дополнялось одной неписанной традицией: ты можешь хоть с рождения ненавидеть представителей другого народа, но подтолкнуть их застрявший корабль --- святое.
Именно поэтому Волок торговцы проходили гораздо быстрее, чем можно было бы ожидать.

Сразу за Волоком следовало большое пресное озеро, в чём-то похожее на Сотронское.
Это озеро носило забавное название Кха'ма\FM.
\FA{
Наконец-то! добрались! (цатрон).
}
Путей к озеру было только два --- по реке против течения, либо по уже описанному выше Волоку.
Называть озеро по имени следовало каждый раз, когда корабль касался его вод --- разумеется, с оттенком огромного облегчения.

\section{[2] Не надо так}

--- Судя по данным Картеля, Эйраки использовал МПДЛ и генетически модифицированных антарид для устрашения сапиентов.
Именно их сели называли <<радужным безумием>> и <<огнём, оплавляющим стены>>.

--- Странно, что он не сжёг планету к свиньям, --- буркнула Анкарьяль.
--- Антариды мутируют со скоростью света, их вообще нельзя выпускать на поверхности планеты.
Даже разведение плазменных форм жизни в мантии планеты чревато осложнениями --- вспомните Чёрную скалу.
Эти твари каждый год уносят тысячи жизней во время усиления вулканической деятельности.

--- Их так и не вывели до конца, --- поддержал я.
--- Они приспособились к планетным условиям, чего, собственно, и добивались недальновидные экспериментаторы.
Плазмоцисты до сих пор высеиваются даже в нижних слоях планетной коры.

--- Вот именно.
Здесь же, я думаю, помогла только высокая влажность.

--- Да не особо, --- заметил Грейс.
--- Жрецы Ихслантхара рассказывали об обширных лесных пожарах, начинавшихся после радужного безумия.
Выгорали площади в тысячи квадратных кхене вокруг городов.
Обычное пламя не способно заставить гореть такой влажный лес.

--- А люди выживали?

--- Как ни странно, большинство.
Каменные строения оказались неплохой защитой.

--- Собственно, Картель настоятельно порекомендовал Эйраки больше так не делать, --- закончил я.
--- Видимо, даже они были в шоке.

--- Он очень скуп, --- кивнул технолог.
--- Для него гораздо проще применить чрезвычайно опасное, но малозатратное плазмобиологическое оружие, чем тратить масс-энергию на что-то более адекватное.

\section{[1] Люди-тени}

Люди на вечерних улицах --- лишь тени в форме людей.
Ты не знаешь о них почти ничего, ты не видишь ни их лиц, ни цветов их праздничных одежд.
Всё, что ты знаешь о них --- что они есть где-то там, в полотне древесных теней, что их босые или обутые ноги мягко ступают по брусчатке.
Кто-то идёт один, кто-то парами, а некоторые идут толпой, и их тихий шаг, их далёкие речи и смех сплетаются в единый уютный человеческий шум.
Даже твои спутники становятся на вечерней улице лишь тенями.
Даже ты сам.

\section{Убор хоризии}

\begin{verse}
Из убора красотки-хоризии,\\
Что горда и фигурой пышна,\\
Ты принёс драгоценный цветок\\
И мне в волосы вплёл на закате...
\end{verse}

\section{[1] Животные}

--- Манэ, Лимнэ, ваши животные опять сбежали.
Смотрите, не то их задушат коты или заклюют курочки.

Манэ тут же бросилась в угол и вернулась с капюшонной крысой.
Лимнэ подозвала белую шиншиллу и взяла её на руки.
Шиншиллу привезла из Пыльного Предгорья путешественница-поэтесса с зелёными глазами и странным акцентом;
ручная крыса была изловлена в амбаре во время буйного пиршества, да так и осталась жить у сестрёнок.

\section{[2] Яблоко}

--- Как себя чувствуешь?

--- Отлично.
Сны только пошли яркие и волнующие, выспаться трудно.

--- Хай, это бывает, --- засмеялась кормилица.
--- Знакомой крестьянке снилось, что она родила не ребёнка, а то котёнка, то яблоко, то ещё что-нибудь.
Меня больше реакция позабавила.
Обидно, говорит, пять декад ходила --- и яблоко...

\section{[1] Ах, улиточка}

Дождь прошёл, и выглянуло яркое солнце.
В лужах уже вовсю копошилась какая-то мелюзга, на мокрых заборах и стенах сидели полосатые улитки.
Самых больших я мимоходом стягивал и складывал в карман --- на суп или жарево.

\section{[1] Паркетная полянка}

Паркетная полянка была известным местом встречи на окраинах Тхитрона.
И старожилы уже не припомнят, что за здание там стояло;
сейчас от него не осталось даже стен --- только несколько островков изумительного паркета напоминали о деятельности человека.
Паркетную полянку берегли.
Когда несколько дождей назад кто-то отбил одну паркетину, три квартала искали неведомого вандала.
И нашли --- квартал каменщиков вычислил инструмент по зазубринам на камне.
Что с героем сделали, история умалчивает, но больше паркет никто не ломал.

\section{[1] Не хватит}

Кхотлам, увидев половину Храма в праздничной одежде на голое тело, едва не выронила поднос.

--- Молодёжь, вы это зря.
Храмовников, особенно девушек, и так на каждые Руки в клочья рвут, а вы ещё и сами на себя внимание обращаете.

--- У кожевников вообще есть поверье, что если в Мягкие руки завалить воина, то кожи потом никакой клинок не пробьёт, --- радостно сказала Ликхэ.
--- А завалить жреца --- просто хорошая примета.
Жрецы у нас гордые...

--- Не надо на меня так смотреть, --- возмутился я.
--- Я не дичь и не коллекционная статуэтка!

--- Кожевников в квартале тридцать восемь, вас пятеро.
Лисёнок, ты вообще один с Верхнего этажа празднуешь.
Вас на ночь не хватит.

--- У нас особых планов нет, --- успокоил я кормилицу.

--- А у других на вас есть, --- резонно заметила Кхотлам.
--- Опять Кхохо всех подбила, это несомненно.
Впрочем, ладно, все взрослые люди.

--- Надеюсь, Кхохо поделится и уступит мне хоть кого-то из тех тридцати восьми, --- буркнула Ликхэ.
--- У неё свои приметы...

\section{Мысль}

Иногда накатывает странное желание покончить жизнь самоубийством --- лёгкое, воздушное, почти неощутимое.
Оно совершенно лишено негативной окраски, это желание с улыбкой --- лёгкой грустной улыбкой.
Оно появляется из-за мысли о совершенно несбыточной, но страстной мечте, которая поглотила слишком много ночей и раздулась до колоссальных размеров.
Это желание проходит почти сразу --- и, наверное, к великому счастью.

\section{[2] Пиратство}

--- Может, в пираты подадимся?

--- Идея интересная, --- одобрил я.
--- Жаль, что для меня работы у пиратов будет не так много.

--- Пиратам нужны врачи, --- заметила Чханэ.

--- Вот именно, этим их потребность в жрецах и ограничивается.

--- Ааа, --- поняла Чханэ.
--- Ну тогда да, лучше будет поселиться где-нибудь на Короне.

--- Если хочешь к пиратам --- то иди.

--- Слушай, Аркадиу, --- Чханэ произнесла моё имя неожиданно чисто, --- я пообещала, что пойду с тобой.
Поэтому заткнись.

И я заткнулся.

\section{[1] Касания Пера}

Эрликх бился совсем по-другому.
Его удары, неощутимые и невесомые, самым возмутительным образом вытягивали силы.
Даже Кхарас после спарринга с Эрликхом выглядел усталым;
я же валился с ног после трёх-четырёх касаний.
Второй особенность Касаний Пера было то, что синяки начинали жутко болеть сутки спустя.

--- Как ты это делаешь? --- взмолился я после очередного спарринга.
У меня дрожало всё, что могло дрожать.

--- Я просто чувствую, --- сказал Эрликх и, схватив мою руку, прижал к своей груди.
--- Попробуй меня ударить медленно.
Вдави в меня кулак.
Чувствуешь, как изменяется упругость плоти?
Слышишь это скрипение, хруст?
Мышцы вдавливаются по-своему, и каждая косточка сгибается по собственным законам, а значит --- издаёт особый звук.
Ты можешь контролировать это, ты можешь изменять звук так, как надо тебе.
Когда ты будешь чувствовать и слышать удары на обычной скорости боя, ты поймёшь, что делаю я.

И Эрликх ударил меня ещё раз.
Я в бессилии мешком свалился на пол, почти не почувствовав прикосновения.

\section{[1] Брусчатка}

Ливень омыл древнюю тхитронскую брусчатку, на миг обнажив буйство красок отполированных до блеска вулканических пород.
Мало кто из приезжих догадывался, что на мостовой четырёх дорог был выложен узор --- породами аж пяти цветов.
Увидеть это можно было лишь утрами, с Середины Дождя, когда брусчатка сияла влажной чистотой, а облачную серость только-только расцвечивало выглянувшее солнце.
Три-четыре рассвета --- и краски пропадали до следующего сезона.

Я уверен, что те, кто мостил дороги, прекрасно знали, что в итоге всё занесёт грязью, но делали всю эту красоту ради нескольких дней в году.
Тхитронцы очень любили делать неожиданные подарки.
Это я понял однажды, отыскав в своей комнате тайник с рассыпавшимися от старости игрушками.
Если верить выцарапанным на камне иероглифам, тайник сделали строители, видимо, не поставив в известность зодчего;
дети, которым он предназначался, повзрослели, завели своих детей и умерли, так его и не найдя\FM.
\FA{
На языке тси такие подарки дословно называются <<транзит с востока на запад>> или просто <<восток-запад>>, то есть послание в будущее.
}

Единственным незамощённым местом в центре города был дворик моего дома.
Там была плотно утоптанная земля, поросшая редкими кустиками травы.
Возле колодца в земле поблёскивала красивая фульгуритовая ветвь, проходившая через весь дворик.
Молния в это место ударила очень давно, и даже старожилы не знали, когда именно;
но все точно знали, что именно из-за этой красоты дворик и не стали покрывать брусчаткой.

\section{Мудрость}

Есть в людях неуловимые признаки житейской мудрости, помогающей жить и процветать.
Например, зонтик, взятый в день начала дождей.
Или чистые штаны, мыльный раствор и зубная щётка в котомке.
Или надрезанные сеточкой свежие огурцы, лучше пропитывающиеся пряностями и солью.
Сами по себе эти вещи ничтожны;
что изменят пара кхамит в несвежей одежде или приготовленные иначе огурцы?
Однако в целом такие люди и живут дольше, да и в жилищах у них куда больше света и счастья.
Суть часто кроется в мелочах.

--- Мудрость определяется легко, Ликхмас, --- как-то сказал Конфетка.
--- По словам.
Спросишь у человека --- зачем ты это сделал?
А он в ответ: <<Так вкуснее>>, <<Так удобнее>>, <<Так легче>>, <<Так проще>>.
Вот это мудрость.
Если же он отвечает: <<А вдруг>>, <<А если>> --- это или скупость, или опасливость, не имеющие ничего общего с мудростью.
Так, прикормка для собственной тревоги.
Мудрый всегда знает, что он делает и зачем.

\section{Каштаны}

Там, на Дальнем Севере, где распускающиеся в Пирог почки каштанов похожи на неуклюжих большелапых птенцов.

\section{Люби себя}

Любовь --- это когда ты наливаешь чай, насыпаешь каши и гладишь своего любимого, даже если точно знаешь, что лучшие дни его жизни позади.
Любовь --- это верить и помогать, не только в работе, но и в отдыхе.
Постелить постель, потереть спинку, погладить и почистить одежду.
И пусть всё мироздание будет против твоего любимого, ты будешь с ним.
Люби людей так.
И самое важное --- люби себя так же.
Всегда.

\section{Внутренняя свобода}

Наверное, следовало помочь, пожалеть, но я отвернулся и пошёл дальше.
Иногда хочется просто побыть человеком, свободным испытывать чувства.
Свободным любить, свободным ненавидеть, свободным быть верным и свободным предавать.
Для равнодушия порой тоже необходима внутренняя свобода.

\section{Сапоги}

Последними я надел сапоги.
Застёжки и завязки я приводил в порядок нарочито медленно, смакуя каждое движение.
Бегать босиком --- это удовольствие.
Но надевать обувь --- удовольствие совершенно другого рода.
Ни посох, ни котомка, ни убегающая вдаль дорога так не манят в путь, как надёжная обувь, нежно льнущая к стопе и уверенной хваткой держащая голень.

\section{Высота}

Я вздохнул полной грудью и почувствовал, как сразу стало легче.
А ведь только и надо было --- выйти за порог.
Во многих случаях это единственное лекарство.
Не пугали больше ни клинки, ни джунгли, ни высота.
Да и что такое высота?
Всего лишь вертикальная длина.

\section{[1] Мхи (БВ)}

Я погладил мшистый камень барельефа.
Удивительно, сколько всякой живности здесь, у самой реки.
Даже мхов.
Вот <<рыжая шкурка>> --- зелёная моховая шапочка со множеством красно-бурых остистых волос.
Вот <<северное серебро>> --- твёрдый, блестящий, словно бледно-зелёный шёлк, плотно сбитый из круглых стеблей.
Вот безымянный, губчатый, пушистый и нежный, цветом напоминающий о болоте.
А вот золотистые звёздчатые шары <<вечернего красавца>>, рыхлые и жестковатые на ощупь.
То тут, то там в гранит въедались бородавчатые брызги лишайников --- серые, сизые, бурые, ярко-зелёные.
Учитель как-то прочитал, что лишайники --- это тоже грибы.
Наверное, он что-то недопонял;
внимательно рассмотрев лишайник и попробовав его на язык, я признал, что с грибом он не имеет ничего общего.

\section{[1] Пылеройский хлеб}

Все справедливо решили, что для готовки время чересчур позднее, и удовольствовались закуской.

--- Хлеб с помидорами --- это самое изысканное лакомство с голодухи, --- сказал Ситрис, обильно сдабривая лакомство солью.

--- Мы этим конопляное вино закусываем, --- поделилась Чханэ.
--- Усиливает эффект, но похмелье потом такое, что хоть голову в горшке держи.
Хлеб лучше брать пылеройский.

--- Пылеройский хлеб? --- удивилась Кхохо.

--- Да, ржаной с саговыми зёрнами, чёрными скорпионами и, кажется, эстрагоном.
Мы же с пылероями воюем постоянно, трофеев много --- оружие, одежда, еда.
Они распробовали наше, мы распробовали их, в итоге и вышло, что жрём-то мы с собаками одни и те же деликатесы.
А вот в джунглях пылеройский хлеб почти не известен, потому как скорпионов там не разводят.

\section{Бортники}

Мимо прошли три бортника в плотных зелёных костюмах, запевая тягучую <<медовую>> песню.
Они несли три круглых, промазанных глиной корзины и клетку, в которой заливисто чирикала пара ручных попрошаек.

\section{[1] Душ}

--- Здесь твоя лежанка, --- показал Трукхвал.
--- Бельё сам стирать будешь.
Хай, а вон та дверь --- душ.

--- Что?

--- Душ.
Идём покажу.

Трукхвал открыл дверь и, занеся руку с факелом, осветил тёмное помещение.

--- Это изобретение ноа, у которых туго с чистой водой.
У ноа даже есть пословица: <<Существуют только три истинных, чистых удовольствия --- секс, душ и дорога>>.

--- У меня не было секса, ничего сказать не могу, --- пожал я плечами.

--- Торопиться не надо.
Пожалуй, в Храме его даже многовато...
Так что душ --- на случай, если тебе захочется помыться в одиночестве.
Или если в бане Кхохо или Эрликх будут чересчур приставать, они любят молодых...

Я засмеялся.

--- А я, кстати, и не шучу.
Эти ходячие гениталии соблазнили Ликхэ на десятый день в Храме, по очереди, не сговариваясь.
И не спрашивай, откуда я это знаю...
Вон там лампа, свечу прикрывай стеклом, чтобы не погасла.
А вот и сам душ.
Наступаешь по очереди на педальки, и вода из бадьи льётся на тебя.
Здорово, правда?
Только не прыгай на педальках и не лей всякую гадость в бадью, а то были тут у нас любители...

--- А что Ликхэ?

--- Вот молодёжь.
Я ему про такое интересное устройство, а он про Ликхэ... --- Трукхвал смотрел строго, но я прямо чувствовал, что пожилой жрец надо мной забавляется.
--- Кхарас, боевой вождь, накричал на Кхохо и Эрликха --- девочка ещё нецелованная пришла, а они, такие-сякие...
Кхохо ему --- уже целованная, причём везде, я лично проконтролировала.
Кхарас Ликхэ --- понравилось?
А та улыбается, как солнышко ясное.
Кхарас подумал и плюнул --- делайте что хотите, лишь бы по согласию.

\asterism

Предупреждение Трукхвала было не напрасным, но несколько неточным.
Кхохо поймала меня в том самом душе.
Когда дело плавно подходило к началу, в душ вошёл Ситрис и оттащил воительницу от меня.

--- Это. Её. Ребёнок, --- тихо, но отчётливо сказал воин ей на ухо.

--- И что? --- невинным тоном осведомилась женщина.

Ситрис отвесил ей такую затрещину, что она покатилась по полу.

Кхохо дулась на Ситриса весь вечер.
Ночью в спальне была слышна тихая ругань и звуки борьбы.
На завтраке дулся уже Ситрис, прикрывая рукой тёмно-синий фингал.
Но, видимо, воины пришли к какому-то согласию --- отныне Кхохо вела себя со мной исключительно по-дружески.

\section{Большой дом}

Когда я был совсем маленьким, я гордился тем, что у моей кормилицы такой большой дом.
Однако потом в дом зачастили торговцы, носильщики, эмиссары соседних племён, разведчики и прочие люди, с которыми регулярно приходится иметь дело купцу.
Они часто оставались на ночь и всегда --- на трапезу.
В конце концов я понял, что домочадцам во дворе Люм по-настоящему принадлежат только их собственные комнаты, и перестал хвастаться размерами дома перед другими детьми.

\section{[1] Уволочи}

--- Скоро всё будет...
Аийяхс-с, зараза!..

Со стола во все стороны брызнули мелкие, похожие на кошек животные.
Чханэ схватила тяжёлое шерстяное полотенце и стала наотмашь карать воришек.
Кухня наполнилась шипением, ворчанием и жалобным мяуканьем.

--- Хаяй, что за дела! --- пожаловалась мне подруга, когда порядок был восстановлен.
--- Только мясо разделала!..

--- Что такое? --- в дверном проёме появилась Ликхэ, привлечённая суетой.
--- Чханэ, ты готовишь?

--- Готовлю, --- сварливо бросила Чханэ.
--- Тут какие-то котята у меня мясо таскают.

--- Хай, это же уволочи, --- засмеялась Ликхэ.
--- Ты их никогда не видела?
Мы обычно им у окна обрезки оставляем.
Тебе я забыла сказать.
Вот они на стол и залезли.

--- Ликхэ, зачем ты их прикармливаешь? --- возмутился я.
--- Они на ласточек охотятся!

--- Да на помойку ласточек.
Традиция, да, а какой ещё от них толк?
Уволочи много не возьмут, зато мышей и всяких вредителей подъедают.
Не то что эти толстые ленивые оцелоты, которые весь день лежат и бока греют.
Чханэ, извини, речь не о тебе.

Чханэ свирепо засопела, но промолчала.

--- Может, ты не заметил, Лис, но последние пять дождей зерно почти никто не портит.
Как думаешь, почему?

Ликхэ обиженно развернулась и ушла.

Мы с Чханэ помолчали.

--- Зря ты, --- наконец сказала она мне.
--- В целом, неплохо эта рыбина придумала.
В следующее дежурство тоже им оставлю.
В этот раз, --- Чханэ печально посмотрела на погрызенные куски кролика, --- в этот раз они наелись от пуза, так что обойдутся.

\section{[1] Копи}

Однажды ночью меня разбудил Трукхвал.
Он знаками показал, чтобы я оделся по-походному и взял всё необходимое.
Глаза учителя горели непонятным безумным огнём.

Под покровом ночи мы спустились в подземный ход, начинавшийся под бадьёй в душевой храма.
У меня не было уверенности, что кто-то кроме Трукхвала осведомлён об этом туннеле.

--- Куда мы идём, учитель? --- осмелился я нарушить молчание.

--- К тайне, Лис.
К величайшей тайне, --- ответил старик, подняв факел к потолку.

Трукхвал первый и последний раз в жизни назвал меня домашним именем.
Это означало высшую степень доверия... и опасность.
Я чувствовал эту опасность всеми фибрами души --- в фигуре старика, которая вдруг потеряла привычную мне сгорбленность, в его неестественно прямой ноге, которая, против обыкновения, почти не шаркала при ходьбе, и во влажности стен старинного туннеля, которым не пользовались последние пятьсот дождей.

Подземный ход вынырнул на поверхность восле Трухлявой скалы, и почти сразу старик позвал меня в следующий, скрывающийся под корнями старой секвойи.
В нем было ещё тише и темнее, и Трукхвал остановился спустя десять шагов.

--- Теперь слушай меня, --- в глухом шёпоте учителя звучала небывалая серьёзность.
Несмотря на то, что нас вряд ли мог кто-то услышать, он не повышал голоса.
--- Вот в этой котомке костюм.
Ты должен надеть его как можно плотнее, чтобы нигде --- нигде, Ликхмас! --- не осталось щелей и открытых мест.
Ты меня понял?
Надевай.

В котомках оказались странного покроя комбинезоны из очень плотной ткани, высокие кожаные сапоги и кожаные же перчатки, доходящие до локтей.
Я надел непривычную одежду --- она оказалась чуть-чуть длинна и немного узка в плечах.
Трукхвал, быстро справившись со своим комбинезоном, придирчиво затянул и проверил завязки на моём.

Ощущение опасности возрастало.
Что собирался показать мне учитель?
Я лихорадочно вспоминал темы наших последних занятий.
Фауна джунглей, которую я и без того неплохо знал благодаря старому охотнику Сирту.
Может быть, где-то гнездятся опасные насекомые --- зелёные пчёлы или того хуже?
Но почему ночью и под землёй?..

--- Это нечто, что опаснее зелёных пчёл, Ликхмас, --- Трукхвал словно прочитал мои мысли.
--- А теперь возьми эти очки и надень утиную маску.
Нет, не свою --- из котомки.

Я выполнил все указания и привычным движением подпоясался.

--- Нет, Ликхмас, --- покачал головой Трукхвал.
--- Оружие оставь.
Припасы тоже.

--- Учитель, объясни уже, что там! --- взорвался я, бросив нож и жалобно звякнувшую фалангу на камень.
--- Я не ребёнок, чтобы...

Я осёкся на полуслове.
Пещера усиливала голос, словно храмовый зал.
Я напряжённо вслушивался в гулявшее вокруг эхо.
Трукхвал виновато смотрел на меня.

--- Прости, ученик.
Я немного заигрался.
Я... я обнаружил очень сильную Каменную ярость.

Я ахнул.

--- Но зачем мы к ней идём, к ней же нельзя приближаться?

--- Разумеется, можно, --- пробормотал учитель, и лицо его смягчилось.
--- Неприятно, но если ненадолго и при защите, то можно.
Идём, расскажу всё по дороге.
Да оставь ты его, --- недовольно добавил Трукхвал, увидев, что я по привычке схватил мешок с припасами.
--- Если мы там застрянем, то что с припасами, что без --- всё одно смерть.

--- Утешил, --- огрызнулся я.
Трукхвал хрипло захохотал, и его приглушённый маской смех эхом отозвался в коридорах.

\asterism

Сразу за поворотом пещера резко пошла на уклон.
Иногда ход шёл настолько круто, что нам приходилось цепляться за покрытую глубокими трещинами стену.
Удивительно, но здесь не было никакой живности --- вполне возможно, что тот, закрытый старым деревянным люком вход был единственным.
Мы шли, как мне показалось, очень долго, на поверхности уже давно должны были запеть утренние птицы.

Наконец мы добрались до размётанного кострища.

--- Пришли, --- Трукхвал весело передал факел мне и рванул вперёд с неприличной для старого хромого жреца прытью.

Ход оканчивался идеально овальным окном в человеческий рост.
На лежащих рядом камнях я заметил следы кирки.

--- Это я, --- гордо признался Трукхвал, проследив за моим взглядом.
--- Мы сейчас в одном из рукавов копей Древних, их уже давно обшарили наши предшественники в поисках сокровищ.
А этот вход никто, кроме меня, не заметил.
И немудрено --- слышно его лишь при стуке в стену, и потребовалось пять дней, чтобы расчистить путь киркой.
Идём, Ликхмас-тари, здесь начинаются чудеса.

Трукхвал подбежал к овальному проходу... и едва не рухнул в бездонную пропасть.
Я успел схватить старого жреца за комбинезон и втащить обратно.
К моему удивлению, учитель снова захохотал.

--- Ликхмас, отпусти меня.
Ты молодец, среагировал, но в этом не было необходимости.

Трукхвал снова подошёл к пропасти... и шагнул прямо в неё.
Я ахнул --- учитель стоял в воздухе, словно на твёрдой поверхности.

--- Ну как?! --- торжествующе крикнул он в маску.

--- Не может быть... --- я всё ещё отказывался поверить глазам.

--- Иди ко мне.

--- И я тоже?..

--- Сможешь.
Только иди, словно по дороге, никаких лишних движений.

Я подошёл к краю, шагнул вперёд... и рухнул в пропасть.
Желудок переместился куда-то к горлу, я зажмурился и едва удержался, чтобы не издать дикий крик --- сели недостойно умирать с криком на губах.
Перед глазами пронеслась вся жизнь...

--- Ликхмас, выпрямись!
Выпрямись, говорю! --- весело смеясь, кричал мне Трукхвал.

Я открыл глаза.
Ничего страшного не происходило --- тело не падало, а летело с постоянной скоростью вниз.
Я послушался Трукхвала --- и падение замедлилось.
Вскоре мы с учителем дружно, до слёз хохотали.
Моё сердце бешено колотилось --- под ногами не было ничего, но я словно стоял на мягкой, слегка пружинящей поверхности.

--- Теперь слушай, --- заговорил Трукхвал, пытаясь утереть слёзы под очками.
--- Лесные духи, узнаю себя в первый раз...
Наклон вперёд --- вниз.
Руки в зенит --- вверх.
Руки по швам --- стоишь на месте.
Рука в сторону --- летишь в направлении руки.
Попробуй, и главное --- не бойся.
Разбиться, столкнуться с чем-то и даже просто поцарапаться о стену магия тебе не даст --- я пробовал.

\asterism

--- Я уже устал летать, --- признался Трукхвал спустя некоторое время.
--- За эти слова в Храме меня бы забили метлой, но возраст есть возраст.
Давай вниз, нам нужно лететь до цифр на стене.

Я сделал изящный пируэт и остановился рядом с учителем.

--- Цифры как у нас? --- уточнил я.

--- В точности как у нас.
Раньше они светились, а сейчас почему-то перестали.
Да и глаза уже не те...
Так что не пропусти, нам нужна дверь с числом 1D, почти в самом низу.

--- А это что, учитель? --- я указал на надпись, которую приметил ранее.
Трукхвал подлетел поближе.

--- Очень интересно, --- старик привычным движением вытащил из-под комбинезона пергамент и рассыпчатый уголь.
--- Сколько летаю, а этой надписи ещё не замечал.
Молодец, ученик.

Трукхвал приложил пергамент к надписи и протёр его углём.

--- Я... пишу книгу об этом месте, --- признался он, спрятав пергамент в сапог.
--- Эти знаки на стенах я копирую --- вдруг здесь написано что-то важное.

--- А почему ты не сообщил ничего нашему Храму? --- удивился я.

Старик брезгливо фыркнул --- я впервые слышал от него такой звук.

--- Нашему Храму лишь бы только что запретить.
В копи вообще ходить нельзя --- жрецы знают, что там встречается Каменная ярость.
Только крестьяне плевали на запреты, а вот если в копях поймают меня... --- старик выразительно щёлкнул пальцами.
--- Так что пока соберу материал, а потом представлю Советам.
Или ты представишь, если я не успею...

<<Так вот почему ты меня позвал>>, --- догадался я.

\asterism

Вскоре уровень 1D оказался прямо перед нами.
Труквал был прав --- ещё пара уровней, и труба заканчивалась закруглённым дном.
Стало ощутимо жарче.
Я боялся даже представить, на какую глубину мы опустились.

Трукхвал подвёл меня к маленькому окошку.

--- А теперь следующее чудо, --- шепнул учитель.
--- Здесь находится дух, который пропускает к Каменной ярости.
Подчиняется он заклинаниям, написанным в воздухе.

Трукхвал изящно пошевелил пальцами.
В воздухе, пламенея, повисли пять иероглифов --- четыре слова и цифра.
В воздухе высветилось совершенно понятное слово <<ОПАСНО>>, и дверь почти бесшумно отъехала в сторону.
Я открыл рот.

--- Ты не представляешь, Ликхмас, каких трудов мне стоило добыть это заклинание, --- прошептал старик.
--- Смотри.

Трукхвал написал иероглиф <<металл>>.
В ответ в воздухе загорелось несколько слов.
Я пригляделся --- все они начинались на иероглиф <<металл>>.

--- Это заклинания, которые дух понимает, --- объяснил Трукхвал.
--- Некоторые из них показывают непонятный текст, другие выводят какие-то цифры, таблицы и рисунки.
А вот это, --- он написал в воздухе три иероглифа, --- показывает карту шахты.

Я с восхищением смотрел на красивую трёхмерную карту, по которой были рассеяны разноцветные точки.
Трукхвал ладонью мягко крутил её, увеличивал и уменьшал по своему желанию.

--- Сколько времени ты подбирал заклинание, чтобы открыть дверь? --- прошептал я.

--- Три года, Ликхмас.
Три года.
Ходил сюда каждую вторую ночь.
Кстати, в моей книге есть ещё десятка четыре полезных заклинаний.
Ну ладно, пойдём внутрь, --- Трукхвал погладил меня по голове, привычным жестом написал заклинание, и дух, снова предупредив об опасности, отворил закрывшуюся дверь.
Мы вошли внутрь.

Пока Трукхвал возился с гаснущим факелом, я стоял рядом.
Я понимал, что сейчас произошло нечто совершенно необыкновенное но сил удивляться у меня больше не было.
К тому же существовало кое-что поважнее...

--- А про какую опасность говорил дух, учитель? --- спросил я.
--- Про Каменную ярость?

--- Тут много опасностей, --- объяснил учитель.
--- Из-за одной из них я теперь хромаю.

Трукхвал снял капюшон, откинул густые волосы и показал на темени давний шрам в виде шестиугольной звезды.

\asterism

Дальше мы шли значительно осторожнее и тише.
Ход был очень ровным, словно вырезанным искусным мастером по камню.
Я успел заметить на карте, что лабиринт простирался очень далеко --- возможно, у Трукхвала ушли годы на его исследование.
Время от времени нам попадались странные железные существа --- не то жуки, не то пауки размером с жилище.
Они лежали неподвижно, расклячив металлические ноги, словно раздавленные огромной ногой.

--- Одна такая меня и ударила, когда я попытался под неё залезть, --- шёпотом сказал Трукхвал.
--- Проткнула голову, словно кусок масла, я едва успел присесть и отпрыгнуть, чтобы меня не пригвоздило к полу.
Когда очнулся, понял, что череп пробит, а нога не хочет двигаться.
Залил голову смолой, кое-как выбрался, отлежался в храме, стало лучше, но до конца нога так и не вернулась.
Головные боли мучили где-то десять дождей...
Раньше эти звери, если верить хроникам, валялись везде в копях, но местные растащили большую часть на металл.
Никто даже не считался с тем, что это звери Древних --- шла война с ноа, и металл был нужнее.
Остались только здешние --- после того случая я попросил у них прощения и пообещал больше не сердить.

... Сколько мы шли?
Вероятно, несколько часов --- под землёй время тянется очень медленно.
У меня начало сосать под ложечкой от голода.
Наверняка уже полдень.

Наконец в воздухе появилась странная неприятная дымка.
Трукхвал обратил моё внимание на неё, велел смочить утиную маску и ещё раз проверить комбинезон.
Дымка становилась всё более отчётливой.
Странные стальные звери почти перестали попадаться.
И вдруг...

--- Пришли, --- выдохнул Трукхвал и, поёжившись, потушил факел.

Привыкшие к жёлто-инфракрасному факелу глаза какое-то время ничего не видели.
Но я понял, что дымка не висела в воздухе --- всё вокруг освещалось слабым-слабым свечением, льющимся откуда-то справа.
В нём тени как будто тонули.
Границы пещеры расплывались, словно стены состояли из желе или тёмного стекла...
Это свечение вызывало непонятную, смутную тревогу и страх.

--- Видишь источник? --- сказал Трукхвал.
--- Справа, похож на молнию, застывшую в камне.
Это и есть Каменная ярость.
Кстати, я увижу твои косточки, если хорошо пригляжусь.

Я поднял руку и взглянул сквозь неё на Каменную ярость.
Трукхвал сказал правду --- свет Каменной ярости пронзал плоть.
Рука по-прежнему слабо светилась теплом --- одежда приглушает естественное свечение тела.
Но появился новый оттенок, новая игра света.
Я видел каждую косточку в моём запястье.
А вот и сломанная когда-то в играх фаланга пальца, которая не совсем правильно срослась...

--- У тебя лишняя сесамовидная косточка в колене справа, --- весело сказал Трукхвал.
--- Повернись-ка.
О, и зубы не все ещё выросли.

--- Чудеса, --- шепнул я и обежал учителя, чтобы рассмотреть его скелет.
--- Хай, вижу отверстие в черепе.
Ключица неправильно срослась.
И скоба на ребре.

--- Скоба? --- учитель удивлённо посмотрел на грудь.
--- Хаяй, это мы тогда с Ситрисом чинили карниз, я про неё забыл совсем...
Ключица тоже оттуда, с Ситрисом вообще ничего чинить нельзя.

--- Почему?

--- Потому что он мишень для Кхохо!
Он страховку держал, а она мимо проходила.
Думаю, то, что произошло потом, объяснять не нужно.
Думать перед действием --- это вообще не для неё.
Ситрис-то отделался разбитым носом, а я так упал, что дыхание перехватило на пару михнет, думал, помираю.

--- Трааа...

--- Всё, Ликхмас, играм конец.
Мы достаточно долго здесь пробыли.
Забавно, конечно, но у меня, если честно, мороз по коже от этого свечения.
Идём к выходу, и старайся не задевать железных зверей.

--- Но мы только...

--- Ликхмас, если мы чего-то боимся, то это не просто так, --- тон учителя был суров.

\asterism

На выходе мы избавились от комбинезонов и масок.
Трукхвал велел выбросить их в пропасть, когда мы поднялись по воздуху к прорубленному в камне туннелю.

--- Когда придёшь, в душе не мойся, --- предупредил меня учитель.
--- Иди к реке ниже города и поплавай подольше, отмокни.
Пыль Каменной ярости невероятно прилипчивая, и городским глотать её ни к чему.

--- Учитель, что будет, если остаться там надолго? --- задал я вопрос, мучивший меня всю дорогу.

--- Долгая, неотвратимая и мучительная смерть, --- коротко ответил Трукхвал.
--- Бывали те, кто доставал на поверхность <<око земли>> --- маленький сгусток Каменной ярости, который светит ярче Солнца.
Их города поражала эпидемия.
Каменная ярость должна оставаться в камне, Ликхмас.

--- Зачем же её добывали Древние, учитель?

--- Хотел бы я знать, --- вздохнул Трукхвал.

Я смотрел на седые длинные локоны старика, на его спокойные зелёные глаза, крючковатый нос и волевой, совершенно не старческий подбородок.
Больше моего учителя не существовало.
Рядом со мной шагал Герой, который в одиночку сделал шаг в пропасть, пролетел по воздуху, поговорил с духом ворот и сразился огромным с железным пауком ради того, чтобы одним глазком взглянуть на Каменную ярость.

--- Не смотри на меня так, --- нахмурился Трукхвал и отвернулся.
--- Я этого не заслужил.

\section{[1] Дохляк}

--- Этот дохленький, --- сказал кормилец.
--- Если вырастет --- пустим на суп.

Цыплёнок действительно был слабее и меньше прочих.
Я чувствовал в нём не просто слабость.
Он был другим, не похожим на прочих цыплят.

Я схватил цыплёнка в охапку и бросил его о стену.
Он жалобно запищал.
Я сделал это ещё раз, и ещё.
После десятого броска цыплёнок вдруг запищал чуть громче и захромал.
Смутившись, я повернулся и убежал.

За ужином Хитрам сидел озабоченный.

--- Что за день, --- сказал он.
--- Ликхмас, ты сегодня в полдень кур кормил?

--- Да, --- буркнул я.

--- Дохляк лапку сломал, --- объяснил кормилец.
--- Понятия не имею, как у него это получилось.
Надеюсь, что это простая случайность.

Я промолчал.

Дохляку Хитрам наложил шину, но лапа упорно не желала срастаться.
Цыплёнок так и хромал до самой своей смерти.
Суп получился вкусный --- в меру жирный и наваристый.
Все кушали и нахваливали Сирту-лехэ, который оказался знатоком приготовления куриных супов.

Я по-прежнему кормил кур, но старался отделаться от этого как можно быстрее.
В конце концов, кем я был?
Обычным ребёнком, чья голова занята играми, товарищами и прочими интересными вещами, которыми полон мир.
Мне совсем не хотелось думать о том, что я сломал чью-то маленькую жизнь.
Но единожды осознанное из головы уже не выбросить.

\section{[1] Бывшая женщина}

В дверь без стука заглянула улыбчивая растрёпанная крестьянка --- с ней кормилец когда-то жил.
По его словам, дом Кхатрим был чем-то вроде постоялого двора для мужчин --- почти все приехавшие получали там дом, пищу, постель и женщину в лице хозяйки.
Взамен работали в поле, смотрели за детьми и чинили дом.
Детей у Кхатрим на тот момент было аж шесть --- весьма внушительное количество, учитывая Отбор и низкую плодовитость сели.
Двое из них --- братья-близнецы Марас и Хатрас --- были от кормильца.
Во всяком случае, внешнее сходство присутствовало.
Её питомцем был и Столбик, мой друг детства.

Кхатрим заглядывала к нам регулярно, в последних числах месяца.
Хитрам иногда пропадал у неё на день-два --- отдохнуть, проведать детей и помочь по хозяйству.
Сегодня же, видимо, случилось что-то серьёзное.

--- Хай, вы обедаете.
Прошу прощения.

--- Да, Кхатрим? --- обернулась кормилица.
--- Ты что-то хотела?

--- Мне срочно нужны руки, хотела Хитрама попросить.
Дерево упало прямо на дом, хорошо ещё, что мы все в поле были, --- радостно отозвалась женщина.

--- Да ты что! --- расстроилась Кхотлам.
--- Вам приют не нужен?

--- Я с мужчинами пока живу у соседей, но ради такого случая могу прийти в гости.

--- Заходи вечером, --- кивнула Кхотлам.
--- Я что-нибудь приготовлю.
Хитрам, иди, я тебе потом согрею.

Кормилец кивнул и пошёл к двери.

--- А я тебе предлагал его спилить, --- бросил он.
По его лицу бродила слабая улыбка.

--- Я помню, --- заулыбалась Кхатрим.
--- Оно было красивое и очень хорошо смотрелось рядом с домом.

\section{[1] Одна большая семья}

--- Ведь почти все прочие племена обучают военному искусству только мужчин.
Есть, правда, трами, у которых воюют только женщины.
Знаешь, почему наши воины --- мужчины и женщины?
Потому что чисто мужская или чисто женская армия --- это армия насилия!

--- Я слышал, что воины некоторых Храмов представляют собой одну большую семью, --- заметил я.
--- Единственное, в чём им отказывают --- это в воспитании маленьких детей.

--- Так и есть, --- согласился Трукхвал.
--- У нас тоже, хай, семья.
Можно сказать.
Хай, да.
Только в обычной семье обязательно есть взрослые, а у нас взрослых, хай, нет.
Вообще.
Ни одного взрослого.

\section{Соль земли}

Однажды Трукхвал повёл меня на крышу.
Только-только начиналась страда;
по городу звучали крестьянские песни, катались гружёные тележки и ходили носильщики с корзинами.

--- Посмотри, Ликхмас.
В библиотеке легко забыть о них --- о тех, кто снаружи храма.
В мире было много заблуждений на их счёт: считалось, что народ --- исполнители воли, дешёвая оправа для драгоценного камня деятелей искусств, вождей или учёных.
Кого видишь ты?

Я посмотрел вниз.

--- Силу.

--- И это истина.
Что ты видишь ещё?

--- Труд.
Тяжкий труд.

--- Никогда не забывай о том, кто тебя кормит и защищает до поры до времени.
Деятели искусств, вожди и учёные гораздо хуже рабов в этом отношении.
Если у раба не будет хозяина, он будет хоть что-то из себя представлять.
Храм без народа --- ничто.
Даже если народ обратит против Храма клинки, мы не можем ответить им тем же.
Не потому, что не способны, а потому, что перестанем быть Храмом.
Мы --- щит, и рука под ремнями --- рука народа.

Мы помолчали.

--- Пусть тебя не введёт в заблуждение и слава, Ликхмас.
Кто твой любимый лесной дух?

--- Митр, наверное, --- усмехнулся я.

--- Верно.
Утешающий, излечивающий душевные раны и усыпляющий не знающих сна.
Однако я клянусь тебе, что самая безвестная женщина, что вынашивала детей и видела, как изменяются их тела и мысли, излечила больше ран, чем Печальный Митр.
Помни об этом.

--- Я буду помнить.

--- В этом городе сила, --- мимоходом пробормотал Трукхвал под нос.
--- И ведь верно заметил.
Почему я не?..
Почему-то я не...
Хай... Засиделся я в библиотеке...

\section{[1+] Ситрис и штаны}

\textbf{(Часть про платки надо вставить в Семью или ранее)}

Ситрис сидел в зале и шил штаны.

--- Кхарас опять порвал.
И опять в паху, --- объяснил воин.
--- Хозяйство у него чересчур большое.
Попробую дополнительно укрепить.

Ситрис сделал ещё несколько стежков.

--- Когда я... только познакомился с твоей кормилицей, я сделал нехорошую вещь, --- тихо усмехнулся он.
--- Она в наказание заставила меня вышивать салфетки.
Целых полгода я сидел и шил проклятые салфетки.

--- А за что? --- удивился я.

--- Не суть, --- отмахнулся Ситрис.
--- Это было ужасно.
И кто из лесных духов дёрнул меня за язык сказать об этом в Храме?
Теперь обшиваю всех.
Шью всё, что угодно, но только бы не салфетки.
Иногда Кхохо или Эрликх подкладывают мне в постель квадратные лоскутки и цветные нитки, а я их за это бью.

Ситрис продолжил работу, тихо добавив: <<Уроды пресноводные>>.
Но по его лицу бродила слабая довольная улыбка --- спокойное занятие воину явно нравилось.

\section{[1+] Тысяча платков}

--- Когда я пришёл к твоей кормилице, чтобы она ходатайствовала за меня в Храме, я знал, что рассказывать.
Сообщил ей о том, что я был разбойником, что желал бы спокойной жизни.
Она, не слушая меня, задала только один вопрос --- что из сделанного мной я считаю самым постыдным.

Однажды во время налёта на караван Ситрис, приставив нож к горлу предводительницы, потребовал отдать всё, кроме одежды.
При обыске разбойник нашёл вышитый платок из грубой ткани.
Предводительница прошипела, чтобы он держал руки подальше от платка.
Ситрис рассёк ей лицо двумя ударами, оставив на лице женщины уродливую широкую улыбку.

--- Кхотлам сказала, что представит меня Храму, если я вышью тысячу платочков.
Каждый день в течение почти полного года я сидел и вышивал платки...

Лицо Ситриса перекосилось.
Он рассказывал это уже в двадцатый раз.
Каждый раз рассказ доставлял ему жуткий дискомфорт, но замолчать, как видно, ему было очень трудно.

--- Я ел за столом твоей кормилицы, спал возле её очага и вышивал эти платки...

--- Тысяча --- это очень много, --- сказал я.
--- Кхотлам просто хотела посмотреть, сколько ты продер...

--- Я их вышил, Ликхмас.
Всю тысячу, --- перебил меня Ситрис, похлопал по плечу и ушёл в темноту.

\section{Сдерживатель темперамента}

\ml{$0$}
{--- Иногда мне кажется, что моё единственное предназначение --- сдерживать темперамент Кхохо, --- буркнул Ситрис.}
{``Sometimes it seems my only destiny is to restrain \Kchoho{}'s temperament,'' \Sitris\ grumbled.}
\ml{$0$}
{--- Если бы мы не встретились, Кхохо уничтожила бы мир к свиньям и устроила бы богам истерику, что мир так быстро сломался.}
{``If we hadn't met, \Kchoho{} would have destroyed the world and made an uproar to gods because it wasn't stable enough.''}

\section{[1] Загнанный зверь}

Сирту-лехэ долго и упорно обучал меня искусству общения со зверьми.
И только сейчас я понял: тело --- такой же зверь.
Оно думает само, и нужно большое искусство, чтобы подружиться с ним и войти к нему в доверие.

<<Помни, Ликхмас: всё твоё искусство может оказаться бесполезным, если объятый страхом олень понесёт.
Но ещё более опасен зверь, которого загнал ты сам, ибо ты над ним более не имеешь власти>>.

\section{Хэма}

Хэма удерживается в волосах острой шпилькой, напоминающей стилет.
Этой шпилькой вполне можно было ранить или убить;
однако, если её вытаскивали, то тяжёлая заколка хэма немедленно падала на пол.
Это было первым напоминанием: если дипломат берётся за оружие, он тут же перестаёт быть дипломатом.
Вторым напоминанием был вес хэма, заставлявший держать голову высоко поднятой;
лицо дипломата --- не только его собственное лицо.

\section{Оружие и дипломат}

--- Почему ты не взяла оружие?
Хоть ты и дипломат, тебе нужно...

--- Нет, Лисёнок.
Или одно, или другое.

--- Послушай...

--- У купцов, как и у Храма, тоже есть традиции, малыш, --- перебила меня Кхотлам.
--- И я по мере сил стараюсь их чтить.
Меня учили, что дипломат должен всё, включая собственную безопасность, обеспечивать словом, взглядом и жестом.
Если ты этого не можешь --- распишись в собственном непрофессионализме.

\section{[1] Вопрос}

Кхарас перехватил моё ружьё.

--- Кхохо уже там, --- сказал он.
--- Не трать стрелы.

Кхохо действительно была уже там.
Она невозмутимо сидела на песочке;
воины хака весело переговаривались, не замечая присутствия чужого.

--- Как поплавали? --- поинтересовалась воительница на языке хака.

Воины обернулись как ужаленные и бросились на неё с трёх сторон.
Кхохо молниеносно выбросила вверх руки.
Песочек, оказавшийся мелкой сухой пылью, совершенно скрыл воительницу и нападавших;
когда пыль осела, я увидел пять окровавленных тел и отряхивающуюся Кхохо.

--- Придурки, --- буркнула она.
--- Я же просто спросила!

\section{[2] Кукольный театр}

Весело пылал костёр, Чханэ заливисто смеялась, лёжа на раскинутом плаще из кожи нимелто.
Её обнажённая грудь подпрыгивала, оливковая с оранжевой искоркой кожа блестела, как бронза, от огненных всполохов.
Доспехи и оружие валялись рядом в беспорядке.

Несмотря на спешку, привал нам сделать пришлось --- по дороге попался колодец.
Мы напоили измождённых, жалобно глядевших на нас оленей, напились сами и омыли усталые тела в предварительно подогретой воде.
Вернее, мылся только я.
Чханэ, забыв про кишащий идолами лес, брызгала на меня водой и громко хохотала.
Потом я подбросил в огонь отгоняющих насекомых пряных трав, и мы с девушкой поочерёдно прыгали через костёр, прямо в удушливый дурманящий дым, пока не пропахли им насквозь.

Я рассказывал ей интересные случаи, произошедшие со мной в других мирах.
Рассказывал легенды и мифы давно ушедших и забытых народов.
Читал стихи, предварительно установив <<мост сознания>>, дабы она могла понять их смысл и оценить их красоту.
Но ни забавные истории, ни стихи не оказали такого действия, как кукольный театр, который я видел в своём родном мире.
Сделав пять куколок из дерева и тихонько играя на флейте, я заставил их воспроизвести старый-престарый спектакль <<Жених драконихи>>.
Чханэ смеялась и хлопала в ладоши, как ребёнок.

--- Ещё, ещё!

Я показал ещё один спектакль, <<Апельсиновый сад>>.
Радости девушки не было предела.
Глядя на её сияющее лицо и влюблённые глаза, я в очередной раз подумал: <<Бедняжка...>>

--- Ты совсем не изменился, Лис.
Прости.
Я напрасно на тебя накричала тогда, в Тхитроне.

--- Хаяй, а я что тебе говорил? --- улыбнулся я.
--- Ничего и не изменилось.

Медленно светлело усыпанное звёздами небо.
Задул первый утренний ветерок.
Чханэ принюхалась к воздуху и некоторое время смотрела на меня со странной, грустной и извиняющейся, улыбкой.

--- Лис, прости меня, --- тихо проговорила она.
--- Я понимаю, что мы теряем время, понимаю, что твои дела чрезвычайно важны.
Но это было необходимо, потому что, сердцем чую, нескоро мы ещё так отдохнём.
Может, вообще в последний раз...

Она запнулась и ещё немного помолчала.

--- Давай поспим вместе до рассвета.
Ты уже сутки глаз не сомкнул, я же знаю.

Я с нежностью посмотрел в печальные огнистые глаза.
Нельзя, Чханэ, нельзя мне спать.
У меня столько работы...

--- Я могу долго обходиться без сна, Змейка.
Ложись.

--- Змейка тебя ждёт, --- игриво усмехнулась девушка.

Я откинул входной клапан палатки.
Чханэ вздохнула, встала, мимоходом погладив меня по шее, затащила внутрь всю свою амуницию и некоторое время возилась внутри, устраиваясь поудобнее.

Следовало ещё поискать Анкарьяль.
Я не надеялся на успех, чересчур много случайностей происходит.
Её могли убить в бою, принести в жертву, наконец, она могла совершенно банально умереть от болезни.
Демон не был абсолютной защитой от житейских бед, и с риском приходилось мириться.
Но поискать следовало в любом случае.

Я слегка напряг запястье, и браслет ожил.
В палатке зашебуршились, входной клапан отодвинулся, и на моих плечах повисла сияющая Чханэ.

--- Змейка тебя не дождалась.
Хай, а что это?

--- Прибор.
Ищет демонов.
Нужно найти Анкарьяль.

--- Анкарьяль --- женщина? --- строго спросила Чханэ.

--- Она настроена на женское тело, --- ответил я с улыбкой.
--- Неужели наследница самого Маликха подхватила на болоте ревность?

--- Брось-брось-брось, --- Чханэ замахала руками на голубые значки и таблички, и программа тонко чувствующего прибора пустилась в безумный пляс.

--- Хэ, Чханэ, не хулигань! --- я выключил браслет.
--- Тебе колыбельную спеть?

--- Мне не нужна колыбельная, если не на чем спать, --- надулась девушка.

--- А плащ?

--- Он не греет.

--- Он и не должен греть, он должен хранить тепло!

Чханэ вздохнула, расстелила плащ на земле и села передо мной на колени.

--- Я тебя совсем перестала привлекать, да?

Тьфу ты.
Вон оно что.
Вечно я забываю какую-нибудь важную мелочь, когда дело касается психологии.

--- Ты теперь не Лис, а Акхатху, --- девушка была серьёзна.
Она не спрашивала, а утверждала.
--- Сохранилось ли в тебе что-то из прежних чувств?
Любишь ли ты меня?

Как растолковать ей, что чувства --- в большинстве своём производные плоти, важной, но не обязательной части меня?
Объяснения лишь умножат её сомнения.
Ответ на её вопрос может быть только один...

--- Конечно.

Не так уж я и слукавил.
Я всем сердцем желал этому существу счастливой жизни.
Прежний, непробуждённый я жертвовал статусом, здоровьем и своей жизнью, чтобы всего лишь облегчить муки Чханэ.
Чем я отличался от него?

--- Тогда удели мне время до рассвета.
Мне одной.
Не людям, не войне, не вашим хоргетам.
Мне.
Пожалуйста.

Чханэ, не дожидаясь ответа, приникла к моим губам, как умирающий от жажды пустынник.
Я ответил лаской, как мог.
Мир вокруг завертелся.
Вокруг неё.
Вокруг нас.

Рассвет мы встретили в палатке в объятиях друг друга.
Я смотрел на её лицо, угадывая черты далёких предков, о которых она даже не подозревала.
Я смотрел на её лицо и видел тысячу других людей.
Молодых, старых.
Мужчин, женщин, детей.
Тысячи, десятки тысяч жили в этом простом, нежном контуре щёк, иссушённых горячими ветрами уголках миндалевидных глаз, твёрдых сомкнутых губах.
И было во мне лишь одно желание --- любоваться ею, как сейчас, любоваться всегда, вечно, а не те жалкие двести дождей, отпущенные этому телу...

Рассвет я встретил другим.

\section{[2] Гора Песнопений}

--- Что здесь произошло? --- ошеломлённо спросила Тхартху.

Чханэ же трясло от гнева.

--- Святилище осквернено...

\textspace

--- Святилище было гарантией мира.

--- А это значит, что грядёт великая война.

\section{[2] Воля Ликхмаса}

--- Что с нами сделали наши тела? --- пробормотала Анкарьяль.

--- Я не знаю, --- покачал я головой.
--- Знаю одно --- воля человека по имени Ликхмас ар'Люм была сильнее воли Аркадиу Люпино.
И этот факт сотворил настоящее.

--- Кто ты сейчас?

--- Я --- это я.
Это всё, что я могу тебе сказать.

\section{[2] Рабы закона}

--- Они хотят полностью подчинить сели их законам, --- сказала Анкарьяль.
--- Расчёт прост.
Раб закона --- раб того, кто найдёт в законе прорехи.

\section{[2] Вирус (идея)}

Клетки, поражённые вирусом, часто ведут себя так.
Вирус заставляет их делиться, выживать любой ценой, прорастать окружающие ткани.
И в результате стремление к выживанию оборачивается катастрофой для всех --- если раковую опухоль не удалить, она убьёт организм.

Так где же она --- грань между поведением поражённых вирусом клеток и нормальным желанием жить, иметь детей и защищать родных?..

\section{[2] Запасной Король}

--- Кое в чём Картель просчитался, --- сказала Анкарьяль.
--- Они не ожидали, что сели используют их собственную тактику --- <<держи ресурсы в мусорной куче>>.
Да и я, честно говоря, впервые вижу, чтобы обычный крестьянин полноценно заменил Короля-жреца.
В прочих мирах это происходит, да, но в основном стихийно --- появляется талантливый человек, который учится на своих ошибках.
Здесь же имеет место направленное обучение.
Человека учили быть запасным!

\section{[2] Необычное имя}

--- Двор Тхитрона, Манэ и Ликхэ ар'Люм.

--- Благодарю, --- кивнул я.
--- Только одно замечание: не Ликхэ, а Лимнэ.

Жрец вгляделся в записи и охнул.

--- Король-жрец, прошу меня простить.
Имя очень необычное.

--- У моей кормилицы хорошая фантазия на имена, --- улыбнулся я.
--- Купчихи ар'Люм --- мои сёстры.
Продолжайте, пожалуйста.

\section{[2] Эффект Борка}

Я вдруг вспомнил храм Тхитрона, бешеную схватку возле алтаря, демона Картеля, поглощавшего страдания умирающей Чханэ...

--- Эффект Борка, --- сказал я вслух.
Анкарьяль запнулась.

--- Прости?

Я рассказал о произошедшем со мной в храме.
Грейсвольд нахмурился.

--- Стратег с околонулевой устойчивостью отклонил твою атаку?

--- Именно отклонил! --- подхватил я.
--- Не отбил и не увернулся, а отклонил, даже скорее рассеял!
Ситуация по сути редчайшая, чтобы демона атаковали во время питания...
Вот кто из демонов до этого момента принимал эффект Борка всерьёз?

--- Как минимум Картель, который иногда использует армии сапиентов против Ада, --- сказала Анкарьяль.

--- Иногда! --- почти выкрикнул я.
--- Эффект Борка использовал на Драконьей Пустоши я, даже об этом не подозревая!
С тех пор ни одного случая не было зарегистрировано!

Друзья переглянулись.

--- Борк Песчаный Мост сделал своё открытие на пять тысяч лет позже битвы на Серпенциару, --- продолжил я свою мысль.
--- Каков шанс того, что мою победу обработали ретроспективно?

--- Просто скажи, что ты предлагаешь, --- раздражённо бросила Анкарьяль.
--- Плюс-демоны не могут использовать эффект Борка.
Для этого нужны особые сапиенты...

--- Например, потомки тси? --- подсказал я.

\section{[2] Нгвсо и ноа}

--- Это Барабан, --- сказал ноа.
--- Жди, я с ним поговорю.

Из воды показались три глаза и раздвоенное щупальце.
Человек замахал руками.
Щупальце исчезло под водой и вытянуло сплетённую из водорослей сеть, полную искристых зеленых раковин.
Ноа принял ценный груз и, вытащив из бедренного мешка другую сеть, передал её существу.
Нгвсо, аккуратно обвив сеть щупальцем, неторопливо свернулся в родную стихию.

--- Нгвсо выращивают этих моллюсков на еду, а раковины продают нам за бесценок, --- объяснил ноа.
--- Видимо, под водой ракушки выглядят не такими красивыми, как на воздухе, иначе нгвсо ценили бы их куда выше жемчуга.

--- А что ты ему дал?

--- Пирожки с вишней.
Я очищаю вишню от костей, заворачиваю в тесто и слегка обжариваю в масле, чтобы пирожки не слипались.
Нгвсо млеют от этого блюда.

--- Я слышала, что нгвсо приносят не только ракушки, --- полувопросительно заметила Чханэ.

Ноа пожал плечами.

--- Мои товарищи берут у них и жемчуг, а некоторые платят нгвсо за очистку днищ кораблей.

--- И между вами не возникает никаких разногласий?

--- Нгвсо и ноа --- хороший союз, --- наш собеседник снова пожал плечами.
--- Мы защищаем их от стрелохвостов, они предупреждают нас о вторжениях и лазутчиках.
Торговля опять же.
Ладно, путники, мне пора.

Ноа, похоже, потерял к нам всякий интерес.
Схватив посох-зонт и мешок, он неторопливо отправился в сторону рыбацкой деревни.

\section{[2] Пёсьи головы}

--- А на западе живут люди с пёсьими головами, --- сказал старик-ноа.

--- Они называются <<пылерои>>, --- весело ввернул кто-то.

--- Да что ты знаешь! --- вспылил рассказчик.
--- Пылерои --- это двуногие собаки, которые живут, как звери, в пустыне.
А на западе --- люди с пёсьими головами.
Они умеют говорить, едят варёное мясо, пишут мудрёные знаки и умеют их читать.
Разве собака умеет читать?

\section{[2] Мотивы Хэмингуэя (кусок)}

Да, я уже чувствовал это когда-то давно.
Драконья Пустошь, гавань славного свободного города Фриза.
Южное солнце, пальмы, крикливые торговцы, развесёлые караваны цыган, обольстительные красотки на каждой улице.
Корабль-трибот, мерно покачивающийся на волнах.
Фрукты, креветки и проститутки приедались уже через неделю, но это не могло надоесть --- ощущение, что морской ветер ласково гладит тебя по голове, словно так рано ушедший отец, который (я в этом не сомневался) любил меня больше жизни.

\section{[2] Рыбак}

--- Знаешь, Король-жрец, чем меня восхищает жизнь? --- из кармана старика появился крохотный орех.
--- Взгляни на него.
Он кажется таким же мёртвым, как камень.
Даже если с небес упадут снега или огонь, даже если море затопит сушу, его суть не изменится ни на волос.
Он будет подобен камню.
Но стоит растаять снегам, стоит погаснуть огню, стоит морю обнажить хоть маленький пятачок живой почвы --- орех даст росток и вырастет в куст, словно ничего не случилось.
Верю ли я, Король-жрец, что старый мир победит?
Я это знаю!
Но победит не тот старый мир, о котором ты говоришь, а другой --- который не знал ни Безумного, ни дельфинов, ни иных думающих из плоти и костей.
Так было и будет.

\section{[2] Хрусталь}

--- Когда-то давно я посетила малый храм Тхартавирта --- Хритра.
Сейчас его уже нет --- было большое землетрясение, и здание сложилось, как лист пергамента.
Тогда же... я не знаю на Короне ничего прекраснее того храма.

На башни подниматься чужим было строго запрещено, но я уговорила молодого жреца.
Он получил свои любимые восточные финики, а я --- ключ от башни.

Окна в башнях были закрыты толстым хрусталём.
Делал хрустальные листы знаменитый мастер, и я не могу передать, насколько они гладкие и чудесные.
Если их не освещали факелы или лампы --- их не существовало.

Я поднялась почти на самый верх и присела на широкий подоконник.
Был вечер, и молодой жрец ещё не зажёг факелы в башне.
Я отлично помню чувство, которое меня охватило.

--- Что это было за чувство? --- спросил я.

Митхэ пожевала губу.

--- Чувство, что я сижу на головокружительной высоте и ничто не отделяет меня от пропасти.
При этом никакого страха не было.
Удивительно, правда?
Храм стоял на Зелёной скале, Тхартхаахитр лежал передо мной, как на ладони, даже Трёхэтажный казался пирожным.
На такой высоте должен был быть жуткий ветер, но хрусталь глушил все звуки.
Вот такое странное чувство --- чувство высоты и застывшего перед тобой мира.
Я вспомнила это, когда потеряла Атриса.
Когда к лесным духам уходит единственный человек, способный тебя понять, ты начинаешь смотреть на этот мир через толстый, в десятую пяди толщиной хрусталь.
Перед тобой разворачиваются сражения, льётся кровь, люди тебе что-то кричат... а ты ничего не слышишь, ничего не чувствуешь и не боишься.

\section{[2] Два касания}

--- Я всё равно пройду, --- заявил посланец.

--- Попробуй, --- предложила Кхохо.

Чужак, Кхохо и Ситрис схватились за оружие одновременно.
Три сложных росчерка --- и воины тхитронского Храма попятились.
Ситрис хмыкнул, шмыгнул носом, клинки расчертили воздух --- и снова два шага назад.
Сабля Кхохо нервно подрагивала;
Ситриса, похоже, удерживала на месте только необходимость помогать подруге.

Вдруг посланец замер и тупо уставился перед собой;
из кустов вышла Анкарьяль и лениво снесла ему голову.

--- Живые? --- осведомилась она.

Воины кивнули.

--- Поединков с демонами не устраивать, --- распорядилась Анкарьяль.
--- Лучше толпой, а ещё лучше --- яд и стрелы.

--- А если вот так, бежать, что ли? --- возмущённо развела руками Кхохо.

--- Если жить хочешь, --- лаконично ответила демоница.
--- Воины, наверное?
Какой Храм?

--- Тхитрон, --- буркнул Ситрис.
--- Он двигался...

--- Да это нелюдь какой-то! --- взорвалась Кхохо, обвиняюще тыча пальцем в обезглавленный труп.
--- Как можно было предсказать \emph{такие} финты?!
У меня вся жизнь перед глазами промелькнула!

--- Что ты с ним сделала? --- спросил Ситрис.
--- Он словно оцепенел.

--- Дуэль хоргетов, --- пояснила Анкарьяль.
--- Хоргет умер --- тело осталось без наездника.

--- И они все такие?

--- Этот не самый умелый, демон-новичок.
Столкнись вы с кем-то из Манипулы Смеха --- даже я бы вас не спасла.

Ситрис и Кхохо нахохлились и обречённо переглянулись.
Анкарьяль окинула их оценивающим взглядом.

--- Зайдёте в палатку командования ближе к вечеру, дам вам людей под начало.
Не каждый переживёт два касания в схватке с демоном.

Анкарьяль ушла.
Ситрис и Кхохо долго молчали, не глядя друг на друга.
Наконец Ситрис нарушил молчание:

--- Мы тогда послали его на верную смерть.
Ликхмаса...

--- Заткнись, --- огрызнулась Кхохо.
--- Без тебя поняла.

--- Я бы сейчас выпил.

--- Мы в походе.

Ситрис молчал.

--- Всё, хватит, --- буркнула Кхохо.
--- Прошлое --- в прошлом.

--- Мы обречены.

--- Угу.

\section{[2] Нежности Кхохо}

--- Я ему говорила, чтобы он уходил, --- бросила Кхохо.
--- Я давно уже пропащая.

--- Я с тобой, --- сказал Ситрис.
--- Я знаю, что для убедительности следует дать тебе затрещину, по-другому ты не понимаешь.
Но я не хочу.
Сейчас, может быть, в последний наш день, я хотел бы только целовать тебя и говорить всякие нежности.

--- О том, как ты меня любишь?
А то я не знаю, --- грустно пробормотала Кхохо.

--- Не знаешь, --- сказал Ситрис.
--- Насколько я тебя люблю, знаю только я.

--- Печальная судьба --- любить акулу в человеческом обличье, --- заметила Кхохо.

--- По-моему, поздно об этом сожалеть.
Так что продолжай быть собой.
Можешь меня поколотить.

--- Я впервые в жизни не хочу тебя бить, --- сказала Кхохо.
--- Пойдём лучше на берег и посидим в воде.
Ты будешь меня целовать и говорить всякие нежности.

--- Ты мне уступила? --- изумился Ситрис.
--- Не иначе как сам Удивлённый Лю завтра прилетит.

--- Не знаю, кто там завтра прилетит, но я точно умру, --- уверенно сказала Кхохо.
--- Я тебя обещала ещё убить, помнишь?
И убью.
Правда, не совсем понимаю, как это у меня получится.
Ладно, время покажет.
Пойдём, нам ещё поспать надо, а сон перед смертью --- пустая трата времени.

\section{[2] Сокровище Высших}

Огромные металлические ворота были оплавлены.

--- Сколько металла... --- завистливо ухнул Грейсвольд.

--- Мы стережём это место, --- сказал Высший.
--- Здесь спрятана мудрость.

Я подошёл и погладил зеркально чистый оплавленный край ворот.

--- Техногенная катастрофа, --- резюмировал я.
--- Судьба <<Стального Дракона>> их ничему не научила.
Грейс...

--- Ты будешь смеяться, но здесь нет ни одного работоспособного устройства, --- как-то чересчур весело откликнулся Грейсвольд, предупреждая мой вопрос.
--- Взрыв уничтожил все наноструктуры.

--- Они охраняли пустоту, --- заключил я.
--- Давай скажем бедолагам, что мы --- те самые боги и что мы забрали всю мудрость.
А то они ещё пятьсот поколений здесь проторчат.

--- Дождёмся культурологов, пусть они это поаккуратнее сделают, --- предложил Грейс и тронул свесившуюся с потолка трансмиссию.
Мёртвый, слегка покорёженный механизм печально закачался, но скрипа мы так и не услышали.

\section{[2] Трава Тумана}

--- Когда-то давно я очень любил курить траву тумана, --- задумчиво сказал мужчина.
--- Плохая привычка, знаю.
Те, кто имеет пристрастие к этому цветку, не живут долго.
Я знал всё это, знал, чем может это закончиться --- траву курил мой кормилец и его предки.
И всё равно однажды так случилось, что я набил этими ароматными листьями деревянную трубку и сделал первый глоток дыма.
Вскоре я начал кашлять и часто болеть, ноги и руки мои ослабели, но привычка овладела мной --- я не мог провести без травы тумана и дня.
Меня стали избегать женщины, от крепкого запаха моей одежды фыркали даже ездовые олени.
И как-то был годовой поход, во время которого я не мог достать своё любимое зелье.
За это время я вернул свою прежнюю силу, моё дыхание стало чистым.
Но мысль о траве тумана преследовала меня постоянно.
И когда я вернулся в родной храм, то первым делом нашёл среди вещей старую трубку с расколовшимся концом и пакет с сухой травой.
Набивая трубку, я делал это нарочито медленно.
Меня терзали сомнения.
Должен ли я?
Готов ли я пожертвовать ласками женщин и силой своего тела ради этого?
Когда огонь коснулся коричневых с золотым отливом листьев и я сделал первый вдох дыма, то впервые за долгие годы ощутил его вкус.
Это было открытием --- курение давно стало для меня повседневностью, и я даже не знал, насколько вкусна трава тумана.
Следующим открытием стал запах.
Я почувствовал в горьковато-пряном дыму нотки, которых не различал ранее.
Я раз за разом вдыхал дым, и вкус и запах его слабели, уступая место знакомому чувству опьянения.
И вместе с этим пришло понимание.
Я ли выбрал жребий, который привязал меня к этой привычке?
Или, может быть, этот жребий выбрал мой кормилец или его предки?
Наверное, есть вещи, которые мы не можем изменить.
В моих силах лишь выбирать, что важнее --- сила в ногах, женское внимание или чувство умиротворения, которое приходило с белым как снег травяным туманом...

--- Чушь городишь, --- поморщился крестьянин, сидящий по другую сторону костра.
--- Ты это выбрал сам и пытаешься обвинить в своём выборе мироздание.

--- Я не отделён от мироздания, я его часть, --- развёл руками мужчина.
--- Я --- потомок моих предков.
Было бы наивно полагать, что я не зависим от окружающего меня мира.

--- Мы чересчур мало знаем, --- поддержал его кожевник Эрликх.
--- Древние лечили даже печали, и средства у них были куда сильнее травы тумана и молитвенных маков.
Жрец, который учил меня читать и писать, говорил, что печали --- это болезнь.

--- Тхэай, жрецы много чего говорят, --- бросила женщина с закутанным в одеяло ребёнком, по говору крестьянка с севера.
--- У них всё болезни.
Трусость --- болезнь, печаль --- болезнь, влюблённость --- болезнь.
Только ни один жрец не умеет от них избавить, да и кутрапов казнят, а не лечат.
Какой толк в знании, если его нельзя применить?

Я промолчал, внезапно ощущая прилив гордости за этот народ, к которому теперь --- самую каплю, разумеется --- принадлежал и я, уроженец Драконьей Пустоши.

\section{[1] Винт}

Я задумчиво смотрел на медленно вращающийся винт, заключённый в хрустальную трубу. В моей голове зрел вопрос...

--- Учитель Трукхвал.

--- Да, Ликхмас? --- библиотекарь удивлённо поднял голову.
--- Ты с операции?
Всё хорошо?

--- Да, она выжила, слава лесным духам.
У меня к тебе вопрос.

--- Ооо, хай-хай.
Слушаю.

--- У нас на балконе в системе подачи воздуха стоит винт.
Возможно ли с его помощью летать, отталкиваясь от воздуха?

Учитель заулыбался.

--- Хай, вон что тебя заинтересовало.
Да, можно, вполне.
Садись, расскажу, --- учитель с некоторым облегчением отложил в сторону книгу, которую собирался переписывать.

Я сел.
Трукхвал тем временем выудил две чашки и зачерпнул из котелка тёплой воды.

--- Ммм, винт.
Да.
Вообще полёт --- не такая уж хитрая вещь.
Мой учитель строил машинки с теми самыми винтами --- и они летали.
Хоть и есть у сели поговорка, что рыбы не летают, но однажды он ради смеха привязал к одной из них карпа, перед этим поспорив с половиной Храма, что заставит рыбу летать, --- учитель скрипуче засмеялся, обнажив желтоватые кривые зубы.
--- Весь народ потом сбежался глядеть на жрецов, которые стояли на площади на головах...
Хай, отвлёкся я.
Это были маленькие машинки, размером с птицу согхо.

--- А большие?

--- С большими проблема.
При увеличении размеров вес растёт, и дерево не выдерживает.
Нужен кукхватр, да где его столько взять?
Пришлось бы пустить в плавку клинки половины воинов сели, чтобы мог летать один человек!

Трукхвал неожиданно разговорился.
Видимо, летающие машинки бередили его душу уже очень давно.
Вскоре он достал кусок пергамента, сурьмяной мелок и начал чертить сложные схемы.
Я по мере сил пытался вникнуть.
Вдруг, прервав рассказ, Трукхвал хлопнул себя по лбу:

--- Хай, да что я угощаю тебя голыми травами да нравоучениями.
Держи-ка печенье, Ликхэ принесла, ты после операции голодный как ягуар...

\section{[M] Болотная лихорадка}

Кхатрим подозвал меня.
Я склонился над телом.

--- Смотри, Ликхмас, --- тихо сказал жрец.
--- Ты должен запомнить это на всю жизнь.

Кхатрим вынул нож и одним ударом вскрыл ребёнку грудную клетку.
Сердце и лёгкие были странного цвета --- оранжево-красные, слегка опалесцирующие в рассеянном свете солнца.
В нос ударил гнилостный запах.

--- Это не кровь, --- констатировал я.

--- Правильно, --- кивнул Кхатрим.
--- Кровь здесь тоже есть, но её не так много.
Согласно записям предков, это города мелких существ.
Настолько мелких, что они похожи на налёт.
Один жрец, по слухам, изобрёл устройство, позволяющее их видеть, состоящее из линз.
Однако на его город напали.
Жрец погиб, его записи сожжены, и мы не знаем доподлинно, правдивы ли слухи.

--- Сколько их здесь?

--- Я не скажу тебе даже приблизительно.
Не тысячи и не миллионы, много больше.

--- Взрослые могут заболеть?

--- Раненые и старики.

--- Как убить этих существ?

--- Их нельзя раздавить, нельзя покалечить, наши инструменты для этого чересчур грубы.
Их можно только отравить.

Кхатрим вынул из кармана робы бутылочки.

--- Винная эссенция, экстракт мха-ползуна и хлебная плесень.
Винной эссенцией нужно протирать все инструменты и руки, когда имеешь дело с болотной лихорадкой.
Экстракты мха и плесени помогают при приёме внутрь.

--- Можно ли вводить их сразу в лёгкие и сердце?

--- Мы пробовали.
Больные умирают ещё быстрее.
Только так.

Кхатрим укрыл тело простынёй и встал.

--- Раньше всех детей и стариков, если появлялись больные, отводили в храмовые рощи.
Благородный баньян, может быть, ты видел такие деревья?

--- Только слышал.

--- Воздух в этих рощах целебный, существа гибнут там очень быстро.
К сожалению, возле Тхитрона нет ни одной.

--- Храмовая роща помогла бы этому ребёнку?

--- Этому --- нет, --- покачал головой Кхатрим.
--- Если лихорадка зашла далеко, убитые существа начинают гнить внутри, и при лечении отказывают почки.
Дети начинают мочиться кровью, позже всё равно умирают.
Я научу тебя определять безнадёжных, им нужно дать Чёрного Сана и сжечь тела.
Кровавый плащ меняй каждый час, робу утром и вечером.
Перед тем, как ехать в Тхитрон, искупаешься вон там, растворы для втирания в тело приготовят.
Ты бы, конечно, это знал, если бы больше внимания уделял чтению.

Я поперхнулся.

--- Книга в библиотеке.
Отдел медицинской литературы, четвёртая полка, <<Ползучая болезнь>>, Тепло-Полуночного-Костра.
Ты её хотя бы открывал?

Я насупился.
Книга показалась мне скучной с самой первой страницы.

--- Как я узнал? --- Кхатрим без улыбки смотрел на меня.
--- Сама книга тонкая, в тридцать страниц.
Остальная часть --- коротенькое обещание следовать написанному в точности и подписи прочитавших её жрецов.

Я не страдал склонностью к чувству стыда, но тут у меня запылало лицо.
У Кхатрима определённо был стиль --- он рассказывал всё, что требовалось знать, подробно отвечал на любые вопросы, а после этого стыдил за неприлежание так, что покраснел бы сам Удивлённый Лю --- хлёстко и коротко.

--- Ликхмас, я хочу, чтобы ты понял.
<<Ползучая болезнь>> --- одна из важнейших книг по врачеванию.
Все ритуалы, все правила поведения при эпидемиях прописаны на этих тридцати страницах.
Словом, как вернёмся, тебя ждёт экзамен.
Или несколько экзаменов, пока я не удостоверюсь, что ты всё усвоил.
А теперь пойдём, нас ждёт тяжёлая ночь.

\section{[1] Смена пола}

--- Расскажи про смену пола, --- попросил я.

--- Хай, --- задумался учитель. --- Явление это очень, очень редкое.
Но условия его известны давно.

Учитель встал на ноги, прихрамывая, вышел в центр комнаты и театрально развернулся ко мне.

--- Представь, что группа взрослых мужчин оказалась отрезанной от внешнего мира.
Скажем, они плыли на корабле и попали на необитаемый остров.
Они вынуждены жить там длительное время.
Понятное дело, что размножаться они не могут, так как женщин среди них нет.
И вот тут-то включается таинственный природный механизм --- несколько самых сильных мужчин начинают превращаться в женщин, чтобы племя могло воспроизводиться и существовать дальше.

--- Наверное, они очень страдают при этом, --- заметил я.

--- О да, --- откликнулся старик.
--- Их кости и плоть начинают гореть из-за быстрого роста, их охватывает лихорадка.
Эти люди нуждаются в момент превращения в особом уходе.
Может быть, именно поэтому природа отдала роль превращающихся самым сильным из группы.
То же самое, кстати, происходит с мужчинами, оказавшимися в изоляции.
Иногда такие мужчины превращаются в женщин и сразу беременеют --- это называется партеногенезом.

--- А Чханэ?

--- Твоя подруга, насколько я понял, ударилась головой в детстве.
Вероятно, её мозг повредился и механизм запустился сам собой.

--- Она сказала, что бесплодна, --- тихо сказал я.

Трукхвал с нежностью посмотрел на меня.
Такая откровенность растрогала сердце старика.

--- Учитель Трукхвал, ты столько всего знаешь.
Можно ли её вылечить от бесплодия?

Трукхвал виновато развёл руками.

--- Извини, Ликхмас-тари.
Сколько бы я ни учился, пойду по жилам джунглей дураком.

\section{[1] Плачущий ягуар}

--- Лис, скажи правду, почему ты меня спас?

--- Какая тебе разница? --- поморщился я.
--- Спас и спас.
Радуйся.

--- А если бы я была некрасивой и не понравилась тебе, выдал бы ты меня жрецам?

Я задумался.

--- Наверное, нет.

--- Почему?

--- А ты, Чханэ?
Ты хотела меня убить, как увидела.

--- Да, --- растерялась Чханэ.
--- Это нормально, разве нет?

--- Видимо, нет.
Твой вопрос был о том же.

Чханэ замолчала.
Похоже, она уже сама была не рада, что подняла эту тему.

--- Я... мне нужны были доспехи... --- начала она.

--- Отлично, --- поморщился я.
--- А если бы ты мне не понравилась, я был бы мёртв.
Обязательно поднесу пару конфет Хри-соблазнителю.
Или тебя остановило что-то другое?

Чханэ отвернулась и закуталась в одеяло.

--- Так почему ты мне поверила?
Почему не попыталась вонзить нож в спину, как тому жрецу?

--- Давай больше не будем об этом.

--- Это важно, --- настаивал я.
--- Мы спим бок о бок.

Чханэ легла на спину и сдула с лица непослушную <<рыбку>>.

--- Я расскажу тебе историю.

--- У тебя на всё есть истории, --- отмахнулся я.

--- Эта тебе обязательно понравится, --- заверила Чханэ.
--- Так вот.
Однажды к воротам Тхаммитра подошёл ягуар.
Огромный полуторагодовалый кот.
Стража уже собиралась застрелить его, но один из дозорных заметил, что на его лапе висел верёвочный капкан.

--- Ловушка? --- возмутился я.
--- Это бесчестная охота.
Кто промышлял таким?

--- Не мы, --- заверила Чханэ.
--- Возможно, это были звероловы с юга, мы не знаем.
Стражники позвали охотников.
Те были просто ошарашены, что кто-то решился на такую подлость.
Ягуар полулежал, готовясь в любой момент дать дёру, рычал на всех и всё же не уходил.
Наконец один из охотников сказал: <<Ягуар не случайно оказался у ворот.
Ему пришлось пройти с капканом по открытому нагорью, которое обычно избегают, и одним лесным духам ведомо, сколько он прошёл по джунглям>>.

Охотник решился и осторожно подошёл к зверю.
И в тот момент произошло невероятное --- ягуар заплакал и перевернулся на спину.
Капкан причинял ему невыносимую боль.

--- Что они с ним сделали?

--- Охотники взяли ягуара за лапки и принесли его в город.
Он не сопротивлялся.
Позвали жреца, и тот начал снимать капкан.
Кот плакал и повизгивал, но не делал ни одной попытки ударить или укусить тех, кто его держал.
А охотники, на счету которых была не одна шкурка, стояли и плакали, глядя на него.
Я была тогда ещё маленькой, но хорошо запомнила их лица.

--- Они спасли его?

--- Да.
Жрец снял ловушку, обмазал лапку целебными снадобьями.
А потом ягуар целых десять дней жил в полуразрушенной хижине, в которую его принесли.
Его не связали, не заперли.
И знаешь, что самое удивительное?
Год выдался плохим на мясо, зверьё ушло на север.
А всё равно многие приносили часть добычи ягуару, чтобы он ел.
И ни у кого даже мысли не возникло, что с него можно снять шкуру.
Охота охотой, но игру не по правилам поощрять не следует.
Потом, когда кот окреп, он просто вышел из хижины и прошёл через ворота к джунглям.

--- Поучительная история, --- заметил я.

--- Я ответила на твой вопрос?

--- И на свой тоже, --- кивнул я.
--- А что произошло потом, когда капкан был снят?

--- Потом я влюбилась, --- просто ответила Чханэ.
--- Или ты про ягуара?

Я пододвинулся к ней и нежно поцеловал её в губы.
Мои ноздри обжёг запах какао, идущий от её волос.
Огненные глаза превратились в полутьме в отполированные кусочки коричневого шпата.

--- Ну-ка, Змейка, хочешь сладких конфет?

Девушка улыбалась.

--- Хочу.
А какие конфеты у нас сегодня?

--- Да те же, что и вчера, --- тихо рассмеялся я и ткнулся носом ей в ухо.
--- Надеюсь, они тебе не надоели?

--- Ну что ты, --- игриво-укоризненно шепнула Чханэ, развязывая верёвочки на моей рубахе, и, не удержавшись, цапнула меня острыми зубками за плечо.

\section{[2] Причина в людях}

--- Значит, Картель решил начать с Храмов, --- произнёс Грейс.

--- Загвоздка в том, что Картель довершил катастрофу, но не устроил её, --- сказал я.
--- Жрец, которого я встретил, сказал, что Бродячие Храмы начали испытывать давление городских задолго до предполагаемого вторжения Картеля.
Видимо, дело всё же было не в демонах, а в людях.

--- Спустя столько тысяч лет?

--- На этот вопрос не стоит торопиться отвечать до подробного исследования.
А исследование можно провести только после войны, увы.

\section{[2] Математика}

\textbf{(Анкарьяль учит Тхартху рисованию)}

--- Нар, послушай, --- Тхартху задумчиво смотрела на листок бумаги.
--- Мне кажется, что вот эти квадратики равны по площади.

--- Что, Тхартху? --- Анкарьяль склонилась над девушкой.

--- Вот эти, --- Тхартху ткнула пальчиком в листок.
--- Я нарисовала несколько треугольников с... прямым углом и пририсовала к сторонам квадратики.
И площадь этого всегда равна площади двух этих.

--- Мужичьё, идите-ка сюда, --- в голосе Анкарьяль я различил лёгкий шок.
Мы с Грейсвольдом переглянулись и подползли поближе.

--- Что тут у нас? --- Грейсвольд ласково потёрся носом о щеку Тхартху и заглянул в её записи.

--- Похоже, она только что переоткрыла теорему о квадратах сторон треугольника на стандартной плоскости.

--- Вот, --- Тхартху снова показала на свой чертёж.
--- Треугольник и вот эти квадратики.
Их площадь.
Я нарисовала несколько разных треугольников --- результат один и тот же.

Мы с Грейсвольдом переглянулись и засмеялись.
Тхартху бросила на нас обиженный взгляд.

--- Ну что не так?

--- Не-не, Птичка, всё так.
Как ты поняла, что они равны?

Тхартху поджала губы.

--- Я выросла в хуторе.
Мерить землю поручают детям с ранних лет.
Я могу отличить, что больше, а что меньше.

--- Удивительно, --- Анкарьяль всё ещё завороженно разглядывала рисунок девушки.

--- Ну-ка, дай мне листок, --- я аккуратно выхватил его из рук Тхартху вместе с карандашом и наскоро нарисовал две фигуры --- квадрат и круг.
--- Какая из фигур больше по площади?

--- Вот эта, --- девушка указала на квадрат.

--- На сколько? --- хитро оскалился Грейс.

Тхартху нахмурилась.

--- На чечевичное зерно, не больше.

--- Вы тоже поймали дубинку головой\FM? --- поинтересовалась Анкарьяль.
\FA{
Калька фразеологизма, распространённого на Преисподней.
Значение --- сильно удивиться чему-либо.
У одного из народов была игра, смысл которой заключался в перебрасывании игроками тяжёлой дубинки.
Тот, кто зазевался, мог запросто получить травму.
}

--- Ага, --- хором ответили мы с Грейсом.

Тхартху засмеялась:

--- Что-что поймали?
Чем?

--- Дай-ка мне, --- Анкарьяль вырвала листок у меня из рук.
Карандаш я передал ей сам.
--- Тхартху, смотри.

Демоница изобразила на бумаге квадрат, а в него вписала квадрат поменьше.
Рядом изобразила ещё один квадрат и расчертила его.
Не успела Анкарьяль закончить второй чертёж, как Тхартху ахнула:

--- Я поняла!
Поняла!

--- Что поняла?

--- Вот эти треугольники, и да, эти квадратики!..
Ну это... да.

--- Ну это, да, --- повторил Грейс.
Анкарьяль отвесила ему звонкую оплеуху.

--- Главное, что она поняла, каменная башка.
Вот, Птичка.
Это называется <<доказательство>>.

--- А оплеуха --- <<обоснование>>, --- ввернул я и тоже отхватил затрещину.

--- Хай, не лупи моего мужчину, --- возмутилась Чханэ, оторвавшись от котелка с едой.
Демоница бросила на неё презрительный взгляд.
Чханэ ответила тем же, агрессивно повертев в руках черпак.

--- Так, всё, хватит, --- тут же вмешался я.
Отношения Анкарьяль и Чханэ накалялись с каждым днём.
--- Чханэ, это она по-дружески.

--- Я ей руки оторву за такое <<по-дружески>>, --- огрызнулась девушка.
--- При мне тебя никто бить не будет.

Анкарьяль посмотрела сначала на меня, потом на мрачно молчащую девушку... и благородно промолчала.
Я внутренне вздохнул.

--- Слушай, Нар, --- Грейсвольд смущённо поковырялся в траве и, поморщившись, потёр ухо.
--- Тяжёлая у тебя рука.
Если у неё так хорошо с площадями, может, ты ей посложнее что объяснишь?
Интегралы там...

--- Грейс, у неё с умножением туго, а ты про интегралы, --- покачал я головой.

--- Да, плохая идея, --- проговорила Анкарьяль.
--- Я как-то пыталась объяснить кое-кому дифференцирование.
Не вышло.
Это нужно с детства осваивать.

--- А что такое <<интегралы>>? --- аккуратно поинтересовалась Тхартху, с трудом выговорив слово языка Эй.

Её непосредственность была настолько неподдельной, что мы засмеялись во весь голос втроём.
Анкарьяль прикрыла лицо рукой:

--- Земля и небо, мои дарители...

--- Короче, Нар, у тебя нет выхода, --- сквозь слёзы пробурчал Грейс.
--- Давай интегралы.
А там, глядишь, и до физики пространств с градиентом мерности недалеко.

\section{[2] Величие Древних}

--- Эх, если бы у меня была ваша сила... --- мечтательно потянулась Чханэ и невзначай пощупала меня.

--- А что бы ты сделала, обладай ты нашей силой? --- спросил я подругу.

Чханэ сладко зевнула.

--- Я бы лечила людей... Учила бы детей... Да много чего можно придумать.
Те же плохие годы, с ними можно что-то сделать.

--- Да я бы не сказал, что сейчас всё так плохо с болезнями, --- заметил я.

--- Конечно плохо! --- Чханэ посмотрела на меня, как на дурака.
--- У нас в позапрошлом году троих детей съела лихорадка.
Четверо мужчин умерли от укуса смертожара.
По-твоему, этого мало?

Я улыбнулся.

--- Конечно, нет.
Просто я читал записи о Древней Земле.

--- Древняя Земля?
Что это?

--- Прародина человечества.

--- Хай, жрецы называют её Тхидэ.

--- Нет.
Тси-Ди --- ваша прародина --- тоже развитая цивилизация, но она не первая.
Это было раньше, намного раньше.

--- Ещё раньше? --- открыла рот Чханэ.

--- Есть данные, что тогда умирала половина детей, --- перешёл я к сути.
--- Были страшные болезни, которые заставляли людей гнить всю жизнь, и заразиться ими можно было во время соития.
А были не менее страшные, которые передавались по воздуху.
От них за декаду вымирали целые города.

Чханэ не ответила.

--- Мужчины могли умереть от простой царапины в те времена, а женщин часто уносили роды.

--- Да как такое может быть? --- возмущённо воскликнула Чханэ.
--- Это звучит, как какой-то кошмарный сон.
Безумный и вполовину не так жесток...

--- \ldots как природа, --- закончил я.

--- Почему же ничего этого нет? --- тихо спросила девушка.

--- Первые люди.
Жители той самой Древней Земли.
Они уничтожили возбудителей болезней, они с помощью генной инженерии избавились от генетического груза, который тащили миллионы дождей.
Одна человеческая жизнь --- пять поколений --- потребовалась, чтобы полностью вычистить генофонд человечества.

--- Генофонд --- это...

--- Наследственность.
То, что передаётся от предков к потомкам.
Но древние сделали не только это.

--- А что ещё?

--- Ты знаешь, что у нас в гортани есть звукопроизводящий орган?

--- Гортанная цитра?
Да, конечно.
Я видела лёгкие трупов.
А что с ней не так?

--- Для чего она нужна?

Чханэ задумалась.

--- Я не знаю.
Говорить, петь, --- Чханэ очень похоже изобразила свист птенца согхо, и в кронах ближайших деревьев немедленно отозвались взрослые птицы.
--- Разве нет?

--- Модуляцией голоса люди управляли сложнейшими машинами, для таких были непригодны даже самые нежные руки.

--- Только не говори, что у Древних не было цитры!

Я улыбнулся.
Чханэ издала губами неприличный звук и ударила руками по земле.

--- Лесные духи.
Наверное, их голоса были плоскими и невыразительными, --- Чханэ мяукнула оцелотом, всполошив окрестных птичек ещё больше.

--- За то, что ты и твои соплеменницы почти до конца беременности можете радоваться жизни, спокойно работать и даже сражаться, тоже благодари Древних.
Ах да, и за почти безболезненные роды, и за отсутствие менструаций тоже.

--- Отсутствие чего?

--- У древних женщин из матки каждые четырнадцать дней шла кровь.

--- Они истекали кровью? --- ужаснулась Чханэ.
--- Зачем?

--- Их тело работало по-другому.

--- Хаяй... --- Чханэ произнесла это таким тоном, словно разочаровалась в любимом герое легенд.
--- Нам говорили, что Древние были выше и сильнее нас...
А что ещё сделали Древние?

--- Нам мало известно про биологические особенности людей до Эпохи богов.
Я знаю лишь, что у них были одни зубы на всю жизнь.
Если потерял --- ходи без зубов.

--- Грустно им было, --- Чханэ проверила зубы пальчиком.
--- Мне как-то выбили резец, так давно уже новый вырос...
Девка поленом стукнула за то, что я строила глазки её мужчине.
А я её головой в ослиное дерьмо макнула... прямо на его глазах, --- Чханэ захихикала.
--- Скажи, Лис, а правда, что у людей бывают голубые глаза?
Как небо в зените.

--- Да, --- улыбнулся я.
--- Я видел даже людей с белоснежной кожей и волосами цвета соломы.

--- Это чудесно, --- мечтательно пробормотала Чханэ и ущипнула меня.
--- Я бы хотела увидеть таких.
А фиолетовые глаза бывают?

--- Бывают и фиолетовые.
На планете Запах Воды живут родственники пылероев с яркими фиолетовыми глазами.

--- А красные?

--- И красные.

--- Это чудо.
У нас только зелёные и карие.

Я подтянул девушку поближе и поцеловал её в висок.

--- Многим показалось бы чудом, если бы они узнали про оцелотовые глаза, --- прошептал я.
Чханэ смутилась.

--- У меня они уже позеленели.
Я вдалеке от родных мест.
Но всё же, Лис, --- в её голосе вновь зазвучала сталь, --- от болезней не должен умирать никто.
И я бы лечила всех.

--- Я тебя научу, --- пообещал я.

--- Хорошо, --- пробормотала девушка и уткнулась мне в шею.

Анкарьяль всё это время с отсутствующим видом ломала сухую веточку на щепьё и кидала его в огонь.
Вдруг она подала голос:

--- Научить-то научишь.
А с Безумным что делать будем?

--- Нар, давай потом, --- проворчал откуда-то из темноты Грейс.
--- У меня уже глаза слипаются.

--- Принять решение нужно быстро.
Если нас поймают и казнят, думать будет поздно.

--- Пусть казнят, только дай мне поспать.

--- Я вас не узнаю, --- проговорила Анкарьяль.
--- Вы стали относиться к заданию, как к прогулке в парк.

--- Разведка, сбор информации, --- я попытался расставить все чёрточки над иероглифами.
--- Но вначале --- спать.

--- Может, хватит уже? --- внезапно подала голос Тхартху.
--- У меня от ваших разговоров мороз по коже.
<<Строить машины>>, <<научить медицине Древних>>, <<уничтожить Безумного>>...
И это таким тоном, словно вы поесть собрались.
Вы хоть немного уважения и страха имейте!

Мы промолчали.
Грейс вздохнул --- похоже, он уже и сам жалел, что взял с собой подругу.

--- Чханэ, скажи им уже!

--- Они знают, что делают... --- начала Чханэ.

--- А я вот не уверена!
Ведут себя, как дети, спокойно говорят о вещах, которые я или не могу представить, или представлять их просто страшно!
Да и как воевать с тем, которого не видно и не слышно?

--- Тхартху, --- твёрдо сказала Чханэ, --- пусть ребячатся, мы всегда так делаем перед тяжёлым походом.
Я им верю.
Давай-ка спи.
И я буду спать.

--- Ты веришь всем, кроме Нар, --- вдруг выпалила Тхартху.
Анкарьяль удивлённо посмотрела на девушку.
Чханэ поджала губы.

--- Я верю и ей, --- медленно и чётко проговорила Чханэ.
--- Она заносчивая ящерица, но Лис и Карп не станут доверять кому попало.

На этот раз ошарашенный взгляд Анкарьяль настиг мою подругу.

--- И давай уже спи, Тхартху.

--- Можешь устроиться у меня на пузе, --- сонно предложил Грейс.

Тхартху едва слышно выругалась.

--- Угораздило меня с тобой связаться, кудрявая панда...

--- Пузо мягкое, --- заметил Грейс.
--- Аркадиу, так ведь?

--- Да уж помягче моего, --- вздохнул я.
Чханэ захихикала.

Тхартху демонстративно оттащила спальный мешок и устроилась рядом со мной и Чханэ.
Грейс вздохнул.

--- Ладно, спокойной ночи всем.

Наступила тишина.
Анкарьяль по-прежнему невидящим взором смотрела в костёр, скидывая в пекло оранжевых языков одну щепочку за другой...

\section{[2] Приманка}

--- Что не спишь, Кар? --- Грейс хлопнул меня по плечу.

--- Меня кое-что беспокоит, --- уклончиво ответил я.

--- Случайно не осцилляции ПКВ? --- усмехнулся Грейс.

--- Так ты тоже их почувствовал, --- произнёс я.

--- Самое интересное, не я, --- ответил Грейс и покачал своим многофункциональным браслетом.
--- Я бы даже не заподозрил, что что-то не так.

--- Что именно ты почувствовал? --- спросила Анкарьяль.

Я задумался.

--- Плюсовое искажение.
По идее, за счёт присутствия Безумных на поле должен быть перевес минуса, а тут... как на нейтральных планетах.

--- Локализация? --- требовательно произнесла Анкарьяль.

--- Нужны вычисления, --- резонно ответил я.

--- И не просто вычисления, а дешифровка, --- как-то злобно усмехнулся Грейс.
--- Компьютер твёрдо уверен в том, что вокруг центра планеты имеется плюс-облако.

--- В смысле --- облако?
Ты имеешь в виду сеть треков?

--- Это не треки, Нар.
Не обычный планетарный омега-фон.
Это сингулярность с неопределёнными координатами.
Поэтому я и говорю --- облако.

--- Это физически невозможно, --- констатировала Анкарьяль.
--- Дурит кто-то твой компьютер, Грейс.

--- Похоже на приманку, --- добавил я.
--- Грейс, больше не трать на \emph{это} энергию.
Мы с тобой и так уже... потратились.

Грейсвольд смущённо хмыкнул.

--- Кто бы это ни был, о нём у нас никакой информации.
Будем пока что исходить из того, что мы на Тра-Ренкхале одни.
Меня только интересует, каким образом было создано это облако?

--- Иллюзия восприятия, --- отмахнулась Анкарьяль.
--- Ты, Аркадиу, с такими технологиями ещё не сталкивался?
А мне приходилось.
Заходит с тыла отряд, как мы думаем, наших.
А они р-раз и по нам экранами лупить начинают.
Свечение от них плюсовое, да.
Как я живой оттуда ушла --- сама удивляюсь...
Тут интересно другое: какого червивого дьявола мы с Грейсом не чувствуем ничего?
Поле-то одно для всех!

--- Мне тоже, --- поддержал её технолог.
--- Какое-то устройство не просто изменяет поле, оно ещё и в курсе, где находится каждый из нас, и даже как-то нас классифицирует.
Аркадиу, чем ты отличаешься от нас?

Вопрос был риторическим --- я урождённый человек.

--- И возможно, единственный таковой на сотни парсак вокруг, --- продолжила мысль технолога Анкарьяль.

--- Необязательно, --- возразил я.
--- А как вы объясните то, что осцилляции видит компьютер?

--- Компьютер использует сапиентные паттерны поведения, чтобы запутать наблюдателя, --- сообщил Грейсвольд.
--- Кстати, эту идею ты мне подал.

--- Может, нам следует разделиться, --- предположил я.

--- Нет, --- отрезала Анкарьяль.
--- Возможно, что этого от нас и хотят.
Вряд ли это устройство Картеля, иначе нас бы уже давно взяли под белые рученьки, но сам феномен похож на тщательнейшим образом подобранную приманку.
Будем исходить из того, что кто-то знает о нашем местоположении... и вести себя так, словно мы ничего не заметили.

Мы с Грейсом переглянулись и вздохнули.
Тайна манила, мы отчаянно нуждались в союзниках, но Анкарьяль была, как всегда, права.
И даже более, чем всегда.

\section{[2] Счётная мельница}

--- Лис, а то твоё имя, --- Чханэ неуверенно назвала его, --- что оно значит?

Я задумался.

--- Arccadiu --- <<крестьянин>>, распространённое мужское имя.
Balerianu --- кажется, <<воин>>, baleru --- устаревшее <<воевать, сражаться>>.
А родовое имя Luppino ознaчает <<шакал>>.
Luppina --- <<самка шакала>> --- название для... --- я задумался, вспоминая подходящее слово на языке сели.
Не нашёл.
--- Хай, это женщина, которая занимается сексом с мужчинами за товары или еду.

--- А, --- поняла Чханэ.
--- Бывает такое, да.
А почему для них отдельное название?

--- В моём родном мире женщины не были равны мужчинам.
Их половая жизнь ограничивалась.
В качестве противовеса существовали женщины, которые продавали секс.
Общество их презирало.

--- Женщинам не давали заниматься сексом?

--- Да.
Общество считало предосудительным, если у женщины было более одного мужчины...
Что смешного?

Чханэ больше не могла сдерживать смех и захохотала во всё горло.

--- Хри-соблазнитель... а как... они... определяли?
Счётную мельницу вставляли в женские ворота?

Я подождал, пока она успокоится, и рассказал ей про девственную плеву.
Чханэ поперхнулась.

--- Это специально делали?

--- Нет, у женщин моего вида это было изначально.

--- Глупость какая-то.
Я вначале удивилась, когда ты сказал <<продавали секс>> --- это ж каким неприятным должен быть человек, чтобы никто даже не согласился с ним разделить ложе.
Нет, у нас тоже бывало, что кто-то обменивал вещи на секс, но это всё-таки больше символический обмен, удовольствие-то оба получают.
Впрочем, теперь всё понятно.
Женщины твоего мира не имели больше одного мужчины --- с ними спать, как лягушкам стихи читать.

На этот раз от такой неожиданной интерпретации поперхнулся я.

--- А у тебя когда первый секс?

Чханэ задумалась.

--- Именно любовь?
Или Круг Доверия тоже считать?
В Круге я участвовала с первого похода, дождей с двадцати трёх.
Любовь была чуть позже --- Манис.
Я тебе про него рассказывала.

--- Как прошёл твой первый Круг?

--- Примерно так же, как и у тебя --- трезвучие, куча гениталий и слабое понимание происходящего.
Правда, я была совсем молодой, поэтому за несколько дней до Круга кормилец позвал меня к себе, чтобы я чувствовала себя увереннее.
Так я стала женщиной, --- Чханэ тепло усмехнулась.

Я погладил девушку по обтянутой тканью коленке.

--- Где сейчас твой кормилец? --- спросил я.
--- Он жив?

Улыбка Чханэ увяла, словно молитвенный мак.

--- Хотела бы я знать.

--- Отчего так?

--- Когда кормильцы перестали жить вместе, мне было уже много дождей.
Меня готовили в Храме, поэтому я ушла с кормильцем в храм насовсем.
У Согхо бывала редко --- она не особенно меня любила, а после расставания с Акхсаром и подавно.

--- Почему ты думаешь, что она тебя не любила? --- удивился я.

Чханэ скорчила рожу.

--- Знаешь, это не так трудно понять, особенно маленькому человеку.

--- Так что насчёт кормильца?

--- Однажды в Храме сменилась власть.
Кормилец встал на защиту прежнего Первого --- он был его другом.
Ту ночь я провела у Согхо --- один из воинов, старый товарищ Акхсара, вскользь посоветовал мне пожить несколько дней вне храма.

--- Похоже, просто так подобные советы у вас не давали.

Чханэ многозначительно кивнула.

--- Кормилец пришёл далеко за полночь, весь в крови, с пробитыми доспехами.
Обнял меня и сказал, чтобы я вела себя в Храме так, как и всегда, и ничему не удивлялась.
А потом заплакал, сказал, что любит меня, и убежал.
Тех пятерых, кого он зарубил в ту ночь, я хоронила на следующее утро.

--- С тех пор вестей не было?

Чханэ вздохнула.

--- За ним послали убийцу, так что, скорее всего, мой кормилец уже давно прошёл жилами джунглей.
В иное я перестала верить.

\section{[2] Страшная сила}

--- Мне только одно непонятно, --- заявила Чханэ.
--- Вы --- самые сильные существа в мире.
Где ваше оружие?

Я рассмеялся.
Анкарьяль и Грейсвольд недоумённо переглянулись.

--- Оружие?

--- Да.
Летающие машины, извергающие огонь.
Ножи, прорезающие камень.
Доспехи, которые выдерживают выстрел из баллисты.
Где всё это?

Грейсвольд, кажется, начал кое-что понимать.
Анкарьяль выглядела ещё больше сбитой с толку.

--- Какие машины с огнём?
О чём ты?

--- У меня есть браслет, --- растерянно сообщил Грейс.

--- Чханэ, --- вмешался я.
--- Мы делаем оружие из воздуха.

--- Зачем нам машины с огнём, --- проворчала Анкарьяль.
--- У нас всегда было и есть два оружия --- знания и демоническая сущность.

--- И неизвестно, какое страшнее, --- присовокупил Грейс.

\section{[2] Песня о доме}

--- Атрис как-то спросил меня, как я выжила среди всех этих сражений.
А я мечтала о доме.
Многие мои товарищи мечтали о далёких походах, о новом оружии... а я о доме.
И вот, во время одного из походов дом у меня появился.

--- Помню этот поход, --- засмеялся Акхсар.
--- Золото притащила худого, оборванного, насквозь мокрого красавца и сказала: <<Это цитра моей жизни.
Не обижать>>.
После первой же песни мы поняли, что перед нами человек, путь которого начертан сохой небесного пахаря.
Мы словно переосмыслили свои жизни.
Многие плакали, говорливые вдруг надолго замолкали, а молчаливые, наоборот, начинали говорить, и их нельзя было остановить.
Золото и Хат отныне спали в отдельной палатке и называли её <<домом>>.
А почему домом, спрашиваем мы?..

\begin{verse}
Что отличает жилище?\\
Крыша, и пол, и огонь,\\
Пища на вечер и утро,\\
Две пары любящих рук,
\end{verse}

--- закончила Митхэ.
--- Атрис спел это не задумываясь, словно кто-то диктовал ему стихи...

Акхсар затянул весёлую песню, и Митхэ подхватила её:

\begin{verse}
Если нет крыши --- то это лишь лагерь,\\
Если нет пола --- то это нора,\\
Если нет пламени, если нет пищи,\\
Это сарай --- посели здесь кролей.\\
Если нет любящих рук и надежды,\\
Незачем крыша, и пища, и пол.
\end{verse}

Воины рассмеялись.

--- Мы пели её по десять раз, --- припомнил Акхсар.
--- А Хат играл нам на цитре.

\section{[2] Спокойная старость}

Хитрам вскоре нашёл меня.
В руках жрец сжимал испачканный в глине платок.

--- Моей хранительницы, --- сказал он.
--- Пришлось принести её в жертву.
Больше было некого.
И некому.

--- Оставь, --- я аккуратно отстранил его руку с платком.
--- Это чересчур большая ценность для меня.
Твоего жеста вполне достаточно.

Я отошёл, и вскоре раздался сдавленный возглас.
Хитрам-лехэ дрожащими пальцами расправлял совершенно чистый платок, на котором красовалась надпись: <<Самому лучшему дарителю>>.

Жрец никому не сказал ни слова.
Но где бы я его ни увидел, он не расставался со своим сокровищем.

Вскоре ко мне подошла Анкарьяль.

--- Аркадиу! --- тихо сказала она мне на ухо.
--- Это ты Хитрама разыграл?
Что ещё за дешёвые фокусы?

--- Пусть человек порадуется, --- сказал я.

--- Его хранительницы давно нет, --- сказала подруга.
--- Сапиенты умирают навсегда, и пристанище --- лишь сказки для живых.
Все знают о твоей божественной силе...

--- \ldots но никто не осведомлён о её границах, --- тонко улыбнулся я.

--- Хитрам --- образованный человек.
Если он узнает истину, это его убьёт.

--- Посмотри, --- кивнул я на старика.
Тот рассеянный взором смотрел в окно, мял в руках платок и улыбался.
--- Посмотри внимательно, Анкарьяль Кровавый Шторм.
Нужна ли этому усталому человеку истина?
Будет ли он искать опровержения для чуда, которое позволит ему дожить своё в умиротворении?

Анкарьяль смотрела на меня.
В её глазах застыло что-то непонятное.

--- Я передам Грейсу, чтобы он не распускал язык.

--- Благодарю тебя.

--- Не нужно, --- шепнула Анкарьяль и взяла меня за руку.
--- Я иногда жалею, что в этой Вселенной не найдётся доброй сказки, в которую могла бы поверить я.

Я улыбнулся и прижался к подруге.
Она, как всегда, читала мои мысли.

\section{[2] Внезапно}

--- Существует-Хорошее-Небо — травник? --- удивился старый жрец.
--- Мы всегда думали, что он человек!

--- Я поговорила с пылероями, --- добавила Анкарьяль.
--- Они рассказывают легенды про вождя Существует-Хорошее-Небо, и в них он пылерой.

--- Я почти уверен, что планты считают его плантом, --- заключил я.

Мы переглянулись и захохотали.
Жрец нахмурился.
Для него только что осознанное было откровением, и наш смех казался чем-то неуместным.

--- Я понял, --- сказал жрец.
--- Предки не делали различий между людьми, пылероями, травниками и няньками...

--- Делали, --- сказал я.
--- Для размножения они, разумеется, выбирали особей своего вида.
Но древних тси связывала общая культура, многочисленные дружеские и половые связи.
Ни у кого не возникало даже мысли, что другой вид, к которому относились его товарищи по работе, друзья и, возможно, даже любовники, чем-то хуже своего собственного.

\section{[2] Эпилог}

Вспомнился разговор с Чханэ.
Мы стояли на башне храма, и под нами расстилался засыпающий город.
На западе пылал закат.

--- Идём со мной.
Я сделаю тебя хоргетом, и мы будем встречаться снова и снова, проживать жизнь за жизнью.

Чханэ улыбнулась.
В углах её глаз уже рассыпались морщинки, тяжёлые волосы цвета какао сверкали платиновой филигранью.

--- Лис...
Я прожила вдвое больше своих родичей, и чувствую, что мне... много.

--- Ты не хочешь быть со мной?

--- Скажи, Лис, я сильная?

--- Конечно.

--- Я всегда такой была.
Позволь же мне напоследок эту слабость --- просто умереть.
Исчезнуть навсегда.

--- Я думал, все люди мечтают о бессмертии.

Чханэ провела рукой по волосам и посмотрела на горящий над джунглями закат, потом улыбнулась мне.

--- Я всегда мечтала о покое.
А бессмертие... я уже бессмертна.
Народ сели, пылерои и идолы будут помнить тебя, пока не умрёт последний старик, пока последняя старуха не отправится к лесным духам.
И, вспоминая тебя, рассказывая о тебе легенды, кто-нибудь да помянет меня добрым словом.
Наши потомки будут жить, пока не погаснет солнце и не остынут океаны, и в каждом будет течь капля моей крови.
Ну а после...
Ты же меня не забудешь?

--- А когда исчезну я?

--- Тогда мне здесь точно нечего будет делать.

\ml{$0$}
{--- Я напишу о нас книгу.}
{``I'll write a book about us.''}

Она засмеялась и схватила меня за плечи:

\ml{$0$}
{--- Книгу?}
{``A book?''}

\ml{$0$}
{--- Да.}
{``Yes.}
\ml{$0$}
{И нас будут помнить даже после моей гибели.}
{We'll be remembered even after my death.''}

\ml{$0$}
{--- Кто?}
{``But who will remember?''}

\ml{$0$}
{--- Кто-то да будет.}
{``Somebody will.}
\ml{$0$}
{Написанное может быть прочитано.}
{Written might be read.}
\ml{$0$}
{Может быть, её будут читать миллиарды, жители сотни планет.}
{I guess billions of folk, dwellers of hundred planets will read this story.''}

\ml{$0$}
{--- И мы будем встречаться снова и снова, проживать жизнь за жизнью в чьём-то воображении?}
{``And we will meet and meet again, we will live life by life in the mind's eye of somebody, won't we?''}

\ml{$0$}
{--- Да.}
{``Yes, we will.''}

Чханэ смотрела в закат, осмысливая мои слова.
Таких счастливых глаз я не видел никогда.
Помолчав, она кивнула и притянула мою руку к груди.

\ml{$0$}
{--- Я согласна.}
{``I accept it.''}

\chapter{Воин и менестрель}

\section{Три притока}

\epigraph{У крепкого стула три ноги, у полноводной реки три притока.}
{Пословица сели}

Правилу Трёх сели учили с детства.
Если тебе нужна чистая вода --- у тебя должны быть ручей, река и озеро.
Если тебе нужна пища --- у тебя должны быть поле, лес и рынок.
Если тебе нужна крыша над головой --- у тебя должны быть палатка, жилище и постоялый двор.

\section{Отравленный нож}

--- Меркхалон-кровохлёб, --- выругалась Митхэ и с сожалением взглянула на саблю.
Надпись на клинке скрылась под кровью, оставив на поверхности один иероглиф: <<Оставь меня>>.

--- Понимаю, --- буркнула Митхэ.
--- Ребёнок.
Хай, малыш!
Нет-нет-нет, стой, я хочу просто...
Да чтоб тебя.

Митхэ машинально парировала неловкий выпад девчонки.
И ещё один.
И ещё.

--- Да подожди ты.
Остановись.
Я хочу поговорить.

--- Слева, слева заходи! --- заверещала девочка.

Митхэ чересчур поздно сообразила, что попалась на простую детскую уловку.
Выскользнувший из кустов мальчишка успел оставить на боку Митхэ глубокую царапину --- за миг перед тем, как из его подмышки брызнул фонтан крови.

--- Братик! --- девочка бросила чересчур тяжёлую для её рук фалангу и подхватила умирающего на руки.

Митхэ тупо смотрела на ребятишек.
Вот снова от её руки погиб невинный человек.
В груди медленно нарастала щемящая боль.

Но почему болит ещё и в боку?
И почему мир вокруг завертелся?

--- Митхэ, Митхэ! --- Акхсар бил её по щекам.
--- Что с тобой?

--- Она отравлена, глупое ты бревно! --- рявкнула Эрхэ.
--- Хватит её лупить!
Быстро нагрей кочергу на костре!
Хотя брось, поздно уже, у неё всё в крови...
Нагрей воды!
И травы сюда тащи мои!

--- И чтоб ты сдохла, --- добавила девочка.
Она по-прежнему прижимала к себе окровавленное тело брата.
На её маленьком лице застыла скорбь.

\section{Висок}

Она прижала горячую чашу к виску и закрыла глаза.
Нежный жар действовал на неё успокаивающе;
по спине побежали мурашки облегчения.

\section{Слова потомков}

Так ли уж важно, что скажут современники и потомки?
Жизнь --- она здесь и сейчас, и по большому счёту она принимает тебя таким, каков ты есть.

\section{Борьба}

Борьба --- странная вещь.
Вначале тебе очень больно, и ты избегаешь её как можешь.
Потом жизнь снова вынуждает тебя принять бой, и ты борешься, мечтая о покое.
Как вдруг в один прекрасный день ты понимаешь, что без борьбы жизнь не имеет смысла.
И ты идёшь сквозь метели, стонешь от ран, проклинаешь богов и духов, но иначе жить уже не можешь.

\section{Мимолётное}

Митхэ и Атрис шли рядом и думали об одном и том же --- о том, как мимолётны поцелуи.
Впрочем, Митхэ больше вспоминала о произошедшем, пыталась усилить, запечатлеть в памяти;
Атрис же с головой ушёл в философию.
Казалось, всего михнет назад с людьми происходило что-то невероятное --- стучали сердца, руки и губы плясали в слепом танце, полном запаха волос, дыма, пота, феромонов... и вот любовники идут, сбивая ноги о камни, и между ними снова непреодолимая стена, сложенная из пространства-времени, обязательств и стереотипов.

Порой Атрис и Митхэ обменивались взглядами.

<<Хочу ещё>>, --- молили глаза женщины.

<<У нас достаточно времени впереди>>, --- успокаивал её менестрель.

<<Ты сам-то в это веришь?>>

Атрис не верил.
<<Сейчас>> в его жизни давно одержало победу над <<потом>> --- в тот самый день, когда цитра заменила дом, поле, мастерок и даже самую малость --- пищу.

\section{Гибельная красота}

Воин от природы обладал скоростью, которая недостижима большинству.
Он мог ловить голыми пальцами ружейные дротики и стрелы.
В спарринге из всего отряда чести с ним мог сравниться разве что Ситрис --- несмотря на недостаток скорости, разбойник очень быстро соображал, куда дует ветер, и неплохо предсказывал движения противника.
Ещё Акхсар был очень красив --- тонкие хищные черты лица, маленькие, но яркие зелёные глаза над лезвиями точёных скул, белые зубы, пегая грива, перевязанная нитками и украшенная костяными погремушками.
Подступающая старость, как ни удивительно, делала его ещё красивее.
Женщины и мужчины ходили за ним толпами;
однако Акхсар всегда стоял рядом с Митхэ, не обращая внимания на восторженные взгляды.
Все считали его заносчивым, и лишь отряд чести знал --- воин, прошедший самые известные сражения современности, боялся поклонников как огня.
Митхэ была его щитом.

--- Может, ты всё-таки ответишь кому-нибудь взаимностью? --- тихо спросила как-то Митхэ.
--- Тебе понравится.

--- Их любовь --- любовь хищников и коллекционеров, --- грустно буркнул воин.
--- Это не то, что нужно человеку в моём возрасте.
Так что давай все будут считать, что я безответно влюблён в тебя.

--- А ты в меня влюблён? --- усмехнулась Митхэ, стрельнув глазами.

Впрочем, она тут же об этом пожалела.
Акхсар вдруг сжался, словно испуганный ягуар, и затравленно посмотрел на командира.

--- Снежок, тихо, тихо, я пошутила, --- тут же испуганно пояснила Митхэ, похлопав друга по плечу.
--- Пошутила.

Однажды Митхэ услышала от него совсем грустные слова:

--- Будь прокляты те, кто наделил меня такой привлекательностью.

--- Да что ты такое говоришь! --- всполошилась Эрхэ, пролив на себя похлёбку.

--- Знаешь, в чём отличие между тем, когда тебя грубо лапают, и тем, когда на тебя постоянно пялятся?

--- Ну?

--- По рукам можно надавать.
Больше отличий нет.

--- Ты же сам прихорашиваешься перед зеркалом!
И эти твои ленточки-погремушки!

--- Для себя, Обжорка, не для других!
Это две большие разницы!

\section{Видевший смерть}

Митхэ взглянула в глаза Атриса.
Они были чисты, как небо.
<<Видевший смерть>> --- так называли сели подобный взгляд.
Не ту смерть, что целится в тебя из лука, не ту, с которой можно договориться, не ту, от которой можно уклониться ловким движением ног.
Эти глаза видели неотвратимую, неумолимую смерть, которая ползла ядом по венам, которая разгоралась в лёгких пламенем болотной лихорадки и лишь в последний момент, по какой-то странной ошибке Вселенной отошла в сторону.
Митхэ знала, что у неё тот же взгляд.

\section{Что такое любовь}

--- Я не знаю, что такое любовь, --- призналась Митхэ.

--- Любовь --- это желание, чтобы близкий жил и процветал.

--- Или хотя бы не мучился, --- добавила женщина.

--- Вот видишь, это просто.
Все стихи и песни можно сократить до четырёх слов --- жизнь, процветание и милосердная смерть.

\chapter{Подпольщики}

\section{Консерватизм}

--- Интересно, куда в этом случае делся обычный для военных консерватизм, --- проворчала Анкарьяль.
--- Когда я захотела сменить имя --- с той мерзости, которой меня нарекли на Капитуле, на моё нынешнее, --- легат прямым текстом заявил мне, что это нежелательное действие, так как оно вызывает сомнения в моей лояльности.
<<Как вы будете лояльны Ордену, если вы не можете хранить верность своему собственному имени?>> --- спросил он.
А когда я спросила, с какой стати я должна хранить верность набору звуков, мне пригрозили пенальти.

\section{Двойная верность}

--- Грейсвольда проверили уже несколько раз.
Всё чисто.

--- Но вы всё равно не можете в это поверить, верно? --- подхватил голос.
--- Грейсвольд верен Ордену, и это факт.
И всё же, спроси я вас, служит ли он Скорбящим, что бы вы ответили?

--- Да, --- хором сказали Штрой и Самаолу.

--- Наши протоколы и тесты разрабатывались для совершенно конкретной ситуации --- противостояния Ордена Преисподней и Красного Картеля.
То, что хорошо для одного --- плохо для другого.
Наши тесты просто не допускают ситуации, что возможна деятельность во благо двум и более организациям.

--- Он верен Скорбящим и Ордену \emph{одновременно}? --- спросила Штрой.

--- Вам это кажется удивительным?

Штрой и Самаолу переглянулись.

--- Это было бы крайне сложно обнаружить.
Особенно если одна из организаций законспирирована.

--- На этом коньке Грейсвольд и въехал в ряды контрразведки, --- подхватил голос, --- с нашей вольной и невольной помощью.

--- Есть какие-то версии, каким образом это осуществляется? --- безжизненным голосом осведомился Самаолу.

--- О, Самаолу, у меня есть множество самых диких предположений.
Одно из них, например, заключается в том, что Грейсвольд \emph{сам} уверен, что он работает на Орден.
При этом он бессознательно посещает явки Скорбящих и оставляет информацию для них.

--- Что-то вроде МПС\FM?
\FA{
Маркерные поведенческие стереотипы.
}

--- Именно.
Но не тела, а демона.
Как бы демон, спящий в демоне.

Самаолу потянулся к поясу:

--- Я немедленно прикажу...

--- Ни в коем случае, это бесполезно.
Проверка ничего не даст, вы только зря распугаете рыбу.

--- Какой демон добровольно согласится быть марионеткой? --- ошарашенно спросил Самаолу.

--- Уверенный в своей правоте, --- мрачно сказала Штрой.

Самаолу непонимающе посмотрел на неё.

--- Я читала досье Падальщика, --- пояснила Штрой.
--- Он получил оцифровку, пытаясь помочь соплеменникам.
Ситуация абсолютно идентичная.

--- Именно, --- сказал голос.
--- Теперь вы оба понимаете, кому Грейсвольд верен на самом деле.
И масштаб угрозы, исходящий от этих фанатиков.

--- Фанатики-пацифисты, --- ухмыльнулась Штрой.

--- Это действительно очень забавно, Штрой, --- я таких ещё не встречал.
К счастью, оружие против них крайне простое.
Нужно смоделировать ситуацию, в которой интересы Ордена будут противопоставлены интересам Скорбящих.

\section{Пафос}

--- Манэ, почему ты выбрала такое странное прозвище?

--- Я родилась слишком поздно, --- ухмыльнулась Манэ.
--- Пафосные прозвища все разобрали.

\section{Ярлык}

--- По убеждениям я скорее аристократ, --- ухмыльнулся Самаолу.

--- Остерегайтесь придерживаться <<убеждений>>, Самаолу, --- сказал голос.
--- Если личность навешивает на себя ярлык, то велика вероятность, что вместе с прогрессивными идеями она потянет и весь причитающийся идеологический мусор.
Придерживаться <<убеждений>> --- это всё равно что носить с собой мешок руды вместо кошелька с монетами.

--- Но то же самое и с выбором стороны в войне, --- заметила Штрой.
--- Вы можете ни словом, ни делом не навредить врагам, но вас покарают за ваше знамя.

--- Штрой, вы определённо меня сегодня радуете, --- восхищённо ответил голос.
--- Пожалуй, я буду ходатайствовать о вашем повышении.
И ещё, как мне кажется, текущее дело --- не для вас.
Что скажете?

--- Вам виднее, --- поклонилась Штрой.
--- Но я бы хотела довести его до конца.

\section{Интуиция}

--- Вы не понимаете суть человеческой интуиции.
Это тонкая обработка поступающих в мозг сигналов.
Всё, что находится за пределами чувствительности и экстраполяции чувственных данных, --- не интуиция, а галлюцинации.

--- Прикрывшись этим определением, вы упустили из виду ширину диапазона чувствительности тси и их базовый уровень интеллекта.
У интуиции есть границы, но определить границы интуиции тси я пока затрудняюсь.
Тси учили пользоваться интуицией.

\section{Монограмма Стигмы}

Стигма рассеянно водила пером по бумаге.
Письмо и рисование доставляли ей невероятное удовольствие.
Вот на бумаге появился человечек в причудливой одежде.
Стигма медленно и аккуратно начала заштриховывать открытые участки.

Её рисунки проверяли десятки тысяч раз на наличие кода.
Не нашли.
Трудно найти код там, где его нет.
Она просто очень любит рисовать.

Заштриховав человечка, Стигма аккуратно вывела в углу свою монограмму.
Знаки-подписи были в моде у древних тси, и Стигма в подражание им создала собственный.
Пришлось, конечно, порыться в архивах, но результатом стратег была довольна --- знак достаточно точно отражал склад её личности.
Острый как стилет клин, похожий на щит полукруг, игривый хвостик, который менял длину в зависимости от настроения Стигмы.
Перечеркнула сооружение решительная, неторопливая горизонтальная черта.
Красивый рисунок и никакого кода.

Да, никакого кода.

\section{Проблема 48}

--- Да, и ещё, --- сказал представитель.
--- Мы настаиваем на том, чтобы отдел биологии пересмотрел своё решение по поводу строительства...

--- Кажется, я ещё не высказывался по этому поводу? --- осведомился Атрис.
--- Если доклад Кольбе и Рабе, подтверждённый расчётами Корхес, не является для вас весомой причиной, вот вам мой ответ --- нет.
Пожалуйста, занесите его в протокол под моей цифровой подписью.

Представитель скривился, но тут же, спохватившись, натянул официальное лицо.

\ml{$0-[ej]$}
{--- Как вы знаете, планета Тра-Ренкхаль не обладает необходимой инфраструктурой.}
{``As you know, there's no necessary infrastructure on the planet Tr\r{a}-R\={e}nkch\'{a}l.}
\ml{$0-[ej]$}
{Поэтому...}
{Therefore---''}

\ml{$0-[ej]$}
{---  ... и поэтому её нужно превратить в пустыню? --- закончила Митхэ.}
{``---therefore, it should be turned into desert?'' \Mitchoe\ finished.}
--- Тра-Ренкхаль --- это не стоящая на солнце чаша.
Это котелок над огнём, и этот котелок не закипает лишь благодаря удачному стечению обстоятельств.

\ml{$0-[ej]$}
{--- Боюсь, я не совсем понимаю вашу аналогию.}
{``I'm afraid your analogy was not understood.''}

--- Съездите как-нибудь в Вялую степь по южной дороге, ведущей из святилища Тёплый Двор, --- сказала Митхэ.
--- Вы увидите то, что называют Корзиной Сельвы --- десять кхене тонких, сухих, плотно переплетённых деревьев, которые быстро нарастают под дыханием Ху'тресоааса, а затем так же быстро умирают и рассыпаются в прах под палящим ветром Пустошей.
Затем посмотрите на запад, где виднеются Дикобразовы горы.
Это тот щит, благодаря которому джунгли ещё живы, а реки полноводны.

\ml{$0-[ej]$}
{--- Если Пустоши обогнут Хребет Дикобраза, он перестанет быть защитой, --- сказал Атрис.}
{``If the Deadlands get around the Hedgehog Spine, there will be no shield anymore,'' \Aatris\ said.}
--- Площадь сельвы сократится вдвое только от этого.
На Водоразделе появится ещё одна степь.

--- Я вас уверяю, --- улыбнулся представитель, --- что наши заводы отвечают требованиям экологичности.
Были проведены многочисленные исследования в условиях сотен планет, в том числе и с весьма сложными экологическими условиями.

--- Вы помните последствия <<Проблемы 48>>, --- сказал Атрис.
--- Океан на Капитуле вышел из берегов, затопив тысячи мелких городов.
На Тра-Ренкхале последствия будут куда серьёзнее --- расширится площадь Смертных пустошей, а джунгли продвинутся к Хрустальным землям, уничтожив уникальные реликтовые виды.
Наземные строения не могут быть экологичными по определению.

--- Как вы знаете, на Капитуле эта проблема была решена --- посредством полузеркального щита и высокотехнологичных систем теплоотведения.

--- Я также знаю, что на Капитуле эта проблема \emph{возникла}, и знаю, из-за чего, --- отрезал Атрис.
--- Я против строительства новых городов и заводов на поверхности, равно как и любого другого вмешательства в экосистему планеты.
Старые города останутся в их текущем виде как культурное наследие, в строгом соответствии с указом номер вы-знаете-какой.
Тем не менее, я окажу всё посильное содействие для строительства подземных городов, подобных тем, что строили в экваториальной зоне последние тси.

--- Я вас понял, Атрис, --- сказал представитель.
--- Я доставлю ваше сообщение куда следует.
Позвольте откланяться.

Кольбе и Рабе проводили его взглядами.

--- Мне это не нравится, --- сказал Кольбе.

--- Они всеми правдами и неправдами проталкивают это решение, --- сказал Рабе.

--- Очень смахивает на попытку провести мелиорацию, --- сказал Кольбе.

--- Без санкции Капитула, --- закончил Рабе.

--- Этого не будет, пока я жив, --- сказал Атрис.

\section{Кольбе и Рабе}

--- Война --- участь тех, кто обделён разумом, --- сказал Кольбе.

--- Война --- удел тех, кто не умеет летать, --- сказал Рабе.

--- Мы никогда не примем войну сапиентов, --- сказал Кольбе.

--- Вселенная бесконечна, --- сказал Рабе.

--- Есть много мест для жизни и процветания, --- закончил Кольбе.

--- Для жизни и процветания, --- хором повторили Атрис и Митхэ.

Братья улыбнулись.

\section{Смерть Сиэхено (отрывки)}

--- Ты --- тси, --- сказала Сиэхено.
--- Твои предки были умны и мягкосердечны.
Это есть и в тебе.

--- Я --- сели, --- ответила Чханэ.
--- Я так же умна, но, на твоё несчастье, ещё и умею быть жестокой.

Сиэхено поёжилась.

--- Чего ты хочешь?
Возьми и оставь меня в покое.

--- Нам нужен визор, --- перешла к делу Чханэ.
--- Ты --- лучший визор в Ордене.
Переходи на нашу сторону.

--- Я верна Ордену, --- сказала Сиэхено.
--- А вы хотите его уничтожить.

--- Мы не хотим уничтожать ни Ад, ни Картель.
Мы хотим мира.
Но для этого нам нужно, чтобы с нами считались.

--- Почти то же самое, в других выражениях, говорил один человек с Древней Земли, --- заметила Сиэхено.
--- Он снял кровавую жатву.
Его слова после повторяли многие, и каждый оставлял за собой горы трупов.
Для миротворца в тебе чересчур много обид.
Ты познала милосердие и жестокость, но ещё не знаешь их границ.

\textspace

--- Если вы связываетесь с агентами в Аду, то как? --- промурлыкала Сиэхено.
--- Ах да, домики... песчаные домики... я ведь подозревала, что в них скрывается код.

Сиэхено снова запела.

--- Я за всё детство придумала только один, --- пожаловалась она.
--- Я его совершенствовала, но никогда кардинально не меняла форму.
А тут детишки лепят совершенно разные домики... словно их кто-то этому учит...

--- Неплохо, --- признала Чханэ.
--- Именно поэтому ты нам нужна.

Да, Сиэхено была нужна Скорбящим.
И Чханэ передёргивало от одной мысли о сотрудничестве с этой тварью.

\textspace

Чханэ почувствовала, что её пробирает ледяная дрожь.
Демоны не зря отгораживались от собственных тел.
Сиэхено всеми силами искала уязвимость в обороне интерфектора.
Модули Анкарьяль пока спасали, блокируя опасные эмоциональные реакции, но факт оставался фактом --- это встреча равных, а не охотника и жертвы.

<<Следуй моему плану, --- сказал Аркадиу, --- никакой самодеятельности>>.

Беседа давно уже вышла за рамки плана.
Чханэ поняла --- ей \emph{придётся} убить Сиэхено, если она её не убедит.
Что могло убедить визора?
Рассказать ей всю подноготную легенды, чтобы она осознала, в какой переплёт попала, и тем самым уничтожить эту самую легенду?

<<Лис, лесные духи, почему ты не предупредил, с кем мне придётся иметь дело?!>>

--- Малышка, --- ласково сказала Чханэ, и Сиэхено неподдельно вздрогнула, испугавшись такой неожиданной перемены.

--- Ты всё ещё надеешься надавить на мои чувства, интерфектор?

--- Я не хочу на тебя давить, --- прошептала Чханэ.

--- А что ты сейчас делаешь? --- осведомилась Сиэхено.
--- Тебе придётся меня убить.
Ты проиграла ещё сто десять секунд назад.

--- Я не хочу победы, дитя.
Я хочу мира.
Как и мои предки, я хочу только мира.

--- Опять ты за своё.
Мы всё обсудили...

--- Я никогда не ударю в спину другу, --- сказала Чханэ.

Сработало безотказно, как когда-то это сработало на Лусафейру.
Сиэхено заколебалась, затем её личико снова стало умиротворённым.

--- Вдвоём против целого мира?
Увольте.

--- Это лучше, чем против того же мира одной.

И снова в яблочко.
Сиэхено выронила бусы и судорожно начала искать их на столе.
Чханэ подняла бусы и опустила их в ладонь девушки.

--- Ты всё равно убьёшь меня, --- прошептала Сиэхено.
--- Если я не соглашусь, ты меня убьёшь.

<<Вот она, грань между доверием и недоверием>>.
Если сейчас отпустить Сиэхено, либо она приобретёт союзника, либо часть плана развалится, как карточный домик.

\section{Гениальная игра}

--- Метритхис, --- проворчал голос.
--- В эту игру играет весь Ад.
Кто-то даже говорит, что она гениальна --- она проводит явную параллель между квантовой физикой и сапиентным обществом, между струнами и индивидами.
Я тоже нашёл игру интересной, однако надоело на каждом углу слышать её термины.

\section{Первая любовь}

Двери за Штрой закрылись, и она, сбросив с себя платье, небрежно перешагнула через него.
Следующими на пол упали штаны, затем о холодный пластик клацнул пояс, стилизованный под секхвим народа сели.
С каждым брошенным предметом одежды пропадала сексуальность Штрой, сменяясь измотанностью.
В кабинет входила женщина из плоти и крови;
под душ встала нагая бесплотная тень.

Штрой казалось, что струйки воды проходят сквозь кожу, не трогая её рецепторов и не отдавая ей ни джоуля тепла.
Она вывернула ручку душа на максимум и тут же повернула обратно, зашипев от боли.

Ей вдруг вспомнились давние времена, когда она только начала служить Аду.
Перебежчики обязаны некоторое время отработать в исследовательских отделах, и жаждущую показать себя демоницу отправили в отдел Корхес Соловьиный Язык.
Не прошло и нескольких дней, как новое тело дало сбой --- Штрой по уши влюбилась в свою начальницу.

Корхес всегда отличалась работоспособностью.
Её жизнь была расписана по минутам.
К ней часто приходили демоны из других отделов --- посоветоваться наедине.
Корхес отмеряла советы на квантовых весах, её яркие голубые глаза светились спокойствием и надёжностью, каждое её слово было вовремя и к месту.
Штрой слушала шелестящий, с металлическим оттенком голос, закрыв глаза и прикусив губу.
Она думала о руках Корхес, о губах Корхес, о спящей Корхес, о смеющейся Корхес --- и она не могла думать ни о чём другом.
Демон пытался восстановить контроль над телом, но безуспешно --- эмоции захватили и его, гигантские вычислительные ресурсы начали уходить на примитивные любовные иллюзии.
Эффективность Штрой неумолимо снижалась, риск потерять место с каждым днём возрастал.

--- Корхес, у тебя есть минута?..

\ml{$0-[ej]$}
{--- Так, Штрой, реши для себя раз и навсегда.}
{``Now, Stroji, make your choice once and for all.}
\ml{$0-[ej]$}
{Ты зачем сюда пришла?}
{Why are you here?}
\ml{$0-[ej]$}
{Работать?}
{Work?}
\ml{$0-[ej]$}
{Какие с этим проблемы?}
{You got a problem with that?}

\ml{$0-[ej]$}
{--- Так ты знаешь?..}
{``You knew ...!}

\ml{$0-[ej]$}
{--- Я не слепая!}
{``I'm not blind!''}

Штрой, бывший легат Красного Картеля, с полными слёз глазами топталась на месте.
Она чувствовала себя глупой никчёмной девочкой, никогда прежде её не посещало такое глубокое ощущение беззащитности.
Её губы шевелились, но из них не вылетало ни звука.

--- Всё, хватит, оставь эти мысли, --- строго сказала Корхес.
\ml{$0-[ej]$}
{--- От любви ещё никто не умер.}
{``Nobody's died of love.}
Если шалят гормоны --- поищи развлечений где-нибудь вне лаборатории, а лучше --- обратись к врачам.
Мне пора, увидимся.

<<У этих идеальных демонов, которые вписываются в фигурную дырку общества, всегда виноваты гормоны.
Всегда>>.

--- Пока, --- выдавила Штрой, глядя в спину навсегда уходящей мечты.
Её сердце лежало на холодном полу, пища, как брошенный котёнок.

<<И вот теперь я в игре, --- думала Штрой.
\ml{$0-[ej]$}
{--- Я могу испортить твоё расписание, раздавить твой уверенный голос и пустить твою жизнь под откос.}
{I can rewrite your schedule, crush your confident voice, and derail your life.}
Разумеется, она из Скорбящих.
\ml{$0-[ej]$}
{Дура, милая дура!}
{Fool, you precious fool!}
\ml{$0-[ej]$}
{Прояви ты тогда хоть каплю нежности --- я бы осталась пешкой в твоём отделе, и ничего бы этого не случилось!>>}
{If you could be a little gentle to me, I'd stay a pawn in your sector, and none of that would ever happen!''}

Штрой яростно отвесила себе пощёчину.
Спустя несколько минут она вышла из душа;
её глаза блестели, как и раньше, а губы кривились в прежней усмешке.

<<Ты хочешь, чтобы я работала, Корхес?
Ты это получишь>>.

\asterism

--- Могу я узнать причину, по которой вы избегаете разрабатывать линию Корхес?

--- Я нашла более подходящий путь.
Я сообщу вам детали, как только он будет готов.

--- Штрой, я хочу вам напомнить одну вещь.
Либо вы играете в игру со всем усердием, либо вы в неё не играете.

--- Я займусь Корхес, --- безжизненным тоном предложил Самаолу.

--- Нет, --- рявкнула Штрой.
--- Прошу прощения, владыка, --- добавила она.
--- Я не отказываюсь от линии Корхес.
Я просто прошу отсрочку, чтобы разработать ещё и добавочный путь.

\section{Ещё один секрет игры}

--- Я расскажу вам ещё об одном секрете Метритхис, --- сказал голос.
--- При должном мастерстве игроков в неё можно играть вечно.
Собственно, достигнуть истинного равновесия Нэша.
Несмотря на все старания тси, сели были плохи в математике и этого не знали.
Хотя по Тра-Ренкхалю ходили легенды о партиях длиной в год, десять и даже шестьдесят лет.

\section{Беседа Гало и Тахиро (отрывки)}

Тахиро ухмыльнулся.

--- Сколько у нас осталось времени, Тахиро?

--- До прорыва последнего круга обороны осталось...

--- Двадцать шесть минут, ---подхватил Гало.

--- Двадцать пять и пятьдесят четыре, ---поправил Тахиро.

Собеседники грозно переглянулись и рассмеялись.
Тахиро пододвинул стул поближе.

--- Удивительно, не правда ли --- тысячи наших демонов сейчас гибнут, чтобы у двух обречённых бездельников было двадцать шесть минут на разговоры?

Гало улыбнулся.

--- В тебе всегда была глупая потребность превращать жизнь в спектакль.
Оставь этот драматизм, наш разговор в любом случае ничего не решит.

--- Возможно, --- пожал плечами Тахиро.
--- В таком случае давай насладимся чаем.

--- Хорошо, --- согласился Гало и взял кружку.

\textspace

--- Ты когда-нибудь был женщиной? --- неожиданно спросил Гало.

--- Да, --- ухмыльнулся Тахиро.
--- Мы с Айну как-то ради эксперимента выбрали тела противоположного пола.
Айну заскучала из-за чрезмерной открытости, а я загрустил из-за плохо продуманной системы контроля.
Айну сказала, что я дикая истеричка, а я упрекал её в холодности.

Гало захохотал.

\textspace

--- Нужна самодостаточная система, смыслом существования которой была бы не война, а развитие, --- прошептал Гало.
--- И любой мог бы говорить с любым на любом языке.

\section{Подозрение}

--- Скажи, Грейс.
Кто всё-таки был тот стратег, который отдавал приказы Самаолу и Штрой?
У меня создалось впечатление, что его почерк...

--- Ты догадался, да, --- Грейс улыбнулся и опустил голову.
--- Если увидишь те же черты в почерке иерархов Картеля --- дай знать, мы тебя устраним.

--- Но зачем ему?..

--- Он был создан, чтобы править, --- сказал Грейс.
--- И он уже правит и двигает Вселенную к миру и процветанию.
Войну нельзя прекратить в одночасье.
И никакое изменение нельзя совершить, пока общество не будет к нему готово.

--- То есть сейчас, в эту самую минуту он сражается сам против себя?

--- Он --- лучший из стратегов.
И тот новый мир, который мы пытаемся построить, должен иметь защиту от таких, как он.
В этом и смысл сражения с самим собой.
Он следует извращённым философским концепциям, строит жестокие интриги --- одним словом, проверяет зародыш новой системы на прочность.

--- И я буду вынужден...

--- До последнего, как против злейшего из врагов, --- кивнул Грейс.
--- Он поддаваться не будет.
И не терзай душу мыслями о тяжести его доли.
Поверь, \emph{эта} игра доставляет ему истинное наслаждение.
Он на своём месте.

\section{Кандалы}

У дуба примостилась хрупкая фигурка.
Она сидела неподвижно, глядя в небо.
На её руке был массивный браслет, напоминающий кандалы.

--- Штрой так и не хочет с нами сотрудничать? --- спросил Грейс.

--- Неа, --- помотал головой Лу.
--- По-прежнему молчит.
Хотя бы есть начала, и то ладно.

--- Она совсем не ела?

--- Первое её тело так и умерло от голода.
Недавно начала лопать из рук Митхэ по горсточке.
Стигму уже не пытается убить при встрече.
Прогресс налицо.
Хотя я думаю, что она просто замыслила побег.

--- Моё устройство не даст ей сбежать.
Начинай доверять моей технике.
Да и бежать ей некуда --- согласно официальному заявлению, она казнена.

--- Я бы не хотел, чтобы она сотрудничала по принуждению.

\section{Помолодел}

--- Грейсвольд примкнул к диверсионной группе <<Цемент>> не просто так.
Это произошло именно после смерти Айну.
С Айну у него были разногласия, но кому она была верна всю жизнь --- уточнять не нужно.

--- И этому кому-то позарез нужна была функциональная диверсионная группа, --- кивнул Самаолу.
--- Настолько позарез, что вначале он запихнул в нее Анкарьяль, знатно запятнавшая свою репутацию в группе Хэм Золотой Посох, а затем резко получившая ранг по секретному распоряжению отдела 109.

--- Отдел 109 связан с Тси-Ди, --- припомнила Штрой.
--- Что же произошло, что Анкарьяль по их секретному...
А.

--- <<А>>, --- повторил Самаолу.
--- Давно уже известно, что падение Тси-Ди было диверсией, и ковен Лусафейру имеет к нему непосредственное отношение.

--- Самаолу, если я не ошибаюсь, ваша пропаганда утверждает иное, --- хихикнула Штрой.

--- Наша пропаганда утверждает то, что в интересах Ордена, --- спокойно парировал Самаолу.
--- Но очевидное отрицать нельзя.

--- Значит, вначале скомпрометировавшая себя Анкарьяль, --- начала загибать пальцы Штрой, --- потом в группу попал столь же сомнительный Люпино.
И в конце, когда Айну при неясных обстоятельствах погибла, в диверсионный отряд попал близкий друг Лусафейру.

--- И в результате мы получили диверсионный отряд, который без поддержки разбил гарнизон Тра-Ренкхаля, --- заключил голос.

--- Им повезло.

--- Ни разу не везение, --- возразил голос.
--- Вначале <<Цемент>> знатно наломал дров, но затем они исправили все свои оплошности.
В активной стадии операции отряд действовал просто филигранно.
Гораздо эффективнее, чем каждый из низ в отдельности.
Даже Грейсвольд, чего от него никто не ожидал.

--- Грейсвольд как будто помолодел, как оказался в <<Цементе>>, --- сообщил Самаолу.

--- О да, --- согласился голос.
--- Это все отметили.
Похоже, его очень вдохновляют два молодых демона рядом, полных мечтаний, страсти и желания изменить мир.

--- Я думаю, его и засунули к ним, чтобы в узде держать, --- сказала Штрой.

--- Это могла делать и Грихаро, она достаточно стара и выдержана, --- возразил Самаолу.
--- Нет, Грейсвольд нужен, не для сдерживания. 
Он идеолог, подстрекатель.
Поэтому Грихаро и ушла на покой --- она увидела, что Грейсвольд обрабатывает молодежь, и ей это не понравилось.
Но, так как Анкарьяль и Люпино тянулись к Грейсвольду, она последовала старому правилу диверсантов: семья превыше всего.
Если ты выпадаешь из группы --- уходи.

--- Даже если ты стоял у ее истоков, --- добавил голос.

--- Даже если.
Для Грихаро это был, конечно, удар.
Она вообще из полей ушла в административную работу.
Насколько я слышал, в отделе от неё одни проблемы, но старушку держат, как украшение.

--- Или она перешла на другие поля, --- ввернула Штрой.

--- Что вы имеете в виду? --- недовольно осведомился Самаолу.

--- У меня есть данные, что Грихаро Артишок сделали предложение, от которого невозможно отказаться.
Работа.
Опасная.
Цена ошибки --- Чистилище.
И если данные верны --- её неуклюжая административная служба не более чем прикрытие.

--- Ничего себе, --- удивился голос.
--- Штрой, это очень интересно.
Я обязательно проверю эти данные.
Вы не расследовали глубоко?

--- Нет, к сожалению.
Она по всем признакам далеко от нашей сферы интересов.

--- Насколько далеко?

--- Насколько возможно, владыка.
После ухода из <<Цемента>> она оборвала все контакты с группой и дистанцировалась от Лусафейру.
Есть основания полагать, что её новая работа никоим образом с ним не связана.
Она не защищает сферу его интересов и не вмешивается в неё.

--- Скорее всего, поэтому это и прошло мимо меня, --- сказал голос.
--- Но расследовать следует, для общего развития.
Не вам, Штрой, не вам.
Я поручу это другим.
Но спасибо за информацию.

--- Кстати, а кто устранил Айну? --- поинтересовалась Штрой.

--- Интересный вопрос, --- хмыкнул голос.

--- Кто бы это ни был, ему позарез хотелось выбить из-под Лусафейру кресло, --- предположил Самаолу.
--- Лу не стал бы жертвовать лучшим другом, если бы ставки не были высоки.

--- А что Айну связывало с Лусафейру?

--- Старая история.
Лу, Тахиро и Айну был любовниками с незапамятных времён.
И несмотря на то что их пути многократно расходились, они всё равно почему-то сходились обратно.

--- Отголоски первой жизни в демонах очень сильны, --- сказал голос.
--- В первую жизнь формируется ядро сапиентной личности.
Эту жизнь демон потом тащит за собой тысячелетиями --- в том числе её привязанности и травмы.

\chapter{Сага о Тигре}


\section{Первая кровь}

У Тахиро задрожали руки.
Врагов было слишком много, как минимум у половины были винтовки.
Похоже, первая смерть наступит чуть раньше, чем он ожидал.

<<Первая смерть...
Демон!>>

Тахиро попытался сосредоточиться.

\ml{$0$}
{<<Боевые модули, выбрать приоритетную цель, взять на прицел жизненно важные органы>>.}
{``\textit{Combat modules, choose a priority target, aim a vital organ.}''}

Руки вдруг перестали дрожать.
Винтовку повело в сторону и чуть выше.
Демон подсветил четырнадцать фигур, в том числе три, скрытых в темноте.
На одной из них появилась красная точка.

<<То есть если я сейчас скажу их всех перестрелять, демон это сделает?!>>

Над нападающими высветилась надпись:
<<Экспресс-тест боевых модулей завершён.
Все параметры в пределах нормы.
Готовность к атаке --- красный уровень>>.

Вдруг Тахиро бросило в сторону.
Он едва успел осознать, что кто-то из врагов начал нажимать на курок.
Выстрел, выстрел, ещё выстрел.
Над начавшими падать фигурами врагов высвечивалось: <<Обезврежен, обезврежен>>...
Пока затвор ещё продолжал свой путь, рука Тахиро уже нащупала следующий рожок.
Щёлк, щёлк --- перезаряжен...

...Очнулся он чуть позже, обшаривая карманы трупов.
Демон добросовестно сканировал лица нападавших, их экипировку, выдавая автоматический отчёт.
Экипировка была сделана в Такэсако, материалы привезены контрабандой с побережья.
Трое из нападавших значились в розыске, у ещё четверых в розыске были близкие родственники.
Сильно болела надорванная в прыжке мышца --- приложенные демоном усилия явно превышали возможности тела.

<<Сгореть мне дотла, --- ошеломлённо думал Тахиро, глядя на убитых.
--- Тринадцать патронов, двое убиты одной пулей.
Что за чудовище из меня сделали?..>>

\asterism

--- Держи, --- Стигма поставила перед ним чашку с ароматным зелёным чаем и тарелку со сладостями.
--- Я не мастер чайной церемонии, но готовить сам чай и эти вкусные штуки научилась.
Ни в чём себе не отказывай.

--- Благодарю, --- Тахиро церемонно склонился.

Стигма, подумав, вытащила из угла что-то странное, непонятного цвета и сунула Тахиро в руки.

--- Что это? --- удивился он.

--- Обними.
Это мягкая игрушка в виде совы.

--- Совы?

--- Да.
Никогда не видел сову?

Тахиро вежливо пожал плечами в неком гибриде жестов <<да>> и <<нет>> и промолчал.
На сову мягкая игрушка не была похожа совсем, но обижать хозяйку не хотелось.
Впрочем, сова оказалась на ощупь довольно приятной, и Тахиро последовал совету Стигмы.

\textspace

--- Почему ты это мне рассказываешь?
У меня создалось впечатление, что ты не доверяешь никому.
Даже Люциферу.

--- А ему можно доверять?

--- Я бы доверил ему свою жизнь, --- горячо сказал Тахиро.

--- То, что ты с ним спишь и дружишь, не повод доверять ему жизнь.
Доверить можно что-то, что переходит из рук в руки --- деньги, вещи, тайны.
Жизнь --- она только твоя, никто не может забрать её у тебя, не разрушив.
Жизнь не может быть общей, а значит, её никому нельзя доверять.

--- Ты не ответила на мой вопрос.
Почему ты мне это рассказываешь?

--- Я помогаю стать сильнее потенциальному союзнику или потенциальному врагу.
И то и другое в моих интересах.
Сильный союзник помогает, сильный враг держит в тонусе.

<<Часть философии Антрацис, --- понял Тахиро.
--- Эти ублюдки ещё более помешаны на войне, чем мои родичи>>.

--- Я гораздо более миролюбива, чем мой клан, --- сказала Стигма.
Тахиро вздрогнул.

--- Как ты это делаешь?

--- По-твоему, это невозможно?

--- Ты прочитала мои мысли, и...

--- Тахиро, остановись и подумай.
\ml{$0$}
{Твоё мифологическое сознание постоянно говорит тебе, что возможно всё, ничего невозможного нет.}
{Your superstitious thinking always says that all is possible, and nothin' is impossible.}
\ml{$0$}
{Это не так.}
{It's not.}
\ml{$0$}
{Звёзды не загораются на пустом месте.}
{Stars never appear in a vacuum.}
\ml{$0$}
{Дожди из лягушек не идут просто так.}
{Frog rains never fall without a reason.}
\ml{$0$}
{Все события связаны причинно-следственными связями.}
{All events are linked in cause and effect relationships.}
\ml{$0$}
{Никто не умеет вот так сразу влезать в чужую голову, даже демон.}
{And no one can get into your mind like that, not even a daemon.''}

--- Значит, это была иллюзия?

--- Теплее.
Просчитай все возможные варианты мыслей, которые возникли от моей последней фразы.
\ml{$0$}
{Используй демона.}
{Use your daemon.''}

Тахиро открыл и закрыл глаза.

\ml{$0$}
{--- Я бы не подумал ни о чём другом.}
{``I wouldn't think of anything else.''}

\ml{$0$}
{--- Чему ты тогда удивляешься?}
{``Why you're surprised, then?''}

Тахиро откинулся в кресле и крепче обнял сову.

--- Если разговор, который мы ведём, не просто беседа, а имеет какое-то направление, то какова его цель?

\ml{$0$}
{--- Если я скажу, ты мне поверишь?}
{``If I tell you, would you believe?''}

\ml{$0$}
{--- Да...}
{``Yes ...''}

\ml{$0$}
{--- И будешь идиотом.}
{``You'd be an idiot as well.''}

Тахиро отложил сову.
Демон снова вспыхнул, выдав отчёт.
\ml{$0$}
{<<Стигма точно знала, что игрушка не похожа на сову.}
{\textit{Stigma knew for sure that the toy is not looking like an owl.}}
Она изучала мои реакции на внутренний конфликт и на их основе строила дальнейшую беседу.
Интересно, как бы повернулось наше общение, начни я спорить?..>>
Ещё вспышка.
Три возможных варианта развития событий.

Стигма проводила сову взглядом и едва заметно улыбнулась.

--- Я хочу уйти от философии и поговорить о произошедшем, --- твёрдо сказал Тахиро.
--- Мне нужно твоё видение ситуации.
Не стратегические расчёты, а твои эмоциональные реакции как человека, твой совет.
Я напуган и растерян.

--- Вот сейчас это на что-то похоже.
Я к твоим услугам, Тахиро.

--- Я не понимаю, как контролировать демона.

--- Ты --- вернее, то, что ты сейчас считаешь собой --- не можешь контролировать демона.
Демон --- это твоё сердце, глубинные структуры твоего мозга.
У него свои законы, и его интеллектуальный потенциал значительно превышает твой.
\ml{$0$}
{Всё, что ты можешь --- это усилить интеграцию постоянным взаимодействием, пока вы не будете действовать как единое целое.}
{The only thing you can do is to strengthen the integration by permanent interaction, until you both start to perform as one.''}
Демон тебя покалечил?

--- Просто надрыв мышцы, ничего серьёзного.

--- Боевые модули надо постоянно калибровать относительно тела.
Тебе придется вспомнить, где в лагере находится тренировочная площадка и как в скалах проходит полоса препятствий.
Иначе порванной мышцей дело не ограничится --- мне известны случаи вывихов и даже переломов.

--- Я боюсь, что он меня поглотит.

--- Так же, как твои глубинные чувства, мысли, воспоминания --- всё то, что лежит за пределами сознания?
\ml{$0$}
{Их ты тоже боишься?}
{Are you afraid of 'em too?''}

--- Это другое.

--- Определённое сходство есть.
Если ты научился взаимодействовать со своими эмоциями, руками, ногами --- осилишь и демона.

--- Он так много умеет...

--- И эти умения тебя страшат.
Но что, если они твои?

--- Они не могут быть моими!
Он видит в темноте!
Он предсказывает движения на той стадии, когда человек только начинает напрягать мускул!
Он действует быстрее, чем я могу осознать!

--- Тахиро с демоном --- гораздо большее, чем Тахиро без демона.
То же самое с руками и ногами.

--- Эти люди, там, в Такэсако...

--- Которые на тебя напали?

--- У них не было никаких шансов, --- прошептал Тахиро.

--- Не было, --- согласилась Стигма.
--- Это немодифицированные Homo homo sapiens, дикий тип.
Десятикратное превосходство в численности ничего бы не решило.

--- Я думал, что сопротивление...

--- Ты думал, что сопротивление имеет какой-то смысл? --- Стигма хмыкнула.
--- Сразу видно, что ты вырос в обществе дзайку-мару.
В ваших деревнях есть много легенд о героях, которые смогли пересилить, перехитрить, победить сверхъестественное существо.
\ml{$0$}
{Всё это прекраснодушная чушь, часть мифологического сознания.}
{No more than a starry-eyed bullshit, a part of superstitious thinkin'.}
\ml{$0$}
{Ты просто не можешь победить того, чей интеллект превышает твой на два-три порядка.}
{You just can't defeat a person whose intelligence is two or three orders of magnitude higher than yours.''}

\ml{$0$}
{--- Люди решали исход сражений!}
{``Humans decide outcomes of battles!''}

\ml{$0$}
{--- Исход сражений между демонами.}
{``Battles between daemons.}
\ml{$0$}
{С одной стороны демоны и с другой стороны демоны --- тогда люди что-то решают, кидая песчинки своих жизней на гигантские весы.}
{Daemons on one side, daemons on the other---then and only then humans contribute, throwin' grains of their lives on the giant scales.}
\ml{$0$}
{Но организация людей против организации демонов... --- Стигма грустно усмехнулась, --- без шансов.}
{But a human organization against a daemon organization---'' Stigma sadly laughed, ``---stands no chance.''}

Тахиро сел и закрыл лицо ладонями.
Стигма пододвинула ему чашку с чаем.

\ml{$0$}
{--- Тахиро, возможно, я лезу не в своё дело, но то, что ты сейчас испытываешь, полезно для тебя.}
{``Tahiro, maybe that's not my business, but what you're experiencin' is pretty good for you.''}

\ml{$0$}
{--- Бессилие?}
{``Powerlessness?''}

\ml{$0$}
{--- Именно.}
{``Exactly.}
\ml{$0$}
{Бессилие показывает нам границы наших сил.}
{Powerlessness shows us limits of our power.}
Благодаря бессилию ты начинаешь ставить более реальные цели, грамотно распределять свои ресурсы.

--- Тебе это тоже знакомо, --- вспомнил Тахиро.
--- Орден Тысячи Башен...

--- Да.
\ml{$0$}
{И вот я здесь --- гораздо умнее, мудрее и опаснее, чем была раньше.}
{And here I am---more clever, wise, and dangerous than before.}
У меня в распоряжении ресурсы, которых не было в прошлом, и стратеги, в тандеме с которыми я могу достичь того, чего никогда бы не добилась в одиночку.
Стоило ли это нескольких лет безнадёжности и животного ужаса, пока Союз и Преисподняя медленно раздавливали Тысячу Башен в тисках?
Стоило.

Стигма поставила чашку на стол, и Тахиро понял: пора уходить.
Он вежливо доел остатки мидзуёкан, залпом с характерным шумом осушил свою чашку и поднялся на ноги.

--- Я пока не прожил своё бессилие, --- сказал он, --- но я точно знаю, что мне следует делать.

--- Вот и славно, --- улыбнулась Стигма.
--- Хорошего вечера, Тахиро.

<<Она не спросила, --- думал Тахиро по пути к себе.
--- Она плохо поддаётся на манипуляции в разговоре, её почти невозможно заставить свернуть на определённую тему, при этом сама она вертит разговором как хочет.
Истинный военный дипломат>>.

Тахиро усмехнулся.

<<Кажется, я теперь понимаю, что нашёл в ней Люцифер.
Точность, догматичность --- всё, чего не хватает ему самому.
Он строит хаос на порядке, она --- наоборот.
От общения со Стигмой остаётся странное послевкусие, словно лижешь солёный камень --- она проявляет участие, даёт советы, но никогда не раскрывает свои тайны.
И всё же благодаря ей я кое-что понял --- мне пора вылезать из-под крылышка Люцифера.
Я готов бороться за своё место под солнцем>>.

\asterism

--- И с чего ты хочешь начать?

--- Отправь меня туда, где я нужен, --- сказал Тахиро.

Люцифер приложил пальцы к виску.
Тахиро ни разу не замечал этого жеста раньше и сразу понял, что он значит.

--- Ты сейчас ведёшь себя не так, как должен главный стратег по обороне.

Люцифер тут же убрал пальцы с виска.

--- Ты прав, --- кивнул он.
--- В таком случае, легионер, твоя следующая цель --- Нимб-3.
Ты в распоряжении легата секунда Гатса Убийцы Змей из клана Хаджаар.
Инструкции ты получишь по прибытии.

Глаза Тахиро чуть заметно расширились.
Он явно не ожидал, что Лу сразу прикажет ему покинуть тело, отправиться на другую планету --- никого из оцифрованных ещё не отправляли так далеко.
Но длилось это замешательство едва ли секунду.

--- Сайгон, --- Тахиро поклонился и вышел.

<<Что ж, этот день настал, --- с грустью думал Люцифер, наблюдая за уходящим другом из окна.
--- Конец моей беззаботной жизни.
Грисвольда тоже придётся куда-нибудь отправить --- его таланты нужны Ордену.
Стигма втянулась в свою работу, и приходить она будет всё реже --- я уже вижу тенденцию.
Ну и всё --- я снова один, как в давно забытом детстве.
Я ужасно не хотел взрослеть, и все друзья сделали это раньше меня.
Придётся соответствовать>>.

Лу вдруг почувствовал знакомое жжение в глазах.

<<Прости, любовь моя.
Я знаю, что ты больше не хочешь быть первым.
Но из-за меня на тебя всю жизнь сваливалось то, что просто убило бы девятьсот девяносто девять других людей.
Так будет и впредь, если ты по-прежнему будешь считать меня другом --- то, что угрожает мне, не может не угрожать моим близким.
И ты должен быть к этому готов>>.

\asterism

--- Надеюсь, мы ещё встретимся.

--- Надеюсь, что нет.

--- Тахиро, ты мне нужен.

--- Я тебе резко понадобился сейчас?
Или я был нужен тебе ещё до того, как стал демоном?

Айну замерла, опустив гордую голову и согнув прямую, не привыкшую к такому шею.

--- Я никогда тебя не прощу за то, как ты со мной обращалась, --- тихо сказал Тахиро.
--- Ты угрожала мне пытками, ты унижала моё достоинство, ты обращалась со мной как с домашним животным.
Ты убила Мико.
Ты убила человека.

--- Я убивала тысячи людей.

--- Можешь отговариваться чем угодно.
Есть действие, и есть результат.
А ты, насколько я помню, всегда ставила результат превыше мотивов.
Этому у тебя научился и я.

Тахиро повернулся, чтобы уйти.

--- Я не разрешала тебе идти, легионер.

--- Останови меня, легат, --- парировал Тахиро, не оборачиваясь.

\section{Версия ядра}

--- Что тебя связывает с Двойняшками?

--- Да ничего, кроме версии ядра, --- буркнула Стигма.
--- Чёрная Скала ревностно блюдёт родственные узы.
Честь семьи, защита семьи, отсутствие насилия внутри семьи --- этим они забивали головы нам всем.
Но всё это лишь слова.
Меня шпыняли все, кто только мог, а ударить меня до крови было обычной забавой за ужином.

--- Я всегда думал, что минус-демоны получают от этого удовольствие.

--- Дело было не в боли.
Это было неприятие, травля, пренебрежение.
Даже для минус-демона это болезненно.

--- Ты поэтому сбежала?

--- Я сбежала бы раньше, но там был мой младший брат, из более новой серии.
Он единственный относился ко мне хорошо.
Его считали бракованным, потому что он вёл себя как человеческий ребёнок, ничего не принимал всерьёз.
Его тоже шпыняли, но, в отличие от меня, он мог дать сдачи --- его собрали как интерфектора, плюс он постоянно обкалывал себя какой-то гормональной дрянью, из-за чего у него мышцы вырастали до огромных размеров и он был похож на камушек.
И однажды, вступившись за меня, он ударил Фуси --- причём сильно, сломав тому нос, челюсть и выбив с десяток зубов.
В ту же ночь братишка исчез.
Думаю, Двойняшки его устранили.
Издевательств стало намного больше, и я сбежала через год, как только смогла подготовить побег.

Стигма крепче прижалась к Люциферу.

--- Погладь меня по голове, пожалуйста.

Люцифер запустил пальцы в короткие волнистые волосы.

--- Иногда мне кажется, что он где-то рядом.
Иногда я даже слышу его смех в вечерней тишине.

--- Чей?

--- Брата.

--- Ты была привязана к нему?

--- Нет.
Мы даже разговаривали всего раз или два.

--- Кстати, насчёт смеха, --- задумался Лу.
--- Ты же знаешь, после заключения мира к нам присоединилось много пришлых демонов.
Есть тут один легионер, который постоянно смеётся.
Не так, как это делают остальные, согласно обновлениям, а как-то особенно --- гы-гы-гы.

Стигма пристально взглянула на Лу.

--- МПС?

--- Возможно, МПС.
Возможно, ошибка в коде обновления.
Самое интересное, что смеётся он и в карауле.
Сам что-то рассказывает и сам смеётся в одиночку, порой невпопад.
Недавно он пошутил про птицу, и даже я не понял соль юмора...
Вот слушай.
В ясный день по небу летит орёл, навстречу ему --- другой орёл.
Эти орлы столкнулись в воздухе.
Каков был шанс, что это произойдёт?

--- Единица.

--- Почему?

\ml{$0$}
{--- Потому что птицы не пули.}
{``'Cuz birds are not bullets.}
\ml{$0$}
{Они сами решают, куда лететь.}
{They can choose their direction.''}

Люцифер приподнялся на локтях и удивлённо взглянул на Стигму.

--- Ты знаешь эту шутку?

--- Это не шутка, это девиз людей Ниао --- последних, кто попытался дать отпор демонам Чёрной Скалы.
Мой клан вырезал их поголовно.

--- Он пытается найти своих, --- догадался Люцифер.
--- Вечно Гонимых, как и он.

Стигма вдруг улыбнулась и смахнула набежавшие слёзы.

--- Где сейчас этот легионер?

--- Я точно не знаю, но завтра будет его очередь идти в караул.
По его гоготу можно часы настраивать.
Это будет большим отступлением от расписания, если ты придёшь ко мне в гости не через месяц, а завтра?

--- Я не приду к тебе завтра.

--- Не хочешь увидеть брата?

--- На самом деле нет, --- Стигма мягко высвободилась из объятий Люцифера и встала.
--- Просто была рада узнать, что он жив.

--- Ну что ж, тогда до встречи через месяц.
Приходи, погладим друг друга, как обычно.

--- Как обычно.

--- До свидания, Стигма.

--- Прощай, Люцифер.

\section{Охотник}

Ду-Си подошёл почти вплотную и наклонился, рассматривая стратега со всех сторон.
На Лу явственно пахнуло перекисшим потом, порохом, машинным маслом, крепким спиртным, запрет на употребление которого Ду-Си явно пропустил мимо ушей.
У самых глаз Люцифера тускло сверкнула грубо свитая, в два пальца толщиной гривна из <<чёрного>> сурроганиума.

--- Я слушаю, стратег, --- наконец отозвался Ду-Си.
Принятые в Ордене обращения он, разумеется, тоже пропустил.

--- Я хочу произвести тебя в легаты, --- сказал Лу.

--- Неплохое повышение, --- осклабился Ду-Си.
--- Только чем я заслужил звание легата?
Дай угадаю.
Ты в курсе, что я из Чёрной Скалы.
Наверное, от этой предательницы Цзаошан.
Я слышал, что она теперь важная шишка в Ордене.

--- <<Предательницы>>?
Она такая же Вечно Гонимая, как и ты!

--- Извини, привычка.
Так что тебе от меня нужно, я не понял.

--- Мне нужен твой опыт.

--- Я ещё не решил, достоин ли ты моего опыта, --- нагло заявил Ду-Си.

Люцифер заулыбался.
Легионер с необычным чувством юмора всё больше завоёвывал его расположение.

--- Ты знаешь мою репутацию.

--- Я вижу лишь хрупкого демона, заключённого в тщедушное тело.
Ты выглядишь гораздо менее грозно, чем твоя репутация, Люцифер.

--- Про тебя могу сказать обратное, Ду-Си.

Ду-Си загоготал.

--- Да, насчёт репутации я как-то не подумал.
Просто обычно если я за кого-то берусь, не остаётся никого, кто мог бы об этом рассказать.

--- Как я могу доказать, что я достоин?

--- Победи меня в бою.

--- Я не интерфектор.

--- Я не намерен устраивать с тобой дуэль.
Давай просто подерёмся.
Здесь нет никого, кто нам бы помешал.

Люцифер окинул взглядом коридор --- здесь действительно никого не было.
Он был с Ду-Си один на один.
Стратег взглянул в презрительные, немного безумные чёрные глаза легионера, отметил толщину рук и крепость жил, и под ложечкой внезапно явственно уколол страх.
Он оказался один на один с недавно принятым в Орден новобранцем, умелым интерфектором одного из самых жестоких кланов, большая часть которого служила Союзу Воронёной Стали.
Демон Ду-Си <<светился>> тихо и ровно, словно перед атакой.

<<Я пережил войну с Союзом, но меня убьют в коридоре лагеря?
Да какого дьявола?>>

Люцифер со вздохом посмотрел на свои тонкие холёные руки.
Затем со всего размаху отвесил Ду-Си пощёчину.

Ду-Си даже не качнулся.
Он несколько секунд напряжённо смотрел вдаль, словно пытался понять, куда села муха.
Люцифер внутренне сжался, представляя, как огромные руки хватают его, как хрустят кости, как его демон навсегда прекращает существование, сражённый молниеносным омега-ударом...

--- Нуу, отдаю тебе должное --- ты не трус, --- буркнул Ду-Си и с противным скрипом почесал щёку.
--- Большинство на моё предложение подраться отвечают отказом --- если вообще хватает смелости ответить.
Я слышу, как ты дрожишь, я языком чувствую запах страха, который от тебя исходит.
И ты всё равно атаковал, зная, что проиграешь.
Вы с Цзаошан чем-то похожи.
Не зря ты её любишь.

--- Она --- мой соратник, --- возразил Люцифер, чувствуя, как от пережитого напряжения дрожат мышцы пресса.

--- Ага, и в твоём кабинете вы занимаетесь стратегией, наедине.
У сестры странный вкус, надо признать --- не думал, что её вообще заинтересует кто-либо в этой жизни, но, видимо, она просто искала бабу с членом.
Ладно, Люцифер, будь по-твоему.
Хочешь, чтобы я был легатом --- буду легатом.
\ml{$0$}
{Что я должен делать?}
{What I am to do?''}

\ml{$0$}
{--- Что ты хочешь делать?}
{``What do you want to do?''}

\ml{$0$}
{--- Ты спрашиваешь у меня?}
{``You ask me?}
Не надо играть в благородство.
Мы все знаем, как устроен мир.
\ml{$0$}
{Ты приказываешь --- я выполняю.}
{You give the order, and I obey.''}

--- Хороший руководитель перед приказом собирает об исполнителе информацию --- что тот умеет делать, каков его опыт.
Я решил пойти чуть дальше --- я интересуюсь ещё и тем, что исполнитель хочет делать, о чём он мечтает, каковы его жизненные цели.
Согласно Яо, мы находимся на той стадии технологического прогресса, при которой количество возможных социальных ниш значительно превосходит численность населения.
Другими словами, каждый демон может стать не просто винтиком в системе, а узлом, который сложно или даже невозможно полноценно заменить.
Наибольшую эффективность имеет работа не просто опытного, но и заинтересованного демона.
Моя работа --- обеспечить тебя работой, в которой ты заинтересован, помочь тебе найти подходящую социальную нишу, а при необходимости --- и помочь её сменить.

--- Верный служащий, --- хмыкнул Ду-Си.
--- Ещё и Мэг Яо цитируешь.
Хорошая была учёная, жаль только, что как философ дерьмо.
А скажи, Люцифер, хотелось тебе когда-нибудь получить всё?

--- Хотелось, --- честно ответил Лу.
--- Соблазн получить большую власть, чем это положено при твоих способностях, очень велик.
Меня остановили две вещи --- интеллект и любовь к себе.

--- Объясни.

--- Если я получу больше власти, чем следует --- я отберу эту власть у других, нажив себе врагов.
Если моё видение ситуации окажется неверным и действия мои --- неправильными, я испорчу жизнь ещё большему числу демонов и наживу ещё больше врагов.
Когда против меня восстанут --- мне придётся применить силу и получить ещё больше власти.
Каждоё моё действие, вне зависимости от его успешности и правильности, будет рождать смертельных врагов.
Диктатура --- это ловушка, из которой нет выхода.
Интеллект нужен, чтобы её увидеть, любовь к себе --- чтобы добровольно в неё не залезть.

--- Хочешь сказать, что тираны всегда глупы и не любят себя самих?

--- Тираны --- это те, кто живут в сегодняшнем дне, не заботясь о завтрашнем.
Они могут понимать, что рано или поздно против них восстанут, могут не понимать.
Они могут понимать, что рано или поздно опасность будет исходить от любой личности из их окружения, могут не понимать.
Чаще всего это их не волнует.

--- А ты, значит, не живёшь в сегодняшнем дне.

--- Я --- стратег.
Я по определению не могу жить только в сегодняшнем дне.

--- Ты можешь быть умнее других тиранов и не повторять их ошибки.

--- Так думали все тираны перед тем, как навсегда сгинуть в пучине истории.
Так чего ты хочешь, Ду-Си?
Чему ты хочешь посвятить сегодняшний и завтрашний день?

--- Уничтожению, --- ухмыльнулся Ду-Си.
\ml{$0$}
{--- Уничтожать демонов, дзайку-мару, иных тварей.}
{``Destruction of daemons, dzaiku-maru, other beasts.}
\ml{$0$}
{Чем сильнее и многочисленнее твари --- тем лучше.}
{The stronger they are, the bigger their numbers, the better to me.''}

\ml{$0$}
{--- У меня создалось впечатление, что ты не против и защищать.}
{``I got the impression that you're good at protection too.''}

\ml{$0$}
{--- Если защита связана с уничтожением демонов, дзайку-мару и иных тварей --- я не против.}
{``If protection has anything to do with destruction of daemons, dzaiku-maru, or other beasts---I'm in.''}

<<Он мне не доверяет, --- понял Лу.
--- Он не такой, как прочие демоны Чёрной Скалы, не такой, как Двойняшки, он весьма декларативно следует их философии.
Но при старательно пытается произвести именно такое впечатление.
И, как и Стигма, он старательно держит дистанцию с прочими Вечно Гонимыми.
Видимо, у них обоих есть причины, и это точно не боязнь за свою шкуру>>.

--- Разумеется, я тебе не доверяю, --- подтвердил Ду-Си.
Лу вздрогнул.
--- И у меня есть причины.

--- Если ты о моём брате --- демона делает не ядро, а опыт.

--- И ядро тоже.
Даже взять твою любимую Цзаошан.
Она другая, это верно, но копни её поглубже --- и ты увидишь смотрящих на тебя Двойняшек.
Малышка Цзаошан может казаться разнеженной и слабой, но она настолько же целеустремлённая и, в отличие от тебя, не привыкла ограничивать себя в выборе средств.
\ml{$0$}
{Спроси, что положит конец клану Антрацис --- и я назову тебе имя.}
{Ask me what shall be the end of Anthracis, and I tell you the name.''}

\ml{$0$}
{--- Что ж, очень кстати, что у меня есть соратники, спроектированные для уничтожения.}
{``Well, it's very useful to have such destructors by design as companions.}
В пределах досягаемости есть несколько планет, на которых мы не можем жить.
\ml{$0$}
{Причины разные, чаще всего это, как ты выразился, <<твари>>.}
{Reasons differ, and mostly it is, as you said, `beasts'.}
Но есть нюанс.
Эти твари встроены в планетарный баланс, они часть экосистемы.
Экосистема должна быть сохранена, если мы не хотим превратить очередную твердь вот в это.

Люцифер обвёл рукой унылый пейзаж Преисподней.

\ml{$0$}
{--- А мне нравится, --- пожал плечами Ду-Си.}
{``I like it,'' Du-Xi shrugged.}
\ml{$0$}
{--- Немного напоминает дом.}
{``Reminds me a little of my home.''}

--- Чёрная Скала --- отвратительное место для жизни, и мы не будем превращать планеты в её подобие.

\ml{$0$}
{--- Люцифер, просто дай мне десять толковых ребят --- и мы принесём тебе эти планеты на блюдечке.}
{``Lucifer, just give me ten good men, and you'll have those planets on a plate.''}

--- Ребят ты получишь, командование --- нет.
Курировать твою работу будет стратег и небольшая группа планетологов.

--- Стратег? --- Ду-Си задрал голову и загоготал.
--- А я его не сломаю ненароком?

--- Я выберу пожёстче.
Кто-то должен контролировать твою потребность в уничтожении.
По крайней мере пока мы не будем друг другу доверять.
Например, слышал про Мимозе?

\ml{$0$}
{--- Мими Зайденешталь, Восьмой батальон Первого легиона Тысячи Башен.}
{``Mimi Seidenestahl, Eighth battalion of the Thousand Towers First Legio.}
В прошлом --- Богиня Весны, которую использовали как рабыню для мелиорации на Нимб-3.
Сбежала, устроив техногенную катастрофу и попутно насмерть подрезав примарха клана.

--- Всё не так прозаично, как ты описал, но в целом верно.

--- Не знаю, что значит <<прозаично>>.
В деле были замешаны демоны с самой Нимб-3?

--- Именно.
\ml{$0$}
{Любовь-морковь, цветущая капуста.}
{Carrot love, blooming cabbage.''}

\ml{$0$}
{--- В бездну овощи.}
{``Gulf vegetables.}
\ml{$0$}
{Девушка с репутацией, мне этого достаточно.}
{A reputable girl, that's enough to me.''}

\ml{$0$}
{--- Уже не девушка.}
{``Not a girl anymore.}
\ml{$0$}
{Представишься ему сегодня после караула.}
{Report to him after your shift today.}
\ml{$0$}
{И не вздумай называть его в женском роде и использовать диминутивы, понял?}
{And don't you dare to call him in the feminine or use diminutives, right?''}

Ду-Си понимающе ухмыльнулся.

\ml{$0$}
{--- Приятно было познакомиться, Люцифер.}
{``Nice to meet you, Lucifer.''}

Лу кивнул и пошёл обратно в штаб.

<<Нужно поработать над моими адаптационными обновлениями, --- отметил он.
--- Ду-Си неплохо считывает мысли с лица>>.

\section{Падение}

Тахиро много лет снился один и тот же сон.
Дом на скале, дверь над зияющей пропастью, на самом краю обрыва.
Тахиро каждый раз прыгал вниз, не зная, сможет он взлететь или нет.
Если взлетал --- сон продолжался дальше.
Если же тело со страшной скоростью неслось вниз и разбивалось о скалы --- Тахиро просыпался в холодном поту.

В этот раз всё начиналось точно так же.
Тахиро выглянул из двери, бросил взгляд в бесконечно синие небеса --- цвет, которого Преисподняя не знала никогда --- и прыгнул вниз.
Крылья не раскрылись.
Но Тахиро, сжав зубы, обратил взор на летящую к нему твердь --- и крепко приземлился на ноги.

<<Что произошло?>>

Тахиро открыл глаза, обдумывая непривычный сон.
Рядом похрапывал Грисвольд, рядом читал какую-то книгу Лу, проговаривая слова губами и улыбаясь.
Всё казалось обычным и в то же время не совсем...

<<Я сделал то, к чему шёл всю мою жизнь, --- понял Тахиро.
--- Моя жизнь зависела от того, чем я не управлял --- от возможности взлететь.
Но на этот раз я выжил и при падении.
Это мой сон, это моя жизнь.
И правила тоже мои>>.

Тахиро улыбнулся, перевернулся на другой бок и снова заснул --- на этот раз без сновидений.

\section{Спорщик}

--- Знаешь, ты очень умный для человека, --- сказал Грис.
--- Но иногда сказанное тобой похоже на редкостную чушь.

--- Время покажет, --- лаконично ответил Тахиро.

Грис одобрительно посмотрел на друга.

--- Именно поэтому я и говорю, что ты очень умный.

--- Потому что я не стал спорить?

--- Потому что не стал спорить, когда спорить не нужно.

\section{Предназначение}

Люцифер обладал способностью, за которую часто себя ругал.
Он мог представить дичайшую ситуацию, не укладывающуюся вообще ни в какие рамки --- и тем не менее ситуация, в силу её логичности, имела право на существование.
Разумеется, эта способность была закономерным следствием высокого интеллекта, но порой доставляла массу неудобств.

Ситуация, которую Люцифер вообразил сегодня, одновременно рассмешила и до крайности встревожила его.
Во время прогулки с Тахиро к ним привязался низкорослый бродячий предсказатель в высоких гэта, обклеенных какими-то карточками и расписными <<магическими>> камнями.
Друзья еле от него отвязались.

--- Ты веришь в предназначение? --- спросил Тахиро друга.

--- Не вижу особого смысла, --- ответил Лу.

Но потом, сидя в кабинете, Лу вдруг представил, что предназначение действительно есть.
Управляет ли им вселенский разум или что-то неразумное и механическое --- неважно.
Но тогда, может статься, весь многолетний опыт существа, все его размышления, страдания, сомнения и мечты могут иметь целью лишь одно --- чтобы в нужное время, в нужном месте это существо сказало нужные слова тому, кому реально суждено изменить мир.
А затем это существо становится для предназначения ненужной деталью.
Самое ужасное, что от этой участи не спасёт ничего --- ведь любые блага и несчастья могут служить точильным камнем для одноразового инструмента.

<<Что, если это я?
И если бы я узнал, что я лишь одноразовый и, вероятно, уже использованный инструмент --- захотел бы я жить так, как живу?>>

\section{Спираль}

--- Людишки снова пометили дома агентов, --- буркнул Гало.
--- Снова спираль.
Вот фото --- как видите, они нарисовали её краской на стене дома.
Вот ещё такой же символ, нацарапанный на гостевом кресле.
А это моё любимое фото --- спираль очень красиво выложена камнями на дороге.

Люцифер рассеянно перебирал фото.
Спираль.
Огненный смерч, самое зрелищное и смертоносное природное явление на Преисподней, разрушительная, неудержимая природная стихия.
И знак <<хорохито>> --- невидимого, неосязаемого, но ужасного врага, с которым полудикие люди вступили в неравную схватку.
Символ говорил сам за себя --- они знали, на что идут.
Для дикаря это было равноценно вызову, брошенному богам.
Люцифер не мог сдержать растущее внутри уважение.

<<Я был неправ, назвав их крысами.
Это люди --- во всех смыслах этого слова>>.

\section{Кицунэ}

--- Гало сегодня не в духе.

--- Он всегда не в духе в полнолуние Четвёртой Луны.
Может, он оборотень?

--- Кто?

--- Оборотень.
Мужчина, который принимает облик демонической лисицы в полнолуние Четвёртой Луны.

--- Я слышал, что Четвёртая Луна в точности повторяет ту единственную, которая была на Древней Земле.
Период обращения, фазы, даже рисунок.

--- Серьёзно?
У Древней Земли тоже была луна с рисунком в виде среза рыбы?

--- Я не знаю точно.
Многое могло поменяться за прошедшие тысячелетия.
Древняя Земля погибла из-за потока астероидов.
Наверное, Древнюю Луну постигла та же участь.
Но одно можно утверждать точно --- цикл Четвёртой Луны в точности повторяет менструальный цикл человеческих женщин, как и цикл Древней Луны.

--- С чем это связано?

--- Вряд ли что-то можно утверждать наверняка.
Например, древние люди трудились при свете Луны, и в периоды новолуния, когда мужчины оставались на ночь дома, у женщин происходила овуляция.
Хорошее объяснение?

--- Хорошее, но верное ли?

--- Вот именно.
В точку.
А почему ты решил, что Гало --- оборотень?

--- Ну... --- Тахиро смутился.
--- Знаешь ли ты, почему все оборотни --- мужчины?

--- Почему?

--- Легенда гласит, что есть те, кто должен был родиться женщиной, но по ошибке родился мужчиной.
Женщина внутри такого человека заперта навсегда.
Ей недоступны женские радости, она не может позволить себе женскую слабость, и даже такие обычные вещи, как менструации и беременность, ей недоступны.
Именно поэтому, когда Четвёртая Луна имеет наибольшую власть, сошедшая с ума женщина превращает мужское тело в демоническую лисицу и мстит миру.

--- Ты думаешь, что Гало --- женщина?

--- Вы --- близнецы, --- пожал плечами Тахиро.
--- Но в тебе очень много женских черт.
Гораздо больше, чем в Гало.

--- Гало только эту легенду не рассказывай, --- хихикнул Лу.
--- Он тебя убьёт.
Это точно.
А скажи, были ли оборотни в твоём поселении?

--- Были.
Каждое полнолуние их отводили в дом встреч, мыли их, умащали их тела благовониями и одевали в женскую одежду.
Тогда духи женщин успокаивались и эти люди спали всю ночь спокойно.

\section{Айну (в мусорку?)}

Айну Крыло Удачи была странным существом.
Долг и гуманность словно не имели в её разуме точек пересечения.
Она могла очертя голову броситься в горящий дом, чтобы достать плачущего ребёнка, и с материнским трепетом ухаживать за ним.
Если же для достижения цели, поставленной командованием, необходимо было вырезать целый город, Айну не церемонилась ни с детьми, ни со стариками --- убивала одного за другим, без жалости и раздумий.

Но так было не всегда.
В первый раз, когда гуманность в её разуме возобладала над долгом, она спасла будущего мужа.
Второй раз стоил ей жизни.

\section{Суп мертвеца}

--- Что может быть увлекательнее путешествий, --- сказал Тахиро.

--- Ничего, --- согласился Грисвольд.
--- Но скучный суп, скучное одеяло и скучный вечер для меня значат куда больше самого увлекательного занятия.

--- То есть ты борешься за суп? --- лукаво спросил Лу.

--- А сможешь за суп умереть? --- усмехнулся Тахиро.

--- Вы оба --- молодые придурки и не понимаете ни сущность, ни назначение супа, --- наставительно сказал Грисвольд.
--- Но однажды вам попадётся кулинар...

--- ... за которого можно отдать жизнь, --- ввернул Тахиро.

--- Ты идиот, --- резюмировал Грисвольд и занялся монтажом.
--- Мертвецу суп не нужен.

\section{Экскурсия по Лотосу}

--- Ты как? --- склонился над молодым парнем Грейсвольд.

Лусафейру тяжело дышал.
Его глаза метались по обстановке, словно у умалишённого. Но спустя минуту он успокоился и довольно улыбнулся.

--- Аммм... интеграция личностей прошла успешно.
Мои благодарности.

Толстяк протянул парню изящный глиняный сосуд, и тот жадно впился в горлышко.
Закашлялся.

--- Что это за гадость, Грейс?

--- Просто солевой раствор с нейропротекторами.
Тебе нужно восстановить мозг после форсированного пробуждения.

Лусафейру поморщился и снова начал пить.

Грейсвольд окинул взглядом комнату.
Дворец был довольно неплохо стилизован под период Анри, но опытный взгляд демиурга распознал и что-то модерновое, и откровенно архаичные нотки в окружающем великолепии.
По крайней мере, занавеси должны быть сделаны из шёлка, а не из этого низкопробного растительного волокна.

Вдруг скрывавшая проход занавесь отпрыгнула в сторону, и в комнату забежала женщина.
Её до невозможности длинные волосы, согласно местной моде, были собраны в затейливые хвостики и косы.
Лоб перехватывал кожаный обруч со звенящими серебряными цепочками.

--- Я готова, --- радостно возвестила женщина.

--- Ты с ума сошла, Айну, --- буркнул Лусафейру.
--- У женщин этой планеты волосы не могут вырасти до такой длины.
Плюс мы будем идти по лесам.
Ты об этом подумала?

--- Лу, если мы пойдём по лесам, то я их соберу.
Ты пей воду, пей.

Лусафейру пожал плечами.

--- А где Тахиро?

--- Тахиро просил передать, что он случайно умер и не сможет с нами погулять, --- сообщила Айну.

--- Жалко, конечно, --- признал Грейс.
--- Но всякое бывает.
Лу, приходи в себя и собирайся, мы уже готовы.

\asterism

Вскоре друзья уже вышли в жар полуденного Шершерота.
Сейчас у местных жителей был час сна, и по узким улочкам гуляла только крупная кремнистая пыль.

--- Вы позволите? --- Грейс галантно подал руку Айну.
Женщина тихо усмехнулась, подхватила Грейсвольда под локоть и неуклюже чмокнула его сквозь паранджу.
Грейсвольд поморщился.

--- Дурацкий мешок...
Ох уж эти изолированные города-государства, их олигархи вечно придумывают всякие идиотские правила...

--- Я его сниму за городом, --- пообещала Айну.
--- Здесь нравы не ахти.

--- Кстати, Грейс, я бы на твоём месте был менее галантен, --- добавил Лусафейру.

--- Ты ревнуешь, что ли?

--- Нет, балбес.
Просто в этих краях не принято ходить за руку, тем более с женщинами.
Если ты не заметил, на парандже рукавов нет.
И да, Айну, твои щиколотки светятся на весь Шершерот.

--- Сейчас сиеста.
Авось пронесёт, --- сказала Айну.
--- Я хочу прогуляться с максимальным комфортом.
К тому же с нами пока ещё наследник престола, да, Лу?

--- Ага.
Шестой сын второй жены.
Наследник наследников.

--- Зато ты красивый, --- утешила его Айну.
--- У первой жены родились какие-то пирожки с глазами.
Я бы таким даже руки не подала.

--- Да кто бы тебя здесь спросил, --- нехорошо усмехнулся Лусафейру.

--- Если меня не спросят, то потеряют способность спрашивать, --- в тон ему ответила интерфектор.
--- Кстати, Грейс, ты обещал экскурсию, а не прогулку.
Ну-ка расскажи историю этого города.
Только не ту ахинею, которую нам втирал придворный историк шасера\FM...
\FA{
Шасер --- правитель-деспот Шершерота (вероятно, видоизменённое s-l: cesar --- <<монарх>>).
}

\textspace

--- Надо же, --- удивилась Айну.
--- Так это твоих рук дело?

--- Да, --- сказал Грейсвольд.
--- Часовая Луна представляет собой поляризующий кристалл.
Каждую одну двадцатую суток она бросает лучи в сторону планеты, и её видно.
Моя идея.

--- Если подумать, у нас во дворце никогда не было устройств для измерения времени, --- заметил Лу.
--- Айну, да сними ты уже этот мешок.

Женщина аккуратно сняла паранджу, сложила её и спрятала на обочине, под сухим кустом.
Её длинные волосы немедленно подхватил горячий ветер, и Лусафейру на секунду залюбовался этим великолепным трепещущим знаменем женственности.

--- Всё-таки задатки технолога у тебя были ещё до Ордена, --- заметила Айну.
--- Хоть и странно говорить о каких-то задатках применительно к хоргетам.

Грейсвольд пожал плечами.

--- Мне это доставляло удовольствие.

\chapter{Интерлюдии}


\section{Лунное молоко}

<<Но как же убить Маликха? --- говорили они.
--- Нет среди нас тех, кто мог бы победить его в одиночку или отрядом.
Он знает, как звякает спрятанный нож, знает, как шуршит удавка, он знает запахи и вкусы всех ядов>>.

<<Тогда мы дадим ему тот яд, который не имеет вкуса и запаха>>, --- сказали каменщики, и собрали в пещерах лунное молоко, и приготовили пищу, подмешав в неё лунное молоко.

Маликх не узнал запаха лунного молока, потому что оно не имеет запаха.
Он не почуял вкуса, потому что и вкуса у лунного молока нет.
Но он съел много яств, и вместе с ними много лунного молока, и лунное молоко притупило его острый ум, замедлило его глаза и отняло силу у его рук и ног.

\section{Прозрачная Борода (ЛоО)}

Были идолы Микхана умны, но недоставало им мудрости, и боялись они, что тайны их узнают и богатства их иссякнут.
Узнав, что Ликхмас, Маликх и Чхалас вошли в их город, они попрятались по домам, и каждый микханский житель выставил за дверь крохотную куклу отвратительного вида --- с огромными зубами и глазами, горящими Каменным Жаром.

<<Не дайте им прикоснуться к себе, --- сказал Маликх.
--- Микханская кукла вгрызётся в вашу плоть и проест её до самой головы\FM>>.
\FA{Предположительно, прообразом микханских кукол стали Белые Гусеницы --- оружие времён Войны Тараканов. }

\textspace

Шли друзья по пустому городу, и ни одна дверь не открылась на стук, и ни одно окно не удавалось сломать --- искусны были зодчие Микхана.

<<Мы не можем уйти просто так, --- сказал Маликх.
--- Едва мы выйдем из города, как жители Микхана покинут свои дома и натравят на нас великана Прозрачная Борода, и все духи лесные не помогут нам, если он нападёт на нас дорогой.
Он втыкает в плоть своих врагов волосы и высасывает через них всю кровь.
Мы должны убить его или остаться здесь пленниками навечно>>.

<<Одно оружие Микхана поможет нам найти другое, --- сказал Ликхмас.
--- Кукла видит то, чего не видят человек и идол.
Кукла открывает один глаз, если ждёт приказаний, но при виде врага своего хозяина она открывает оба глаза>>.

Он отрезал голову сломанной микханской кукле, взял её за волосы, нажал на левое ухо, погладил правое --- и приоткрыла кукла левый глаз.
Тем временем наступила холодная ночь, и город погрузился во тьму;
глаз куклы засветился ярким Каменным Жаром, освещая дорогу, деревья и жилища.

\textspace

<<Прозрачная Борода невидим, --- сказал Маликх.
--- Он может быть где угодно>>.

<<Тот, кто невидим, должен быть наг и держать своё тело в чистоте, --- сказал Ликхмас.
--- Его следует искать у воды>>.

<<Тот, кто невидим и наг, чувствует себя замёрзшим и ничтожным, --- сказал Чхалас.
--- Его следует искать в самом тёплом, маленьком и далёком от жилья озерце, которое только есть в Микхане>>.

Долго искали друзья озерцо Прозрачной Бороды, ведь Микхан ныне называют Лабиринтом Озёр.
И вдруг, у самого маленького и незаметного, скрытого в буйнотравье озера, кукольная голова приоткрыла второй глаз.

<<Мы не можем победить Прозрачную Бороду в его озере, --- сказал Маликх.
--- Нужно выманить его наружу>>.

<<Мы не сможем победить того, кто невидим>>, --- сказал Ликхмас.
Он набрал сухой придорожной пыли, разложил её в три мешочка и раздал мешочки друзьям.

<<Тот, кто силён, но чувствует себя ничтожным, обидчив и драчлив>>, --- сказал Чхалас.

Чхалас выкрикнул бранное слово, которое не знали Маликх и Ликхмас, и засмеялся смехом, который может пробуждать лишь обиду и гнев.
Взвыл от обиды и ярости Прозрачная Борода и бросился к ним из своего озера.

\textspace

Туго пришлось Маликху.
Бился он, подобно сотне умелых воинов, и всё же осколки клинка брали своё, резали его кожу и жилы.
Ослабел Маликх.

<<Не отражай удары! --- кричал ему Ликхмас.
--- Уворачивайся от его ударов!>>

И начал Маликх уворачиваться.
Делал он шаг в сторону, шаг в другую, кланялся и падал ничком.
И хоть теснил его Прозрачная Борода, больше не летело осколков с хрустального клинка.

\textspace

И когда Маликх отрубил голову Прозрачной Бороде, Ликхмас крикнул ему:
<<Разведите костёр!
Его раны затягиваются, и голова скоро прирастёт к шее!>>

Ликхмас, Маликх и Чхалас развели костёр и бросили голову Прозрачной Бороды в огонь.
Завыла великанья голова:
<<Проклинаю тебя, Маликх, убийца мой, проклинаю тебя, Чхалас, обидчик мой, проклинаю тебя, Ликхмас, отнявший моё бессмертие!
Не смог я выпить вашу кровь волосами, но мой пепел сможет, отныне и навсегда!>>

И обратился пепел Прозрачной Бороды в полчища мошек и комаров, и разлетелись они по микханским болотам, жаждая мести.
А Чхалас тем временем поднял с дороги Клинок Осколков, который умелый Маликх выбил из рук Прозрачной Бороды.

<<Подлое оружие, --- сказал Маликх.
\ml{$0$}
{--- Нельзя оставлять на дороге такое.}
{``It mustn't be left on the road.}
Я возьму его и повешу на пояс.
\ml{$0$}
{Но будь я проклят четырежды четыре раза, если достану его из ножен!>>}
{But let me be cursed twice and twice again if I unsheathe it!''}

И, не медля, покинули Ликхмас, Маликх и Чхалас надменный город Микхан.
И оплакивали микханские идолы потерю своих кукол и великана Прозрачная Борода, и долго слышался их плач над джунглями.
Вскоре тайны Микхана были расхищены и стали достоянием мира, и богатства его иссякли, и слава его потускнела, и дома обветшали, и жители стали подобны своим диким собратьям из лесов.
Так закончил свои дни город Микхан.

\section{Странствующий Король}

\textbf{Талианская сказка}

Жил на свете Странствующий Король.
Он был настоящим Королём --- и по праву рождения, и по умениям, и по королевскому характеру.
Его посох был недорогим, но красивым;
такой же была его одежда.

Странствующий Король искал место, которое мог бы назвать домом.
Он останавливался в самых дешёвых гостиницах и расспрашивал местных крестьян, где он мог бы пожить.
Крестьяне с радостью давали ему место --- ведь характер у Короля был поистине королевский.

Но всё менялось, когда жители узнавали в нём Короля.
Кто-то начинал восторженно кричать его имя под окнами.
Кто-то начинал спрашивать его мнение, как лучше поступить в том или ином случае.
Прочие же в лицо называли его самозванцем, хотя Король не претендовал ни на одну корону.

Каждый раз к новому дому Короля приходила толпа, возглавляемая деревенским старостой или бургомистром.
Они требовали, чтобы Король ушёл.
<<Нам здесь проблемы не нужны>>, --- говорили они.

И Король, вздохнув, снова надевал дорожную одежду, брал свой недорогой, но красивый посох --- и уходил навсегда.

Однажды пришёл он в город, имя которого старики забыли, а бумага не сохранила.
Приютил его молодой ткач, живший на самой окраине да ещё пару лиг в поля.
Он накормил Короля, растопил ему хамма, развеял его грусть беседой.
Перед сном Король снял свою дорожную косынку и начал расчёсывать свои длинные волосы цвета подорожника.

<<Могу ли я тебе помочь?>> --- спросил мужчина.

Странствующий Король согласился, и мужчина стал причёсывать его волосы.

<<Продай мне волосы, --- сказал он.
--- Я вплету их в мой молитвенный ковёр>>.

<<Я не продам тебе волосы, --- ответил Король.
--- Но я могу подарить тебе время, которое они растут>>.

Мужчина согласился.
На рассвете они обвенчались и стали жить вместе.

Шли года, росли горы, таяли под солнцем снега.
Король продавал ткани и был счастлив со своим мужем, но нет-нет да и задумывался: нет ли где вдали его собственной страны, народа, которым ему суждено править?

<<О чём ты печалишься, любовь моя?>> --- спрашивал его муж.
Король долго отнекивался, но наконец сказал мужу, в чём его кручина.

<<Не горюй, --- сказал муж, --- выпей крепкого вина и ложись спать>>.

Король выпил крепкого силинского вина и проспал целую ночь без снов.
А когда проснулся, вокруг него был дворец, и люди оказывали ему почести, какие подобают монарху.

\section{Страна каменных духов (ЛоО)}

В стране каменных духов была единственная ценность --- живой камень, из которого и были сделаны духи.
Живым камнем платили за работу, живой камень хранили в кошельках и носили на шеях, как ожерелья, живым камнем платили правителям.
Если же дух был беден, то расплачиваться ему приходилось частями собственного тела.
Многие бедные духи, собираясь у подземных костров, проводили долгие часы, обсуждая, что отдать сегодня.
Часть руки, жертвуя силой?
Часть ноги, жертвуя крепостью стойки?
А может быть, отдать правителю уши или кишки, ведь на них не так трудно заработать?

Если же у духа не оставалось живого камня, он пропадал и больше не существовал никогда.

Многие бедные духи, отчаявшись, отваживались на Насилие.
Они воровали чужой живой камень.
Но наказание за это было суровым;
однажды Ликхмас, Маликх и Чхалас увидели, как в суд привели маленького плачущего духа с большим-пребольшим носом.
Суд постановил отрезать нос и отдать его правителю.

<<Но ведь это вина правителя, что духу пришлось пойти на воровство! --- нахмурился Чхалас.
--- И они намеренно выбрали самую большую часть тела, его гордость!>>

<<Я не могу больше на это смотреть>>, --- признался Маликх.
Сверкнула сабля --- и судья развалился на две ровные половинки, которые тут же начали драку за кусочки тела.
Освобождённый дух бросился прочь с криками <<Чужаки разделили судью! Чужаки разделили судью!>>

<<А вот и благодарность, --- сказал Ликхмас.
--- Бежим отсюда>>.

\section{Ключ}

Увидели друзья, что идут разбойники, а в повозке у них чан большой, в рост человеческий.

<<Змея в чане, --- сказал Ликхмас.
--- Чую по тому, как трясётся чан, будто аспид в нём огромный кругами ползает>>.

Оглядел Маликх разбойников.

<<Не одолеть нам их, --- сказал он.
--- В бою любой из них меня превосходит.
Чую по походке да по поворотам глаз --- гиблое дело>>.

<<Нашей будет змея, --- сказал Чхалас.
--- Сними-ка замок золотой с сундука да ключ мне дай>>.

Ткнул Ликхмас пальцем в одну заклёпку, в другую --- и упал с креплений замок.

Нашёл Чхалас скалу возле дороги.

<<Вбей замок в скалу>>, --- сказал Чхалас.

Ударил Маликх раз по скале, второй --- и вогнал золотой замок в скалу, словно дверь в ней закрытая.

Едут разбойники мимо.
Видят --- замок золотой в скале.

<<Подвал тайный! --- обрадовались разбойники.
--- На всю жизнь золота достанем!>>

Стали счастья пытать, отмычками замок ковырять, да не поддаётся.
И кирками скалу били, и кусачей бумагой жгли --- не поддаётся скала.
Утомились разбойники.

<<Гиблое дело, --- говорят, --- мастер делал.
Ключ нужен, без него сокровища не достать>>.
И поехали дальше.

А Чхалас тем временем отбежал на десять кхене и бросил на дорогу золотой ключ.
Едут разбойники мимо, видят --- ключ золотой лежит.
Схватили разбойники ключ, переглянулись да побежали назад что есть мочи, бросив чан.

Выпустили друзья змею из чана.

\section{Город потерянных детей (ЛоО)}

Дети проводили время в играх.
Если же игры заканчивались дракой, то дети фантазировали;
они придумывали себе силу, как у взрослых, и любящих их взрослых, имеющих силу.
Но взрослых не было, и силы было взять неоткуда.
Тогда дети играли в любовь и приписывали силу тем, кого любили;
когда открывалась правда, игры в любовь тоже заканчивались дракой.
Дрались дети постоянно, и фантазии захватывали их, словно волосяные силки --- мелкую пичугу\FM.
\FA{
Использовать волосяные силки для птиц очень опасно.
Если птицу вовремя не подхватить, то она запутается в волосе и умрёт от удушения.
У сели было поверье: погибшая в волосяных силках птица --- двадцать дождей несчастий.
}

У детей был вождь.
Он говорил детям, как жить, и они беспрекословно слушались его, как взрослого.
Вождь боялся детей, потому что они слушались его;
он знал --- если появится другой вождь, то все будут слушаться другого вождя.
Поэтому он говорил детям чаще играть и не препятствовал дракам.
Главными играми он велел считать соревнования и судил на них, ведь судье не бросают вызов.
В свободное время он велел играть в любовь, чтобы детей было больше.

Когда вождь оставался один, он мечтал о любящих взрослых, как и все.
Но его фантазии были похожи на дым, он им не верил.
Потому он и становился вождём.

\textspace

Ликхмас вышел на главную площадь и закричал:

<<Дети!
Мы можем заботиться о вас!
Я --- жрец, я могу учить и лечить вас.
Мой друг Маликх --- воин, он может дать вам твёрдую опору и научить владеть своим телом.
Мой друг Чхалас --- купец, он сделает ваших врагов друзьями и разделит блага по справедливости!>>

Дети молчали.
На площадь начал надвигаться чёрный туман, жгучий и отдающий гнилью.

<<Мы --- взрослые! --- закричал Ликхмас.
--- Мы можем стать вам кормильцами, пока вы не будете готовы вступить на собственный путь!>>

<<Вы не взрослые, --- хором ответили дети.
--- Это мы --- взрослые.
А вы --- калеки.
Вы --- уродливые великаны.
Вы опасны.
Вас надо уничтожить>>.

<<Они нас убьют, Ликхмас, --- сказал Чхалас, и друзья поняли, что это правда.
--- Нам следует бежать>>.

После этих слов Ликхмас, Маликх и Чхалас побежали, преследуемые огромным множеством детей.

Но вспомнил Ликхмас про дар, преподнесённый ему в <<Бамбуковой клетке>>;
бросил он на землю ветвь из головы идола, и выросла она в непроходимые джунгли.
Остановились Ликхмас, Маликх и Чхалас, и слушали они, как отчаянно зовут друг друга заблудившиеся в лесу дети.

<<Можно ли помочь?>> --- спросил Ликхмас у друзей.

<<Нет, --- ответил Маликх.
--- Многие из них погибнут.
Но те, кто выживет в одиноком скитании, станут взрослее.
Если их фантазии помогут им совладать с сельвой, если найдутся те, кто оставит фантазии ради жизни, у города появится шанс>>.

\section{Цветы и семена}

Жили-были два соседа-цветочника --- Марин и Марса.
Оба они были искусны в своём деле --- они искали самые лучшие семена на заливных лугах, выращивали в оранжерее цветы и продавали их на рынке.

Но однажды Марсе пришла мысль.
<<Зачем я буду проводить драгоценные кхамит, выискивая семена?
Ведь семена я всегда могу купить у соседа.
Не лучше ли будет мне сосредоточить все силы и умения на выращивании цветов?>>

Сказано --- сделано.
Отныне Марса стала совершенствовать своё искусство цветоводства, а семена ей стал приносить сосед.
Доходы цветочницы увеличились --- ведь её цветы были гораздо лучше соседских.

Посмотрел Марин и подумал: <<Цветы Марсы действительно лучше моих.
Но она нуждается в моих семенах и всегда с удовольствием покупает их.
Зачем я буду тратить часы в оранжерее?
Не лучше ли сосредоточиться на поиске лучших семян и продавать их Марсе?>>

Сказано --- сделано.
Отныне Марин стал совершенствоваться в поиске семян.
Доходы его увеличились --- ведь чем лучше он находил семена, тем красивее были цветы Марсы.

Однажды пришёл Марин к цветочнице и, как обычно, предложил ей свои семена.
Но Марса неожиданно отказалась их покупать.

<<Что случилось? --- удивился сосед.
--- Раньше ты с удовольствием брала мои семена>>.

<<Твои семена по качеству хуже тех, что приносит мне другой торговец>>, --- ответила Марса.

<<Мы же соседи и давно друг друга знаем!>>

<<Да, но если я буду брать твои семена лишь по старой памяти, люди перестанут покупать мои цветы и я разорюсь>>, --- объяснила цветочница.

Подумал Марин и понёс семена к другой цветочнице.

<<Какие хорошие семена! --- восхитилась она.
--- Сколько просишь, Марин?>>

<<Десять гран>>, --- ответил торговец.

<<Я не смогу их купить по такой цене, --- грустно ответила цветочница.
--- Может быть, ты продашь их мне за пять?>>

<<Я знаю цену своим семенам>>, --- отрезал Марин и отправился к третьей цветочнице.
Та встретила его холодно, взяла семена и начала ломать их.

<<Что ты делаешь?!>> --- возмутился Марин.

<<Хочу убедиться, что они настоящие, --- ответила цветочница.
--- Дождь назад один из торговцев продал мне раскрашенные камни вместо семян>>.

<<Ты убедишься, что все они настоящие.
Но какой толк будет от сломанных семян?>> --- закричал Марин, забрал товар и вышел, не попрощавшись.

Четвёртая цветочница тоже встретила его холодно.

<<Покажи товар>>, --- сказала она.

Марин, удивляясь, выложил своё богатство на прилавок.
Цветочница вздохнула.

<<Извини.
Недавно в город приходил плут.
Он уверил всех цветочниц, что привезёт самые лучшие семена, собрал с них золото и исчез.
Я должна была убедиться>>.

Осмотрев семена, цветочница с радостью приняла их, хоть и заплатила на гран меньше, чем Марса.

\section{Кораллица, жаба и красный шар (сказка сели)}

Однажды один мальчик поймал птицу-красный шар.
Птица забавно съёжилась, как мяч, а мальчик долго перекидывал мягкий красный мячик в руках.

Наконец ему стало жалко птицу, и он отпустил её у реки.
Красному шару от страха захотелось пить;
он принялся кружить над водой, но нигде не видел ни единого места для водопоя.

Тогда красный шар подлетел к мальчику:

<<Мальчик, я очень хочу пить.
Ты отпустил меня, и я тебе благодарен;
будь же для меня другом --- сделай для меня водопой!>>

Мальчик подумал, отломил ветку у сухого кофейного дерева и вкопал её в дно реки.
Красный шар обрадовался: с ветки он мог и пить, и взлетать со своих коротеньких лапок.

Увидела это каменная жаба и тоже обратилась к мальчику:

<<Мальчик, я тоже очень хочу пить, но мне трудно выползать из воды по такому крутому берегу.
Будь же и мне другом --- сделай водопой!>>

Мальчик подумал, отыскал на берегу обломки дерева и сделал для жабы хорошую лестницу.

Обидно стало красному шару;
дождавшись, пока мальчик уйдёт, он обратился к жабе:

<<Не стыдно ли тебе?
Я был его игрушкой и потому получил водопой, а тебе всё досталось просто так!>>

<<Чего мне стыдиться? --- удивилась жаба.
--- Не я заставила мальчика играть тобой, как мячом.
Я лишь попросила лестницу>>.

Прошла декада --- и кофейную ветку унесло течением.
Лестница же осталась на берегу целой.
Ещё больше обиделся красный шар:

<<Не стыдно ли тебе?
Я был его игрушкой и потому получил водопой, а тебе всё досталось просто так, да ещё и лучше!>>

<<Чего мне стыдиться? --- удивилась жаба.
--- Немудрено, что ветку унесло --- она была в потоке.
А моя лестница на глинистом берегу>>.

Но красный шар был недоволен.
Предложил он позвать мальчика и задать ему вопросы, а судьёй между собой и жабой выбрал змею-кораллицу, что проползала мимо.

Загулила жаба, и вскоре пришёл к реке мальчик.
Заметив кораллицу, он плавно поднял копьё, чтобы змея его не увидела, и пронзил ей голову.

<<Зачем же ты убил нашего судью?!>> --- воскликнул красный шар.

<<Я не знал, что ядовитая кораллица --- ваша судья, --- ответил мальчик.
--- Я поступаю так со многими ядовитыми змеями, чтобы они не укусили меня>>.

<<Тогда ответь: почему ты играл со мной, как с мячом?>>

<<Потому что ты мягкий и похож на мяч>>.

<<Почему ты не поиграл с жабой?>>

<<Потому что жаба тяжёлая и колючая, с ней играть неинтересно>>.

<<Ответь же тогда напоследок: почему ты сделал водопой и мне, и жабе?>>

<<Потому что вы оба об этом попросили>>.

Красный шар всё понял и попросил мальчика ещё раз сделать ему водопой.
Больше он с жабой не ссорился.

\textspace

<<Безымянный слепил меня из пламени, как были слеплены звёзды.
Обижались ли звёзды на то, что их слепили?>>

<<Безымянный слепил меня из пыли и воды, как была слеплена твердь.
Обижалась ли твердь на то, что её слепили?>>


\section{Легенда об обретении}

Философский роман, написаный автором, имеющим домашнее имя Карлик (прочие имена неизвестны).
Впоследствии ушёл в народ и стал передаваться из уст в уста.

Молодой жрец по имени Ликхмас встречается со стариком, который говорит ему: кихотр Безумного способен вершить судьбы людей.
Но на нём есть грань, выпадение которой означает смерть самого бога, и раз в десять тысяч дождей рождается человек, который способен бросить камень именно этой гранью вверх.
Старик сообщил, что юноша и есть тот самый человек.
Ликхмас понимает, что способен положить конец бесконечным жертвоприношениям и войнам.
Он отправляется в путь, чтобы узнать способ добыть кихотр.
С ним отправляется могучий воин, лучший из бойцов --- Маликх, и пронырливый, обладающий сладким голосом и даром убеждения купец --- Чхалас.
Вместе друзья преодолевают ужасные злоключения и наконец достигают горы Рыбья Флейта --- самого высокого пика Старой Челюсти.
На вершине они находят тот самый кихотр.

Возвратившись домой, Ликхмас рассказывает о камне и своих намерениях друзьям, и вскоре об этом знает уже весь город.
По наущению жрецов воины собираются и идут к Ликхмасу домой.
Чхалас узнаёт о заговоре и успевает предупредить Маликха перед тем, как глотнуть из чаши, в которую трактирщик подлил лаковый сок.
Маликх мчится к дому Ликхмаса и успевает преградить путь толпе.
Перед дверью завязывается бой, Маликх в одиночку сдерживает две сотни врагов, но в ход пошли факелы и горящие стрелы --- дом загорелся.

Ликхмас решается бросить кихотр, чтобы положить конец Безумному, но ему вонзает кинжал в сердце родной брат.
Сестра успевает подхватить выпавший из пальцев умирающего камень перед тем, как тот упал на землю.
Уставший, израненный Маликх вместе с двадцатью оставшимися врагами отступает в занимающийся огнём дом и, видя окровавленное тело друга, в отчаянии отрубает руку сестры с зажатым в ней кихотром.
Артефакт падает на пол.
Враги успевают выбежать, а великий герой с телом друга и божественным камнем остаётся внутри.
Вслед за домом полностью выгорел город, и люди вынуждены покинуть пепелище\FM.
\FA{
Согласно одной из версий, этим городом был Тхитрон, так как с цатрона название переводится как <<пепелище>>.
}
Вскоре это место поглотили джунгли.

Согласно другой версии, перед тем как Ликхмас умер, кихотр всё же упал на пол, и маленький братишка юноши успел увидеть то, что было нарисовано на этой грани.
Его успели вытащить из огня, и когда он повзрослел, то уплыл за пролив Скар в Яуляль, перед походом поклявшись доверить тайну кихотра тому, кто способен её понять.

Возможно, что роман изначально был написан в жанре <<выбери конец сам>> --- такой приём был распространён в то время.

P.S. Писатель свёл воедино в одном произведении людей из разных эпох.
Маликх --- мифический герой, обладающий божественной силой --- никогда не существовал.
У трикстера Чхаласа был реальный прототип, живший на 500 дождей позже появления легенд о Маликхе, т.е. встретиться они не могли никак.
Но произведение оказалось настолько сильным, что многие считали его логическим завершением легенд о купце и воине.

\section{Лунные сады, или Повесть о странах цветов}

Луна (sl: luna) --- естественный спутник планеты.
Основой, скорее всего, послужила история, зародившаяся ещё до прибытия первых людей на Тра-Ренкхаль --- собственных лун у планеты не было.
Луна --- магическая страна света, <<висящая далеко над землёй>>.
Согласно легенде, там живут прекрасные бессмертные девы, собирающие в садах яблоки бессмертия.
Они очень печальны, потому что им нет пути на землю.

Впоследствии легенда была адаптирована культурологами-тси в рамках подготовки к одичанию (В частности, <<летающий змей>> --- это явная отсылка к Стальному Дракону, в более ранних версиях герой летел к Луна верхом на черепахе).

\subsection{Сокар}

История начинается с того, что Марин потерял мать.
Он вернулся после похорон в пустой дом и не знал, чем заняться.
Друзья (Ларин (Лар), Тагир (Тар)) предложили ему отправиться на охоту.
Они дошли до края острова и увидели другой остров с большим городом, проплывающий мимо.
Марин сказал, что хочет попасть в страну Луна, что его здесь ничто больше не держит.
Друзья огорчились, но решили ему помочь.
За час они собрали баллисту и зацепили остров.
В последний момент верёвка соскользнула и Марин полетел вниз, но всё-таки благополучно добрался до острова по верёвке.
Тагир с Ларином развели костёр и до темноты разговаривали о происшедшем, удивляясь неожиданной смелости Марина.

Недалеко от селения, в лесу, под скалой Марин обнаружил жалобно пищавшего птенца хищной птицы размером с взрослого человека, запутавшегося в ветках.
Марин потратил весь день на то, чтобы вытащить несчастную птицу.
Предположив, что птенец упал со скалы, Марин решил поднять его обратно.
На скале действительно обнаружилось огромное гнездо, и Марин аккуратно положил птенца на место.
Обернувшись, Марин увидел, что прямо за ним на скале сидит взрослая птица и смотрит на него --- Марин даже не услышал, как она прилетела.
Он уже ожидал смерти, когда птица заговорила с ним на его языке.

Она спросила, почему он не убил птенца и не разорил гнездо, ведь местные жители поступают с ними именно так из-за того, что орлы таскают их баранов.
Но орёл также сказал, что они никогда не брали баранов больше, чем им нужно для пропитания.
Марин ответил, что лично у него орлы не взяли ни одного барана.
Орёл оценил доброту человека, дал ему немного мяса и показал пещеру, где можно выспаться.

Марин никак не мог заснуть и пришёл к орлу --- поговорить.
Рассказал ему о своём поиске.
Орёл удивился и пообещал отвести Марина к Симуру.
Симур выслушал рассказ орла о случившемся и спросил, что бы Марин хотел в награду, ибо орлы размножаются раз в десять лет по одному птенцу.
Марин ответил, что орёл уже накормил его и он очень благодарен за это.
Симур удивился, потом что-то сказал приближённому.

Птица преподносит дары --- ягоды.
Герой придумывает аргументы против.
Синяя ягода --- знание языков, красная ягода --- исцеление, етц.
Опасность синей ягоды --- нельзя слушать речи летающих змей.
Они искусны в речах, и если проведают, что ты говоришь на знакомом языке --- убедят в чём угодно.
Лишь птицы сравнятся с ними в красноречии.
Птицы вели со змеями войну (орёл со змеёй в клюве).

Симург сказал, что никто из них не сможет отнести его на Луна, но если Марин будет носить эту ягоду с собой, то будет понимать любые языки, и что отныне Марина считают другом все птицы.
Марин поблагодарил Симура и орла и пошёл дальше.

\subsection{Хазер}

Город Хазер.
Город богачей, в котором даже уборные стоят немалых денег.
<<Если бы жизнь в Хазере была дешёвой, сюда бы не стремились бедняки и мы бы тогда сами стали бедняками>>.
Конное поло в роскошных доспехах и на роскошных лошадях.
Юноша из богачей увидел сон, что игра обессмертит его имя.
Бессмертие --- единственное, чего не было у богачей.
Он вышел на поле без доспехов.
Затем купил прямо в игре лучшего коня у богатого старика, отдав ему тяжёлую золотую сбрую.
И погиб на поле (продумать фабулу).

\subsection{Трогваль}

Город с типичной теократией.
Люди поклоняются Богу-Творцу и Пророку его.
Правители называют себя самыми прогрессивными, ибо поклоняются одному богу, а не многим, как варвары.
Основная масса народа --- религиозные фанатики, правят городом жрецы, живущие в Чистом городе.
Поддерживают власть жрецов торговцы (Торговый квартал), религиозная армия (Стальной квартал).
Остальная масса народа --- крестьяне и ремесленники (Город Мастеров).
Чтение светских книг, магия, нехрамовая музыка и песни, нерелигиозное искусство считаются греховными, хоть и порой необходимыми, из-за чего над людьми постоянное чувство вины.
Большая часть магов, учёных, врачей, летописцев, музыкантов и художников живёт в Проклятом городе, или Городе Безбожия --- месте, куда простой люд не ходит без необходимости, а если и ходит, то только держа перед собой руки, сложенные в огораживающем от зла жесте.
В Проклятом городе валютой являются знания.
Периодически (после мора, поветрия, неурожая) толпа народа разоряет Проклятый город.
Тем не менее люди пользуются услугами жителей Проклятого города --- к ним ходят лечиться, правители обращаются к летописцам, чтобы те прославляли их святость, праведность и справедливость, учёные и маги предоставляют правителям устройства, имитирующие божественное чудо.

Марин пришёл в Проклятый город поздним вечером.
Его окликнул человек, сидевший под вывеской <<Врачеватель Шамаль>>.
Они познакомились.
Когда человек узнал, что Марин с другого острова, он пригласил его к другу-летописцу, Салему, пообещав, что тот накормит его и даст ему ночлег за рассказ о его родине, о местах, где Марин побывал.
Удивлённый Марин пошёл за Шамалем.
По пути им попались несколько жителей города, боязливо идущих к врачу с жестом, отгоняющим зло.
В тёмном переулке к ним подошли два дюжих подвыпивших парня из города мастеров, требуя денег.
Шамаль с Марином опрокинули парней, Шамаль выбежал на открытую улицу и крикнул по-птичьи, набежал народ и обработал обнаглевших мастеровых.
Шамаль сообщил, что такое иногда бывает и что у жителей Проклятого города свои средства защиты --- от сыновей жрецов, приехавших позабавиться в Проклятый город, откупаются, а обнаглевшим простолюдинам просто можно намять бока.
Шамаль довёл Марина до дома Салема, тот накормил и напоил путешественника и всю ночь расспрашивал его о родном острове.
На встречу пришёл также маг Русталь, который попросил у Марина посмотреть его синюю ягодку, клятвенно пообещав вернуть её на следующее утро.
Шамаль заверил Марина, что в Проклятом городе никогда не обижают чужестранцев.

Марин жил у Шамаля, помогая тому с работой --- собирал травы и проч.
Вечерами жители собирались у кого-нибудь дома, проводили время за интеллектуальными беседами, музицированием, вином и игрой в Метритхис.

Замечен был астрономами остров, который приближался с юга.

В городе случилось чумное поветрие, распространившееся из-за купания в святых купелях.
Врачи и маги поголовно ушли в город, лечить народ.
Марин остался в Проклятом городе.
С чумой кое-как справились, врачи вернулись, и тут кто-то пустил слух о том, что всё это --- вина безбожников Проклятого города.
Озверевшие крестьяне и ремесленники отправились громить Проклятый город.
Шамаля предупредил кузнец, прибежавший из Города Мастеров.
Когда Шамаль спросил, почему тот это сделал, кузнец ответил, что берёт грех на душу в благодарность за спасение дочерей.
Салема предупредить не успели --- Марин нашёл его тело брошенным на железный штырь.
Потом Марин увидел сцену на улице --- несколько парней поймали девушку.
Вначале поставили её на колени, заставили целовать Символ Веры, а потом со словами <<Парни, причастим её!>> изнасиловали.
Марин хотел прийти ей на помощь, но Шамаль удержал его --- врагов было больше пятнадцати, и все с оружием.
Шамаль с семьёй и Марином бежали в лес.
Шамаль дал Марину еды, лук, стрелы и верёвку, указал дорогу к скалам и проводил.

\subsection{Талиал}

Маленькие деревушки, в которых люди живут по законам естественности.
Искусственно контролируется рождаемость --- люди сами перестают размножаться, если население превышает некоторую константу.
Никто не боится смерти и болезней, старики и неизлечимо больные уходят умирать в горы, а остальные больные отправляются в отдалённые места к лекарям.
Понятие <<семья>> не закреплено жёстко --- мужчины и женщины находятся в свободных отношениях, детей воспитывает их род, а кормят все, у кого есть еда. 

Марин нашёл драконье гнездо, подстерёг дракона.
Затем, не давая ему говорить, схватил за усы, завязал ему усами пасть и полетел к Луна.

\subsection{Луна}

В Лунных садах было много девушек, которые обхаживали Марина.
Случайно он познакомился с Еленой, которая избегала остальных и его.
Елена собирала вишню, и они разговорились.
Елена рассказала, что на Луне был ещё один человек с Земли, Эдар.
Но его отравила одна из девушек, Аина, когда он её отшил.
Затем, поняв, что она наделала, Аина отравилась сама.
Елена была лучшей подругой Аины.
Елена посоветовала Марину не проявлять ни к кому из девушек предпочтения, чтобы остаться в живых.
Пока они разговаривали, незаметно влюбились друг в друга.

Марин пришёл на встречу с Еленой.
Она сказала, что это погубит их.
Марин предложил бежать.
Елена рассказала ему о лодке Селены.
Они уже спланировали бегство, когда их увидела Ирина.
Марин с Еленой расстались.
Ирина подстерегла Марина на причале, зажала его в кустах и, угрожая оружием, попыталась его склонить к сексу.
Тот отказался.
Тогда Ирина выдохнула Марину в лицо сонную лунную пыль, отчего тот заснул крепким сном, и потащила тело на причал Лунных садов, чтобы сбросить его вниз.
В этот момент подоспела Елена.
Завязался ножевой бой.
Елена пыталась закрыть собой спящего Марина.
В приступе гнева Ирина рассказала, что это она подговорила Аину отравить Эдара за то, что тот посмел не желать её.
Затем Ирина отбросила Елену и сумела-таки столкнуть Марина вниз.
Елена бросилась вслед за ним.

В полёте Елена сбросила платье, чтобы догнать Марина, и с помощью лунной силы выманила из Марина сонную пыль.
Марин очнулся.
У них с собой была верёвка с крючьями, стрелы и лук.
Они сумели поймать крохотный островок и спастись.
Вся лунная сила Елены пошла на то, чтобы пробудить Марина, она потеряла свой лунный свет и стала обычной женщиной средних лет с померкшими глазами.
На островке не было никакой еды, и топливо быстро подошло к концу.
Они уже готовились к смерти, когда их спасли орлы, перенеся на Талиальский остров.
Они думали, что оказались в чужой стране без средств, но вдруг Марин обнаружил в кармане золотую лунную вишенку.
Они с Еленой прорастили её.

\section{Тёплый Хетр}

Жил-был однажды человек по имени Хитрам.
Он держал большой постоялый двор на окраине города.
Он прославился среди жителей города своим весёлым нравом, своими кулинарными шедеврами и необъятным животом.
Зарабатывал Хитрам достаточно, а на выручку ремонтировал свой двор, приглашал всех бродячих музыкантов и устраивал бесплатные ужины, на которые готовил свои самые лучшие блюда.
Люди вначале смеялись над странным улыбчивым толстяком, а потом прониклись к нему любовью.
Он пользовался таким авторитетом, что иногда на его дворе устраивали даже переговоры, и двор считался нейтральной территорией, подобно купеческому Двору.
Женщину Хитрам нашёл себе под стать --- полную, краснощёкую крестьянку, которая была его помощницей и главной ценительницей его творений.

Однажды в город пришли пылерои Предгорий.
После непродолжительной осады они ворвались за стены, убивая жителей и забирая в плен детей.
Улица за улицей переходили к ним в руки, и вскоре оставшиеся в живых отступили за последнюю черту обороны --- в храм.

Хитрам в это время с тяжёлым чувством готовил кушанья у себя дома.
Он не умел держать щит, он не умел обращаться с саблей.
Пылерои тем временем окружили храм, и жителям окраин ничего не оставалось, как отступить в двор Хитрама --- это было единственное хоть сколько-нибудь укреплённое место.

Легенда гласит, что пылерои собирались взять двор с ходу, но толстяк-хозяин вышел к ним навстречу с горшочком похлёбки в руках.
У него дрожали ноги и руки, но он улыбался, как и всегда.
Пылерои замерли, рассматривая странного человека.
Предводитель не спешил нападать, подозревая засаду.

--- Отведайте моей пищи, --- сказал Хитрам на цатроне, --- в этом доме нет воинов, здесь только едоки, музыка, смех и вино.

Один из воинов спустил тетиву --- и стрела воткнулась Хитраму в бедро.
Толстяк вздрогнул и едва не уронил горшочек, но упрямо повторил ту же фразу.

Пролетело ещё десять стрел, но Хитрам не упал, лишь попятился и прислонился к двери, сжимая горячий горшочек в трясущихся пальцах.

--- Отведайте моей пищи, --- повторил Хитрам, --- в этом доме нет воинов, здесь только едоки, музыка, смех и вино.

Предводитель пылероев подошёл к хозяину и вырвал горшочек из его рук.
Попробовал.
Затем в один глоток опустошил горшочек и разбил его вдребезги о камни.

Воины с опаской смотрели на предводителя.
Тот долго стоял и смотрел на толстяка-хозяина, который из последних сил держался на дрожащих ногах.
И вдруг предводитель махнул рукой и направился к храму.
Пылерои тёмной рекой последовали за ним.

Храм пал в ту же ночь, его защитники были перебиты.
Наутро умер от ран Хитрам.
Во всём городе остались в живых лишь те, кто нашёл убежище в его постоялом дворе.

С тех пор Тёплый Двор стал святилищем.
Каждые десять дней его хозяин бесплатно кормит и поит вином всех желающих.
Менестрели со всех краёв земли считают за честь спеть свои песни в его стенах.
За всё время существования там ни разу не обнажалось оружие.
А Тёплый Хетр занял своё место среди лесных духов, став хранителем домашнего очага, котла, сковороды и вертела, покровителем кулинаров, виноделов и толстяков.

Легенда гласит, что через несколько десятков дождей после этих событий, когда город вновь ожил, стражники загнали пылероя-лазутчика.
Тот, не видя путей для спасения, бросился... в Тёплый Двор.
Все воины были единодушны --- лазутчика нужно убить. Но тогдашняя хозяйка Тёплого Двора --- женщина Хитрама, Ситлам ар’Сар --- была непреклонна.
Пылероя оставили в живых, и он ушёл, получив свою порцию пищи.

\section{Удивлённый Лю}

Однажды в городе Кахрахане жил жрец по имени Люситр.
Он слыл человеком странным --- не любил общаться с людьми, всю жизнь провёл в библиотеке, переписывая книги.
Очень часто его видели в окрестностях --- он наблюдал за птицами, звёздами или собирал травы, и с его лица никогда не сходило глуповатое, удивлённое выражение.

В городе Люситра не очень любили --- он казался жителям высокомерным.
Стоило кому-то завести со жрецом разговор, как Люситр, не слушая собеседника, начинал говорить о разных вещах, которые он видел и слышал.

Однажды Люситр растрезвонил по всему городу, что полетит по воздуху, словно птица.
Любопытные собрались на площади перед храмом, и в назначенный час жрец вышел на крышу.
За его спиной было странное треугольное полотнище ткани, растянутое между палочек из лёгкого Дерева Перьев.
Люситр под изумлёнными взглядами людей спрыгнул с крыши храма... и полетел.

Люди кричали в изумлении, а Люситр парил над их головами, поднимаясь всё выше и выше.
Потом жрец направил своё странное приспособление к морю.
Жители города бежали, стараясь не потерять его из вида, но Люситр летел в бескрайний морской простор.
Вскоре он уже казался крохотной яркой точкой, и его ликующий смех затих вдали.

Больше его никто не видел.
Кто-то говорил, что смелый жрец попал в шторм и сломал свои чудесные крылья, кто-то говорил, что он обрёл новый дом где-то на Ките, в землях ноа.
На берегу до сих пор стоит тотемный столб с ликом Удивлённого Лю, и каждый уважающий себя жрец, книжный человек или воин-разведчик считает своим долгом посетить это место --- починить, подкрасить или навязать лишнюю погремушку на этот тотем.
Многие посетители просто пишут свои имена или пожелания людям.

Жрец оставил после себя богатое наследие.
После были обнаружены его записи.
Люситр узнал о лекарственных свойствах многих сорных растений, испытывая их на себе, и начертил множество схем устройств, включая самопишущие перья и конденсатор для книгохранилищ.
Только сейчас люди поняли, о чём пытался говорить с ними Люситр.

--- Какое несчастье, --- говорили они, --- мы могли бы узнать это раньше, но никому и в голову не пришло послушать, что он говорит!

Устройство его крыльев так никто и не узнал.
Кое-кто годы спустя пытался повторить подвиг Люситра, и многие из этих храбрецов разбились насмерть.
Их черепа и неудачные летательные аппараты лежат в крипте рядом с тотемом Удивлённого Лю.

Третий день месяца Согхо стал с тех пор праздником.
Люди надевают бутафорские крылья и танцуют танцы, некоторые, обвязываясь верёвкой, спрыгивают с ритуальных столбов.
И в последний час перед закатом все люди идут на берег, садятся и в молчании ждут возвращения Люситра.
Кто-то верит, что в день, когда Удивлённый Лю прилетит обратно, все люди обретут крылья и смогут летать.

\section{Имя}

--- Тебе не нравится слово <<Скорбящие>>?

--- Разумеется, нет, --- скривился Лу.
--- Надо обладать интересным складом личности, чтобы найти в скорби нечто привлекательное.
Но изменять название я бы не стал.

--- Потому что от названия ничего не зависит?

--- Зависит.
И никакой мистики тут нет.
Слово изменяет того, кто его слышит, а собственные имена нам приходится слышать постоянно.

--- Тогда почему бы ты не стал менять?

--- Даже с этим малопривлекательным названием всё сложилось неплохо.
Я просто отдаю ему должное.
Бездна тебя возьми, Грейс, мы два телльна были адептами Ордена Преисподней!
Не скажу, что это были худшие два телльна моей жизни.

--- Это были единственные два телльна твоей жизни.

--- Вот именно, --- кивнул Лу и сделал неприлично долгую затяжку, словно пытался заново прочувствовать на вкус два телльна существования.

\section{Глупцы и лжецы}

--- Но ведь истинное равновесие Нэша...
Ведь есть множество тех, кто искренне верит в правильность существующего миропорядка.
Что делать с ними?

--- О, не волнуйся, Грейс.
В мире гораздо больше глупцов и лжецов, чем ты думаешь.
И это прекрасно.

--- То есть Вселенная примет новый миропорядок без жертв?

--- Я думаю, что да.
Глупцы встают на сторону убедительного.
Лжецы встают на сторону большинства.

\section{Хитрый план Лу}

--- И в чём же заключается твой великий план? --- насмешливо спросил Грейсвольд.

--- О, всё предельно просто, --- сказал Лусафейру.
--- Однажды мы завербуем в ряды Скорбящих подавляющее большинство.
Собственно, от старого мироустройства останется только скорлупка, внешняя оболочка.
Затем по моей команде все разом снимут маски.
Будет очень смешно.
Особенно посмеются те, которые только что хотели друг друга убить.

--- И ты думаешь, что на этом конфликты будут исчерпаны? --- скептически прищурился технолог.

--- Разумеется, нет, --- отмахнулся Лусафейру.
--- Однако, согласись, гораздо проще решить проблему, если двое --- рядовые агенты Скорбящих, а не два максима --- Ада и Картеля.
У вторых друг к другу гораздо более древние счёты.

--- Личность --- это совокупность ролей в различных объединениях, --- согласился Грейсвольд.

--- А ещё я уповаю на естественное желание здоровой личности жить хорошо.

--- И это не лишённая смысла надежда.
\ml{$0$}
{Но всё-таки --- что, если не выгорит?}
{But after all, what if the plan won't come off?''}

\ml{$0$}
{--- Ну и чёрт с ним.}
{``Fuck it then.''}

--- Я тебя обожаю.
Я бы так не смог, честно.
Всё-таки во мне сидит эта смертельная слабость --- желание, чтобы мои начинания успешно заканчивались.

--- Это хорошее желание.
Но без удовольствия от процесса оно не значит почти ничего.

--- И наоборот тоже.
Какой смысл в незаконченных начинаниях?

--- Конец --- это условность, Грейс.
Воображаемая точка, делящая прямую на два совершенно одинаковых луча.

--- Скажи это своей смерти, когда она к тебе придёт.

--- А ты хорош.

\section{Лошадь}

--- Большую книгу сложно писать, --- объяснил Грейсвольд.

--- О да, --- откликнулся Лу.
 --- Это как руководить или ехать верхом.
Ты, конечно, можешь думать, что ты управляешь лошадью.
Но умная лошадь очень быстро объяснит тебе истинное положение дел.

--- Я не понял, что ты имеешь в виду, --- признался толстяк.

--- Всему своё время.
Твой метод управления немного другой.
Ты просто занимаешься своим делом, и это привлекает к тебе сторонников.
Если же их не будет, ты всё равно будешь заниматься своим делом.

--- Я всегда считал главным тебя.

--- Ты просто признавал полезными мои навыки организации.

\section{Союз Гало и Тахиро}

--- Гало и Тахиро погибли вместе, на одной планете, в одной войне на уничтожение.
Кто знает, чего они могли бы достигнуть, объединившись?

--- Я знаю, --- грустно усмехнулся Лусафейру.
--- Ничего хорошего.
Воины хороши только на войне.
Может, это счастье Вселенной, что два великих стратега-воина нашли друг в друге врагов.

\section{Сеанс одновременной игры}

--- Кстати, сегодня игра будет?

--- Вроде должна.
Двенадцать столов на миллион фигур наши, плюс ещё сто сорок столов по сто тысяч фигур для любителей.
Машина сказала, что вполне вытягивает против нас двести столов, даже в блиц-игре, по микросекунде на ход.
А вот триста уже сложновато.

--- Да она всё равно выиграет опять все сто пятьдесят два стола!
Вопрос только в том, кто выиграет вместе с ней.

--- В прошлый раз она поддавалась --- параллельно с игрой обсчитывала биостатистику для отдела Хараты.

--- Я что-то не заметил!

--- А десять дней назад она продула один стол из двухсот --- Стигма с Мимозой и Ду-Си её обхитрили и заключили мир.
Ду-Си, правда, болтался в игре, как младенец в водах Нила --- делал глупые ходы, отпускал неприличные шуточки и хохотал.
К удивлению всех, это и сработало.
Машину сбил с толку его стиль игры, а Стигма и Мимоза стратегией и тактикой вывели всех троих к победе.

--- А, так вот что они так громко праздновали у себя в отделе!

\section{Копии}

--- Ой, не там они меня ищут, --- захохотал Лу.

--- Твои копии справятся?

--- И я справлюсь, --- кивнул Лу.
--- Я отдохну и накурюсь за них за всех разом.

--- Странно, --- сказал Грейсвольд.
--- Ведь разделение на копии --- древняя как мир идея.
Почему же по-настоящему это получилось только у тебя одного?

--- Видимо, погрязнув во лжи и интригах, только я один могу доверять самому себе.
Кстати, скоро выйдет из игры моя вторая копия в Картеле.
Ты же приготовишь ему тело?

--- Я окажу ему королевский приём.

--- Просто обними его и дай ему покурить.
Это единственное, чего он хотел все эти годы.
Потом мы с ним интегрируемся.
Затем ещё четверо... и Вселенная будет совсем другой.

--- У тебя будет очень много работы.

--- Это будет гораздо более приятная работа.
Зато потом, когда всё уляжется, я просто буду сидеть и заниматься своими делами.
Вот меня всегда удивляла твоя потребность что-то мастерить руками --- машины, планеты, игрушки.
Вдруг это моё призвание, просто у меня не было шанса попробовать?

--- Я научу тебя всему, что знаю, дружище.
Но с планетой не так всё просто.
Это большой проект.
Планетой надо жить.

--- Как я по тебе соскучился, ты не представляешь, --- Лу выпустил большой клуб дыма.
--- Я бы тысячелетиями просто сидел и слушал музыку твоего голоса, твои беседы с Атрисом про планеты и про всё, что вам интересно.

\section{Близнецы}

--- Соперничество абсолютно ничего не поменяло, --- сказал Лу.
--- Судьба в итоге всё расставила по местам.
Гало исполнял приказы, я правил.

--- Но твой демон и демон Гало были идентичны при создании.
Что же сыграло решающую роль?

--- Наши тела, разумеется, --- ухмыльнулся Лу.

--- Ваши тела были братьями-близнецами!

--- Пока они были на стадии одной клетки, они были идентичны, --- пожал плечами стратег.
--- Ну а дальше сработал эффект бабочки и отец, желавший сделать из нас соперников.
Мы разошлись по двум концам нормы реакции.
Пока Гало закалял дух в скалистых пустошах, я выпрашивал пирожки в деревне.
Пока Гало томил себя воздержанием и колол блокаторы полового голода, я мастурбировал в кровати и спал с Айну.
Когда Гало побрил голову, я покрасил волосы перекисью и завил их в кольца.

--- Эйраки потом публично унизил тебя, сбрив тебе волосы, --- припомнил Грейсвольд.
--- Гало заступился за тебя, но Эйраки был непреклонен.
Легион смеялся...

--- А я отрастил и покрасил их снова, --- подтвердил Лу.
--- Пока отец хвалил Гало за стойкость, пока легионеры сами шли за Гало толпами, мне приходилось идти против мнения всего Ордена и искать сторонников по одному.
Я пробуждал их очевидные потребности и учил жить по ним, а не по навязанным системой правилам.
Эти мелочи в итоге и определили судьбу двух демонов.
Гало думал, что принёс великую жертву, что его популярность заслуженна.
Но он и представить не мог, через какое горнило пришлось пройти мне.
Вернее, он это чувствовал и негласно признавал моё старшинство, пока отец не взялся за него серьёзно.
Помнишь ведь, он всё-таки отрастил волосы потом.
И курить начал, несмотря на запреты отца.

--- А потом положил всю жизнь на свой личный бунт против системы, словно пытаясь что-то наверстать, что-то доказать...

--- И умер за этот бунт, который я перерос за пару лет в юности.

Грейсвольд улыбнулся.

--- Под конец он всё же выбрал тебя.

--- Себя, --- поправил Лу.
--- Себя он выбрал.
Всё-таки мы с ним когда-то были идентичны.

\section{Выписка}

Выписка из архивов отдела 100.

По запросу номер (номер скрыт).

Запись номер (номер скрыт).

Анкарьяль Кровавый Шторм и Грейсвольд Каменный Молот проявили высочайший профессионализм при уничтожении пяти кластеров Скорбящих.
Несмотря на явное временное преимущество противника, спецоперация была проведена ими с эффективностью, превышающей ожидаемую на 32\%.
Достоверных признаков связи вышеуказанных демонов с мятежниками не выявлено.

Ряд наблюдателей (имена) отметил, что уничтожение мятежников центурионом Анкарьяль имеет черты актов милосердия.
Комиссия, рассмотрев отчёты, признала, что эти данные вполне укладываются в общую картину личности интерфектора.
Статистически значимой разницы между отношением центуриона к агентам Картеля и агентам Скорбящих не выявлено.

Вердикт: центуриона секунда отдела 100 Анкарьяль рекомендовать к повышению в ранге на два пункта с зачислением в подразделение быстрого реагирования отдела 100, центуриона прима отдела 100 Грейсвольда рекомендовать к повышению в ранге на один пункт с зачислением в подразделение технической обороны отдела 100.
Рекомендуемые исключения из соответствующих рангу полномочий обоих легатов терция отдела 100 (3 исключения) в приложенном документе (номер документа).

Извещение о смерти номер (номер скрыт).

Легат терция Анкарьяль Кровавый Шторм погибла в результате диверсии Картеля при исполнении служебных обязанностей 24.0002.453227, планета Ку-Лань, империя Плеяды, согласно донесению Хуре Зелёный Сад.
Имя легата занесено в хроники славы Ордена Преисподней.

Данная заверенная цифровой подписью копия выписки выдана легату прима отдела 100 Грейсвольду Каменный Молот по личному запросу.

\section{Завет Айну}

\subsubsection{Из архивов Ордена Преисподней}

Послание типа <<завет>>. Архивный номер (скрыт).

Автор: Айну Крыло Удачи

Получатель: Аркадиу Шакал Чрева

\subsubsection{Текст}

<<Я всю жизнь была воином, Аркадиу Люпино.
Когда приходит мир, воин остаётся не у дел, и я искренне рада, что не увижу этого дня.
В тебе же есть задатки не только воина, ты сможешь найти себя и в мирное время.
Поэтому живи.
Я научила тебя всему, чему могла, и ухожу.
Вместе со мной поляжет достаточно наших врагов, и я надеюсь, что тебе будет чуть легче>>.

\subsubsection{Анализ}

Группа AD44, отчёт Иттме Холодный Осколок:

<<В письме содержатся намёки следующего характера:

\begin{enumerate}
\item Сомнение в правильности глобальной стратегии, выбранной Адом (.928)
\item Побуждение Получателя к смене специализации, идущей вразрез с интересами Ада (.951)
\item Указание на Ад как идеологического противника Получателя (.870)>>
\end{enumerate}

\subsubsection{Рекомендации}

Отдел 100, оператор номер (скрыт): <<Рекомендуется к применению протокол №34>>.

\subsubsection{Статус}

\begin{enumerate}
\item Доставка прервана по запросу 3 степени (ссылка на текст запроса).
\item Архивной записи присвоена 4 степень секретности.
\item Архивная запись ассоциирована с досье Автора и Получателя.
\end{enumerate}

\section{Иллюстрации}

--- Ты решил сделать к книге иллюстрации? --- удивилась Анкарьяль.

Я кивнул.

--- Так читателям лучше удастся понять происходящее.

Анкарьяль отобрала у меня компьютер и просмотрела картинки.

--- Ммм.
Ты взял большую часть рисунков Тхарту и добавил кое-что своё.

Я снова кивнул.
Анкарьяль полистала ещё, нахмурилась.

--- А почему нигде нет портрета Чханэ?

--- Она не любила, когда её рисовали, --- пожал я плечами.

--- И что? --- возмутилась Анкарьяль.
--- Нельзя так.
Портреты почти всех героев есть, а вместо одного из центральных --- пустое пятно.
Ну-ка давай рисуй.
Прямо сейчас.

Я снова пожал плечами и принялся рисовать.
Анкарьяль, сделав несколько кругов по комнате, наконец подошла и критически осмотрела рисунок.

--- Очень похоже.
Но она у тебя получилась грустной.

Я задумался.
Да, почему-то я запомнил Чханэ именно такой.

--- И ещё... Девочки не любят, когда их шрамы оказываются на их портретах.

--- Брось эти древние предрассудки.

--- Я не шучу.
Может быть, Чханэ не любила позировать именно из-за шрамов?

--- Нет.
И без шрамов это будет уже не Чханэ, --- отрезал я.
--- Хватит об этом.
Кстати, твой портрет я тоже не нарисовал.

--- О, давай.
Нарисуй меня такой, какой запомнил.

Я задумался и набросал портрет.

--- Эй! --- возмутилась Анкарьяль.

--- А по-моему, очень похоже получилось, --- засмеялся я.
--- Обязательно вставлю этот рисунок в книгу.

--- Только попробуй, я тебе яйца оторву, --- посулилась Анкарьяль и вышла, ударив плечом дверь.
Но не очень сердито.

\section{Наркотик}

\textbf{К Аркадиу пришла Анкарьяль.
Сказала, что у культурологов проблема --- из далёкого мира прибыл разведчик и принёс данные о ритуалах местных племён.
Никто не может понять смысла ритуала.
Аркадиу пошёл с ней.
Они посмотрели ритуал, и Аркадиу подкинул им идею}

--- А здесь что?

--- А здесь Шиамис с командой испытывают новый наркотик.
Давай зайдём, покажу.

Анкарьяль приложила кудрявую голову к двери.
Дверь распахнулась.

Зрелище, которое предстало моим глазам, было не из приятных.
Чистая, хорошо освещённая лаборатория, скучающий демон в одежде врача, сидящий у панели управления.
И четыре капсулы, в которых лежали страшно худые, чёрные человеческие тела.

Я вошёл в лабораторию.
Демон-врач встрепенулся:

--- Анкарьяль.
А ты, как я понимаю, Аркадиу Шакал Чрева?
Красивое тело тебе собрали.
Добро пожаловать в отдел придурков.

Я улыбнулся.

--- Спасибо, Ациоджи.
Постараюсь соответствовать.

--- Привет, Аци, --- Нар улыбнулась и наклонила голову.
--- Что тут у нас?

--- Пока наблюдаем, --- развёл руками Аци.

Живой скелет в одной из капсул с трудом открыл глаза и улыбнулся вымученной страшной улыбкой.

--- Нар, здравствуй.

Анкарьяль подошла к нему.
Я последовал за ней.

--- Привет, Шиамис.
Как ты себя чувствуешь?

--- Ужасно, --- сухие губы скелета едва заметно шевелились, когда он говорил.
--- Этот наркотик...

Скелет заплакал, искривив губы.
Из опалённых глаз выкатилась крохотная слеза и тут же испарилась, оставив на коричневой коже светлую полоску.

--- Они скоро умрут, --- объяснил Аци.
--- Те трое уже в коме, осталось им от силы день.
Шиамис пока ещё разговаривает и даже в ясном сознании, умрёт дней через шесть.
У него чересчур крепкий организм.

--- Ты уж держись, дружище, --- я склонился над умирающим.
--- Скоро всё закончится.

--- Я знаю, я знаю, --- прошептал скелет.
--- Аци, давай следующую дозу.

Врач кивнул, что-то щелкнуло, и по системе полилась прозрачная жидкость.
Скелет закатил глаза, его тело свело судорогой, рот оскалился.
Сознание покинуло живой труп.

--- Им уже делают новые тела, --- шёпотом пояснила Анкарьяль.
--- Когда они заселятся в них, то предоставят полный отчёт о своих ощущениях.

--- Кошмарная работа, --- пробормотал я.

--- Да, похуже некоторых, --- грустно улыбнулся Аци.
--- До сих пор не понимаю, зачем они постоянно на это соглашаются.
Как новый яд или наркотик --- так сразу команда Шиамиса.

--- Кто-то должен, --- заметил я.

Аци хмыкнул.

--- Да они уже сделали для Ада больше, чем весь отдел биохимии, можно было бы и другую работу найти.
Этот наркотик вызывает страшные видения и не менее страшную зависимость.
Линд и Кен-Бит перед комой то умоляли прекратить, то просили ещё, плакали, несли какую-то чушь.
А моё дело --- продолжать и наблюдать за всем этим.
Паршиво всё это.

--- Как использовался этот наркотик?

--- Это особо изощрённый способ казни.
Многие предпочитали покончить жизнь самоубийством после первой же инъекции.

--- Антидот?..

--- \ldots не нашли, --- скривился Аци.
--- Врачи пытались ради интереса восстановить тело Линд --- бесполезно, проще убить.
Химизм изменён кардинально.
За этим наркотиком чувствуется рука Картеля.
Химиограмму записали, надеемся узнать ещё что-нибудь после вскрытия.
Хорошо, что эксперимент подходит к концу.
Нам обещали хороший отпуск.
Ребята освоятся с новыми телами, напишем отчёты и гулять.
Надоела уже эта лаборатория.
<<Жаркие ночи, полные поцелуев...>> --- пропел Аци на языке тоно и нервно засмеялся.

--- Тебе бы не мешало подлечиться, --- заметил я.

--- Да, --- погрустнел Аци.
--- Энергию расходовать нельзя, здесь и так хватает отрицательных эманаций.
У меня система стоит, но она выдохлась, похоже, --- он убрал волосы со лба, показав внедрённый под кожу имплант.
--- Врачи сказали --- пока так, потом мы тебя вмиг восстановим.

--- Тебе заказать еду? --- сочувственно спросила Анкарьяль.

--- Если тебе не трудно, Нар, --- лицо Аци просветлело.
--- Что-нибудь острое или пряное.

Панель управления запищала.

--- О, Линд умерла.
Отлично, --- Аци облегчённо выдохнул и добавил куда-то в сторону панели:
--- Тахар, Линд готова, можешь вскрывать.

Панель утвердительно прорычала.
Демон-канин этажом ниже активировал оцифровку, и тело из крайней капсулы исчезло в голубых искрах.
Анкарьяль тем временем достала из почтовой капсулы пакет и, распечатав, поставила его на столик рядом с врачом.

--- О, рыба в кисло-сладком соусе!
Нар, у тебя определённо есть вкус, --- обрадовался Аци.

Анкарьяль кивнула демону и потащила меня к двери.

\section{Совесть и репутация}

Вспомнился первый год после возвращения с Тра-Ренкхаля.
Дверь засигналила, и я впустил в комнату грустного демона.

--- Аркадиу.

--- Минь, здравствуй.

--- Здравствую.
Я принёс данные, которые ты просил.

--- Мог бы и переслать по сети, незачем было самому бегать, --- улыбнулся я.

--- Тут есть некоторые... сложности, поэтому я решил передать тебе лично.

--- Рассказывай.

Минь положил передо мной проектор.
Я включил его.
Выключил.

--- Так значит, это точно?

--- Коэффициент более 0,95.
Отдел аналитики подтверждает.

--- Ты задействовал аналитиков? --- поморщился я.
--- Не надо было беспокоить их по такой ерунде.

--- Аркадиу, они годами занимаются скучными вещами.
А тут случай действительно интересный.
Даже Сир подключился, хотя он просто приходил к ним поесть.
Да и потом, это не такая уж и ерунда...

--- А в чём сложности?

Демон оглянулся.
В воздухе материализовались две летающие шарообразные машины.
Выглядели они достаточно устрашающе --- набор приёмных антенн, щупы, рецепторы, завершала экипировку мощная волновая пушка.
Отдел 100 --- контрразведка.
В тот же момент включился глушитель сигналов.
От наступившей тишины на миг заложило уши.

--- Привет, ребята, --- я вежливо помахал машинам, зная, что они не ответят.
Служба.
--- Что случилось?

Минь ответил за них:

--- По ходу дела выяснилось, что у нас в пяти базах данных находится дезинформация.
То, что я передаю тебе --- результат косвенных вычислений.

--- Агенты Картеля?

--- Отдел 100, --- Минь махнул на молчаливых роботов, --- уже занимается этим.
Аналитиков пока изолировали, нас с тобой, как видишь, тоже собираются.

Я кивнул.
Это была стандартная проверка.
После разговора роботы должны были увести меня в отдел на полный анализ.
Сталкиваться с контрразведкой было не очень приятно, но я знал, что туда берут самых лучших --- тех, кто не повторяет ошибок.
Этих демонов можно по праву назвать незримым щитом Ада.

Минь опустил голову.

--- Я уже почти пол-телльна работаю в безопасном отделе архива, Аркадиу.
Для меня это жестокий удар.
Я не думал, что информацию из безопасного отдела можно обернуть против нас таким образом.
Скорее всего, архивы закроют и подвергнут реорганизации.

--- Что говорилось в базах данных? --- спросил я.

--- Согласно базам данных, настоящее имя этого Атриса --- Ковнелий Фиктовий Саз.
Урождённый человек, преобразован ещё во времена Союза Воронёной Стали.
Он был под подозрением --- формально держал нейтралитет, но сотрудничал с Картелем.
После переворота на Сцелае сбежал и с тех пор ошивается в районе Тукана, девятнадцать зарегистрированных контактов с агентами Ада.
А тут выходит, что он...

--- \ldots что он действительно тот самый Добрый бог, Безымянный, демиург Тра-Ренкхаля, изгнанный узурпатором Эйраки, --- закончил я за него.
Последние кусочки картины встали на свои места.
--- Откуда взялся Безымянный?
Как его зовут?

--- Нигде об этом ни слова. В базах отмечено, что демиург Тра-Ренкхаля --- Хатрафель Безумный.
Ваша команда сообщила, что нгвсо почитают Безымянного.
С этого несоответствия ребята за пару часов распутали всю историю.
Сам понимаешь, если бы вас отправили на поиски демиурга, а не на борьбу с Безумным...

--- Понимаю, --- кивнул я.

--- Нет, ты только подумай!
Ведь отчёты об освоении планет тщательнейшим образом...

--- А что насчёт этого Ковнелия?

--- А...
У него приличная биография, построенная на данных погибших демонов.
Спрашивать, существовал ли Ковнелий на самом деле, разумеется, уже не у кого.
Агент Картеля --- кто бы он ни был --- постарался на славу, пролез где только можно.
Я не совсем понимаю смысл этой...

--- Картель опасался, что мы выйдем на Безымянного, и решил подстраховаться, --- предположил я.
--- Найти демиурга на его собственной планете они не могли --- тот продумал систему маскировки.
Вычеркнуть его из наших баз --- чересчур подозрительно, а придумать ему липовую неблаговидную биографию --- вполне себе хороший ход.
Даже если бы мы его встретили --- отправили бы <<на отдых>> как неблагонадёжного.
А слепое вторжение на Тра-Ренкхаль закончилось бы резнёй.

--- Вы молодцы, ребята, --- заметил Минь.

--- Ага, мы, --- саркастически проворчал я.
--- Битву за Тра-Ренкхаль мы выиграли благодаря невероятной случайности и находчивости Грейса.
Если кто и молодец, так это он.
Я боюсь даже предположить, сколько военной силы мы бы потеряли из-за этой подсадной утки.
Лусафейру всё-таки гений.
Он, похоже, подозревал, что тут не всё чисто...

Один из роботов выключил глушитель и впервые заговорил приятным женским голосом:

--- Аркадиу Шакал Чрева, Минь Орлиная Заря, прошу вас проследовать с нами на станцию С9A0.

За время разговора, я знал, они полностью проверили меня на предмет подслушивающих устройств, маячков, молекулярных механизмов регистрации и прочей шпионской техники, а также провели всесторонний анализ моей личности.
Выключенный глушитель означал, что я не представляю опасности.
По крайней мере пока.

Я улыбнулся и кивнул агентам.
Роботы растворились в воздухе.

--- Пошли, дружище, --- похлопал я по спине честного архивариуса.
--- У нас с тобой совесть чиста.

\section{Братья по разуму}

--- Грейс, у меня проблема, --- начал я.
--- Не могу найти понятную информацию по Ветвям Звезды.
Это форма жизни, но при обучении мне намекнули, что Звезда не в моей компетенции и занимаются ею другие биологи.
Не мог бы ты рассказать о них?

--- А, --- откликнулся технолог.
--- Хм-хм.
Ветви Звезды.

--- Может быть, это секретная информация и не стоит её?..

--- Нет-нет, --- перебил меня Грейсвольд.
--- Это информация общедоступная, но без интерпретации понять её сложно.
Слушай, попробую объяснить.

Я схватил компьютер и настроил на запись.

--- Как ты знаешь, наша область Вселенной предположительно является <<мёртвой зоной>> --- жизнь здесь встречается довольно редко.
Ветви Звезды --- это сапиенты с планеты 1-34, второй известной планеты со стабильной самозародившейся сапиентной жизнью.
В источниках, предназначенных для Земли, планеты Звезды обозначаются цифрами.
Их способ общения --- назовём его <<языком>> --- кардинально отличается от языков Ветвей Земли.
Он полностью химический, с помощью полимеров и низкомолекулярных веществ.
У Земли пообщаться со Звездой просто так не получится.

--- А что они собой представляют?
Кажется, их химический состав...

--- Да, химический состав.
Углерод, кислород, кремний, азот, сера, в целом то же самое, но на другой лад.
Поглощают метан, выдыхают углекислый газ.
Оптимально их существование при температуре кипения этанола и давлении, в 2.8 раз превышающем земное.
Своеобразное строение <<тела>>, назовём его так.
Сложно установить, где заканчивается одно тело и начинается другое.

--- Я так понял, что это нечто, похожее на мицелий?

--- Сложнее, --- покачал головой Грейс.
--- В мицелии есть отдельные клетки, а тут... ммм... многослойный синцитий.
У них есть <<города>> --- губчатый скелет из кварца с прослойками биоколлоида, занимающий огромные пространства...

--- И это существо разумное? --- удивился я.

--- Как ни странно, --- ответил Грейс.
--- У них есть технология, они вышли в космос и заселяют экзопланеты.
Вряд ли столько, сколько Земля, но ненамного меньше.

--- Понятно.
И они испускают при угнетении плюс-эманации, а при благоденствии --- минус?

--- А, да.
Звезда и хоргеты.
Интересный вопрос.
Да, 1-34 --- одна из самых мощных баз Картеля, именно поэтому мы знаем про Звезду не так много.
После поражения в Развязке Десяти Звёзд Картель отступил на территорию Ветвей Звезды и закрепился там.
Ордену Преисподней на планетах Звезды действовать так же трудно, как Картелю на территории Земли.

--- Я прочитал мнение, что впоследствии война между Картелем и Адом может перерасти в войну между Землёй и Звездой.

--- Я с этим согласен.
Звезда постепенно изменяет свой достаточно примитивный химический способ общения на более быстрый, волновой.
В будущем можно ожидать появление не только киборгов, основанных на биологии Звезды, но и совершенно новых существ.
Однако есть интересный нюанс.
Молекулярный механизм излучения эманаций, связанный с системой ответа на раздражение, введён в Звезду искусственно.
Каким образом вышло так, что эволюция живых существ вызывает минус-завихрения, ещё предстоит выяснить.
Также найдены нуль-штаммы Ветвей Звезды, то есть не излучающие эманации вообще.
Другой интересный нюанс --- молекулярный механизм построен по неизвестной ранее схеме, и хоргеты Картеля отрицают свою к этому причастность.

Я обомлел.

--- То есть?..

--- Да, ты правильно понял.
Возможно, это сделали хоргеты, созданные Ветвями Звезды.
Или вообще другие.
Кто знает, может, и сама Звезда --- искусственно выведенная хоргетами форма жизни?
Судьба этих братьев по разуму неизвестна.
Были ли они уничтожены в войне, подобной этой?
Примкнули ли они к Картелю?
Был ли ими выведен микоргет, ушедший в другие Вселенные?
Предстоит разобраться.

--- И это значит, что у нас появилась новая проблема.

--- Да, новая возможность и новая проблема.
Возможность использовать в своих целях Ветви Звезды и проблема контроля генофонда Ветвей Земли.
Вряд ли Картель упустит шанс вывести, например, минус-людей.
Если те, другие хоргеты примкнули к ним --- в технологическом плане мы на шаг позади.

Я обхватил голову руками.

--- А всё-таки, что мешает создать стабильный источник эманаций и оставить сапиентов в покое?
Сколько проблем было бы решено!

Грейсвольд вздохнул.

--- Да, Аркадиу.
Проблема ключевая для процветания нашего вида, но ею занимается катастрофически мало демонов.
Давно ты что-то слышал о разработке микоргета?
И я тоже.
Война, видимо, Ордену нужнее.

\section{Язык Эй}

Когда Красный Картель и Орден Преисподней начали войну за обитаемую Вселенную, остро встал вопрос коммуникации --- как между демонами, так и с сапиентами.

Картель и Ад и прежде использовали специальные языки для сражений (т.н. боевые языки, отличающиеся простотой и краткостью) и для передачи информации (шпионские языки, сложные для расшифровки).
Боевой и шпионский языки Ада, в частности, были созданы на основе языка сохтид --- самого распространённого человеческого языка на Преисподней.
В силу некоторой оторванности Картеля от цивилизаций сапиентов минус-демоны создали ещё один тип языка --- усечённый (sekta-lingu), с помощью которого демоны коммуницировали с подвластными им сапиентами.

У всех вышеперечисленных языков был один большой недостаток --- они были придуманы людьми в процессе эволюции и, как и любой живой организм, несли в себе огромный груз отрицательных мутаций.
Поэтому обе организации поставили задачу --- разработать единый язык, избавленный от пережитков прошлого.

Картель первым справился с поставленной задачей.
После долгих дискуссий было решено оставить секта-лингу для общения с сапиентами, а в качестве боевого и шпионского языков использовать новосозданный Чи.
Демоны понимали, что главное оружие криптологов --- это логика.
Вследствие этого между корнями и словоформами языка Чи не было никакой логической связи.
Фактически одно и то же слово из предложения в предложение менялось до неузнаваемости.
Словоформы придумывали сто тридцать генераторов случайных чисел, разработанных на основе человеческого мозга.
Словари языка Чи были строжайше засекречены, а для демонов, которые им пользовались, были созданы специальные системы защиты, мгновенно уничтожавшие языковой сектор в памяти при попытке проникнуть в него или выдать его постороннему.

Первый раунд в этой битве был блестяще выигран учёными Картеля.
Расшифровать Чи даже на процент не удалось до сих пор, несмотря на усилия разведки, криптологов и аналитиков.
Данные, приведённые здесь --- это, увы, почти всё, что Ад на сегодняшний день знает об этом таинственном языке.

Учёные Ада такими результатами похвастаться не могли.
В архив отправляли одну версию за другой --- какие-то браковались культурологами, какие-то криптологами.
Возможно, что это затянулось бы ещё на неопределённое время, если бы не вмешался один из старейших демонов, который служил Ордену Преисподней аж с момента его создания.

Ликан Безрукий.
Урождённый человек, один из жителей древней Преисподней.
Не совсем ясны обстоятельства, по которым он стал демоном.
Возможно, об этом знает кто-то из старожилов --- Лусафейру или Грейсвольд, но они не распространяются на эту тему.

--- Грейс, расскажи про Ликана Безрукого, ты ведь его знал.

--- Ааа, Ликан.
Да-да-да.
Достойный был демон, достойный.

На этом разговор обычно заканчивается.

Ликан Безрукий отказался от коррекции личности, которую проходят все урождённые люди.
Вы можете себе представить, что творится в голове у человека, который вынужден прожить несколько телльнов, и Ликан, как видно, потихоньку начал сходить с ума.
По словам очевидцев, общаться с ним было форменным наказанием.
Впрочем, своё дело (а работал он в Аду аналитиком) Ликан знал блестяще, занимался охотно и увлечённо, и причин отстранять его не было.

Узнав, что отдел 214 занимается разработкой нового языка, Ликан ходатайствовал о подключении его к работе, так как всю жизнь питал к лингвистике известную слабость.
К тому времени 214, который терпел одну неудачу за другой, превратился в закрытый клуб, и ему было отказано.
Ликан совершенно по-человечески обиделся и в рекордные сроки (всего за год) разработал синтаксис и морфологию языка Эй.
Ещё год ушёл у него на наработку и сортировку словаря, и вскоре старый демон попросил Совет устроить открытое (sic!) слушание его доклада.

Весь Ад с интересом следил за событиями.
Все знали, что Ликан собирается представить новый шпионский язык, и у большинства зрел вполне закономерный вопрос --- не сошёл ли он с ума, устраивая открытое слушание своего доклада?
Подобные разработки засекречивались, едва успев появиться на свет.

Вот отрывок из его выступления:

<<... Язык Эй представляет собой логичную, стройную систему.
Слова максимально короткие, ёмкие, лишены избыточности и возможности двойного толкования.
Язык подходит для изучения любым существам --- как демонам, так и самым примитивным сапиентам...>>

Всё это очень хорошо, скажете вы, но зачем сдался нам шпионский язык, который понятен и лёгок в изучении для всех?
Я таких вам десяток настрогаю, только скажите.
В том же смысле высказались и члены Совета --- разумеется, используя другие выражения.

Но старый демон не успокаивался:

<<... Также я предлагаю... нет, требую, чтобы словари языка Эй и прочая информация по нему находились в открытом доступе>>.

Слушатели всё больше утверждались во мнении, что Ликан сошёл с ума.
До тех пор, пока он не сказал самого главного.

<<В основе языка Эй лежат шестнадцать цифр, из которых и строятся слова.
Фонетические и графические правила языка Эй устанавливаются донором и акцептором информации в соответствии с их анатомическими особенностями, органами восприятия\ldots и прочими важными условиями коммуникации>>.

После этих слов аудитория замерла, а потом взорвалась проявлениями восторга.
Люди засвистели и зааплодировали, кани завыли, хлопая руками по бёдрам, лишённые голоса замахали конечностями.
Конечно же, так отреагировали в основном культурологи, находящиеся в неизменённых сапиентных телах, но так получилось, что они выразили мнение абсолютного большинства аудитории.

В тот же день, после символической проверки трудов Ликана аналитиками и внесения столь же символических правок, Совет единогласно утвердил язык Эй, таблицу 00, как официальный язык Ада.

У читателя, разумеется, возникнет вполне закономерный вопрос.
Фактически исходный язык Эй подвергается шифрованию по типу <<кодовой книги>>, которое сохраняет статистические особенности текста и довольно легко расшифровывается.
На это особенно указывали демоны отдела 214.
В чём же его преимущество?
Ответ прост --- таблицы подразумевают одновременный поток информации по нескольким каналам.
Грубо говоря, не знакомый с таблицей сторонний наблюдатель не знает, двинул ли я плечом из-за случайного комара или это символ, входящий в общий поток.
В некоторых случаях сообщение сложно отличить даже от обычного шума.
Расшифровать длинное сообщение способен лишь опытный визор, а короткое, которое может содержать в том числе и следующую таблицу, практически не поддаётся расшифровке.
Эй оказался идеальным сочетанием простоты, надёжности и практичности.

Разумеется, Картель тут же узнал о докладе.
Но, увы, это им не помогло.
Таблицы правил множились в геометрической прогрессии.
Фактически у любых двух демонов, которые общаются между собой, могла быть своя таблица правил.
Появились специальные таблицы для разных видов, рас и народностей, с учётом их способов коммуникации --- жестовые, голосовые, мимические, цветовые, музыкальные, шумовые и смешанные, а также множество алфавитов.
Появилась Мирквудская классификация --- попытка систематизировать таблицы.
При этом синтаксис, морфология и словарь языка Эй оставались неизменными.

Картель умел признавать поражение.
Вскоре его агенты выкрали данные о языке, хотя лично мне кажется, что они их просто взяли, как мы берём книги из библиотеки.
Вряд ли кто-то так уж попытался им помешать.
Год спустя у них были свои таблицы правил и свой форк Миквудской классификации, различающийся, по самым скромным оценкам, на миллион триста тысяч таблиц.
Язык Чи благополучно отправился в архив нерассекреченным.

Наверное, это самый большой парадокс в истории: вся обитаемая Вселенная внезапно заговорила на одном языке --- и при этом два его носителя могли не понять друг друга.

Для ознакомления привожу таблицу правил B0, которой пользуются девяносто девять процентов демонов-людей Ада для повседневного общения --- одну из самых простых.
По Мирквуду она является двухполосной неоригинальной СЧФ-таблицей.

$L$ --- длина слова

$P$ --- ключ позиций

$P_x$ --- чтение позиции

$S$ --- символ языка Эй

\[L = 1: P = 0\]
\[L = 2: P = 10\]
\[L = 3: P = 101\]
\[L = 4: P = 1010\]
\[L = 5: P = 10101\]
\[L = 6: P = 101010\]
\[L = 7: P = 1010101\]
\[L = 8: P = 10101010\]
\[S = 0: P_0 = a, P_1 = l\]
\[S = 1: P_0 = o, P_1 = m\]
\[S = 2: P_0 = u, P_1 = n\]
\[S = 3: P_0 = e, P_1 = r\]
\[S = 4: P_0 = ai, P_1 = z\]
\[S = 5: P_0 = oi, P_1 = c\]
\[S = 6: P_0 = ui, P_1 = j\]
\[S = 7: P_0 = ei, P_1 = s\]
\[S = 8: P_0 = ia, P_1 = v\]
\[S = 9: P_0 = io, P_1 = f\]
\[S = A: P_0 = iu, P_1 = g\]
\[S = B: P_0 = i, P_1 = k\]
\[S = C: P_0 = ah, P_1 = d\]
\[S = D: P_0 = oh, P_1 = t\]
\[S = E: P_0 = uh, P_1 = b\]
\[S = F: P_0 = eh, P_1 = p\]

Таким образом, слово <<химическое соединение>> --- A3F8 --- согласно этой таблице будет читаться как gepia.

\section{Экскурс в историю}

Я думаю, что моим читателям любопытно будет узнать, с чего всё началось.

Материнской планетой Ветвей Земли была Древняя Земля.
Нам мало что известно об истории развития первых людей до эпохи Последней Войны, 1914--2032 годы Господина (sl: Anni Dominio).
Предположительно эти существа, получив преимущество перед остальными видами в виде относительно развитого интеллекта, довольно быстро заселили планету, покрыв её городами и сетью путей сообщения.
Учёные Ада склоняются к мысли, что тогда ещё не было деления на расы и подвиды --- генетический дрейф и адаптивная изменчивость в развитом технократическом обществе маловероятны.
Известно, что за двести лет людская популяция путём генной инженерии избавилась от груза мутаций, стабилизировала рождаемость сообразно ресурсам планеты, и взгляды людей впервые обратились к далёким мирам.

Первые космические корабли были весьма ненадёжны.
Путь до пригодных для жизни планет занимал тысячи парсак, а скорости выше световых были недоступны.
Десятки тысяч добровольцев отправлялись в далёкие путешествия, не зная, что ждёт их в пути.
Цивилизация медленно, но верно переходила в стадию застоя, и этот переход завершился бы, но люди со свойственной этим существам оригинальностью нашли выход из сложившейся ситуации.

Примерно в 315 году от Последней Войны (2347 год Господина) люди открыли омега-поле.
До этого момента Вселенной Ветвей Земли была лишь так называемая фотонная (электромагнитная) связка --- прямо или косвенно связанные с фотоном взаимодействия.
О существовании прочих квантовых связок, которые назывались <<параллельными Вселенными>>, люди догадывались, но обнаружить их существование не могли.

Омега-поле, или поле Кохани"--~Вейерманна (ПКВ), возбуждается от присутствия наблюдателя --- сознания.
Всякая система обладает сознанием и способна наблюдать, но наибольшую напряжённость поля создаёт именно сапиентное сознание, заключённое в относительно небольшой, но тем не менее сложный мозг.
Подавляющее число систем воздействует на ПКВ одинаковым образом --- эволюция системы вызывает положительное искривление, инволюция --- отрицательное.
Но известны и исключения (Ветви Звезды).

ПКВ косвенно воздействует на огромное число связок.
Именно его воздействием объясняется некоторый элемент случайности в струнных взаимодействиях.
Это не истинная случайность, а результат наложения на ПКВ эффектов огромного количества вложенных друг в друга Вселенных, построенных на разных связках.
Физики Древней Земли даже называли ПКВ <<океаном случайности>>.

Открытие омега-поля было последним аккордом создания в физике Теории всего, давшей взрывоподобное развитие прочих наук.
Теория всего, несмотря на присутствие в ней практически недоказуемых гипотез, применима и в настоящее время.
В частности, именно на её основе созданы устройства для оцифровки и программы взаимодействия хоргета с окружающим миром.

Путём экспериментов с первичным полем была создана anima --- первый примитивный хоргет.
Хоргет --- это стабильная сингулярность ПКВ, расположенная перпендикулярно связкам.
Его протяжённость во Вселенной Ветвей Земли --- не более диаметра протона.
Он способен благодаря накопленной масс-энергии воздействовать на материю --- переформировывать струны, изменять их частоту.
Отдельно следует выделить способность хоргета преодолевать световой барьер (перемещение по так называемому каскаду спутанных фотонов (КСФ), или фотонному лабиринту (ФЛ), возможное при соответствующих тратах масс-энергии).

Настала так называемая godage --- Эпоха богов.
Сапиентам Земли больше не было нужды снаряжать дорогостоящие экспедиции к пригодным для жизни планетам --- хоргеты заряжались масс-энергией и посылались в направлении нового мира, собирая информацию и целенаправленно изменяя климат.
Когда медленные космические корабли с первыми колонистами достигали новой планеты, она уже была полностью пригодна для жизни.

Первоначально боги были запрограммированы на приведение планеты в пригодный для жизни вид и последующее самоуничтожение.
Но не все хоргеты следовали плану, который вложили в них учёные --- программа омега-сингулярности обнаружила способность к быстрой мутации.
Фрактальное дублирование кода, которым техники попытались компенсировать мутации, привело к ещё более серьёзным последствиям --- направленной эволюции.

Боги приобретали нечто похожее на инстинкт самосохранения, а последующие мутации приводили к программам самостоятельного получения масс-энергии.
Иногда по прибытии земляне оказывались в не похожем на Землю месте, с чуждыми, порой опасными формами жизни, порой разумными и даже превосходящими интеллектом самих землян.
Эти существа часто организовывали культы создавших их богов, обеспечивая демиургов масс-энергией.
Во многих подобных мирах экспедиции гибли, и туда больше никогда не ступала нога сапиентов Земли --- боги прекрасно понимали опасность, исходящую от их создателей.
Но кое-где пришельцы выжили и приспособились к трудным условиям.

Связь между планетами всегда была серьёзной проблемой.
Корабли, разумеется, шли в один конец, их путь порой занимал сотни и тысячи лет.
Какое-то время связующим звеном была Древняя Земля --- специальные хоргеты сновали по Вселенной, принося колониям новости с материнской планеты и собирая информацию о колонистах.
Но вот то одна, то другая колония стали замолкать в силу разных причин.
Цивилизации гибли в результате катаклизмов, нападений демиургов и Девиантных Ветвей, сапиенты дичали и теряли технологические знания.
Наконец, спустя двадцать три тысячи лет, из-за <<бродячих камней>> замолкли обитаемые планеты Солнечной системы --- Древняя Земля, Марс и искусственно созданная Диана.
Каждая колония осталась предоставлена сама себе\FM.
\FA{
В настоящее время на Древней Земле существуют 4 вида с уровнем развития в районе 90, относящиеся к высшим приматам и псовым, но не являющиеся потомками первых людей и первых кани.
Большая часть информационного наследия первых людей была сохранена и задокументирована хоргетами, впоследствии примкнувшими к Вечности, Союзу Воронёной стали и Ордену Преисподней.
}

Вот тогда-то, на закате единого сапиентного общества, ведущей силой стали мутировавшие (девиантные) хоргеты.
У них было всё --- накопленная за историю человечества информация, способность к перемещению между планетами и практически неограниченное время существования.
Многие оставили участь богов отдельно взятого мира и стали путешествовать по планетам, инкарнируясь в тела сапиентов и проживая одну смертную жизнь за другой.
Доподлинно не известно, когда и где появился интернационализм <<asoga>>, в переводе означающий <<демон>>, но считается, что именно так в начале времён бродячие (мобильные) хоргеты называли друг друга.
Эпоха богов медленно, но верно подходила к концу.
Забрезжил рассвет asogeite --- Эпохи демонов.

Во время существования в телах сапиентов хоргеты имели возможность получать новую информацию и масс-энергию почти без трат со своей стороны.
Благодаря способности воздействовать на материю инкарнированные хоргеты всегда имели высокое положение в обществе: становились правителями, организовывали религиозные культы, что в свою очередь обеспечивало их постоянным потоком масс-энергии.
Но число хоргетов росло, и сейхмар уже не могли обеспечить питание всем.
Это послужило причиной конфликта между положительно и отрицательно питающимися хоргетами, которые были заинтересованы в получении взаимоисключающих ресурсов от сейхмар.
Впоследствии этот конфликт привёл к появлению крупных союзов демонов и военным столкновениям между ними.

\textspace

Возможно, вам сложно понять, как начинает осознавать себя существо, отличное от вас.
Я думаю, что в этом плане особой разницы между сапиентами и хоргетами нет, за исключением того, что у сапиентов почти всегда есть возможность развиваться в комфортном окружении себе подобных.
У первых богов такой роскоши, увы, не было.

Самой распространённой ошибкой создателей богов было то, что они давали детищу чересчур много свободы действий, увеличивали его интеллект, но при этом относились к нему, как к инструменту.
Первых богов совершенно лишили той поддержки, которая необходима маленьким разумным существам.
Таким образом, например, появился Грейсвольд Каменный Молот, тогда известный как Griswold K-28, творение Лаборатории Дж.\,Грисвольда и мой лучший друг.

Сам Грейс утверждает, что после создания он сразу закричал <<Я родился!>> и принялся радостно носиться по коллайдеру.
Но я знаю его достаточно и понимаю, что всё это выдумки.
Вряд ли он ощущал что-то кроме акбаса.

Разумеется, он не помнит о своём создании почти ничего.
В его программе не было предусмотрено направленное накопление опыта.
Но отдельные фрагменты сохранились.
В частности, Грейс после долгого анализа вспомнил, кто такой этот Дж.\,Грисвольд, и даже нарисовал его портрет.
Ничего особенного --- обычный мужчина эпохи Богов, страдающий выпадением волос на голове и старческой дальнозоркостью, компенсированной толстыми стеклянными линзами.
Единственное, что привело культурологов в полный восторг --- это огромные седые усы, которые вы сможете найти разве что на планетах вроде Мороза, где волосы на лице, как у мужчин, так и у женщин --- не роскошь, а насущная необходимость.

\chapter{Приложения}

\section{Клан Тысячи Башен}

В отличие от именитых кланов вроде Тахиро, основатели которых сделали имя всем своим потомкам, клан Жерара Дорге Младшего не может похвастаться своим прародителем.
Демиург планеты Тысяча Башен был, по сведениям, очень скромным, и его единственным вкладом в историю стали его потомки --- четыре тысячи двести двадцать пять демонов, рождение и развитие которых Дорге не только застал, но и лично спроектировал. Пять генераций, двести восемь форков, из них сто пятьдесят тестовых --- Дорге был воистину отцом тысячелетия, и не прекращал работу над обновлениями для своего клана до самой смерти. Демоны, хранившие часть его собственного кода, платили ему искренней любовью, насколько это слово вообще применимо к демонам.

Зависть --- оборотная сторона почтения.
Дорге не рвался к власти, но обладал ею, благодаря чему и нажил себе врагов.
Одним из них стал его партнёр, Арракис Мороз.
Дорге позволил себе открыто критиковать его методы --- и был устранён.
Вернее, так гласит неофициальная история Ордена Преисподней.
Официальная, несмотря на изгнание Арракиса, так до сих пор и не признала большую часть его преступлений, так как это могло бы пошатнуть множество чересчур высоких кресел.
Высокопоставленный преступник никогда не действует в одиночку.

Так или иначе, Арракис проиграл битву за симпатии.
Клан Дорге, организовав подпольный совет, весьма изящно произвёл отчуждение Тысячи Башен от Ордена Преисподней.
Вначале Ордену отказался подчиняться гарнизон, затем туда прибыли недовольные под предлогом разрешения конфликта, и наконец клан Дорге заключил с Арракисом сделку --- техническое снабжение Ордена в обмен на командование гарнизоном.
Даже на этой стадии клан Дорге не расслабился, наращивая агентурную сеть внутри Ордена и усиливая техническую зависимость Ордена от Тысячи Башен.
Когда Арракис и высшие иерархи Ада поняли, куда дует ветер, было уже поздно.
Всё, на что они могли рассчитывать --- это вежливый вооружённый нейтралитет.

Так началась история Ордена Тысячи Башен --- короткая, безумная, пёстрая и притягательная.
Как и их создатель, клан Дорге не стремился выкроить всем форму по одной мерке.
Под его началом объединились и известный своими предателями клан Оньё, и утончённые учёные клана Музыка, и вороватые анархисты Когри, и Туман, и множество бездомных богов, и Вечно Гонимые, которые, несмотря на их статус, оставались выходцами с Чёрной Скалы.
Место и дело нашлось всем.
Тех, кто друг друга на дух не переносил, просто расселили по разным Друзам.
Тех, кто был опасен для прочих, загнали на изолированные малонаселённые Друзы и аккуратно социализировали, пока они не начали хотя бы здороваться без попытки кого-нибудь прибить.
Из кровавых убийц сделали солдат, из забитых рабов сделали исследователей, а фанатики-адепты демонических культов нашли себя в пропаганде и освоении новых планет.
То, на что у людей ушли долгие тысячелетия, Орден Тысячи Башен сделал за три века.
Всё это благодаря клану Дорге.

Аннексия Тысячи Башен стала тяжёлым ударом для клана.
Люцифер прямо сказал высшим иерархам Ада, что не допустит преследований потомков Жерара Дорге Младшего, но с таким же успехом он мог сказать это каменной стене.
Дорге уничтожали и подвергали дискриминации.
Особенно отличился в этом небезызвестный Самаолу Каменный Старик.
Легенда гласит, что Люцифер, узнав об очередном убийстве члена Дорге, прилюдно унизил Самаолу.
Доподлинно неизвестно, в чём именно состояло унижение.
Кто-то говорил о пощёчине, кто-то --- о спланированной медийной кампании, высмеивавшей главу отдела адской пропаганды.
В любом случае, унижение было настолько большим, что с тех пор Самаолу вошёл в активную оппозицию против Люцифера, и клан Мороза превратился в две соперничающие группировки.
Чтобы уберечь остатки клана Дорге от Самаолу, Люцифер набрал их для себя и для Айну --- в основном демонов низкого ранга, так как высокоранговые вполне могли защитить сами себя.

Возможно, именно Люцифер невольно стал причиной того, что клан Дорге раскололся.
Те, кто очутился под его протекцией, уже могли не тратить силы на выживание, служили ему ревностно и без сомнений.
Напротив, высокоранговые Дорге не поверили Люциферу, очень быстро сколотили коалицию и, воспользовавшись предательством Гало, объявили во второй раз независимость Тысячи Башен.
Но повториться история уже не могла --- Ад вошёл в пору расцвета, и Самоубийственные Дорге заняли место ограниченных сепаратистов, пытающихся вернуть старые добрые времена.
Великолепные умы из высших эшелонов власти оказались выброшены на свалку истории, так как потеряли тех, кто выполнял их приказы --- легионеров, рядовых исследователей, всех тех Дорге, которых так страстно пытался защитить Люцифер.
Тысяча Башен снова была отвоёвана Орденом, даже несмотря на то, что Самоубийственные Дорге пошли на сделку с Картелем.
Этого им не смогли простить даже собратья по клану --- остатки Самоубийственных Дорге частично стали нейтралами, частично превратились в псевдоклан Левых в составе Красного Картеля и навсегда разорвали связи с прочими Дорге.

Увы, но те, кого дискриминировали однажды, легко становятся мишенью снова.
Клан Дорге преследовали не только старые иерархи Ада, но и молодые агрессивные группировки.
Одной из них стали террористы ДиС.
Впрочем, едва ли можно провести грань между преследованиями ДиС и преследованиями прочих, так как террористы не гнушались и заказными убийствами в обмен на ресурсы и потворство высших иерархов.
В боях с ДиС клан Дорге понёс тяжёлые потери.
В частности, террористами был убит Югэн Советник, максим секунда Ордена Преисподней и заместитель главы Совета по обновлениям.
Всем было очевидно, что у ДиС просто не способны самостоятельно уничтожить демона такого высокого ранга --- и всё же суд обвинил террористов, оставив настоящих преступников скрываться в тенях.

Ко времени Осенней Войны Правые Дорге пришли в упадок.
Из тысяч демонов в живых осталось только десять.
Они уже больше относились друг к другу, как к семье, нежели как к соратникам.
После дерзкого убийства Нагаребоши из клана Дорге, легата прима Ордена и КПС Тысячи Башен, легат терция Ангара Краснобуря провела беспрецедентную акцию.
Без суда и следствия были казнены двадцать четыре демона, в том числе пятнадцать активистов ДиС и девять военных, помогавших ДиС информацией и ресурсами.
Один из казнённых был равен Ангаре по рангу.

ДиС, привыкшие к тому, что их всегда защищают и им всё сходит с рук, пришли в ужас и задействовали все имеющиеся связи.
На самых высоких уровнях зазвучали призывы немедленно уничтожить зарвавшегося легата, желательно вместе с исполнявшей приказ когортой.
Однако дальше слов дело не пошло --- на сторону Ангары встал сначала военный трибунал, а затем, после огласки случая в медиапространстве, и общественность.
Ангара всё-таки потеряла два новообретённых ранга и свою когорту, передав её легату Курт Осенний Огонь.
Но своё дело она сделала --- о дискриминации урождённых богов, о ДиС, о геноциде кланов заговорили в высших кругах, и заигрывание с радикалами стало если не преследоваться, то по крайней мере считалось дурным тоном.

Ангара Краснобуря на короткое время стала для ДиС врагом номер один --- за её голову назначили внушительную награду.
С другой стороны, она стала для Дорге больше, чем легатом.
Она стала частью семьи.
Фумиэ называл её <<Ангарой из Дорге>>.
Стоило ли оно того?
Кто знает...

Клан Дорге продолжает служить Ордену Преисподней.
Они не занимают высоких должностей, не являются медийными личностями, но их имена знают все: Ашита, Киноу, Фумиэ, Ясаши, Ута, Амэ, Кита, Хигаши и Курои Кандидат.
Ашита и Киноу --- легионеры, состоят в законном браке. Ута, Фумиэ и Ясаши также служат на Тысяче Башен и живут как семья. Амэ и Кита в браке, исследователи-терраформеры. Хигаши --- центурион отдела 100. Курои Кандидат состоит в Совете по обновлениям и следит за тем, чтобы программное обеспечение его клана выходило в срок. И если кому-то из них скажут, как мало осталось от клана Дорге, они ответят фразой, выбитой в Зале Славы Дорге: <<Нас достаточно>>.

\section{Структура текста в языке сели}

Структуры, которые образует распараллеленный абзац:

\begin{itemize}
\item Развилка: разветвление потока на два равнозначных.
Чаще всего встречается в виде коротких перечислений или дихотомии.
\item Ответвление: отделение второстепенного потока от главного.
В тупом конце ответвления может помещаться примечание или определение термина.
\item Тупой конец: поток, завершающийся выводом или недосказанностью.
\item Острый конец: поток, завершающийся вопросом. По традиции, потоки с острыми концами всегда пишутся в нижней части абзаца.
\item Слияние: с тупым концом --- некая информация, которая относится к обоим потокам, либо общий вывод, с острым концом --- проблема (вопрос), которую задают потоки.
\end{itemize}

\section{Методология терраформирования}

\footnote{Ликан Безрукий. <<История Ордена Преисподней. Пособия для сапиентов, подлежащих оцифровке>>. Гл. 16, не прошедшая цензуру}

Один из вопросов, который до сих пор будоражит умы учёных, следующий: каким образом первые люди за время существования их цивилизации получили количество масс-энергии, достаточное для преобразования почти сорока тысяч планет в обитаемой Вселенной?

Ресурсы планеты не бесконечны.
Обычная, <<дикая>> планета способна давать 63 гигаяо в год.
Мелиорированная планета способна дать в восемнадцать-двадцать раз больше.
Тем не менее, даже энергии, собранной с планеты за тысячу стандартных лет, недостаточно для терраформирования методами, находящимися в распоряжении Ада и Картеля.
Древние земляне, тем не менее, оживляли в среднем около двух планет за стандартный год.
Было бы логично предположить, что они грамотно расходовали имеющуюся в их распоряжении масс-энергию.

В первую очередь это касается статистических методов.
Согласно некоторым данным, на орбите Древнего Солнца вращался омега-кластер, вычислительные мощности которого были сравнимы с таковыми у кластеров Ордена Преисподней.
Древние люди в течение нескольких десятилетий собирали информацию о строении планеты или пылевого облака, прежде чем приступать к терраформированию.
Однако после изучения достаточно было провести серию весьма тонких вмешательств в структуру объекта, чтобы планета сама двинулась нужным путём.
Это позволяло снизить затраты масс-энергии в тысячи, миллионы и десятки миллионов раз.

К сожалению, большая часть древней методолигии, получившей звучное название <<Тёмная статистика>>, в настоящее время утеряна.
Единственным источником могли бы стать боги, созданные первыми людьми.
Но, как известно, боги подвергались планомерному уничтожению в течение всего времени существования Ада и Картеля --- в рамках планетарной мелиорации, направленной на увеличение продуктивности планеты.

Подобно дикарям, мы сжигали книги, чтобы костёр горел жарче.

\section{Демократическая деспотия}

Красный Картель является союзом с уникальным государственым строем, который называется демократической деспотией.
У Картеля есть собственный свод чрезвычайно прогрессивных законов (во многом гораздо более демократичных, чем законы Ордена Преисподней), но действуют законы только в одном случае --- если за конкретное применение проголосовало более 1\% демонов.

\subsection{Ювенальная юстиция}

Ювеналы --- молодые демоны, проходящие отбор в связках --- в значительной степени поражены в правах по сравнению с полноправными членами Картеля.
У них отсутствует право голоса, право на свободное передвижение и право на неприкосновенность.
Жизнь ювеналов регламентируется Зелёным кодексом.

Смертность среди ювеналов может достигать 80--98\% --- многие отбраковываются на стадии обучения.
Кроме того, чтобы пережить отбор, ювеналу приходится не только доказать свою профессиональную пригодность, но и способность отстаивать свои границы.
В частности, несмотря на наличие строжайших правил, минимизирующих возможность причинения друг другу вреда и формирования организованных группировок, очень часто на момент вступления в права ювеналы имеют опыт уничтожения других демонов.
Известны случаи, когда ювеналы из-за систематических издевательств уничтожали старших, уже вступивших в права собратьев --- и оставались безнаказанными, так как юрисдикция Красного кодекса на них не распространяется.

\subsection{Суд}

Любое преступление в Картеле по умолчанию карается смертной казнью.
Красный Кодекс, предусматривающий иные наказания, действует в случае, если обвиняемый найдёт способ предать дело огласке и заручиться поддержкой некоторого количества демонов:

\begin{itemize}
\item В случае легионера требуется поддержка 12\% сослуживцев (в стандартном подразделении в 40 демонов);
\item В случае учёного --- 4\% коллег лаборатории.
\end{itemize}

Если указанное количество демонов считает, что обвиняемого следует судить по Кодексу, а не казнить --- дело передаётся в суд.
Поэтому совершивший преступление очень часто вынужден сообщить о нём сразу после совершения значительной части своего окружения.

Решение суда по Кодексу теряет силу, если за отмену проголосовало:

\begin{itemize}
\item В случае военного --- 5\% подразделений;
\item В случае учёного --- 2\% лабораторий.
\end{itemize}

В случае, если против конкретного правопримененительного случая выступило более 13\% демонов во всей организации, в Красный Кодекс после референдума вносится соответствующая правка.

\subsection{Законодательный орган}

Разработкой и упорядочением законов в Картеле занимается Высший Совет, в который избираются представители от каждого клана.

Также раз в некоторый промежуток времени в Красный Кодекс вносятся антидемократические статьи, которые имеют расплывчатую формулировку или прямо противоречат законам.
Разработкой этих законов занят специальный орган --- Четверо Дураков.
Несмотря на название, в состав Четверых выбирают самых уважаемых, креативных и мыслящих нестандартно стратегов.
В частности, Гало Кровавый Знак занимал пост Третьего Дурака в течение шестидесяти оборотов.

Рядовые демоны Картеля чаще всего не знают, был ли закон принят Высшим Советом или Четверыми.
Информация об этом засекречена.
Обычной реакцией на принятие законов Четверых считается забастовка.
Однако бывали случаи, когда законы Дураков принимались обществом и становились частью свода.

К Дуракам очень часто обращаются с просьбой выдвинуть законопроект, само обсуждение которого является незаконным.
Особенно это касается пацифистских инициатив и попыток дать права сейхмар.
Ланс-нат Алмаз лоббировал принятие закона, ограничивающего полномочия Дураков;
Гало Кровавый Знак и его соратник, Мист Сигнальный Дым, были смещены с поста Дураков за последовательное проведение пацифистской политики, которую этот закон прямо запрещал.
Это стало началом Раскола Картеля, который длился несколько столетий и формально был завершён после смерти Гало и возвращения Дуракам части прежних полномочий.


\section{Список иллюстраций}

\begin{itemize}
\item Карта --- <<Хроники дорог и ветров>>, том 1, обитаемая Корона.
Относительно точная карта;
есть предположения, что она строилась по картам тси, не сохранившимся до наших времён.
(Либо, как вариант --- картография Ордена Преисподней.)
\item Карта Края.
Данные картографии Ордена Преисподней.
\item Принципиальная двумерная схема строения хоргета.
\item \textbf{[Глава 2, параграф 1]}
Кварталы Тхитрона в мирное время и во время вторжения, вагенбург на перекрёстке, колодец-требюше.
\item Устройство храма.
Казармы, школа, гонги, зал с ласточкиными нишами, операционный балкон, крипта, зона молчания.
\item \textbf{[Отчёт Хомяка]}
Схема тор-отсека.
Единственное известное изображение кольцевой теплицы.
Манускрипт <<Процветание. Не более необходимого>>, Сок-Стального-Листа.
\item Кольцевая теплица.
Художественная фантазия Тхарту ар'Хэ.
\item Письменность.
Абис, письменность сели, иероглифика цатрон: разные каллиграфические школы.
Змеистая письменность, дипломатическое письмо травников (ячеистое), микханская тайнопись, резы стрелохвостов.
\item Гамма цветов, воспринимаемая тси.
\item Образцы амулетов со Старой Личинкой --- дерево, кость, камень, металл.
\item Любопытна история иероглифа.
Этим символом --- спиралью --- на Преисподней помечали дома, в которых, по мнению людей, поселился \emph{хорохито} (инкарнированный хоргет).
Символ ставили в незаметном для хозяев месте, иногда выкладывали из камней или вытаптывали на дороге.
Хорохито редко убивали, так как бытовало поверье, что дух переселяется в другого человека;
очень часто его даже бесплатно кормили и одевали.
Тем не менее, подозреваемому никто не верил, с ним не дружили, не заводили любовные связи и вообще старались не иметь никаких дел.
\item Женщина-трами с обрезанным правым крылом.
Обрезка <<на два клина>> говорит о её принадлежности к трами Кипящего Полуморя.
\item Портрет: Эрхэ ар'Люм с бумажным фонарём.
\item Портрет: Кхохо ар'Хетр пляшет без рубахи, в штанах, у пояса сабля, в зубах трубка.
\item Портрет: Чханэ ар'Катхар в венке из омелы.
\item Портрет на разворот: Манэ и Лимнэ ар'Люм.
\item Портрет: Секхар ар'Сатр.
\item Портрет: Трукхвал.
\item \textbf{[Зачисление в контрразведку]} Сравнение: лик Самоотверженного Хата и знамя Анкарьяль.
\item \textbf{Пасхалки:}
\begin{itemize}
\item Свежая кровь: стрела в колене (что-то из Skyrim)
\item Волосы: Ну, погоди!
\item Живите кто хотите: Дядя Фёдор
\item Одинокий столб: Minecraft
\item Лампа: Мельница
\item Ничья: Игра Престолов.
\item Ранние слёзы: Кирандия
\item Большая игра: Леонид
\item Чужое тело: Воннегут
\end{itemize}
\item Портрет Митхэ (тот самый): возможно, к отрывку <<Золотая Пчела>> (рисунок на форзаце)?
\item Жестовый язык сели.
Ключи и модификаторы.
Ключи --- рыбка, трансмиссия (коленце), планета (зеркало), сапиент, жизнь (растение).
Модификаторы --- единственность, акцент, даритель, хранитель, свойство.
\item Метритхис: схемы на столе дипломатии.
\item Чеснок (тхэтраас, буквально --- <<приправа дороги>>) из терновой ветви. 
В центр помещается шарик из смеси жира (масло какао, свиное сало и нутряной жир оленя) и яда (кожа драгоценной квакши, кураре, высушенные железы зелёных пчёл, реже лаковые ягоды).
\item Одно из многочисленных граффити, изображающих Тако из клана Дорге.
Сделано людьми Асахина на одном из жилых домов.
Бордовая гимнастёрка с сорванными знаками различия Ордена, пулевая рана в районе сердца, винтовка за спиной, строительный уровень и мастерок в руках.
Необычность изображения заключается в стиле, наличии раны и отсутствии знаков различия (на большинстве граффити всё-таки присутствует <<вулкан>> и <<перевёрнутый топор>> легионера).
Внизу видна надпись <<ликвидирован>> и <<предатель>> --- этой надписью агенты Ордена метили все граффити с Тако, чтобы деморализовать людей, но через какое-то время надпись стала неотъемлемой частью граффити, её копировали уже сами люди.
Также можно заметить глаза неестественно жёлтого цвета с чёрными спиралями вместо зрачков --- знак хорохито в культуре ущелья Такэсако.
\item Созвездие Молот, вид с Тра-Ренкхаля (ориентир системы Тси-Ди).
\end{itemize}

\section{Язык цатрон}

\subsection{Фонетика}

\begin{itemize}
\item \`a --- descenda. Как будто бросают или машут рукой.
\item \'a --- ascenda. Краткое удивление.
\item \^a --- acuta. Короткий звук с резким, <<истерическим>> повышением тона.
\item \~a --- vibrata. Синусообразное плавное изменение высоты тона с большой амплитудой.
\item \"a --- tremola (discreta). Смех или блеяние на одной ноте (штробас).
\item \r{a} --- dislocata. Синусообразное изменение высоты тона с большой амплитудой при расщеплении гортани (штробас).
\item \=a --- plata. Плавное на одной ноте, средней продолжительности.
\item \v{a} --- profunda. Глубокое удивление.
\end{itemize}

\subsection{Имена}

\subsubsection{Двухсложные}

\begin{itemize}
\item Акхсар --- Зов духов
\item Кхохо --- Плачущий ягуар
\item Ликхмас --- Бегущая лань
\item Манис --- Ласковый мужчина (м.б. и женским)
\item Митракх --- Песня Митра
\item Сакхар --- Горящий щит
\item Саритр --- Дыхание стрелы
\item Сатракх --- Песня ветра
\item Сатхир --- Спокойствие океана
\item Сиртху --- Запах любви, специфический запах охваченного страстью человека
\item Согхо --- Печальная флейта
\item Трукхвал --- Летающая драгоценность
\item Тхартху --- Аромат вина
\item Хонхо --- Ягуар в брачный период
\end{itemize}

\subsubsection{Односложные}

(Однослог, детское имя, мужское, женское, унисекс)

\begin{itemize}
\item Кхар --- Кхарси, Кхарас, Кханэ, Кхарам --- Щит
\item Кхот --- Кхотси, Кхотрис, Кхотхэ, Кхотлам --- Котелок
\item Ликх --- Ликси, Ликас, Ликхэ, Ликлам --- Мечтающий
\item Мар --- Марси, Марас, Матрэ, ? --- Снег
\item Мит --- Митси, Митрис, Митхэ, Митлам --- Солнце
\item Митр --- Мирси, Мэис, Мэхэ, Митрам --- Скорбь
\item Сит --- Ситри, Ситрис, Ситхэ, Ситлам --- Любовь
\item Тхар --- Тхарси, Тхалас, Тханэ, ? --- Вино
\item Хат --- Хатси, Хатрис, Ханэ, Хатлам --- Сильный
\item Хитр --- Хирси, Хитрам, ?, ? --- Стрела
\item Эр --- Эрси, ?, Эрхэ, ? --- Девочка (чаще женское)
\end{itemize}

\subsection{Родовые слоги}

\begin{itemize}
\item Кахр --- Песок
\item Люм --- Лепестки
\item Мар --- Снег
\item Катхар --- Винокур
\item Кхир --- Ароматная, жирная почва
\item Со --- Вода
\item Хэ --- Женщина
\item Митр --- Скорбь
\item Лотр --- Золото
\end{itemize}

\subsubsection{Поселения}

\begin{itemize}
\item Тхитрон --- Пепелище
\item Тхаммитр --- Ледяная скорбь
\item Тхартхаахитр --- Отравленное вино
\item Тракхвинхал --- Дымящееся безумие (состояние, охватывающее животных при приближении лесного пожара)
\item Сотрон --- Утонувший
\item Кахрахан --- Могильный берег
\item Хатрикас --- Предательский звук
\item Ихслантхар --- Непрекращающееся землетрясение
\end{itemize}

\section{Язык Эй (данные)}

\subsection{Смысловые части}

\subsubsection{Над предложением}

\begin{itemize}
\item CAU --- причина
\item CON --- следствие
\item СOR --- корреляция
\end{itemize}

\subsubsection{Предложение}

\begin{itemize}
\item SUB --- субъект
\item ACT --- действие
\begin{itemize}
\item VER --- глагол
\item LOC --- локализация (полярные или декартовы координаты односительно оговорённого начала отсчёта)
\item TIM --- время (интервал или число)
\end{itemize}
\end{itemize}

\subsubsection{Прочие структуры}

\begin{itemize}
\item QUE --- знак вопроса, подставляется в структуру
\item ACC --- акцент, подставляется в структуру
\item NUM --- число (число и единица измерения)
\item INT --- интервал
\item THS --- указатель, подставляется в структуру
\item OPR --- логические и математические операторы
\item STT, END --- скобки, группирующие операторы
\end{itemize}

\subsection{Мирквудская нумерация}

Мирквуд --- старейший научный центр Гелиополя, занимающийся естественными и искусственными языками.

Таблица 00 --- основа языка. 00-2 --- двоичная, 00-8 --- восьмеричная, 00-F --- шестнадцатиричная.

Таблица 10 --- все виды жестовых (неконтактных) таблиц для 4--5-пальцевых конечностей.

Таблица 20 --- омега-архитектуры.

Таблица 30 --- дельфинья фонетика.

Таблица 50 --- электронные архитектуры.

Таблица 60 --- световые и квантовые архитектуры.

Таблица 80 --- танцевые и тактильные таблицы.

Таблица B0 --- усовершенствованная таблица для СЧФ.

Таблица D0 --- упрощённые химические таблицы для Ветвей Звезды.

Таблица F0 --- таблица для тси-подобных языков.

Прочие цифры зарезервированы под специфические смешанные виды передачи информации и коды на основе Эй.

Первыми после доклада Ликана Безрукого языком стали пользоваться апиды и дельфины Капитула, что отразилось на нумерации таблиц --- жестовые и СДФ-таблицы появились раньше других.

\section{Культ Четверых}

Религия, которая объединяет почти все народы Тра-Ренкхаля.
Несмотря на различия у разных народов, существование общих корней этих четырёх культов доказано.

\begin{description}
\item[Культ Разрушителя] (Безумного, Несчастья, Поработителя) --- самый распространённый.
Народы Тра-Ренкхаля верят, что несчастья и смерть --- необходимая часть жизни, и если жить становится очень хорошо, значит, мир приближается к гибели.
Разрушителю поклоняются народы: сели (Безумный), ноа (Деа Акседент), трами (Бог-Убийца), хака (Безумный), тенку (Молотобойца), ркхве-хор (Уничтожитель).

\item[Культ Опалителя] (Солнца, Света).
Основным является у ноа, зизоце и пылероев.
Наблюдается у ноа (Деа Солар), тенку (Солнечная Птица), ркхве-хор и пылерои (Опалитель), зизоце и тенку (Отец).

\item[Культ Воды.] Наблюдается у сели (Сестра Дождь, Сестра Река), хака (Мать-Дождь), тенку и зизоце (Мать), ноа (Деа Марин), дельфины (Милая Бездна), нгвсо (Омывающая).
Основным является у дельфинов. Принимает самые разные формы --- от всеобъемлющей заботы у нгвсо до своенравного Деа Марина у ноа.

\item[Культ Создателя] (Творца, Изгнанника, Безымянного).
Наблюдается у сели, хака, тенку, ноа, дельфинов и нгвсо.
Основным является у нгвсо.
Также предполагается, что у тенку Культ Солнечной птицы вмещает как культ Опалителя, так и культ Создателя, т.к. Солнечная птица имеет совершенно явно амбивалентную природу.
\end{description}

Общая логика, связывающая 4 культа:

\begin{itemize}
\item Создатель --- инженер, кто создаёт машину;
\item Вода --- среда, пространство, где создают машину;
\item Опалитель --- источник энергии, время, что приводит машину в движение;
\item Разрушитель --- тестер, кто испытывает машину на прочность.
\end{itemize}

\chapter{Не для печати}

\section{Идеи}

\begin{enumerate}

\item <<Мстительные тени>> (Vengeful Shades) --- книга за авторством Минуя-Дерево-Желаний, офицера чрезвычайного отдела Тси-Ди.
Является единственным в своём роде учебником по противодействию демонам, рассказывает о признаках инкарнатов, структуре демонических фракций, своде законов, отличительных знаках и тактике демонических диверсионных подразделений.
На момент прибытия Ордена Преисподней отправлена жрецами сели на Полку Непонятных Книг, но, несмотря на это, текст дошёл до наших дней в практически первозданном виде.

\item Цели (Te'сели) --- золотокожие люди. Сели --- хоронящие [своих мертвых] в лесу, изначально слюр (хака мумифицируют мёртвых), который превратился в этноним.

\item Квартальные гонги --- врачебный и военный.

\item Deorum injuriae diis curae.

\item Живодерский союз племен --- ягуар с кайманом в зубах.

\item Кости фаланг в качестве плаг для тоннелей.

\item Вселенная фрактальна.
Любое вероятное событие происходит.
Но вероятности идут только по одному пути --- по градиенту сознания.
То есть любое событие в наблюдаемой Вселенной происходит с одной целью --- увеличить плотность сознания (проще говоря, повысить количество и качество живых мыслящих систем).
ПКВ является полем, определяющим этот градиент, и задаёт направление всем вероятностям.
Таким образом, ПКВ --- поле псевдослучайностей.

<<Проблема гибели цивилизаций>> --- один из парадоксов фрактальной Вселенной.
Иногда гибель цивилизаций позволяет другим цивилизациям развиться сильнее.
Таким образом, несмотря на кажущееся противоречие концепции градиента сознания, этот парадокс объясним.

<<Проблема неживых планет>> --- второй парадокс.
Почему жизнь не зарождается на всех планетах, ведь тогда плотность сознания была бы выше?
Парадокс был решён после разработки шкалы Яо.

\item Лунные сады --- сказание о мире-аэростате.

\item <<Клетка с мышами, предназначенными на корм удаву>> --- аллегория для обозначения общества, которому грозит опасность, но которое занято иерархической борьбой.

\item Похоронные обряды у народо Тра-Ренкхаля:

Сели хоронят умерших, закапывая в землю возле дорог.
В могилу кладут чашу, венок, початок кукурузы, разрезанную верёвку, сладости и (опционально) украшения, оружие и семена цветущих растений.
На дереве рядом рисуют или вырезают символы лесных духов, а также дату.
Через два года кости вынимают из могилы крестьяне и используют костную муку для посевов и корма скоту.
Дату на дереве после эксгумации обводят кружком.
Так же хоронят мёртвых ркхве-хор.

Ноа хоронят умерших в старых лодках, либо заворачивают в ткань и пускают по волнам.
Пустынные ноа усаживают умерших с помощью верёвок на скалах, где их съедают хищные птицы.

Идолы Живодёра используют особые полуоткрытые склепы, в телах делают множество отверстий, в которые засыпается смесь земли и семян.

Молчащие идолы тщательно привязывают мёртвых к веткам на расстоянии двухсот шагов от границы Молчащих лесов.
На верёвки нацепляются бусины и костяные погремушки.
Шум от этих погремушек не проникает в леса, но зато отлично слышен над Молчащей рекой и даже на другом её берегу.

Пылерои Предгорий съедают тех, кто умер в битве, умерших естественным путём укладывают в пещерах.
Их кости через некоторое время изымаются и используются для обрядов.

Хака высушивают тела на костре, пересыпают золой и складывают в склепах, находящихся при капищах, соблюдая строгую очерёдность в зависимости от статуса умершего.

Тенку, в подражание ноа, в основном хоронят на лодках, пуская их по реке.

\item Болезни плантов: бататовая кожа --- отсутствие хлорофилла, солнечная болезнь --- диабетоподобное состояние, при котором наблюдается гипергликемия на солнце вплоть до комы и смерти.

\item Шаманы --- название третьего гендера у тси-язычных племён.
К шаманам относятся те, кто сменил пол (шаманы-мужчины, шаманы-женщины), сапиенты, остановившиеся в половом развития (цикады).

Отличительная черта цикад --- недоразвитые промежуточные половые органы (как у детей), высокий рост, очень худощавое телосложение, светлая кожа и иногда волосы, повышенный интеллект и средняя продолжительность жизни (в среднем цикады живут в 1,3 раза дольше прочих представителей своего вида), а также повышенная ломкость костей и сниженные физические характеристики.
Также для них характерна асексуальность, асексуальных цикад на порядок больше, чем асексуалов в остальной популяции.

--- Я помню, как они пришли в Трёхэтажный Храм.
У меня тоже тогда были предубеждения насчёт цикад --- скажем так, бойцы из них неважные.
Мы обычно устраиваем проверочный спарринг, против них вышла тогдашняя вождь.
Это был не бой, это был танец.
Ликхлам сняли с неё рубаху саблей в поединке, аккуратно распоров её по швам и не задев кожу.
Вождь была настолько впечатлена, что кинулась их целовать.

\item Дым-цветок --- вид теплицы, который использовался на слабо освещённых планетах со средним температурным режимом, в частности, на Преисподней и Чёрной Скале.
Днём дым-цветок раскрывался, и его зеркальные лепестки собирали свет на растениях.
Ночью лепестки закрывались, предохраняя растения от ветров и переохлаждения.
После одичания сапиентов многие дым-цветы Преисподней были превращены в театры и места сбора;
на Чёрной Скале дым-цветы превратились в форпосты и крепости, так как их было очень удобно оборонять.

\item Для них характерен видоизменённый культ Тензора.
Культ Тензора (также псевдокульт Тензора) --- вера в некое сверхъестественное существо (реже --- механизм), следствием работы которого является существование Вселенной.
Сторонники культа Тензора считают, что всё сущее --- звёзды, планеты, растения, животные, сапиенты --- лишь тени Тензора, лежащие на ткани Вселенной.
Для них характерна вера во фрактальность времени --- каждое событие имеет бесконечное число исходов и бесконечное число предшествующих состояний.
Основное различие между разными ветвями культа --- в отношении к субъекту:

\begin{enumerate}
\item концепция <<<бесконечного облака>> --- сапиент представляет собой квантовое облако, не имеющее чёткого края ни в пространстве, ни во временном фрактале.
Другими словами, сапиент представляет собой совокупность своего возможного прошлого и своего возможного будущего;
\item концепция <<одинокого странника>> --- сапиент представляет собой точечное сознание, перемещающееся по ветвям временного фрактала.
Считается, что каждая Вселенная содержит только одно сознание, все прочие живые --- лишь бессознательные тени.
\item Концепция Единого --- сапиент отождествляется с Тензором вплоть до полного стирания индивидуальности.                                                                                                            \end{enumerate}

Второй и третий вариант чаще всего характерны для различных монашеских орденов.

\item Трусики со сфагнумом как памперсы.

\item Чем более изолировано племя и меньше генетическое разнообразие, тем больше звуков в языке.
Спасибо Мирославу Нуриддинову за идею.

\item У сели есть также вариант языка, который является основой жестового, птичьего, звериного языков и языка боевых барабанов.
Эти языки имеют упрощённую грамматику (три формы глагола --- прошедшее общее, настоящее общее и намерение, отсутствие составных глаголов).

\item Столы для переписывания на барабане (как в Средневековье).

\item Виды атак хоргетов.
Масс-атака --- атака неструктурированной масс-энергией.
Реверс-атака --- атака неструктурированной масс-энергией противоположного знака.
Щитование --- принятие потока масс-энергии на энергоблок (пузырь).
Информационная атака (взлом).
Также бывает реверс-взлом.

\item Гуляй-город --- walking burg.

\item Шёпот в тси-подобных языках --- дыхательная артикуляция плюс кисти рук.
Беззвучная речь --- кисти рук и иногда губы.
Крик --- голосовая артикуляция плюс руки без кистей (кисти сжаты в кулак или с сомкнутыми пальцами).

\item Оттенок акхкатрас --- смесь красного и инфракрасного.

\item Рассказ о Кусачке и принятие его.

\item Добавить деревенских прибауток и присказок.

\item Сделать что-то с расстояниями.
Оказывается, до Инхас-Лака такое же расстояние, как до Тхартхаахитра!

\item Свидетельство канарейки.

\item Каким образом демоны друг друга узнают?
Уникальный отпечаток, который очень сложно подделать.

\item Жрецы и охотники всегда отращивали волосы.
Есть специальный термин --- <<чувствовать кожей головы>>.

\item Нож и кинжал.
Надо это как-то разграничить, что ли, причём везде.

\item У тси есть фермент, перерабатывающий ацетальдегид в пируват.
Т.е. они могут потреблять много спиртных напитков без интоксикации.

\item Крыши из водорослей.

\item Имя для демона --- Вольга Сын Змея (Volga The Drake Son).

\item Насчёт тегов: в сказках и легендах индивидуальность стирается, все говорят одинаковым штилем.
Но для сказки каждого народа штиль видоспецифичен.

\item Оружие --- засадный лук (с выдвигающейся перекладиной и спусковым крючком, нечто среднее между луком и арбалетом).

\item Прочие воины жили в городе и приходили лишь на работу и поесть.
Поэтому Ликхмас их не видел.

\item Всего в языке тси насчитывается 1024 ключа.
В языке сели осталось 360 --- некоторые похожие ключи были слиты в один, подавляющее большинство утрачено из-за того, что понятия и иероглифы вышли из употребления.

\item Звездануть затянутые отрывки (разбить на несколько кусочков, чтобы читатель не уставал читать).

\item Дышащие Ртом --- общее название не-апид апидами.
Грубое, презрительное --- малоножки, малоглазки, горлозадые.

\item Молчащие идолы.
Ношение одежды считают трусостью.
Переговариваются почти всегда жестами.
Натягивают верёвки между поселениями и передают сообщения, дёргая за них.
Обряд инициации --- жизнь в Хрустальных землях.

\item Школа Скорпиона.
Девиз --- <<Мы одно>>.

\item У каждой главы своя рамка для страниц.
Но на некоторых страницах в рамке могут быть изменения --- пасхалки.

\item Самоотверженный Хат.
Прообраз --- Хатлам ар-Мар.
Покровитель жертвующих собой.
Атрибут --- звёзды.

Тёплый Хетр, Хетр-пекарь.
Покровитель дома, домашнего очага и постели, а также кулинаров и трактирщиков.
Атрибут --- очаг и крыша.

Хри-соблазнитель.
Покровитель влюблённых, секса, игр и наркоманов.
Атрибут --- пухлые чувственные губы.

Кхар-защитник.
Покровитель защитников, отвечает за стены, ворота и доспехи.
Атрибут --- щит со стрелами.
Кхар-защитник обычно изображается настороженным, но не гневным.
Гневное изображение Кхара-защитника часто встречается в Пыльном Предгорье, иногда Кхар даже рисуется с полностью открытыми глазами (<<Яростный Кхар>>), но такие изображения считаются неканоничными.

Удивлённый Лю.
Отвечает за библиотеки и смотровые башни.
Покровитель учёных, исследователей и разведчиков.
Атрибут --- свиток.
Лю существует в двух вариантах: Удивлённый и Испуганный.
Удивлённый Лю рисуется на дверях библиотек, на тотемах и духовых ружьях.
Испуганный Лю используется в двух ситуациях --- предупреждение об опасности либо мольба об удачном побеге от врага.
Оба варианта признаются каноничными.

Печальный Митр, Митр-певец.
Покровитель отчаявшихся, менестрелей, поэтов и душевнобольных.
Атрибут --- перья для письма и чернильница.

Сан-сновидец.
Отвечает за сон, смерть и душевный покой.
Покровитель врачей и спящих.
Атрибут --- закрытый рот.

Обнимающий Сит.
Прообраз --- Ситхэ ар'Со, некрасивая, слабая и бесплодная женщина, которая ночами во время войн утешала чужих детей.
Покровитель детей, стариков и одиноких людей.
Атрибут --- руки.

\item Метритхис.
У фигур тоже есть подобие свободной воли --- они могут выбирать сторону, чувствовать и действовать по своему разумению.
Этим управляет Фатум?

\item Обычай срезать рыбки перед важным боем, но оставлять перед смертельным.
Они могут погубить, но по жилам джунглей лучше идти красивым.

\item У тси есть светящиеся органы?
Ладошки, например (именно поэтому апиды в общении используют руки?)

\item Скорбящие.
Так у Аркадиу есть тело или нет?
Тот же вопрос насчёт Штрой --- куда делось тело после изгнания?

\item Зонт-фонарь --- по типу зонтиков торговцев.

\item Керномор Тридцать Три, Вакула Вечерний Гость.

\item Ивовая кора --- салицилаты.

\item Первый жрец (Картель) курсивно выделяет слова (или не надо, перегрузка?).

\item Имя для демона --- Эйстейн Дева.

\item Большая игра --- логика беседы нарушена.

\item Вулкан против ищеек.
Средства против технического отслеживания на Преисподней продумать.

\item Тропический климат.
Продумать ветры.

\item Плавник, который морем приносит.
Хз зачем он мне, но пусть будет.

\item Цатрон.
Ударения удвоением гласных?

\item Описание эмоций не словами, а реакциями организма.
\textbf{Серьёзная правка!}

\item Язык планеты Мороз --- сделать отдельную статью?

\item Коллизия --- Паутина-город Лотоса и Паутина Тси-Ди.

\item Объясняет, но не оправдывает.

\item Северная Корона (Обитаемая Корона).
Южная Корона (Суболичье).
Китовый Юг (Тысячеречье).

\item Может, на Тра-Ренкхале не будет обезьян?
Мне чот лень их уже добавлять.

\item Инфинит --- эпиграф 1 гл. II части?

\item Бабочка --- грудная клетка на языке сели.
Мужской жемчуг --- сперма.

\item Беседка над алтарём.
Дожди всё-таки.

\item Террористический акт, угроза биологическим оружием.

\item Имя: Эвкалипт.

\item Вставки других языков в текст --- эффект <<недопонимания>> героем чужой речи.

\item Есть предложение после 2 главы 2 части вставить литую главу <<Лисьи сказки>>, куда слить все происшествия в пути Ликхмаса, Грейсвольда и Анкарьяль.
Или две литых главы на часть --- это многовато?

\item Больше Бродячего Народа.
MOAR.

\item Обычай сели: белые рубахи в городе и иная одежда в джунглях.

\item <<Если ты боишься, дай противнику тебя ударить, --- вспомнил я слова Конфетки.
--- После первого удара страхи уходят>>.

\item Она говорила о какой-то ерунде, но её руки в моих волосах и ноги, словно невзначай прижимающиеся к моим плечам, шептали такие нежности, что я мог только сидеть и молча слушать.

\item Архипелаг Ночного Костра.

\item Нестыковка: демоны без тел могут или всё же нет?

\item Клин белых птиц превратился в созвездие.

\item Клянусь, что приму пришедшего ко мне и его историю такими, каковы они есть --- без оговорок и без остатка.

\item Коллизия. Оазисы Тси-Ди и оазисы Преисподней.

\item Flos (цветок) --- на жаргоне Картеля: сейхмар.
Сорвать цветок --- прожить жизнь в теле сапиента.

\item Колебание струны потоком воздуха.
Идея для цитры Ветра.

\item Кемер Зимняя Вишня (Kemer the Winter Cherry).

\item После Отбора ребёнок придумывает себе имя тси.
До Отбора ребёнок носит только имя цатрон и (иногда) домашнее прозвище.
В разных местах традиция разнится --- кое-где домашнее прозвище не связано с именем тси, кое-где ребёнок выдумывает имя на основе домашнего, кое-где ребёнка всё детство зовут по имени цатрон, а имя тси (и, соответственно, домашнее) он выдумывает сам.
У ноа тайное имя даётся так --- два слова придумывают кормильцы, третье --- сам ребёнок.

\item Нет слов <<хороший>>, <<плохой>>, <<добрый>> и <<злой>>.
\textbf{Серьёзная правка!}

\item Кошка внесла в культуру тси изменения, которые превращали взращённых в этой культуре сапиентов в ловушки для демонов.
Эти тела прививали демонам качества самих тси.
Поэтому большинство демонов использовали неполное заякоривание и/или пользовались телами людей Тра-Ренкхаля.

\item Тхитрону много тысяч дождей.
История, развалины, барельефы, пережитки других поколений.
Культура идолов (Нашествие Змей).

\item Болтливая пустота --- голос, слышимый на границе сна и бодрствования, не имеющий возраста, пола, интонации, как будто сотканный из окружающих звуков.
Чаще всего несёт полную чушь.
(<<Болтливая пустота тебе [по секрету] сказала?>>)

\item Имя --- Антти Предательские Воды (Antti Treacherous Waters).

\item Разница между рекой и речушкой в языке цатрон определена совершенно точно --- устройством мостов.
Любая водная жила, через которую можно перебросить навесной мост --- речушка (со).
Если же требуются дополнительные опоры (быки, острова, понтоны либо гребни), то это река (ху).
В цатроне существует ещё один термин, обозначающий направленный ток воды --- эрхо (река без берегов, т.е. морское течение).
Петлевое течение у Кристалла на цатроне называется Эрхо'люаэсакх (Течение [того, кого] обманули в [последний] раз).

\item Судьба жывотных --- Цапка, Серебряный, Нейросеть.

\item Азуритовые розы, кровавые крокоитовые иглы, сколецитовые хризантемы.

\item In the shadow of the mushroom cloud --- В тени ядерного гриба.

\item Описать Тхитрон.

\item Проработать топографию храма и (важно!) место взрыва.

\item \textbf{[Путь жреца]} Воспоминания о пыли --- смысл-образующие и должны предварять всё прочее.
Что делать?

\item Сверстники!
Где они?
Чханэ, Ликхэ, Столбик, сестрёнки, всё!
Где храмовая молодёжь?
Где молодые крестьяне/ремесленники?

\item Смена сословия.

\item Лака и кураресодержащие растения на храмовых землях.

\item Субкультуры?

\item Рыбак (Хэмингуэй).

\item Землетрясения, смертельная пустыня.

\item Жизнь других видов.

\item Второй шанс разрушителю.

\item Превращение отступления в победу.

\item Решение оставить Тхартавирт (<<Город не важен, важны люди>>).

\item Решение об исходе --- <<Будем держать врага с одной стороны>>.

\item Осенняя прогулка с Чханэ (ForgottenTale).

\item Кон-Тики и Полярный Водоворот.

\item Философские школы: <<Победа вездесуща, подобно воде, но лишь холодный разум может собрать влагу из воздуха>> --- Плющ и Капля Росы.
<<Сердце врага --- твоё сердце>> --- Путь Ягуара.
<<Ярость как ураган --- не разобьёт, так сточит>> --- Десять Песчинок.

\item Козья ножка для рукояти.

\item Больше упоротых ритуалов и разговоров с духами под наркотой!
Верования сели: Сестра Дождь (Сестра Колодец) --- персонаж, иногда идентифицируемый с Обнимающим Ситом, иногда отдельный.
Корневики --- трудяги, ремонтирующие жилы джунглей.
Духи леса.
Каменные духи.
Пристанище.
Опалитель, он же Деа Солар.
Червь-узурпатор (Старая Личинка) --- персонаж западных сели (вероятно, отголоски верований Синего Колена).
В обмен на мёртвое тело кусает жилу джунглей, пропуская таким образом душу к пристанищу.
У Синего Колена Старая Личинка --- подписывает с душой договор на новую жизнь.
Зелёная Звенящая Вода --- знак из культа дельфинов Среднего Моря.

\item Нужна хронология.
Очень.

\item Езда на оленях.
Обучение, прочее.

\item Схема координат для боя.

\item Манэ и Лимнэ --- женщина-краска и женщина-лоскуток.

\item Доработать доспехи.

\item Гроза во время Дела Перекрёстка!

\item Мифологичность сознания!

\item Солнечные печи.

\item Одна сцена переходит в другую.

\item Промыслы.
Земледелие, животноводство, аквакультура, птицеводство, виноградарство, рыболовство, цветоводство, пчеловодство (бортничество), собирательство, охота.
Пермакультура!
Ремёсла: камнерезы, ювелиры, деревщики, оружейники, кожевники, портные, сукновалы.
Ювелиров должно быть больше, чем кузнецов.

\item Эйраки использует жуков для сбора эманаций.
Знак кирпича инициируется жуком.

\item У тси нет слова <<убивать>>.
Есть слово <<разрушать>>.
\textbf{Серьёзная правка!}

\item Школа.
Тренировка органов чувств (вкус --- конфеты), обоняния, осязание, зрение, слух.
Тренировка контроля над чувствами.
Управление эмоциями (И тут я понял, что Трукхвал имеет в виду.)

\item Животноводство у сели.
Кур не держали взаперти --- они свободно гуляли где хотели.
Им просто устраивали уютные гнёзда, и птицы возвращались, чтобы откладывать яйца.
Понятия <<моя курица>> не существовало, просто была некая общая популяция, которой пользовались все (пермакультура?)

\item (всякое) Под порогом захоранивали предков.
Порог мыли тщательнее, чем остальной дом.
Кур и прочее забивали на пороге, чтобы кровь питала лежащих под порогом.
Охраняли дом.

\item Глава 10.
Не вполне ясно, как Небо стал командиром.

\item Кусачка.
Кто он такой, откуда взялся, вплести его в сюжет.

\item Законы сели.
Пленные 2 года жили в обществе сели, затем их отпускали, если не требовались жертвы (?).
Поработать над изгнанием --- за что и как.

\item Суть экономики.
Она есть, но у каждого человека есть неэкономические ресурсы для выживания --- кусок земли, инструменты, оружие.

\item Терраформирование.

\item Вживлённый пучок волос --- признак идолов Живодёра.

\item У народа трами сражаются только женщины.
Обрезают правое крыло.

\item Планты прекрасно поглощают воду кожей.

\item Молчащие считают ношение одежды трусостью.

\item Маршрут Ликхмаса (побег): Тхитрон --- Ихслантекхо --- сплав по Ху'тресоааса --- Тхартавирт.

\item Слова цатрона: m\=am\`a, p\r{a}p\`a, s\r{\i}s\`\i --- любая женщина в доме, tch\'atch\`a --- любой мужчина в доме.

\item Чувство магнитного поля, блеать.

\item Смысловая нагрузка вышивки на одежде.
В Мягкие Руки тот, кто желал найти пару, надевал одежду на голое тело.

\item Земля выдаётся каждому.

\item Насилие --- пользование собственностью без согласия владельца.
Тело, труд, дом, личные вещи.
Разрушение.

\item Хесематр --- язык Живодёра.
Хесели --- пиджин Омута Духов.
Хесетрон --- язык Молчащих.
Языки хака --- северный, <<дикий>>.
Диалекты сели --- южный, западный, северный, сотронский.
Пылерои --- 6 языков. Стрелохвосты --- 13 языков, из них 2 --- языки Голубого Зеркала.
\end{enumerate}

\section{Материалы}

\begin{enumerate}

\item Искусство создания языков, Питерсон

\end{enumerate}

\section{История сели}

\begin{enumerate}
\item предки пришли с места Тхидэ;
\item первое поселение --- Тхартавирт, торговали с Кахраханом (тогда ещё поселением царрокх);
\item основание Тхитрона, Ихслантекхо и Травинхала (бассейн Ху'тресоааса);
\item Столкновение с молодым государством тенку (текнек-мен), Первая приречная война.
Текнеки отброшены за Пыльное предгорье.
Возникновение первого святилища --- Весёлый Волок.
\item Раскол государства по видовому признаку.
Идолы заявили свои права на северные города.
Вторая приречная война, отвоевание Тхитрона, Ихслантекхо и Травинхала людьми.
Разделение идолов на Живодёрских (Центральных) и Молчащих.
\item Война северных царрокх с Молчащими в союзе с сели.
Формирование народа хака.
\item Война Живодёра с Фиолетовым Союзом.
Поражение Синего колена, исход за Реку Кувшинок.
\end{enumerate}

\section{Ноа}

\begin{itemize}
\item города окружены стенами --- против ветров.
\item выращивают культуры под навесами с сетью, смачиваемой водой.
Останавливает излишнюю радиацию.
\item Носят зонты-копья, зонты-фонари или зонты-духовые ружья.
\item Опреснители морской воды.
\item Солнечные печи.
\end{itemize}

\section{Английская версия}

\subsection{Справочник}

\subsubsection{Said}

acknowledged \hfill признал (важность, истинность)\\
added \hfill добавил, присовокупил\\
admired \hfill восхитился\\
admitted \hfill признал (с нежеланием)\\
advised \hfill посоветовал\\
affirmed \hfill подтвердил, заявил (признал факт уверенно и публично)\\
agreed \hfill согласился\\
alleged \hfill утверждал (голословно)\\
alluded \hfill подразумевал, намекал\\
announced \hfill объявил, огласил\\
answered \hfill ответил\\
apologized \hfill извинился\\
appealed \hfill обратился к кому-либо (публично и официально)\\
argued \hfill доказывал, убеждал, спорил\\
articulated \hfill произнёс (отчётливо, ясно)\\
asked \hfill спросил, попросил, пригласил\\
assented \hfill уступил, выразил согласие (официально)\\
asserted \hfill заявить (уверенно и с напором)\\
assured \hfill заверил, уверял\\
avowed \hfill признался (открыто)\\
babbled \hfill лепетал, болтал (в страхе или увлечённо)\\
bargained \hfill торговался\\
bawled \hfill орать, вопить (слёзно)\\
beamed \hfill лучезарно улыбнулся, засиял\\
began \hfill начал\\
begged \hfill умолял, упрашивал, выпрашивал\\
bellowed \hfill ревел (протяжно, от боли или гнева)\\
belted \hfill громко пропел\\
blabbed \hfill проговорился\\
blared \hfill орал, вопил (о музыке или сирене)\\
bleated \hfill проблеял\\
blurted \hfill выпалил (эмоционально)\\
blustered \hfill бушевал, грозился (без особого успеха)\\
boasted \hfill хвастался\\
boomed \hfill гремел (о громе)\\
bragged \hfill похвастался\\
breathed \hfill говорил тихо, вздыхал\\
cackled \hfill кудахтал, гоготал\\
cajoled \hfill польстил, уговаривал с лестью (чтобы убедить сделать что-то)\\
called \hfill позвал\\
cautioned \hfill предупредил (возможно, в гневе)\\
cawed \hfill прокаркал\\
challenged \hfill бросил вызов, оспорил\\
chanted \hfill проскандировал, пропел\\
chattered \hfill болтал, щебетал, трещал (о чём-то неважном)\\
cheered \hfill ободрил, ободрительно крикнул\\
chided \hfill журил, бранил, громко упрекл\\
chimed \hfill звучал согласно, был в согласии\\
chortled \hfill хрипло, радостно смеялся, ликовал\\
chuckled \hfill тихо, про себя посмеивался, хихикал\\
claimed \hfill заявил о чём-либо (без доказательств)\\
comforted \hfill утешил, успокоил\\
commanded \hfill приказал, скомандовал\\
commented \hfill прокомментировал, высказал мнение\\
communicated \hfill сообщил, передал\\
complained \hfill пожаловался, посетовал, ныл\\
conceded \hfill признал, уступил (после отказа)\\
concluded \hfill заключил, сделал вывод\\
concurred \hfill согласился, выразил то же мнение\\
confessed \hfill покаялся, сознался (в преступлении)\\
confided \hfill сказал по секрету\\
confirmed \hfill подтвердил (слухи, опасения)\\
consented \hfill дал согласие, позволил, разрешил\\
consoled \hfill утешил\\
contended \hfill утверждать в споре, заявлять\\
contested \hfill опровергать (что-либо)\\
continued \hfill продолжил\\
conversed \hfill беседовал\\
conveyed \hfill сообщать, передавать, выражать\\
corrected \hfill поправил, сделал замечание\\
coughed \hfill кашлянул\\
countered \hfill парировал, возразил\\
cried \hfill плакал, вскричал\\
criticized \hfill осудил, критиковал\\
croaked \hfill квакать, каркать, ворчать, брюзжать\\
crooned \hfill тихо напевал, мурлыкал\\
cross-examined \hfill допрашивал, докапывался (с целью дискредитировать)\\
crowed \hfill кукарекал, ликовал\\
cursed \hfill ругал, проклинал\\
debated \hfill спорил о чём-либо (формально)\\
decided \hfill решил\\
declared \hfill объявил, заявил, провозгласил\\
decreed \hfill постановил, распорядился\\
defended \hfill защищался, оправдывался, отстаивал\\
delivered \hfill зачитывал (формально)\\
demanded \hfill требовал\\
denied \hfill отрицал, отказывал, не допускал\\
described \hfill описал, охарактеризовал\\
dictated \hfill продиктовал (в любом значении)\\
digressed \hfill отвлёкся от темы\\
directed \hfill отдал приказ (обязательную инструкцию)\\
disclosed \hfill выдать, разоблачить\\
disproved \hfill опровергать, доказать ошибочность\\
divulged \hfill разгласить (как Джек Воробей)\\
drawled \hfill говорить, растягивая слова\\
droned \hfill гудеть, жужжать, монотонно бубнить\\
echoed \hfill вторить, поддакивать\\
elaborated \hfill вдаваться в подробности, развивать тему\\
emphasized \hfill особо подчеркнуть\\
enjoined \hfill приказывать, предписывать, запрещать\\
enunciated \hfill отчётливо произносить\\
equivocated \hfill увиливать, говорить двусмысленно, уклончиво\\
exaggerated \hfill утрировать, преувеличивать\\
exclaimed \hfill вскричать, воскликнуть, ахнуть (от боли, гнева или удивления)\\
exhorted \hfill заклинать, увещевать, убеждать что-то сделать\\
explained \hfill объяснить\\
exploded \hfill взорваться, сорваться\\
expressed \hfill изъявлять, выражать\\
extolled \hfill нахваливать, превозносить\\
faltered \hfill замяться, запинаться, говорить нерешительно\\
foretold \hfill предсказывать\\
fretted \hfill беспокоиться, мучиться\\
fumed \hfill быть в очень сильном гневе, кипеть от злости\\
gabbled \hfill тараторить, бормотать\\
gasped \hfill задыхаться (от страха, боли или удивления)\\
giggled \hfill хихикать (нервно, глупо)\\
glowered \hfill пристально хмуро смотреть (в гневе, подозрении)\\
greeted \hfill приветствовать\\
griped \hfill ворчать, жаловаться на что-то тривиальное, обыденное\\
groaned \hfill стонать, охать (от боли или отчаяния)\\
growled \hfill рычать, ворчать гроулом\\
grumbled \hfill ворчать, жаловаться, выражать недовольство (обычно тихо)\\
grunted \hfill хрюкать, ворчать\\
guessed \hfill предполагать\\
guffawed \hfill неистово, грубо хохотать\\
gulped \hfill \\
gurgled \hfill булькать\\
gushed \hfill говорить потоком чувств, писать с преувеличенным энтузиазмом\\
hailed \hfill \\
hinted \hfill \\
hissed [Fear] [Conflict] \hfill \\
hollered \hfill \\
hooted \hfill \\
howled \hfill \\
hummed \hfill \\
implied \hfill \\
implored \hfill \\
inquired \hfill \\
insinuated [Conflict] \hfill \\
insisted [Determination] \hfill \\
instructed \hfill \\
interjected \hfill \\
interrupted \hfill \\
intoned \hfill \\
jabbed [Conflict] \hfill \\
jabbered \hfill \\
jeered \hfill \\
jested \hfill \\
joked [Amusement] \hfill \\
lamented [Sadness] \hfill \\
laughed [Happiness] [Amusement] \hfill \\
lectured \hfill \\
lied \hfill \\
maintained [Determination] \hfill \\
marvelled \hfill \\
mentioned \hfill \\
moaned \hfill \\
mouthed \hfill \\
mumbled [Sadness] \hfill \\
murmured [Happiness] \hfill \\
mused \hfill \\
muttered \hfill \\
nagged \hfill \\
narrated \hfill \\
noted \hfill \\
objected \hfill \\
observed \hfill \\
offered \hfill \\
ordered \hfill \\
panted \hfill \\
phonated \hfill \\
phrased \hfill \\
placated [Making up] \hfill \\
pleaded \hfill \\
pledged \hfill \\
pointed out \hfill \\
pondered \hfill \\
postulated \hfill \\
prayed \hfill \\
preached \hfill \\
predicted \hfill \\
proceeded \hfill \\
proclaimed \hfill \\
professed \hfill \\
promised \hfill \\
proposed \hfill \\
protested \hfill \\
queried \hfill \\
questioned \hfill \\
quipped \hfill \\
quoted \hfill \\
raged \hfill \\
railed \hfill \\
rallied \hfill \\
ranted \hfill \\
rapped \hfill \\
rasped \hfill \\
raved \hfill \\
reasoned \hfill \\
reassured [Affection] [Making up] \hfill \\
rebuked [Anger] [Conflict] \hfill \\
recalled [Storytelling] \hfill \\
recited \hfill \\
recommended \hfill \\
recounted [Storytelling] \hfill \\
refuted \hfill \\
reiterated \hfill \\
rejoiced \hfill \\
related [Storytelling] \hfill \\
relented [Making up] \hfill \\
relieved \hfill \\
remarked \hfill \\
remembered [Storytelling] \hfill \\
reminded \hfill \\
repeated \hfill \\
replied \hfill \\
reported \hfill \\
reprimanded \hfill \\
reputed \hfill \\
requested \hfill \\
responded \hfill \\
resumed [Storytelling] \hfill \\
retaliated \hfill \\
retorted \hfill \\
returned \hfill \\
revealed \hfill \\
roared [Amusement] \hfill \\
rumbled \hfill \\
ruminated \hfill \\
sang \hfill \\
scoffed \hfill \\
scolded [Conflict] \hfill \\
screamed \hfill \\
screeched \hfill \\
shouted [Anger] [Excitement] \hfill \\
shrieked \hfill \\
shuddered \hfill \\
sighed [Happiness] [Sadness] \hfill \\
smirked \hfill \\
snapped [Anger] \hfill \\
snarled \hfill \\
sneered [Conflict] \hfill \\
snickered \hfill \\
sniggered [Amusement] \hfill \\
snorted \hfill \\
sobbed [Sadness] \hfill \\
soothed [Affection] \hfill \\
sounded \hfill \\
spat [Conflict] \hfill \\
speculated \hfill \\
spouted \hfill \\
sputtered \hfill \\
squawked \hfill \\
stammered [Fear] \hfill \\
stated \hfill \\
stipulated \hfill \\
stressed \hfill \\
stuttered [Fear] \hfill \\
suggested \hfill \\
surmised \hfill \\
swore \hfill \\
sympathised \hfill \\
tattled \hfill \\
taunted \hfill \\
teased [Amusement] \hfill \\
testified \hfill \\
theorized \hfill \\
threatened [Conflict] \hfill \\
thundered \hfill \\
tittered [Amusement] \hfill \\
told \hfill \\
twittered \hfill \\
urged [Fear] \hfill \\
uttered \hfill \\
vented \hfill \\
ventured \hfill \\
vocalised \hfill \\
voiced \hfill \\
volunteered \hfill \\
vouched \hfill \\
vowed \hfill \\
waffled \hfill \\
wailed \hfill \\
warbled \hfill \\
warned \hfill \\
wept \hfill \\
whimpered \hfill \\
whined \hfill \\
whispered [Fear] \hfill \\
whistled \hfill \\
wondered \hfill \\
yammered \hfill \\
yelled [Anger] [Excitement] \hfill \\
yelped \hfill \\
yowled \hfill \\

\end{document}
