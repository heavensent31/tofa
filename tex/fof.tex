\documentclass[a4paper,10pt,fleqn]{book}
\usepackage{polyglossia} 
\setdefaultlanguage[babelshorthands=true]{russian}
\setotherlanguage{english}
\defaultfontfeatures{Ligatures=TeX,Mapping=tex-text}

\usepackage{xcolor}
\newcommand{\ml}[3]{#2}

% ----------------------

\usepackage{amsmath,amssymb,amsfonts,xltxtra,microtype,graphicx,textcomp}
\usepackage{svg}

% ------ GEOMETRY ------

\usepackage[twoside,left=2.5cm,right=3cm,top=3cm,bottom=4cm,bindingoffset=0cm]{geometry}

% ------ FONT ------

\setmainfont{Linux Libertine}
\definecolor{darkblue}{HTML}{003153}

% ------ HYPERLINKS ------

\usepackage{hyperref}
\hypersetup{colorlinks=true, linkcolor=darkblue, citecolor=darkblue, filecolor=darkblue, urlcolor=darkblue}

% ------ EPIGRAPH ------

\usepackage{epigraph}
\renewcommand{\epigraphsize}{\footnotesize}
\epigraphrule=0pt
\epigraphwidth=8cm

\usepackage{etoolbox}
\AtBeginEnvironment{quote}{\itshape}
\makeatletter
\newlength\episourceskip
\pretocmd{\@episource}{\em}{}{}
\apptocmd{\@episource}{\em}{}{}
\patchcmd{\epigraph}{\@episource{#1}\\}{\@episource{#1}\\[\episourceskip]}{}{}
\makeatother

% ------ METADATA ------

\newcommand{\tofaauthor}{\ml{$0$}{Эмиль~Весна}{Emil~Viesn\'{a}}}
\newcommand{\tofatitle}{\ml{$0$}{ПЛАМЯ~ОСЕНИ}{Flame~of~the~Fall}}
\newcommand{\tofastarted}{01.10.2021}

% ------ FANCY PAGE STYLE ------

\usepackage{fancyhdr}
\pagestyle{fancy}
\fancyhead[LE,RO]{\thepage}
\fancyhead[LO]{{\small\textsc{\tofatitle}}}
\fancyhead[RE]{{\small\textsc{\tofaauthor}}}
\fancyfoot{}
\fancypagestyle{plain}
{\fancyhead{}
\renewcommand{\headrulewidth}{0mm}
\fancyfoot{}}

% ------ NEW COMMANDS ------

\newcommand{\asterism}{\vspace{1em}{\centering\Large\bfseries$\ast~\ast~\ast$\par}\vspace{1em}}
\newcommand{\textspace}{\vspace{1em}{\centering\Large\bfseries<...>\par}\vspace{1em}}
\newcommand{\FM}{\footnotemark}
\newcommand{\FL}[2]{\footnotetext{См. \textit{\hyperlink{#1}{#2}}.}}
\newcommand{\FA}[1]{\footnotetext{#1 \emph{\ml{$0$}{---~Прим.~авт.}{---~Author.}}}}

\newcommand{\theterm}[3]{\textbf{\hypertarget{#1}{#2}} --- #3}
\newcommand{\thesynonim}[3]{\textbf{#2} --- см. \textit{\hyperlink{#1}{#3}}.}
\newcommand{\theorigin}[3]{\textit{#1:} #2 --- #3}


\begin{document}

% ------ TITLE PAGE ------

\begin{titlepage}
{\centering{~\par}\vspace{0.25\textheight}
{\LARGE\tofaauthor}\par
\vspace{1.0cm}\rule{17em}{1pt}\par\vspace{0.3cm}
{\Huge\textsc{\tofatitle}\par}
\vspace{0.3cm}\rule{17em}{2pt}\par\vspace{1.0cm}
{\Large\textit{\ml{$0$}{Фантастический~роман}{Science~fiction}}\par}
\vspace{0.5cm}\asterism\par\vspace{1.0cm}
{\textbf{\ml{$0$}{Начато:}{Started:}}~\tofastarted\par}\vfill
{\Large\ml{$0$}{Создано~в}{Created~by}~\XeLaTeX}\par}
\end{titlepage}

\tableofcontents

\chapter{Пламя Осени}

--- Башня Дьявола рухнула!
Вставай, девчонка, вставай!

С этих слов началось для Курц Штайгер утро, полное криков, топота и звона оружия.

Башней Дьявола издревле называли огромный кристаллический столп, росший на краю друзы Хербст.
Шли века и тысячелетия, эрозия подтачивала Башни, и они рушились одна за другой.
Некоторые падали неудачно, раздавливая целые города и деревни.
Некоторые падали удачно, на несколько сезонов превращаясь в каменный мост с одной Друзы на другую.
Чаще всего, разумеется, односторонний --- наклон и положение Башни редко позволяли одинаково хорошо идти вверх и вниз по ней.

О Башне Дьявола знали давно, к ней обращались мечты завоевателей и торговцев.
Мелкие люди, которые умирали, не дождавшись своего часа, а стареть начинали и того раньше, ждали и молили Всестроителя, чтобы Башня упала.
%{Overbuilder}
Но Башня Дьявола стояла.
По крайней мере, до сегодняшнего дня, когда её основание надломилось и исполинский кристалл встал строго горизонтально, соединив друзы Хербст и Гарда Викка.

Курц встала и протёрла глаз.
К чему вся эта беготня?
К Башне никто не подойдёт в ближайшие десять-пятнадцать дней.
Нужно время, чтобы она утряслась, плотно легла в своём каменном ложе, может быть, даже треснула напополам и отправилась в чёрную пучину Заалвира, разом похоронив все страхи и надежды людей.
%{Saalweird}
К чему эта беготня?

\asterism

Да, для неё процесс опорожнения выглядел так --- снять мехи, выдавить содержимое, обварить кипятком, затем обработать спиртом.
Достать новые мехи, промазать уплотнителем, аккуратно вставить, привычно поморщившись от боли.
И так каждый день.

Курц знала, что от нее всегда исходит неприятный специфический запах.
Раньше её это волновало --- она натиралась маслами, пытаясь отбить вонь;
сейчас ей было всё равно.

\asterism

Курц сняла маску, обнажив то, на месте чего должно было быть лицо.
Заросший глаз, стянутая кожа, кривое отверстие на месте рта, обнажённые носовые ходы...
Среди старых шрамов багровел ещё свежий, похожий на сколопендру.
Однажды женщине надоел не до конца закрывающийся рот, с уголка которого вечно капала слюна.
Местные врачи отказывались помогать, считая девушку безнадёжной.

--- Красивее ты от этого не станешь, --- напрямик сказала одна из них.

--- Я к тебе не за красотой пришла, идиотка, --- ответила Курц.

В конце концов она наткнулась на баронского полевого хирурга --- рассеянного, впадающего в деменцию старичка.
Тот отказался её оперировать, сославшись на трясущиеся руки, но подробно рассказал, как бы он это сделал.
Курц долго думала, примерялась, размышляла, а потом взяла бритву и исправила дефект.

Когда-то давно девчонка из Валленбергов --- местная красавица --- бросила ей в лицо:
<<Я бы лучше умерла, чем жила с этим!>>
Девчонка умерла при родах вместе с ребёнком.
Курц жива до сих пор.
И даже рот закрывается.

\asterism

Кленовые листья лежали на траве, словно язычки пламени.
Огнём горела и тут же сохла рябина, держа в высохших старческих руках горсти красных ягод.

\asterism

--- Умри, навозная глыба.

Движения Курц были выверены до мелочей.
Она отшатнулась, словно от неожиданности, пропустив вражеский клинок слева.
Имитация неуклюжего везения, вводящая противника в заблуждение.
Затем быстрым движением справа она перерезала врагу горло.

Налётчик взмахнул руками, словно пытаясь поймать и вернуть обратно фонтан крови, брызнувший из его шеи.
В его глазах застыло удивление.

<<На Хербст тебе не рады.
Я лишь вложила это чувство в движение клинка>>.

\asterism

Однажды она любила человека.
Молодой парень с жилистым телом и горячими руками краснел и смущался.
Он очень хотел узнать, что она прячет под маской.
Она показала ему.

Спустя год, когда Курц уже уверилась в том, что её приняли, он ушёл к другой --- молодой, красивой, по выражению Курц, <<у которой жопа на правильном месте>>.
Иногда она встречала его на окраинах города --- уже начавшего седеть, полненького, лысого мужчину с пушистыми усами.
Он не испытывал особых чувств к женщине, с которой жил, но очень любил своих детей --- забавно говорил с ними и таскал их на загривке.
Когда они случайно встречались, он смущённо улыбался и говорил <<Привет, Курц>>.
Курц кивала в ответ.

\asterism

--- Орден Преисподней --- просто террористы.

--- Да, для многих это будет выглядеть как хаос, --- кивнула Анкарьяль.
--- Отцы обратятся против детей, внучки пойдут против дедов.
Но это лишь ширма, внешняя оболочка.

--- А ты не думаешь, что то, как вещь выглядит, порой и есть её суть?

Анкарьяль на несколько долгих мгновений замолчала, думая, стоит ли сказать то, что вертится на языке.

--- Ты гораздо красивее внутри, чем снаружи, Курц, --- наконец проговорила она.

--- Но ведь внешность отражает мою суть, --- грустно возразила Курц.
--- Я привыкла быть уродливой, быть калекой.
Я сломана внутри.

--- Ты не...

--- Не перебивай.
Я сломана, это правда.
Даже если я шагну за пределы своих возможностей, даже если я сменю десять тел --- моё самое первое искалеченное тело останется со мной навсегда.
Внешнее всегда связано с тем, что внутри.

--- Я достаточно повидала войн, чтобы делать выводы.

--- Ты никогда не видела войну так, как видят её простые жители.
Ты видишь то, что тебе позволяют видеть командиры --- маску войны.
Ты бессознательно дорисовываешь войне лицо, и оно кажется тебе красивым.
Но те, кто никогда не обладал твоей властью и твоими знаниями, будут видеть войну такой, какая она есть, без маски, без дорисованной красоты.

--- Я не вижу другого пути, --- буркнула Анкарьяль.
--- Если ты считаешь меня террористкой --- я приму это как справедливую цену за правильный поступок.

--- Скажи, почему ты называешь себя человеческим именем?

Анкарьяль непонимающе посмотрела на подругу.

--- Я слышала, как твоя командир называла тебя Тальяной, --- пояснила Курц.
--- Но ты продолжаешь называть себя Ангарой.

Баронесса отвернулась.

--- Это личное.

--- Сколько времени тебе потребовалось, чтобы понять, что ты Ангара, а не Тальяна?

--- Нисколько.
Я просто услышала это имя и поняла, что оно принадлежит мне.

--- Тогда я уверена, Анкарьяль, что однажды ты меня поймёшь.

--- Почему?

--- Потому что все самые важные изменения происходят без войны.
Ты просто понимаешь, что новый порядок вещей --- единственно правильный для тебя.


\end{document}
