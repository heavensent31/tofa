
\author{Эмиль Весна}
\title{Пламя осени}
\date{01.10.2021}
\maketitle

\tableofcontents

--- Башня Дьявола рухнула!
Вставай, девчонка!

С этих слов началось для Курц Штайгер утро, полное криков, топота и звона оружия.

Башней Дьявола издревле называли огромный кристаллический столп, росший на краю друзы Хербст.
Шли века и тысячелетия, эрозия подтачивала Башни, и они рушились одна за другой.
Некоторые падали неудачно, раздавливая целые города и деревни.
Некоторые падали удачно, на несколько сезонов превращаясь в каменный мост с одной Друзы на другую.
Чаще всего, разумеется, односторонний --- наклон и положение Башни редко позволяли одинаково хорошо идти вверх и вниз по ней.

О Башне Дьявола знали давно, к ней обращались мечты завоевателей и торговцев.
Мелкие люди, которые умирали, не дождавшись своего часа, а стареть начинали и того раньше, ждали и молили Всестроителя, чтобы Башня упала.
Но Башня Дьявола стояла.
По крайней мере, до сегодняшнего дня, пока её основание не надломилось и исполинский кристалл не встал строго горизонтально, соединив друзы Хербст и Гарда Викка.

Курц встала и протёрла глаза.
К чему вся эта беготня?
К Башне никто не подойдёт в ближайшие десять-пятнадцать дней.
Нужно время, чтобы она утряслась, плотно легла в своём каменном ложе, может быть, даже треснула напополам и отправилась в чёрную пучину Заалвира, разом похоронив все страхи и надежды людей.
К чему эта беготня?
