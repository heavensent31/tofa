% \documentclass[a4paper,12pt,fleqn]{book}\usepackage{polyglossia}\setdefaultlanguage[babelshorthands=true]{russian}\setotherlanguage{english}\defaultfontfeatures{Ligatures=TeX,Mapping=tex-text}\usepackage{xcolor}\newcommand{\ml}[3]{#2}

\documentclass[a4paper,12pt,fleqn]{book}\usepackage{cooltooltips}\usepackage{polyglossia}\setdefaultlanguage[babelshorthands=true]{russian}\setotherlanguage{english}\defaultfontfeatures{Ligatures=TeX,Mapping=tex-text} \usepackage{xcolor}\definecolor{lightgray}{HTML}{bbbbbb}\color{lightgray}\newcommand{\ml}[3]{\textenglish{\textcolor{black}{#3}}}

% ----------------------

\usepackage{amsmath,amssymb,amsfonts,xltxtra,microtype,graphicx,textcomp}
\usepackage{svg}

% ------ GEOMETRY ------

\usepackage[twoside,left=2.5cm,right=3cm,top=3cm,bottom=4cm,bindingoffset=0cm]{geometry}

% ------ FONT ------

\usepackage{ebgaramond}
\definecolor{darkblue}{HTML}{003153}

% ------ HYPERLINKS ------

\usepackage{hyperref}
\hypersetup{colorlinks=true, linkcolor=darkblue, citecolor=darkblue, filecolor=darkblue, urlcolor=darkblue}

% ------ EPIGRAPH ------

\usepackage{epigraph}
\renewcommand{\epigraphsize}{\footnotesize}
\epigraphrule=0pt
\epigraphwidth=8cm

\usepackage{etoolbox}
\AtBeginEnvironment{quote}{\itshape}
\makeatletter
\newlength\episourceskip
\pretocmd{\@episource}{\em}{}{}
\apptocmd{\@episource}{\em}{}{}
\patchcmd{\epigraph}{\@episource{#1}\\}{\@episource{#1}\\[\episourceskip]}{}{}
\makeatother

% ------ METADATA ------

\newcommand{\tofaauthor}{\ml{$0$}{Эмиль~Весна}{Emil~Viesn\'{a}}}
\newcommand{\tofatitle}{\ml{$0$}{ПОДЗЕМЬЕ}{The~Underground}}
\newcommand{\tofastarted}{15.08.2022}

% ------ FANCY PAGE STYLE ------

\usepackage{fancyhdr}
\pagestyle{fancy}
\fancyhead[LE,RO]{\thepage}
\fancyhead[LO]{{\small\textsc{\tofatitle}}}
\fancyhead[RE]{{\small\textsc{\tofaauthor}}}
\fancyfoot{}
\setlength{\headheight}{15pt}
\fancypagestyle{plain}
{\fancyhead{}
\renewcommand{\headrulewidth}{0mm}
\fancyfoot{}}

% ------ NEW COMMANDS ------

\newcommand{\asterism}{\vspace{1em}{\centering\Large\bfseries$\ast~\ast~\ast$\par}\vspace{1em}}
\newcommand{\textspace}{\vspace{1em}{\centering\Large\bfseries<...>\par}\vspace{1em}}
\newcommand{\FM}{\footnotemark}
\newcommand{\FL}[2]{\footnotetext{См. \textit{\hyperlink{#1}{#2}}.}}
\newcommand{\FA}[1]{\footnotetext{#1 \emph{\ml{$0$}{---~Прим.~авт.}{---~Author.}}}}

\newcommand{\theterm}[3]{\textbf{\hypertarget{#1}{#2}} --- #3}
\newcommand{\thesynonim}[3]{\textbf{#2} --- см. \textit{\hyperlink{#1}{#3}}.}
\newcommand{\theorigin}[3]{\textit{#1:} #2 --- #3}

\begin{document}

% ------ TITLE PAGE ------

\begin{titlepage}
{\centering{~\par}\vspace{0.25\textheight}
{\LARGE\tofaauthor}\par
\vspace{1.0cm}\rule{17em}{1pt}\par\vspace{0.3cm}
{\Huge\textsc{\tofatitle}\par}
\vspace{0.3cm}\rule{17em}{2pt}\par\vspace{1.0cm}
{\Large\textit{\ml{$0$}{Повесть}{Novella}}\par}
\vspace{0.5cm}\asterism\par\vspace{1.0cm}
{\textbf{\ml{$0$}{Начато:}{Started:}}~\tofastarted\par}\vfill
{\Large\ml{$0$}{Создано~в}{Created~by}~\XeLaTeX}\par}
\end{titlepage}

\tableofcontents

\epigraph
{Проблема Подземья на Капитуле была и остаётся.
Но могу вас уверить --- опасности оно не представляет.
И этого демона на Капитуле нет и не будет.}
{Самаолу Каменный Старик на вопрос об Уэсиба Серозмее}

--- Слушай, дружище, если ты ищешь новый смысл жизни --- ты не по адресу.
Здесь его нет.

--- Для начала мне нужно просто, чтобы меня не нашли, --- ответил Митракх.
--- Не нашли --- это значит, не нашёл никто: ни Ад, ни Картель...

--- ...ни Скорбящие, --- закончил Кор, --- потому что они не любят предателей так же, как и все остальные.

--- Предатель --- это тот, кто сделал выбор, который боялись сделать остальные, --- сказал Стенающий. --- Они сами это говорили.

--- Для них хороши лишь предатели, сделавшие выбор в их пользу.
Вот увидишь --- едва Скорбящие станут конкурентами Ада и Картеля, они будут выкорчёвывать предательство не хуже этих отбросов из клана Антрацис.
Это судьба любой военной организации, даже если её идеолог --- гениальный пацифист Лу.

--- И так как Лу отнюдь не глуп, он это понимает, но никогда не скажет этого своим фанатикам.

--- Ложь правит Вселенной, дружище.

--- Поэтому я и пришёл в Подземье.

--- Хороший выбор.
Но, сдаётся мне, пользы от тебя мало.

--- Я кое-что знаю и могу быть полезен.

--- Ты тси, верно?

--- Был когда-то.

--- Это неважно.
Спросил из пустого любопытства --- вашего брата нечасто встречаешь.
Тем более в демоническом обличье.

--- Ну так что?

--- Мне твои умения ни к чему.
Ставлю миллион, что они вообще здесь ни к чему.
Кто ты?
Учёный, вояка, визор?

--- Меня собирали как технолога.

--- Здесь всем плевать.
Унитаз мы и сами можем починить.

--- В таком случае чем я могу быть полезен?

--- Мне нужен бармен.

--- Прости, кто?

--- Прощаю.
Это тот, который разливает напитки и развлекает обдолбанных посетителей, слушая их бессвязный бред.
Как твоё имя тси?

--- Бездонное-Стенающее-Пятно.
Можно просто Стенающий. % Wail

--- Ну и бред.
Без обид.
Будешь Глыбью. % Dypth

--- Я уже прошёл имянаречение, и...

--- ...и всем плевать.
Ты хочешь на меня работать или нет, Глыбь?

--- Хорошо.

\section{Кровавая Дыра}

Стенающий любил Кровавую Дыру.
Несмотря на название, это был сравнительно уютный район Тридцать четвёртого уровня.
Магазинчики, чистящие оружие жители с татуировками местных банд, множество граффити.
Всё это почему-то до боли напоминало мирный цветущий Миситр...

<<Ордену --- орденское, всё остальное --- нам>>

<<Р.К.Е.М.>> (Режу Конкурента Ещё Младенцем)

<<Думаешь скрыться в новом теле?>>

<<Втяни усы, Улитка>>

<<Убивай, насилуй, подчиняй>>

<<Ну почти>>, --- закончил про себя Стенающий, увидев последнее граффити.
Разумеется, в самой Дыре дальше надписей дело не шло.
Здесь такое количество автоматических боевых устройств и орудия у жителей, что в карман залезть нельзя без опасности быть превентивно пристреленным.

Стенающий кивнул группе у крохотной транспортной мастерской.
У всех пятерых были татуировки <<Улиток Перехода Двух Труб>> и массивные самодельные гауссганы.
Ему кивнули в ответ --- соседей здесь принято приветствовать.
Но в этот раз с ним решили ещё и заговорить.

--- Ты из Микродатчиков, брат?

Стенающий тут же изящным, отточенным движением сплёл пальцы в сложный жест.
Улитки ухмыльнулись.

--- Да не баись, мы знаем.
Пошли подбросим, мы на Перемычку.

Отказываться было нельзя.
Это не простая вежливость.
В Кровавой Дыре не существовало простой вежливости, как не существовало и многих других обыденных вещей.

Вскоре из мастерской вылетел транспорт.
Старый, залатанный, покрытый стикерами и надписями гравиколь закашлял, а затем медленно и задумчиво лёг на пузо.

--- Да что такое, --- проворчал водитель.

--- Братан, ты нас прогулять хочешь или что? --- поинтересовался один из Улиток.

--- Тут какая-то шняга, --- пояснил водитель и обратился к Стенающему.
--- Слухай, брат, открой панель сбоку.

Стенающий ухмыльнулся.
Старый как мир приём.
Панелей было несколько, и Улитки просто проверяли его технологические навыки.

--- Что такое <<панель>>? --- поинтересовался он самым невинным голосом.

Улитки разочарованно крякнули.

--- Лезь на моё место, --- сказал один из них Стенающему.
--- Щас всё сделаю...

Ни о какой серьёзной поломке речи не шло.
Судя по звуку, в магнит просто нагнало чуть больше заряда, чем позволяла температура контура.
Выкрутить соленоид --- заземлить --- зафиксировать.
Не прошло и михнет, как гравиколь уже бодро парил по разрисованному переулку к Развязке.

\asterism

--- Бармен, значит? --- хмыкнул водитель.
--- Неплохо, неплохо.
А сам-то ты чей?

--- Винодел с Ватермилла, --- пожал плечами Стенающий.
Этой легенды посоветовал придерживаться Кор.

--- Кто по масти?

--- Гнедой.

--- Не демон, не нейтрал, не бог?

--- Родился, поиграл, подох, --- отозвался Стенающий.
Типичная рифмовка сапиентных банд Подземья.

Улитки ухмыльнулись.
Гравиколь притормозил у <<Грязной кальки>>.

--- Заходи, если с транспортом беды, --- водитель кивнул на прощание.
Стенающий выпрыгнул из транспорта и показал Улиткам жест <<буду должен>>.

\section{Эквилибриум}

--- В среде сапиентов естественное локальное равновесие, Глыбь, --- Кор выразительно покачал узкими ладонями.
--- Те, кто пострадал от рук федералов, идут за справедливостью в банду.
Те, кто пострадал от банды, пополняют ряды федералов.
И там, и там они всё больше убеждаются в своей правоте, пока не начинают творить с другими то, что сломало жизнь им.
Я --- другое дело.
Я не местный и не имею эмоциональной привязки к ситуации.
Поэтому я правлю.

Стенающий уже начал уставать от его болтовни.
<<Кем ты правишь, пустобай?
Пылью, песком и девятью ветрами?..>>

\asterism

Телохранители тоже не вмешивались.
Они застыли столбами по бокам от своего босса.
Кор, кривясь, наблюдал и ждал, пока драка закончится.

Вскоре она закончилась.
Дебоширы лежали на полу клуба без движения.
Силач Фервайкт по-дружески кивнул Стенающему и начал растаскивать дебоширов по углам.

--- Глыбь, это работа вышибалы, --- недовольно подал голос Кор.
--- У нас есть вышибала --- Фервайкт.
\ml{$0$}
{Ты вышибала, Глыбь?}
{Are you a muscle, Dypth?''}

\ml{$0$}
{--- Нет.}
{``No.''}

\ml{$0$}
{--- Нет?}
{``No?''}

--- Нет, босс, --- поправился Стенающий.

--- Тогда через какие обоссанные кусты ты сюда полез?

\ml{$0$}
{--- Они на меня напали, босс.}
{``They assaulted me, boss.''}

--- Так надо было бежать, Глыбь! --- Кор обвёл руками зал.
--- Смотри на остальных барменов --- они бежали.
Почему ты не бежал?

--- Прошу прощения, босс, --- угрюмо сказал Стенающий.

--- Последний раз, --- кивнул Кор.
--- Иди на своё рабочее место.

\asterism

--- Тобой уже заинтересовались, --- процедил Кор сквозь зубы.
--- Подходили сапиенты, которые спрашивали, откуда у бармена такие навыки физической интерфекции.
Спрошу и я.

--- Я из тси.

--- И?

--- Этого недостаточно, босс?

--- Вопросы здесь задаю я, Глыбь.
Ты на них отвечаешь, отвечаешь прямо и недвусмысленно, без этих ваших дикарских иносказаний.
И если ты этого не усвоишь прямо сейчас, наше сотрудничество можно считать оконченным.
Итак, ещё раз: откуда у тебя такие боевые навыки?

--- Это стандартное обучение в обществе народа сели, --- ответил Стенающий.
--- Мы все проходим через Храм и учимся владеть телом.

--- И твоё обучение у Скорбящих не имеет к этому отношения?

--- Я выжил на Могильном берегу, значит, моих навыков уже для этого хватало.

--- Я не могу понять, ты тупой или тебе шкура не дорога?

--- Простите, босс?

--- Видимо, тупой, --- махнул Кор.
--- Иди на своё рабочее место.
Завтра возьмёшь оплачиваемый выходной.

--- Спасибо, босс.
В честь чего?

--- Просто так, в качестве благодарности.
Ситуацию ты разрулил, и Фервайкт тебя похвалил.
Ты ему понравился, сечёшь?

--- Да, босс.

--- Иди, Глыбь.

--- Босс.

\asterism

--- А ты красивый, --- сказал киборг, обнажив белые, явно сделанные под заказ, зубы.
--- Давно к Дядюшке ходишь?

--- Давно, --- кивнул Стенающий.

--- Ну не до его рождения же ты начал ходить? --- удивился канин с косами.
--- Дядюшка О, конечно, бессмертный, но ты на бессмертного не похож.

--- Бессмертный? --- переспросил Стенающий.

--- Да.
Он демон.
Умирает, потом рождается снова, и снова к плитке.

--- А мы стережём его лавочку, пока он не родится снова! --- вмешалась беззубая и толстая мохнатая бабулька --- видимо, Род Медведя.
--- Я стерегла с ружжом, бандюков отгоняла!
Мать моя сказала --- вернется Дядюшка.
И он вернулся!

--- И давно он так?

--- Кто знает? --- пожал плечами канин, потеребив косы.
--- Историков в Перешейке нет, следить некому.
Если знаешь демонов --- поспрашивай их.
Они поболе нашего живут.

\asterism

Дядюшка О протянул Стенающему пакет.
Тот обычной улыбкой поблагодарил Дядюшку и только потом нащупал на пакете бугорки.
Язык Эй, обфусцированная таблица E4.

<<Зайди после закрытия>>.

\asterism

--- Мой добрый старый клиент, --- Дядюшка похлопал себя по толстому животу.
--- Как здоровье?

--- На твоих обедах, Дядюшка?
Здоров как ягуар.

--- Как ягуар, говоришь.
Грозные кошки.
Да только шкуры их чересчур ценны.

--- Моя шкура ценна только мне, Дядюшка, --- хихикнул Стенающий, кладя мужчине руку на плечо.

<<Ты хотел меня видеть?>> --- забарабанили пальцы.
Таблица 2F с обфусцированным инициальным блоком публичного ключа.

--- Ошибаешься, внучок, --- Дядюшка грузно повернулся и звякнул грязными сковородками.
Палец пробежался по столу: <<Важный разговор>>.
Фрагмент ключа.
--- Я тут подумал... кхм.
Да.
Не хочет ли, кхм, такой ценитель поработать в моём, ммм, заведении?

Тоны.
Интервалы.
Покашливания.
Публичный ключ получен.
Стенающий захихикал, завершив шифрованное соединение.

--- Я бы с удовольствием, кормилец, да я уже при работе.
Бар <<Грязная Калька>> знаешь?

--- Слыхал.
И кто ты там?

--- Бармен.

--- В повары тебе надо, --- затряс щеками Дядюшка О.
--- В повары.
Вот смотри, как я это делаю...

Спустя час Стенающий вышел от Дядюшки --- сытый, уставший и узнавший кучу неприятных для себя вещей.
Едва подойдя к дому, он убедился в истинности этих вещей.
Среди бандитских граффити краснела свежая надпись, корявые иероглифы тси, явно полученные от грубого автоматического переводчика:

\ml{$0$}
{<<Оглядывайся, цель>>.}
{\textsc{Watch your back, target.}}


\asterism

Стенающий бросил на пол баллончик.

--- Ааа, ДиС, --- Кор небрежно пнул баллончик носком ботинка.
--- Знаю таких.

--- Кто они?

--- Тебе вкратце или подлиннее?

--- Как можно короче.

--- Охотники на урождённых богов.

--- Я же не бог!

--- Я тебя предупреждал, чтобы ты...

--- Ладно, ладно, --- Стенающий поднял руки.
--- Суть я уловил, босс.
Я накосячил, и меня начали травить.
Не так важно, кто они.
Как я могу исправиться?

--- Пока ты в баре --- ты под моей защитой, Глыбь.
Для меня это вопрос репутации.

--- Мне нужно знать о них побольше, чтобы, выйдя из бара, я мог хотя бы дойти до дома.

--- Судя по твоему описанию, это какая-то радикальная группа, --- Кор щёлкнул зажигалкой и раскурил сигару.
--- На богов охотиться у них остроты не хватает, а вот на всякую мелочь из урождённых сапиентов --- вполне.

--- Твои ребята не хотят ими заняться?

--- Мне они не мешают.

--- А могут?

--- Глыбь, --- Кор выпустил огромный клуб дыма, --- не наглей.

--- Босс, ты же не хочешь лишиться бармена?

--- Нет, Глыбь.

--- Тогда расскажи, пожалуйста, с кем я имею дело.

--- Я дам тебе совет, малыш: попробуй договориться с теми, кто их крышует.
Радикалы --- всегда мелочь, которая плавает под крупными китами.
И вот этим крупным китам ты можешь быть полезен.
Адрес я тебе дам.

--- У тебя и адрес есть.
Интересненько.

--- На что ты намекаешь?

--- Это ты продал меня ДиС, Кор.

Телохранители подобрались.
Кор, напротив, расслабленно откинулся в кресле.

--- И почему это я?

--- Твои телохранители вряд ли бы оценили, что ты разбрасываешься персоналом направо и налево.
Ещё меньше они бы оценили, если бы ты сказал им выпутываться из проблем самим.
Вывод простой --- я уже не часть банды, и деньги за меня ты получил.

--- Превосходно.
Не такой уж ты и тупой, Глыбь.

Кор потянулся за ещё одной сигарой.

--- И сколько ты за меня получил?

--- Ты не поверишь --- я отхватил боевое устройство федералов.
Парни уже им занялись.

--- <<Радикальная мелочь>>, у которой есть боевые устройства федералов, --- Стенающего начала пронимать улыбка.
Телохранители смотрели на эту улыбку непонимающе и настороженно, их боевые модули <<вспыхивали>> и <<гасли>>, выдавая чрезвычайную нервозность;
Кор, напротив, расслабился ещё больше.

--- Не свежак с конвейера, Глыбь.
Не обольщайся насчёт своей значимости.
Сломанное устройство, конечно же.
Списанное, предназначенное для утилизации.
Но ты их заинтересовал.
Как я уже говорил, вашего брата нечасто встретишь.

--- Они хотят меня изучить?

--- Господи Боже, Глыбь, ты не в Мирквуде, ты в сраном Подземье!
Никто никого здесь не изучает, это вопрос репутации.
Одно дело --- подрезать сраного божка, и совсем другое --- первую цифру-тси, которая имела неосторожность сунуться в Подземье.

--- Мог бы и подержать меня, как редкий экспонат, --- Стенающий уже скалился во все зубы.

--- Босс, --- тихо сказал телохранитель справа.
Его демон тихо <<пульсировал>>.

--- Подожди, дорогой, мы разговариваем.
Глыбь, кажется, я забыл тебе сказать: я не коллекционер, я бизнесмен.
ДиС по сути такие же партнёры моей банды, как и...

Стенающий ударил.
Ударил жёстко, сочетая омега-атаку и физическое разрушение мозговых структур.
Два телохранителя упали на пол.
Кор обмяк и выронил сигару.

--- Кажется, я забыл тебе сказать, что Скорбящие собрали меня как интерфектора, --- задумчиво сказал Стенающий.
--- Какая жалость.

Он поднял с пола баллончик и написал на стене над телами:

\ml{$0$}
{Д.и.С.}
{\textsc{D.o.D.}}

<<Разбирайтесь сами теперь>>.

\asterism

Это он, вне всяких сомнений.
Сигнатура совпадает в точности.
Стенающего поразили глаза старика --- добрые и лукавые, окружённые тонкими морщинками.

--- Так зачем ты меня искал?

Стенающий промолчал.
Он и сам не мог понять, зачем он так страстно пытался найти Уэсиба.

--- Не знаешь, --- кивнул старик.
--- Это хороший знак.

--- Почему хороший? --- удивился Стенающий.

--- Все, кто ищет меня, преследуют какую-то цель.
Выслужиться, обрести знание, самоутвердиться.
Почти никто не ищет меня просто так.
Меня привлекают такие, хоть и нечасто я таких вижу.

--- Как ты узнал?

--- Я сижу тут каждый день и прошу милостыню.
Разговоры до меня долетают.

--- Хочешь, я посижу с тобой? --- вдруг предложил Стенающий.
--- Я умею петь и играть на флейте.
Ты заработаешь больше.

Старик улыбнулся и кивнул.

--- Только не вздумай петь песни с Тра-Ренкхаля, --- предупредил он.
--- А не то тебя быстро вычислят.

--- Моя сигнатура уже у ДиС и у банды Кора, --- грустно сказал Стенающий.

--- Нет, --- коротко ответил Уэсиба.
Стенающий не стал расспрашивать.

\asterism

Уровни ниже двадцатого на Перешейке Скаге --- оживлённое место.
Здесь кипит жизнь.
Стенающий с флейтой быстро набрал печатей на еду и одежду.

--- Держи, --- Стенающий протянул Уэсиба пакет с Дорожным Обедом Дядюшки О.
--- Это вкусно, я у них уже третий раз беру.
Правда, к Дядюшке такая очередь, что вставать надо с утра, когда он даже печку не разогрел, и предзаказ он принципиально делать не хочет --- видимо, чтобы федералы не обложили налогами...
Курить будешь?
Конопля хороша, у Колодезного Люка купил.

Уэсиба оценил Дорожный Обед Дядюшки О и коноплю.

--- Это генномодифицированная, --- сказал он, выпуская колечки дыма.
--- Скорее всего, контрабанда из биобанков Наверху.
Обычное местное зелье не настолько крепкое.
Между делом, ты не особо светись с флейтой, а то кто-нибудь тебя закрышует...

--- С сейхмар проблем точно не будет.

--- Я не про сейхмар, Стенающий.
Здесь и демонических банд хватает...
Кстати, раз уж мы так хорошо сегодня покушали --- не хочешь прогуляться?

\asterism

--- Я не знаю.
Тогда, на Могильном Берегу, я думал, что сражаюсь за правое дело.
Но нас просто вели на забой, как скот...
Потом, вступив в ряды Скорбящих, я думал, что смогу положить этому конец.
Но они меня обманули.
Скорбящие --- такие же детали системы.
Они не способны положить конец ничему.

--- Никто не способен, Стенающий.
Более того --- я вообще не знаю, зачем трогать систему.

--- Тебя называют великим бунтарём, легендой.

--- И всё моё бунтарство заключается лишь в том, что я не желаю играть в их игры.
Не желаю быть фигурой на поле боя, не желаю быть деталью в науке, не желаю служить, отстаивать и верить.
Все мои умения служат лишь тому, чтобы быть вне системы, выстроенной демонами.

--- А чего ты хочешь, Уэсиба?

--- Жить хочу, Стенающий.
Ты когда-нибудь жил?

Стенающий кивнул.

--- Я гулял по джунглям.
А ещё по Миситру, когда только занималась заря и солнце подсвечивало Винтовой храм.
В эти моменты я жил, а не служил кому-то.

--- Тогда ты меня понимаешь.
Расскажи мне про Миситр.
Я хочу знать, как выглядит Винтовой храм...

\asterism

--- Может быть, я загляну на Тра-Ренкхаль, --- задумчиво сказал Уэсиба.
--- Выглядит очень привлекательно.
Этот Безымянный... бог с фантазией на биологию, отдаю ему должное.
Тра-Ренкхаль ещё не мелиорировали?

--- Нет.
Стараниями Скорбящих планету объявили заповедником и трогать её биосферу категорически воспрещено.
Тси, разумеется, частью биосферы не считаются, и их трогать можно.

--- Ничего себе.
Это вообще впервые в истории такое.
Мои знакомые боги умерли бы от зависти...
Не знаешь, где сейчас Безымянный?

--- На Тси-Ди, следит за планетой вместе с местным компьютерным разумом.
Официально Скорбящие его используют как мелиоратора, но после изгнания с Тра-Ренкхаля он, скажем так, немного в стороне от их дел.
На Тси-Ди ему дали полную свободу, и он играет с планетой как хочет --- завозит какие-то истреблённые виды с других планет, создаёт свои, разворачивает биомы, подпиливает и шлифует биосферу...
Тси-Ди сейчас очень напоминает Капитул, так же много видов сапиентов и зверья, но гораздо красочнее и меньше урбанистики.
Стрекозодраконы летают, соволюди...

--- Стрекозодраконы?
Это случайно не печально известные сюзерены с Драконьей Пустоши, которых истребили по приказу максима Смока?

--- Они самые.
Безымянный их развёл по просьбе Шакала.
Была целая операция, чтобы стащить ДНК из биобанка Ордена, и сейчас они телепатически связаны с Безымянным, как ранее были связаны с Кох.
Оказывается, связь с богом для них жизненно необходима, и изменить это пока не смогли...
Грозные твари.

--- Вот, значит, как.
Тоже своего рода нейтрал, этот Безымянный.
Думаю, он на своём месте.

--- А давно ты заделался попрошайкой?

--- В этой жизни.
В прошлой был монахом.
В позапрошлой, дай земля памяти, попробовал себя в коневодстве.
Был ещё и геологом, пастухом, каллиграфом, танцором, секс-работником, предсказателем, зубным врачом.
Ещё ни разу не повторился, представь?

--- А если профессии закончатся?

--- Тогда пойду по второму кругу.

\asterism

--- Ух, какая холодная вода, --- Стенающий быстро выбежал из моря и завернулся в полотенце.

Уэсиба хихикнул и похлопал товарища по спине.

--- Приятно, да?
Есть что-то в этом, когда холод раздражает твои рецепторы, когда напрягаются мышцы волос и когда становится пупырчатой?
А капли на коже?
Это потрясающе.

--- И мокрая рубаха, --- поддержал Стенающий.
--- У меня на планете дожди были стеной.
Дышать надо осторожно, чтобы не нахлебаться...
А по залитым дорогам плавали испуганные и ничего не понимающие змеи.

Уэсиба захохотал.

--- Никогда не видел ошарашенных змей.
Наверное, это забавно.

--- Забавно ровно до того момента, пока одна из них тебя не цапнет.

Стенающий вдруг замолчал.

--- Слушай, Уэсиба, я вдруг подумал...
Вот мы с тобой живём... вот так.
Ничего не производим, ничего не делаем полезного для общества.
Но что будет, если все так будут жить?

--- А ты не думай за всех, --- крякнул Уэсиба.
--- Думай за себя.
Все остальные пусть думают о чём хотят.

--- А расскажи про свою планету.

--- Про какую из?

Стенающий смутился.
Он по-прежнему считал <<своей>> лишь одну, ту самую...

--- Про какую хочешь.

\asterism

--- Я вдруг понял, чем я хочу заниматься.

--- Значит, пришло нам время расстаться.

--- Тебе не грустно?

--- Нет.
Я очень за тебя рад.

--- Мне грустно.

--- Если ты реально нашёл себя --- ты переживёшь эту грусть, Стенающий, --- Уэсиба похлопал товарища по плечу и кивнул в знак прощания.

...Переходя пути, Стенающий обернулся.
Старик-попрошайка сидел на прежнем месте и дремал, склонив седую голову.

\end{document}
