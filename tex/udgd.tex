\epigraph
{Проблема Подземья на Капитуле была и остаётся.
Но могу вас уверить --- опасности оно не представляет.
И этого демона на Капитуле нет и не будет.}
{Самаолу Каменный Старик на вопрос об Уэсиба Серозмее}

--- Слушай, дружище, если ты ищешь новый смысл жизни --- ты не по адресу.
Здесь его нет.

--- Для начала мне нужно просто, чтобы меня не нашли, --- ответил Митракх.
--- Не нашли --- это значит, не нашёл никто: ни Ад, ни Картель...

--- ...ни Скорбящие, --- закончил Кор, --- потому что они не любят предателей так же, как и все остальные.

Митракх кивнул.

--- Я кое-что знаю и могу быть полезен.

--- Ты тси, верно?

--- Был когда-то.

--- Это неважно.
Спросил из пустого любопытства --- вашего брата нечасто встречаешь.
Тем более в демоническом обличье.

--- Ну так что?

--- Мне твои умения ни к чему.
Ставлю миллион, что они вообще здесь ни к чему.
Кто ты?
Учёный, вояка, визор?

--- Меня собирали как технолога.

--- Здесь всем плевать.
Унитаз мы и сами можем починить.

--- В таком случае чем я могу быть полезен?

--- Мне нужен бармен.

--- Прости, кто?

--- Прощаю.
Это тот, который разливает напитки и развлекает обдолбанных посетителей, слушая их бессвязный бред.
Как твоё имя тси?

--- Бездонное-Стенающее-Пятно.
Можно просто Стенающий. % Wail

--- Ну и бред.
Без обид.
Будешь Глыбью. % Dypth

--- Я уже прошёл имянаречение, и...

--- ...и всем плевать.
Ты хочешь на меня работать или нет, Глыбь?

--- Хорошо.

\razd

--- То есть ты продал меня ДиС.

--- Это бизнес, Глыбь.
ДиС по сути такие же партнёры моей банды, как и...

Стенающий ударил.
Ударил жёстко, сочетая омега-атаку и физическое разрушение мозговых структур.
Кор и два его телохранителя упали на пол.

--- Кажется, я забыл тебе сказать, что Скорбящие собрали меня как интерфектора, --- задумчиво сказал Стенающий.
--- Какая жалость.

Он поднял с пола баллончик и написал на стене над телами:

Д.И.С.

<<Разбирайтесь сами теперь>>.

\razd

Это он, вне всяких сомнений.
Сигнатура совпадает в точности.
Стенающего поразили глаза старика --- добрые и лукавые, окружённые тонкими морщинками.

--- Так зачем ты меня искал?

Стенающий промолчал.
Он и сам не мог понять, зачем он так страстно пытался найти Уэсиба.

--- Не знаешь, --- кивнул старик.
--- Это хороший знак.

--- Почему хороший? --- удивился Стенающий.

--- Все, кто ищет меня, преследуют какую-то цель.
Выслужиться, обрести знание, самоутвердиться.
Почти никто не ищет меня просто так.
Меня привлекают такие, хоть и нечасто я таких вижу.

--- Как ты узнал?

--- Я сижу тут каждый день и прошу милостыню.
Разговоры до меня долетают.

--- Хочешь, я посижу с тобой? --- вдруг предложил Стенающий.
--- Я умею петь и играть на флейте.
Ты заработаешь больше.

Старик улыбнулся и кивнул.

--- Только не вздумай петь песни с Тра-Ренкхаля, --- предупредил он.
--- А не то тебя быстро вычислят.

--- Моя сигнатура уже у ДиС и у банды Кора, --- грустно сказал Стенающий.

--- Нет, --- коротко ответил Уэсиба.
Стенающий не стал расспрашивать.

\razd

Уровни ниже двадцатого на Перешейке Скаге --- оживлённое место.
Здесь кипит жизнь.
Стенающий с флейтой быстро набрал печатей на еду и одежду.

--- Держи, --- Стенающий протянул Уэсиба пакет с Дорожным Обедом Дядюшки О.
--- Это вкусно, я у них уже третий раз беру.
Правда, к Дядюшке такая очередь, что вставать надо с утра, когда он даже печку не разогрел, и предзаказ он принципиально делать не хочет --- видимо, чтобы федералы не обложили налогами...
Курить будешь?
Конопля хороша, у Колодезного Люка купил.

Уэсиба оценил Дорожный Обед Дядюшки О и коноплю.

--- Это генномодифицированная, --- сказал он, выпуская колечки дыма.
--- Скорее всего, контрабанда из биобанков Наверху.
Обычное местное зелье не настолько крепкое.
Между делом, ты не особо светись с флейтой, а то кто-нибудь тебя закрышует...

--- С сейхмар проблем точно не будет.

--- Я не про сейхмар, Стенающий.
Здесь и демонических банд хватает...
Кстати, раз уж мы так хорошо сегодня покушали --- не хочешь прогуляться?

\razd

--- Я не знаю.
Тогда, на Могильном Берегу, я думал, что сражаюсь за правое дело.
Но нас просто вели на забой, как скот...
Потом, вступив в ряды Скорбящих, я думал, что смогу положить этому конец.
Но они меня обманули.
Скорбящие --- такие же детали системы.
Они не способны положить конец ничему.

--- Никто не способен, Стенающий.
Более того --- я вообще не знаю, зачем трогать систему.

--- Тебя называют великим бунтарём, легендой.

--- И всё моё бунтарство заключается лишь в том, что я не желаю играть в их игры.
Не желаю быть фигурой на поле боя, не желаю быть деталью в науке, не желаю служить, отстаивать и верить.
Все мои умения служат лишь тому, чтобы быть вне системы, выстроенной демонами.

--- А чего ты хочешь, Уэсиба?

--- Жить хочу, Стенающий.
Ты когда-нибудь жил?

Стенающий кивнул.

--- Я гулял по джунглям.
А ещё по Миситру, когда только занималась заря и солнце подсвечивало Винтовой храм.
В эти моменты я жил, а не служил кому-то.

--- Тогда ты меня понимаешь.
Расскажи мне про Миситр.
Я хочу знать, как выглядит Винтовой храм...

\razd

--- Может быть, я загляну на Тра-Ренкхаль, --- задумчиво сказал Уэсиба.
--- Выглядит очень привлекательно.
Этот Безымянный... бог с фантазией на биологию, отдаю ему должное.
Тра-Ренкхаль ещё не мелиорировали?

--- Нет.
Стараниями Скорбящих планету объявили заповедником и трогать её биосферу категорически воспрещено.
Тси, разумеется, частью биосферы не считаются, и их трогать можно.

--- Ничего себе.
Это вообще впервые в истории такое.
Мои знакомые боги умерли бы от зависти...
Не знаешь, где сейчас Безымянный?

--- На Тси-Ди, следит за планетой вместе с местным компьютерным разумом.
Официально Скорбящие его используют как мелиоратора, но после изгнания с Тра-Ренкхаля он, скажем так, немного в стороне от их дел.
На Тси-Ди ему дали полную свободу, и он играет с планетой как хочет --- завозит какие-то истреблённые виды с других планет, создаёт свои, разворачивает биомы, подпиливает и шлифует биосферу...
Тси-Ди сейчас очень напоминает Капитул, так же много видов сапиентов и зверья, но гораздо красочнее и меньше урбанистики.
Стрекозодраконы летают, соволюди...

--- Стрекозодраконы?
Это случайно не печально известные сюзерены с Драконьей Пустоши, которых истребили по приказу максима Смока?

--- Они самые.
Безымянный их развёл по просьбе Шакала.
Была целая операция, чтобы стащить ДНК из биобанка Ордена, и сейчас они телепатически связаны с Безымянным, как ранее были связаны с Кох.
Оказывается, связь с богом для них жизненно необходима, и изменить это пока не смогли...
Грозные твари.

--- Вот, значит, как.
Тоже своего рода нейтрал, этот Безымянный.
Думаю, он на своём месте.

--- А давно ты заделался попрошайкой?

--- В этой жизни.
В прошлой был монахом.
В позапрошлой, дай земля памяти, попробовал себя в коневодстве.
Был ещё и геологом, пастухом, каллиграфом, танцором, секс-работником, предсказателем, зубным врачом.
Ещё ни разу не повторился, представь?

--- А если профессии закончатся?

--- Тогда пойду по второму кругу.

\razd

--- Ух, какая холодная вода, --- Стенающий быстро выбежал из моря и завернулся в полотенце.

Уэсиба хихикнул и похлопал товарища по спине.

--- Приятно, да?
Есть что-то в этом, когда холод раздражает твои рецепторы, когда напрягаются мышцы волос и когда становится пупырчатой?
А капли на коже?
Это потрясающе.

--- И мокрая рубаха, --- поддержал Стенающий.
--- У меня на планете дожди были стеной.
Дышать надо осторожно, чтобы не нахлебаться...
А по залитым дорогам плавали испуганные и ничего не понимающие змеи.

Уэсиба захохотал.

--- Никогда не видел ошарашенных змей.
Наверное, это забавно.

--- Забавно ровно до того момента, пока одна из них тебя не цапнет.

Стенающий вдруг замолчал.

--- Слушай, Уэсиба, я вдруг подумал...
Вот мы с тобой живём... вот так.
Ничего не производим, ничего не делаем полезного для общества.
Но что будет, если все так будут жить?

--- А ты не думай за всех, --- крякнул Уэсиба.
--- Думай за себя.
Все остальные пусть думают о чём хотят.

--- А расскажи про свою планету.

--- Про какую из?

Стенающий смутился.
Он по-прежнему считал <<своей>> лишь одну, ту самую...

--- Про какую хочешь.

\razd

--- Я вдруг понял, чем я хочу заниматься.

--- Значит, пришло нам время расстаться.

--- Тебе не грустно?

--- Нет.
Я очень за тебя рад.

--- Мне грустно.

--- Если ты реально нашёл себя --- ты переживёшь эту грусть, Стенающий, --- Уэсиба похлопал товарища по плечу и кивнул в знак прощания.

...Переходя пути, Стенающий обернулся.
Старик-попрошайка сидел на прежнем месте и дремал, склонив седую голову.
