\documentclass[a4paper,12pt,fleqn]{book}\usepackage{polyglossia}\setdefaultlanguage[babelshorthands=true]{russian}\setotherlanguage{english}\defaultfontfeatures{Ligatures=TeX,Mapping=tex-text}\usepackage{xcolor}\newcommand{\ml}[3]{#2}

% \documentclass[a4paper,12pt,fleqn]{book}\usepackage{cooltooltips}\usepackage{polyglossia}\setdefaultlanguage[babelshorthands=true]{russian}\setotherlanguage{english}\defaultfontfeatures{Ligatures=TeX,Mapping=tex-text} \usepackage{xcolor}\definecolor{lightgray}{HTML}{bbbbbb}\color{lightgray}\newcommand{\ml}[3]{\textenglish{\textcolor{black}{#3}} }

% ----------------------

\usepackage{amsmath,amssymb,amsfonts,xltxtra,microtype,graphicx,textcomp}
\usepackage{svg}

% ------ GEOMETRY ------

\usepackage[twoside,left=2.5cm,right=3cm,top=3cm,bottom=4cm,bindingoffset=0cm]{geometry}

% ------ FONT ------

\usepackage{ebgaramond}
\definecolor{darkblue}{HTML}{003153}

% ------ HYPERLINKS ------

\usepackage{hyperref}
\hypersetup{colorlinks=true, linkcolor=darkblue, citecolor=darkblue, filecolor=darkblue, urlcolor=darkblue}

% ------ EPIGRAPH ------

\usepackage{epigraph}
\renewcommand{\epigraphsize}{\footnotesize}
\epigraphrule=0pt
\epigraphwidth=8cm

\usepackage{etoolbox}
\AtBeginEnvironment{quote}{\itshape}
\makeatletter
\newlength\episourceskip
\pretocmd{\@episource}{\em}{}{}
\apptocmd{\@episource}{\em}{}{}
\patchcmd{\epigraph}{\@episource{#1}\\}{\@episource{#1}\\[\episourceskip]}{}{}
\makeatother

% ------ METADATA ------

\newcommand{\tofaauthor}{\ml{$0$}{Эмиль~Весна}{Emil~Viesn\'{a}}}
\newcommand{\tofatitle}{\ml{$0$}{НЕБЕСНЫЕ~СКАЛЫ}{The~Skycliff}}
\newcommand{\tofastarted}{29.06.2020}

% ------ FANCY PAGE STYLE ------

\usepackage{fancyhdr}
\pagestyle{fancy}
\fancyhead[LE,RO]{\thepage}
\fancyhead[LO]{{\small\textsc{\tofatitle}}}
\fancyhead[RE]{{\small\textsc{\tofaauthor}}}
\fancyfoot{}
\fancypagestyle{plain}
{\fancyhead{}
\renewcommand{\headrulewidth}{0mm}
\fancyfoot{}}

% ------ NEW COMMANDS ------

\newcommand{\asterism}{\vspace{1em}{\centering\Large\bfseries$\ast~\ast~\ast$\par}\vspace{1em}}
\newcommand{\textspace}{\vspace{1em}{\centering\Large\bfseries<...>\par}\vspace{1em}}
\newcommand{\FM}{\footnotemark}
\newcommand{\FL}[2]{\footnotetext{См. \textit{\hyperlink{#1}{#2}}.}}
\newcommand{\FA}[1]{\footnotetext{#1 \emph{\ml{$0$}{---~Прим.~авт.}{---~Author.}}}}

\newcommand{\theterm}[3]{\textbf{\hypertarget{#1}{#2}} --- #3}
\newcommand{\thesynonim}[3]{\textbf{#2} --- см. \textit{\hyperlink{#1}{#3}}.}
\newcommand{\theorigin}[3]{\textit{#1:} #2 --- #3}

\begin{document}
 
% ------ TITLE PAGE ------

\begin{titlepage}
{\centering{~\par}\vspace{0.25\textheight}
{\LARGE\tofaauthor}\par
\vspace{1.0cm}\rule{17em}{1pt}\par\vspace{0.3cm}
{\Huge\textsc{\tofatitle}\par}
\vspace{0.3cm}\rule{17em}{2pt}\par\vspace{1.0cm}
{\Large\textit{\ml{$0$}{Фэнтези}{Fantasy}}\par}
\vspace{0.5cm}\asterism\par\vspace{1.0cm}
{\textbf{\ml{$0$}{Начато:}{Started:}}~\tofastarted\par}\vfill
{\Large\ml{$0$}{Создано~в}{Created~by}~\XeLaTeX}\par}
\end{titlepage}

\tableofcontents

\chapter{Небесные скалы}

\section{Райские врата}

Херувим говорил долго, употребляя изысканные восточные эпитеты и многоступенчатые предложения.
Он говорил настолько долго, что я успела отойти от шока, вызванного падением, смириться с собственной смертью, осознать, где я нахожусь, и заскучать.

Я не была праведницей, живущей по священному писанию, я не была йогом, начитывающим мантры в священном экстазе.
Я была обычной девушкой, которая жила как все, радовалась обычным вещам и никому не делала плохого.
Видимо, этого всё-таки хватило на путёвку в Рай.

--- Неважно, богата ты или бедна, умна или простодушна, --- говорил херувим.
--- Перед лицом Бога все равны, и врата Рая --- одни для всех.
Да, кстати, ты умеешь читать?
Вообще должна, глаза умные.
Да?

На этом его речь закончилась.
Это было неожиданностью.
Несколько минут --- если здесь вообще есть время --- я тупо смотрела на то, как херувим вкладывает в ножны меч, шарит по карманам, перебирает бронзовые ключи на огромной связке и один за другим открывает массивные замки.

--- Читать? --- наконец переспросила я.
--- Да, конечно.

--- Отлично, --- с облегчением сказал херувим.
--- Надоели эти безграмотные, которых мне по долгу службы надо провожать.
Мрут чаще, в рай попадают чаще --- агнцы, чтоб их...

--- А что дальше? --- спросила я.

--- Дальше следуйте информационным указателям, читать умеете, --- сухо сказал херувим.
--- Проходите, проходите, не задерживайте очередь...

Информационный указатель встретил меня на развилке сразу же за вратами Рая.
Он гласил: <<Мужчины --- направо, женщины --- налево>>.
Третья дорога --- вернее, тропинка, широкая и утоптанная, но не замощённая и лишённая указателей --- вела от развилки прямо, куда-то в поля.
Именно тогда у меня появились первые подозрения.

\section{Холод}

Дьявол, как холодно.
Я не думала, что я буду так мёрзнуть в Раю.

--- Да чтоб тебя черти в Аду драли, --- выругалась я в адрес Бога, которого мне так и не довелось увидеть.
Удивительно, но никто не покарал меня за богохульство.
Никто вообще не обратил на меня внимания.
Узкий переулок, в котором я решила провести холодную ночь, был пуст, если не считать нескольких жмущихся друг к другу собак.
<<Псы попадают в Рай>>.
Если это правда, то здесь их должны быть сотни миллионов.

Чем дальше я шла по Раю, тем больше он напоминал мегаполис --- зеркальные небоскрёбы, шлагбаумы, закрытые двери, непонятные вывески.
Информационным указателям я перестала следовать после шестого.
Он гласил: <<Высшее образование --- направо, два и более высших --- прямо, нет высшего образования --- налево>>.
Вместо этого я свернула с дороги и пошла между небоскрёбами.

Когда солнце опустилось за горизонт, я отбросила остатки брезгливости и пошла греться к собакам.
Я погладила белого самоеда, легла рядом с ним и зарылась промёрзшими пальцами в густую шерсть.
Прочие псы, не издав ни звука, улеглись прямо на меня, словно разношёрстное меховое одеяло.
Тяжесть и тепло быстро окутали мою усталую голову, и я погрузилась в сон.

\section{Часть стаи}

Утром первым моим ощущением был голод.

--- Да какого хрена, я же уже мертва! --- возмутилась я.

Собаки, с которыми я провела ночь, смотрели на меня.
Я вдруг поняла, что они тоже голодны.
Вокруг был лишь пустынный город --- ни забегаловок, ни помоек.
Они могли разорвать меня, они могли съесть друг друга, если бы только...

<<Если бы это было решением проблемы>>, --- закончил белый самоед.

<<У нас впереди вечность, --- собаки каким-то образом доносили свои мысли без единого звука.
--- Твоего мяса хватит на сутки, но твоего тепла может хватить на века>>.

<<Оставайся с нами, --- просил самоед.
--- Здесь холодные ночи>>.

--- Это точно Рай? --- пробормотала я.

<<Не ходи туда, куда велят тебе таблички, --- говорили собаки.
--- Останься с нами>>.

И я осталась.

\section{Привычный маршрут}

Прошло много дней.
Голод, как ни странно, не усиливался со временем --- достигнув какой-то точки, он превратился в постоянное фоновое чувство.
Он был не настолько сильным, чтобы свести с ума, и не отнимал силы.
Я по-прежнему была способна ходить и спать.

Собаки целыми днями шатались по окрестностям, и я составляла им компанию.
Вскоре я заметила, что их маршрут повторяется --- поворот направо, прямо, три поворота налево, ещё два направо.
Всего собаки преодолевали тысячу сто двадцать пять дорожных развилок за день, возвращаясь на то же место, где и проводили ночь.
Любые мои предложения изменить маршрут встречали непонимание.

--- Я не могу ходить по одним и тем же дорогам вечность, --- заявила я как-то вечером.
--- Голод и холод --- сущие пустяки по сравнению с этой рутиной.

<<Другие дороги опасны>>, --- сообщили псы.

--- Вы проверяли? --- допытывалась я.

<<Мы видели места гораздо хуже.
Это лучшее место для нас>>.

--- Тогда почему вы так страдаете? --- спрашивала я.
Ответом было молчание.

\section{Глупая смерть}

--- А ты как сюда попала?

--- По глупости, --- коротко ответила я.

--- А поподробнее?

--- Десять лет занимаюсь скалолазанием, упала с табуретки и ударилась головой.

--- Извини, --- пробормотал он.

--- А ты?

--- Рак яичек.

\section{Благодарность для Бога}

Я обернулась к Роме.

--- Знаешь, ты мог бы мне понравится, будь мы живы, --- сказала я.

Рома грустно улыбнулся.

--- Вряд ли.

--- Ты смелый.

--- Просто не хочу стать куском мяса, лежащим в Уплотнителе.

--- Это и есть смелость.

Рома опустил голову.

--- Я прожил жизнь глупо.
Я чаще молился, чем говорил женщинам о своих чувствах.
Даже если бы Рай оказался тем Раем, о котором нам говорили, эта мысль мучила бы меня до конца вечности.

--- Пойдём со мной.
Пожалуйста.

Рома покачал головой.

--- Я буду обузой на этой скале.
Она твоя.

Я подошла к Роме и поцеловала его в губы.
Поцелуй был призрачным, словно дуновение ветра;
губы Ромы на вкус были подобны тёплой воде.

--- Прощай, Вика.
Больше не падай.
И если увидишь Бога --- передай ему от меня благодарность, по морде, со всей силы.

\section{Первая и последняя попытка}

Я посмотрела вниз.
Солнце стояло высоко, и я увидела широкие и глубокие расщелины, располосовавшие камень у подножия скалы.
Стены расщелин были абсолютно гладкие, словно трещины в стекле;
если я упаду в одну из них, второй попытки у меня не будет.

Я запустила пальцы в мешочек с магнезией и ухватилась за следующий выступ.

\section{Свобода}

--- Почему? --- закричала я ангелам.
--- Почему вы ему служите?!

Зверь захохотал.

--- Они будут служить любому, кто сидит на Небесном престоле.
Они будут выполнять все его приказы.
Те, кто хотел служить по совести, уже давно в Преисподней.

--- Будь ты проклят! --- заорала я.

--- В Уплотнитель её, --- буркнул Зверь.

Ангелы рывком поставили меня на ноги, вывернув руку.
Но я этого не чувствовала.
Моё внимание поглотили две вещи, попавшие в поле зрения.
На поясе ангела висел мой крюк, незакреплённый, с тросом достаточной длины, чтобы сделать размах.
А на стене в углу зала примостился рычаг, похожий на стоп-кран в метро.
Над ним была небольшая наклейка с красными буквами, которые гласили:

<<СВОБОДА>>

...Я с размаху вогнала крюк ангелу в шею, точно в не прикрытую пластиной доспеха щель.
Раздался дикий визг, похожий то ли на человеческий крик, то ли на звук сирены скорой помощи.
В следующий миг я рванулась к рычагу стоп-крана и дёрнула его вниз.
Ещё миг спустя три ангельских копья, три полоски невообразимо блестящего белого металла намертво пригвоздили меня к каменному паркету зала.

Ослеплённая болью, я ещё успела услышать отчаянный рёв Зверя.
Зал распадался на куски, превращаясь в водоворот камня, стекла и древесных щепок.
То же происходило и со скалой.
Вскоре волна разрушения добралась до нижних уровней Рая.
Мироздание огласил хор миллионов, миллиардов голосов;
освобождённые души летели во все стороны, чтобы навсегда исчезнуть в глубинах космоса.

--- Вика, Вика! --- голос Ромы звучал откуда-то издалека.
--- Дай мне руку, дай мне руку! 
Я отнесу тебя домой...

\section{Невидимые цепи}

Первым пришло чувство запертости.
Меня сковали невидимые цепи, подвесили в пустоте.
Я пыталась закричать, но не могла даже открыть рот.
Вскоре до меня дошло, что рот и так открыт;
из него торчала пластиковая трубка.

\section{Кусочек прежней жизни}

--- Я никогда больше не смогу лазать?

--- Улучшение возможно, настолько, чтобы передвигаться на коляске и, возможно, на костылях.

Я закрыла глаза и отвернулась к стене.

--- Лёша не приходил?

--- Я ему сказала, что ты в коме, --- извиняющимся тоном ответила мама.
Продолжения не требовалось.

--- Принеси мне мою снарягу, --- попросила я.

--- Вика, пожалуйста!
Врачи...

--- Мне всё равно.
Принеси снарягу мне сюда.

\section{Окно в больнице}

Я сидела у больничного окна, привалившись к холодной, крашенной голубой краской стене.
Расстояние до окна отняло больше сил, чем скалодром.
Пользуясь одной рукой и тросами, я подняла себя в сидячее положение, подтянула табуретку, кое-как зафиксировала тело на табуретке и медленно, сантиметр за сантиметром, подтащила табуретку к окну.

--- Да что ж ты делаешь, тебе постельный режим нужен! --- всполошилась медсестра.

--- Я хочу посидеть у окна.
Да не надо меня отвязывать!

Сестра, безуспешно поколупавшись с узлами и карабинами, устало села на кровать.

--- Ну и что теперь с тобой делать?
Как я тебя обратно положу?
У меня и без тебя работы невпроворот!

--- Я сама отвяжусь и лягу.
Просто дайте мне посидеть у окна.

--- Ты ногу перетянула!
Холодная нога совсем!
Синяя!

--- Это вы мне ногу перетянули, пока пытались меня развязать!

Я не без труда ослабила трос, охватывающий бедро.
Нога медленно стала приобретать нормальный цвет.

\section{Чай и тортик}

Ещё неделю я провела в больнице, настойчиво пытаясь вернуться к нормальной жизни.
Я научилась перемещаться по всей палате --- к окну, к раковине, чтобы почистить зубы и умыться, к креслу-каталке.
Падала каждый час, и после каждого падения на руках оставались крупные чернильные синяки.
В конце концов, поглядев на протянутые по палате тросы, лечащий врач велел оставить мне одну каталку.
Так я начала выбираться в коридор.

Ещё через неделю начала шевелиться нога.
Всего одна.
Совсем немного.
Впервые с момента пробуждения я заплакала.

Больницу я покинула уже на костылях.

--- Пойдём, Вика, я такси вызвала.

--- Я хочу пешком.

--- Дочка, дом далеко!

--- Пара километров.
Не хочу в машину.

Мы с мамой останавливались у каждой скамейки и подолгу сидели на ней.
Я слушала мамину болтовню, слушала стрекотание кузнечиков, слушала шорох травы и листьев берёз --- и не могла наслушаться.

--- Я тебе купила твой любимый чай и тортик.
Ой, а папу нашего вчера повысили, он с работы не вылезал с тех пор, как ты в больницу попала...
Сашка тебе тоже подарок приготовила, она такая умница, вот увидишь, тебе точно понравится...

\end{document}
