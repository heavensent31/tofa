% \documentclass[a4paper,10pt,fleqn]{book}\usepackage{polyglossia}\setdefaultlanguage[babelshorthands=true]{russian}\setotherlanguage{english}\defaultfontfeatures{Ligatures=TeX,Mapping=tex-text}\usepackage{xcolor}\newcommand{\ml}[3]{#2}

\documentclass[a4paper,10pt,fleqn]{book}\usepackage{polyglossia}\setdefaultlanguage{english}\setotherlanguage{russian}\defaultfontfeatures{Ligatures=TeX,Mapping=tex-text}\usepackage{xcolor}\definecolor{lightgray}{HTML}{bbbbbb}\color{lightgray}\newcommand{\ml}[3]{\textcolor{black}{#3}}

% ----------------------

\usepackage{amsmath,amssymb,amsfonts,xltxtra,microtype,graphicx,textcomp}
\usepackage{svg}

% ------ GEOMETRY ------

\usepackage[twoside,left=2.5cm,right=3cm,top=3cm,bottom=4cm,bindingoffset=0cm]{geometry}

% ------ FONT ------

\setmainfont{Linux Libertine}
\definecolor{darkblue}{HTML}{003153}

% ------ HYPERLINKS ------

\usepackage{hyperref}
\hypersetup{colorlinks=true, linkcolor=darkblue, citecolor=darkblue, filecolor=darkblue, urlcolor=darkblue}

% ------ EPIGRAPH ------

\usepackage{epigraph}
\renewcommand{\epigraphsize}{\footnotesize}
\epigraphrule=0pt
\epigraphwidth=8cm

\usepackage{etoolbox}
\AtBeginEnvironment{quote}{\itshape}
\makeatletter
\newlength\episourceskip
\pretocmd{\@episource}{\em}{}{}
\apptocmd{\@episource}{\em}{}{}
\patchcmd{\epigraph}{\@episource{#1}\\}{\@episource{#1}\\[\episourceskip]}{}{}
\makeatother

% ------ METADATA ------

\newcommand{\tofaauthor}{\ml{$0$}{Эмиль~Весна}{Emil~Viesn\'{a}}}
\newcommand{\tofatitle}{\ml{$0$}{ПЛАМЯ~ОСЕНИ}{Flame~of~the~Fall}}
\newcommand{\tofastarted}{01.10.2021}

% ------ FANCY PAGE STYLE ------

\usepackage{fancyhdr}
\pagestyle{fancy}
\fancyhead[LE,RO]{\thepage}
\fancyhead[LO]{{\small\textsc{\tofatitle}}}
\fancyhead[RE]{{\small\textsc{\tofaauthor}}}
\fancyfoot{}
\fancypagestyle{plain}
{\fancyhead{}
\renewcommand{\headrulewidth}{0mm}
\fancyfoot{}}

% ------ NEW COMMANDS ------

\newcommand{\asterism}{\vspace{1em}{\centering\Large\bfseries$\ast~\ast~\ast$\par}\vspace{1em}}
\newcommand{\textspace}{\vspace{1em}{\centering\Large\bfseries<...>\par}\vspace{1em}}
\newcommand{\FM}{\footnotemark}
\newcommand{\FL}[2]{\footnotetext{См. \textit{\hyperlink{#1}{#2}}.}}
\newcommand{\FA}[1]{\footnotetext{#1 \emph{\ml{$0$}{---~Прим.~авт.}{---~Author.}}}}

\newcommand{\theterm}[3]{\textbf{\hypertarget{#1}{#2}} --- #3}
\newcommand{\thesynonim}[3]{\textbf{#2} --- см. \textit{\hyperlink{#1}{#3}}.}
\newcommand{\theorigin}[3]{\textit{#1:} #2 --- #3}


\begin{document}

% ------ TITLE PAGE ------

\begin{titlepage}
{\centering{~\par}\vspace{0.25\textheight}
{\LARGE\tofaauthor}\par
\vspace{1.0cm}\rule{17em}{1pt}\par\vspace{0.3cm}
{\Huge\textsc{\tofatitle}\par}
\vspace{0.3cm}\rule{17em}{2pt}\par\vspace{1.0cm}
{\Large\textit{\ml{$0$}{Повесть}{Novella}}\par}
\vspace{0.5cm}\asterism\par\vspace{1.0cm}
{\textbf{\ml{$0$}{Начато:}{Started:}}~\tofastarted\par}\vfill
{\Large\ml{$0$}{Создано~в}{Created~by}~\XeLaTeX}\par}
\end{titlepage}

\tableofcontents

\chapter{Пламя Осени}

\section{Крушение легенды}

--- Башня Дьявола рухнула!
Вставайте, вставайте!

С этих слов началось для Курц Штайгер утро, полное криков, топота и звона оружия.

Башней Дьявола издревле называли огромный кристаллический столп, росший на краю друзы Хербст.
Шли века и тысячелетия, эрозия подтачивала Башни, и они рушились одна за другой.
Некоторые падали неудачно, раздавливая целые города и деревни.
Некоторые падали удачно, на несколько сезонов превращаясь в каменный мост с одной Друзы на другую.
Чаще всего, разумеется, односторонний --- наклон и положение Башни редко позволяли одинаково хорошо идти вверх и вниз по ней.

О Башне Дьявола знали давно, к ней обращались мечты завоевателей и торговцев.
Мелкие люди, которые умирали, не дождавшись своего часа, а стареть начинали и того раньше, ждали и молили Всестроителя, чтобы Башня упала.
%{Overbuilder}
Но Башня Дьявола стояла.
По крайней мере, до сегодняшнего дня, когда её основание надломилось и исполинский кристалл встал строго горизонтально, соединив друзы Хербст и Гарда Викка.

Курц встала и протёрла глаз.
К чему вся эта беготня?
К Башне никто не подойдёт в ближайшие десять-пятнадцать дней.
Нужно время, чтобы она утряслась, плотно легла в своём каменном ложе, может быть, даже треснула напополам и отправилась в чёрную пучину Заалвира, разом похоронив все страхи и надежды людей.
%{Saalweird}
К чему эта беготня?

Курц зевнула и тут же ощутила запах свежеиспечённого куриного пирога --- с луком, картофелем и перцем.
Мама готовила его очень редко --- в последний раз это было года два назад, когда...

--- Ма-ам, --- сонно позвала Курц.
И тут же осеклась, вспомнив, что ей никто не ответит.

\section{Запах пирога}

Когда Курц вошла на кухню, пирог уже покинул печь и ждал на столе, сверкая подрумяненными, смазанными маслом боками.
У печи копошилась незнакомая женщина лет пятидесяти --- щуплая, седая, но ещё сохранившая гибкость и силу.
Услышав шаги Курц, женщина обернулась и нахмурила брови.

--- Садись и ешь, --- не особенно вежливо сказала она.

Курц молча повиновалась.

Традиция <<объедания>> существовала на Хербст издревле.
Всегда находились двое-трое стариков, которые освобождали молодоженов или вдовцов от необходимости готовить.
Разумеется, не бесплатно --- объедалы питались тем, что приготовили из хозяйских продуктов, а порой и утаскивали всё, что плохо лежит.
Никто не испытывал к объедалам светлых чувств, но их и не гнали --- традиция есть традиция.
Это могло продолжаться несколько дней или десятков дней после свадьбы или похорон --- всё зависело от наглости стариков или характера хозяев.

Даже в этот раз нашлась приживалка.

--- Я к тебе только на сегодня, --- сообщила незнакомка за едой.
--- Дел по горло.

--- Если хватит сил утащить все продукты за раз, можешь даже до вечера не ждать, --- съязвила Курц.
--- Кстати, пирог очень вкусный.
Мама так точно не умела, покойная головушка.

--- Земля пухом, --- поклонилась женщина.
--- Только ты меня не позорь зазря, я не за продуктами пришла.

--- Но и не из вежливости.
Поэтому выкладывай, по возможности покороче.

Женщина разгладила юбку и вздохнула.

--- Я знала Сабину, Курц.
Достойная женщина, смелая и умная.
Такая потеря...

--- Ещё короче.

--- Ты знаешь, у меня недавно сын замуж вышел.
Такого прелестного мальчика в дом привёл, любо-дорого глядеть...

--- Что именно в словах <<ещё короче>> тебе непонятно?

Женщина поджала губы.

--- Я хочу купить у тебя дом.
Для одной тебя он чересчур велик, признай это.
Ты только убираться здесь будешь три дня.

--- Сложно отрицать очевидное.
Мне просто интересно, что именно тебя привело?
Упавшая Башня Дьявола или моя безвременно ушедшая мама?

--- А это моё дело, --- подбоченилась женщина.
--- Тревожно сейчас, облачно на горизонте, и у меня и у тебя.
\ml{$0$}
{А я предлагаю тебе стабильную валюту в нестабильные времена.}
{All I offer you is stable currency for unstable times.}
\ml{$0$}
{Грех отказываться.}
{Too good a chance to miss.''}

--- Сколько?

--- Сорок аркан.

--- А я твоему сыну отсосать не должна? --- спокойно осведомилась Курц.
--- И его мужу в придачу?

--- Я вошла в твоё положение, войди и ты в моё! --- вспылила женщина.
--- Пятьдесят --- последнее слово, не будь я Анна Вагнер!

--- Вошла она, как же, --- усмехнулась Курц.
--- Сто аркан.
Никаких баронских кредитов, займов по соседям и долгов под честное слово, чистые деньги из чистых рук.

--- Идёт, --- буркнула Анна.

--- Ещё тридцать за меблировку и вещи, исключая моё рабочее снаряжение и походную тележку.

Гостья вздохнула.

--- По рукам.
Вечером принесу деньги.
Приятного аппетита.

Анна пододвинула пирог ближе к Курц.

--- Плохая ты женщина, Курц, --- укоризненно сказала она.
--- Злая.
Сразу видно --- неименоваха.

--- Следи за словами, Анна, --- Курц задумчиво укусила пирог.
--- Ко мне ещё твои внуки придут.
Не хочу им рассказывать, как их бабка наживается на чужом горе.

Гостья рывком встала из-за стола, замоталась в шаль и выбежала из дома, хлопнув дверью.

\section{Запах зла}

Да, для неё процесс опорожнения выглядел так --- снять мехи, выдавить содержимое, обварить кипятком, затем обработать спиртом.
Достать новые мехи, промазать уплотнителем, аккуратно вставить, привычно поморщившись от боли.
И так каждый день.

Курц знала, что от нее всегда исходит неприятный специфический запах.
Раньше её это волновало --- она натиралась маслами, пытаясь отбить вонь;
сейчас ей было всё равно.

\section{Творец}

\epigraph{Начало Творца было положено самой сутью мироздания.
Творец есть ответ на незаданный вопрос.
Но тот, кто вылепил из глины первых людей, не был равен тому, кто есть лишь ответ;
Творец сотворил себя сам --- и лишь тогда обрёл тот гений, что позволил ему стать Творцом Вселенной.}
{Хакем-Аят, 2:8--10}

Курц сняла маску, обнажив то, на месте чего должно было быть лицо.
Заросший глаз, стянутая кожа, кривое отверстие на месте рта, обнажённые носовые ходы...
Среди старых шрамов багровел ещё свежий, похожий на сколопендру.
Однажды женщине надоел не до конца закрывающийся рот, с уголка которого вечно капала слюна.
Местные врачи отказывались помогать, считая девушку безнадёжной.

--- Красивее ты от этого не станешь, --- напрямик сказала одна из них.

--- Я к тебе не за красотой пришла, идиотка, --- ответила Курц.

В конце концов она наткнулась на баронского полевого хирурга --- рассеянного, впадающего в деменцию старичка.
Тот отказался её оперировать, сославшись на трясущиеся руки, но подробно рассказал, как бы он это сделал.
Курц долго думала, примерялась, размышляла, а потом взяла бритву и исправила дефект.

Когда-то давно девчонка из Валленбергов --- местная красавица --- бросила ей в лицо:
<<Я бы лучше умерла, чем жила с этим!>>
Девчонка умерла при родах вместе с ребёнком.
Курц жива до сих пор.
И даже рот закрывается.

\section{Кленовое пламя}

Кленовые листья лежали на траве, словно язычки пламени.
Огнём горела и тут же сохла рябина, держа в высохших старческих руках горсти красных ягод.

\section{Знак от Вселенной}

Однажды она любила человека.
Молодой парень с жилистым телом и горячими руками краснел и смущался.
Он очень хотел узнать, что она прячет под маской.
Она показала ему.

Спустя год, когда Курц уже уверилась в том, что её приняли, он ушёл к другой --- молодой, красивой, по выражению Курц, <<у которой жопа на правильном месте>>.
Иногда она встречала его на окраинах города --- уже начавшего седеть, полненького, лысого мужчину с пушистыми усами.
Он не испытывал особых чувств к женщине, с которой жил, но очень любил своих детей --- забавно говорил с ними и таскал их на загривке.
При встрече он смущённо улыбался и говорил <<Привет, Курц>>.
Курц кивала в ответ.

--- Не переживай, девочка, --- сказала тогда мама за обедом.
--- Будут другие, получше.

--- Мне бы твой оптимизм.

--- Это не оптимизм, это признание очевидного.
Я тебя люблю --- не только потому, что ты моя дочь, но ещё и потому, что ты сильная и смелая, какой я никогда не была.
И он тебя любил.
Если бы не любил --- не был бы с тобой так долго.
Полюбил один --- найдутся и другие.

--- Твои-то <<другие, получше>> где задерживаются? --- пошутила Курц, брызнув на маму чаем.
Обе улыбнулись --- Сабина Штайгер никогда не страдала от недостатка поклонников.

--- С меня хватит, --- отмахнулась мама.

<<Вот и с меня тоже>>, --- сказала тогда Курц, но про себя, чтобы не огорчить маму.
С тех пор прошло пятнадцать лет, и они по-прежнему вместе.

Да и кого встретишь на крохотном островке камня, окружённом бездонной газовой пропастью?
Кого встретишь в постоянном хороводе <<работа-дом-работа>>?
Друзья юности постепенно завели семьи, умерли, улетели на другие Друзы...
Курц знала половину жителей Хербст по именам, вторую половину знала в лицо.
Она была первой, кого видели дети, покинув дома.
Она была последней, кого видели авантюристы, летящие сквозь туманную тьму в поиске счастья.

<<Я --- Штайгер.
В мир, где правят хук и глайдер, меня привела семья, которая учит детей обращаться с ними.
%{hook-en-gleider}
Разве не благородно посвятить этому делу всю жизнь, всю себя?>>

Эта нехитрая молитва помогала много лет.
Но одним промозглым дождливым днём, стоя над свежей могилой мамы, Курц поняла --- кроме неё на похороны пришёл лишь могильщик.
Никто из тех, в чьи руки Сабина Штайгер вложила хук, не нашёл времени, чтобы её проводить.

<<Мне нужны друзья>>, --- решила тогда Курц, засыпая.
Было очень непривычно не слышать дыхание мамы в другом углу комнаты.
Вселенная ответила на её призыв в своей обычной непонятной манере --- на следующее же утро рухнула Башня Дьявола.

\section{Чёрный нож}

--- Пожалуйста, мама, пей! --- Курц со слезами пыталась влить в рот мамы воду.
Сабина металась в горячке, выкрикивая несвязные слова;
вода расплёскивалась, не достигая горла.

--- Злокачественный туберкулёз лёгких, --- вынес вердикт пришёдший врач.
--- Также известный как <<Пламя осени>>.
Я бы на твоём месте переночевал где-нибудь вне дома, а завтра провёл влажную уборку с отваром полыни.
Сейчас она заразна.

--- Она была совсем здорова вчера!

--- Так оно обычно и бывает, --- развёл руками врач.
--- Заболевание развивается в течение десяти-двадцати дней, иногда года, а потом человек сгорает, быстро и безвозвратно.
Судя по цвету кожных покровов, она вряд ли переживёт ночь.
Если решишь, что с неё хватит...

Врач вынул из-за пояса чёрный нож в ножнах, скреплённых сложной печатью из смолы и полосок ткани с надписями.
У Курц ёкнуло сердце.

--- Вот сюда, --- врач вытащил хирургический маркер и нарисовал на груди Сабины шестилучевую звёздочку.
--- Строго вертикально, резко, изо всей силы, проворачивать не надо.
Сломанную печать и нож принесёшь мне.

--- Можно это сделает кто-то другой? --- прошептала Курц.
В её горле стоял комок.

--- Я не могу, таков закон.
Сделать может любой человек, которому ты доверяешь, но нож и печать должна принести ты.
И не вздумай потерять печать, Курц Штайгер, иначе тебя обвинят в убийстве, ты меня поняла?

--- Поняла.
У меня нет никого, кому я могла бы это доверить.

--- В таком случае у тебя два выхода: либо дождись, пока она умрёт сама, либо возьми себя в руки и сделай то что должно.
\ml{$0$}
{Если нужно отпеть --- я могу прислать Ворона.}
{If she needs last rites, I'll send the Raben.''}

--- Не нужно, --- процедила сквозь зубы Курц.
\ml{$0$}
{--- Пусть Ворон своих птенцов отпевает.}
{``Let the Raben give last rites to his birdlings.''}

С Вороном Биркендорфом Курц связывали давние тёплые чувства.
Когда Сабина принесла пищащую обожжённую девочку на именины, Ворон отказался проводить обряд.

--- Я даже не знаю, что за зверя ты мне принесла.

\ml{$0$}
{--- Она вышла из моей утробы, --- прорычала мама.}
{``She came out of my womb,'' mama growled.}
\ml{$0$}
{--- Я, по-твоему, кто --- ящерица?}
{``Who do you think I am---a lizard?''}

\ml{$0$}
{--- Штайгер, ты слышала мой ответ.}
{``Steiger, you heard my answer.''}

\ml{$0$}
{--- У тебя имена закончились, Биркендорф?}
{``Are you out of names, Birkendorf?''}

\ml{$0$}
{--- Имён предостаточно, а вот времени на безнадёжных нет.}
{``I have plenty of names, I'm out of time for a lost cause.}
\ml{$0$}
{Она умрёт в течение десяти дней с такими ожогами.}
{She'll die in ten days with burns like those.''}

--- Чёрта с два, --- выплюнула Сабина, подхватила ребёнка и вышла из костёла.
%{``Like hell she will,'' Sabina spitted}

Именно так Курц получила своё прозвище.
По традиции Хербст, настоящее имя мог дать только Ворон, только на тридцать первом дне жизни.
Причин отказать в именинах было немного, но такое случалось.
Ходило поверье, что неименованных преследуют неприятности и ждёт ранняя смерть.
Так оно и было, впрочем --- люди заранее относились к неименованным настороженно, а порой и неприязненно.
Ведь если тебя не нарекли --- на это была причина, верно?

Как выяснилось впоследствии, Курц не только выжила, но и научилась устраивать неприятности именованным.
В детстве девочка несколько раз пробиралась в костёл и поджигала чёрную мантию Ворона.
Один раз он её за этим поймал.
Подняв за шкирку рычащее слюнявое существо, похожее на детёныша грелла\FM, он вдруг вспомнил злосчастные непрошедшие именины.
\FA{Грелл --- существо из мифологии друзы Хербст, одноглазый тролль, который смотрит в окна, выманивая людей на улицу по ночам.}

\ml{$0$}
{--- Считай, что мы квиты, Курц Штайгер, --- лаконично сказал Биркендорф.}
{``Call it even, Kurz Steiger,'' Birkendorf succinctly said.}
\ml{$0$}
{--- Ещё раз поймаю за этим --- урою.}
{``If I catch you again, you're done.''}

\section{Баронская дочь}

Однажды Курц довелось увидеть баронскую дочь.
Большеглазая смуглая девочка лет семи шла в сопровождении учителя и нескольких солдат.
Увидев Курц, она испуганно прижалась к учителю.
Курц было не привыкать к такой реакции, поэтому она не обратила на ребёнка особого внимания.

<<Как же давно это было, --- думала Курц.
--- Должно быть, ей сейчас около двадцати.
Ко мне её не приводили, что неудивительно --- по слухам, баронских детей хуку и глайдеру обучает заезжий мастер.
Зачем же я понадобилась баронессе сейчас?>>

\section{Выкидыш}

Служанка поклонилась.

--- Агнес, фрау.

--- Скажи, Агнес, где я могу найти баронессу... баронин Ангару?
Она посылала за мной.
Я...

--- Я знаю, кто вы, --- снова поклонилась Агнес.
\ml{$0$}
{--- Вы майстерин из Штайгеров, дочь Сабины?}
{``You're the \textit{meisterin}, Steiger kin, Sabina's daughter?}
Я вас сразу узнала, у вас, эээ... --- взгляд Агнес прыгнул на заросшую глазницу Курц и тут же испуганно отбежал в сторону, --- ...волосы как у мамы.

--- Да, меня все узнают по волосам, --- ухмыльнулась Курц.
Агнес виновато улыбнулась.
--- Можешь проводить меня к баронин?

Служанка замялась и бросила взгляд в раскрытые двери комнаты.

--- Что такое?

\ml{$0$}
{--- Фрайфрау была беременна, --- шёпотом пояснила служанка.}
{``Freifrau was pregnant,'' the servant quietly explained.}

\ml{$0$}
{--- Была? --- уточнила Курц.}
{``Was?'' Kurz asked.}

--- Была, --- грустно ответила Агнес.
\ml{$0$}
{--- Выкидыш.}
{``She lost the baby.''}

--- Сам по себе?
\ml{$0$}
{Может, случилось чего?}
{Maybe, something's happened?''}

--- Я не знаю.
Я услышала её стон поутру, захожу --- а её перекорёжило, пена изо рта идёт и зубы скрипят...
Доктора позвала, пока доктор встала, пока прибежала --- а фрайфрау уже и плод выбросила.
Вся постель в крови, как будто свинью резали.
Не приведи Всестроитель...

Служанку передёрнуло, она поспешно потеребила радужный браслет с полумесяцем и прошептала молитву.
Курц решила сменить тему.

--- Агнес, не знаешь случайно, зачем она за мной послала?

--- Не знаю, майстерин, --- покачала головой Агнес.
--- У нас некого хуку и глайдеру учить, дети все выросли.
Меня, кстати, мама ваша учила.
До сих пор помню её <<три чечётки...

--- ...две трещотки, на глазок>>, --- усмехнулась Курц и хлопнула в ладоши.
--- Сколько скалолазов сорвалось из-за того, что они не знали это правило.
А мама умела доходчиво объяснять детям сложные вещи.
Мне бы ещё кто доходчиво объяснил, какого дьявола я здесь делаю...
В письме, в лучших традициях баронства, ни слова.

--- Вам лучше у управляющих разведать, --- улыбнулась служанка.
--- Вон туда по коридору и направо, спросите Рихарда.

\section{Шнелль}

Парень был худ, высок.
Тонкие губы, высокие скулы, кривой нос, тонкая бородка, хитрые зеленые глаза;
таким рисовали образ дьявола.

--- Я ищу Рихарда.

--- Ты ищешь его там, где его нет.

--- А кого я в таком случае нашла?

--- Шнелль, --- осклабился парень.
--- Вообще я Лангзам Хетвертак, но у меня и имя неудачное, и фамилия непроизносимая.
Поэтому здесь я просто Шнелль.

--- Курц Штайгер, учитель хука и глайдера.

--- Еще одна неименованная, --- Шнелль протянул мягкую тонкую ладонь.
--- Нечасто встречаю своих.
Из-за этих ожогов?

Курц кивнула и пожала руку Шнелля, отметив, как легко он говорил о её увечьях.

--- А ты?

--- А я в безворонье родился.
Старый Рольф преставился, и...

--- И тебе двадцать два года.

--- Все всегда угадывают мой возраст, --- хихикнул Шнелль.

--- Кем ты здесь работаешь?

--- Я гонец.
Почту приношу, уношу, проверяю.
Хорошо обращаюсь с чернилами и ножом для конвертов.

\ml{$0$}
{--- Может быть, ты случайно знаешь, зачем баронин Ангара меня вызвала?}
{``Maybe you know by chance, why Baronin Angara paged me?''}

\ml{$0$}
{--- Знаю, конечно.}
{``Of course I know.}
\ml{$0$}
{Профессия у меня такая --- знать.}
{That's my job, to know.}
\ml{$0$}
{Если мне не изменяет память, тебя отправляют на разведку к Башне Дьявола.}
{If I recall correctly, you're ordered to scout the Teufel Tower.''}

Курц вздохнула.
Ну конечно, зачем же ещё её могли позвать.

\ml{$0$}
{--- А если память тебе изменяет? --- с надеждой спросила она.}
{``Is there any chance you recall incorrectly?'' she asked with hope.}

\ml{$0$}
{--- Редко она мне изменяет, Курц.}
{``That happens very rarely, Kurz.}
\ml{$0$}
{Но, если хочешь, можешь поискать Рихарда и уточнить у него.}
{But, if you insist, you can look for Richard and clear it with him.}
Только медлить не советую --- сама понимаешь, какая ситуация.

Курц окинула парня оценивающим взглядом.

--- Ты сказал, что хорошо обращаешься с ножом для конвертов, --- припомнила она.
\ml{$0$}
{--- Ты ведь не только конверты режешь?}
{``Your knife is good not only for envelopes, isn't it?''}

Шнелль бросил на неё хитрый взгляд зелёных глаз.

\ml{$0$}
{--- Не только.}
{``Not only.}
\ml{$0$}
{Ты хочешь взять меня с собой?}
{Do you want me to go with you?''}

--- Кто-то должен доставить мой отчёт в баронство.
\ml{$0$}
{Мне бы не помешал спутник, знающий толк в конвертах.}
{I could use a companion who knows envelopes.''}

Шнелль задумался.

--- Ангара пока что в отключке, так что, думаю, я разделю с тобой прогулку.
\ml{$0$}
{Только хочу предупредить: хук --- это не моё.}
{Just a fair warning: hook is not my piece of pie.}
\ml{$0$}
{Предпочитаю равнинную местность.}
{I prefer flat terrain.''}

\section{Линейка}

--- А почему ты пошёл?

--- Ты мне интересна, --- напрямик сказал Шнелль.
--- Ты чем-то похожа на баронессу.
Та тоже гордая и за словом в карман не лезет.

--- Если что, я завязала с мужчинами, --- предупредила Курц.

--- Я не об этом, --- поморщился Шнелль.
--- Но если уж разговор зашёл, это ты зря.
Ни к чему хорошему такое отношение не приведёт.

--- Почему?

--- Кому-то действительно никто не нужен.
Но такие люди встречаются не так часто, и ты не из них.
Для тебя потребность в партнёре --- это как потребность в еде.
Даже если тебе больно, ты всё равно ешь, верно?

--- Плохое сравнение.

--- Достаточно хорошее.
В юности у меня была одна неудача за другой, и после каждой я говорил себе <<больше никогда>>, и неудачи всё равно повторялись.

--- А почему тебя преследовали неудачи, Шнелль?

--- Да по той же причине, что и тебя.
Неименоваха не может рассчитывать на равного партнёра.
Даже будь ты красив, красноречив, трудолюбив, ты останешься неименовахой.

--- Я, если честно, списывала свои неудачи на внешность, --- призналась Курц.

--- Возможно.
Мужчины обычно менее суеверны, когда дело доходит до партнёра.
Со мной женщины если и встречались, то тайком и если других вариантов не было совсем.

--- Что изменилось сейчас?

--- В принципе ничего, --- пожал плечами Шнелль.
--- Я просто научился брать то, что мне нужно, не отвлекаясь на мелочи.

--- Ты всё ещё хочешь быть любимым.

--- Да, --- неожиданно легко согласился Шнелль.
--- И всё ещё чего-то жду, как и ты.

--- Я думаю, что мне уже поздно чего-то ждать.

--- Думаешь, что тебя никто не полюбит?
Тоже зря.
Судя по тому, что я про тебя слышал, ты --- одна из лучших мастеров хука и глайдера на Хербст.
Человек, который в чём-то настолько хорош, обречён быть любимым.

--- Я рада, что я взяла тебя с собой.

--- Я рад, что я согласился.

Курц спешилась и подхватила козла под уздцы.
Шнелль молча подал ей бинокль.
Утренний туман на секунду рассеялся, показав редкозубую пасть Гребня Троллей.
А вот и Линейка, Башня-ориентир, на которой сделала ровные насечки сама природа.
В руках Курц щёлкнул навигационный транспортир, взволнованно завертелась освобождённая из плена магнитная стрелка.
Три, четыре, учёт азимута, сезонный сдвиг магнитного полюса, фаза Съедобных Лун, положение Железной Луны\FM... четыре тысячи шестьсот метров до Башни Дьявола.
\FA{Тысяча Башен имеет два кольца и один связанный спутниковый комплекс, называемый Съедобными Лунами (Эсслунен, Essloonen).
Две из них видны с поверхности планеты, имеют ярко-жёлтый с синевой (<<сырный>>) цвет, третья видна только из космоса.
Этимология названия не вполне ясна, но оно в той или иной форме встречается у всех народов Тысячи Башен.
Также при навигации учитывается положение массивного астероида с мощным магнитным полем, называемого Железной Луной (Айзенлун, Eisenloon).
Железная Луна не видна с поверхности, её положение высчитывают по засечкам на кольцах.}

--- Приехали, --- сказала Курц.
--- Отсюда пойдём пешком.

Шнелль кивнул и, вытащив верёвку, начал аккуратно стреноживать козлов.

\section{Фоуф}

Ложбина точно между каменным ложем и башней.
Опасный отрезок пути, который никак не обойти.
Если Башня внезапно качнётся, от Курц останется только горсть мясного пюре...

Курц спрыгнула в ложбину и побежала со всех ног.
Триста метров, двести, сто...
На лету, не сбив дыхания, женщина выстрелила лёгким гарпуном в торчащее сверху дерево, аккуратно подтянула трос и полезла.
И только оказавшись пятьюдесятью метрами выше ложбины, она без сил свалилась в кусты, чтобы отдышаться.

Башня Дьявола лежала крепко, как влитая.
Курц в бинокль оценила контакт на четырнадцать баллов из пятнадцати, постоянно сверяясь с истрёпанным маминым справочником.
Такие мосты могут стоять годами, их долговечность зависит исключительно от прочности корневого кристалла Башни и сейсмической активности региона.
Учитывая, что Гарда Викка и Хербст были одними из самых сейсмически спокойных Друз, а прочность Башни Дьявола не вызывала никаких сомнений --- речь действительно о годах.

Это плохо, очень плохо.
Гарда Викка не отличалась спокойствием.
Все хотели урвать кусочек плодородного берега Кровавой Чаши.
Хербст не имела прямых переправ на Гарда Викка, и это в какой-то степени помогало держаться от войн подальше.
Но теперь, благодаря Башне Дьявола, в поле сражения может превратиться и этот островок спокойствия.
Как только о мосте узнают соседние Друзы --- золотые леса будут кишеть отрядами авантюристов, идущих по направлению к Гарда Викка.

Существовала возможность взорвать мост.
Это сложно, но осуществимо.
Выдолбить несколько ниш, заложить взрывчатку и молиться, чтобы инженерия победила первобытную мощь кристалла.
Но тех, кому выгоден мост, чересчур много.
Это не только авантюристы, но и торговцы, и обычные разбойники, и мелкие правители.
Все они попытаются наложить на мост лапу как можно скорее.
И если отряды с Гарда Викка уже на Башне Дьявола, задача усложняется в разы.
Про Войны Мостов рассказывали всем детям без исключения, это самые кровавые страницы в истории Тысячи Башен.
Если же враг сумеет фортифицировать подходы по эту сторону моста --- пиши пропало.

Едва Курц об этом подумала, как до неё долетели чьи-то голоса.
Она обратилась в слух.
Говорили двое мужчин.
Их непонятная мяукающая речь перемежалась вполне знакомыми обсценными словами.

Фоуф-у, язык наёмников.

Курц соскользнула в ложбину и побежала.
Она больше не думала о том, что Башня может раздавить её.
Надо предупредить город любой ценой.

\section{Налётчики}

--- А кто это у нас тут, неужели женщина? --- мужчина опустил и наклонил голову --- поза охотящегося медведя.
Курц знала, что после этого последует.

--- Ты уверен? --- засомневался второй.
--- Какая-то она странная.
Почему лицо закрыто?
Это у них обычай такой?
Подай-ка голос, красавица.

Курц молча попятилась.

--- А куда это мы собрались? --- первый откровенно глумился, широко расставив руки и закрыв собой выход из пещеры.

--- Слушай, оставь её, --- поморщился второй.
--- От неё ещё и воняет как от падали.
Даже за деньги в такое не засуну...

--- Да, верно, --- согласился первый, понюхав воздух.
--- Очень жаль, фруктик с гнильцой...
Умри, навозная куча.

Движения Курц были выверены до мелочей.
Она отшатнулась, словно от неожиданности, пропустив вражеский клинок слева.
Имитация неуклюжего везения, вводящая противника в заблуждение.
Затем быстрым движением справа она перерезала врагу горло.

Налётчик взмахнул руками, словно пытаясь поймать и вернуть обратно фонтан крови, брызнувший из его шеи.
В его глазах застыло удивление.

<<На Хербст тебе не рады.
Я лишь вложила это чувство в движение клинка>>.

Второй неожиданно легко увернулся от серии выпадов.
Курц в жизни не видела, чтобы кто-то двигался с такой скоростью.
Он присел, перекатился, оттолкнулся от стены и подпрыгнул, целясь ногой в голову противницы.
Курц отклонилась... и тут же получила от висящего в воздухе мужчины удар второй ногой.

<<Обманный манёвр! --- успела подумать Курц.
--- Вот это сила...>>

В следующий миг она грохнулась спиной о камень и сползла вниз.
Дыхание перехватило, к горлу подступила паника, Курц не могла сделать ни малейшего вдоха.
Инстинкт выживания заставил её вскинуть руку и нажать на спусковой крючок.
Лёгкий скалолазный гарпун вошёл точно в глаз бросившегося на неё врага.
Звякнул о камень шипастый кастет, зажатый в жилистой руке;
опустевшее тяжёлое тело свалилось прямо на Курц, чувствительно ударив её ещё и в пах.

Вдохнуть наконец удалось.
Сначала маленькими глотками, вскоре Курц раздышалась совсем.
Но встать она не могла ещё долго, потрясённо рассматривая ещё тёплый труп человека, который едва не забрал её жизнь.

<<Их было всего двое, --- с ужасом думала она.
--- Сколько их там ещё на мосту?..>>

\section{Огнестрел}

Снаружи пещеры раздались тяжёлые шаги, и Курц притворилась мёртвой.
Сквозь полуприкрытое веко она увидела, как в пещеру нетвёрдой походкой зашёл незнакомый человек и привалился к стене, пытаясь устоять на ногах.
Вслед за ним появился Шнелль и вонзил нож незнакомцу в спину, оборвав его мучения.
Увидев Курц, он охнул и подбежал к ней.

--- Фух, ты жива, --- сказал он, вытирая пот со лба.
--- Я боялся, мне придётся докладывать Ангаре, что я ушёл в самоволку, и разведка прошла неудачно.

--- Да, всё хорошо.
Как у тебя дела?

--- Минус три, --- отрапортовал Шнелль, вытирая лезвие ножа.
--- Ходят пятёрками.
Ещё одна группа скаутов прошла на юг, в сторону города.
Ребята очень жёсткие, лучше спрятаться, чем драться.
Впрочем, ты и сама поняла, как я вижу...

--- Лучше бы я спряталась, --- согласилась Курц.
--- Давай-ка к козлам, и побыстрее.

--- Дай мне минуту, --- Шнелль начал вытряхивать из сумки бинты и жгуты.

--- Шнелль, ты ранен?

--- Самую малость.

--- Ложись, я осмотрю.

--- Курц, меня немного крутит, так что если я лягу --- я уже не встану.

--- Хватит нести чушь.
Подними рубаху.

Курц аккуратно брызнула на рану водой из фляжки и нахмурила брови.

--- Это огнестрел, --- наконец выдала она.

--- Ты уверена?
Мне показалось, это был пружинный самострел, и стрела срикошетила...

--- Он хлопнул?

--- Скорее фыркнул, --- задумался Шнелль.
--- Вот такой маленький, размером с ладошку.
Улетел в расщелину, когда я этого гада порезал...

--- Этот <<маленький, размером с ладошку>> понаделал в тебе пять дырок одним залпом.
Дьявол...

Глаза Шнелля расширились.
Он настойчиво скашивал глаза, пытаясь разглядеть рану.

--- Курц, ты же не дашь мне сдохнуть здесь, правда?

--- Дыши мелкими глотками, пожалуйста, и заткнись.
Одна пулька застряла в ребре.

--- Ай!
Чёрт бы тебя побрал, Курц!

--- Да всё уже, вытащила, --- Курц критически осмотрела окровавленный шарик.
--- Какой-то мягкий металл, точнее сказать сложно.
Он расплющился от удара о кость.
Сиди, я приведу козлов.
До баронства не поедем, попросим помощи в ближайшей деревне.

Курц положила пульку в карман и выбежала из пещеры.

\section{Отравление}

--- Как ты себя чувствуешь?
Эй, Шнелль!

--- Живот, --- сдавленно сказал Шнелль.
Его взгляд замер, белки глаз покраснели, с уголка рта безвольно сочилась слюна.
В следующую секунду его вырвало с кровью.

\section{Через силу}

--- Дьявол, --- ругалась сквозь зубы Курц.
--- Дьявол...

Солнце уже давно зашло за Гребень Троллей, и в лесу быстро темнело.
Курц стянула со Шнелля испачканные штаны и обмыла его в озерце, аккуратно придерживая за голову.
Он был в спутанном сознании уже несколько часов.
Вода с углём помогла, но ненадолго;
на всякий случай Курц промыла и засыпала углём ещё и рану.

Ясное дело, пули были отравлены.
Когда Шнелль немного успокоился, Курц рассмотрела вытащенный ею снаряд.
Внутри оказалась полость.

<<Четыре пули всё ещё там>>.
Курц чертовски хотелось искупаться, поесть и поспать.
Козлы тоже клевали носом.
Она наскоро прополоскала мехи с мылом в озерце --- чистые закончились, оставшиеся не было времени обработать как следует, от них смердило.
Как назло, подходили ещё и месячные, внизу живота разливалась знакомая тянущая боль.
Едва прожевав половинку бутерброда с грудинкой, Курц посадила безвольное тело Шнелля на козла и повела упирающееся животное в темноту.

\section{Ртутная сердцевина}

--- Это отравление ртутью, --- сказала женщина.
--- В твоей пульке были следы ртути.
К счастью, лопнула только одна пулька из пяти, и ты вытащила её почти сразу.
Если бы не ты --- участь парня была бы предрешена.

\section{Единственный, кто вернулся}

Ни один разведчик не вернулся.

\section{Висячие руины}

--- А что там?

--- Висяше руины, нэй, --- отозвался проводник с сильным саркортским акцентом.
--- Очень давно никто не живёт.

--- Я ни разу не видела висячие руины, --- задумалась Анкарьяль.
--- А ты, Курц?

--- Я даже не слышала, чтобы на Хербст были висячие руины, --- пожала плечами Курц.
--- Может, сходим?
Склон не выглядит сложным для восхождения.
Арслан, что скажешь?
Ты там был?

--- По западной кромка иди, нэй, хороше дорога, надёжнэ, --- улыбнулся проводник.
--- Был давно, красиво там.
Подарки жене принёс, нэй.

Анкарьяль кивнула и обратилась к легионеру:

--- Три кислородных баллона, запас на десять часов каждый.
Ещё ручной огнемёт и мои хуки.

\section{Технология}

<<Так, а это что? --- вдруг услышала Курц.
--- Иди сюда срочно!
Здесь какая-то технология!>>

Курц подошла и вгляделась в заснеженные камни.
Нет, это были не камни.
Металлический блеск, следы заклёпок...

<<Отойди, расчищу>>.
Анкарьяль вскинула огнемёт.

\section{Оцифровка}

<<Что ты делаешь?>> --- Курц уже научилась распознавать этот странный взгляд в себя и замершую позу.

<<Оцифровка аппарата>>.

<<Зачем?>>

<<Моделировать, изучить, воспроизвести, --- объяснила Анкарьяль.
--- Первое, что нужно сделать с незнакомым устройством --- оцифровать.
Это превосходящая технология первых людей.
Очень странно, что они находятся в стабильном состоянии уже много тысячелетий...>>

<<Что находится в стабильном состоянии?>>

<<Ты не поверишь, но внутри люди>>.

<<Люди?!>>

<<Это капсулы анабиоза, двенадцать штук.
Целостность восьми точно нарушена, внутри снег и конденсат.
Ещё четыре как будто в порядке...
При этом никакого питания, технически система мертва.
Такого я точно не встречала и даже не слышала.
Подкопайся с того боку, командная панель там>>.

<<Что такое анабиоз?>>

<<Заморозка.
Вечный сон.
Помогала людям преодолевать космос, сохраняя молодость>>.

<<Так, может, те четверо ещё живы?!>>

<<Я бы на это не рассчитывала, но ради интереса хочу заглянуть внутрь>>.

Курц перехватила огнемёт и прыгнула в расщелину.

<<Бесполезно, --- фыркнула Анкарьяль.
--- Без питания не заработает.
Схема в целом понятна, но у меня только базовые знания по технологии такого уровня, так что без внятной инструкции...
Курц, спустись вниз за генератором>>.

<<Анкарьяль, тут сбоку табличка!>>

<<А, это табличка?
На омега-сканировании мне показалось, что какая-то заплатка, она прикрывает дефект кожуха>>.

<<Буквы знакомые, это латиница.
А, дьявол...
Надпись почти стёрлась>>.

<<Не беда.
Мой демон считает.
Таааак...
Энглис, Древняя Земля, ранняя Эпоха Богов.
Ясно, ясно.
Расшифровалось не всё, но генератор точно можешь не тащить, его мощности не хватит.
Как же они запитывали эту махину?>>

<<Тут какие-то провода!
Толстые, как моя рука!
Ощущение, что оторвались...
Нет, их обрезали>>.

<<Разморозь их, пожалуйста, насколько можно, --- попросила Анкарьяль.
--- Сколько у нас ещё кислорода в основных?>>

<<Пять часов работы и обратный путь>>.

<<Отлично, успеем>>.

\section{Риск}

<<Длины хватает впритык, --- пожаловалась Курц, дёргая кабель.
--- Запас в полпальца.
Я даже на знаю, как их соединить>>.

<<Не волнуйся, я сама всё сделаю.
К генератору я могу подобраться только демоном, он глубоко подо льдом, но в целом он рабочий, в нём ядерное топливо.
Сразу скажу --- я не технолог, и если что-то пойдёт не так и он рванёт, нам однозначно конец>>.

<<Давай уже.
Мне так любопытно, что я готова рискнуть>>.

<<Так и знала, --- ухмыльнулась Анкарьяль.
--- Нарежь изоляции, пожалуйста, мы прикроем провода от снега...>>

\section{Драгоценный воздух}

<<Фуууух, заработало, --- Анкарьяль вытерла со лба холодный пот.
--- Слышишь гул?
Пошли к капсулам>>.

<<Анкарьяль, стой.
Если люди живы, им нужно будет дышать!>>

<<Вот дьявол, я не подумала...
Запасные маски есть?>>

<<Только две!>>

<<Будем дышать по очереди>>.

<<А на склоне?>>

<<Не занудствуй, Курц.
Ставлю девяносто девять из ста, что они все мертвы.
Но если что, просто спустишься за дополнительными баллонами.
Я запускаю разморозку?>>

<<Давай по очереди>>.

\section{Красный Крест}

<<Этот мёртв, --- констатировала Анкарьяль.
--- Совсем маленький, года четыре.
Видимо, капсула всё-таки была повреждена>>.

<<Анкарьяль, во второй кто-то шевелится!>>

<<Открывай, аккуратно... Ах ты дьявол!>>

<<Он задыхается!
Маску, маску!>>

<<Он не от воздуха задыхается! --- остановила её Анкарьяль.
--- Смотри!>>

Мальчик вдруг кашлянул, извергнув брызги красноватой пены.
Из его груди и живота торчали блестящие, острые как бритва осколки толстого стекла.
По белой, как снег, одежде медленно расплывались кровавые пятна.

<<Что мне делать, Анкарьяль?>>

<<Ничего, --- отозвалась демоница.
--- Он сейчас умрёт.
Смотри --- на крышке был налеплен красный крест, наклейка выцвела, но различить ещё можно.
Это распространённый у древних знак врачей.
Мальчишку не зря в капсулу положили.
Видимо, рассчитывали, что смогут вылечить через какое-то время...>>

Несколько секунд --- и мальчик затих навсегда.
Курц с ужасом наблюдала, как из небесно-голубых глаз медленно утекает вернувшееся было сознание.

<<Пошли к третьей>>.

<<Нет! --- Курц схватила её за рукав.
--- А вдруг там тоже тяжелораненые?>>

<<На этих крестов нет>>.

<<Забыли нарисовать, стёрлись, мало ли что произошло за это время!>>

<<Ты предлагаешь им подождать до конца света? --- осведомилась Анкарьяль.
--- Они всё равно умрут, Курц!
Никто, кроме нас, в ближайшие тысячи лет не сможет их разморозить!>>

<<Ладно, --- грустно сказала Курц.
--- Давай следующую.
Но маску я на него всё-таки одену.
Теперь-то их точно хватит...>>

<<На неё, --- поправила Анкарьяль.
--- Здесь лежит девочка>>.

\section{Холод}

<<Дьявол, мы ничего тёплого-то и не взяли, --- бормотала Курц.
--- Не знали, что вас найдём.
Дышите, дышите, маленькие.
Всё хорошо...>>

<<Они тебя не понимают>>, --- заметила Анкарьяль.

<<Неважно, --- махнула рукой Курц.
--- Главное --- интонация>>.

Она погладила по головам жмущихся к ней детей.
Вначале мальчик и девочка отстранялись, в страхе разглядывая изуродованное лицо женщины.
Потом холод взял своё.
Анкарьяль и Курц пожертвовали свои куртки, но ребятишек всё равно колотило на пронизывающем ветру.

<<Они замёрзнут насмерть, если мы сейчас их не спустим, --- констатировала Курц.
--- Кто вообще придумал одеть детей в эти тонкие холодные костюмы?>>

<<Здесь были другие высоты и другой климат тысячелетия назад, --- пояснила Анкарьяль.
--- Вполне возможно, что тогда их костюмы были по погоде>>.

<<Мальчишка ещё и ранен, его надо полечить.
Я физически сильнее, возьму его.
А ты бери девчонку, закутай во что-нибудь, примотай к спине и пошли.
Чем быстрее, тем лучше>>.

<<Да нам самим бы не околеть по пути>>, --- буркнула Анкарьяль и сняла с пояса верёвку.

\section{Добрые намерения}

Курц сидела и пила чай из жестяного термоса.
В палатку вошла Анкарьяль и устало опустилась на груду спальников.

--- Бесполезно, --- выдохнула она.
\ml{$0$}
{--- Детишки не понимают англисский, чайнис, эллатино...}
{``The children don't speak Englis, Chainis, Ellatino ....}
Ни один из языков Древней Земли, о котором сохранились какие-то данные.
Может, они вообще не оттуда?

\ml{$0$}
{--- Но табличка-то на ангельском, --- заметила Курц.}
{``But that plate is written in Engels,'' Kurz noted.}

\ml{$0$}
{--- Разумно, --- признала Анкарьяль.}
{``Makes sense,'' Angaralle admitted.}
--- Ладно, неважно.
Выучат со временем наши языки.

--- А сейчас с ними кто?

\ml{$0$}
{--- Арслан.}
{``Arslan.}
\ml{$0$}
{Пытается разговаривать на саркорте.}
{Trying to speak Sarqort to them.}
\ml{$0$}
{Тоже безрезультатно, но у парня явно самые добрые намерения, и дети к нему тянутся.}
{Result is the same, but the guy apparently has the best intentions for them, and the children reciprocate.}
\ml{$0$}
{Едят пока что только из его рук.}
{Thus far, they take food from his hands only.''}

--- Пойду-ка и я с ними поговорю, --- Курц встала на ноги и надела маску.

\section{Этикетка}

Вдруг внимание Курц привлекла этикетка на одежде мальчика.
Она протянула к ней руку.
Мальчик тут же вскочил и выставил вперёд нож.

--- Всё в порядке, всё в порядке, --- ласково сказала Курц.
--- Я только посмотрю...

Она медленно подошла к мальчишке, словно к пойманному в капкан волчонку, говоря самые нежные слова, которые только могла вспомнить.
Этих слов не слышал ни её давно оставшийся в прошлом любовник, ни один из её учеников, ни мама.
Наконец мальчик сдался и опустил оружие.
Курц обняла его и аккуратно развернула этикетку.
На этикетке стояли три совершенно понятных слова:

\ml{$0$}
{<<Сделано в Восточной Стране>>}
{\textsc{Made in Eastern Realm}}

\section{Фонетический дрейф}

--- Я не понимаю! --- без предисловий выдала Курц, зайдя в палатку Анкарьяль.
--- Не понимаю!

--- Судя по твоему возбуждённому тону, это качественно другое непонимание по сравнению с тем, что мы имели ранее, --- весело откликнулась Анкарьяль, отложив в сторону записи.
--- Рассказывай, что накопала.

Курц вкратце рассказала об этикетке.

--- Я попробовала написать ему сообщение на бумажке, и он его прочитал! --- развела руками Курц.
--- Но сказанного по-прежнему не понимает.
Как такое может быть?

--- Ааа, --- на лице Анкарьяль проступило понимание.
--- Фонетический дрейф.

--- Что это такое?

--- Ваша письменность всё это время была почти неизменной, --- пояснила Анкарьяль.
--- Вас учили на старых книгах.
Грамматика тоже если и изменилась, то незначительно.
А вот фонетика нигде не сохранялась.
Если проще --- вы не знаете, как именно произносили буквы ваши предки.
Если между вами несколько тысяч лет --- фонетика может измениться до неузнаваемости.

--- Это очень странно.
Почему, например, фонетика не изменилась с другими Друзами?
Мы тоже разделены тысячелетиями!

--- Во-первых, всё-таки изменилась.
Вспомни --- даже между жителями разных городов есть разница в говорах.
Во-вторых, вы не совсем разделены, есть люди, которые путешествуют между Друзами.
И в-третьих, фонетический дрейф ускоряется в сотни раз при перелёте на другие планеты.
Причина какая-то чисто биологическая, я точно тебе не скажу, так как не специалист в физиолингвистике.

--- И как долго длится ускорение?

--- Обычно первые три-четыре поколения, максимум до десяти.
Чаще всего уже правнуки первых поселенцев произносят звуки совсем не так, как их прадеды.

--- Получается, эти дети... первые люди, которые прибыли сюда?

--- Да.
И если судить по языку, это твои прямые предки.

Курц кивнула.

--- Пойдём-ка, --- сказала Анкарьяль.
--- Напиши какие-нибудь стихи на бумажке и попроси его прочитать.

\section{Раскодирование}

Курц наблюдала за Анкарьяль.
Та снова замерла с отсутствующим выражением лица, полуприкрыв глаза.
Её зрачки плавали, как у спящего.
Едва мальчик дочитал стихи до конца, баронин очнулась.

--- Отлично, --- удовлетворённо сказала она.
--- Мой демон раскодировал фонетику.
Сейчас поболтаем...

\section{Ашита}

--- Легионер, --- подозвала Курц.
--- Как ваше имя?

--- Ашита из клана Дорге, --- поклонился легионер.

--- У вас необычное имя, --- сказала Курц.
--- Вам много лет?

--- Я был создан ещё во времена первого Ордена Преисподней, --- кивнул Ашита.

--- Удивительно, что вы до сих пор в ранге легионера.

--- Я не желаю для себя другой роли, --- с достоинством сказал Ашита.

--- Я приношу извинения, если я случайно вас обидела.

--- Обиды нет.

--- У вас с Анкарьяль ведь существует, кхм, связь?
Я хочу знать, о чём они говорят.

--- Легат переслала мне раскодированную фонетику.
Код успешно интегрирован в мой биологический нейроконтур.

--- Можно тогда вас попросить перевести?

--- Легат разрешает, --- кивнул Ашита и приник к уху Курц.

\section{Демиург}

--- С кем вы сражались? --- спросила Анкарьяль.

Мальчик ответил.
Анкарьяль странно усмехнулась и промолчала.
Ашита тоже не спешил переводить.

--- Что он сказал? --- Курц подёргала Ашиту за рукав.

--- Они сражались с демиургом планеты, --- ответила за подчинённого Анкарьяль.
--- Они называют его Демоном.

--- А мы называем его Дорге Основателем, --- в словах Ашиты промелькнуло благоговение, словно перед уважаемым предком.

--- Я слышала в его голосе вопрос, --- настаивала Курц.
--- Мальчик о чём-то спросил.

--- Он спросил, кто в итоге победил, --- Анкарьяль смущённо потёрла лоб.
--- Я даже не знаю, что ответить.

Вдруг небо озарила яркая вспышка.
Облака засияли белым светом.

--- Что за дьявол? --- вскочила Курц.
--- Это же не молния?

В следующую секунду громыхнуло так, что чайник слетел с подставки и с шипением упал в костёр.

--- Упс, --- смущённо выдавила Анкарьяль.
--- Похоже, тот генератор всё-таки рванул.
Неудобно вышло.
Арслан, извини, но тебе придётся возвращаться в обход.
Я оплачу сверх оговорённого.
Скажи своим, чтобы ближайшие пятьдесят-сто лет там не ходили.
И пошли-ка отсюда побыстрее, пока мы не нахватались рад...

Проводник покачал головой и, буркнув <<Женщины>>, начал разбирать палатку.

\section{Отчёт}

--- Что ты делаешь?

--- Пишу отчёт, --- ответила Анкарьяль.
--- Не мешай.

--- А разве ты не можешь просто писать его, ну, в голове?

--- Могу.
Но это не очень хорошо для мозга.
Писать и рисовать руками очень полезно --- это помогает сохранять синхронизацию демона и тела.
А бумага прекрасно горит --- когда я закончу, я просто брошу рукопись в костёр.
В моём демоне останется копия.

Курц подняла один лист и нахмурилась.

--- Это рассказ Михаэля? --- спросила она.

--- Да.

--- Тут так много подробностей...

--- Лёгкий гипноз --- и я вытянула из парня чуть больше, чем доступно его пониманию.
Полезные исторические сведения из первых рук.
Когда Орден Преисподней только создавался, была осознана важность любой информации и научного исследования.
\ml{$0$}
{Каждый военный в Ордене --- это в первую очередь учёный.}
{Every military in the Order is a scientist in the first place.}
\ml{$0$}
{Если же нет --- он военный не ахти.}
{Otherwise, they're not much of a military.''}

\ml{$0$}
{--- Ты не против, если я почитаю?}
{``Would you mind if I read?''}

--- Конечно, читай.
Никаких военных тайн здесь нет.
Можешь взять всё, что я уже написала.

Курц аккуратно подняла стопку исписанных листов, села в кресло и углубилась в чтение.

\section{Обманутые ожидания}

--- А почему ты отправилась, Лида?

--- Это был побег от проблем на Земле, --- сказала она.
--- Как только люди узнавали, что я внучка Коханого --- они ожидали от меня чего-то выдающегося.
Я не хочу, чтобы от меня что-то ожидали.

--- Я никому не скажу, --- кивнул Йокудль.

--- А что ты здесь делаешь, Йокудль?

--- Кораблю нужен навигатор, --- сказал мужчина.
--- Я вызвался потому, что понял --- если не соглашусь я, не согласится никто.
Признай --- эта экспедиция изначально была ошибкой.

--- Даже не буду спорить.
Лететь с двумя циклами к недотерраформированной планете --- что может пойти не так?

\asterism

--- Терраформирование прошло не по плану, --- сказала Лида.
--- Она другая... она совсем другая, ничего о том, что говорил Дорге!
Откуда эти кольца и луны?
Что с поверхностью?
Она как будто растрескалась...

--- Мам, я боюсь, --- сказал Михаэль.
--- С нами ведь всё будет хорошо?

--- Да, конечно, золотко.
Иди ко мне.

--- Йокудль, вызови демиурга, --- сказал Пауль.
--- Нам нужна связь с Землёй.
Я хочу нормальное объяснение происходящего.

--- На Земле уже должно было пройти около пятисот лет.

--- И что?
Кто-то же у них должен быть на связи.
Надеюсь, коды доступа, которые нам дал Жерар, всё ещё действуют...

\asterism

--- С соединением что-то не так, --- пробормотал Йокудль.
--- Такое ощущение, что омега-модуль связи демиурга повреждён.
Я не могу локализовать его.
Но сигнал идёт.

\asterism

--- Земля, говорит <<Первая ночь>>.
Повторяю, говорит <<Первая ночь>>.

--- <<Первая ночь>>, кто говорит?

--- Говорит Йокудль Гуннарссон Глазенапп, навигатор корабля.

--- <<Первая ночь>>, вас понял.
К сожалению, у нас всего один специалист по языкам вашей эпохи, и мы ничем не можем вам помочь.
Перенаправляю сигнал в Исследовательский центр <<Стармист>>, Торонто.
Согласно нашим данным, вашей экспедицией сейчас занимаются они.
Оставайтесь на связи.

--- <<Стармист>> на связи, <<Первая ночь>>.
Рады вас слышать.
Говорит Ким 40-А.

--- <<Стармист>>, простите, но я не совсем расслышал, как к вам обращаться.

--- Ким 40-А.
Я хунев, кибернетический организм на основе культуры человеческих нейронов.
Можете звать меня просто 40.

--- Очень приятно, 40.
Я впервые слышу о хунев.
Похоже, на Земле происходит что-то интересное?

--- Я понимаю ваше удивление, Йокудль.
Вам было бы привычнее слышать немодифицированных людей, но, к сожалению, в Торонто сейчас ночь, и все люди спят.
У хунев чувствительные глаза, поэтому мы предпочитаем ночное время.

--- Я очень рад говорить именно с вами, 40.

--- Для меня тоже большая честь, Йокудль.
Я много читали о вашей команде.
Всё ли в порядке с кораблём?
Насколько стабильна связь?

--- С кораблём всё отлично, 40.
Со связью есть некоторые проблемы, поэтому будем кратки.
Мы увидели не то, что ожидали.

--- Понимаю ваше беспокойство, Йокудль.
Буду кратки.
Планета пригодна для жизни, но пригодна не вся.
Как вы заметили, поверхность разделена на континенты, которые мы называем Друзами.
То, что между ними --- Трещины, содержащие воду и вулканические газы.
На семидесяти Друзах пригодный для жизни людей климат, есть источники пищи и питьевой воды.
Для посадки могу порекомендовать три из них, относительно спокойных сейсмически и с мягким умеренным климатом.
Трещины для жизни непригодны.
Более подробно можете узнать в справочнике, который я вам выслали.
Просмотрите его так быстро, как сможете, и сообщите нам, потому что справочник составлялся примерно через двести лет после вашего отлёта, и кое-какие вещи могут быть для вас непонятны.

--- Вас понял, 40.
Премного благодарны!
Лида, Пауль, займитесь справочником.

--- Рады помочь, Йокудль.
У вас есть какие-то срочные вопросы для обсуждения?

--- 40, можете подробнее рассказать, что произошло с планетой?

--- Вас поняли, Йокудль.
Ошибка в коде демиурга.
Он нефункционален и не поддаётся контролю с Земли.
Терраформеры едва смогли довести планету до относительно приличного вида с помощью нескольких дополнительных богов, но демиург всё равно своевольничает.
Странная форма планеты --- это его работа.

--- Не могли бы вы рассказать поподробнее про ошибку в коде?

--- К сожалению, не могу.
Мы потеряли большую часть данных по взаимодействию с демиургом по нелепой случайности --- в Бремене сгорел датацентр Бременского Университета.
После этого два проекта по терраформированию, в том числе ваш, достались Институту <<Стармист>>, и ещё два --- Лаборатории Кошкина в Калькутте, так как Бремен больше не имел базы для дальнейшей работы.
Это всё произошло задолго до моего рождения.
Жерар Дорге был настолько потрясён этим несчастьем, что умер от инфаркта на следующий день.
Ему было всего 94 года.

--- 40, прискорбно слышать, что его жизнь оборвалась так трагически.
Я сообщу его друзьям.
Скажите, представляет ли демиург опасность сейчас?

--- Мы около ста лет не замечали следов его активности, так что, предположительно, он израсходовал батарею и заснул.
Если модуль связи всё ещё функционирует, ваши программисты могут узнать поподробнее.
Нам отсюда чересчур энергозатратно лезть в его системы.

--- Я боюсь, что колонисты не захотят заселять планету.

--- Мы это предусмотрели, Йокудль.
Слишком много вещей пошло не так, и приземляться мы вам советуем только после тщательнейшего анализа ситуации.
В справочнике есть маршрутная карта для обратного пути.
Мы позаботимся о том, чтобы вас встретили.

--- Благодарим вас за добрые слова, 40.
На Земле всё спокойно?

--- Беспокоиться не о чем, Йокудль.
Как вы уже поняли, здесь много интересного и нового для вас.
Надеюсь, что и ближайшие 500 лет будут такими же.
Возможно, я даже доживу до встречи с вами.

\asterism

Наконец Пауль выдал:

--- Всем занять свои места и приготовиться к запуску двигателей.
Мы возвращаемся на Землю.

\asterism

--- Да что такое?

--- Что?

--- Двигатель не запускается!
Топливо...
Утечка топлива!

--- Быстро в технические капсулы!
Надо заделать пробоину, пока...

--- Капитан, мы уже потеряли тридцать процентов.

--- Сколько?!

Все потрясённо замолчали.
И в полной тишине прозвучал голос Пауля:

--- Приготовиться к посадке.
Цель --- Друза номер шесть.

\asterism

--- Башни, башни, башни... --- Пауль вглядывался в туманную даль.
--- Одни башни.
Тысячи башен.
Это точно самые пригодные для жизни регионы?
Я хочу горы, но тут даже горы выглядят как нагромождение упавших столбов.
Выглядит ужасно.

--- Не говори так, --- оборвала его Лида.
--- Эта планета будет домом для нас и для Михаэля.
Про неё будут сочинять стихи, с неё будут рисовать картины.
Мы должны первыми увидеть в этой планете красоту, а до той поры хотя бы её не обижать.

--- Я уже её ненавижу, --- пробормотал Пауль.
--- Будь моя воля...

--- Пауль, перестань.
Я понимаю твои чувства, но держи свою ненависть при себе, пожалуйста.
Не надо подавать детям плохой пример.

\asterism

--- Дядя Йокудль, что ты здесь делаешь?

--- Читаю, Михаэль.

--- Ты ещё не прочитал все книги, которые нам дали в путешествие?

--- Это новая.
Помнишь того... ту персону, с которой мы разговаривали?

--- Странное существо из человеческих нейронов?
У него ещё имя --- циферка.

--- Да, 40.
Их зовут 40.
Они прислали нам кроме справочника ещё книжек.
Историю, новости, научные труды.
И ведь как будто специально собирали эту библиотеку --- с любовью, вдумчиво.
Знали, что нам захочется узнать, и выбирали лучшее.

--- И что нового ты узнал?

--- Я как раз читал про хунев.
40 --- это киборг, хунев.
Оказывается, за время нашего полёта люди создали несколько видов разумных существ.
Помнишь собачек?
Я тебе показывал собачек на картинках.

--- Помню, дядя Йокудль.

--- Из них создали прямоходящих разумных существ --- кани.
Люди воспитали кани в своём обществе, и кани стали равными людям.

--- Ты говорил про хунев.

--- Это была предыстория, Михаэль.
Люди сразу признали кани за равных, потому что кани были живыми существами из плоти и крови, почти как люди.
А история в том, что хунев изначально были игрушками --- няньками, поварами, прислугой.
Они целых пятьдесят лет боролись за свои базовые права --- за право на существование, неприкосновенность, независимость.

--- Это несправедливо!

--- Несправедливо, --- согласился Йокудль.
--- И мне вначале было не совсем понятно, почему для меня эта несправедливость очевидна, а для тех, кто годился мне в правнуки --- нет.
Ведь давно известно, что с каждым поколением люди становятся лучше, добрее, честнее, сильнее.
Но, кажется, теперь, глядя на нас, я понимаю, в чём дело.
Когда я говорил с 40, я видел лишь равного мне собеседника.
Я не знал, боролись ли они с дискриминацией с самого появления на свет, или же росли как равный член общества.
Мимо меня прошёл весь исторический контекст --- появление хунев, рост их самосознания, сомнения и протест консервативно мыслящих людей, борьба, трагедии, победы.
Я получил собеседника-хунев на блюдечке и принял его таким, какой он есть.

Йокудль засунул электронную книгу в сумку и тяжело поднялся на ноги.

--- 40 пообещал, что будет нас ждать, --- сказал он.
--- И какое-то время я грустил, что мы уже точно не встретимся.
Но сейчас я думаю, что мы просто были недостойны вернуться на Землю.
Все бури из этих с любовью подобранных книжек прошли мимо нас.
Мы --- дезертиры, отголосок давно забытой истории.
Мы устарели.
И наше место --- там, где мы есть, под другим Солнцем, на другой Земле.

--- Вы герои! --- возмущённо сказал мальчик.

Йокудль засмеялся и погладил Михаэля по голове.

--- Что ещё может сделать дезертир?
Придумать себе героическую историю и поверить в неё.
Я много разговаривал с командой, Михаэль.
У всех, кто здесь, так или иначе не сложилась жизнь на Земле.
Все счастливые остались там.

Йокудль вздохнул.

--- Самое обидное --- ты, например, ещё можешь свалить всё на родителей, которые по молодости и дурости ломанулись на край Вселенной.
А мне, старому дураку, даже и винить некого.
На что я рассчитывал, когда записывался в команду?
Что я какой-то уникальный, что я незаменим?
Придумал какой-то долг, какую-то судьбу, какое-то предназначение.
Дураки --- это универсальный и бесконечный расходник истории: один струсит, обязательно появится второй.

--- Если бы ты не записался в команду, ты бы ничего этого не понял и жалел бы всю жизнь, --- сказал Михаэль.
--- Я же тебя знаю, дядя Йокудль.

--- Aus dem Munde der Unmündigen und Säuglinge, --- хихикнул Йокудль и, завернувшись в шарф, пошёл в сторону лагеря.

\asterism

--- Она была в полном порядке, --- сказала Лида.
--- Есть только одна возможность.

--- Какая?

--- Демиург.

--- Хватит фантазировать, Лид, --- отмахнулся Пауль.

--- У тебя есть другие версии, почему диагностически полностью исправный двигатель дал такую огромную пробоину?

--- Я согласен с Лидой, --- тихо сказал Йокудль.
--- Демиург не выходит на связь с самого момента аварии.
Возможно, попыткой связаться с Землёй мы вывели его из режима ожидания.

Пауль потёр голову.

--- Разбудите программистов и настройте омега-связь.
Мне нужно знать, где находится демиург и в каком он состоянии.

\asterism

--- Он выводит из строя одно устройство за другим, но не трогает нас, --- Пауль поморщился.
--- Что это вообще значит?
Зачем ему портить наши машины?

\asterism

--- Провода загорелись!

--- Огнетушители?

--- Ни одного!

--- Забрасывай землёй!
Быстро!

--- Бесполезно.
Лида, их надо перерезать.
Капсулы стабильны даже без питания.
Если обесточить систему, Демон не станет выводить её из строя!

--- Знаю!
Михаэль, слушай меня внимательно!
Я тебя вытащу!
Я тебя обязательно вытащу!
Ложись быстро!

--- Мама!

--- Я люблю тебя, Михаэль!

Лида поцеловала сына в лоб, затолкнула его в капсулу и закрыла крышку.
Последнее, что он видел --- это Пауль, который занёс плазменный резак над горящим кабелем.

\section{Слёзы}

Курц бросила бумаги на стол.
Её маска и шарф были мокрыми от слёз.
Она украдкой вытерла слёзы и бросила взгляд на Анкарьяль.
Та как будто ничего не...

--- Можешь не стесняться, --- вдруг сказала демоница, не отрываясь от записей.
--- Я бы тоже поплакала, если бы не знала истории похуже этой.

--- Они не смогли его вытащить?

--- Или не успели.
Согласно историческим хроникам, первые поселенцы укрылись в лесах Хербст.
С его родителями могло произойти что угодно.
Обратный переход был труден, у людей было много забот на новом месте.
Наверняка у них не было медицинского оборудования, чтобы спасти тяжелораненных.
Вполне возможно, они так и умерли, лелея мечту о том, что однажды смогут починить систему и воссоединиться с родными...

--- Там был всего один обрезанный кабель.

--- Именно, --- подтвердила Анкарьяль.
--- Такая мелочь, и такая непреодолимая преграда...
Наверное, это было мучительно осознавать.

--- История Михаэля занимала всего десять листов.
Что ты пишешь в остальных?
Я не знаю этот язык.

--- Мои комментарии.
Исторический контекст, какие-то психофизиологические аспекты.
Для тебя это точно не интересно, но для учёных будет полезно.

Курц вдруг охватило странное чувство уюта.
Анкарьяль сидела, склонившись над столом, и писала --- маленький островок тепла, очерченный походной лампой.
Её золотистые прямые волосы матово светились в ламповом свете, косой срез оставлял открытой белую шею.
Полосатая мягкая кофта из шерсти ламы была ей чуть велика, подбородок и щека Анкарьяль утопали в толстом пушистом воротнике.
Пол шатра был покрыт зелёной травой вместо надоевшего утоптанного снега.
Рука Курц гладила траву и всё не могла насладиться этой прохладной зеленью.
Как же не хочется идти к своей палатке...

Не прошло и пяти минут, как Курц Штайгер крепко спала, завернувшись в два спальных мешка.

\section{Чужая микрофлора}

Лоб мальчишки был горячим.
Губы пересохли и растрескались.

--- Нэй, рана плохо выглядит, --- тихо сказал Арслан.
--- Грязь попала, нэй.
Я его травой напоил --- не такой горячэ.
Врача нужен, нэй.

--- Если это дети первых поселенцев --- у них ещё нет иммунитета от микрофлоры Тысячи Башен, --- хмуро сказала Анкарьяль.
--- Любое микроскопическое зверьё с земли может стать для них смертельным, не говоря уже об антропонозах.
Нужны антибиотики.

--- Я не поняла ни слова, --- призналась Курц.
--- Просто скажи, что мне достать.

--- Врач им нужен.

--- Да спасибо, поняли мы уже, --- буркнула Курц.
--- Два дня ещё идти.
Не потерять бы мальчишку...

\section{Курточка}

--- Я курточку ей шью, нэй.
У меня обрезки ткань есть, тепло ей будет.

\section{Древняя вражда}

Мальчишка стоял неустойчиво.
Руки расставлены чересчур широко.
Нож недостаточно плотно у шеи, находится напротив твёрдой гортани, а не у сонного треугольника.
Голова выдаётся из-за тела заложника за пределами разброса выстрела.

\ml{$0$}
{<<Ашита мог бы справиться с ним в долю секунды, --- думала Курц.}
{\textit{Ashita could handle him in a split second,} Kurz thought.}
\ml{$0$}
{--- Кто-то из легионеров может легко пристрелить Михаэля.}
{\textit{One of the legionaires could easily shoot Michael.}}
\ml{$0$}
{Почему же они медлят?..}
{\textit{Why do they hesitate ...?}}
\ml{$0$}
{А, ясно.}
{\textit{Ah, I see.}}
\ml{$0$}
{Анкарьяль велела ничего не предпринимать>>.}
{\textit{Angaralle told no moves.}}

Курц на мгновение пробрала дрожь.
Понятное дело, что для демона потеря тела --- всего лишь небольшая неприятность.
Но что бы сделал Ашита, если бы Анкарьяль приказала ему пожертвовать собой?..

\section{Клан Дорге}

--- Клан Дорге? --- Михаэль рассмеялся безумным смехом, прижимая нож к горлу Ашиты.
\ml{$0$}
{--- Он убил моих родителей!}
{``It killed my parents!}
\ml{$0$}
{Он убил моих друзей!}
{It killed my friends!}
\ml{$0$}
{Мы вели с Демоном войну с того момента, как ступили на Таузендтурмеланд!}
{We were fighting the Demon since we landed Tausendt\"{u}rmeland!}
\ml{$0$}
{Как смеет это существо называться именем друга моей семьи, своего создателя?}
{How dare that thing to call itself the name of my family's friend, the name of its creator!''}

--- Успокойся, --- Анкарьяль опустила пистолет.
\ml{$0$}
{--- Отпусти моего легионера, ты всё равно не сможешь его убить.}
{``Release my leggionaire, you can't kill him anyway.}
\ml{$0$}
{Мы все здесь демоны.}
{We're all daemons here.}
\ml{$0$}
{Дорге давно уже нет в живых, многие из нас знают о нём лишь из исторических хроник, даже члены его клана.}
{Dourgue had gone long ago, most of us know him as but a part of history chronicles, even his kinsmen.}
\ml{$0$}
{Мы не виноваты в том, что случилось с твоей семьёй.}
{We're not responsible for what happened to your family.''}

--- Вы все демоны, --- потрясённо прошептал Михаэль.
\ml{$0$}
{--- Никто из вас не человек...}
{``No one of you is human ...''}

--- Я человек, --- сказала Курц, пытаясь удержать извивающуюся Грету.
\ml{$0$}
{--- Арслан --- тоже человек.}
{``Arslan is a human too.}
\ml{$0$}
{Арслан, скажи ему.}
{Arslan, tell him.''}

\ml{$0$}
{--- Я человек, нэй, --- закивал Арслан.}
{``I am a human, n\ae{},'' Arslan nodded.}
\ml{$0$}
{--- Меня эсей выносила, атай воспитал.}
{``\textit{\OE{}s\ae{}} carried me, \textit{at\ae{}} teached me.}
\ml{$0$}
{Отпусти его, егет.}
{Let him go, \textit{jeget}.}
\ml{$0$}
{Не надо драться...}
{There is no need in fighting ...''}

--- На этой планете бок о бок с демонами живут миллионы людей, потомки твоих сородичей, --- продолжала Курц.
--- Твои сородичи выжили и обрели здесь дом.

\ml{$0$}
{--- Я тебе не верю, --- прошептал Михаэль.}
{``I don't believe you,'' Michael whispered.}
Прежде чем кто-то успел сообразить, он оттолкнул Ашиту и прыгнул в Трещину.
Чернота Заалвира молча сомкнулась над ним, и вслед ему прозвучал лишь отчаянный крик Греты.

\section{Разными дорогами}

--- Идём, --- говорил Арслан.
--- Моя жена тебя полюбит, нэй, она любит детей.
У тебя брат-сестра будут, нэй.
Мы с жена больше детей не можем, один родили, ещё два дети взяли, сиротки, и тебя возьмём.
Хук-глайдер тебя научим, нэй, читать-писать умеем.
Хорошие люди у нас в город живут, не обидим, нэй.

Девочка кивнула.
Арслан застегнул ей курточку, проверил крепления на ремнях и взвалил Грету на плечи.

--- Курц, --- Арслан протянул женщине руку, --- хорошая ты женщина.
Не теряйся, нэй, в гости заходи, как рядом будешь.

\ml{$0$}
{--- Ау бул, дускай, --- Курц ответила на рукопожатие.}
{``\textit{Au bul, du\th{}k\ae{},}'' Kurz answered the handshake.}

--- Анкарьяль, --- кивнул проводник демонице.
Та кивнула в ответ.

--- Ты помнишь, что через перевал лучше не ходить?

--- Помню, --- ответил Арслан, подняв звякнувшие хуки.
--- В обход пойду.
Есть ещё один перевал, далеко от висяше руины, нэй.

Перед тем, как скрыться за углом, Грета обернулась и бросила на Курц ненавидящий взгляд.
<<Предательница>>, --- прочитала в этом взгляде Курц.
Анкарьяль похлопала её по плечу.

--- Привыкай, --- лаконично сказала она.
--- Когда люди узнают, кто мы, по-другому на нас не смотрят.

\section{Ещё одна переправа}

--- Есть ещё одна переправа.

--- Что ты имеешь в виду?

--- То, что сказал.
Есть ещё одна переправа, о которой знаю только я.
Мы пользовались ею с друзьями семь или восемь раз, перевозили товары, никого и ничего не потеряли.

--- Она достаточно надёжна, чтобы перенести отряд?

--- Она достаточно надёжна, --- кивнул старик.
--- Но есть нюанс.

--- Она сезонная? --- Курц закрыла глаза.

--- И сейчас не сезон.
Шанса на ясный день у вас точно нет, молитесь, чтобы не было бури.
Бури начинаются в пятом месяце, во второй или третьей четверти.
То есть завтра утром, скорее всего, ваш последний шанс на переправу.

--- У тебя есть карта?

--- Есть, но я её тебе не дам, пока не пойму, что здесь происходит.

\section{Коварная Башня}

--- Ты слышала про Багатурева Рыковице?

\ml{$0$}
{--- Перчатка Рыцаря?}
{``Knight's Mitten?''}

\ml{$0$}
{--- Именно.}
{``Exactly.}
\ml{$0$}
{Коварная двойная Башня, которая притягивает бури.}
{A treacherous double Tower, which attracts storms.}
Переправа находится рядом с ней.
Пока медведь не проснулся, рядом с ним безопаснее всего.

\section{Багатурева Рыковице}

--- Не летай, дочка.

--- Ты сам сказал, что это наш последний шанс!

--- Я сказал это вчера, когда погода еще была хорошей!

\section{Маячок}

--- Я полечу по маячку, --- заявила Курц.
--- Радио --- вспомогательный способ, не более.

--- Как знаешь, --- ответила Анкарьяль.
--- Всем приготовиться к отлёту.
Послать распоряжение об изменении в цепи командования --- пока мы не перелетим Трещину, приказы Курц Штайгер равносильны моим.

\section{Туман}

<<Туман.
Чёрт бы тебя побрал.
Мы сейчас потеряем половину отряда...>>

Курц завела маячок на левой руке, затем, выждав паузу, завела правый.
Чик-коц, чик-коц, чик-коц...
Вскоре снизу, справа и слева отозвались маячки ещё троих координаторов.

<<Центральная группа, держим дистанцию, --- тут же бросила в радио Анкарьяль.
--- Крайние группы, не нажимаем и не теряемся>>.

Курц улыбнулась.
Эта баронин быстро учится.

Один за другим яркие треугольники глайдеров нырнули в мягкий серый туман.

<<Если ты ничего не видишь, закрой глаза>>.
Эта фраза была вытатуирована у Курц на лопатках, как и у любого Штайгера --- фамильный девиз, мудрость прошедших поколений.
Курц закрыла глаз и начала ощупывать карту переправы.

\section{Тангенс}

Вдруг один из маячков повело вправо --- он начал отдаляться.

<<Я поймал тангенс, --- тут же сообщил по радио координатор.
\ml{$0$}
{--- Шестая, восьмая, тринадцатая группа --- вверх до упора, ориентир три --- отбой>>.}
{``Group six, eight, thirteen, full up, beacon three is off.''}

Маячок координатора умолк.
Курц поразилась его голосу --- он был спокоен.
Завихренный тангенциальный поток мог унести его куда угодно, в том числе и в Трещину...

<<Франциск, ты как?>> --- спросила Анкарьяль через полминуты.

<<Паршиво, крутит, но держусь>>, --- сообщил координатор.

<<Если кто-то увяз в тангенсе, следуйте за Франциском, --- вмешалась Курц.
--- Есть такие?>>

Трое отозвались положительно.

<<Франциск, переключись на другой канал, веди своих, --- распорядилась Анкарьяль.
--- Если выберешься --- ищи нас в условленном месте>>.

\ml{$0$}
{<<Понял>>.}
{``Roger that.''}

Маячок Франциска вновь затрещал где-то далеко внизу.

\section{Далёкое эхо}

Вдруг Курц услышала свой собственный маячок.
Слабый <<чик-коц>>, всего один.

Ещё один вне очереди.

И ещё.

Курц испустила тонкий и короткий заливистый вопль.
Всё её существо обратилось в слух, бессознательно считая щелчки маячка...

Шесть.

<<Всем группам, взять десять градусов направо и приготовиться к повороту, --- тут же приказала Курц, сверившись с картой переправы.
--- Азимут триста, крупное препятствие в километре.
\ml{$0$}
{Скорее всего, это Багатурева Рыковице, от неё лучше держаться подальше>>.}
{Most likely, it's Bagaturewa Rykovitze, better to keep a fair distance.''}

Поток резко закрутил глайдер.
Против часовой стрелки, по часовой стрелке, снова против...
Крылья угрожающе затрещали.
Курц привычно потянула на себя стабилизатор, повернула руль...
Следующий порыв вырвал один из тросов вместе с замком крепления.

--- Ах ты сволочь.

Курц попыталась перехватила конец, но ухватила лишь воздух.
Трос хлестнул по телу, словно хороший сыромятный кнут, и мехи на животе Курц отсоединились от стомы.

--- Дьявол, --- вполголоса выругалась Курц, чувствуя, как по ноге потекла отвратительная жижа.

\ml{$0$}
{<<Что такое, Курц? --- тут же отозвалась Анкарьяль.}
{``Whazzup, Kurz?'' Angaralle immediately answered.}
\ml{$0$}
{--- Мне послышался твой голос>>.}
{``I thought I heard your voice.''}

\ml{$0$}
{<<Всё нормально.}
{``I'm all right.}
\ml{$0$}
{Я просто немного дала течь>>.}
{Just sprung a leak.''}

Курц ожидала смешков и шуточек, но эфир оставался чистым.

<<Командир, я в десяти метрах слева, --- отозвался один из легионеров.
--- Если есть проблемы, я могу пристыковаться и перехватить тебя>>.

<<Всё в порядке, легионер, благодарю, --- ответила Курц.
--- Это подождёт до земли>>.

\ml{$0$}
{<<Они другие>>.}
{They're different.}

Курц не глядя вытащила из сумки с расходниками новый замок, на ощупь прикрутила.
Крыло снова стабилизировалось, каждый трос дублировался.
Но этот конец...
Он весело болтался в потоках воздуха, ощерившись металлическими зубами.
Дотянуться до него было нельзя, и любой турбулентный поток грозил превратить его в бешеную нагайку, полосующую плоть и ткань.
Даже если его поймать, вкрутить обратно будет непросто.

<<Впервые вижу, чтобы новый проксимальный замок ломался в полёте, --- удивлённо думала Курц.
--- Ванты и то рвутся чаще.
Даже мама о таком никогда не упоминала...
\ml{$0$}
{Впрочем, она говорила, что если о проблеме никто не сообщал --- вполне вероятно, что проблему никто не пережил>>.}
{\textit{She told, however, if the problem were not reported, the problem more than likely were not outlived.}}

<<Это четырнадцатый, --- доложило радио.
--- Похоже, меня поймал кольцевой поток Башни>>.

<<Четырнадцатый, следуй за потоком, держись как можно выше, не позволяй увлечь себя вниз, --- затараторила Курц.
--- Как только поток повернёт на север, что есть крыльев лети на запад, по ровной земле где-нибудь да сядешь>>.

\ml{$0$}
{<<Понял.}
{``Roger that.}
\ml{$0$}
{Ориентир два --- отбой...>>}
{Beacon two is off---''}

Передача прервалась звуком удара и скрежетом.
Треск маячка затих.

<<Дьявол, ещё один>>.

<<Группа семь и два, до упора вправо, готовимся к повороту!
Она ближе, чем я думала>>.

Едва группа прошла поворот, как налетел шквал.

Курц потеряла представление о верхе и низе;
она дёргала рычаги более по наитию, чем по расчёту.
Поток швырял её из стороны в сторону.

<<Точка два, двенадцать часов! --- выкрикивала она в радио.
--- Точка четыре, два часа!
Группа семь, вверх до упора!>>

Несколько раз Курц попыталась поймать злосчастный вырванный трос.
Конец полосовал её по рукам и ногам, оставляя после себя горящие ссадины, извивался и выскальзывал из захвата, как речной угорь.
Прежде чем Курц успела его закрепить, он снова вырвался и хлестнул её по очкам.
Оправа вдавилась в плоть, глаз немедленно залила кровь.

Курц заорала, вложив в этот крик всю охватившую её ярость.
Вдруг время словно замедлилось, чувства обострились;
она почти видела, как летит трос, с тихим визгом рассекая воздух.
Ещё миг --- трос оказался у неё в руках, крыло пошло вниз, и шипастая головка с размаху вошла в гнездо.
Трос натянулся струной и запел привычную уху воздухоплавателя песню.

<<Курц, как слышно? --- взывала Анкарьяль.
--- Что случилось?>>

<<Просто продолжаем полёт>>.

<<У тебя глаз повреждён!>>

<<Как она узнала?>>

<<К дьяволу глаз, глаза здесь не нужны! --- рявкнула Курц.
--- Освободи частоту!>>

Исчез отряд, исчез туман;
это была её личная битва с Башней.

\section{Пробуждение}

--- Глаз не задет, --- сухо ответила Анкарьяль.
--- Кровь я смыла.

--- Могла бы поручить это легионерам.

Анкарьяль промолчала.

Курц не помнила приземление.
Она продолжала лететь за пределами карты переправы;
она продолжала лететь, когда кто-то крикнул <<Просвет!>>;
она продолжала лететь, когда Анкарьяль несколько раз переспросила <<Курц, тебя перехватить?>>.
Только когда сильные руки обняли ее за плечи и щёлкнули складываемые крылья глайдера, Курц окончательно потеряла сознание.

--- Скольких мы потеряли?

Взгляд Анкарьяль потеплел.

--- Всего десятерых, благодаря тебе.
Я в жизни не видела такого проводника, как ты.

--- Франциск?

\ml{$0$}
{--- Он не выбрался.}
{``He didn't make it.''}

\section{Презрение}

--- Ты говорила, что они презирают людей.

--- Не мои легионеры, --- ответила Анкарьяль.
--- И не тебя.

\section{Ругательство}

--- Почему ты говоришь <<дьявол>>?

--- Просто привычка, --- осклабилась Курц.
--- Мама меня ругала за неё, но мне нравится.

--- Похоже, мне тоже, --- хихикнула Анкарьяль.

\section{Террористы}

--- Орден Преисподней --- просто террористы.

--- Да, для многих это будет выглядеть как хаос, --- кивнула Анкарьяль.
--- Отцы обратятся против детей, внучки пойдут против дедов.
Но это лишь ширма, внешняя оболочка.

--- А ты не думаешь, что то, как вещь выглядит, порой и есть её суть?

Анкарьяль на несколько долгих мгновений замолчала, думая, стоит ли сказать то, что вертится на языке.

--- Ты гораздо красивее внутри, чем снаружи, Курц, --- наконец проговорила она.

--- Но ведь внешность отражает мою суть, --- грустно возразила Курц.
--- Я привыкла быть уродливой, быть калекой.
Я сломана внутри.

--- Ты не...

--- Не перебивай.
Я сломана, это правда.
Даже если я шагну за пределы своих возможностей, даже если я сменю десять тел --- моё самое первое искалеченное тело останется со мной навсегда.
Внешнее всегда связано с тем, что внутри.

--- Я достаточно повидала войн, чтобы делать выводы.

--- Ты никогда не видела войну так, как видят её простые жители.
Ты видишь то, что тебе позволяют видеть командиры --- маску войны.
Ты бессознательно дорисовываешь войне лицо, и оно кажется тебе красивым.
Но те, кто никогда не обладал твоей властью и твоими знаниями, будут видеть войну такой, какая она есть, без маски, без дорисованной красоты.

--- Я не вижу другого пути, --- буркнула Анкарьяль.
--- Если ты считаешь меня террористкой --- я приму это как справедливую цену за правильный поступок.

--- Скажи, почему ты называешь себя человеческим именем?

Анкарьяль непонимающе посмотрела на подругу.

--- Я слышала, как твоя командир называла тебя Тальяной, --- пояснила Курц.
--- Но ты продолжаешь называть себя Ангарой.

Баронесса отвернулась.

--- Это личное.

--- Сколько времени тебе потребовалось, чтобы понять, что ты Ангара, а не Тальяна?

--- Нисколько.
Я просто услышала это имя и поняла, что оно принадлежит мне.

--- Тогда я уверена, Анкарьяль, что однажды ты меня поймёшь.

--- Почему?

--- Потому что все самые важные изменения происходят без войны.
Ты просто понимаешь, что новый порядок вещей --- единственно правильный для тебя.

\section{Поцелуй}

--- Я не удержалась и поцеловала тебя тогда, когда ты лежала без сознания.

--- Не удержалась? --- хихикнула Курц.

--- Я сняла с тебя маску, чтобы обработать тебе раны, и...
Если честно, я ожидала чего-то худшего.
А ты...
Ты была похожа на...

--- На грелла?

--- Наверное.
Не на то чудовище, которым пугают детей, а на принцессу греллов --- самую прекрасную женщину из всего их племени.

--- Наверное, это твой демон, --- предположила Курц.
--- Мне все говорили, что моё лицо --- это выжженная дотла деревня.

--- Да, это был демон.
Он экстраполировал твоё лицо и показал мне проекцию.

--- Скажи ему больше так не делать.
Я не хочу, чтобы ты целовала маску.
Даже ту, которую на меня надел твой демон.

Анкарьяль прижалась к Курц и погладила обтянутую шрамами грудную клетку.

--- Почему мне с тобой так хорошо?

--- Потому что ты меня любишь, --- прошептала Курц.

--- Я боюсь того, что это значит.

--- Что это значит?
Чего ты боишься?

--- Я не могу чересчур близко к сердцу принимать то, что происходит с людьми.
Иначе я забуду о долге.

--- Зачем нужен долг, если тебе некого обнять по возвращении домой?

--- Поцелуй меня и засыпай.

--- Это приказ, баронин?

--- Это приказ.

\section{Возвращение}

--- Курц, ты жива!

--- Да, я имитировала свою смерть.

--- Но зачем?

--- Чтобы быть с тобой, конечно.
Пойдём.

Курц потянула Анкарьяль за руку, но та не сдвинулась с места.

--- Я не понимаю, --- прошептала Анкарьяль.

--- Мы уйдём от Ордена, мы уйдём ото всех.
У нас будет домик у озера, огород, сад...

--- Курц, я не могу.
Я служу Ордену.
Я не могу уйти.

Курц отпустила руку Анкарьяль.

--- Ты сказала, что хочешь быть со мной.

--- Я хочу, но я думала...

--- ...думала, что я стану террористкой, как и ты? --- холодно завершила Курц.

--- Курц, пойдём со мной.
Мы будем вместе, и у нас будет общее дело, и...

--- Да не нужно мне твоё общее дело, --- буркнула Курц.
По её лицу текли слёзы.
--- Я за тобой пришла.

--- Я не могу уйти с тобой, Курц, --- прошептала Анкарьяль.
--- Я не могу...
Курц, подожди.
Вернись.
Пожалуйста.
Не уходи...

Но было поздно.
Туманы Вольке поглотили Курц Штайгер, надёжно скрыв направление её пути.
Анкарьяль упала на колени и взвыла от боли и одиночества, но ей ответило лишь эхо в опутанных туманом скалах.

\section{Эпилог}

Посреди палатки стояла на коленях женщина в цепях.
Её и без того изуродованное, похожее на череп лицо было изборождено обветреными морщинами.
Волосы --- наполовину черные, наполовину седые --- были растрёпаны, в них застряли сухие листья и трава.

--- И снова здравствуй, --- поприветствовала её женщина в камуфляже, по виду --- офицер.
--- Как здоровье?
Давно тебя не было видно.
Три года?
Четыре?
Ты стала умнее и хитрее, чем раньше.

Пленница не ответила.
Офицер подняла красную маску.

--- <<Охотники на демонов>>.
Серьёзно?
И сколько демонов вы убили?

Молчание.

--- Ты пыталась снять чертежи с аппарата для оцифровки, а также нескольких машин, использующих омега-интерфейс.
Отдам тебе должное, в этот раз направление правильное, но расстояние ты недооценила.
Даже если ваши специалисты поймут принцип (в чём я уже сомневаюсь), то чтобы собрать сколько-нибудь работающее оборудование, вам потребуются огромные производственные мощности.
С учётом ресурсов Тысячи Башен, на это уйдут десятки лет --- даже если бы я вам не мешала.

Пленница вздохнула.

--- Демоны --- это не люди, это сгустки информации.
Ты не можешь уничтожить нас, просто проделав в нас несколько дырок.
Или ты собираешься выявлять демонов среди людей, устроить новорождённым тотальный скрининг, чтобы избежать появления инкарнатов?
Идея не новая, это пытались реализовать в десятках более технологически развитых миров, не получилось ни у кого.
Люди размножаются стихийно.
Тебе не под силу поставить вопроизводство даже под статистический контроль, не говоря обо всём остальном.

Молчание.

--- Ты понимаешь, что ставишь себе невыполнимую задачу?
Ты отдаёшь себе отчёт, что каждая твоя попытка забирает жизни людей, не принося тебе взамен ничего?
То, что ты собираешься сделать, не под силу одному человеку, не под силу сотне и десятку тысяч тоже.
Твоему телу осталось жить сорок лет, плюс-минус удача, но твои умственные ресурсы подойдут к концу намного раньше.

Молчание.

--- Не хочешь говорить --- не надо, --- пожала плечами женщина в камуфляже и обратилась к подчинённому:
--- Отведите её за город, снимите оковы и оставьте там.

--- Хватит!

Отчаянный крик пленницы разорвал пустоту, словно удар молнии.
Женщина в камуфляже подняла брови.

--- Тринадцать раз! --- хрипло выплюнула пленница.
--- Тринадцать раз я попадаю в плен, и тринадцать раз ты меня отпускаешь!
С меня хватит!
Просто убей меня!

--- Я всё сказала тебе еще в нашу третью встречу, --- терпеливо пояснила женщина в камуфляже.
--- Однажды я увидела нечто большее за изуродованным лицом.
И сейчас я вижу нечто большее за твоими бесконечными бесполезными попытками организовать восстание против Ордена.

--- Анкарьяль, просто убей меня, --- прошептала пленница.
По её лицу градом катились слёзы, теряясь в извилинах шрамов и морщин.

--- Нет, Курц, --- отрезала женщина в камуфляже.
--- Ты мне нужна.
Легионер, вы слышали мой приказ.

Легионер вздёрнул пленницу на ноги и вывел её из палатки.
Анкарьяль осталась одна.

Солнце давно ушло за полдень и наконец заглянуло в проем палатки.
Анкарьяль на секунду оторвалась от своих записей и подставила лицо теплому оранжевому свету.
Он был таким ласковым, таким нежным, как...

--- Командир.
К вам пленница, которую вы велели отпустить.
Пришла пешком.
Просит слова.

--- Впустите, --- встрепенулась Анкарьяль.

Курц вошла в палатку и остановилась, скрестив руки на груди.

--- Я отказываюсь признавать власть Ордена здесь.

--- Я в курсе, --- пожала плечами Анкарьяль.
--- Могла бы приберечь эти слова до своего следующего пленения.

Курц молча встала на одно колено.

--- Я клянусь тебе в верности, Анкарьяль.
Я буду делать то, что ты прикажешь.
Я отказываюсь признавать власть Ордена над Тысяче Башен, но я признаю твою власть надо мной.

Анкарьяль откинулась в кресле и взглянула в оплетённое заходящим солнцем лицо Курц.
Молчание тянулось долго, словно мёд...
Наконец Анкарьяль встала на ноги.

--- Легионер, --- позвала она.

--- Да, командир?

--- Отвезите Курц Штайгер в штаб и подготовьте к оцифровке.
Настройте оборудование под мои сигнатуры.
Отправьте сообщение на Капитул: я нашла кандидата на пост командующего вооруженными силами Ордена на Тысяче Башен.

\section{Нос}

Анкарьяль долго молчала, не зная, что сказать.

--- Хорошо выглядишь, --- наконец проговорила она.

Курц ухмыльнулась.

--- Ну, я теперь не грелл, --- женщина немного шепелявила.
--- Пока не освоилась с губами.

Анкарьяль погладила Курц по грудной клетке.

--- Грудь не сделала?
Тебе и ее могут вырастить.

--- Я отказалась.
Привыкла уже, ну и в моём деле грудь будет только мешать.
Нос пока что тоже мешает... но дышать с ним проще, признаю.

Анкарьяль промолчала, смотря на вышитый ворот рубашки Курц.

--- И куда ты теперь? --- поинтересовалась Курц.

--- В поля, --- улыбнулась Анкарьяль.
--- В команде Хэм Золотой Посох освободилось место, и я попросилась к ним.

--- Кажется, я уже слышала про команду Хэм.
Диверсанты?

--- Одни из лучших.
Я не подхожу для административной работы.

--- С чего ты взяла, что для неё подхожу я?

Анкарьяль хихикнула.

--- Да мне плевать, подходишь ты или нет.
Я просто не хочу, чтобы ты стала террористкой, как я.

Курц опустила голову.

--- Сейчас у тебя есть хороший шанс принести людям хорошую жизнь.

--- Командующий армией --- это не та должность, которая приносит счастье, --- возразила Курц.

--- Именно та.
Не используй армию без крайней необходимости.
Цени жизни людей и легионеров.
Покажи всем, что мир для тебя важнее всего.

--- Как скажешь, Анкарьяль.

--- Я не хочу, чтобы ты делала всё это из-за обещания, которое дала мне.
Я хочу, чтобы ты делала всё это, потому что это правильно.

Курц кивнула.
Анкарьяль вдруг повела плечами, словно ей стало холодно.

--- Слушай, Курц.
Моя новая работа очень интересная, но и опасная тоже.
Особенно сейчас.
В общем, я...

--- Я знаю, --- сказала Курц.
--- Ты думаешь, что не вернёшься.

--- Да.

\ml{$0$}
{--- А если вернёшься, мы никогда не будем так близки.}
{``And even if you come back, we'll never be as close as we are now.''}

--- Я никогда ни с кем не расставалась, --- призналась Анкарьяль.

--- Ты рассталась со мной тогда.

Анкарьяль вздохнула.

--- Чёртовы люди.
У нас с вами разное чувство времени и расстояния.
Мы с тобой были на одной планете, Курц, это почти что рядом, да и несколько лет --- это всего лишь мимолётное расставание.
Я знала, что ты где-то здесь, в пределах досягаемости, и это грело мне душу.
А сейчас...

--- Ты никогда не будешь для меня чужой, --- прошептала Курц.
\ml{$0$}
{--- Из моего сердца обратной дороги нет.}
{``There's no way back from my heart.''}

--- Значит, я вернусь.

--- Если ты не вернёшься, я буду считать себя свободной от данного обещания и уничтожу Орден.

--- Зная твоё упорство... --- поёжилась Анкарьяль.
--- Мне и вправду лучше не умирать.

Повинуясь внезапному порыву, Курц подошла ближе и поцеловала Анкарьяль в тонкие губы.
На секунду солнце вдруг стало тем, осенним, ласковым;
вокруг закружились в танце чуть влажные золотые листья.

--- Мне не следовало уходить, Анкарьяль, --- прошептала Курц.

--- Тебе нужно было время, --- улыбнулась Анкарьяль.
--- Мне тоже, похоже, нужно время.
Непонятно для чего, но определённо нужно.

--- Прощай, Анкарьяль.

--- Прощай, Курц Пламя Осени.

\section{Карта стратегии}

Курц уже начала осваиваться со стратегическими модулями.
Проблема --- щёлк --- решение --- щёлк --- новая проблема --- щёлк --- новое решение --- щёлк...

<<Так вот как вы нас сдерживали.
Умно.
Но эта цитадель отнюдь не неприступна...>>

Курц вытащила из стопки бумаг чистый лист и карандаш.
Слабо улыбаясь, она нанесла на карту стратегии первые два узла и соединила их пунктиром.
\ml{$0$}
{Демон тут же подсветил их в её сознании:}
{The daemon immediately highlighted them in her mind:}

\ml{$0$}
{<<Курц Штайгер - - - - - Ангара Краснобуря>>}
{\textit{Kurz Steiger - - - - - Angara of Redbreeze}}

\end{document}
