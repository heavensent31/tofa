\documentclass[a4paper,10pt,fleqn]{book}\usepackage{polyglossia}\setdefaultlanguage[babelshorthands=true]{russian}\setotherlanguage{english}\defaultfontfeatures{Ligatures=TeX,Mapping=tex-text}\usepackage{xcolor}\newcommand{\ml}[3]{#2}

% \documentclass[a4paper,10pt,fleqn]{book}\usepackage{polyglossia}\setdefaultlanguage{english}\setotherlanguage{russian}\defaultfontfeatures{Ligatures=TeX,Mapping=tex-text}\usepackage{xcolor}\definecolor{lightgray}{HTML}{bbbbbb}\color{lightgray}\newcommand{\ml}[3]{\textcolor{black}{#3}}

% ----------------------

\usepackage{amsmath,amssymb,amsfonts,xltxtra,microtype,graphicx,textcomp}
\usepackage{svg}

% ------ GEOMETRY ------

\usepackage[twoside,left=2.5cm,right=3cm,top=3cm,bottom=4cm,bindingoffset=0cm]{geometry}

% ------ FONT ------

\setmainfont{Linux Libertine}
\definecolor{darkblue}{HTML}{003153}

% ------ HYPERLINKS ------

\usepackage{hyperref}
\hypersetup{colorlinks=true, linkcolor=darkblue, citecolor=darkblue, filecolor=darkblue, urlcolor=darkblue}

% ------ EPIGRAPH ------

\usepackage{epigraph}
\renewcommand{\epigraphsize}{\footnotesize}
\epigraphrule=0pt
\epigraphwidth=8cm

\usepackage{etoolbox}
\AtBeginEnvironment{quote}{\itshape}
\makeatletter
\newlength\episourceskip
\pretocmd{\@episource}{\em}{}{}
\apptocmd{\@episource}{\em}{}{}
\patchcmd{\epigraph}{\@episource{#1}\\}{\@episource{#1}\\[\episourceskip]}{}{}
\makeatother

% ------ METADATA ------

\newcommand{\tofaauthor}{\ml{$0$}{Эмиль~Весна}{Emil~Viesn\'{a}}}
\newcommand{\tofatitle}{\ml{$0$}{ПЛАМЯ~ОСЕНИ}{Flame~of~the~Fall}}
\newcommand{\tofastarted}{01.10.2021}

% ------ FANCY PAGE STYLE ------

\usepackage{fancyhdr}
\pagestyle{fancy}
\fancyhead[LE,RO]{\thepage}
\fancyhead[LO]{{\small\textsc{\tofatitle}}}
\fancyhead[RE]{{\small\textsc{\tofaauthor}}}
\fancyfoot{}
\fancypagestyle{plain}
{\fancyhead{}
\renewcommand{\headrulewidth}{0mm}
\fancyfoot{}}

% ------ NEW COMMANDS ------

\newcommand{\asterism}{\vspace{1em}{\centering\Large\bfseries$\ast~\ast~\ast$\par}\vspace{1em}}
\newcommand{\textspace}{\vspace{1em}{\centering\Large\bfseries<...>\par}\vspace{1em}}
\newcommand{\FM}{\footnotemark}
\newcommand{\FL}[2]{\footnotetext{См. \textit{\hyperlink{#1}{#2}}.}}
\newcommand{\FA}[1]{\footnotetext{#1 \emph{\ml{$0$}{---~Прим.~авт.}{---~Author.}}}}

\newcommand{\theterm}[3]{\textbf{\hypertarget{#1}{#2}} --- #3}
\newcommand{\thesynonim}[3]{\textbf{#2} --- см. \textit{\hyperlink{#1}{#3}}.}
\newcommand{\theorigin}[3]{\textit{#1:} #2 --- #3}


\begin{document}

% ------ TITLE PAGE ------

\begin{titlepage}
{\centering{~\par}\vspace{0.25\textheight}
{\LARGE\tofaauthor}\par
\vspace{1.0cm}\rule{17em}{1pt}\par\vspace{0.3cm}
{\Huge\textsc{\tofatitle}\par}
\vspace{0.3cm}\rule{17em}{2pt}\par\vspace{1.0cm}
{\Large\textit{\ml{$0$}{Повесть}{Novella}}\par}
\vspace{0.5cm}\asterism\par\vspace{1.0cm}
{\textbf{\ml{$0$}{Начато:}{Started:}}~\tofastarted\par}\vfill
{\Large\ml{$0$}{Создано~в}{Created~by}~\XeLaTeX}\par}
\end{titlepage}

\tableofcontents

\chapter{Пламя Осени}

--- Башня Дьявола рухнула!
Вставайте, вставайте!

С этих слов началось для Курц Штайгер утро, полное криков, топота и звона оружия.

Башней Дьявола издревле называли огромный кристаллический столп, росший на краю друзы Хербст.
Шли века и тысячелетия, эрозия подтачивала Башни, и они рушились одна за другой.
Некоторые падали неудачно, раздавливая целые города и деревни.
Некоторые падали удачно, на несколько сезонов превращаясь в каменный мост с одной Друзы на другую.
Чаще всего, разумеется, односторонний --- наклон и положение Башни редко позволяли одинаково хорошо идти вверх и вниз по ней.

О Башне Дьявола знали давно, к ней обращались мечты завоевателей и торговцев.
Мелкие люди, которые умирали, не дождавшись своего часа, а стареть начинали и того раньше, ждали и молили Всестроителя, чтобы Башня упала.
%{Overbuilder}
Но Башня Дьявола стояла.
По крайней мере, до сегодняшнего дня, когда её основание надломилось и исполинский кристалл встал строго горизонтально, соединив друзы Хербст и Гарда Викка.

Курц встала и протёрла глаз.
К чему вся эта беготня?
К Башне никто не подойдёт в ближайшие десять-пятнадцать дней.
Нужно время, чтобы она утряслась, плотно легла в своём каменном ложе, может быть, даже треснула напополам и отправилась в чёрную пучину Заалвира, разом похоронив все страхи и надежды людей.
%{Saalweird}
К чему эта беготня?

--- Ма-ам, --- сонно сказала Курц.
И тут же осеклась, вспомнив, что ей никто не ответит.

\asterism

Да, для неё процесс опорожнения выглядел так --- снять мехи, выдавить содержимое, обварить кипятком, затем обработать спиртом.
Достать новые мехи, промазать уплотнителем, аккуратно вставить, привычно поморщившись от боли.
И так каждый день.

Курц знала, что от нее всегда исходит неприятный специфический запах.
Раньше её это волновало --- она натиралась маслами, пытаясь отбить вонь;
сейчас ей было всё равно.

\asterism

Курц сняла маску, обнажив то, на месте чего должно было быть лицо.
Заросший глаз, стянутая кожа, кривое отверстие на месте рта, обнажённые носовые ходы...
Среди старых шрамов багровел ещё свежий, похожий на сколопендру.
Однажды женщине надоел не до конца закрывающийся рот, с уголка которого вечно капала слюна.
Местные врачи отказывались помогать, считая девушку безнадёжной.

--- Красивее ты от этого не станешь, --- напрямик сказала одна из них.

--- Я к тебе не за красотой пришла, идиотка, --- ответила Курц.

В конце концов она наткнулась на баронского полевого хирурга --- рассеянного, впадающего в деменцию старичка.
Тот отказался её оперировать, сославшись на трясущиеся руки, но подробно рассказал, как бы он это сделал.
Курц долго думала, примерялась, размышляла, а потом взяла бритву и исправила дефект.

Когда-то давно девчонка из Валленбергов --- местная красавица --- бросила ей в лицо:
<<Я бы лучше умерла, чем жила с этим!>>
Девчонка умерла при родах вместе с ребёнком.
Курц жива до сих пор.
И даже рот закрывается.

\asterism

Кленовые листья лежали на траве, словно язычки пламени.
Огнём горела и тут же сохла рябина, держа в высохших старческих руках горсти красных ягод.

\asterism

--- Умри, навозная глыба.

Движения Курц были выверены до мелочей.
Она отшатнулась, словно от неожиданности, пропустив вражеский клинок слева.
Имитация неуклюжего везения, вводящая противника в заблуждение.
Затем быстрым движением справа она перерезала врагу горло.

Налётчик взмахнул руками, словно пытаясь поймать и вернуть обратно фонтан крови, брызнувший из его шеи.
В его глазах застыло удивление.

<<На Хербст тебе не рады.
Я лишь вложила это чувство в движение клинка>>.

\asterism

Однажды она любила человека.
Молодой парень с жилистым телом и горячими руками краснел и смущался.
Он очень хотел узнать, что она прячет под маской.
Она показала ему.

Спустя год, когда Курц уже уверилась в том, что её приняли, он ушёл к другой --- молодой, красивой, по выражению Курц, <<у которой жопа на правильном месте>>.
Иногда она встречала его на окраинах города --- уже начавшего седеть, полненького, лысого мужчину с пушистыми усами.
Он не испытывал особых чувств к женщине, с которой жил, но очень любил своих детей --- забавно говорил с ними и таскал их на загривке.
При встрече он смущённо улыбался и говорил <<Привет, Курц>>.
Курц кивала в ответ.

--- Не переживай, девочка, --- сказала тогда мама за обедом.
--- Будут другие, получше.

--- Мне бы твой оптимизм.

--- Это не оптимизм, это признание очевидного.
Я тебя люблю --- не только потому, что ты моя дочь, но ещё и потому, что ты сильная и смелая, какой я никогда не была.
И он тебя любил.
Если бы не любил --- не был бы с тобой так долго.
Полюбил один --- найдутся и другие.

--- Твои-то <<другие, получше>> где задерживаются? --- пошутила Курц, брызнув на маму чаем.

--- С меня хватит, --- отмахнулась мама.

<<С меня тоже>>, --- сказала тогда Курц, но про себя, чтобы не огорчить маму.
С тех пор прошло пятнадцать лет --- ни мама, ни Курц так никого и не встретили.

Да и кого встретишь на крохотном островке камня, окружённом бездонной газовой пропастью?
Кого встретишь в постоянном хороводе <<работа-дом-работа>>?
Друзья юности постепенно завели семьи, умерли, улетели на другие Друзы...
Курц знала половину жителей Хербст по именам, вторую половину знала в лицо.
Она была первой, кого видели дети, покинув дома.
Она была последней, кого видели авантюристы, летящие сквозь туманную тьму в поиске счастья.

<<Я --- Штайгер.
В мир, где правят хук и глайдер, меня привела семья, которая учит детей обращаться с ними.
%{hook-en-gleider}
Разве не благородно посвятить этому делу всю жизнь, всю себя?>>

Эта нехитрая молитва помогала много лет.
Но одним промозглым дождливым днём, стоя над свежей могилой мамы, Курц поняла --- кроме неё на похороны пришёл лишь могильщик.
Никто из тех, в чьи руки Сабина Штайгер вложила хук, не нашёл времени, чтобы её проводить.

<<Мне нужны друзья>>, --- решила тогда Курц, засыпая.
Было очень непривычно не слышать дыхание мамы в другом углу комнаты.
Вселенная ответила на её призыв в своей обычной непонятной манере --- на следующее же утро рухнула Башня Дьявола.

\asterism

--- Пожалуйста, мама, пей! --- Курц со слезами пыталась влить в рот мамы воду.
Сабина металась в горячке, выкрикивая несвязные слова;
вода расплёскивалась, не достигая горла.

--- Злокачественный туберкулёз лёгких, --- вынес вердикт пришёдший врач.
--- Также известный как <<Пламя осени>>.
Я бы на твоём месте переночевал где-нибудь вне дома, а завтра провёл влажную уборку с отваром полыни.
Сейчас она заразна.

--- Она была совсем здорова вчера!

--- Так оно обычно и бывает, --- развёл руками врач.
--- Заболевание развивается в течение десяти-двадцати дней, иногда года, а потом человек сгорает, быстро и безвозвратно.
Судя по цвету кожных покровов, она вряд ли переживёт ночь.
Если решишь, что с неё хватит...

Врач вынул из-за пояса чёрный нож в ножнах, скреплённых сложной печатью из смолы и полосок ткани с надписями.
У Курц ёкнуло сердце.

--- Вот сюда, --- врач вытащил хирургический маркер и нарисовал на груди Сабины шестилучевую звёздочку.
--- Строго вертикально, резко, изо всей силы, проворачивать не надо.
Сломанную печать и нож принесёшь мне.

--- Можно это сделает кто-то другой? --- прошептала Курц.
В её горле стоял комок.

--- Я не могу, таков закон.
Сделать может любой человек, которому ты доверяешь, но нож и печать должна принести ты.
И не вздумай потерять печать, Курц Штайгер, иначе тебя обвинят в убийстве, ты меня поняла?

--- Поняла.
У меня нет никого, кому я могла бы это доверить.

--- В таком случае у тебя два выхода: либо дождись, пока она умрёт сама, либо возьми себя в руки и сделай то что должно.
Если нужно отпеть --- я могу прислать Ворона.

--- Не нужно, --- процедила сквозь зубы Курц.
--- Пусть Ворон своих птенцов отпевает.

С Вороном Биркендорфом Курц связывали давние тёплые чувства.
Когда Сабина принесла пищащую обожжённую девочку на именины, Ворон отказался проводить обряд.

--- Я даже не знаю, что за зверя ты мне принесла.

--- Она вышла из моей утробы, --- прорычала мама.
--- Я, по-твоему, кто --- ящерица?

\ml{$0$}
{--- Штайгер, ты слышала мой ответ.}
{``Steiger, you heard my answer.''}

\ml{$0$}
{--- У тебя имена закончились, Биркендорф?}
{``Are you out of names, Birkendorf?''}

\ml{$0$}
{--- Имён предостаточно, а вот времени на безнадёжных нет.}
{``I have plenty of names, I'm out of time for a lost cause.}
\ml{$0$}
{Она умрёт в течение десяти дней с такими ожогами.}
{She'll die in ten days with burns like those.''}

--- Чёрта с два, --- выплюнула Сабина, подхватила ребёнка и вышла из костёла.
%{``Like hell she will,'' Sabina spitted}

Именно так Курц получила своё прозвище.
По традиции Хербст, настоящее имя мог дать только Ворон, только на тридцать первом дне жизни.
Причин отказать в именинах было немного, но такое случалось.
Ходило поверье, что неименованных преследуют неприятности и ждёт ранняя смерть.
Так оно и было, впрочем --- люди заранее относились к неименованным настороженно, а порой и неприязненно.
Ведь если тебя не нарекли --- на это была причина, верно?

Как выяснилось впоследствии, Курц не только выжила, но и научилась устраивать неприятности именованным.
В детстве девочка несколько раз пробиралась в костёл и поджигала чёрную мантию Ворона.
Один раз он её за этим поймал.
Подняв за шкирку рычащее слюнявое существо, похожее на детёныша грелла\FM, он вдруг вспомнил злосчастные непрошедшие именины.
\FA{Грелл --- существо из мифологии друзы Хербст, одноглазый тролль, который смотрит в окна, выманивая людей на улицу по ночам.}

\ml{$0$}
{--- Считай, что мы квиты, Курц Штайгер, --- лаконично сказал Биркендорф.}
{``Call it even, Kurz Steiger,'' Birkendorf succinctly said.}
\ml{$0$}
{--- Ещё раз поймаю за этим --- урою.}
{``If I catch you again, you're done.''}

\asterism

--- Есть ещё одна переправа.

--- Что ты имеешь в виду?

--- То, что сказал.
Есть ещё одна переправа, о которой знаю только я.
Мы пользовались ею с друзьями семь или восемь раз, перевозили товары, никого и ничего не потеряли.

--- Она достаточно надёжна, чтобы перенести отряд?

--- Она достаточно надёжна, --- кивнул старик.
--- Но есть нюанс.

--- Она сезонная? --- Курц закрыла глаза.

--- И сейчас не сезон.
Шанса на ясный день у вас точно нет, молитесь, чтобы не было бури.
Бури начинаются в пятом месяце, во второй или третьей четверти.
То есть завтра утром, скорее всего, ваш последний шанс на переправу.

--- У тебя есть карта?

--- Есть, но я её тебе не дам, пока не пойму, что здесь происходит.

\asterism

--- Ты слышала про Багатурева Рыковице?

\ml{$0$}
{--- Перчатка Рыцаря?}
{``Knight's Mitten?''}

\ml{$0$}
{--- Именно.}
{``Exactly.}
\ml{$0$}
{Коварная двойная Башня, которая притягивает бури.}
{A treacherous double Tower, which attracts storms.}
Переправа находится рядом с ней.
Пока медведь не проснулся, рядом с ним безопаснее всего.

\section{Багатурева Рыковице}

--- Я полечу по маячку, --- заявила Курц.
--- Радио --- вспомогательный способ, не более.

--- Как знаешь, --- ответила Анкарьяль.
--- Всем приготовиться к отлёту.
Послать распоряжение об изменении в цепи командования --- пока мы не перелетим Трещину, приказы Курц Штайгер равносильны приказам легата секунда.

\asterism

<<Туман.
Чёрт бы тебя побрал.
Мы сейчас потеряем половину отряда...>>

Курц завела маячок на левой руке, затем, выждав паузу, завела правый.
Чик-коц, чик-коц, чик-коц...
Вскоре снизу, справа и слева отозвались маячки ещё троих координаторов.

<<Центральная группа, держим дистанцию, --- тут же бросила в радио Анкарьяль.
--- Крайние группы, не нажимаем и не теряемся>>.

Курц улыбнулась.
Эта баронесса быстро учится.

Один за другим яркие треугольники глайдеров нырнули в мягкий серый туман.

<<Если ты ничего не видишь, закрой глаза>>.
Эта фраза была вытатуирована у Курц на лопатках, как и у любого Штайгера --- фамильный девиз, мудрость прошедших поколений.
Курц закрыла глаза и начала ощупывать карту переправы.

\asterism

Вдруг один из маячков повело вправо --- он начал отдаляться.

<<Я поймал тангенс, --- тут же сообщил по радио координатор.
\ml{$0$}
{--- Шестая, восьмая, тринадцатая группа --- вверх до упора, ориентир три --- отбой>>.}
{``Group six, eight, thirteen, full up, beacon three is off.''}

Маячок координатора умолк.
Курц поразилась его голосу --- он был спокоен.
Завихренный тангенциальный поток мог унести его куда угодно, в том числе и в Трещину...

<<Франциск, ты как?>> --- спросила Анкарьяль через полминуты.

<<Паршиво, крутит, но держусь>>, --- сообщил координатор.

<<Если кто-то увяз в тангенсе, следуйте за Франциском, --- вмешалась Курц.
--- Есть такие?>>

Трое отозвались положительно.

<<Франциск, переключись на другой канал, веди своих, --- распорядилась Анкарьяль.
--- Если выберешься --- ищи нас в условленном месте>>.

\ml{$0$}
{<<Понял>>.}
{``Roger that.''}

Маячок Франциска вновь затрещал где-то далеко внизу.

\asterism

Вдруг Курц услышала свой собственный маячок.
Слабый <<чик-коц>>, всего один.

Ещё один вне очереди.

И ещё.

Курц испустила тонкий и короткий заливистый вопль.
Всё её существо обратилось в слух, бессознательно считая щелчки маячка...

Шесть.

<<Всем группам, взять десять градусов направо и приготовиться к повороту, --- тут же приказала Курц, сверившись с картой переправы.
--- Азимут триста, крупное препятствие в километре.
\ml{$0$}
{Скорее всего, это Багатурева Рыковице, от неё лучше держаться подальше>>.}
{Most likely, it's Bagaturewa Rykovitze, better to keep a fair distance.''}

Поток резко закрутил глайдер.
Курц дёрнула на себя стабилизатор, перехватила конец.
Трос хлестнул по телу, словно хороший сыромятный кнут, и мехи на животе Курц отсоединились от стомы.

--- Дьявол, --- вполголоса выругалась Курц, чувствуя, как по ноге потекла отвратительная жижа.

\ml{$0$}
{<<Что такое, Курц? --- тут же отозвалась Анкарьяль.}
{``Whazzup, Kurz?'' Angaralle immediately answered.}
\ml{$0$}
{--- Мне послышался твой голос>>.}
{``I thought I heard your voice.''}

\ml{$0$}
{<<Всё нормально.}
{``I'm all right.}
\ml{$0$}
{Я просто немного дала течь>>.}
{Just sprung a leak.''}

Курц ожидала смешков и шуточек, но эфир оставался чистым.

<<Командир, я в десяти метрах слева, --- отозвался один из легионеров.
--- Если есть проблемы, я могу пристыковаться и перехватить тебя>>.

<<Всё в порядке, легионер, благодарю, --- ответила Курц.
--- Это подождёт до земли>>.

\ml{$0$}
{<<Они другие>>.}
{They're different.}

<<Это четырнадцатый, --- вмешался голос.
--- Похоже, меня поймал кольцевой поток Башни>>.

<<Четырнадцатый, следуй за потоком, держись как можно выше, не позволяй увлечь себя вниз, --- затараторила Курц.
--- Как только поток повернёт на север, что есть крыльев лети на запад, по ровной земле где-нибудь да сядешь>>.

\ml{$0$}
{<<Понял.}
{``Roger that.}
\ml{$0$}
{Ориентир два --- отбой...>>}
{Beacon two is off---''}

Передача прервалась звуком удара и скрежетом.
Треск маячка затих.

<<Дьявол, ещё один>>.

<<Группа семь и два, до упора вправо!
Она ближе, чем я думала>>.

\asterism

--- Орден Преисподней --- просто террористы.

--- Да, для многих это будет выглядеть как хаос, --- кивнула Анкарьяль.
--- Отцы обратятся против детей, внучки пойдут против дедов.
Но это лишь ширма, внешняя оболочка.

--- А ты не думаешь, что то, как вещь выглядит, порой и есть её суть?

Анкарьяль на несколько долгих мгновений замолчала, думая, стоит ли сказать то, что вертится на языке.

--- Ты гораздо красивее внутри, чем снаружи, Курц, --- наконец проговорила она.

--- Но ведь внешность отражает мою суть, --- грустно возразила Курц.
--- Я привыкла быть уродливой, быть калекой.
Я сломана внутри.

--- Ты не...

--- Не перебивай.
Я сломана, это правда.
Даже если я шагну за пределы своих возможностей, даже если я сменю десять тел --- моё самое первое искалеченное тело останется со мной навсегда.
Внешнее всегда связано с тем, что внутри.

--- Я достаточно повидала войн, чтобы делать выводы.

--- Ты никогда не видела войну так, как видят её простые жители.
Ты видишь то, что тебе позволяют видеть командиры --- маску войны.
Ты бессознательно дорисовываешь войне лицо, и оно кажется тебе красивым.
Но те, кто никогда не обладал твоей властью и твоими знаниями, будут видеть войну такой, какая она есть, без маски, без дорисованной красоты.

--- Я не вижу другого пути, --- буркнула Анкарьяль.
--- Если ты считаешь меня террористкой --- я приму это как справедливую цену за правильный поступок.

--- Скажи, почему ты называешь себя человеческим именем?

Анкарьяль непонимающе посмотрела на подругу.

--- Я слышала, как твоя командир называла тебя Тальяной, --- пояснила Курц.
--- Но ты продолжаешь называть себя Ангарой.

Баронесса отвернулась.

--- Это личное.

--- Сколько времени тебе потребовалось, чтобы понять, что ты Ангара, а не Тальяна?

--- Нисколько.
Я просто услышала это имя и поняла, что оно принадлежит мне.

--- Тогда я уверена, Анкарьяль, что однажды ты меня поймёшь.

--- Почему?

--- Потому что все самые важные изменения происходят без войны.
Ты просто понимаешь, что новый порядок вещей --- единственно правильный для тебя.

\asterism

--- Почему мне с тобой так хорошо?

--- Потому что ты меня любишь, --- прошептала Курц.

--- Я боюсь того, что это значит.

--- Что это значит?
Чего ты боишься?

--- Я не могу чересчур близко к сердцу принимать то, что происходит с людьми.
Иначе я забуду о долге.

--- Зачем нужен долг, если тебе некого обнять по возвращении домой?

--- Поцелуй меня и засыпай.

--- Это приказ, баронесса?

--- Это приказ.
\end{document}
