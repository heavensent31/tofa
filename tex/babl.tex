\documentclass[a4paper,12pt,fleqn]{book}\usepackage{polyglossia}\setdefaultlanguage[babelshorthands=true]{russian}\setotherlanguage{english}\defaultfontfeatures{Ligatures=TeX,Mapping=tex-text}
\usepackage{xcolor}\newcommand{\ml}[3]{#2}

% \documentclass[a4paper,12pt,fleqn]{book}\usepackage{cooltooltips}\usepackage{polyglossia}\setdefaultlanguage[babelshorthands=true]{russian}\setotherlanguage{english}\defaultfontfeatures{Ligatures=TeX,Mapping=tex-text} \usepackage{xcolor}\definecolor{lightgray}{HTML}{bbbbbb}\color{lightgray}\newcommand{\ml}[3]{\textenglish{\textcolor{black}{#3}} }

% ----------------------

\usepackage{amsmath,amssymb,amsfonts,xltxtra,microtype,graphicx,textcomp}
\usepackage{svg}

% ------ GEOMETRY ------

\usepackage[twoside,left=2.5cm,right=3cm,top=3cm,bottom=4cm,bindingoffset=0cm]{geometry}

% ------ FONT ------

\usepackage{ebgaramond}
\definecolor{darkblue}{HTML}{003153}

% ------ HYPERLINKS ------

\usepackage{hyperref}
\hypersetup{colorlinks=true, linkcolor=darkblue, citecolor=darkblue, filecolor=darkblue, urlcolor=darkblue}

% ------ EPIGRAPH ------

\usepackage{epigraph}
\renewcommand{\epigraphsize}{\footnotesize}
\epigraphrule=0pt
\epigraphwidth=8cm

\usepackage{etoolbox}
\AtBeginEnvironment{quote}{\itshape}
\makeatletter
\newlength\episourceskip
\pretocmd{\@episource}{\em}{}{}
\apptocmd{\@episource}{\em}{}{}
\patchcmd{\epigraph}{\@episource{#1}\\}{\@episource{#1}\\[\episourceskip]}{}{}
\makeatother

% ------ METADATA ------

\newcommand{\tofaauthor}{\ml{$0$}{Эмиль~Весна}{Emil~Viesn\'{a}}}
\newcommand{\tofatitle}{\ml{$0$}{НАВИГАЦИЯ ПО ГОЛУБКАМ}{Babel}}
\newcommand{\tofastarted}{06.05.2021}

% ------ FANCY PAGE STYLE ------

\usepackage{fancyhdr}
\pagestyle{fancy}
\fancyhead[LE,RO]{\thepage}
\fancyhead[LO]{{\small\textsc{\tofatitle}}}
\fancyhead[RE]{{\small\textsc{\tofaauthor}}}
\fancyfoot{}
\fancypagestyle{plain}
{\fancyhead{}
\renewcommand{\headrulewidth}{0mm}
\fancyfoot{}}

% ------ NEW COMMANDS ------

\newcommand{\asterism}{\vspace{1em}{\centering\Large\bfseries$\ast~\ast~\ast$\par}\vspace{1em}}
\newcommand{\textspace}{\vspace{1em}{\centering\Large\bfseries<...>\par}\vspace{1em}}
\newcommand{\FM}{\footnotemark}
\newcommand{\FL}[2]{\footnotetext{См. \textit{\hyperlink{#1}{#2}}.}}
\newcommand{\FA}[1]{\footnotetext{#1 \emph{\ml{$0$}{---~Прим.~авт.}{---~Author.}}}}

\newcommand{\theterm}[3]{\textbf{\hypertarget{#1}{#2}} --- #3}
\newcommand{\thesynonim}[3]{\textbf{#2} --- см. \textit{\hyperlink{#1}{#3}}.}
\newcommand{\theorigin}[3]{\textit{#1:} #2 --- #3}


\begin{document}

% ------ TITLE PAGE ------

\begin{titlepage}
{\centering{~\par}\vspace{0.25\textheight}
{\LARGE\tofaauthor}\par
\vspace{1.0cm}\rule{17em}{1pt}\par\vspace{0.3cm}
{\Huge\textsc{\tofatitle}\par}
\vspace{0.3cm}\rule{17em}{2pt}\par\vspace{1.0cm}
{\Large\textit{\ml{$0$}{Рассказ}{Short~story}}\par}
\vspace{0.5cm}\asterism\par\vspace{1.0cm}
{\textbf{\ml{$0$}{Начато:}{Started:}}~\tofastarted\par}\vfill
{\Large\ml{$0$}{Создано~в}{Created~by}~\XeLaTeX}\par}
\end{titlepage}

\tableofcontents

\chapter{Столпотворение}

<<Море>>.

Мне в штанину снова залезла прохладная волна, нанеся в складки ещё щепотку скрипучего чёрного песка.
Я сразу понял, что это море --- по запаху, крикам птиц и жжению в расцарапанной спине.

<<Транспорт>>.

Я приоткрыл глаза, с трудом разлепляя спаянные ресницы.
Транспорт лежал неподалёку, в полосе прибоя.
Догорающее солнце мрачно отражалось в треснувшем оранжево-красном стекле.
Можно было даже не пытаться проводить осмотр --- идущий от сверхпроводящих магнитов пар ясно говорил, что машина мертва.

<<Ребята!>>

Я схватил губами воздух и приподнялся на локтях;
меня охватил приступ дурноты, в глазах засверкали звёзды.
Песок вокруг меня был испещрён следами обуви, уже частично расплывшимися от влаги.
Судя по всему, друзья походили вокруг меня, поняли, что я жив, и ушли куда-то в сторону далёких чахлых кустов.

<<Бросили в зоне прилива>>, --- содрогнулся я, посмотрев на небо.
Вода должна начать прибывать меньше чем через пару часов.

Непохоже на них.
Может, кто-то из них серьёзно ранен и ему требовалась помощь?

Я кое-как поднялся на ноги.
Меня мотало из стороны в сторону.
Я упал на колени и, порывшись в пл\'{а}внике, выудил длинную палку --- мокрую, тяжёлую, но достаточно крепкую, чтобы выдержать мой вес.

В путанице следов можно было разобрать четыре пары ног --- йоранские трекинговые сапоги, высокие каблуки, мягкие мужские кеды и огромные протекторы с инженерными креплениями.
Я с сожалением посмотрел в сторону моря.
Мать с ребёнком, которых мы успели подхватить на башне, остались в море навсегда.
Та же участь, видимо, постигла нашего станционного кота.

\asterism

--- Ну наконец-то, --- буркнул я, увидев вдалеке огонь.

Сови, Кирд, Тан-шая и Кевар молча сидели на жёстких голых камнях.
Никто даже не поднял глаза, когда я, бросив палку, уселся напротив.

--- Спасибо, что оставили меня в безопасном месте, --- съязвил я.

Кевар вздрогнул, а Тан-шая поёжилась, словно я пригрозил ей оружием.

Ветер изменил направление, бросив клубы дыма прямо мне в лицо.
Я пересел чуть подальше, подтащив палку.

--- Жрать есть? --- снова обратился я к друзьям.
--- Или вы ждали главного повара в лице меня?

И снова молчание.

--- Да в чём дело-то? --- спросил я уже помягче.

Ответа не последовало и в этот раз.

--- Ладно, ладно, извиняюсь, --- буркнул я.
--- Погорячился.
Сови, пошли со мной в лес, до темноты успеем что-нибудь набрать.
ДА ХВАТИТ УЖЕ МОЛЧАТЬ!

--- Haar't'werrah r'quiwarr! --- вдруг злобно рявкнул в ответ Кирд.

--- Lemanasatu limero virimanatusa nikulata, --- затараторил Кевар, успокоительно подняв руки.

--- K'wee! --- Кирд поднялся с явным намерением продолжить разговор кулаками. 
Сови и Тан-шая повисли у него на рукавах куртки, выкрикивая какие-то непонятные фразы.
Кирд оттолкнул их, продолжая ругаться на своём, похожем на медвежий рёв наречии.

--- Так, ребята, стоп.
Сови.
Кирд.
Кирд.
Стоп.
СТОП!

Я неаккуратно влетел в человеческую свалку и тут же получил локтем по голове, и так не оправившейся от жёсткой посадки.
Перед глазами поплыли жёлто-зелёные круги.

--- Ну и хрен с вами, --- пробормотал я, привстав на локтях и выплюнув ком песка.
Думаю, это единственное, что было в тот день у меня во рту.
Нащупав палку, я кое-как поднялся на ноги и пошёл к морю, пока друзья продолжали препираться --- каждый на своём языке.

Когда я дошёл до линии прибоя, уже стемнело.
Вода поднималась всё выше, следуя гравитации восходящего Харона-2.
Жёлтыми лампами в небе загорелись Близнецы --- ближайшая двойная звезда, предвещающая летнюю ночь и полное Плеяд небо.
Разбитый транспорт, на прощание сверкнув серебристым крылом, скрылся под набегающими волнами.

\asterism

Усевшись на какую-то корягу, я попытался собрать воедино картину происходящего.

Да, проект <<Вавилон>>.
Мы строили четвёртый ретранслятор, который должен был связать телепатической связью две обитаемые планеты нашей звёздной системы с прочими мирами империи Плеяды.
В моей памяти засел дикий вой ветра на головокружительной высоте --- с одной из антенн ретранслятора, казалось, можно было увидеть половину Харона и почти все его хтонично-красноватые небеса.
Магистрали мегаполисов Роде и Сагар казались трещинами, сквозь которые тускло просвечивало остывающее ядро планеты;
далёкий Имперский Приют, напротив, был пятном черноты, и даже тысячи Малых Солнц --- весь свет, который могли дать Плеяды --- были бессильны его осветить.

Мы со всех ног бежали к взлётной площадке под звук трещащих тросов и вой ветра.
Огромную башню трясло, как былинку...

Я почесал голову и вздохнул.

Друзья не понимали ни слова из того, что я им говорил.
Впрочем, никто из них вообще друг друга не понимал.
Я звал Кирда и Сови по именам, но они реагировали на собственные имена так же, как и на прочий шум, выходивший из моего рта.

Объяснение могло быть только одно.
Ретранслятор испустил мощный телепатический импульс, из-за которого у всех нас произошло полное ремоделирование речевых центров головного мозга.
Импульс не нарушил общую логику и структуру --- я по-прежнему был способен думать словами;
но едва ли тот язык, на котором я говорил и который понимал, был моим родным х'эакур.
Каждый из нас теперь говорил на своём собственном, понятном только ему языке.

Мои сапоги нежно облизала бурая, пахнущая фукусом волна.
Пузырьки морской пены рассыпались в песке и умерли с тихим шипением.
Я снова вздохнул и встал на ноги.
Придётся искать способ сообщить друзьям, что произошло.

\asterism

Друзья прекратили крикливую перепалку, мрачно сидели вокруг костра и молчали.
Когда я подошёл, все с надеждой повернулись ко мне.

Я не знаю, что двигало мной в тот момент.
Я просто сделал первое, что пришло мне в голову.
Вытащив из песка круглую гальку, я показал на неё пальцем и заявил:

--- Камень.

Друзья вытаращили глаза.

Я ещё раз указал на гальку и повторил:

--- Камень.

Сови вдруг понимающе ахнула.
Она подхватила гальку поменьше, указала на неё и произнесла:

--- Хамэн.

Вскоре понимание загорелось и в глазах Кирда.
Он указал на гальку в моей руке:

--- Кам-мен.

Я положил камень обратно на песок, встал и приложил руку к груди:

--- Кулав.

Друзья поняли.
Один за другим они вставали, прикладывали руки к груди и называли свои новые имена.
Так Сови, Кирд, Тан-шая и Кевар превратились в Уст, Хрыкры, Жест и Фонитара.

В тот же день я начал учить их собственному языку.

\asterism

Труднее всего язык давался Кирду, то есть Хрыкры --- его речевые центры ремоделировало так, что с его языком едва справлялся даже его собственный артикуляционный аппарат.
Лингвист Кевар, то есть Фонитар, овладевал языком со скоростью света.
Уже на следующий день мы с ним болтали на некоторые не особо сложные темы.

--- Я вода рыба поймал, --- похвастался Фонитар, показывая лупоглазую океаническую щуку.
--- Вкусный, дома часто ел.
Ты что идёт делает?

--- Ветки для дома, --- сказал я.
--- Дождь может быть.

--- Дож может быть? --- не понял Фонитар.

--- Дождь.
Вода сверху кап-кап-кап.

--- Ааа, понял!
Я делает с тобой.

Работа пошла пободрее.
К вечеру, когда остальные вернулись, мы с Фонитаром уже выстроили вполне приличный шалаш и промазали его глиной.
Инженер Хрыкры, покачав головой, вырвал нашу печь из пола и перенёс её на воздух;
затем с помощью жестов, двух палок и комка глины объяснил, как от уличной печи провести под полом отопление.

Изучение языка продолжалось.
Письменность, как оказалось, у каждого из нас тоже своя;
это было неудивительно, учитывая, что телепатический модуль был связан и с центром языковых знаков.
Друзья исписали стены пещерки собственными словариками, пытаясь транслитерировать слова моего языка на свой манер.
Самую большую площадь занимал словарик Фонитара;
в его языке слова оказались настолько длинными, что я даже обрадовался, что не придётся учить его язык.

\asterism

--- Кулав, дай мне камень, --- командовала Сови, она же Уст.
--- Ещё один.
Поставь его во-он туда, на метку.
А этот на противоположную.
А теперь иди отсюда.

Я усмехнулся и отошёл.
Уст тем временем легла на песок и сосредоточенно вперила взгляд в светлеющее небо, скрестив две палки.
Насколько я понял из её слов, это был древний способ навигации, который использовался где-то на Древней Земле.
Уст пролежала в песке уже четыре дня;
за это время доверительные интервалы наших координат сократились с двух тысяч километров до пятисот.

В целом всё начало налаживаться.
Смерть от голода в этих широтах нам не грозила;
океан был полон рыбы и водорослей, а в кустах созревали ягоды.
Тан-шая, она же Жест, накануне притащила из кустов ошарашенное животное наподобие оленя, которое она предварительно отлупила палкой до полубессознательного состояния;
животное мы зажарили, полив кислым ягодным соком.

Но остров был лишь вынужденным пристанищем.
Мы скучали по дому.
К тому же всегда существовала опасность серьёзно заболеть --- ни лекарств, ни медицинского оборудования у нас не было.

Впервые о возвращении мы поговорили спустя месяц пребывания на острове.

--- Что думаешь, остальные то же самое произошло? --- спросила Жест.

--- Скорее всего, --- буркнул Хрыкры.
--- Планета весь воздействие.

--- Это могло вызвать война, --- заметила Уст.

--- Согласен, --- сказал я.
--- Инфраструктура точно парализована.
Я боюсь, нам вообще повезло, что мы сейчас на необитаемом острове.

--- Тогда что делать? --- спросила Уст.

--- Доберёмся до большой земли, а там видно будет, --- сказал я.
--- Что бы там ни происходило, на большой земле хотя бы есть оружие, транспорт, лекарства и пища.
Ладно, ребята, давайте спать.
Силы нам понадобятся.

\asterism

Жест, которую я прежде знал как Тан-шаю, ска ма-Ракон, была одной из тех немногих йоранцев, которые смогли жить на тёмной планете Харон.
Её кожа казалась синеватой из-за особого пигмента, способного задержать мощную радиацию йоранских солнц.
Она постоянно носила очки --- глаза йоранцев не приспособлены к местному светилу, требовалось многократное усиление света, чтобы йоранцы начали хоть что-то различать.
Второй проблемой был воздух --- влажный и достаточно холодный.
Очки Жест имели специальный осушитель и согреватель воздуха, сопла которого вставлялись в её узкий нос.
Без осушителя женщину постоянно мучил кашель из-за отходящего из лёгких секрета --- в нормальных условиях Йорана этот секрет испарялся, защищая лёгкие от пересыхания.
Разумеется, её костюм тоже был не простым.

Тем не менее, несмотря на экипировку, на необитаемом острове Жест страдала больше других.
Она постоянно жаловалась на холод, и мы стелили ей постель на плите, закрывающей систему отопления.

Но был в её происхождении и плюс --- она единственная из нас могла пить воду любой солёности, в том числе и морскую.
Первые дни пресную воду приходилось экономить и нас мучила жажда --- всех, кроме жилистой йоранки.
Вторым её талантом был талант находить пищу --- тоже, как я думаю, общий для всех уроженцев раскалённой планеты.
Большую часть нашего рациона составляло то, что нашла, поймала или подстрелила Жест.

\asterism

В тот день добыча Жест была скудна.
Она принесла четыре диких яблока, пару горстей каких-то злаков и угря.
Угря съела сама йоранка --- даже неприхотливый Хрыкры отказался от специфически пахнущей рыбины.

Ужин был коротким, и мы как никогда серьёзно разговорились о побеге с острова. 

--- Над морем опасно лететь на аэростате, --- сказала Жест.
--- Если мы попадём в полосу шторма --- это будет очень красивое и последнее в нашей жизни зрелище.
Кроме того, аэростат не утащит провизию на пятерых.
Единственный способ сделать его управляемым --- мускульная тяга, мы будем очень много есть и пить.

--- Плот тоже не годится, --- буркнула Уст.
--- Это неуправляемое судно, а особенности течений мы не знаем.

--- Попробуем сделать примитивный парусник, --- сказал я.
--- Палуба, киль, вёсла.
Хрыкры, ты инженер.
Сможешь спроектировать?

--- Угу, --- отозвался Хрыкры.
--- Но корабль мелкий.
Дерево мало.
Вода с собой мало.

--- Нужен опреснитель, --- догадался Фонитар и тут же, помрачнев, схватился за голову.

--- Угу, --- угрюмо хмыкнул Хрыкры.
--- Металл нет.
Полимер нет.
Обработка ноль.
Но надо.
Жизнь и смерть вопрос.

--- Придумай что-нибудь, --- попросил я Хрыкры.
Друг кивнул.

\asterism

Утром нас разбудил скрип и треск.
Хрыкры, не дожидаясь рассвета, ушёл в одиночку валить деревья.

Мы сраведливо решили, что работать на пустой желудок не стоит, и потратили утро на добычу пищи.
Хрыкры не настаивал.
К обеду у нас было четыре рыбины и целая куча мидий.
Инженер к тому времени осилил десять сосен и сломал походный топорик.

--- Чем мы дальше рубить будем? --- поинтересовался Фонитар, жуя рыбину и разглядывая обломок топора.

--- <<Мы>>? --- буркнул Хрыкры.

После обеда у топорика появилась новая длинная рукоять из сосновой ветки, и инженер снова отправился в лес.
К ужину упало ещё десять сосен.
Ужинать Хрыкры не стал --- он ополоснулся в ручье, упал на свою лежанку и заснул как убитый.

Можно было начинать строить корабль.

\asterism

Каждое утро я начинал с того, что выходил из пещерки и, жмурясь и дрожа от холода, направлялся к расщелине с пресной водой.
Красноватое солнце только-только выходило из-за двух скал, превращая морские волны в холодную радужную плазму.

В то утро солнце меня не встретило.
Светило полностью закрыла собой обшивка лодки, стоящей на скальном стапеле.

Уст была недовольна.

--- Тяжёлая.
Хрупкая.
Неуправляемая, --- бормотала она.
--- Ребят, можно ли как-нибудь сделать смолу погуще?
Ну позорище же.
Лодка растает на полпути.

--- Счастье, что здесь вообще есть такое количество смола, --- буркнул Хрыкры.
Друзья молчаливо с ним согласились.

Пилить доски было нечем.
Хрыкры делал их старинным способом, забивая в щель клин.
Получившийся продукт обладал всеми достоинствами и недостатками, присущими обычным сосновым веткам, в том числе известной прочностью и неизвестной кривизной.
Обработка паром и сушка решали проблему весьма относительно.

Самой сложной задачей оказалось скрепить доски вместе.
Я, Жест и Уст терпеливо, часами высверливали дырки в сосновых досках, стараясь, чтобы дерево не потрескалось.
Фонитар свою норму высверлил гораздо быстрее и веселее --- он обжёг на костре глиняный маховик, сплёл из травы верёвку и сделал верёвочную дрель.
Правда, потрескавшихся досок у него было побольше.

Второй по сложности задачей были паруса.
Нечего было и надеяться целиком соткать их из растительных и животных волокон, даже впятером.
Жест терпеливо собирала на острове любые волокна, которые попадались ей под руки;
этого хватило на редкую, но прочную основу.
Промежутки мы закрыли выделанной кожей, плотными циновками, а также самодельной толстой бумагой, сделанной из опилок, сухих листьев и ваты, а затем пропитанной рыбьим клеем.

--- Франкенштейновский монстр, --- отрекомендовал парус Фонитар, когда работа была закончена.

\asterism

Хрыкры тем временем занимался опреснителем.
Он промазал деревянный бочонок смолой, закрыв все щели, а затем долго и скрупулёзно полировал поршень деревянного ручного насоса.
Возможно, не будь опреснителя, мы вышли бы в море гораздо раньше.
Но Хрыкры не оставил работу над ним, пока не убедился, что каждая деталь имеет по крайней мере один дубликат на случай поломки.

Результат нас обескуражил --- полчаса не самой лёгкой работы давали всего стакан дистиллята.

--- С другой стороны, конечно, это лучше чем ничего, --- признал я.
--- Но, думаю, пресную воду всё-таки придётся взять с собой.

Хрыкры, улыбаясь, вытащил ещё четыре бочонка.

--- Я всё предусмотрел.

\asterism

--- А давайте устроим праздник!
Когда мы ещё сможем оторваться на необитаемом острове!

Предложение Уст было встречено с энтузиазмом.
Мы наготовили всяких деликатесов и сделали маракасы, насыпав в кружки твёрдых бобов.
Фонитар запел протяжную песню на своём переливчатом, многосуставчатом языке.


Я ненадолго отлучился в пещерку, чтобы налить себе отвара.
Жест подошла сзади тихо.
Я не слышал её шагов.

--- Нужно поговорить, --- шепнула она.

Я кивнул, недоумевая.

Жест повела меня подальше от торжества, на берег --- не на песчаный берег, золотистый и ветрено-пряный, а в скалы.
Мы с трудом продрались через сухие заросли.
Зачем Жест попадобилось идти сюда?

Наконец мы оказались в узкой тихой расщелине.
Ветер сюда не долетал.

--- Жест, что случилось? --- спросил я.

Жест вместо ответа сняла очки.
Из них хлынул яркий свет.

--- Эй, тебе это вредно! --- запротестовал я, невольно засмотревшись на её глаза --- странные, бело-золотистые, лишённые зрачков...
В свете очков они выглядели словно большие, наполненные дымом стеклянные шары.
Я ни разу до этого не видел её глаз.

--- Уже всё равно, --- грустно сказала Жест, слепо хлопая веками.
--- Они почти разряжены.

--- Подожди, --- я схватил очки и открыл батарейный отсек.
--- Это же рекомбинируемый атомный аккумулятор, его должно хватать на сто лет!

--- Он и проработал сто лет, --- Жест сухо всхлипнула.
--- Это не мои.
Они очень старые, мне их отдали, когда я прилетела сюда.

Я промолчал и вернул очки ей.

--- В общем, --- Жест с трудом подбирала слова, --- в общем, я скоро буду совсем бесполезна.
Хотя кого я обманываю, --- женщина усмехнулась и села на корточки.
--- Обузой я буду, Кулав.
Кашляющей слепой обузой.

\asterism

Вечером Жест при всех сняла очки, нащупала лежанку и завернулась в одеяло.
Никто не сказал ни слова.
Хрыкры повертел в руках мёртвые очки и вздохнул.
Фонитар и Уст легли рядом с Жест и обняли её, чтобы она не замёрзла.

Никто из нас так и не смог заснуть.
Ночью Жест начала кашлять.
После каждого приступа она выплёвывала несколько грамм прозрачной жидкости, нащупывала в темноте полотенце и вытирала рот.
Вначале она шутила, потом тихо ругалась, потом ругань стала злой, затем злость превратилась в ярость.
К утру она жалобно скулила и плакала, страдая от кашля, ноющего живота и невозможности заснуть.

Едва небо начало светлеть, я растолкал товарищей.

--- Подъём.
Выводим корабль.

--- Что??? --- хором спросили все.

--- Ты сойти? --- осведомился Хрыкры.
--- Корабль не!
Паруса делать, палуба не конец!

--- Кулав, я ещё не определила точное местоположение, --- сказала Уст.
--- Очень большой разброс по долготе.
Мы можем заблудиться!

--- А я спать хочу, --- заявил Фонитар, потирая отёкшие глаза.
--- У меня болит абсолютно всё.
Какое плавание, скажи?

--- Она умрёт, --- просто сказал я, кивнув на Жест.
--- Сегодня хорошая погода.
Ждать больше нельзя.

\asterism

Когда мы грузили на борт опреснитель, Фонитар не выдержал.
Он бросил шланги и сел на песок.

--- Что случилось? --- удивилась Уст.
--- Фонитар!
Эй!

--- Я долго молчал, --- буркнул он.
--- Нет, не так.
Я \emph{чересчур} долго молчал, но больше молчать не могу.
Я не хочу рисковать шкурой из-за неё.

--- Она --- наша подруга, дружище, --- удивился я.
--- Мы работали вместе с ней над <<Вавилоном>>, она вместе с нами строила корабль.

--- И что?

Друзья смутились и переглянулись, не зная, что ответить.

--- Корабль не достроен, и вы это знаете, --- сказал Фонитар.
--- Выйдем сейчас --- и велика вероятность, что мы не переживём и первого шторма.
Сколько штормов у нас впереди, я не знаю.
Да, да, да, --- Фонитар поднял руки, видя, что я собрался что-то сказать, --- я её люблю и уважаю и всё такое, но моя шкура --- это моя шкура.
Можете думать обо мне что угодно, мне не стыдно.
Я не готов идти на смерть из-за того, что она умирает.

--- Она бы ради тебя пошла, --- тихо сказала Уст.

--- Она --- не я.

Я помолчал, собираясь с мыслями.

--- Фонитар, --- начал я.
--- Я мог бы сделать кучу разных неприятных для тебя вещей.
Например, устроить голосование и заставить тебя подчиниться большинству.
Или предложить остаться на острове.
Или, раз уж разговор зашёл о смерти, я мог бы предложить тебе переночевать ещё раз рядом с Жест, чтобы ты слушал, как умирает человек.
Но я этого делать не буду.

--- Интересно почему, --- рявкнул Хрыкры.
--- Пусть!

--- Фонитар много сделал для нас, --- сказал я.
--- Это достойно уважения.
Даю вам час на то, чтобы его убедить --- без криков, давления и прочего.
Просто попробуйте найти нужные слова.

--- А если нет? --- Уст вытаращила глаза.

--- Тогда мы остаёмся.

Фонитар поднял на меня несчастное лицо.

--- Ты действительно готов пойти на это?

--- Я не твой хозяин, --- сказал я.
--- И ты мне не раб.

Фонитар с несчастным видом встал.

--- Загружайте корабль, --- глухо сказал он.
--- Но дайте мне время подумать.

\asterism

Дальше работа шла в молчании.
Хрыкры молча проверял опреснитель.
Уст пыталась натянуть на лодку остатки парусной ткани, чтобы сделать подобие тента.
Я фиксировал балласт.

--- Пойду проверю Жест, --- наконец буркнул я, закончив свою часть работы.

Жест не было в лагере.
Я проверил туалет, пещерку, осмотрел берег, но её не было нигде.
Наконец, догадавшись, я побежал в скрытую зарослями расщелину в скалах.

Вскоре я услышал далёкий кашель и пошёл быстрее.
Я обнаружил Жест очень вовремя --- она легла на каменистую землю, обвязала шею платком и слабыми пальцами пыталась затянуть его туже.

Я без слов стянул платочную петлю с шеи Жест, обхватил женщину за талию, перебросил через плечо и потащил в лагерь.
Она лишилась чувств у меня на руках.

Пробираясь по скалистой тропинке с телом Жест, я нос к носу столкнулся с Фонитаром.
Он всё понял по моим глазам.

\asterism

--- Поверни её набок, --- командовал я.
--- Вот так.
Старайся, чтобы она приняла позу, в которой отходит больше жидкости.
Ей станет легче.

Фонитар молча выполнял мои указания.

--- Ну что, есть на этом свете кто-то, кто достоин риска? --- спросил я.

Фонитар молча поднял на меня полные слёз глаза.

--- Наверное.
Я не знаю.
Просто не хочу, чтобы она умирала.

--- У тебя по-прежнему есть выбор.
Тебя никто не будет обвинять.

--- Ты сам-то в это веришь? --- голос Фонитара стал сухим, из него почти пропали слёзы.
--- Да будут, конечно.
Ты будешь, ребята будут.
И я сам буду.

Я промолчал.

--- Я так хочу жить, Кулав.

--- Я тоже.
И она тоже хочет жить.

--- Если мы утонем, --- голос Фонитара вдруг сделался мечтательным, --- я столько всего не увижу, столько всего не узнаю, столько всего не почувствую...
У меня чувство, что я никогда не был по-настоящему живым.

--- Даже если мы утонем, Фонитар, я тебе обещаю --- ты никогда не будешь настолько живым, как во время плавания на дырявом судне.

Фонитар кивнул и встал.

--- Посмотри за ней.
Я пойду готовить корабль.

\asterism

Отчалили мы в молчании.
Даже лодка, которая должна была скрипеть напропалую, притихла.
Перестала кашлять и Жест;
видимо, за те двадцать минут, которые она провисела вниз головой на моём плече, из неё вышла большая часть жидкости.
Мы положили женщину у штевня, и она тут же заснула как убитая.

Остров, уже успевший стать милым и родным, тихо скрывался в красноватой дымке.

--- Мы когда-нибудь вернёмся? --- пробормотала Уст.

--- А надо? --- осведомился Фонитар.

Мы промолчали.

--- Как ты чувствовать? --- спросил Хрыкры у Фонитара --- всё ещё немного ершисто, но уже по-дружески.

--- Лучше, чем я ожидал, --- лаконично ответил Фонитар.

Вскоре над головой перестали попадаться чайки.
Проводив глазами последнюю одинокую птицу, Хрыкры хмыкнул и вытащил из-под багажа весло.
Мы последовали его примеру.
Одно за другим весла тихо вошли в уключины и так же тихо взрезали водную гладь;
мы очень старались не потревожить сон Жест.

--- Раз, --- нежным голосом командовал я.
--- Раз.
Раз.

Уст похлопала меня по плечу, и я замолчал.
Женщина запела колыбельную, отмечая ею ритм;
я улыбнулся --- её идея оказалась куда лучше моей.

\asterism

Плавание продолжалось уже восемь дней.
Пару раз мы попали в полосу шторма, но всё обошлось;
третья по счёту встреча со стихией грозила неминуемой гибелью, но нам повезло и тут --- на горизонте показался крохотный вулканический остров.
Выбирать между штормом и извержением долго не пришлось.

Мы аккуратно вынесли Жест на небольшую возвышенность, которая была недосягаема для потоков лавы.
Сами же, выйдя на берег с концами, нашли камни потяжелее и приготовились к шторму.

Это было самое потрясающее приключение в моей жизни.
Фалинь рвался из моих рук так, что трещали сухожилия, вокруг дымилась лава, от которой за пять метров несло нестерпимым жаром.
Вода соприкасалась с огнём, поднимая столбы пара;
мы с друзьями весело перекликались, пытаясь перекрикнуть ревущие волны и одновременно удержать мечущееся по огромным волнам судно.

К ночи шторм утих.
Мы кое-как заклинили носовой фалинь корабля в скалах и без сил, кто где, попадали на горячий базальтовый песок.

Следующие пять дней мы провели на острове --- собирали мёртвую рыбу и вялили её над остывающей лавой.
Друзья повеселели, и даже Жест стало гораздо лучше.
Вулкан притих, словно огромный дикий зверь, который внезапно увидел детей.

Ночью Уст сделала неожиданное наблюдение: небо пересёк межпланетный коммуникационный модуль.
Его орбиту она знала досконально и смогла скорректировать курс --- следовало взять ещё на пять градусов к юго-востоку.

На рассвете шестого дня изрядно потрёпанное судно, гружёное вяленой рыбой и прессованными водорослями, оставило вулканический островок за кормой.

\asterism

--- Это ещё что такое?!

Я очнулся мгновенно, словно мне дали пощёчину.
В голосе Фонитара звучала паника.

Уст поняла сразу.

--- Кливер на левый борт! --- заорала она, едва открыв глаза.
--- Поворот на пять румбов!

Тент в мгновение ока оказался у нас под ногами.
Мы вдвоём с Хрыкры кое-как перекинули кливер, спотыкаясь о бочонки, ящики с рыбой, какую-то утварь...
Жест повисла на руле, пытаясь довернуть отчаянно сопротивляющееся судно;
её слепые глаза светились, словно фонари.

--- Давай, милый, давай! --- отчаянно умоляла корабль Уст.
--- Ещё немножко...

Вдруг в темноте слева от нас отчётливо вырисовался громадный бык моста.
Грот-мачта затрещала, выбив сноп искр из свесившегося высоковольтного провода.
Мы в ужасе застыли, наблюдая, как корма стремительно приближается к бетонной стене... и начинает удаляться, не дойдя всего нескольких сантиметров.

--- Руль закрепить, --- бросила Уст в сторону кормы и бессильно рухнула на дно лодки.

\asterism

Мост продолжался и продолжался.
Мы не имели ни малейшего понятия, откуда и куда он идёт.

--- Это не может быть часть Большой Дороги, --- рассуждала Уст.
--- Транспортных линий на нём нет.
Это стопроцентно какой-то технический мост --- один из тринадцати, соединяющих станции Узкого Перешейка.
Но какой именно?
А ведь это важно --- если мы попадём куда-нибудь на Северную станцию, где одна бетонная колонна и ничего нет, придётся плыть двести миль обратно.

Хрыкры тем временем волновало другое.

--- Как залезть, минуя высоковольтные провода? --- говорил он Фонитару.
--- Если буря, придётся отойти от моста, и мы рисковать теряться.
Лучше был бы залезть и идти вдоль.
Наверху всегда есть знаки для инженеров, что куда идти.
Может быть, есть даже транспорт короткое расстояние.

Вскоре все вопросы отпали.
Вдали показалась огромная башня добывающей станции с заметной надписью на боку.

--- Кто-нибудь может прочитать эту надпись? --- грустно спросил Фонитар.
--- Ну или хотя бы сделать разумное предположение, что она означает?

--- Не нужно, --- ухмыльнулся Хрыкры.
--- Цветовая кодировка.
Жёлтый круг --- никель, голубая полоса --- средняя глубина разработки.
Таких станций всего две --- Дождливая и Штормовая Скала.

--- Мост идёт в северном направлении, --- подхватила Уст.
--- Это Штормовая Скала.
Плывём вдоль следующего моста, он выведет на большую землю.

\asterism

Земля.

Мы с Хрыкры даже не стали дожидаться, пока нос лодки зароется в мокрый прибрежный песок.
Мы подхватили Жест под руки, выпрыгнули за борт и пошли вброд.

--- Где мы? --- поинтересовалась Жест, оглядевшись невидящими глазами.

--- Хотел бы я знать, где мы, --- буркнул я.

--- На суше, это главное, --- сказал Фонитар.
--- Можно мне теперь паниковать?

--- Пожалуйста, --- разрешили мы хором.

Фонитар заорал благим матом;
эхо разнесло его крик по прибрежным скалам.
Фонитар заорал ещё раз, и ещё.

--- Вот теперь всё, --- удовлетворённо сказал он после седьмого крика --- слегка охрипшим голосом.
--- Надо было сразу, стало бы гораздо легче.

Вскоре приковыляла Уст, волоча ящик с рыбой.

--- Ну вы вообще нормальные --- жратву бросать на произвол судьбы? --- осведомилась она.
--- Разводите костёр, скоро ночь.
Завтра будем ночевать уже в нормальных постелях.

--- А я думаю --- уже сегодня, --- улыбнулся Фонитар.

Мы обернулись в направлении его взгляда.
По берегу шла команда людей в композитной броне с длинными винтовками.
Они остановились и начали переговариваться, показывая на нас.

--- Ну, если они нас за что-нибудь не расстреляют, --- встревоженно добавил Фонитар.

Группа военных подошла поближе и взяла нас на прицел.
Мы нерешительно подняли руки вверх.
Хрыкры мрачно дожёвывал рыбину.

Оглядев наш скромный багаж и помятые лица, старший что-то сказал подчинённым.
Винтовки опустились.
Старший поманил нас за собой.

\asterism

--- Ты меня слышишь?
Эй!

Я открыл глаза и тут же зажмурился от яркого света.

--- Тьфу, чёрт...
Сейчас уберу лампу, подожди...

Свет погас, и я смог разлепить веки полностью.
Молодой парень в медицинской маске сосредоточенно наблюдал за мной.

--- Ты меня понимаешь?

--- Да, --- просипел я.
Язык слушался с трудом.

--- Отлично.
Я убрал седативную медикацию.
Через несколько минут сможешь ходить и улыбаться, но может быть повышенная тревожность и слуховые галлюцинации.
Ясно?

--- Да, --- повторил я.

--- Вставай-оживай и налево по коридору, на глубокую проверку речевых центров.

--- Где ребята? --- спросил я.

--- Твои друзья в комнате 223, тремя этажами ниже.
Но вначале на диагностику.

--- Мне нужно их увидеть, --- настаивал я.

Молодой врач подумал и оценивающе оглядел меня.

--- Хорошо, --- наконец сказал он.
--- Зайдёшь завтра.
Ну или если вдруг почувствуешь, что забыл, как говорить...
Резко не вставай.
Лифт по коридору направо.

Каждый шаг давался с трудом.
Я понимал, что я ещё не готов ходить, но мне нужно было их увидеть.
Срочно.
Сейчас.
Это была потребность наравне с жаждой или желанием опорожнить мочевой пузырь.

Когда лифт прибыл, я уже почти не держался за стену.
Дверцы мягко разъехались в сторону, открывая немного обшарпанные, старомодные больничные покои...

Вот и комната 223.
Оттуда звучали знакомые приглушённые голоса и тихая музыка.
Я положил пальцы на дверную ручку и, чуть помедлив, резко повернул её.

Дверь распахнулась.

\chapter{Диктатура}

\section{Правда}

--- И что?

--- Да ничего, --- сказала Тан-шая.
--- Они...

В этот момент в комнату вошли вооружённые люди.

--- Имплант на проверку, --- сказал старший.
Все послушно убрали волосы.

Пик.
Пик.
Пик.

--- Что здесь делает йоранка?

--- Я работала на проекте <<Вавилон>>.

--- Как долго?

--- Пять лет.

--- Увести, --- распорядился старший.

--- Стойте!
Стойте! --- закричала Сови.
--- Куда вы её ведёте?

Однако дверь захлопнулась, и вслед за этим на окна опустились решётки.

\section{Экран}

--- Просто выслушайте, --- повторял юноша, --- просто выслушайте меня.
Я могу вам помочь.
Пожалуйста.
Пожалуйста!

Отчаянный крик сменился автоматной очередью.

--- Мало нам было Императора, --- устало сказал Боян.
--- Теперь ещё и Орден Преисподней сюда ломанулся...
Талип, проверь их на демонов и на фиден-паттерны.

--- Что это? --- с ужасом спросил Кевар.

--- Это экран, --- ответила одна из связанных.
--- Кустарный локализованный генератор экрана модели Кельвинова Тьма, эпоха Механик планеты Тси-Ди.
Убить демона не может, мощности маловато, но распознаёт почти всегда, даже спящего.
У спящего демона тоже есть рефлексы защиты.

--- А ты откуда знаешь?

--- Я --- демон, --- пожала плечами женщина.
--- Дьявол бы побрал эти Плеяды.

\section{Экран}

--- И ты вот так просто об этом сообщаешь? --- спросил я.

--- А смысл уже скрывать? --- мрачно осведомилась женщина.

--- Нет смысла, если ты тупая напыщенная малолетка, которая отказывается выполнять приказы, --- прошипела ещё одна связанная рядом.

--- Я объясню тебе всё позже, Хэм.

--- Я с удовольствием выслушаю тебя на трибунале, Ангара Краснобуря.
Если мы до него доживём...

--- Повысить мощность будки до летальной, --- бросил Боян.

--- Покинуть тела, --- прошипела Хэм.
--- Немедленно.

В тот же момент пятеро человек, включая её, бессильно завалились.
Ангара хихикнула.

--- Что произошло? --- удивилась Сови.

--- Они эвакуировались, --- буднично объяснила Ангара.
--- Тела пустые.

--- А ты?

--- А я ещё подожду.

--- Но ты же можешь погибнуть!

--- Насколько я знаю, у вас нет технологий удержания демонов по типу <<Бамбуковой клетки>>.
Только кустарный экран с небольшим зарядом.
Я могу покинуть тело в любой момент.
Мне просто интересно.

--- Ты правда демон? --- спросил Кирд.

Ангара кивнула.

--- Если я дам тебе ключи от наручников, сможешь нас вытащить?
У меня есть ключи, но я не могу ими воспользоваться, потому что наручники считали сигнатуру моего импланта и знают, что я пленник.

--- Я заинтересована.

\ml{$0$}
{--- Пообещай, что ты нас вытащишь.}
{``Promise you will set us free.''}

\ml{$0$}
{--- Не буду.}
{``I won't.}
\ml{$0$}
{Ты либо веришь мне, либо нет.}
{You either believe me, or you don't.''}

Кирд хмыкнул и, придвинувшись к Ангаре, передал ей металлическую капсулу.
В тот же миг наручники на тонких запястьях Ангары расстегнулись.
Миг --- и в её руках непонятно откуда появился пистолет.

--- Работайте в команде, --- шепнула нам Ангара и выстрелила Бояну точно в рот.
В тот же миг завыли сирены.

\end{document}
