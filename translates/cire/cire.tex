\documentclass[twoside,a4paper,12pt,fleqn,openany]{extbook}
\usepackage{polyglossia}
\setdefaultlanguage[babelshorthands=true]{russian}
\setotherlanguage{english}
\defaultfontfeatures{Ligatures=TeX,Mapping=tex-text}
\usepackage{xcolor}
% ----- XELATEX SYMBOL -----
\usepackage{xltxtra}
% ----- XELATEX SYMBOL -----

% ----- HYPHENATION -----
\usepackage{hyphenat}
% ----- HYPHENATION -----

% ----- GEOMETRY -----
\usepackage[left=1.5cm,right=1.5cm,top=2cm,bottom=2cm,bindingoffset=0.5cm]{geometry}
% ----- GEOMETRY -----

% ----- INCLUDE PDF AS PAGES -----
\usepackage{pdfpages}
% ----- INCLUDE PDF AS PAGES -----

% ----- DROPPING CAP -----
\usepackage{type1cm,lettrine}
% ----- DROPPING CAP -----

% ----- FONTS -----
\renewcommand{\baselinestretch}{1.2}
\setmainfont{Linux Libertine}
% ----- FONTS -----

% ------ HYPERLINKS ------
\usepackage{hyperref}
\definecolor{LinkColor}{HTML}{0969DA}
\hypersetup{colorlinks=true, linkcolor=LinkColor, citecolor=LinkColor, filecolor=LinkColor, urlcolor=LinkColor}
% ------ HYPERLINKS ------

% ------ FANCY PAGE STYLE ------
\setlength{\headheight}{15pt}
\usepackage{fancyhdr}
\pagestyle{fancy}
\fancyhead[LE,RO]{\thepage}
\fancyhead[LO]{{\small\textsc{\booktitle}}}
\fancyhead[RE]{{\small\textsc{\bookauthor}}}
\fancyfoot{}
    \fancypagestyle{plain}{
    \renewcommand{\headrulewidth}{0mm}
    \fancyhead{}
    \fancyfoot{}
}
% ------ FANCY PAGE STYLE ------

% ------ ELEMENTS ------
\newcommand{\asterism}{\vspace{1em}{\centering\Large\bfseries$\ast~\ast~\ast$\par}\vspace{1em}}
\newcommand{\textspace}{\vspace{1em}{\centering\Large\bfseries<\dots>\par}\vspace{1em}}
\newcommand{\FM}{\footnotemark}
\newcommand{\FL}[2]{\footnotetext{См. \textit{\hyperlink{#1}{#2}}.}}
\newcommand{\FA}[1]{\footnotetext{#1 \emph{\ml{$0$}{---~Прим.~авт.}{---~Author.}}}}
\newcommand{\theterm}[3]{\textbf{\hypertarget{#1}{#2}} --- #3}
\newcommand{\thesynonim}[3]{\textbf{#2} --- см. \textit{\hyperlink{#1}{#3}}.}
\newcommand{\theorigin}[3]{\textit{#1:} #2 --- #3}
% ------ ELEMENTS ------

% ------ EPIGRAPH ------
\usepackage{epigraph}
\renewcommand{\epigraphsize}{\footnotesize}
\epigraphrule=0pt
\epigraphwidth=8cm
\usepackage{etoolbox}
\AtBeginEnvironment{quote}{\itshape}
\makeatletter
\newlength\episourceskip
\pretocmd{\@episource}{\em}{}{}
\apptocmd{\@episource}{\em}{}{}
\patchcmd{\epigraph}{\@episource{#1}\\}{\@episource{#1}\\[\episourceskip]}{}{}
\makeatother
% ------ EPIGRAPH ------


\begin{document}

ICNC MONOGRAPH SERIES
Civil Resistance
Tactics in the 21st Century
Michael A. Beer

Front cover image (left)
Description: Black Lives Matter Plaza, Washington, DC USA on June 9, 2020.
By Ted Eytan and is licensed with CC BY-SA 2.0.
This image has been modified by cropping.
Link to license: https://creativecommons.org/licenses/by-sa/2.0

Front cover image (center)
Description: Hong Kong Protests at Exeter Uni, England on October 1, 2019.
By Ben David Godson and is licensed with CC BY-SA 2.0.
This image has been modified by cropping.
Link to license: https://creativecommons.org/licenses/by-sa/2.0/

Front cover image (right)
Description: Protest actions in Minsk (Belarus) near Stella on August 16.
By Максим Шикунец and is licensed with CC BY-SA 4.0.
This image has been modified by cropping.
Link to license: https://creativecommons.org/licenses/by-sa/4.0

Civil Resistance Tactics in the 21st Century
by Michael A. Beer (2021)
Published by ICNC Press

Publication Disclaimer: The designations used and material presented in this publication
do not indicate the expression of any opinion whatsoever on the part of ICNC.
The authors hold responsibility for the selection and presentation of facts contained in
this work, as well as for any and all opinions expressed therein, which are not necessarily
those of ICNC and do not commit the organization in any way.

International Center on Nonviolent Conflict
600 New Hampshire Ave NW, Suite 710 • Washington, D.C. 20037 USA
www.nonviolent-conflict.org

ICNC MONOGRAPH SERIES EDITOR: Dr. Maciej Bartkowski
VOLUME EDITORS: Julia Constantine and Steve Chase
CONTACT: icnc@nonviolent-conflict.org

© 2021 International Center on Nonviolent Conflict
Michael A. Beer
All rights reserved.
ISBN: 978-1-943271-40-5

ICNC MONOGRAPH SERIES
Civil Resistance Tactics in the 21st Century
Michael A. Beer

Table of Contents

\chapter*{SUMMARY}

\textbf{AS WITH WEAPONS OF VIOLENCE}, the weapons of civil resistance are numerous, diverse, and ever-evolving.
In addition to strikes, boycotts, mass demonstrations and other widespread
actions, new tactics are regularly being invented as civil resisters adapt to opportunities, challenges, and tactics by their opponents.

The expanding repertoire of nonviolent tactics (sometimes referred to as methods by researchers like Gene Sharp) is a testament to the ingenuity and creativity of activists around the world. Exploring new tactics—the primary purpose of this monograph—is not just a simple documentation or classification exercise. Studying each individual method opens up a world of civil resistance stories in various places and times. Each method offers insight into people’s
perseverance and resilience in the face of repression, demonstrating not only a drive to fight for rights, freedom, and justice, but also the need to be innovative and adaptive in leading resistance struggles.

This monograph opens by introducing terms and fundamental concepts in civil resistance, followed by trends and underlying factors driving the growth of new civil resistance tactics worldwide. It then identifies shortcomings in the current categorization of tactics and offers an expanded list of new tactics as well as a refined framework for cataloging them. Finally, it offers clear takeaways for activists and practitioners, experts, researchers in the field, and others who are interested in supporting nonviolent movements effectively.

\chapter*{Introduction}

NONVIOLENT CIVIL RESISTANCE OCCURS DAILY across many societies in a variety of forms. Examples include indigenous blockades against resource extraction in the Amazon, anti-corruption hunger strikes in Russia, street protests against dictators in the Middle East and North Africa, illegal same-sex wedding ceremonies in India, and whale protection by boat interventions in the Antarctic Ocean.

Coverage of civil resistance movements is increasingly available in many countries. Books\footnote{Some of the key generic books on civil resistance include (this list is not exhaustive or in a particular order): Erica Chenoweth and Maria Stephan, How Civil Resistance Works (New York: Columbia University Press, 2011); Peter Ackerman and Jack DuVall, A Force More Powerful: A Century of Nonviolent Conflict (New York: Macmillan, 2000); Maciej Bartkowski, ed., Recovering Nonviolent History. Civil Resistance in Liberation Struggles (Boulder CO: Lynne Rienner Publishers, 2013); Shaazka Beyerle, Curtailing Corruption: People Power for Accountability and Justice (Boulder CO: Lynne Rienner Publishers, 2014); Robert J. Burrowes, The Strategy of Nonviolent Defense: A Gandhian Approach (Albany: SUNY Press, 1996); Maia Carter Hallward and Norma M. Julie, Understanding Nonviolence: Contours and Context (Cambridge: Polity Press, 2015); Robert L. Helvey, On Strategic Nonviolent Conflict: Thinking About the Fundamentals (Boston: The Albert Einstein Institution, 2004); Sharon Erickson Nepstad, Nonviolent Revolutions Civil Resistance in the Late 20th Century (Oxford: Oxford University Press, 2011); Adam Roberts and Timothy Garton Ash, eds., Civil Resistance and Power Politics: The Experience of Non-Violent Action from Gandhi to the Present (Oxford: Oxford University Press, 2009); Kurt Schock, Unarmed Insurrections: People Power Movements in Nondemocracies (University of Minnesota Press, 2004); David Cortright, Peace: A History of Movements and Ideas (Cambridge University Press, 2008); Mary King, Mahatma Gandhi and Martin Luther King, Jr.: The Power of Nonviolent Action (UNESCO/Cultures of Peace, 1991); Maria J., Stephan, ed., Civilian Jihad: Nonviolent Struggle, Democratization, and Governance in the Middle East (New York: Palgrave Macmillian Series on Civil Resistance, 2010); Gene Sharp, From Dictatorship to Democracy: A Conceptual Framework for Liberation, Fourth Edition (Boston: Albert Einstein Institution, 2010); Gene Sharp, The Politics of Nonviolent Action: Power and Struggle (Part One), The Methods of Nonviolent Action (Part Two), and The Dynamics of Nonviolent Action (Part Three) (Boston: Porter Sargent Publishers, 1973).} and films\footnote{Some of the documentaries on civil resistance struggles include: Bringing Down a Dictator; Orange Revolution; Egypt: Revolution Interrupted?; and A Force More Powerful, all of which are available to stream for free in numerous languages on ICNC’s website, https://www.nonviolent-conflict.org/icncfilms/; and Singing Revolution; Armenian Velvet Revolution; among others.} recount the stories of nonviolent struggles and the ordinary people that led them. The A Force More Powerful documentary, which features historical civil resistance campaigns in India, South Africa, Chile, Denmark, the United States, and Poland, has reached tens of millions of viewers in numerous languages.\footnote{The A Force More Powerful documentary is available at: https://www.nonviolent-conflict.org/force-powerful-english/.}
The Digital Library of Nonviolent Resistance\footnote{The Digital Library of Nonviolent Resistance is available at: http://nonviolence.rutgers.edu/s/digital.} (which houses training manuals collected by Nonviolence International in partnership with the Rutgers International Institute for Peace), and the ICNC Online Resource Library (which offers civil resistance resources translated in many languages by the International Center on Nonviolent Conflict\footnote{The International Center on Nonviolent Conflict (ICNC) was founded in 2002 by Peter Ackerman and Jack DuVall and is a private operating foundation focused on how ordinary people nonviolently struggle and achieve rights, justice, and freedom worldwide: https://www.nonviolent-conflict.org/.}) are just two of many websites that have made “how-to” and research information on civil resistance available on a global scale.

However, despite widespread interest in the subject, the most comprehensive effort to catalog the wide array of nonviolent tactics is still Gene Sharp’s 198 nonviolent methods, an extensive list complete with descriptions, examples, and categories that was published in 1973. Since then, there has not been a comprehensive effort to significantly update this acclaimed list. Nonviolent tactics or methods\footnote{In this monograph, the term “tactics” will be used interchangeably with the term “methods.” Tactics will be defined in detail on page 17. Though broader in its meaning, the term “nonviolent actions,” unless indicated in the text otherwise, is also oftentimes used synonymously with “nonviolent tactics.”} can be thought of as nonviolent weapons or tools that are typically utilized as alternatives to violent (or armed) resistance. As with weapons of violence, the weapons of nonviolent conflict are numerous, diverse, and ever-evolving. A few notable examples include boycotts, strikes, teach-ins, parallel governments, blockades, and marches.

Since 2016, Nonviolence International (NI)\footnote{Nonviolence International (NI) was founded in 1989 by Mubarak Awad to continue his efforts to promote nonviolent action around the world after being expelled from Palestine by the Israeli government. NI is a network of resource centers that researches and promotes nonviolent action and a culture of peace and seeks to reduce violence and passivity worldwide. http://nonviolentaction.net/.} has been collecting and identifying new methods of civil resistance in the Nonviolent Tactics Database.\footnote{The Nonviolent Methods Database was created in 2016 by NI to expand the documentation of methods worldwide. It documents more than 300 methods, with descriptions and examples. The data are available for universal use on https://www.tactics.nonviolenceinternational.net/, and can be made available for use on other websites. Corrections and additions are welcome at info@nonviolenceinternational.net.} This monograph emerged out of this cataloging process and answers three questions:

\begin{enumerate}
\item What tactics did Sharp not identify and what new tactics of civil resistance have emerged since 1973?
\item What new categorization of tactics can be helpful in documenting and understanding this common human activity?
\item How can this new knowledge—on tactics and classification—be helpful to practitioners and scholars of civil resistance, as well as those who would like to assist nonviolent movements?
\end{enumerate}

This new effort also builds on the work of others. Groups including New Tactics in Human Rights, Beautiful Trouble, Gadjah Mada University’s database of nonviolent methods,\footnote{The Department of International Relations at Universitas Gadjah Mada maintains an ongoing database of more than 6,000 events of Indonesian nonviolent actions and methods. This is not yet publicly available.} Şiddetsizlik Eğitim ve Araştırma Dernegi,\footnote{The Nonviolent Education and Research Center in Istanbul, set up by NI in 2012, has a dataset of nonviolent actions in the Turkish language.} the Global Nonviolent Action Database,\footnote{The Global Nonviolent Action Database can be accessed at: https://nvdatabase.swarthmore.edu/.} and the Meta-Activism Project\footnote{The Meta-Activism Project analyzes the field of digital activism. It is available at: www.meta-activism.org.} are also collecting and cataloguing tactics and examples, many of which are now collated into the Nonviolent Tactics Database and introduced in this monograph.\footnote{An additional resource is the website Actipedia, which describes itself as an “open-access, community-generated wiki to document, share, and inspire activists to conduct creative action.” The site serves as an archive of tactics. The website does not currently organize the featured actions into categories that could be useful for comparative analysis. https://actipedia.org/.}

\section*{Why Study Civil Resistance Tactics?}

Studying tactics may inspire and promote action, deepen scholarly understanding, help
recover nonviolent history, improve skills through education and training, and improve stra-
tegic planning.

\subsection*{To Inspire and Promote Action}

Sharp’s list of 198 methods (1973) has been translated into many languages and has inspired
countless activists and educators. Unfortunately, the list is over 45 years old and the described methods were deployed in historical contexts that are sometimes unknown or far removed from contemporary activists and analysts.

Often the biggest barriers to action are despair, ignorance, and fear. These human emotions discourage or blind activists from seeing the wide range of possible nonviolent methods at their disposal. Worse, these emotions sometimes push activists to consider engaging in counter-productive violence. Thus, studying hundreds of nonviolent tactics and examples of their deployment helps to convey the enormous range of actions that are available to humanity, regardless of the political systems they live under.

\subsection*{To Deepen Scholarly Understanding}

The study of tactics is a basic foundation for understanding and researching the nature,
dynamics, and effects of nonviolent struggle. Ronald McCarthy in Protest, Power, and Change
(1997, 320) argues that tactics of nonviolent action: (1) are based on observable phenomena,
independent of views of nonviolent resistance that might vary according to time and locality,
and (2) provide an indicator for the occurrence of nonviolent action in conflicts that, in turn, allows verifiable and replicable research findings to be made. Identifying and cataloguing new civil resistance tactics in new contexts—a primary undertaking of this monograph—therefore enriches our understanding of how the creative agency of ordinary people drives nonviolent resistance and change.

\subsection*{To Recover the History of Nonviolent Resistance}

Violent methods have had a powerful impact on historical change, yet modern life has been strongly influenced by other forms of collective action, including methods of nonviolent resistance and defiance. Maciej Bartkowski (2013) is one of many researchers who have used knowledge of nonviolent tactics to excavate history and uncover protests, strikes, and other nonviolent actions in many societies under colonialism and foreign occupation.
Nonviolence International’s book on Tibetan nonviolent tactics and struggle, entitled Truth
is Our Only Weapon (2000), changed many Tibetans’ understanding of their own history of
resistance. The book quickly became a standard textbook in schools for Tibetan exiles.\footnote{Fourteen years later, in 2014, the International Center on Nonviolent Conflict published a monograph by a Tibetan diaspora activist Tenzin Dorjee that offered a comprehensive historical analysis of the Tibetan uprisings and resistance, starting from the 1950s until the last revolution of 2008 and its aftermath. See The Tibetan Nonviolent Struggle: A Strategic and Historical Analysis, ICNC Monograph Series, 2014: https://www.nonviolent-conflict.org/the-tibetan-nonviolent-struggle-a-strategic-and-historical-analysis-2/.}
Other scholarly works on nonviolent action have changed the foundational myths for the British, American, Russian, and Cuban revolutions that previously centered on armed revolt or legislative enactments.\footnote{These scholarly works include: Ackerman, Peter, and Christopher Kruegler. Strategic Nonviolent Conflict: The Dynamics of People Power in the Twentieth Century. Westport, CT: Praeger, 1994; Bartkowski, Maciej J. Recovering Nonviolent History: Civil Resistance in Liberation Struggles. Boulder, CO: Lynne Rienner Publishers, 2013; McManus, Philip, and Gerald Schlabach. Relentless Persistence: Nonviolent Action in Latin America. Eugene, Or.: Wipf & Stock, 2004; Schell, Jonathan. The Unfinished Twentieth Century: The Crisis of Weapons of Mass Destruction. London: Verso, 2003.} Updating the repository of nonviolent tactics provides us with additional markers to highlight the use of civil resistance throughout history and across different geographies and cultures.

\subsection*{To Educate and Train}

Beyond studying nonviolent tactics, one may also train to use them. Studying tactics is helpful for planning, but training is usually necessary for successful action. In a 2016 study on training for civil resistance campaigns, Nadine Bloch notes that tactics are commonly shared in the form of training manuals, classes, and how-to videos. Bloch points out that training programs improve unity, discipline, and successful attainment of goals (2016, 14).

In order to be successful, nonviolent actions, such as a strike, a banner-hang, or a tree
sit against logging, require skill development. Using a tactic in a particular context often
requires many constituent components, including planning, organizing and logistics, and
training. In case of a possible arrest or risk of injury, additional actions must be undertaken, such as medical care, legal assistance, jail support and solidarity, bail money, documentation, and long-term psycho-social support.

\subsection*{To Improve Strategic Planning}

Without efforts to analyze nonviolent tactics, one cannot discern or devise a strategy for an effective campaign. The strategy that takes into consideration, among others, timing, duration, choice of tactics, required resources, and their sequence of deployment is particularly significant in the battle that the Nobel Laureate, Thomas Schelling (1968), compared to bargaining:

\begin{quote}
The tyrant and his subjects are in somewhat symmetrical positions. They can deny him most of what he wants—they can, that is, if they have the disciplined organization to refuse collaboration. And he can deny them just about everything they want—he can deny it by using the force at his command... It is a bargaining situation in which either side, if adequately disciplined and organized, can deny most of what the other wants; and it remains to see who wins.
\end{quote}

Many nonviolent actions can fail because there is little tactical understanding of how to prepare for, perform, and reap strategic benefits from them. In some cases, activists under- or over-estimate the resources and organizing required, or fail to accurately calculate the costs and risks associated with an action. Identifying which tactics will be most successful in a conflict provides campaigners with opportunities to effectively deploy them as well as identify and assess their desired outcomes.

\section*{Main Findings of this Study}

\textbf{There are more methods of nonviolent civil resistance beyond the 198 methods previously documented} by Sharp and other scholars. Our current Nonviolent Tactics Database includes more than 346 methods (see Universe of Nonviolent Tactics Appendix). Sharp was the first to admit that his compilation of methods was incomplete.\footnote{The Nonviolent Tactics Database is available at: https://tactics.nonviolenceinternational.net.} New civil resistance tactics are regularly being invented or recognized precisely because civil resistance is a widespread, continuous occurrence in a multitude of societies and contexts, perpetually reinventing itself to accommodate for shifting conditions or to utilize new technologies of the day. Taking this into account, this monograph seeks to build on the enormous contribution that Sharp’s work has made to the study of civil resistance worldwide.

\textbf{Digital tactics of civil resistance are numerous, diverse, and widespread.} The most striking new development since Sharp’s book was published is the invention and growth of digital communication, primarily through the Internet. The Internet includes the worldwide web, mobile phones, email, apps, cloud computing, lasers, drones, robots, and “the Internet of things,” such as household appliances, vehicles, and medical devices. Cyber actions, such as the creation of online petitions, Twitter bombs, and social media profile photos, are primarily acts of expression as defined and categorized in this monograph (see Table 1 below). The emergence of online communities and cyber-based work has birthed many new tactics of noncooperation, active disobedience, and constructive action, such as doxing (publicizing someone’s personal information) and crowd-sourcing apps. Modern electronic media amplifies nonviolent actions to wide audiences, thus impacting tactics and messaging.

\section*{Classifying Tactics: A Guiding Framework}

This monograph argues that the universe of civil resistance tactics can be usefully organized
into three general categories:

\begin{itemize}
\item Saying (acts of expression)
\item Not doing (acts of omission)
\item Doing and creating (acts of commission)
\end{itemize}

each of which can be enacted in:

\begin{itemize}
\item Confrontational (coercive) or
\item Constructive (persuasive) ways.
\end{itemize}

The table below illustrates the guiding framework this monograph uses to classify the universe of civil resistance tactics.

TABLE 1: The Universe of Civil Resistance Tactics

Resistance behavior
Saying (acts of expression)\footnote{Saying (expression) is identical to Sharp’s protest and persuasion (1973).}
Not doing (acts of omission)\footnote{Sharp fully developed the understanding of the category of noncooperation.}
Doing or creating (acts of commission)\footnote{Burrowes (1996) may have been the first to divide Sharp’s intervention category into disruptive and creative intervention and to introduce a useful strategic distinction between tactics of concentration (coming together to engage in a specific action) and dispersion (spreading out across different locations to engage in a specific action).}
NATURE OF TACTIC INDUCEMENTS
CONFRONTATIONAL (COERCIVE)
CONSTRUCTIVE (PERSUASIVE)
Protest
Communicative actions to criticize or coerce (example: a march)
Appeal
Communicative actions to reward or persuade (example: a teach-in)
Noncooperation
Refusal to engage in expected behavior (examples: strikes and boycotts)
Disruptive intervention
Direct action that confronts another party to stop, disrupt, or change their behavior (example: a blockade)
Refraining
Halting or calling off a planned or ongoing action to reward or persuade (example: suspending a strike)
Creative intervention
Direct action that models or constructs alternative behaviors and institutions or takes over existing institutions (examples: a parallel government or kiss-in)

Table 1 introduces the concepts and categories of nonviolent tactics, which are the weapons or tools of nonviolent action and vary widely depending on their use in different conflicts. The categorization of tactics is based on broad behavioral resistance domains of saying, doing, and not doing.\footnote{These behavioral domains match precisely Sharp’s categories of protest and persuasion, noncooperation, and intervention.} Tactics are further categorized by the nature of their inducement as either coercive or persuasive. There is a dashed line between protest and appeal actions to signify that many acts of expression can be used for both constructive and confrontational purposes. Some expressive tactics like pray-ins are overwhelmingly on the appeal side. Other expressive tactics like rude gestures are overwhelmingly on the protest side. Yet others such as a speech of defiance can be used to appeal as well as to protest.

It also must be noted that disruptive interventions clearly involve coercive/confrontational inducements, whereas creative interventions are more complex. Even though the latter are clearly associated with the constructive/persuasive side, they might sometimes also operate coercively. For the purpose of this study, this monograph adopts an ideal analytical division with creative interventions assigned only to the constructive/persuasive inducements.

\section*{Monograph Overview}

This monograph begins with an introduction to conflict, nonviolent action, and the components of civil resistance, including the importance of tactics. It then highlights general trends driving the innovation of new civil resistance tactics and discusses the underlying factors that account for that growth.

The study then reviews relevant literature on civil resistance tactics to identify significant contributions on the subject and point out the existing gaps and shortcomings in the current categorization of tactics. It also clarifies the need to compile and examine the new types of tactics that activists have invented and deployed over the past few decades.

The monograph then lays out the criteria for selecting new civil resistance methods and introduces a refined categorization for their cataloging. It presents and elaborates on a sampling of new methods, as identified in Table 1 and Table 4. It also discusses a variety of resistance tactics whose inclusion in civil resistance campaigns and nonviolent movement scholarship and education are controversial.

Lastly, the monograph offers insights into how the understanding of these new tactics and their classification can be helpful for activists and practitioners, experts, researchers in the field, and others who are interested in supporting nonviolent movements effectively.

\chapter{Basics of Civil Resistance}

Large-scale political conflict can occur within routine and regulated resolution procedures such as elections, parliaments, and courts. Conflicts can also be conducted through mass direct action in violently disruptive (e.g., guerrilla war) or nonviolently contentious ways (e.g., strikes or demonstrations). Most conflicts are power struggles. Both violent and nonviolent struggles employ social, economic, political, psychological, and physical pressures and incentives to obtain their goals. In contrast to armed or violent resistance, nonviolent tactics are the key building blocks to civil resistance.

\section*{Defining Civil Resistance}

Close to 100 years ago, Gandhi adopted the English term “civil resistance” as he felt that it most appropriately and comprehensively described the Indian independence struggle against British colonization.\footnote{In his 1935 letter “Servants of Indian Society,” Gandhi wrote: “The statement that I had derived my idea of civil disobedience from the writings of Thoreau is wrong. The resistance to authority in South Africa was well advanced before I got the essay of Thoreau on civil disobedience. But the movement was then known as passive resistance. As it was incomplete, I had coined the word satyagraha for the Gujarati readers. When I saw the title of Thoreau’s great essay, I began the use of his phrase to explain our struggle to the English readers. But I found that even civil disobedience failed to convey the full meaning of the struggle. I therefore adopted the phrase civil resistance.” See Peter Ackerman, “Strategic Nonviolence is not Civil Resistance,” Minds of the Movement, International Center on Nonviolent Conflict, September 21, 2017.} In the decades that followed, several scholars have offered various definitions of civil resistance. This monograph will adopt the definition that Véronique Dudouet provided in her special report, Powering to Peace: Integrated Civil Resistance and Peacebuilding Strategies (2017, 5):

\begin{quote}
Civil resistance is an extra-institutional conflict-waging strategy in which organized grassroots movements use various [...] nonviolent tactics such as strikes, boycotts, demonstrations, noncooperation, self-organizing, and constructive resistance to fight perceived injustice without the threat or use of violence.
\end{quote}

This definition helps us view nonviolent tactics broadly. For example, a specific tactic such as civil disobedience is too often understood as equivalent to civil resistance. This single tactic of breaking laws that are perceived as unjust to achieve a political goal is only a tiny part of the larger civil resistance phenomenon. Many other nonviolent tactics are disruptive but not necessarily illegal such as boycotts, street theater, and hunger strikes. Strikes that may be routine in some countries are considered illegal in others. Context matters, and our definition of civil resistance tries to capture this by emphasizing the extra-institutional nature of actions. Civil resistance has been called by several other names in different contexts, including people power, nonviolent struggle, nonviolent mass action, nonviolent conflict, and
political defiance.

The power of civil resistance derives from the simple fact that rulers cannot sustain their rule and may fall when their soldiers, civil servants, or general population refuse to obey. Physical and coercive repression to stop civil resistance are typical responses from movement opponents. Yet research shows that compared to repression against armed or violent resistance, repression against civil resistance often backfires against the government,\footnote{Brian Martin’s research on backfire documents the frequent political loss of legitimacy and public support when violent repression is used to suppress nonviolent opposition. See https://www.bmartin.cc/pubs/backfire.html.} and helps strengthen the movement by triggering loyalty shifts and even defections among key sectors of society such as the civil service, police, or military, which further limits the government’s power.\footnote{For more information on defections, see: Binnendijk, Anika Locke. “Holding Fire: Security Force Allegiance during Nonviolent Uprisings,” 2009; Binnendijk, Anika Locke, and Ivan Marovic. “Power and Persuasion: Nonviolent Strategies to Influence State Security Forces in Serbia (2000) and Ukraine (2004).” Communist and Post-Communist Studies 39, no. 3 (2006): 411–29; Nepstad, Sharon Erickson. Nonviolent Revolutions: Civil Resistance in the Late 20th Century. Oxford: Oxford University Press, 2011.} Withdrawing cooperation and consent and eliciting defections from regime supporters without violence are hallmarks of civil resistance.

Violent collective action and actions as part of routinized politics within established institutions (such as legislative processes, government regulations, courts, and elections), are not considered tactics of civil resistance, even if they are sometimes intertwined or concurrent with nonviolent campaigns (Bond, et. al. 1997).\footnote{Bond, Jenkins, Taylor, and Schock (1997, 556-559) developed a comprehensive framework for differentiating between nonviolent civil resistance, violent mass action, and legislative/routinized action. The framework is based on three dimensions: contentiousness (non-routine versus routine action), coerciveness (strong to weak inducements), and the outputs and outcomes (violent to nonviolent). Contentious actions can range from routine governmental lawmaking to disruptive illegal occupations and terrorist attacks. Some nonviolent action becomes so routine and predictable (getting arrested for sitting down in front of a government building) that these actions can be considered institutional/regulated actions. Coercive action refers to the negative and positive sanctions or inducements made on an opponent, often measured by the strength or intensity of the action. Coercive actions can include anything from fines and killing at the hands of rulers, to armed theft and blockades. Violent/nonviolent outcomes are the third distinction, which generally refers to damage to persons or property (or not).}

\section*{Mechanisms of Change}

George Lakey (1963) was the first to identify the range of successful outcomes of civil resistance campaigns against a movement opponent. Sharp (1973) later elaborated upon this work. The range of success outcomes includes conversion, accommodation, coercion, and disintegration of a movement opponent.\footnote{Bond further explains the four mechanisms in his chapter, “Nonviolent Direct Action and the Diffusion of Power,” in Justice Without Violence by Wehr, Burgess, Burgess, eds. (Lynne Rienner Publishers, 1994, 70-74).} Withdrawal is potentially a fifth opponent response to a campaign.

\begin{itemize}
\item Conversion is when a movement opponent fully accepts movement goals and rationale.
\item Accommodation is the partial acceptance of movement demands based on the movement opponent’s cost/benefit analysis.
\item Coercion is surrendering to movement demands when sufficient pillars of support for the movement opponent have been removed because of loyalty shifts and defections.
\item Disintegration happens when the public aligns its support with a new government and support for the old government collapses.
\item Withdrawal occurs when an opponent moves away and at least temporarily cedes the field to a movement challenger.
\end{itemize}

Accommodation, coercion, and disintegration are the results of different degrees of power shifts within a society. Civil resistance is based on the fundamental insight that Sharp (1979) articulates as follows: “All governments depend on the sources of power which come from the society, and are made available because of the assistance, cooperation, and obedience of members of that society.” While seeking to persuade the opponent, if possible, civil resistance is primarily focused on wielding popular power to withdraw and restrict social assistance, cooperation, and obedience from key sectors of society.

\section*{Historical Examples}

Some successful examples of civil resistance to defeat dictators include the Solidarity movement (Solidarność) in Poland (1980s), the nationalist movements in the Soviet Union (1990s), the anti-Apartheid movement in South Africa (1980s), and the Islamic Revolution in Iran (1979) (for more, see Ackerman and DuVall, 2000; the Global Nonviolent Action Database; ICNC Conflict Summaries\footnote{ICNC Conflict Summaries are available at: https://www.nonviolent-conflict.org/nonviolent-conflict-summaries/.}). Examples of partially successful global nonviolent movements for human rights and liberation in the last 200 years include women’s rights, lesbian, gay, bisexual, transgender, and queer (LGBTQ) rights, workers’ rights, rights for people with disabilities, indigenous rights, and a great range of environmental causes. Examples of nonviolent campaign failures include the uprising at Tiananmen Square, China (1989), the environmental and minority rights campaign by Ken Saro Wiwa in the Niger Delta (1995), and Zimbabwe’s democracy movement (2010s).

These examples of civil resistance or, as Schell calls it, nonviolent cooperative power (2001) demonstrate the fragility of rulers or majority cultures when they do not have strong support or consent from their society.

\section*{Components of Civil Resistance}

Civil resistance is complex. It usually involves many people and institutions with varying needs, interests, inter-dependencies, and values interacting in complicated societies. One way to deconstruct civil resistance into understandable parts is to break it into stages of conflict, such as Ebert’s three stages (for more on Ebert’s classification, see Table 3)\footnote{There are many models on how civil resistance develops over time. One group of models emerges from the perspective of civil resistance practitioners including: Bill Moyer’s eight stages for social movements; George Lakey’s five components of a nonviolent revolution; the six stages of Kingian nonviolence; and Gandhi’s four stages: first, they ignore you, then they laugh at you, then they fight you, then you win. A second group of civil resistance stages is outlined by conflict managers and peacebuilders who look at conflict from “above” and see variations of stages such as latent, overt, settlement, perceived, felt, manifest, and aftermath in conflict. See for example Curle (1971).} Another way to understand civil resistance (as well as violent resistance) is to look at its common components. These components are described in Table 2, which divides civil resistance into three levels: actions, campaigns, and movements. It shows the relationship between tactics and other common elements within civil resistance. This, in turn, allows us to differentiate a tactic from other components of civil resistance and define it in more precise terms. The three levels of civil resistance also serve as a useful guide for targeting training and education to various audiences.

TABLE 2: Components of Civil Resistance

LEVELS OF RESISTANCE
Action
Campaign
Movement

PURPOSE/ENDS
MEANS
Limited (tactical) objectives
Tactic
Goal(s)
Strategy
Vision(s)
Grand strategy

TYPICAL DURATION
Hours and days (much variation)
Months and years
Years and decades

\subsection*{Tactics}

Nonviolent action, as opposed to passivity, routine activity, and institutional procedures, is the basis of civil resistance. Tactics or methods are akin to nonviolent “weapons” and are “the many individual forms of action” (Sharp 2005, 445). Tactics can be best understood as discrete methods deployed to achieve a limited goal. A hunger strike in prison to demand better food and visiting privileges is a tactic because it is a specific action and has an objective. From 1993 to 1996, in Jakarta, Indonesia, East Timorese students seeking self-determination—for which there was no plausible legal recourse—climbed over fences onto the grounds of the U.S., Swedish, Finnish, Russian, and other embassies, all of which were territories immune from Indonesian arrest without embassy approval. Their tactical objective was to use serial and sustained occupations to make visible, and gain support for, their demands for self-determination to Indonesian and global audiences.

Civil resistance practitioners typically use the term “tactics” instead of “methods.” “Tactics” is also the standard term used to describe forms of action on a limited scale by most social and political scientists. Although the distinction between “method” and “tactic” can be useful for research and educational purposes, the use of the word “methods” in many contexts is an obstacle to public understanding of civil resistance because of its general, ambivalent meaning. In many contexts, the term “nonviolent tactics” substitutes entirely for “methods” and, in other situations, as it will be the case in this monograph, methods and tactics are used interchangeably.

\subsection*{Duration of a Tactic}

The duration of a tactic depends on the objective and therefore varies according to context. Implementing a tactic such as a march or a die-in takes time, resources, and effort—and often risk. Typically, many methods of expression and intervention (flash-mob, parade, etc.) are short-lived and last from minutes to days. Different movement actors therefore deploy methods for short durations and serially or progressively. Tactics such as Palestinians raising their flag in occupied territories during the 1980s had various durations and were implemented serially, perhaps thousands of times. One limited objective of this tactic was to break an unjust law banning display of the Palestinian flag. Another objective was to pressure Israel to use armed actors and resources to respond to the symbolic protests and to exhaust their resources and collective will to enforce the prohibition.

Tactics of omission such as boycotts and strikes are sometimes more sustainable and can last for an extended period of time. In the 18th century, residents of American colonies withheld taxes from the British for five months. When the British repealed the tax, the tax withholding campaign ended. Many other tax withholding efforts have lasted years.

Most nonviolent tactics are deployed in the context of campaigns, but some tactics are deployed with no campaign goals and without a campaign strategy. Uncoordinated tactics are also used by movement allies who are motivated to do something on their own regardless of other strategically planned movement actions.

To be implemented effectively, some methods might require years of study and training. For example, entire books have been published on boycotts, civil disobedience, strikes, and fasting.\footnote{For more information, see: Awad, Mubarak, and Laura Bain. Fasting: Pamphlet No.3 in a Series on Organizing Tactics for Nonviolent Action. Nonviolence International, 2014; Herngren, Per. Path of Resistance: the Practice of Civil Disobedience. Philadelphia, PA: New Society Publishers, 1993; Feldman, David. Boycotts Past and Present: from the American Revolution to the Campaign to Boycott Israel. Cham, Switzerland: Palgrave Macmillan, 2019; Reichard, Richard W. From the Petition to the Strike: a History of Strikes in Germany, 1869 - 1914. New York: Peter Lang Inc., International Academic Publishers, 1991.} Other methods require specialized skills such as sky-writing, or rope-climbing for banner hangs.

\subsection*{Campaigns}

Civil resistance campaigns use nonviolent tactics and organizing to achieve a specific and long-lasting political or social change goal. Campaigns commonly last six months to two years. The Salt Satyagraha (1930-31) in India was a campaign consisting of a salt march to the town of Dandi on the Arabian Sea coast that began on March 12, 1930, as well as several subsequent marches and attempts to occupy salt works. In parallel, the campaign called for nation-wide civil disobedience and boycotting of British goods. This lasted almost a year and ended when Gandhi negotiated with Lord Irwin, the British Viceroy of India, a truce that led to the formalized pact signed between the two on March 5, 1931. Gandhi’s campaign attained two advances: the salt tax was revoked in response to the act of mass civil disobedience near the sea coast, and the British were forced to negotiate in a way that legitimized the equality of Indians.

Some tactics of nonviolent resistance are so impactful that they are strongly identified with a particular campaign. “Defining methods” (Schock, 2012) are used as the central technique around which campaigns are organized. For example, according to Beyerle (2014), Koreans identified political corruption as the largest hinderance for the betterment of the country. Civil society organizations, with the support of the public, created a blacklist of corrupt candidates considered unfit to run for parliamentary seats and designed a nonviolent campaign around it in order to oust corrupt officials from government. In Brazil, landless peasants used occupation as the primary method to meet their needs for subsistence and to campaign for more equal access to natural resources.

The use of one or many nonviolent tactics alone is usually not sufficient for waging a successful nonviolent campaign. A strategy is needed, reflected, among others, in:

\begin{itemize}
\item Sustained, multitudinous participation (Chenoweth and Stephan, 2011)
\item Resilience in the face of repression (Sharp, 1973)
\item Effective targeting of the opponents’ support networks (Schock, 2005)
\item Adherence to 12 strategic principles (Ackerman and Kruegler, 1994), later reduced to six key factors that make movements effective (Ackerman and Merriman, 2015), including unity, strategic planning, nonviolent discipline, participation growth, managing repression, and facilitating defections from opponents’ allies in favor of a movement.
\end{itemize}

A valuable source of information about strategically driven campaigns is the Global Nonviolent Action Database, which is based at Swarthmore College and has documented more than 1,400 nonviolent campaigns in various areas of human activity. It systematically registers the major actors, issues, strategies, and methods used in each campaign as well as the responses of the campaign’s adversaries. It is an invaluable resource for activists, teachers, and researchers.

\subsection*{Movements and Grand Strategy}

A civil resistance movement is a collection of campaigns that aim to implement a similar vision for society. Most movements can take years to achieve substantial success. A few are fairly centrally managed. Others can be geographically spread-out with autonomous local groups adhering to the general goals and directives set by the movement thought leaders. Some examples include the Burmese democracy movement (1990s), the East Timorese independence movement, and the Solidarity movement in Poland. Others are more loosely coordinated, such as the South African anti-Apartheid struggle, the current global movement for environmental justice, and various Arab Spring movements (2010s).

Grand strategy is needed to coordinate campaigns, or at least establish policies that seek to maximize mutual support and minimize internal conflict. For example, suffrage movements in various countries often split between suffragettes who carried out confrontational (coercive) tactics and those who employed constructive (persuasive) tactics and legal methods. Differing visions within movements also create conflict. In Burma in the 1990s, Burmans (the majority ethnic group) wanted an end to military rule because they wanted democracy, while the ethnic minorities wanted an end to military rule because they wanted autonomy and decentralized ethnic rule.

Grand strategy in movements serves as a constant reference—a guiding star. This is particularly helpful during times of rapid growth in movement participation, or when internal conflict arises. Movements can refer to their grand strategy for insight into which strategies and tactics to use, where their priorities should lie, and how to coordinate sequencing of campaign actions. Sometimes adopting a grand strategy helps activists mitigate disagreements over vision, especially when movement support is low or during repression.

Civil resistance movements frequently co-exist with external allies, and sometimes even with autonomous armed elements. In such circumstances, nonviolent movements must develop a viable grand strategy in order to successfully coordinate their strategies, negotiate and compromise with allies, and advance or defend their nonviolent stance.

In the 1990s, nonviolent resistance to the military regime was active in the heartland of Burma through the National League for Democracy, led by Aung San Suu Kyi. Simultaneously, on the perimeter of the country, ethnic minorities primarily engaged in armed resistance against the regime through the National Democratic Front. Internationally, the exile government was called the National Coalition Government of the Union of Burma. Finally, the National Council of the Union of Burma served as an over-arching coordination group for these various opposition groups. The grand strategy of the Burmese opposition required geographic, demographic, and organizational separation to avoid contamination of the nonviolent campaigns as well as to maintain coordination among various opposition groups to maximize synchronicity and cooperation.

\section*{Universality and Context}

Civil resistance occurs in all societies but is an observable phenomenon that varies greatly according to context. For example, the use of civil disobedience varies widely depending on the laws or established norms it challenges. Chewing gum in Singapore, swimming with individuals of different ethnicities in apartheid South Africa, eating in public during the daytime fast of Ramadan in Morocco, or speaking Kurdish in Turkey are all actions that could be considered routine and insignificant elsewhere but are acts of civil disobedience in the specified contexts. However, regardless of the contexts of time, culture, belief system, and location, there are common patterns of defiant methods of action employed to induce a change in opponents’ behavior. Just as atoms and quarks are essential to understanding the physical universe, non-
violent tactics are essential to understanding civil resistance and many modern conflicts.

Now that we have explored a number of broad notions about civil resistance, we can examine nonviolent tactics more deeply.

\chapter{Accounting for Tactical Innovation and Variety of Nonviolent Tactics}

Since Gene Sharp’s publication of 198 methods and, in particular, in the last two decades, researchers have noticed an uptick in the frequency of new, nationwide, anti-authoritarian campaigns (Chenoweth and Stephan 2016). Simultaneously, thousands of anti-corruption, pro-climate, and social justice civil resistance campaigns, to mention just a few, began using of a wide range of highly innovative nonviolent tactics.

What can account for this tactical innovation and deployment of such a variety of nonviolent tactics? Some of the contributing factors, described in greater detail below, include:

\begin{enumerate}
\item Digital technology: growth and documentation
\item Arts-based and cultural resistance
\item Human rights activism
\item Diffusion of knowledge about civil resistance
\item Tactical innovation from women and sexual/gender minorities
\item Resistance to the rise of global corporate power
\item Ongoing repression
\item Competition for public attention
\item Competition for resources among groups within a movement
\item Natural or human-induced disasters
\end{enumerate}

\section*{Digital Technology: Growth and Documentation}

The digital revolution has spurred information technologies that promote expression and communication through digital screens, cameras, electronic speakers, and microphones. Governments use some of these tools to concentrate power, such as mass surveillance and databases of social media activity. But citizens also utilize these tools to disperse power and information, and to sous-veil governments.\footnote{Sousveillance is defined as recording an activity from the point of view of someone involved, usually by a portable recording device.}

Mass numbers of people around the world use mobile phones to text, call, and communicate. Social media networks such as Facebook are being increasingly accessed on mobile devices and have accumulated more than one billion users. Opportunities are ever-present to express disruptive views or make positive appeals, particularly as information transcends geographical boundaries and people are exposed to new political and cultural ideas.

In addition to social media networks, there are a number of major online platforms focused on supporting activism. Avaaz.org, for example, is the largest advocacy organization on the web, with currently upwards of 55 million “members” in 194 countries. Avaaz mobilizes global citizens in 17 languages around topics such as environmental justice, human rights, social justice, and peace issues. It uses electronic petitions and targeted letters as its defining tactics, raises funds for campaigns and humanitarian issues, and crowd-sources new strategies and information. It also mobilizes people to protest in the streets and lobby to challenge governments, corporations, and religious institutions. As an example of its impact, Avaaz claims partial credit for establishing the Brazilian Anti-Corruption Law, based on a petition and thousands of phone calls.

Another example of digital civil resistance is the work of the group Anonymous, a network of online hackers who attack opponents such as governments, corporations, organizations, and private individuals whom they believe are causing harm. Some of the network’s targets have included Sony, Hong Kong Police, a “revenge porn” site founder, and the Saudi Arabian government. In addition to redesigning, caricaturing, defacing or graffitiing websites, Anonymous often reveals private information to the public such as phone numbers, addresses, finances, and social media profiles (known as “doxing”). There is much debate about the ethics and effectiveness of doxing and other forms of secretive or mass attacks in the digital world.

Avaaz and Anonymous are just two of thousands of organizations and campaigns using digital technology in innovative ways to protest and transform perceived injustices. For example, with a blast announcement on Twitter calling for action, hundreds or thousands of individuals can take part in a targeted action by overwhelming a business with calls and other communications.

Today, the opportunity to amplify one’s message or to quickly coordinate actions is unprecedented. It is not a surprise that multiple large, coordinated protests worldwide (organized repeatedly and rapidly) have occurred during the digital information age. According to the BBC, the 2003 protest against the U.S. war in Iraq mobilized up to 11 million people in 60 countries; climate change protests have frequently mobilized more than 1,000 coordinated same-day actions; within just days of a deadly attack on a French newspaper office, the #JeSuisCharlie hashtag was used more than 5 million times on Twitter; and the 2016 Avaaz.org campaign claimed 6 million signatures for an open Internet. Millions around the world mobilized on online platforms such as Tiktok and joined the “I Can’t Breathe” and “Black Lives Matter” protests in the summer of 2020.

With this unprecedented growth in digital methods of civil resistance, various attempts have been made to collect, systematize, and group together a variety of digital resistance tactics. One example is the Civil Resistance 2.0 project. The brainchild of scholar and activist Mary Joyce, Civil Resistance 2.0 is a crowd-sourced effort to explore the possibilities of new technology to expand and update Sharp’s original list of nonviolent methods.\footnote{For more information, see the webinar presentation by Mary Joyce on Civil Resistance 2.0: https://www.nonviolent-conflict.org/civil-resistance-2-0-digital-enhancements-to-the-198-nonviolent-methods/} The initial intention was to focus on the impact of information and communication technologies, but the list was opened up to include any sort of digital or otherwise technological advancement since 1973. The project is housed in a Google Sheet for maximum flexibility. Civil Resistance
2.0 mirrors Sharp’s system of classification. Next to the original method, there are columns for 1) technical enhancements to the original, 2) an entirely new take on the method (or “method 2.0”), and 3) speculative creative variations that have not yet been attempted.

Joyce has observed that traditional methods of non-digital resistance can be amplified via the use of new technologies. Examples include live-tweeting events on Twitter, distributing digital recordings of offline speeches, and publishing text transcripts on websites.

\section*{Arts-based and Cultural Resistance}

Resistance that utilizes arts and culture, including cultural memes, practices and/or traditions, has been an integral part of historical and contemporary nonviolent struggles (Bartkowski, 2013). Some even called this “transformative resistance,” a type of “sophisticated struggle which focuses on mind, speech and action to work towards [individual and collective] liberation ...” (Dorjee 2015, 16). This kind of cultural struggle consists of engaging in self-constructive actions in the midst of repression, such as “promoting [...] language and other self-reliance and cultural capital-building activities“ (18).

Cultural workers, including artists, musicians, poets, and writers have long played a major role in resistance and social change struggles. They have challenged societal norms and assumptions and used art and cultural memes or traditions explicitly to support civil resistance and social change. Many have used their art to protest, hold powerholders accountable, and appeal for social justice. Their solo activities along with the collective work of orchestras, dance troupes, and mass artistic protest have significantly expanded the repertoire of non-violent resistance in the cultural sphere, compelling us to create a more systematic approach for categorizing such actions.

The uptick in cultural resistance in North America has been particularly apparent since the protests against the World Trade Organization (WTO) in Seattle in 1999. Enormous puppets and innumerable props played a central role in the marches and blockades led by protesters. Since then, many large assemblies and protests have been accompanied by art-builds, which are warehouses or spaces that produce large quantities of banners, props, large papier-mâché puppets, large-scale balloons of the earth, and stilts. 350.org, a global environmental organization dedicated to stopping climate change, employs a full-time art director and many of its protests are visually and theatrically staged for maximum appeal and attention. The Amplifier Foundation, an “art machine for social change,” is entirely dedicated to the production and distribution of art as a means to support grassroots campaigns.

Beautiful Trouble (Boyd and Mitchell, 2016) is a book that highlights effective tactics of creative resistance utilized in many campaigns. It was co-written by scores of activists in the Internet cloud and has an accompanying website, network, and training group that seek to promote effective, creative nonviolent actions.\footnote{For more information, see: http://beautifultrouble.org/.} The book’s subtitle, A Toolbox for Revolution, describes its purpose accurately, as it is meant to be readily accessible to activists in need of concrete support and advice.

To account for the massive expansion in cultural resistance methods, Bloch (2015), a long-time trainer, artist, and associate of Nonviolence International, has proposed 12 categories of cultural resistance methods (Sharp did not identify or categorize them).

Bloch’s cultural resistance tactics include:

\begin{enumerate}
\item 2-dimensional arts (graphics/images/words: murals, banners, posters, stickers,
comics, caricatures, logos)
\item 3-dimensional arts (puppets, props, objects, costumes, mascots, sculpture)
\item sound/musical arts (drumming, noisemaking, spoken word)
\item theater arts (guerrilla and invisible theaters,\footnote{Guerilla theater is when activists put on a surprise public performance that is designed to shock the audience, while invisible theater does not seek to be recognized as such and is presented as reality.} traditional, identity correction)
\item movement arts (dance, martial arts, walks, marches, circus arts)
\item media/documentation arts (video, radio, film, archives)
\item literature (newspapers, leaflets, books, poetry)
\item delineation of space (physical spaces/structures which exclude or resist violence/militarization, peace parks/demilitarized zones, peace villages, peace abbies)
\item cultural institution-building (houses of cultural preservation, native language schools, traditional arts schools)
\item crafts and traditions (clothing, food)
\item rituals (national/spiritual/religious celebrations, funerals)
\item language preservation (linguistic rights activists)
\end{enumerate}

Many of these cultural resistance tactics are primarily artistic and expressive in function. Most cultural and artistic activities are typically performed in commercial settings outside of highly charged social conflicts. When they are intentionally used in a conflict setting to appeal to the public or to protest against opponents, we label them as nonviolent actions.

The first seven categories of cultural resistance tactics listed above have been included in this monograph’s updated classification of tactics of expression (see Tables 1 and 4) The cultural resistance tactics in categories 8-12 have anchored or contributed to a form of constructive direct action that this monograph classifies in a self-standing category of constructive intervention. Stellan Vinthagen (2015) refers to these constructive tactics as examples of developing a “nonviolent society’s culture of resistance alternatives,” as long as they are dually “intervening in relationships of domination and implementing an alternative way of living.”\footnote{Vinthagen (2015) outlines nine components of nonviolent culture: 1) socialization, 2) socio-material reproduction, 3) cultural reconstruction, 4) movement stories in daily life, 5) movement rituals and ceremonies, 6) free zones, 7) group boundaries, 8) movement-oriented interpretive frameworks, 9) the politicization of daily life.}

\section*{Human Rights Activism}

Another reason for the development of new civil resistance tactics relates to the global movement for human rights. In the 1970s, human rights activism and institutions grew in prominence with the support of Amnesty International, Human Rights Watch, the United Nations, and iconic dissident figures of rights-based movements like Solzhenitsyn, Mandela, Havel, and Mothers of the Plaza de Mayo. A powerful new framework for local grassroots activism based on international law has continued to grow, supported by many allies in the international human rights community including labor unions and human rights groups such as Human Rights First PEN, Frontline, and a number of national organizations.

Beyond defense lawyers, human rights defenders now include anyone upholding international laws and conventions, such as journalists and writers defending Article 19 of the UN Declaration for Human Rights, women activists campaigning for gender equality, and indigenous people struggling for survival. Other defenders include minority groups fighting non-violently for self-determination, political parties demanding free and fair elections, students advocating for universal education, and workers defending labor laws enacted by the International Labor Organization.

International human rights law establishes minimum standards of dignity that governments have voluntarily signed to uphold but often fail to effectively implement. There is now a UN Rapporteur for Human Rights Defenders, as well as a reformed UN Human Rights Council with Universal Periodic Reviews monitoring how member states implement human rights law. The combination of local and international human rights pressure has proved formidable in bringing social change to countries such as Colombia and El Salvador (Wilson, 2017).

The New Tactics in Human Rights (NTHR) database is a valuable resource for human rights defenders. The NTHR is a multi-language site designed to help human rights campaigners deploy tactics used in other campaigns. NTHR has documented hundreds of tactics promoting international legal standards. In many communities, human rights defense has supplanted nonviolent resistance as the major strategic avenue for social change. Human rights defenders traditionally work more closely with the courts and legislatures, seeking to hold them accountable and to uphold their stated ideals. Still, many tactics that human rights defenders deploy, such as vigils and funeral protests, are methods of nonviolent resistance. Some of the examples of nonviolent action harvested from NTHR for the purpose of this
monograph are mock tribunals, guerilla lawyering, and the creation of fake money to combat bribery (see Universe of Nonviolent Tactics Appendix).

\section*{Diffusion of Knowledge About Civil Resistance}

Although a digital divide still exists in many parts of the world, and digital surveillance and censorship are common practices, the Internet has contributed significantly to the worldwide spread of knowledge about civil resistance. Before the Internet, the efforts of scholars, educators, and trainers to teach nonviolent resistance were confined to in-person interaction or telephone calls—and the financial and logistical limitations that came with them. Improved Internet access, both in terms of infrastructure and affordability, as well as connectivity advancements in many parts of the world, has enabled educators to explore less costly and more easily scalable educational avenues. Today, a number of universities and civil society actors offer online, offline, and hybrid courses on nonviolent civil resistance and nonviolent action. Measuring the impact of online courses on participants remains a challenge and is still rather limited. However, refined survey tools are being developed to provide important insights into the short and long-term changes in attitudes, skills, and knowledge of participants as a result of online learning on civil resistance (Bartkowski, 2019).

Documentary films have also served as a powerful educational tool. Documentaries examining cases of civil resistance around the world such as A Force More Powerful, Bringing Down a Dictator, and Orange Revolution have reached tens of millions of viewers. More broadly, one may speak of a sort of “documentary resistance,” which, drawing on the work of Bartkowski (2013) and movement historian Jacques Semelin, Amber French (2017) describes as “[e]fforts to document, preserve, and transmit information about past nonviolent movements [as] acts of civil resistance, [including] raw traces of movements (photos, videos, etc.)... .” Two primary groups can engage in documentary resistance:

\begin{enumerate}
\item Activists/movements, through generating leaflets, posters, videos, leaders’ speeches, interviews, and so on.
\item Historians, journalists, documentary directors, and others who write and create films about, and/or study past movements, producing accounts and analysis that today’s movements can refer to, consult, and learn from.
\end{enumerate}

The underlying concept of documentary resistance is defying repression (whether in the form of movement repression or of censored historical narratives) so that future movements can learn from the experience and actions of past movements.

Language translation of resources on civil resistance is another important development in the dissemination of knowledge. Sharp’s works have been translated into 36 languages, and the ICNC online Resource Library features an extensive and growing list of English-
language and translated resources in over 70 languages available for free download.

Information technologies have provided citizens with ways to circumvent communication restrictions established by authoritarian regimes who routinely seek to limit and/or regulate information about resistance campaigns at home and abroad. The use of information technologies to broadcast the 2010-11 protests against Tunisian dictator Zine El Abidine Ben Ali spurred a contagion effect of protests first in Egypt and then throughout the Arab world, known as the Arab Spring.

With the diffusion of knowledge about nonviolent actions across regions, countries, and communities, there is greater potential for reflective learning that spurs tactical innovation and helps wage nonviolent campaigns more creatively and effectively.

\section*{Tactical Innovation from Women and Sexual/Gender Minorities}

Women and girls have been deeply engaged in many nonviolent campaigns throughout history, often displaying strong tactical innovation and creativity. While men have often used mass violence in conflicts, women have mostly waged conflict through nonviolent action. According to scholars Mary Elizabeth King and Anne-Marie Codur:

Women pioneered many of the hundreds of nonviolent methods that have been observed throughout history... Women activists have been able to capitalize on five sociological features that gave them an edge and a specific advantage in civil resistance movements, compared to men:

\begin{enumerate}
\item Women’s presence lowers the level of violence and repression from security forces.
\item Women are often the best keepers of nonviolent discipline inside the movement.
\item Women tend to organize in horizontal networks, which prove to be more efficient and resilient for a movement than male-run hierarchical systems.
\item Women display fewer internal rivalries and can frequently achieve greater unity than male activists.
\item Women show a greater ability than men to build ties of solidarity across and beyond societal lines of division (be they religious, ethnic, or socio-economic) and political rivalries. (Kurtz and Kurtz, 2015, 433).
\end{enumerate}

Historically, women have championed dispersed mass actions such as boycotts and expressive protests. The cacerolazo form of protest, which consists of banging on pots and pans as part of a march or similar event, was invented by women and used in dispersed actions in repressive places.

Yet many women-led nonviolent struggles using concentrated tactics have also left their mark on history. Since the early 2000s, Women of Zimbabwe Arise (WOZA) has participated in the annual “March for Love” on St. Valentine’s Day, campaigning to end torture and authoritarian rule. In the 1970s and 80s, the Mothers of the Plaza de Mayo in Argentina catalyzed mass protests and citizen mobilization by holding a public vigil with portraits of their disappeared children.

Women’s capacity for tactical innovation has time and again transcended restrictive gender roles. Many women have risen up to exploit these societal expectations to their own strategic advantage. King and Codur argue that traditional gender roles can offer women:

\begin{quote}
... a strategic advantage, because they allow them to function within customary stereotypes while utilizing the tools that can be employed as weapons against certain aspects of the society’s patriarchal structures... As they challenged the authorities in the name of those “superior” family values, as good wives and devoted mothers, they were able to create powerful, irresistible dilemma actions.\end{quote}

At the same time, there are numerous examples of women engaging in civil disobedience against restrictive, gender-based laws and cultural expectations in societies around the world. Examples include wearing pants instead of skirts and dresses, voting, driving vehicles, participating in a non-conforming marriage, smoking cigarettes, making legal contracts, and refusing sati (self-immolation of widows).

Furthermore, the evolution of modern gender roles in many regions of the world has opened up new resistance horizons for women in professional, social, and political arenas. Women have run underground abortion and medical clinics in numerous countries. In Liberia in 2003, women exposed their nude bodies in public in a successful effort to shame men and pressure them to sign a cease-fire agreement. In Burkina Faso in 2014, women played a prominent role in resisting dictatorship by “emerging from their homes, waving spatulas in the air—a rare sign of disapproval in Burkinabé culture which signaled to others across the country the degree of seriousness the resistance was reaching” (Zunes 2017).

Importantly, King and Codur, as well as Marie Principe (2016) point out the regenerative and reinforcing effects that women’s nonviolent organizing and mobilization have had for broader struggles in many societies. For example, the women’s petition movement of Iran in 2006 served as a foundation for the 2009 Green Revolution. Conversely, women’s leadership in the U.S. civil rights movement helped catalyze feminist tactics of the 1970s and later decades, including decentralized affinity-group based mobilizations, de-genderizing bathrooms, breast-feeding in public, creating women-centered religious groups and rituals, and using non-sexist language.

In the 20th century, gay, bisexual, and lesbian people publicly broke gender norms by romantically and sexually loving people of the same sex. This has led to all sorts of resistance such as kiss-ins, gay and lesbian sex, erotica and pornography, underground pharmacies, and cross-dressing protests. Tactical innovations included handcuffing oneself to live news broadcasters, running underground pharmacies, throwing glitter, and developing the rainbow flag as an identity symbol.

Understanding civil resistance tactics through a gender lens remains a relatively unexplored area. More research is needed to identify and document the contributions of gender and sexual minorities, including transgender people, intersex individuals, and many others, to nonviolent resistance in general and tactical innovation in civil resistance, specifically.

\section*{Resistance to the Rise of Global Corporate Power}

For hundreds of years, a driving force behind economic globalization has been transnational corporations (TNCs). These entities were backed by European powers during colonial times and, in modern times, they have been supported by world powers such as the United States and China as well as former colonial powers such as France and the United Kingdom, among others. TNCs dominate local governance, which can have harmful effects on multiple levels of society, particularly in countries with weak governance. Examples of TNCs currently in operation include oil companies Shell in Nigeria, Total in Burma, and Hunt Oil in Peru; gold extraction companies Freeport-McMoRan in West Papua and Centerra Gold in Kyrgyzstan; and timber extraction companies operating in the Congo Basin, the Amazon Basin, and in various Southeast Asian countries.

Relatively weak local communities with little leverage over powerful TNCs obtain few tax revenues or profits from resource extraction and are stuck with the long-term health and environmental costs. Most of the related nonviolent campaigns that local communities lead, often with international allies, have been focused on strategic frame that has driven tactical innovation called \textbf{points of intervention} (as identified in the quote below). Effective opposition to these global firms has required a global response, facilitated by modern communication and travel. As noted by Patrick Reinsborough and Doyle Canning (2008) at StoryBasedStrategy.org:

\begin{quote}
[tactical] interventions come at many places—from the \emph{point of destruction} where resource extraction is devastating intact ecosystems, indigenous lands, and local communities, to the \emph{point of production} where workers are organizing in the sweatshops and factories of the world. Solidarity actions spring up at the \emph{point of consumption} where the products that are made from unjust processes are sold, and inevitably communities of all types take direct actions at the \emph{point of decision} to confront the decision makers who have the power to make the changes they need.
\end{quote}

Another common intervention occurrence comes at the \emph{point of transportation} in the form of blockades of ships, trains, and other vehicles. Greenpeace and Indonesian campaigners have been working to preserve rainforests from palm oil plantations. In addition to direct efforts to halt logging, activists have also tried to blockade ports, blockade palm oil factories, promote boycotts of palm-oil products, and organize protests aiming to pressure palm oil executives (Schlegel, 2016).

Environmentalists have innovated at the point of destruction through the use of tripod blockades in which a large tripod device is set up on roads or in front of properties and a single person climbs to the top of the device. The tripod cannot be moved without risking serious harm to the person. A single device and one person can be used to block a large expanse which makes this tactic quite efficient.

Another innovation is identity correction as practiced by the Yes Men, who impersonate corporate executives who “apologize” and “offer reparations” for transgressions such as the Bopal industrial accident. Dow, the accident perpetrator, chose to quickly disavow that apology and reparations offer—an act that caused further reputation damage. Anti-whaling groups have intervened non-lethally with whaling harvesters by shining laser light at whalers, to interpose between the whales and the poachers, to chase and shadow whalers and seize and damage drift nets at sea. Given the ubiquity of transnational corporation activities and the lack of international laws and mechanisms to rein in their power, civil resistance campaigners have expanded their repertoire of methods to intervene accordingly.

\section*{Ongoing Repression}

Authoritarian regimes routinely repress and retaliate against the independent actions of civil societies and are constantly inventing and experimenting with new means of doing so (Burrows and Stephan 2015). This authoritarian pushback tends to spur tactical innovation on the part of repressed but still mobilized groups. Authorities are not prepared for this. To minimize the costs as well as risks of being repressed, activists are incentivized to experiment with a variety of nonviolent resistance tactics.

In response to the Chinese government’s attempts to censor Internet “speech” by banning the use of specific words, netizens relentlessly invent new homophones and homonyms to communicate important information and to support free speech (Shaou and Dodge 2017). In another creative response to repression, the Sahrawi people abandoned the capital city of Laayoune in 2010 in a mass protest against Moroccan occupation of their region and built an alternative tent city. The Moroccan government eventually destroyed the tent city. Sahrawi activists responded to the repression and the ban on kheimas (the traditional nomad tents and the platforms of social life) by pitching them on the roofs of their homes. In Hong Kong in 2019, citizens formed a symbolic 32 kilometers of human chain, held up thank you signs to many foreign supporters, and deployed the mass use of laser pointing in response to police repression.

\section*{Competition for Public Attention}

Movements often engage in competition for the public’s attention through mainstream news media. Campaigns also compete in “bandwidth” with business and charity sector advertising. The frequent desire to be noticed by the mainstream news, combined with the understanding that the same repetitive actions can fade in their effectiveness and newsworthiness over time, drives some movements in a relentless creative process and search for “new” attention-grabbing actions.

People for the Ethical Treatment of Animals (PETA) is famous for its motto, “All news is good news.” This motto suggests that even controversial actions that may anger many in the public, such as throwing blood on people wearing fur coats, or disrobing, are successful because they garner publicity and raise awareness about a grievance. Most movements do not subscribe to this extreme strategy of attention-getting, but some do.

\section*{Competition for Resources among Groups within a Movement}

Cunningham, Dahl, and Fruge (2017) statistically analyzed the diffusion and diversification of tactics used in self-determination struggles from 1975 to 2005 and found that: “Given limitations on their capabilities, competition among organizations in a shared movement, and different resource requirements for nonviolent strategies, we show that organizations have incentives to diversify tactics rather than just copy other organizations.”

They statistically demonstrated that when some organizations use resource-intensive methods such as boycotts or mass protests that are successful, others are incentivized to copy them. However, in many cases, activists try less resource-intensive tactics such as a blockade or an online petition. Internal competition for resources, the impact of tactics, and public support for using specific tactics all drive creativity and diversification of tactics.

One campaign was able to achieve an impact with a less resource-intensive tactic called maptivism. This method was developed in conjunction with the women’s rights campaign in Egypt and resulted in the 2015 creation of Harassmap, an interactive app that displays crowd-sourced reports of harassment on a map. This tactic required only a few app developers working as volunteers and willing citizens to report incidents. In Tunisia in 2011, one group initiated an online campaign of self-portraits of individuals holding signs reading “Enough Ali” to protest then President Zine El Abidine Ben Ali. These photographs were collated online in a mosaic fashion to demonstrate the enormous number of Tunisians who were calling for the president’s removal.

\section*{Natural or Human-induced Disasters}

Pandemics and ecological disasters are not new to humanity. However, the change in scale, severity and speed of these disasters has generated new opportunities for tactical innovation. The 2020 COVID pandemic is an example of a disaster which led to numerous tactical innovations. For example, KPop fans on Tiktok flooded the Trump Campaign with ticket requests to his Oklahoma rally, in order to deprive his supporters of the opportunity to attend the rally and to mislead the organizers about the number of attendees. California activists deployed a virtual (decentralized) art build, photographing their work which was stitched together to spell a message. Millions of people produced face masks on home sewing machines and communities mobilized to collect and distribute protective equipment without government support. In Honduras, activists turned bank notes into face masks as a symbol of their demands for government transparency. Online rallies for numerous causes were held.\footnote{For more information, see: Chenoweth, Erica, and Jeremy Pressman. “Collective Action & Dissent under COVID: Crowd Counting Consortium.” Google Sites, 2020. https://sites.google.com/view/crowdcountingconsortium/dissent-under-covid.}

The points in this section are not exhaustive but instead aim to explore some of the dominant trends in tactical innovation. Given the deployment of nonviolent tactics in an array of civil resistance movements, in all countries and in most sectors of modern societies, there are surely other factors that are driving the use of new nonviolent methods and the tactical innovation of old ones.

\chapter{Categorizing Nonviolent Tactics}

This chapter starts with an overview of Gene Sharp’s work on nonviolent methods because of its significance and popularity. It then adds insights from Ebert, Gandhi, Bond, Bloch, and others to synthesize a more inclusive classification system of nonviolent tactics.

The chapter concludes by presenting alternative research and ideas for classifying tactics such as those proposed by Beyerle, Vinthagen, and others, with the hope that this area of civil resistance study may be more fully integrated into our understanding of old and new tactics and how activists use them.

Sharp’s Classification of Nonviolent Methods

Sharp may not have been the first to begin identifying nonviolent methods, but he was the first to begin systematically documenting them. In the 1950s, Sharp identified and catalogued various types of nonviolent resistance methods that he came across through a painstaking review of numerous historical conflicts and movements. After first compiling a list of nonviolent tactics, which he called methods, he attempted to classify them systematically. Sharp developed a typology that organized this complex field of human behavior into three distinct categories of resistance: A) protest and persuasion, B) noncooperation, and C) intervention. These categories have proven durable and yet are flexible enough to easily include newly emerging methods.

\begin{enumerate}
\item Protest and persuasion tactics are group actions designed primarily to “say” or express protester grievances and demands. This category commonly includes assemblies, processions, theater performances, public speeches, puppets, petitions, videos, and virtual marches. Protest and persuasion convey the two primary purposes of expressive action: to communicate what a movement is campaigning “for” and what a movement is campaigning “against.” However, “persuasion” can be misconstrued to imply that actions that do not succeed in persuading an adversary don’t satisfy the definition of a nonviolent method. In this monograph, we call this functional method “appeal” which has a more unilateral meaning.
\item Noncooperation tactics are acts of “not doing” and typically operate by withdrawing cooperation from opponents. Sharp collected and categorized 100 different kinds of noncooperation methods and divided them into three subcategories: social, economic, and political. Social noncooperation includes ostracizing people, suspending social and sports activities, staying at home, and collective disappearance. Economic noncooperation includes 25 forms of boycotts (refusal to buy) and 22 forms of strikes (refusal to sell). Political noncooperation includes citizen and civil servant noncooperation with government, noncooperation actions by states, and tactics such as slow compliance, disguised disobedience, hiding, and civil disobedience of laws considered illegitimate.
\item Intervention tactics are generally disruptive acts of “doing” that seek to physically or psychologically interfere with opponents’ actions or manipulate the conditions. These include nonviolent land seizures, reverse trials,\footnote{Reverse trials are defined as courtroom trials where the accused assumes the role of a prosecutor putting on trial a law, policy, or an issue (Sharp 1973).} subvertising,\footnote{Subvertising is defined as co-opting an advertisement for a subversive or originally unintended goal.} tree sits, and guerrilla theater. This category also includes tactics that develop alternative institutions and social norms and practices such as alternative markets, transportation, communications systems, social institutions, and parallel governments.
\end{enumerate}

Sharp’s classification system also designates numerous sub-categories of methods. Some sub-categories refer to arenas of action (social intervention) or specific actors (economic noncooperation on the part of workers and producers). Others point to intentions (pressure on individuals, honoring the dead) or the material medium used (drama, music). However, the methods of nonviolent action are highly diverse and context-driven, making it difficult to subcategorize them effectively and consistently. For example, Sharp’s sub-categories of protest and persuasion are particularly eclectic. Table 4 attempts to catalogue these methods more consistently by grouping them according to the medium in which they are conducted: digital technology, material arts, the human body, or language.

Sharp’s definition of tactics has a few drawbacks. Some identified methods might not easily fit into one category because they involve a mixture of doing, not doing, and saying something. Sharp also equates his nonviolent methods with military weapons used against an adversary and, in doing so, excludes many actions that engage society but do not directly target a movement’s adversary. Excluded from Sharp’s methods are, for example, constructive actions such as educational, advocacy, and organizational efforts designed to pull support and loyalty away from the opponent and enhance the social cohesion and self-organization of the repressed society. In reality, most campaign successes are determined by obtaining majority support (rather than unanimous support) from society or other stakeholders; conversion of leading opponents rarely happens. Another drawback is his decision to ignore categorical distinctions between coercive and persuasive methods.

Sharp also warned against inferring that there is a ready-made template for selecting and sequencing his identified categories of tactics to execute successful nonviolent struggles. By and large, however, the simplicity of Sharp’s classification system has been instrumental in its popularity among activists and near-universal adoption among civil resistance trainers, teachers, writers, and scholars.

Keeping in mind Sharp’s enduring classification typology, what follows is an overview of other ways civil resistance practitioners and analysts classify tactics, all of which inform the universal framework developed in this monograph (Tables 1 and 4).

\section*{Disruptive and Constructive Resistance}

No discussion of nonviolent tactics is complete without recognizing the central contributions of Mohandas Gandhi. Gandhi invented much of what we consider modern-day nonviolent action. He coined and popularized many of the field’s terms, practices, and philosophies. Gandhi called his overarching philosophy satyagraha (by which he meant civil resistance). He strongly believed in the importance of consistency between the vision of a new India and the means it deployed to achieve those goals, including maintaining empathy for its opponents, the British. Persuasion, nonviolent coercion, and constructive action were all strategies that he deployed at various stages of the pro-independence struggle (Ackerman and DuVall 2000; Cortright 2008; Sharp 1979).

Gandhi is well-known for his use of hunger strikes, utilizing the dramatic and contentious nature of these fasts to draw attention to his concerns. He often used this method to induce cooperation from other members of the Indian independence movement, in addition to pressuring the British government. His support for economic boycotts of British goods had a strong coercive impact on British society and the British government. These, along with tactics such as filling the jails and overwhelming the judicial system, are examples of disruptive and confrontational actions.

Gandhi coined the popular term “constructive program” to describe his efforts to improve the lives and discipline of his peers, as well as to strengthen their self-reliance. Through the use of cotton weaving, agricultural ashrams, and salt-making, and the development of new social norms to reduce caste and gender barriers, Gandhi attempted to lay the foundation for Indians to build alternative social and economic institutions to serve as competition to the oppressive British system.

As part of constructive actions, Gandhi also practiced the unusual tactic of unilateral rewards. For example, Gandhi called off (active abstention) a mass march on January 1, 1914 in South Africa, because of an unrelated railway strike. He stated that his campaign did not want to take advantage of the weakness of the government hit by the ongoing strike. The British responded favorably to this unilateral goodwill gesture and entered into negotiations which produced an agreement that gave Indians limited civil rights. Unilateral rewards or offers are unusual in contentious conflict.

Constructive resistance was practiced well before Gandhi—for example, in the 18th century American colonial struggle against the British (Conser, McCarthy, Toscano and Sharp 1986). Some researchers have also highlighted the role of routine, constructive actions in resistance. Bartkowski (2011), for example, showed that this type of action was part of the repertoire of nonviolent resistance that strengthened the social fabric of the emerging Polish nation in the late 19th century and built up cultural and economic institutions that proved instrumental in re-establishing the Polish state after World War I. Polish families sent their children to Polish language schools, often at some political or economic cost to them, to strengthen Polish identity, shared narratives, communication, and social bonds. Such daily actions made up a constructive program of cultural, linguistic, and socio-economic organizing, which was disruptive to the German, Austrian, and Russian authorities. Bartkowski (2011), along with Burrowes (1994), sub-classifies these constructive methods as creative interventions that are distinct from disruptive interventions, such as blockades and hunger strikes.

A challenge in classifying constructive actions is differentiating between inward-oriented actions and outward-oriented (direct) actions. Inward-oriented constructive actions are designed primarily to strengthen internal cohesion and skills of a resisting group or a campaign, such as educational activities and training. Outward-oriented constructive actions, such as persuasive pray-ins or Indians making salt in colonial India, intend to directly induce an opponent to change their actions or even worldviews.

This monograph incorporates into its analysis outward-focused constructive actions that challenge an opponent via persuasion or coercion, or both, and excludes from its analysis constructive actions aiming internally to build or strengthen a movement, as they fall more into training and capacity-building activities and out of the scope of this analysis.

\section*{Ebert’s Categorization of Nonviolent Tactics}
% Классификация ненасильственных тактик по Эберту

Subsequent to Mohandas Gandhi, Clarence Marsh Case (1923), Krishnalal Shridharani (1939), Joan Bondurant (1958), and Anthony de Crespigny (1964) were among the first to formulate an understanding of nonviolent tactics. Theodore Ebert (1970) built on their work and developed a robust collection of different forms of nonviolent action and a theory on how these methods work. He classified nonviolent methods as either confrontational or constructive. According to Ebert, confrontational actions are intended to stop or reverse an opponent’s actions; constructive actions are persuasive or innovative in nature and intended to build a
just order in society.

Confrontational actions include:

\begin{itemize}
\item protests (flyers and marches);
\item legal noncooperation (boycotts and slow-down strikes); and
\item civil disobedience (tax refusal and general strike).
\end{itemize}

Constructive actions include:

\begin{itemize}
\item presenting alternatives (seminars);
\item legal innovative activities (founding educational institutions, newspapers, and mutual
aid societies); and
\item civil usurpation (invasions, takeovers, and self-governing bodies). Ebert posits that civil
usurpation actions are used in increasingly forceful phases or stages.                                           \end{itemize}

TABLE 3: Ebert’s Classification System (1970)

NATURE OF ACTION

Stage of escalation
1. Actions that bring the issue into the public arena
2. Legal actions that deal with the issue
3. Illegal actions that deal with the issue

CONFRONTATIONAL ACTION
Actions that are directed against injustice in society
Protest
(Demonstration, petition, leafleting, vigil)
CONSTRUCTIVE ACTION
Actions that help construct a just order in society
Presenting alternatives
(Teach-in, lectures, show alternative)
Legal noncooperation
(Strike, consumer boycott, go-slow)
Legal innovative activities
(Fair trade, free school, alternative economy, ethical investment, nonviolent intervention)
Civil disobedience
(Sit-in, blockade, tax resistance, strike, war resistance)
Civil usurpation
(Sanctuary movement, pirate radio, reverse strike, nonviolent intervention)
How it works
Publicizing/convincing
Raising the stakes (costs) and minimizing the rewards for those committing injustice
Redirecting power

Ebert argues that tactics are usually deployed in a manner that matches an ideal strategic escalation. He claims that in the early phases of a campaign, activists typically use methods designed to publicize a problem and persuade people that change is necessary, in order to put an issue on the public agenda. His second stage of escalation comprises legal actions that are intended to raise the costs for an opponent. In his third phase of escalation, Ebert states that illegal (direct) actions are both contentious and coercive and result in redirecting power. This means that the collective actors are using tactics to actually shape new power relations and relationships.

Another contribution to understanding nonviolent methods is Ebert’s focus on how non-violent action works. He argues that expressive activities seek to publicize and convince; that legal actions to control resources are intended to raise the costs and minimize rewards for an adversary; and lastly that activists’ illegal actions to control resources aim to redirect and usurp power. Grouping methods by how they work has strongly influenced our understanding and classification of methods.

One of the biggest challenges with Ebert’s distinction between legal and illegal actions is the fact that laws vary by locale. A sit-down strike may be legal in one country, but the same act could be illegal in another and therefore categorized differently. Furthermore, Ebert’s emphasis on stages can be misinterpreted as prescriptive. In reality, campaigns can begin in stage two or stage three and certain actions may be more escalatory in stage one than in stage two or three. This study and the Nonviolent Tactics Database incorporate Ebert’s dualistic categorization of nonviolent tactics into confrontational and constructive groups (see Table 3).

\section*{Mechanisms of Nonviolent Direct Action}

Mechanisms of nonviolent direct action\footnote{Mechanisms of direct nonviolent action are often confused with the mechanisms of change described earlier in this monograph. Mechanisms of change are how a movement’s opponent reacts to a successful tactic or campaign. Mechanisms of direct action, in turn, refer to the underlying processes by which methods of action operate to affect change in a conflict situation.} are yet another categorization of tactics focused specifically on how the tactics work. Bond (1994) identifies three relevant mechanisms of nonviolent direct action: discrete manipulation, public coercion, and demonstrative appeals.

\begin{enumerate}
\item Discrete manipulation refers to the “ability to control resources,” such as mobilizing
people for a protest or organizing a trucker strike to block roads with vehicles.
\item Public coercion primarily refers to threats or imposed costs to another’s interests.\footnote{Bond (1994) mentions another mechanism of direct action: physical force. It usually operates with public coercion to affect the will of an opponent. However, without coercion, it operates through displacement (physically moving the opponent), incapacitation (injuring the opponent), or elimination (killing the opponent). The target is the opponent’s body, as opposed to mind. While moving an opponent can be considered nonviolent, injuring or killing an opponent is clearly not.} For example, this could include the threat of a citizen arrest of a political or corporate leader. It also could include shutting down factories through a worker occupation or stay-away.
\item Demonstrative appeals are authoritative, persuasive, or altruistic appeals to shared logic, identity, or ideals directed toward an adversary to influence its will or preferences.
\end{enumerate}

Bond’s focus on mechanisms clarifies the relationship between an instance of direct action (or tactic) and its functional purpose. The purpose of political action is to affect the will of opponents (and ultimately their interests) through appeals (incentives) or coercion or physically manipulating the environment. Walking a dog or cooking a meal in most contexts does not manipulate the environment, appeal, or coerce, and therefore does not constitute non-violent action. However, when Buddhist monks in Burma refused alms (food) that military leaders offered them by turning their bowls upside-down, this simple coercive action was widely believed to threaten the leaders’ ability to obtain merit for their afterlife and thus undermined their legitimacy as devout Buddhists.

\section*{Categorizing Based on Constructive/Persuasive and Confrontational/Coercive Inducements}

Bond expands on Ebert’s categorization of constructive (persuasive) tactics (see Table 3) when he introduces three types of appeals:

\begin{enumerate}
\item Convincing appeals that invoke a common set of assumptions, rules, logic, or evidence,
\item Authoritative appeals that invoke mutual authority from which legitimacy is drawn, and
\item Altruistic appeals that invoke a common identity.
\end{enumerate}

Additional positive inducements where resisters try to invoke common interests with opponents to affect their will can include:

\begin{enumerate}
\item Social, material, or political rewards that benefit another party, or
\item Constructive program, which constructs alternative behaviors, practices, or institutions, or takes over existing institutions, to better or more equitably fulfill societal needs and attract defections and loyalty from the opponent and its allies.\end{enumerate}

Convincing (A), authoritative (B), and altruistic (C) communication constitute the constructive or persuasive inducements in the monograph’s category of tactics known as appeals. Social, material, or political rewards (D) are the constructive or persuasive inducements in the monograph’s category of tactics called refraining. Constructive program (E) falls within the constructive or persuasive inducements for the monograph’s category of creative intervention. See Table 1 and Figure 1.

Positive
Constructive/Persuasive
Inducements
Tactics Category
A. Convincing Communications
B. Authoritative Communications
Appeals
C. Altruistic Communications
D. Social, Material or Political Rewards
Refraining
E. Constructive Program
Creative Intervention

FIGURE 1: Constructive or Persuasive Tactics

In juxtaposition to these positive inducements, this monograph characterizes negative—confrontational or coercive—inducements as the use of threats (F) or the imposition of costs (G) (e.g., social, political, psychological, or economic costs).\footnote{Coercive methods can use violent or nonviolent threats and negative sanctions. With physical coercion, the body is targeted to influence the will or mind of the opponent. Except for the narrow use of physical force, all the mechanisms of direct action operate through social power, i.e., through the will of others.} They fall under the monograph’s categories of tactics such as protest, noncooperation, and disruptive intervention as presented in Figure 2 and adopted in Table 1.

Negative
Confrontational/Coercive
Inducements
F. Threats, Criticism, and Ridicule
G. Impostion of Social, Political, Psychological or Economic Costs
Tactics Category
Protests
Noncooperation
Disruptive Intervention

FIGURE 2: Confrontational or Coercive Tactics

Bond’s categorization informs the universe of civil resistance tactics introduced in this monograph to the extent that the nonviolent actions (saying, not-doing, doing or creating) include either coercion or persuasion or both.

Kreisberg (1998) proposes a slightly different connection between methods and outcomes in his studies on what he refers to as “constructive conflict.” He identifies three classes of methods: persuasion, coercion, and reward. Like Bond, Burrowes, and others, Kreisberg focuses on the importance of affecting another’s will. He argues that nonviolent (social) power operates through the will of others by affecting their interests (which may be symbolic, material, social, political, or economic) and involves tangible costs, threats, rewards, or risks to the opponent. Kriesberg’s persuasion is akin to Bond’s appeal, which seeks to convince or sway another party’s mind or will through logic and reason. Kreisberg includes reward as an inducement, whether through a bribe (though civil resisters would usually not consider this because it would contradict their anti-corruption messages) or a generous offer at the negotiating table.

Although rare in civil resistance conflicts, unilateral rewards matching movement strategies and visions are used. For example, in Armenia on May 2, 2018, the opposition to the ruling party launched a general strike that forced the nation to a halt for one day. Protests were unilaterally suspended until May 8, when a successful parliamentary vote for the opposition to take power was held.

Bond and Kreisberg share an analysis of mechanisms and inducements of nonviolent action that are grounded in positive “constructive” sanctions (appeals, persuasion, rewards) and negative “confrontational” sanctions (nonviolent threats and coercion). They are also framed in similar terms by Ebert’s (1970) “confrontational and constructive action” and Barbara Deming’s (1971) “two hands of nonviolence.”\footnote{In her book, On Revolution and Equilibrium (1971), Barbara Deming presents the metaphor of the two hands of nonviolence: “On the one hand (symbolized by a hand firmly stretched out and signaling, ‘Stop!’) I will not cooperate with your violence or injustice; I will resist it with every fiber of my being. And, on the other hand (symbolized by the hand with its palm turned open and stretched toward the other) I am open to you as a human being” (16).} This monograph adopts this dualistic understanding of the confrontational/coercive and constructive/persuasive inducements of nonviolent action.

\section*{Alternative Classifications of Nonviolent Tactics}

Researchers created alternative classifications of tactics in light of specific nonviolent struggles, including campaigns or movements a) against occupation, b) for civil defense, c) fighting corruption, and d) against institutional and class domination.

\subsection*{Civil Resistance Against Occupation}

Andrew Rigby uses a unique typology of nonviolent tactics to analyze the Palestinian struggle (Rigby 2010; Darweish and Rigby 2015). These categories of tactics include:

\begin{itemize}
\item Symbolic resistance: I remain what I was and communicate to others by means of gestures, actions, or dress continued allegiance to my cause and its values.
\item Polemical resistance: I oppose the occupier by voicing my protest and trying to encourage others of the need to maintain the struggle.
\item Offensive resistance: I am prepared to do all that I can to frustrate and overcome the oppressor by nonviolent means, including strikes, demonstrations, and other forms of direct action.
\item Defensive resistance: I aid and protect those in danger or on the run, and thereby preserve human life and human values endangered by the occupying power.
\item Constructive resistance: I challenge the existing imposed order by seeking to create alternative institutions that embody the values that I hope to see flourish more widely once we are free. (3)
\end{itemize}

Rigby’s categories of methods are perhaps more suited to occupied peoples. A small improvement to his framing of methods would be to substitute “I” for “we,” since collective resistance is the central organizing principle. These categories re-arrange existing tactics of nonviolent resistance, rather than identify new tactics.

\subsection*{Civil Defense}

Anders Boserup and Andrew Mack (1975) classify methods of nonviolent action based on their strategic function in civil defense against foreign invasion, occupation, and internal coups as: a) symbolic, b) denial, and c) undermining. Schock (2015a, 8) explains their analysis as follows:

\begin{itemize}
\item “Symbolic actions demonstrate unity and strength, define the challengers as a moral
community, and force the uncommitted to take a stand.
\item Denial actions deprive the opponent of what is taken through coercion or accumulated
through exploitative or illegitimate exchange relations.
\item Undermining actions attempt to exacerbate or exploit the divisions among opponents
and inhibit the cooperation of third parties with opponents.”
\end{itemize}

The functional analysis of Boserup and Mack’s classification system has utility beyond strategic defense; it also applies to campaigns for social change. Boserup and Mack’s (1975) expansive definition also includes indirect actions against an opponent (such as engagement methods explained by Beyerle below).

\subsection*{Civil Resistance Against Corruption}

Beyerle (2014) has pioneered analysis on how civil resistance curtails corruption. She has documented many nonviolent campaigns and methods against corruption, including a campaign against the mafia in Sicily, social audits in Kenya and Afghanistan, and a campaign against corrupt politicians in South Korea. The tactics she identified can be categorized into three functional categories: disruption, engagement, and empowerment.

\begin{itemize}
\item Disruptive tactics confront and constrain an adversary. Examples of disruptive methods include information gathering, citizen inspections, festivals, games, and exposing corruption through SMS, e-petitions, and online banners.
\item Engagement tactics pull people (and various sectors) toward the campaign and, in some cases, shift loyalties to produce defections, strengthen citizen participation, and weaken corrupt adversaries or their enablers. Engagement can also mean joining forces with institutional activists, such as politicians or civil servants. Examples of engagement methods include negotiation, press conferences, citizen contributions of money and resources, citizen-powerholder meetings, solidarity activities, accompaniment, and displaying resistance symbols on T-shirts, hats, ribbons, and so on.
\item Empowerment tactics “shift power relations through the power of numbers,” (Beyerle 2014, 32) such that widespread participation generates palpable and un-ignorable social pressure. Examples include trainings, civic education, and coalition and alliance building.
\end{itemize}

Beyerle’s tactics typology differs from the standard classification of nonviolent tactics in two ways. On the one hand, anti-corruption methods include tactical support activities that are not typically direct actions against an opponent in other conflicts. Beyerle argues that training and funds solicitation, for example, serve as nonviolent weapons because they can activate a neutral/passive citizenry, not just serve as a logistical support activity in advance of a rally. In addition, every coin donated to the opposition is one less coin available to the adversary.

Beyerle describes a phenomenon that emphasizes the powerful, central role of disruption, engagement, and empowerment actions directed toward the passive and compliant public in which there is no clear opponent or, if there is one, a large population that is neutral
or itself corrupted. Alternatively, anti-corruption methods are often a mix of nonviolent resistance with electioneering, lobbying, and legal fights that advance specific dynamics of nonviolent conflict, e.g., eliciting defections and magnifying backfire potential.\footnote{See more on backfire in “Basics of Civil Resistance” section of the monograph.} In addition to formal processes, anti-corruption campaigns can include routine activities such as negotiation, civic education, and advertising activities.

The embedding of formal and routine actions in anti-corruption campaigns blurs the distinction between more traditional means of influencing politics and non-institutionalized and disruptive civil resistance actions. The categorization framework in this monograph does not include some of Beyerle’s anti-corruption methods in order to maintain a clear separation between formal (institutionalized) and informal (extra-institutional) means of waging conflict.

Tactical support activities—single or routine actions that enable a tactic to succeed—are not considered in this monograph to be weapons or tools that directly target an adversary through persuasion, coercion, or manipulation. As such, these types of support activities are not included in this monograph’s analysis. For a detailed list of support activities and related information, see page 64.

\subsection*{Everyday Resistance Against Structural and Institutional Dominance}

Vinthagen and Johansson (2013), drawing on the works of theorists like Michel Foucault and James C. Scott, have documented an under-examined form of civil resistance: “everyday resistance.” They characterize everyday resistance as “quiet, dispersed, disguised, or otherwise seemingly invisible” (4).

They consider everyday resistance, which can encompass attitudes, habits, conversational behaviors, and actions, to constitute an important and undervalued repertoire of civil resistance struggles. Using Foucault’s truism that “where there is power, there is resistance” as a starting point, Vinthagen and Johansson have argued that many everyday actions and behaviors constitute resistance to power structures; they are subtle, often unintentional or unsuccessful, and are incorporated into normal social life. Methods of micro, disguised, and hidden resistance include squatting, foot-dragging, desertion, feigned ignorance, or evasion. For example, rather than organizing a mass land occupation, resisters may encroach slowly upon a property. Instead of a mass soldier strike, a steady trickle of daily desertions or deliberate bureaucratic inefficiencies can achieve a similar result in breaking down or slowing military capability.

Scott (1989, 27) documents four categories of disguised resistance:

\begin{itemize}
\item Everyday resistance methods to material domination including theft, evasion, and foot dragging (see “Everyday Resistance” subsection below for more on the debate surrounding criminality).
\item Hidden transcripts of anger, aggression, and a discourse of dignity in response to humiliation or disprivilege. An example of this can be tales of revenge passed from one generation to the next.
\item Development of a dissident subculture combating ideological domination through use of tactics such as folk religion, myths of social banditry, and class heroes.
\item Direct rebellious action by disguised resisters.
\end{itemize}

A few methods of everyday resistance are included in this monograph’s catalogue of civil resistance (squatting, silence, popular non-obedience, spreading rumors, ID correction with gender markers, etc.). However, this monograph primarily focuses on tactics that aim for social and political goals and are explicitly nonviolent.

\subsection*{“Power-Breaking” Categorization}

Vinthagen (2015) proposed a range of tactics of “strategic nonviolence” that he calls the “power-breaking” component of nonviolent resistance.

These include:

\begin{enumerate}
\item Counter-discourse: Communicating with well-supported counter-arguments and counter-images (discursive strategies) that disrupt the power’s propaganda (fact-finding, symbolism, countering enemy images by counter-behavior, e.g., nonviolent clowning).
\item Competition: Creating alternative and competing nonviolent institutions (in cultural, political, and economic areas).\footnote{Vinthagen uses the term “competition” essentially as a power contestation to reflect elements of what Gandhi called the constructive program. An example he offers is from the Kosovar pro-independence struggle in the 1990s, in which local government services, such as trash pickup and education, were organized as an alternative service in competition with the official Serbian government.}
\item Noncooperation with the existing system’s roles or functions (including boycotts) combined with cooperation with people who focus on legitimate and mutual needs (such as relief work during a natural catastrophe).
\item Withdrawal: Removing oneself from destructive power relations (flight and the creation of free zones).
\item Hindrance: Stopping or preventing the processes of oppressive power systems (blockades, occupations, and interventions).
\item Dramatizing injustices with humor (self-irony, redefinition, and shock).
\item The strength of this categorization is that it is grounded in the common practices of many nonviolent campaigns and can thus resonate with many activists and campaigners. Vinthagen’s method categories are represented in this monograph’s framework \emph{by protest and appeal} (encompassing his identified expressive and dramaturgical methods, including counter-discourse methods); \emph{noncooperation and refraining} (encompassing his withdrawal and noncooperation methods); \emph{disruptive intervention} (equivalent to Vinthagen’s hindrance methods); and \emph{creative intervention} (encompassing his competition methods). Vinthagen’s categorizations are not well known but they are a useful way of classifying nonviolent tactics. See Tables 1 and 4 for
relevant examples.
\end{enumerate}

This chapter cites a number of efforts at categorizing nonviolent tactics that put important foundations underneath the monograph’s own analysis of the universe of nonviolent tactics. Consequently, the monograph builds on Sharp’s categories and integrates elements from Ebert (confrontational and constructive action), Bond (appeals), Kriesberg (inducements), Bloch (expression through materials), Joyce (digital resistance), Burrowes (creative and disruptive intervention) and refines them with new contributions to present an updated universal typology of civil resistance tactics.

\chapter{Mapping New Civil Resistance Tactics}

In this monograph, we categorize civil resistance tactics into three major categories: saying, not doing, and doing.
\begin{description}
\item[Saying:] Also called acts of expression
\begin{itemize}
\item Protest and appeal: What we say, or how we say things
\end{itemize}
\item[Not doing:] Also called acts of omission
\begin{itemize}
\item Noncooperation and refraining: What we don’t do, or how we don’t do things
\end{itemize}
\item[Doing:] Also called acts of commission
\begin{itemize}
\item Disruptive intervention: What we do or how we disrupt things
\item Creative intervention: What we make or how we create things
\end{itemize}
\end{description}

Saying (acts of expression) includes protesting through communication such as yelling and chanting, as well as persuasive communication forms such as humor and prayers. Most acts of expression can be used to penalize or reward, so in this monograph we follow Gene Sharp’s methodology of grouping expressive methods together in a category called protest and appeal.

Not doing (acts of omission) includes noncooperation (not doing something that adversaries want you to do) and a new category for the constructive technique we call refraining (halting something that adversaries don’t want you to do). The acts of omission category can be misleading in implying passivity or that no action is involved. However, most strikes and boycotts are very active. Sharp’s use of the infinitive “to withdraw cooperation” as a description of noncooperation methods highlights the active dimension of these tactics. Refraining tactics as part of the acts of omission category may include the suspension of a labor strike or the halt of disruptive actions that inflict costs on one’s adversary. It is often used as a positive incentive that is unilaterally initiated by the movement, though does not include actions taken as a result of a negotiated agreement between a movement and its opponent.

Doing or creating (acts of commission) includes disruptive intervention and creative intervention. Examples of disruptive intervention include blockades, phone jamming, and sit-ins. Creative interventions construct and model the society that activists seek to create. Sharp correctly points out that the latter methods “establish new political and social patterns and behaviors,” but he does not classify them separately from methods that “disrupt... established behavior patterns” of adversaries. Bartkowski (2013) and Dudouet (2014) label these tactics as “creative nonviolent interventions.” They encompass some elements called constructive programs and pre-figurative actions such as mock elections, parallel governments, and alternative schools. Creative intervention also includes what Vinthagen calls “utopian enactments” such as new social behaviors or institutions that prefigure the world that proponents want to create. It could also involve taking over and operating the other party’s existing institutions such as a police station, checkpoint, or factory.

All civil resistance tactics are a form of communication in a social environment. Acts of saying, specifically, are often used similarly to a “propaganda of the deed” to inspire a revolt by demonstrating that the opponent’s control was indeed not total and that ordinary people had both the will and space to rebel. Acts of saying include communication through direct action, as made famous by the Industrial Workers of the World (IWW or Wobblies)—the labor activists who engaged in many intense labor strikes in the early 20th century to communicate their opposition to the owning class. Acts of not doing are powerful weapons of nonviolent coercion or pressure on the opponent while acts of doing or creating are primarily used to control resources and/or usurp power, often through engagement (intervention) in alternative practices, parallel institutions, and self-organizing.

Table 4 incorporates the framework introduced in Table 1 to map corresponding new tactics of civil resistance. Only one new tactic was assigned for each major category (in total, 23 tactics), though many more new tactics have been identified in the Universe of Nonviolent Tactics Appendix and in the Nonviolent Tactics Database.

TABLE 4: Mapping New Civil Resistance Tactics

Resistance behavior
Saying (acts of expression)
HUMAN BODY
Flash mob
NATURE OF TACTIC INDUCEMENTS
CONFRONTATIONAL (COERCIVE)
Protest
Communicative action to criticize or coerce
MATERIAL ARTS
DIGITAL TECHNOLOGY
HUMAN LANGUAGE
Cacerolazo
Mic check
Digital game
Not doing
(acts of omission)
POLITICAL
Inter-agency noncooperation
Noncooperation
Refusal to engage in expected behavior through boycotts and strikes in order to penalize or increase costs on the opponent
SOCIAL
Withholding religious rites
ECONOMIC
Divestment
Disruptive Intervention
Direct action that confronts another party to stop, disrupt, or change their behavior
Doing or creating something
POLITICAL/JUDICIAL
ECONOMIC
SOCIAL
PHYSICAL
Parliamentary disruption
Business whistle-blowing
Outing
Die-in
(acts of commission)
PSYCHOLOGICAL
Self-mutilation


HUMAN BODY
Growing hair
Mapping New Civil Resistance Tactics, cont.
NATURE OF TACTIC INDUCEMENTS
CONSTRUCTIVE (PERSUASIVE)
Appeal
Communicative actions to inform or persuade
MATERIAL ARTS
DIGITAL TECHNOLOGY
Murals
Sousveillance
HUMAN LANGUAGE
Public advertisement
Resistance behavior
Saying (acts of expression)
Refraining
Halting or calling off a planned or ongoing action to reward or persuade the opponent
SUSPENDING
Suspending an ongoing nonviolent disruptive action
ACTIVE ABSTENTION
Active abstention from a planned nonviolent action
Not doing
(acts of omission)
Creative Intervention
Direct action that models or constructs alternative (competing) behaviors and institutions or takes over existing institutions
POLITICAL/JUDICIAL
ECONOMIC
SOCIAL
PHYSICAL
PSYCHOLOGICAL
Citizen inspections
Property expropriation
Marriage inclusion
Critical mass
Self-imposed transparency
Doing or creating something
(acts of commission)

\chapter{New Civil Resistance Tactics: Selection Criteria, Descriptions, and Examples}

\section*{The Criteria for Selecting New Civil Resistance Tactics}

This monograph uses specific criteria to identify new civil resistance tactics along the lines of Gene Sharp’s previous efforts. Consequently, to be included in this monograph’s categorization, the new civil resistance tactics must:

\begin{enumerate}
\item coerce or persuade a group or society to change the status quo they support (a governing system, a policy or practice, etc.) in a contentious conflict
\item not physically threaten or injure others or most kinds of property
\item not be routine or institutional processes with known procedural outcomes such as laws, courts, elections, lobbying, and commerce
\item be plausibly replicable in a variety of conflicts or contexts
\item not be an action by a non-partisan third-party (external) actor such as nonviolent civilian protection activities
\item be unilateral and not require the cooperation of an adversary
\item not be primarily logistical campaign activities such as training and fundraising
\end{enumerate}

The above criteria were used to identify hundreds of new tactics listed in the Universe of Nonviolent Tactics Appendix. Below is a sampling of the newly identified tactics in the new proposed categorization. Many tactics, as Sharp wrote, can fit into more than one category based on function or, in some cases, intended impact.

The tactics are divided into broad categories of resistance behaviors such as saying (acts of expression), not doing something (acts of omission), and doing or creating something (acts of commission). We also divide methods according to whether they are confrontational and coercive, or constructive and persuasive. Many new tactics beyond the ones described below have been identified for each of the categories and their full list is provided in the Universe of Nonviolent Tactics Appendix and described in the Nonviolent Tactics Database.

In this chapter, we offer a couple of examples of new tactics falling into each category or subcategory to highlight tactics not included in Sharp’s list of 198 nonviolent methods.

\section*{Tactics of “Saying” Something (Protest and Appeal)}

Resistance behavior
NATURE OF TACTIC INDUCEMENTS
CONFRONTATIONAL (COERCIVE)
CONSTRUCTIVE (PERSUASIVE)
Protest
Appeal
Communicative action to criticize or coerce
Communicative actions to inform or persuade
Saying
(acts of expression)
HUMAN BODY
MATERIAL ARTS
DIGITAL TECHNOLOGY
HUMAN LANGUAGE
Growing hair
Flash mob
Cacerolazo
Digital game
Mic check
Murals
Sousveillance
Public advertisement

This broad category includes verbal, linguistic, behavioral, and visual expressions of positive appeals and negative criticisms. All actions in society—including occupations and fair-trade markets—are a form of communication. However, this category of tactics is designated for actions whose purpose is primarily and commonly seen as communicative. Expressive tactics can be used for confrontation and threats (protest) as well as for constructive expressions or offers (appeal); their categorization in negative or positive boxes is fairly arbitrary and highly contextual.

We organize expressive methods into four sub-categories based on the primary medium in which they are conducted:

\begin{enumerate}
\item Human body;
\item Material arts;
\item Digital/Internet technology; and
\item Human language.
\end{enumerate}

The following sections provide two examples of tactics for each of the aforementioned subcategories.

\subsection*{Human Body as the Primary Medium of Expression}

\subsubsection*{FLASH MOBS (PROTEST)}

Flash mobs are gatherings of people, usually recorded on digital media, performing actions such as street theater, a sit-down protest, or a blockade, at a specific time and place unbeknownst to the public ahead of time. A flash mob is typically performed in a busy area where it will be heavily witnessed, such as a train station, a park square, or the middle of a college campus. It can be an unrehearsed action that observers can join.

Flash mobs began as a form of participatory art and has evolved into a mechanism of political protest. One political example of this tactic is a 2009 flash mob in which hundreds of protesters dressed in business attire stormed Wall Street and took part in a mass pillow fight to “demand their bailouts.” In Thailand, Belarus, and Zimbabwe, protesters have used flash mobs to show outrage over laws banning protests or public gatherings. In Thailand specifically, flash mobs were used in 2014 as a means to quickly disperse and reorganize in the event of police repression. In contrast to Thailand, security forces in Belarus arrested protesters during a flash mob in which participants simply clapped their hands. The absurd images of brutal arrests in response to clapping helped bolster anti-Lukashenko sentiment in Belarus (Mitchell and Boyd 2012).

\subsubsection*{GROWING HAIR (APPEAL)}

Growing hair was a cultural tactic in many Western nations in the 1960s and 70s. It was part of a generational effort to challenge societal assumptions and norms. Growing long hair was about making a “back to the land” and “natural” statement and was associated around the world with anti-war movements. Feminists stopped shaving underarm and leg hair in efforts to challenge different gender grooming standards in society. Growing beards has also been utilized as a form of protest. In 1981, New York City police officers grew beards during stalled contract talks.

\subsection*{Material Art as the Primary Media of Expression}

These tactics use material objects, sound, and light including puppets, banners, colored clothing, flags, food, music, and headlights. Art has been at the forefront of many nonviolent campaigns serving the purpose of both protest and appeal (Bloch, 2015). Two examples not included in Sharp’s classic list include:

\subsubsection*{CACEROLAZO (PROTEST)}

Invented in the 1970s, cacerolazo is a much louder form of protesting than most. Participants bring metal cookware into the streets to bang them noisily. In a poignant 2016 example, Venezuelan protesters reintroduced cacerolazos as a way to express their frustrations with President Nicolás Maduro. They protested the state of the economy—which for many meant lack of food—making the symbolism of the using pots as protest props all the more meaningful. Some cacerolazo actions have been conducted from numerous homes simultaneously to limit police retaliations. Women have historically participated in cacerolazos in large numbers. To sustain noise levels and to prevent headaches, earplugs are commonly used by those participating (Ulmer, 2016; Christoff).

\subsubsection*{MURALS (APPEAL)}

Murals use material arts in a way that is constructive and persuasive, attempting to inspire an audience. Murals are large-scale paintings on the sides of buildings or walls. During the dictatorship of General Pinochet in Chile, photographer Andrés Romero Spethman followed multiple mural brigades that drew vivid depictions of the social and political issues in the country. The murals, often covering entire walls, featured satirical representations of political figures and images of hope for the Chilean people. Soon after their completion, however, the government removed them (Harvard Digital Collection, n.d.).

Murals have been deployed in conflicts across the globe, from East Timor to Palestine to Northern Ireland. In the 2007 Face 2 Face project, Palestinians and Israelis were photographed engaging in the same work, and their portraits were glued to the separation wall in cities on both sides of the wall. A variation of this project is the creation of “sister murals.” For example, the creation of twin murals was coordinated between Olympia, Washington (USA) and Rafah in the Gaza Strip to raise awareness and solidarity. Unfortunately, mural projects are susceptible to “redecoration” and modification by others. For example, after the fall of Egyptian President Hosni Mubarak, competing political groups added many layers of murals over those made in Tahrir Square.\footnote{Author observed this on a trip to Cairo in 2012.}

\subsection*{Digital/Internet Technology as the Primary Medium of Expression}

The digital revolution, both as a communication delivery device and an environment for engaging in power struggle and conflict, has led to the formation and emergence of new civil resistance methods. Most of these methods fall into the appeal and protest category and their development has been closely tracked by the work of Dr. Mary Joyce and colleagues in Civil Resistance 2.0, cited earlier in this monograph. Most digital methods can be used for protest and/or appeal. Here are two examples:

DIGITAL GAMES (PROTEST)

Digital games have been used to strengthen public opposition for a political cause. In Mexico, programmers developed an online game to criticize the values of Donald Trump, who was at the time a U.S. presidential candidate. The game consists of users playing the role of Donald Trump, trying to collect as much money as possible, marginalizing minorities, and whipping his hair around (Matthews, 2015). In 2019, protesters in Hong Kong developed an interactive, virtual reality-based video game to show players what it is like to protest on the streets. The game allows players to learn about important events which have occurred during the movement, and to perform maneuvers such as dodging tear gas or police. The game developers noted that the creation of the game was a form of protest in and of itself in the wake of backlash from Google and Blizzard Entertainment, which had recently banned pro-Hong Kong players and games from their platforms (Reuters, 2019).

SOUSVEILLANCE (APPEAL)

Sousveillance, in contrast to surveillance, is when a participant in an activity records the activity. For example, in 2007, mobile phones were used in Sierra Leone and Ghana for investigating malpractice and intimidation during elections (Green, 2008). A subset of sousveillance is \emph{inverse surveillance}, which typically involves the bottom-up surveillance of surveillance systems, institutions, and power structures. Examples of this range from privacy watch-dog groups to journalistic activities such as The Intercept’s Secret Surveillance Catalogue, which logs the technologies that the U.S. government uses for surveillance. Activists commonly use sousveillance to deter police repression, even in cases where audio or video are not immediately broadcasted. Another subset is \emph{alibi sousveillance}, in which individuals or groups record their activity as a way to defend against narrative manipulations. Nonviolent protesters may choose to live-stream their protest as a way to document their actions in case participants are arrested or falsely accused of wrong-doing.

\subsection*{Human Language as the Primary Medium of Expression}

Some tactics focus on how humans express themselves through various languages. Several of these methods are described in terms of written media (newspapers, journals, books, etc.) which are manifested as methods when employed to protest or persuade in conflicts. Tactics using human language can make both appeals and threats in a variety of constructive and confrontational arenas.

\subsubsection*{HUMAN MICROPHONE OR MIC CHECK (PROTEST)}

Popularized by the Occupy movement, the human microphone is a form of verbal communication utilized to help large audiences hear a speaker. Speakers deliver their message in short phrases and sentences which are immediately shouted by those within earshot and subsequently repeated two or three times by those standing farther away. The message travels throughout the crowd in a ripple effect, eventually reaching those who could not hear the speaker initially. It has the power to enable thousands of people to hear a speaker, in the absence of a microphone and sound system. This method can also help unify a crowd, improve communication and understanding, and have a powerful emotional impact on the speaker hearing one’s words repeated by many others. It also tends to limit side conversations, which can help maintain a large crowd’s attention and focus.

However, the process of repetitive shouting can be tiring. Secondary and tertiary repeats can become frayed and cacophonous, and the necessity of using short phrases can hinder the communication of complex thoughts. This method is increasingly being used as a form of protest to talk-over or interrupt speakers against whom campaigns are organizing, such as political candidates during election rallies (YouTube user “noplatform for IMF,” 2013).

\subsubsection*{POLITICAL ADVERTISING (APPEAL)}

Mainstream media continues to be a central platform for community and society conversations. A common tactic to inform or persuade as part of the human language subcategory of appeals is political advertising. Political advertising by corporations, political parties, and wealthy people is so routine in many countries that usually this would not be considered a nonviolent action tactic. However, when paid for by large numbers of individuals or groups who are commonly shut out of public discourse, paid political advertising is a means to publicize one’s views and to advocate for different policies supported by a campaign or a movement. The targets of these messages can include movements’ adversaries or the general readership and viewership of the media. Crowd-funded advertising has been run on TV, social media, billboards, newspapers, magazines, and radio. In 2015, Amnesty International crowd-funded from some 1,000 British citizens to pay for public advertisements to prevent the government from repealing human rights (Amnesty International UK, 2015).

\section*{Tactics of “Not Doing” (Noncooperation and Refraining)}

Resistance
behavior
N ATUR ECON FRON TATIO N A L
(COE RCIVE )
O FTAC TICIN D UC EMEN TS
C O N ST R U C T I V E
( P E R S UAS I V E )
Noncooperation
Refusal to engage in expected behavior
through boycotts and strikes in order to penalize
or increase costs on the opponent
Not doing
(acts of
POLITICAL
 SOCIAL
 ECONOMIC
omission)
Inter-agency
 Withholding
 Divestment
noncooperation
 religious rites
Refraining
Halting or calling off a planned or ongoing action
to reward or persuade the opponent
SUSPENDING
Suspending an
ongoing nonviolent
disruptive action
ACTIVE ABSTENTION
Active abstention
from a planned
nonviolent action
Doing nothing is not usually seen as an act. But in the context of civil resistance, inten-
tionally doing nothing in an environment where you are expected to do something is an act
of omission. Tactics of omission can be divided into sub-categories called noncooperation
and refraining.
Noncooperation is the refusal to engage in expected or required behavior. These tactics,
that aim to withdraw cooperation from an adversary, constitute the vast majority of acts of
55
omission. Sharp collected over 100 methods of noncooperation largely categorized into
economic, social, or political spheres. This monograph maintains Sharp’s subcategories of
political, social, and economic noncooperation, while identifying new methods in each of
these noncooperation subcategories.
In addition to noncooperation as a constructive act of omission, refraining involves halting
or suspending noncooperation and disruptive actions or expressions in order to reward or
persuade. Pioneered by Gandhi, these methods have not been categorized by analysts until
now. These acts of omission are positive actions conducted unilaterally to communicate
good-will to an adversary and/or the general public. They theoretically can be used to reward
an opponent for good behavior, policies, or statements. They are a relatively rare practice.
Noncooperation Tactics: Confrontational Acts of Omission
The following sections provide two examples of tactics for each of the three subcategories:
political noncooperation, economic noncooperation, and social noncooperation.
■ Political Noncooperation
Political noncooperation is an act of omission that involves withdrawing cooperation from
political entities or requirements. Here is an example that is not included on Sharp’s list of
nonviolent action methods:
I N T E R -AG E NCY NO NCO O P E RAT IO N
Inter-agency noncooperation happens when a governmental or other agency withholds
information or resources to prevent action, such as government repression of activists. During
large protests in South Korea in 2015, the fire department turned off the water supply to fire
hydrants so that the security forces could not water-hose protesters (Lee, 2016). Following
an announcement from the US Immigration and Customs Enforcement agency (ICE) of
planned mass raids to search for undocumented residents in major cities all across the coun-
try, several U.S. cities made declarations of noncooperation with the agency. The mayor of
Chicago, Lori Lightfoot, ordered the city’s police force not to cooperate with ICE and to cut
off ICE’s access to their database, specifically the information concerning immigration.
■ Social Noncooperation
Social noncooperation is an act of omission that involves refusing to engage in expected or
required behavior in social and civil society settings. These methods are often categorized
on a continuum of ostracization of individuals, noncooperation with specific events, customs,
or institutions, and withdrawal from an entire social system.
56
W I T HHO L DING O F RE L IGIO US RIT ES
Religious authorities may selectively withhold religious rites to worshippers who violate the
teachings of the religious body or who are in conflict with the religious body. Interdicts and
excommunications are additional methods of religious noncooperation that clergy use.
One example is Buddhist monks resisting Burmese generals in 1990 and in 2007. The
monks turned over their alms bowls, refusing to accept food offerings from the military leaders
who had assaulted monks and civilians. Buddhists believe that failure to provide food to monks
will damage one’s religious merit. Muslim clerics refused to cooperate with the Islamic State in
France by refusing burial rites to a Muslim murderer of a Catholic priest (La Porte, 2016).
■ Economic Noncooperation
Economic noncooperation is an act of omission that involves refusing to sell one’s labor
(strike) or to buy a service or product (boycott).
D I V E ST ME NT
In the 1980s, major church denominations in the United States forced the seven major U.S.
banks to divest their church pension investments from businesses profiting from apartheid
in South Africa. Labor and national pension funds have also exerted influence by divesting
from companies whose actions they oppose. Norway’s Global Pension Fund of $900 billion
divested from coal companies in 2015 to withdraw support from companies contributing to
climate disruption (Crawford-Browne, 2017).
Refraining: Constructive Acts of Omission
Most examples of refraining are deployed as part of a constructive approach to conflict where
those who use refraining want to persuade or appeal to an adversary or the public using an
incentive of postponing or stopping a resistance action. While noncooperation is not doing
what the opponents want you to do, refraining refers to not doing or doing more slowly what
the opponent does not want you to do. This choice is strategic and deliberate, undertaken
by activists to advance, not suspend, the struggle and to move closer to the realization of its
goals. These methods are not a result of backing down or giving up. Refraining can also be
understood as a form of voluntary cooperation to reward an adversary or the general public
in the short term, in order to change the dynamics of the struggle to help achieve campaign
goals in the long term.
Refraining methods intentionally stop or suspend an ongoing action or prevent a planned
action. They should not be confused with suspensions or collapses of strikes, blockades, or
occupations as a result of campaign failures. Clever strategists undoubtedly seek to re-frame
57
their retreats (refraining) in positive terms in order to re-group and fight another day. The
purpose of these tactics can also aim to unify or strengthen the will of the campaigners.
Refraining tactics, like most constructive methods of action, entail risk, particularly stra-
tegic risk. The opponents or adversaries may not react positively to rewards or incentives.
They may question the sincerity or motivations of the nonviolent actors. An opponent might
also misinterpret these methods as a sign of movement weakness and, instead of agreeing
to negotiate, take the opportunity to instigate more violence. Thus, lessening pressure on
an adversary could incite stronger attacks. Still, such tactics can be used effectively.
Refraining has no subcategories. Instead, we have identified two kinds of tactics: actions
that aim to cease or freeze an ongoing tactic (suspending) and actions that prevent a planned
action from taking place (active abstention). Here are two examples of refraining tactics:
■ Suspending
An act of omission in which one temporarily stops an ongoing action that the opponent
opposes.
SU SPE N D I N G AN O NGO ING NO NV IOLEN T DISRUPTIVE ACTION
In 2017, airport workers in Newark, New Jersey (USA) unilaterally suspended their strike
against unfair labor practices as a goodwill gesture to the public and the authorities (News12,
2017). The strike began when the airlines rejected the workers’ attempts to organize a
union and negotiate better hours and pay. The strike was suspended when American
Airlines finally agreed to meet with their union representative. The airport went on to rec-
ognize the union and since then mutual cooperation has continued.
■ Active Abstention
An act that prevents a planned action against the opponent from taking place.
AC T I V E A BST E NT IO N FRO M A P L AN N ED N ON VIOLEN T ACTION
On March 30, 1981, the Solidarity trade union in Poland planned a general strike to protest
police brutality against workers. Movement leader Lech Wałęsa unilaterally called off the
general strike after the Communist government agreed to negotiate and offered
concessions.
58
Tactics of “Doing or Creating” Something (Disruptive and Creative Interventions)
Resistance
behavior
N ATUR ECON FRON TATIONA L
(COE RCIVE )
O FTAC TICIN D UC EM EN TS
C O N ST R U C T I V E
( P E R S UAS I V E )
Creative Intervention
Disruptive Intervention
Direct action that models or constructs
Direct action that confronts another party
alternative (competing) behaviors and institutions
to stop, disrupt, or change their behavior
or takes over existing institutions
Doing or
creating
POLITICAL/
 ECONOMIC
 SOCIAL
 PHYSICAL
 PSYCHO-
 POLITICAL/
 ECONOMIC
 SOCIAL
 PHYSICAL
 PSYCHO-
some-
 JUDICIAL
 LOGICAL
 JUDICIAL
 LOGICAL
thing
 Business
 Outing
 Die-in
 Property
 Marriage
 Critical
(acts of
 Parlia-
 whistle-
 Self-mu-
 Citizen
 expropri-
 inclusion
 mass
 Self-im-
commission)
 mentary
 blowing
 tilation
 inspec-
 ation
 posed
disruption
 tions
 trans-
parency
This monograph divides Sharp’s large category of intervention methods into creative and
disruptive interventions. Disruptive interventions are direct actions with the purpose of
pressuring another party to stop, disrupt, or change its behavior. This category of methods
seeks to “step more directly in between the adversary and the achievement of its purpose”
(McCarthy 1997, 323). Traditional examples include blockades, invasions, hunger strikes,
and denial of service. A few everyday resistance tactics are included such as counterfeiting
and desertion.
Creative interventions are direct actions that prefigure (model or construct) alternative
behaviors, norms, and institutions. Common examples include parallel government, alter-
native currency, alternative newspapers, or alternative language. Also included are rare
interventions of unilateral rewards or offers, as well as elements of constructive programs
that don’t just strengthen internal campaigns but have elements that disrupt or challenge
an adversary’s interests. A few everyday resistance tactics are included such as black mar-
kets. This monograph maintains Sharp’s subcategories for methods of intervention (political,
economic, social, physical, and psychological), and identifies and adds new tactics for each
of these sub-categories.
59
Disruptive Interventions: Confrontational Acts of Commission
The following sections provide one example of tactics for each of the five subcategories of
disruptive interventions: political, economic, social, physical, and psychological.
■ Political/Judicial Disruptive Intervention
An act of commission, this intervention is used in political and judicial arenas to obstruct or
subvert an opponent’s government or legal system.
PA R L I A ME N TARY /L E GISL AT URE /COUN CIL DISRUPTION
This method is conducted by legislators who stop or slow down legislative proceedings by
extra-legal or extra-regulatory means. Disruption often happens through shouting, singing,
or chanting. Sit-ins and other physically disruptive actions are also deployed, usually to stop
votes. Tools or instruments used in parliamentary proceedings can also be taken away, such
as a symbolic mace in Nigeria that was removed to prevent parliament from passing laws
until it was returned (Taylor, 2018). Another example of legislative disruption was in July 2018,
when members of the European Parliament turned off their microphones in support of striking
interpreters (France24, 2018).
■ Economic Disruptive Intervention
This tactical intervention occurs in the economic sphere and is primarily focused on imposing
financial costs upon an opponent.
B U SI N E SS W HIST L E BLOW ING
Whistleblowing refers to when journalists or insiders release information. A whistleblower is
anyone who reports insider knowledge of illegal activities occurring in a business, including
employees, contractors, suppliers, or clients. Whistleblowers can become aware of the illegal
activity by directly witnessing the behavior, reading insider information, or being told about
it. An insider who leaks information secretly or confidentially to a journalist or other whis-
tleblower is referred to as a leaker. Businesses around the world have been exposed and
held accountable by insiders who provided information on dangerous practices.
One example is a Russian auditor named Sergei Magnitsky who blew the whistle on
Russian police who had given confiscated materials to criminals (high-ranking Russian officials
and Kremlin-connected businessmen), who then used the information obtained from the
materials to take over three Hermitage companies and fraudulently reclaim $230 million of
the taxes previously paid by Hermitage (Browder, 2015).
Whistleblowing is typically seen as an act of betrayal by the affected institution, and ret-
ribution and suppression can be severe. As a result of his whistleblowing, Magnitsky was
60
tortured and beaten to death by police in prison. It can be a high-risk but high-impact tactic
and should be considered carefully.
■ Social Disruptive Intervention
This type of tactical intervention “takes the form of direct intrusion in social behavior patterns,
social occasions, and social institutions” (Sharp, 1973).
OUTING
This is a technique of revealing a secret about someone’s personal life that conflicts with
their political or public actions. In the 1980s, members of the gay community exposed politi-
cians’ homosexual behavior to demonstrate a contradiction with their public condemnation
of homosexuality or support for laws and norms that repressed the gay community. Another
example of outing is male politicians being exposed for encouraging their impregnated
partner to seek abortions while publicly advocating to ban the procedure. Such exposure is
typically done through the arts or reputable news organizations but can also be done through
social media or law enforcement.
■ Physical Disruptive Intervention
This type of tactical intervention consists of deploying bodies or things in places where they
are not legal or wanted, in order to obstruct, disrupt, or interfere with an opponent or their allies.
DIE-IN
Also known as a lie-in, this technique involves protesters simulating being dead as a strategy
for occupying a space. Die-ins are often not simply a theatrical production. The point is to
block foot or vehicle traffic and attract attention, so participants will often cover themselves
with signs, fake blood, or other props. Though die-ins can be traced back over 50 years, in
the last decade they have increased in popularity. In 2016, activists in Hong Kong protested
the mass killing of sharks for soup by engaging in a die-in dressed in bloody shark suits (Efe,
2016). The use of blood or fake blood is common, as well as drawing chalk outlines of bodies
that remain long after the protest has ended.
■ Psychological Disruptive Intervention
This type of tactic is a highly contentious form of intervention as it can be implemented through
self-suffering or dramatization meant to force attention to one’s message or needs.
SE L F - MU T IL AT IO N ( O R BO DY ART )
This is a technique of self-sacrifice and often cultural transgression which seeks to disrupt
the narrative of an opponent or society. A common form of self-mutilation is sewing one’s
61
lips shut to dramatize a cause with the visual statement of being silenced or censored and/
or as a way to amplify a fast. Refugees stranded in prisons or camps have engaged in this
practice in many countries. Tattoos are another common way to promote one’s views. Animal
rights activists have branded themselves with hot irons to demonstrate the cruelty of branding
animals. Shaving one’s head can be a temporary form of bodily mutilation since hair typically
grows back.
Self-mutilation is understood as some form of self-harm without suicidal intent. Some
argue that self-mutilation falls into the category of nonviolent action because no one, other
than the actionist, is harmed. Others argue that a method can only fall under the category of
nonviolent action if it does not cause harm to anyone, including the actionist.
Creative Intervention: Constructive Acts of Commission
As noted above, this monograph introduces a self-standing category of nonviolent tactics:
creative intervention (also known as prefigurative methods or constructive program). These
tactics include efforts to establish patterns of behavior and norms or political, economic,
social, and cultural institutions that in some way challenge the established order or an adver-
sary. These actions sometimes use rewards and constructive approaches to persuade and/
or lessen the defensiveness of an opponent or an opponent’s control over society—without
necessarily challenging them directly. This intervention often aims at a long-term transforma-
tion of society, rather than immediately targeting a visible opponent (e.g., head of a regime).
Common tactics include alternative markets, transportation systems, social institutions, com-
munications, economic institutions, and dual sovereignty and parallel government.
American settlers in the 18 th century, Poles and Algerians in the 19th century, and
Ghanaians and Zambians in the 20th century built indigenous economic, social, political, and
religious associations as a form of resistance against colonization—often parallel to the formal
institutions of the subjugating system. Gandhi helped popularize relentless action as a form
of cultural and economic resistance. He famously led Indians to spin cotton to encourage
self-reliance and to decrease dependency on the British textile mills. The spinning wheel
remains the central motif on the national flag of India. Gandhi also famously walked to the
sea and encouraged others to make their own salt (an illegal activity), which he believed was
the right of any Indian. Millions of Indians joined him, and the British colonial regime was
shaken and weakened due to the loss of tax revenue and the practice of massive disobedi-
ence. The constructive intervention used in a conflictual setting has various names including
prefigurative action, constructive program, creative intervention, institution building, parallel
institutions, cultural resistance, acting-out-the-future-today, and nonviolent action to “make
or create.”
62
The reverse strike produces an unusual example of unilaterally rewarding an opponent,
such as in Italy when unemployed men (that the government was not helping) repaired a
much-needed road (that the government was not repairing).
Here are some newly identified tactics for each of the five subcategories of creative
interventions: political, economic, social, physical, and psychological.
■ Political/Judicial Creative Intervention
This intervention is a positive set of physical and material actions that seek to reward or
persuade an opponent or potential ally to tolerate or support a movement’s goals. Examples
include reverse trials, mock elections, and citizen inspections.
C I T I ZE N I NSP E CT IO NS
Citizens inspect the work of government officials to obtain and release data that compro-
mise the agency and draw support for a movement’s cause. In the 1970s in Pennsylvania
(USA), a citizen inspection team secretly “inspected” an FBI field office and removed doc-
uments that proved massive illegal surveillance and sabotage against social movements
on the part of the FBI. While not directly advocating for replicated citizen inspections, this
effort succeeded in pressuring lawmakers to engage in stronger oversight and to curtail
abuses by unaccountable law enforcement agencies. Another example comes from East
German citizens who entered the Secret Police’s main headquarters in East Berlin on
January 15, 1990, and saved important, revelatory files from planned destruction. Other
citizen investigators uncovered Secret Police surveillance equipment in businesses and
the post office in the East German capital.
However, citizen inspections are sometimes theatrical productions that feature large
magnifying glasses and cameras because authorities often deny physical entry to a space.
■ Economic Creative Intervention
These tactics seek to establish new economic institutions or relationships through a reward
system to persuade an opponent or to model better ways of organizing economic affairs.
Examples of this include buycotts (e.g., buying services or products from others instead of from
boycotted opponents), copyleft (alternative to copyright, making materials freely available as
long as attributions are provided), and property expropriation (the latter explained below).
P R O PE RT Y E XP RO P RIAT IO N
This tactic involves workers permanently taking over factories and turning them into coop-
erative worker-owned enterprises. In 2001, more than 300 factories were bankrupted in the
economic collapse in Argentina. Bankrupt owners fled because of charges of fraud or they
63
were simply replaced by the workers who took over management and ownership of the
enterprises (Balch, 2013).
Property expropriation is an example of a powerful kind of nonviolent action that is best
expressed by the term “takeover.” There are many instances in which an abusive state (or in
this case, corporations) is confronted by mass actions in which state institutions are not
replaced with new ones, but are simply taken over and the old employees pushed out. These
methods do not always fit comfortably in our category of persuasive or reward actions.
■ Social Creative Intervention
An act of commission, this category of action is defined by physical or material actions to
change social behaviors and cultural institutions in constructive ways. In self-determination
and identity conflicts, activists sometimes use these methods in conjunction with political and
economic objectives. Examples of this include women becoming religious leaders without
official approval, pray-ins, and marriage inclusion (the latter explained below).
MA R R I AG E INCLUSIO N
Marriage is a global cultural practice strongly tied to religion. Throughout history, there have
been many transgressive examples of people seeking to marry despite social, religious, or
legal taboos. When many people seek to celebrate taboo marriage in a coordinated effort,
it constitutes civil resistance. One recent example is the same-sex marriage equality move-
ment. Beginning in the 1970s, same-sex couples began openly participating in alternative
marriage ceremonies in various European and North American countries. These marriages
were not legally recognized by any jurisdiction.
In many countries, religious institutions conduct marriages which are then ratified by the
state. LGBTQ activists began campaigns to persuade their religious communities to conduct
their marriages, even if the state refused to recognize them. The LGBTQ community, excluded
from most religious communities, organized its own church called the Metropolitan Community
Church (MCC). The MCC has grown to over 220 congregations (as of 2017),44 some of which
are still operating illegally in countries such as Uganda, Nigeria, Iran, and Saudi Arabia, where
members face extreme risk. The MCC conducted its first marriage in 1969. Subsequently, in
the 1970s, various denominations began performing same-sex unions and, by the 1990s,
same-sex marriages. Although not then recognized by law, these denominations helped
apply political, moral, and social pressure on the public, the legal system, and legislators to
support same-sex marriage.
44
For more information, see https://www.mccchurch.org/.
64
■ Physical Creative Intervention
These methods primarily use human bodies or materials to obstruct the opponent by means of
new social practices or material changes, such as digging up tarmac to create a garden. Examples
of this include kiss-ins, defiance of blockades, and critical mass (the latter explained below).
C R I T I CA L MASS ( CYCL ING)
Cyclists organize critical mass actions regularly in an effort to reclaim the streets from motor
vehicles. In these protests, cyclists gather in large numbers—typically on the afternoon of the
last Friday of the month—and hold a mass improvisational ride through city streets, often
disobeying traffic laws. This event originated in its modern form in 1992 in San Francisco,
California (USA) (Garofoli, 2002). The purposes of the event included challenging the domi-
nation of motor vehicles, encouraging respect for cyclists, and modeling a future in which
cycling is the dominant (more environmentally friendly) form of transportation.
■ Psychological Creative Intervention
These methods focus on psychology, using rewards or new paradigms to influence how
people think. Examples include awards as encouragement, flowers in guns, and self-imposed
transparency (the latter explained below).
SE L F - I MP OSE D T RANSPARE NCY
A major challenge for organizers of the 2012 protests against rigged parliamentary elections
in Russia was to convince potential protesters to trust the organizers. In an effort to do so,
the Networked Public TV (SOTV), a new online channel, broadcast discussions about the
behind-the-scenes of organizing protests, including the financials. Voluntary financial trans-
parency is thus a powerful technique, not only to increase organizers’ credibility, but also to
shame an opaque and corrupt government.
The 23 tactics described above are a sampling of a large and growing dataset in the
field of civil resistance. The tactics vary widely in location, cultural or political contexts, size,
and frequency. Some tactics also overlap in terms of their persuasive or coercive character.
Many tactics also fall on the margins of civil resistance movements, which will be discussed
in the next chapter.
65
CHAPTER 6. On the Edges of
Civil Resistance Tactics
On the edges of mass political action are tactics that were excluded purposefully from the
list of nonviolent methods compiled in the Universe of Nonviolent Tactics Appendix. Some
of them are routine (an everyday behavior), yet they might still create a pattern of nonviolent
subversion and resistance. Others are violent, though might be used within nonviolent cam-
paigns and perceived by some to be nonviolent and/or justified because they do not phys-
ically harm others. Given the ongoing debates around these actions and their frequent
entanglement with civil resistance, they are identified with brief elaborations below.
Everyday Resistance
Everyday resistance, according to Scott (2008, 21), is criticized for its inclusion as a civil resis-
tance phenomenon because these actions: “1) [are] unorganized, unsystematic, and individual;
2) [are] opportunistic and self-indulgent; 3) [have] no revolutionary consequences; and/or 4)
imply in their intention or logic an accommodation with the structure of domination.” As a
result of these characteristics, everyday resistance has remained outside of some civil resis-
tance scholars’ purview.
Certainly, challenging the status quo by breaking established norms and disruption is at
the heart of most nonviolent civil resistance. But civil resistance organizers would consider
many of the norm or law-breaking instances of everyday resistance, such as stealing and
lying, as common criminality or examples of means that do not match the ends and, thus, fall
outside of the nonviolent resistance repertoire.
The term “everyday resistance” can cause confusion with the daily resistance of the
constructive program. Both are actions that occur on a continual basis such as spinning cot-
ton, attending a language school for an unrecognized minority language, squatting, tax
evasion, or creating autonomous social space for the assertion of dignity. One distinction
between many of these actions is that many constructive programs are visible and many
everyday resistance activities are disguised.
Property Destruction and Transformation
Many movements frequently use property destruction and transformation (both being
recognizably subjective terms). Sharp and many practitioners exclude this type of action
from the nonviolent method repertoire. Some campaigns focus on property transformation
as constructive program rather than a nonviolent sanction. The Christian anti-nuclear
66
weapons group, Plowshares, physically hammers on nuclear weapons delivery systems
and their infrastructure. Plowshare activists, acting out the Bible’s verse on beating swords
into plowshares, seek to transform nuclear weapons and their delivery systems into
allegorical plowshares.
Environmental activists have burned, destroyed, and damaged property such as animal
cages, dams, fur coats, and ski lodges. Many countries have experienced campaigns in which
citizens transcribed or stamped money with protest messages. Anarchists, suffragettes, and
others have engaged in attacks on private and corporate property as a form of protest.
If an artist joined a campaign and sacrificed their own painting as a tactic, we would
consider their destruction of their own property to be a method of nonviolent action. On the
other hand, when José Bové, a French farmer, smashed a McDonald’s restaurant with a tractor,
it may have been justified, but it is not considered nonviolent action. Destruction of communal
or public property is highly contextual but, generally speaking, it is excluded from our defini-
tion of nonviolent action.
Some argue that property destruction can be considered nonviolent action if no person
is physically targeted or hurt. However, no one considers the burning of the Reichstag
(Germany’s Parliament) in 1933 to be a nonviolent action. Many believe that most forms of
property destruction are akin to a physical threat or violence. Marxists and anarchists, among
others, critique the private ownership of the means of production as a form of profound
oppression of the 99%. As a result, for-profit corporations, particularly banks, have been
regularly targeted with window smashing.
Since the labeling of property destruction tactics as violent or nonviolent is so contextual,
many have focused their attention on the effectiveness of property destruction in a nonviolent
campaign. Tom Hastings has proposed a list of five elements that determine whether the
destruction of something may be helpful to a nonviolent campaign, including the premise
“that no private individual’s property is destroyed” (Hastings, 2020).45
These acts have been controversial in terms of morality and effectiveness. Some activists
believe that public property destruction is ineffective and counterproductive because it can
cause opponents to harden their stance and steer potential allies away from sympathizing
with and supporting the destructive group. Property destruction and transformation tactics
are used (often controversially) in nonviolent civil resistance. Advice on strategies for effec-
tively using property destruction are beyond the scope of this study.
45
For more information, see: Hastings, Tom. “Property Damage, Violence, Nonviolent Action, and Strategy.”
Minds of the Movement (blog). ICNC, June 2, 2020. https://www.nonviolent-conflict.org/blog_post/
property-damage-violence-nonviolent-action-and-strategy/.
67
Suicide
Activist suicides are another method that is sometimes utilized in nonviolent campaigns. These
activists have sacrificed their lives without threatening or injuring their opponents. Irish nation-
alist Bobby Sands fasted to death in British prisons while demanding British expulsion from
Northern Ireland. A Korean farmer killed himself standing on top of police barricades in Mexico
during protests against the World Trade Organization’s policies. Self-immolation has also been
widely used in movements around the world. Examples include Tibetans protesting Chinese
occupation, Vietnamese and Americans protesting U.S. occupation of Vietnam, and a Pole in
2017 protesting the conservative government that he saw as a threat to democracy.
In addition, thousands of citizens have died by exposing themselves to high-risk situations
such as Rachel Corrie who, on March 16, 2003, stood in front of an Israeli bulldozer in Palestine.
In 1913, suffragette Emily Davison died attempting to hang a banner while riding a galloping
horse in front of the British King. In India, the Iron Lady of Manipur tried to kill herself in a
16-year long hunger strike (2000 to 2016) against police violence and impunity. She was
arrested and force-fed regularly by the Indian military.
Suicide is widely perceived as a violent and psychologically brutal act, particularly when
done without the support of fellow campaigners and/or family. Groups or campaigns have
organized very few suicidal acts, and these are therefore considered renegade actions.
However, the right-to-die movement may be an exception. Within this movement, indi-
viduals with limited time to live and poor quality of life are committing suicide with the planned
support of their friends and, in some cases, physicians. Civil disobedience against laws for-
bidding suicide for terminally ill patients is a world-wide phenomenon. Bond et al. describe
suicide as a violent practice with a high level of contentiousness and almost no coercive
component, instead relying on altruist appeals of empathy to pressure for change. Further
study is needed on the effectiveness of suicide as a method of civil resistance in various
contexts. Suicidal actions are excluded from the list of nonviolent methods listed in the
Universe of Nonviolent Tactics Appendix.
Third-Party Nonviolent Actions
Mohandas Gandhi proposed the Shanti Sena (a peace army) as a third-party to intervene
nonviolently in sectarian conflict. Beginning in the early 1980s, this inspired groups such as
Peace Brigades International and Witness for Peace to send individuals to Central America
to protect citizens from harm at the hands of the state and other armed actors. Using their
privilege to monitor without becoming solidarity activists, these individuals developed a set
68
of nonviolent techniques to protect civilians that has now been taken up by many other
groups around the world.
Examples of non-partisan methods include accompaniment of threatened human rights
defenders, protective presence for indigenous people or groups, reporting threats and
humanizing actors, and witnessing. Mahoney and Eguren (1997), Schweitzer (2001, 2010), and
Clark (2009) have documented and analyzed the emergence of non-partisan third-party
action.46 Civil resisters have appropriated some of these techniques and applied them in their
grassroots campaigns for social change. For example, in Colombia, locals (who have some
political or social status) have accompanied human rights defenders to provide security—
something that was previously only carried out by non-partisan foreigners.
The boundaries of partisanship and parties in a conflict vary. The International Solidarity
Movement and Christian Peacemaker Teams use some of these techniques in Palestine and
elsewhere as invited outsiders with a solidarity viewpoint. In some cases, these groups
become direct participants in the conflict and these non-partisan techniques (such as report-
ing) could be considered methods of civil resistance. Foreigners who are seen as actively
intervening on behalf of one side of a conflict are often deported or denied visas, and there-
fore solidarity activists sometimes cannot sustain the techniques. Most third-party nonviolent
interventions are not currently included in the Universe of Nonviolent Tactics Appendix, unless
local actors have appropriated the intervention, as is sometimes the case with internal accom-
paniment led by movement members or their supporters.
Negotiation and Dialogue
Negotiation in the context of civil resistance campaigns is often thought of as a dialogue
across a negotiating table at a point in a conflict when the parties believe it useful or neces-
sary to reach an agreement. Wanis-St. John and Rosen (2017) propose a close relationship
between successful negotiation and civil resistance in which civil resisters generate leverage
46
Beer (1995) presented 11 types of third-party nonviolent interventions (TPNI): accompaniment, delegations, eco-
nomic sanctions such as boycotts, election monitoring, human rights observation and investigations, information
gathering and polling, media, mediation, peace walks/boating, physical interposition, and rescue teams/humani-
tarian assistance. Hunter and Lakey (2003) have identified four types of TPNI: interposition, observing/monitor-
ing, protective accompaniment, and presence. Burrowes (Moser-Puangsuwan and Weber, 2000) classifies
cross-border action on behalf of indigenous nonviolent movements across nine categories: local nonviolent cam-
paigns, mobilization actions, nonviolent humanitarian assistance, nonviolent witness and accompaniment, nonvi-
olent intercession, nonviolent solidarity, and nonviolent reconciliation and development. Dudouet (2015) defines
nonviolent intervention by focusing on relational mechanisms vis-à-vis local campaigners or power-holders: pro-
moting, capacity-building, connecting, protecting, monitoring, and pressuring.
69
for negotiators, who in return obtain benefits from an adversary.47 Negotiations have not
typically been included as a tactic of nonviolent action because they are not unilaterally
initiated; they require an interlocutor. In addition, classic negotiations have procedurally
known outcomes: a) agreements are reached, or b) agreements are not reached. Dialogue
likewise is not considered a civil resistance tactic here because it requires the participation
of an adversary. There are, however, numerous examples of dialogues that take the form of
civil disobedience because they are forbidden.
Wanis-St. John writes that the “symbiosis between civil resistance and negotiation is
long-standing. Martin Luther King Jr.’s logic was to compel segregationists to negotiate in the
pursuit of greater social justice... After mass mobilization and strikes, the Polish Solidarity
movement negotiated its way into power and transitioned Poland away from authoritarianism.
Saul Alinsky, one of the United States’ pioneering community activists, included negotiation
as a critical step in his blueprint for action” (USIP, 2017, 5).
Weber goes further to argue that Gandhian nonviolence incorporates modern notions
of negotiation and conflict resolution, as both are grounded in a win-win conflict framework.
Vinthagen (2016), for his part, argues that nonviolent action is inherently dialogic. He proposes
that all observed human behavior is a form of communication and that all nonviolent actions
are a form of social conflict. Even if tactical actors protest without “listening,” their goal is not
to shout at trees and rocks but to reach the ears of another party. Therefore, all nonviolent
methods unilaterally initiate a form of dialogue.
Smithey and Kurtz (2002, 319-359) offer a fascinating example of a wordless dialogue
between Northern Ireland’s Loyalists and Nationalists by way of nonviolent action. In the
1990s, the Loyalists of Ballyreagh theatrically abstained from a blockade of a nationalist
parade. In response, at the parade, the nationalists self-policed their supporters to prevent
conflict with the police. Over a period of months of reciprocal restraint, these parades were
used as communication vehicles to build non-verbal trust that helped lead to spoken dialogue
and negotiations.
However, because of the lack of universal agreement, we do not include dialogue and
negotiation in our set of nonviolent methods.
47
Wanis-St. John and Rosen define leverage as the “manifestation of power in a negotiation. Leverage involves the
ability to influence a negotiation outcome based on a party’s ability to either confer or withhold benefits desired
by the counterpart or impose or not impose costs on the counterpart.”
70
Lobbying
Lobbying is a routinized activity in representative governance with known outcomes. Typically,
lobbying entails citizens engaging an elected official at a public event or at their offices, or
by means of a letter or phone call. Regardless of the content of the plea or demand, the
outcome is fully in the hands of the representative who has the power to vote, and not the
citizen lobbying. In practice, norms of personal conduct and communication can be disrupted
and lobbying can be contentious, coercive, and still nonviolent. For example, in 2017, the
author of this monograph participated in disruptive lobbying involving a teach-in conducted
in a congressional office. The teach-in lasted for many hours until the lobbyists’ demands for
a public written statement were met. This teach-in was a form of creative and assertive lob-
bying that Medea Benjamin calls extreme lobbying.48 Lobbying itself is generally not consid-
ered a method, but many nonviolent methods such as teach-ins, sit-ins, and call-ins are used
to lobby elected officials.
Logistical Support Activities for Nonviolent Tactics
Tactical support activities are deployed to increase the chances of a nonviolent tactic suc-
ceeding. Because they are not weapons that target an adversary through the processes of
persuasion, coercion, or manipulation, they are not defined as a nonviolent method in this
monograph. However, some researchers (e.g., Beyerle, 2014) see them as important enough
for the conduct of nonviolent conflict that they integrate them into the repertoire of nonviolent
tactics. We have identified three types of support activities with specific examples:
PLANNING
■
 back-up plans
■
 evaluation
■
 messaging
■
 objectives
■
 scouting and assessment
■
 sequence planning for an action from beginning to end
■
 target analysis and selection
■
 timing, duration, and location
48
Labeled as such in conversations with Benjamin in 2018.
71
O R GA N I Z I NG AND LO GIST ICS
■
 finance
■
 food and clothing
■
 fundraising
■
 internal and external media
■
 internal communications
■
 recruiting actionists, allies, and people for support roles
■
 resource design and preparation
■
 sound systems
■
 transport
TRAINING
■
 coordination of roles and leadership
■
 discipline and action guidelines, safety and security awareness, and preparation
■
 training coordination; venues, times, recruitment of trainers and trainees
■
 training guides, manuals, and videos
■
 training of trainers
Psychological Attack
Confrontational tactics that seek to weaken an opponent’s will are commonplace in civil
resistance. Sharp’s classic list of methods includes rude gestures, mockumentaries, vituper-
ative speech and writing, nudity, hoaxes, and rumors. These tactics have the potential power
to cause deep psychological trauma or temporary discomfort. Followers of Gandhian nonvi-
olence generally believe these methods, along with secrecy and property destruction, to be
harmful to others and counter-productive to campaign success. Wilson (2017) distinguished
between nefarious and virtuous nonviolent campaigns by the presence or absence of a
“human rights ethos.” This ethos is based on four principles of non-discrimination (“equal
treatment of others”); non-repression (“goals that advance political rights and political auton-
omy for all”); non-exploitation (“solidarity with the persecuted, offering mutual aid, and reach-
ing out to diverse groups of society to build shared understanding and trust”); and nonviolent
means (“actions that do not threaten or do physical harm to others”).
Such human rights yardsticks could be used to determine what methods might fall into
the category of nonviolent action and whether intentions or agents behind such methods
had virtuous or nefarious (anti-human rights ethos) ends in mind. This could help determine
if they were part of a genuine civil resistance movement or not.
72
Actions Seemingly Without a Strategic Goal
Vinthagen (2015) writes about the functions of social norm regulation (e.g., dressing up in
formal clothes or not striking back) and dialogical communication in nonviolent campaigns
(e.g., actively listening to an adversary), which are important for normalizing nonviolent action
in societies characterized by violence and oppression. This prefigurative and cultural work
seeks to enact marginalized norms as routine behaviors without explicit strategic goals, but
this may depend on the context. We leave these questions open, though this monograph
does not incorporate such actions into its categorization of nonviolent methods.
In conclusion, many actions in civil resistance campaigns are not currently categorized
as nonviolent tactics because there is significant debate about their function, the relationship
between their means and ends, the nature of conflict and violence, and the role of commu-
nication and dialogue. It is beyond the scope of this monograph to determine if such actions
can or should be part of the civil resistance repertoire. However, rather than seeing these
different views of tactics as a sign of weakness or vagueness in the study of nonviolent civil
resistance, we should interpret the diversity of creative actions and surrounding debates as
a sign of the vibrancy and breadth of the field.
73
CHAPTER 7. Key Takeaways
This monograph sought to answer three basic questions:
1.	What civil resistance tactics did Gene Sharp not identify, and what new methods of
civil resistance have emerged or been identified since 1973? In response, this mono-
graph introduces readers to a new and regularly updated database of more than 300
methods of nonviolent resistance (see Universe of Nonviolent Tactics Appendix).
2.	 What new categorization of tactics would be helpful in documenting this enormous
area of human activity? In response, this monograph builds on Sharp’s categorization,
reviews other available classifications, and offers its own refined framework (included
in Tables 1 and 4).
3.	 How can this new knowledge—tactics and classification—be helpful to practitioners
and scholars of civil resistance, as well as to those who would like to assist nonviolent
movements? The answer to this third question is provided in greater detail in the
paragraphs below.
Takeaways for Activists
First, this study clearly demonstrates that many more nonviolent methods exist than have
previously been documented. In fact, the author of the study collected information on more
than 100 new tactics beyond the 198 identified in 1973 by Sharp.
This monograph selected and described only a limited number of new tactics (23) and
mapped each of them onto the new framework in Table 4. The selected tactics cover a diverse
range of characteristics. This is a testament to the ingenuity and creativity of activists around
the world in developing and deploying new nonviolent tactics.
Activists can use this monograph’s framework to better understand the variety of functions
that nonviolent tactics perform. They can also use this framework to map out the tactics that
they have deployed as part of their past and ongoing campaigns. This can help them visualize
the extent to which their actions are tactically diverse and whether they might be placing too
much emphasis on one set of tactics while intentionally or unintentionally disregarding others.
This understanding, in turn, can help activists consider how their efforts can be re-balanced
to increase the diversity of methods deployed and thus enhance their strategic leverage.
Practitioners should not conclude that certain tactics have lesser or greater value based
on their categorization. Cataloguing civil resistance methods is not meant as a prescription
or seal of approval for the actions identified in this monograph. Application of any nonviolent
tactic should be determined by two key variables: appropriateness and efficacy. In other
74
words, is the action the “right” or appropriate thing to do in the given circumstances, and is
the action an “effective” thing to do to achieve short- and long-term goals?
It is also worth reiterating that many of the identified tactics can be categorized in more
than one way. They can also have various helpful impacts for a movement depending on the
movement’s strategies and the overall context of the struggle. In addition to tactical preparation
and skilled delivery, it will be the strategy behind nonviolent tactics that determines their value.
This monograph’s analytical mapping of tactics dissects the nature of various categories, which
aims to help activists undertake more strategic assessment and planning.
Takeaways for Civil Resistance Scholars and Students
Categorizing tactics can be done in myriad ways. Strategists and social scientists are con-
stantly discovering new patterns, relationships, and insights by re-classifying tactics based
on various criteria, including dispersion/concentration, actors’ motives, action audience, target
object (will, body, or environment), stages of conflict, grievance issue, etc.
This monograph provides a number of examples of new methods to deepen understand-
ing of two kinds of civil resistance actions: coercive or confrontational (threats and sanctions)
on the one hand, and persuasive or constructive (appeals and rewards) on the other hand.
Both are important.
In particular, this monograph explores in detail positive inducements such as appeals,
refraining, and creative intervention, which can help researchers further explore the con-
structive dimension of various types of nonviolent methods. This monograph also aims to
emphasize disruptive elements of civil resistance actions. Coercive nonviolent actions are
often visible and dramatic (e.g., demonstrations, protests, and occupations). Thus, they can
seemingly convey greater power and leverage of a movement than positive or constructive
actions which, even if subtler, hidden, or less spectacular (e.g., appeals, persuasion, or alter-
native institution-building), can nevertheless have a profound impact on the trajectory of the
struggle. To balance this, the framework presented in this monograph elucidates, among
others, a new category of positive acts of omission entitled refraining, which is no less pow-
erful than its disruptive counterparts and has been practiced for decades but until now had
not been analytically identified or studied.
Researchers could expand on the framework presented in this monograph to investigate
nonviolent campaigns and movements. This could include examining the examples, sequenc-
ing, effectiveness, and comparison of both coercive and persuasive tactics and thus provide
insights into the overall conduct of a nonviolent struggle.
75
Furthermore, in Chapter 6, this monograph identifies a host of actions and dynamics in
civil resistance campaigns that need further study (including the use of suicide, psychological
intervention, property destruction, social norm regulation, utopian enactment, dialogical
communication, third-party intervention, and everyday resistance) to determine their proper
analytical place in the categorization of civil resistance methods.
More research, data collection, and analysis are required to clarify the criteria for defining
nonviolent methods and tactics and to deepen our understanding of their underlying mech-
anisms. How should we address subcategories of civil disobedience tactics against unjust
laws, which make up a huge array of illegal actions that have little in common with one another,
other than that a government has labeled them illegal? What about tactics that are so con-
text- specific, such as wade-ins (physical entrance to a prohibited space) to integrate pools
and beaches in South Africa, that they are unlikely to be replicated in many other settings
and thus scarcely warrant analytical attention or a framework to capture them? Should the
method of banner display be subdivided into numerous categories of tactics because of the
diversity of techniques, venues, and materials used, and the different deployment skills and
resources that each of them requires? What would a gender perspective of understanding
nonviolent tactics reveal for those studying civil resistance? Several questions remain to be
studied and clarified.
This monograph aims to inspire civil resistance researchers and experts to document
nonviolent tactics and map them into a universal framework that can offer a certain analytical
order and clear systematization of the type, nature, and impact of nonviolent tactics.
Takeaways for Groups Interested in Supporting Nonviolent Movements
External actors can use this new framework to assess a movement’s tactical diversity and
how a movement’s actions are distributed across the spectra of tactical categories. This may
offer them information about the strengths and weaknesses of a movement’s organizational
capacity, strategic planning, and how a movement views the conditions it faces. This, in turn,
can offer external actors hints about what to address in discussions with movement partici-
pants, as well as which tools and techniques external actors can employ to help a movement
become more strategic in terms of design and/or deployment of their nonviolent tactics.
Concretely, this monograph provides an updated table and framework for understanding
and analyzing nonviolent tactics. Any categorization of human endeavors is, to a certain
extent, a simplification of reality, but given the extraordinary breadth of civil resistance, the
methods template provided here can become an invaluable teaching tool. External trainers
or advisors for a group of activists representing a specific campaign could, for example, use
76
Tables 1 and 4 as handouts to distribute as a teaching tool to facilitate reflection, analysis,
and strategy with regards to nonviolent tactics.
In conclusion, exploring civil resistance tactics is not just a simple documentation or
classification exercise. Studying each individual nonviolent tactic opens up a world of civil
resistance stories in various places and times. It offers insight into people’s ingenuity, perse-
verance, and resilience, often in the face of repression, demonstrating the desire to be cre-
ative and strategic in leading resistance struggles.
We hope this monograph and the ongoing Nonviolent Tactics Database will allow all of
us to continue identifying, collecting, and cataloguing new tactics of civil resistance that
activists and organizers around the world use in their ongoing struggles for rights, freedom,
justice, and sustainability.
77
Cited Bibliography
Ackerman, Peter and Jack DuVall. A Force
More Powerful: A Century of Nonviolent Con-
flict. New York, NY: Palgrave, 2000.
Ackerman, Peter. “Skills or Conditions: What
Key Factors Shape the Success or Failure of
Civil Resistance?” Conference on Civil Resis-
tance and Power Politics, Oxford University
(March 15-18, 2007), pp. 1-9. Retrieved from:
https://www.nonviolent-conflict.org/resource/
skills-conditions-key-factors-shape-success-fail-
ure-civil-resistance/
Ackerman, Peter and Christopher Kruegler.
Strategic Nonviolent Conflict: The Dynamics of
People Power in the Twentieth Century. West-
port, CT: Praeger Publishers, 1994.
Ackerman, Peter and Hardy Merriman. “The
Checklist for Ending Tyranny” in Mathew Bur-
rows and Maria J. Stephan (eds.), Is Authoritar-
ianism Staging a Comeback? Washington, DC:
The Atlantic Council, 2015.
“Airport Workers’ Strikes Suspended at Newark
Liberty, 3 Other Airports,” News12 New Jersey,
July 12, 2017. Retrieved from: http://newjersey.
news12.com/story/35862766/airport-work-
ers-strikes-suspended-at-newark-liberty-3-
other-airports
Amnesty International UK. “Huge Response to
Crowdfunded Newspaper Ads Campaign
Opposing Repeal of Human Rights Act,” May
20, 2015. Retrieved from: https://www.amnesty.
org.uk/press-releases/huge-response-crowd-
funded-newspaper-ads-campaign-oppos-
ing-repeal-human-rights-act
Awad, Mubarak and Laura Bain. Organizing
Tactics for Nonviolent Action: Fasting. Washing-
ton, DC: Nonviolence International, 1995.
Balch, Oliver. “New Hope for Argentina in the
Recovered Factory Movement,” The Guardian,
March 12, 2013. Retrieved from: https://www.
theguardian.com/sustainable-business/argen-
tina-recovered-factory-movement
Bartkowski, Maciej. “How Online Courses on
Civil Resistance Can Make Real Impact,” ICNC
Minds of the Movement, February 13, 2019.
Retrieved from: https://www.nonviolent-con-
flict.org/blog_post/online-courses-civil-resis-
tance-can-make-real-impact/
Bartkowski, Maciej. Recovering Nonviolent His-
tory: Civil Resistance in Liberation Struggles.
Boulder, CO: Lynne Rienner Publishers, 2013.
Beer (1995), Mainstreaming Peace Teams, Col-
lated and Published by Nonviolence Interna-
tional.
Beyerle, Shaazka. Curtailing Corruption: Peo-
ple Power for Accountability and Justice. Boul-
der, CO: Lynne Rienner Publishers, 2014.
Bloch, Nadine. Education and Training in Non-
violent Resistance. Washington, DC: United
States Institute for Peace, 2016.
Bloch, Nadine. “The Arts of Protest: Creative
Cultural Resistance.” ICNC Webinar. Retrieved
from: https://www.nonviolent-conflict.org/
the-arts-of-protest-creative-cultural-resistance/
Bond, Doug. “Nonviolent Direct Action and the
Diffusion of Power” in Paul Wehr et. al. (eds.),
Justice Without Violence. Boulder, CO: Lynne
Rienner Publishers, 1994, pp. 59-79.
Bond, Doug, et. al., “Mapping Mass Political
Conflict and Civil Society: Issues and prospects
for the automated development of event data.”
Journal of Conflict Resolution, vol. 41, Aug 1997.
Boserup, Anders and Andrew Mack. War with-
out Weapons: Non-violence in National
Defense. New York, NY: Schocken, 1975.
Browder, Bill. “The Russians Killed My Lawyer.
This Is How I Got Congress to Avenge Him,”
Politico, February 3, 2015. Retrieved from:
h t t p s : / / w w w. p o l i t i c o . c o m / m a g a z i n e /
story/2015/02/sergei-magnitsky-mur-
der-114878
78
Burrowes, Robert J. “Cross-border Nonviolent
Intervention: A Typology” in Yeshua Mos-
er-Puangsuwan and Thomas Weber (eds.), Non-
violent Intervention Across Borders: A Recur-
rent Vision. Honolulu, HI: University of Hawai’i
Press, 2000.
Burrowes, Robert J. The Strategy of Nonviolent
Defense: A Gandhian Approach. Albany, NY:
State University of New York Press, 1995.
Canning, Doyle and Patrick Reinsborough.
“Changing the Story: Story-Based Strategies for
Direct Action Design,” Smart Meme, May 2008.
Retrieved from: https://inthemiddleofthewhirl-
wind.wordpress.com/changing-the-story/
Center for Artistic Activism and the Yes Lab.
Actipedia. Retrieved from: https://actipedia.org/
Chenoweth, Erica. “Trends in Civil Resistance
and Authoritarian Responses” in Mathew Bur-
rows and Maria J. Stephan (eds.), Is Authoritar-
ianism Staging a Comeback? Washington, DC:
The Atlantic Council, 2015, pp. 53-62.
Chenoweth, Erica and Maria Stephan. Why Civil
Resistance Works: The Strategic Logic of Non-
violent Conflict. New York, NY: Columbia Uni-
versity Press, 2011.
Chenoweth, Erica and Maria Stephan. “How the
world is proving Martin Luther King right about
nonviolence,” Washington Post, January 18,
2016.
Christoff, Stefan. “Cacerolazo.” Beautiful Trou-
ble. Retrieved from: http://beautifultrouble.
org/tactic/cacerolazo/
Clark, Howard (ed.). People Power: Unarmed
Resistance and Global Solidarity. London, UK:
Pluto Press, 2009.
Cortright, David. Peace: A History of Move-
ments and Ideas. Cambridge, UK: Cambridge
University Press, 2008.
Crawford-Brown, Terry. “International Sanc-
tions Against Israeli Banks,” Just World Educa-
tion, August 4, 2017. Retrieved from: http://
justworldeducational.org/2017/08/interna-
tional-sanctions-israeli-banks/
Curle, Adam. Making Peace. London, UK: Tav-
istock Publications, 1971.
Deming, Barbara. On Revolution and Equilib-
rium. New York, NY: A. J. Muste Memorial Insti-
tute, 1971.
Dorjee, Tenzin. The Tibetan Nonviolent Strug-
gle: A Strategic and Historical Analysis.
Washington, DC: ICNC Press, 2015.
Dudouet, Véronique (ed.). Civil Resistance and
Conflict Transformation: Transitions from Armed
to Nonviolent Struggle. Abingdon, UK: Rout-
ledge, 2014.
Dudouet, Véronique. Powering to Peace: Inte-
grated Civil Resistance and Peacebuild-
ing Strategies. Washington, DC: ICNC Press,
2017. Retrieved from: https://www.nonvio-
lent-conflict.org/powering-peace-integrat-
ed-civil-resistance-peacebuilding-strategies
Dudouet, Véronique. “Sources, Functions and
Dilemmas of External Assistance to Civil Resis-
tance Movements,” in Kurt Schock (ed.) Civil
Resistance: Comparative Perspectives on Non-
violent Struggle. Minneapolis, MN: University of
Minnesota Press, 2015.
Ebert, Theodore. Gewaltfreier Aufstand: Alter-
native zum Bürgerkrieg. (Nonviolent Uprising:
Alternatives to Civil War). Hamburg, Germany:
Fischer Verlag, 1970.
EFE. “Activists in Hong Kong Call for End to
Shark Fin Trade ahead of New Year,” January
30, 2016. Retrieved from: https://www.efe.
com/efe/english/life/activists-in-hong-kong-
call-for-end-to-shark-fin-trade-ahead-of-new-
year/50000263-2825615
79
French, Amber. “How Do Nonviolent Move-
ments Shape History? An Interview with Jacques
Semelin,” Minds of the Movement, October 16,
2017. Retrieved from: https://www.nonvio-
lent-conflict.org/blog_post/nonviolent-move-
ments-shape-history-interview-jacques-se-
melin/
“MEPs Cut EU Parliament Interpreting Service
to Back Strikers,” France24, July 3, 2018.
Retrieved from: http://www.france24.com/
en/20180703-meps-cut-eu-parliament-inter-
preting-service-back-strikers
Garano, Lorna. “Speaking Mirth to Power,” Pop-
ular Resistance, June 26, 2016. Retrieved from:
https://popularresistance.org/speaking-
mirth-to-power/
Garofoli, Joe. “Critical Mass Turns 10,” San
Francisco Chronicle, September 28, 2002.
Retrieved from: https://www.sfgate.com/poli-
tics/joegarofoli/article/Critical-Mass-turns-10-
A-decade-of-defiance-2767020.php
Geddie, John. “Hong Kong students take pro-
test to virtual world,“ Reuters, October 31, 2019.
Retrieved from: https://web.archive.org/
web/20191031181503/https://www.asiaone.
com/digital/hong-kong-students-take-pro-
test-virtual-world.
Goldman, David and Jose Pagliery. “#JeSuis-
Charlie Becomes One of Most Popular Hashtags
in Twitter’s History,” CNN, January 9, 2015.
Re t r i e v e d f r o m : h t t p s : / / m o n e y. c n n .
com/2015/01/09/technology/social/jesuis-
charlie-hashtag-twitter/index.html
Green, Matthew. “Ghana Puts Faith in Humble
Text Message,” Financial Times, December 8,
2008. Retrieved from: https://www.ft.com/
content/04a981ce-c553-11dd-b516-
000077b07658
Hallward, Maia Carter, and Julie M. Norman
(eds.). Understanding Nonviolence: Contours
and Contexts. Cambridge, UK: Polity, 2015.
Harrebye, Silas F. Social Change and Creative
Activism in the 21st Century. New York, NY:
Palgrave Macmillan, 2016.
Hastings, Tom. “Property Damage, Violence,
Nonviolent Action, and Strategy.” Minds of the
Movement (blog). ICNC, June 2, 2020. https://
www.nonviolent-conflict.org/blog_post/prop-
erty-damage-violence-nonviolent-ac-
tion-and-strategy/
Harvard Digital Collection. “Chilean Protest
Murals.” Retrieved from: http://www.jr-art.net
Hunter, Daniel and George Lakey. Opening
Space for Democracy: Training Manual for
Third-Party Nonviolent Intervention. Philadel-
phia, PA: Training for Change, 2003.
Jordan, John. “Clandestine Insurgent Rebel
Clown Army.” Beautiful Trouble. Retrieved from:
https://www.beautifultrouble.org/toolbox/#/
tool/clandestine-insurgent-rebel-clown-army
Joyce, Mary. “Civil Resistance 2.0: Digital
Enhancements to the 198 Nonviolent Methods.”
ICNC Webinar. Retrieved from: https://www.
nonviolent-conflict.org/civil-resis-
tance-2-0-digital-enhancements-to-the-198-
nonviolent-methods/
Khatib, Kate, et. al. We Are Many: Reflections
on Movement Strategy from Occupation to
Liberation. Chico, CA: AK Press, 2012.
King, Mary Elizabeth. Gandhian Nonviolent
Struggle and Untouchability in South India: The
1924-1925 Vykom Satyagraha and the Mecha-
nisms of Change. Oxford: Oxford University
Press, 2015.
Kurtz, Mariam M. and Lester R. Kurtz. Women,
War, and Violence: Topography, Resistance, and
Hope, volume II. Westport, CT: Praeger, 2015.
La Porte, Amy. “Muslim Leaders Refuse to Bury
French Priest Killer,” CNN, August 1, 2016.
R e t r i e v e d f r o m : h t t p s : / / w w w. c n n .
com/2016/07/30/europe/priest-killer-buri-
al-refused/index.html
80
Lakey, George. “The Sociological Mechanisms
of Nonviolent Action.” Peace Research Review
2 (1968), pp. 1-102.
Lee, Hyun. “South Korea’s Historic ‘One Million
People Protest’ to Oust Washington’s Puppet
President Park Geun-hye,” Global Research,
November 14, 2016. Retrieved from: https://
www.globalresearch.ca/south-koreas-histor-
ic-one-million-people-protest-to-oust-wash-
i n g t o n s - p u p p e t - p r e s i d e n t - p a r k- g e u n -
hye/5556808
Madden, Richard. “London: How cyclists around
the world put a spoke in the motorist’s wheel,”
The Daily Telegraph. December 15, 2003.
Retrieved from: https://www.telegraph.co.uk/
travel/729324/London-How-cyclists-around-
the-world-put-a-spoke-in-the-motorists-
wheel.html
Mahony, Liam and Luis Enrique Eguren.
Unarmed Bodyguards: International Accompa-
niment for the Protection of Human Rights.
Boulder, CO: Lynne Rienner Publishers, 1997.
Maney, Gregory M., et. al. (eds.). Strategies for
Social Change. Minneapolis, MN: University of
Minnesota Press, 2012.
Matthews, David. “Mexican Protestors are
Striking Back at Donald Trump—with Video
Games,” September 2, 2015. Retrieved from:
https://splinternews.com/mexican-pro-
g r a m mers-are-striking-back-at-donald-
trump-w-1793850467
McCarthy, Ronald et. al. (eds.). Protest, Power,
and Change: An Encyclopedia of Nonviolent
Action from Act-Up to Women’s Suffrage. New
York, NY: Garland Publishing, 1997.
Mitchell, Dave O. and Andrew Boyd. “Tactic:
Flash Mob.” Retrieved from: https://www.beau-
tifultrouble.org/toolbox/#/tool/flash-mob
Moser, Yeshua. Organizing Tactics for Nonvio-
lent Action: Organizing Walks and Pilgrimages.
Washington, DC: Nonviolence International,
1993.
Noplatform forIMF [YouTube user]. “Mic-Check
Disruption of the Speech of IMF Managing
Director, Christine Lagarde, 2013.” Retrieved
from: https://www.youtube.com/watch?v=ax-
LA-qG3EWg
Nurhan, Abujidi. Urbicide in Palestine: Spaces
of Oppression and Resilience. New York: Rout-
ledge, 2014.
Powers, Roger et. al. (eds.). Protest, Power, and
Change: An Encyclopedia of Nonviolent Action
from Act-Up to Women’s Suffrage. New York,
NY: Garland Publishing, 1997.
Principe, Marie. Women in Nonviolent Move-
ments. Washington, DC: United States Institute
for Peace, 2016. Retrieved from: https://www.
usip.org/publications/2016/12/women-nonvi-
olent-movements
Rigby, Andrew. Palestinian Resistance and
Nonviolence. East Jerusalem: PASSIA, 2010.
Retrieved from: https://www.academia.
edu/16293156/Palestinian_Resistance_and_
Nonviolence
Rigby, Andrew and Marwan Darweish. Popular
Protest in Palestine: The History and Uncertain
Future of Unarmed Resistance. London, UK:
Pluto Press, 2015.
Schell, Jonathan. The Unconquerable World:
Power, Nonviolence, and the Will of the People.
New York: Metropolitan Books, 2003.
Schelling, Thomas C. “Some Questions on
Civilian Defence,” in Adam Roberts, ed., Civilian
Resistance as a National Defence: Non-violent
Action against Aggression. Harrisburg, PA:
Stackpole Books, 1968.
81
Schlegel, Ivy. “Palm Oil Scorecard: Are Brands
Doing Enough for Indonesia’s Rainforests?”
Greenpeace.com. March 9, 2016. Retrieved
from: https://www.greenpeace.org/usa/palm-
oil-scorecard-are-brands-doing-enough-for-
indonesias-rainforests/
Schock, Kurt. Civil Resistance Today. Cam-
bridge, UK: Polity, 2015.
Schock, Kurt (ed.). Civil Resistance: Compara-
tive Perspectives on Nonviolent Struggle. Min-
neapolis, MN: University of Minnesota Press,
2015.
Schock, Kurt. “Land Struggles in the Global
South: Strategic Innovations in Brazil and India,”
in Maney, Gregory M., et. al. (eds.), Strategies
for Social Change. Minneapolis, MN: University
of Minnesota Press, 2012.
Schock, Kurt. Unarmed Insurrections: People
Power Movements in Nondemocracies.
Minneapolis, MN: University of Minnesota
Press, 2005.
Scott, James. “Everyday Forms of Resistance.”
Copenhagen Papers in East and Southeast
Asian Studies, vol. 4, May 1989, pp. 55-56.
Shaou, Patrick and Whea Dodge. “Imagining
Dissent: Contesting the Façade of Harmony
through Art in China,” in Stephen John Hartnett
et. al. (eds.) Imagining China: Rhetorics
of Nationalism in an Age of Globalization.
East Lansing, MI: Michigan State University
Press, 2017.
Sharp, Gene. Gandhi as a Political Strategist.
Boston, MA: Porter Sargent Publishers, 1979.
Sharp, Gene. The Politics of Nonviolent Action,
Part Two: The Methods of Nonviolent Action.
Boston, MA: Porter Sargent Publishers, 1973.
Sharp, Gene. Waging Nonviolent Struggle,
20th Century Practice and 21st Century Potential.
Manchester, NH: Extending Horizons Books,
2005.
Smithey, Lee A. and Lester R. Kurtz. “Parading
Persuasion: Nonviolent Collective Action as
Discourse in Northern Ireland,” in Patrick G.
Coy (ed.), Consensus Decision Making, North-
ern Ireland and Indigenous Movements. West
Yorkshire, UK: Emerald Group Publishing
Ltd., 2002.
Taylor, Adam. “Intruders Thought Stealing a
Giant Gold Mace Would Disrupt Nigeria’s Par-
liament. It Didn’t Work,” The Washington Post.
April 19, 2018. Retrieved from: https://www.
washingtonpost.com/news/worldviews/
wp/2018/04/19/intruders-thought-stealing-a-
giant-gold-mace-would-disrupt-nigerias-par-
liament-it-didnt-work/?noredirect=on&utm_
term=.afc2152e9b03
Ulmer, Alexandra. “Venezuelans Revel in Pots-
and-Pans Protests after Maduro Humiliation,”
Reuters. September 9, 2016. Retrieved from:
http://www.reuters.com/article/us-venezue-
la-politics-pots-idUSKCN11F22M
Vinthagen, Stellan. A Theory of Nonviolent
Action: How Civil Resistance Works. London,
UK: Zed Books, 2015.
Wanis-St. John, Anthony and Noah Rose. Nego-
tiating Civil Resistance. Washington, DC: United
States Institute for Peace, 2017. Retrieved from:
https://www.usip.org/sites/default/files/2017-
07/pw129-negotiating-civil-resistance.pdf
Wilson, Elizabeth W. People Power and
International Human Rights Law: Creating a
Legal Framework. Washington, DC: ICNC Press,
2017. Retrieved from: https://www.nonvio-
lent-conflict.org/resource/people-power-move-
ments-international-human-rights-creat-
ing-legal-framework/
Zunes, Stephen. Civil Resistance Against
Coups: A Comparative and Historical Perspec-
tive. Washington, DC: ICNC Press, 2017.
Retrieved from: https://www.nonviolent-con-
flict.org/resource/civil-resistance-coups-com-
parative-historical-perspective/
82
APPENDIX
Universe of Nonviolent Tactics
The following Appendix presents 346 tactics of nonviolent resistance, which comprises
Sharp’s 198 methods from 1973, as well as the new nonviolent tactics included in the
Nonviolent Tactics Database.49 The Appendix also presents the refined categorization of all
old and new nonviolent methods. There are many categorization systems to organize meth-
ods of nonviolent resistance as outlined in this monograph (see Chapter 3 and 4). The tactics
are organized in this Appendix through the categorization hierarchy presented in this mono-
graph as illustrated in the chart below.
The tactic number on the far right of each method refers to the unique identification
number assigned to the tactic in the Nonviolent Tactics Database.
The red numbers in parentheticals are the original tactic numbers that Sharp assigned
for each of his 198 methods in 1973.
Expression/Protest and Appeal
(HOW ONE S AYS S OME TH ING)
Expressive Tactics Using Medium of the Human Person
MOVEMEN TS & GESTURES
Dance: Performing or participating in a form of
dance to demonstrate discontent/resilience/
agreement
 1
Rude gestures (30): Using hand or arm ges-
tures that are considered rude or anti-social in
the social context
 2
Martial arts: Practicing martial arts as a means
of protest or appeal
 3
Human banner: Using human bodies to make
a picture or spell a word, usually visible from
above
 65
Hand gesture: Hand gestures (by an individual
or a group) that are used to convey opposi-
tion
 307
Human chain: Many people join hands to form
a giant line as a demonstration of political sol-
idarity
 308
Kneeling: Kneeling in places or during a time
when it is not socially appropriate to do so		
309
Body percussion: Creating man-made sounds
meant to display approval, disapproval, or uni-
fication (e.g., applause)
 310
49
 For more information, see: https://www.tactics.nonviolenceinternational.net/.
83
PROCESSION S
Marches (38): A group of people walking
 Picketing (16): Congregating outside a place
together to reach a particular point as a means
 to protest and deter entry—hence, crossing the
of protest or appeal
 4
 picket line
 10
Parades (39): Similar to a march, but the point
 Walk and trek: A long walk or journey meant
of destination is not politically significant; it is
 to increase public awareness and/or demon-
only a convenient place of termination
 5
 strate support for or opposition against a spe-
cific cause/group/law/etc.; the route is import-
Religious processions (40): A march or a
 ant, as a more publicly visible passage is more
parade with a religious character
 6
 likely to raise awareness
 11
Pilgrimages (41): A walk (by an individual or a
group) that has a significant moral and/or reli-
gious aspect
 7
PUB LIC ASSEMBDeputations (13): A representative delegation
meeting with an opposing party in order to
present grievances or to propose new policy		
9
“Live crowd ‘choreography’ through crowd-
sourced data”: Using apps to coordinate the
movement of the crowd
 12
Check-ins: Using social networking sites to
‘check in’ at a protest and show your support
13
Assemblies of protest or support (47): A gath-
ering of protest/dissent that occurs at places
important to the cause of the demonstration
(e.g., courthouse steps)
 14
Protest meetings (48): Similar to an assembly
of protest or support, but occurring wherever
is convenient
 15
LIES
Camouflaged meetings of protest (49):
Protest meetings that are presented as some-
thing else (e.g., sporting event, religious cere-
mony, etc.), often in order to avoid legal reper-
cussions
 16
Group lobbying (15): A collective group of
constituents presents their argument for a
specific issue to their designated parliamentary
representative
 17
Vigil (34): A period of prayer or keeping watch
to honor some person or event
 18
Coordinated worldwide demonstrations:
Mass international demonstrations that occur
simultaneously in order to draw extensive
attention to a particular issue
 19
RITUALS & TRADITION S
Growing/shaving hair as protest: Common
 Wearing traditional or historical clothing or
expression of mourning or criticism of an action,
 costumes: Appeals to tradition or history 21
policy, etc.
 20
84
National anthem protests: Expressing some
kind of dissent during the performance of a
national anthem
 22
Invocation of magic: The practice of magic
used to condemn the actions of an individual
or group
 23
Prayer and worship (20): Demonstrating moral
disapproval or political protest through reli-
gious acts
 24
HON ORIN G TH E DEAD
Political mourning (43): Utilizing symbols of
 Demonstrative funerals (45): Transformation
mourning such as grief and loss to demon-
 of an actual funeral into a protest, which is
strate dissent with a certain policy, event, or
 especially effective when honoring a victim
action
 25
 who died at the hands of opponents
 27
Mock funerals (44): Organizing a fake funeral
 Homage at burial places (46): Demonstration
that grieves opponents’ violation of some vir-
 at the burial site of a person who was symbolic
tue or belief
 26
 and/or influential to the cause behind the
protest
 28
PERFORMANMock awards (14): Creation of fictional awards
for the sake of making a statement on an
issue
 29
Mock tribunals: Unofficial public tribunals for
victims or protesters to share feelings and
thoughts about certain polices/actions
 31
Flash mobs/smart mobs: Group gathers in a
public place and performs a seemingly out of
place action (i.e., choreographed dance) for a
brief period, then disperses
 32
Humorous skits and pranks (35): Protest
through the use of comedy or satire
 33
CE
Traditional theater (36): Using theater
sketches, plays, and skits to increase aware-
ness, spread information, and/or protest or
appeal
 34
Destruction of own property (23): Willingly
destroying one’s own property as an act of
protest/appeal
 35
One-person protest (with or without aggre-
gation): A single activist doing some action of
protest/appeal that can often provide observ-
ers with the opportunity to join
 36
Wearing/displaying a single color: Wearing a
single color on a pre-determined day to express
support for a movement and/or discontent with
an individual or government
 311
85
W IT HDRAWAL & REN UN CIATION
Walk-outs (51): An act of protest/appeal in
 Renouncing honors (53): Giving up an honor/
which participants walk out during a meeting,
 award as an act of protest or appeal
 42
assembly, etc.
 40
Turning one’s back (54): Turning away when
Silence (52): Maintaining silence in response
 an opponent speaks or performs; can be sin-
to a speaker or event; often used to convey
 gular or en masse
 43
moral condemnation
 41
P RE S SURE ON IN DIVIDUALS
“Haunting” officials (31): Activists follow offi-
 Fraternization (33): Socializing with police or
cials and remind them of their presence and
 soldiers in an attempt to mitigate their actions
dedication to the cause
 44
 and/or persuade them to join the cause 46
Taunting officials (32): Straightforward mock-
ery of an official in a public space
 45
Expressive Tactics using Medium of Things
SOUN D & MUSIC
Symbolic sounds (28): Use of sounds without
 Whistles: The use of whistles or the technique
language to set a tone or send a message 49
 of whistling to create noise
 52
Cacerolazo: Banging on pots and pans as part
 Drumming: Creating noise/rhythms that
of a march or similar event
 50
 amplify chanting/singing/shouting
 53
Car horns: Repeated use of car horns in an
 Musical instruments (36): Delivering protest
organized fashion to raise awareness/voice
 or appeal through instruments producing
discontent
 51
 sound that is presented to the public
 93
2- D IME N SION AL ARTS & MATERIALS
Banners, posters, and other displayed com-
munications (8): Portable written, painted, or
printed communication that is typically dis-
played in public and used to send or amplify a
message
 54
Paint as protest (26): Graffiti, painting over
signs, etc.
 55
Displays of flags and symbolic colors (18):
Expression of political dissent by displaying/
wearing a country’s flag or symbolic colors, a
group’s symbolic flags or colors, etc.
 56
Wearing of symbols (19): Expression of politi-
cal dissent by wearing symbolic clothing, col-
ors, items, etc.
 57
86
Symbolic lights (24): Using lights as expres-
sion (for example, turning off lights, torches,
candles, etc. en masse)
 58
Displays of portraits (25): Displaying portraits
in a public space, often paired with a vigil or
similar event
 59
Comics: Comics that criticize/mock/draw atten-
tion to individuals, groups, companies, or par-
ticular issues
 61
3 - D IME NSION AL ARTSMotorcades (42): Similar to a march or parade
but involves participants driving cars at a slow
speed rather than walking
 8
Food waste (or other farm goods) as protest:
Using food waste as a form of protest/
appeal
 38
Mailing symbolic items (21): Shipping an item
that has a particular symbolic meaning as an
act of protest
 47
Delivering symbolic objects (21): Dropping off
significant objects to an opponent
 66
Puppets: Using puppets (and puppet shows)
to exemplify a message, make fun of oppo-
nents, or critique a particular group/individual/
issue
 67
Stickers: Displaying stickers that show sup-
port/opposition
 62
Logos: Creating or repurposing logos to con-
vey a message/prove a point
 63
Buttons: Displaying buttons that show support/
opposition
 224
Makeup/face painting: Using makeup or face
paint to demonstrate or display a message, or
to make a point
 312
& MATERIALS
Props: The use of props to add creativity and/
or to emphasize the reason or cause behind a
protest
 68
Costumes: Dressing up in a costume relevant
to the issue in order to draw attention
 69
Mascots: Dressing up as a mascot to draw
attention to an issue
 70
Sculpture: Producing sculptures to emphasize
an issue, to memorialize and/or honor a person
relevant to the cause, to demonstrate a point,
and/or to call out an opponent
 71
Vehicles, with 2D and 3D art: Turning vehicles
(trucks, cars, boats, hot air balloons, planes,
etc.) into mobile message-delivering/informa-
tion-spreading mechanisms
 72
Expressive Tactics Using Medium of Electronic Communication
RE CO RD ING & DISTRIB UTIN G N EWS OF N VA
Livestreaming: The live public broadcasting of
an event, incident, or protest
 73
Short form digital video: A brief video detailing
the issue that people are protesting for/
against
 74
Social media photo campaign: Promoting a
particular image through social media platforms
(for example, changing profile pictures)
 75
Database leaks: Releasing entire digital
archives of secret/classified materials in order
to educate the public and/or increase aware-
ness
 313
87
CROW DSOURCIN G IN FORMATION
Sousveillance: Covert surveillance by citizens,
 Digital file sharing applications: Peer-to-peer
frequently of authorities
 76
 file-sharing (uTorrent, etc.)
 78
Maps and maptivism: Using maps, typically digital
ones, to crowdsource data or information
 77
CRE AT ING ON LIN E DIGITAL CON TEN T
Blogging/writing/commenting/tweeting: Cre-
 Digital video and audio art: Using media forms
ating online written content that addresses
 such as videos, photos, photos of art, digital
particular issues, which is especially useful
 art, animations, and silent videos to protest/
if it is too dangerous to speak out or protest
 appeal
 81
in person
 80
Digital games: Digital games that are used to
criticize opponents and their ideas or to model
a new behavior or institution
 314
FAL SE , I MAGIN ARY IN FORMATION
Creating faux identities, websites, videos:
 Mock documents (government forms): Con-
Creating some kind of hoax or fake information
 structing mock documents or forms for use by
that is intended to mock opponents and/or
 the public
 84
shock the public
 82
Deliberately fake money: Creating false cur-
Mockumentaries: A documentary that uses humor
 rency that can be used to combat corruption,
and parody to mock an opponent or issue 83
 spread awareness about the issue, etc.
 85
MASS ACTION
SMS/email/social media bombing: Using text
 “Nonviolently ‘hijacking’ social media”: Hack-
messaging, email, or social media functions to
 ing, posting on, exposing, and/or disabling the
send messages en masse to a target
 86
 social media accounts of an opponent
 90
Forwarding information, retweeting, re-post-
 Social media “challenges”: Using social media
ing, sharing: Sharing information and raising
 to call others to action on a mass scale
 91
awareness through social media or other mes-
saging systems
 87
 Solidarity telethon: Mass calls to spread infor-
mation and solidarity
 315
Trend a hashtag: Using a social media plat-
form’s hashtag feature to call attention to an
 Product review hijacking: Negatively or posi-
issue or event (#)
 88
 tively mass-reviewing a product
 316
Influencing Internet search engines: Chang-
ing the results of a search engine for a specific
term/person
 89
88
Expressive Tactics using MediumSinging (37): Using song as a way to express
a particular viewpoint or to drown out oppo-
nents’ speech
 48
Changing narratives/flipping scripts/editing
song lyrics, etc.: Altering an existing narrative,
song, script, etc. to exemplify how the original
was incorrect, offensive, misleading, etc. 64
Teach-ins (50): Educating attendees on a topic
by allowing speakers representing various
viewpoints to talk without strict limits on time
or scope and encouraging audience engage-
ment; popularized during the Vietnam War 92
Skywriting and earthwriting (12): Writing mes-
sages in the sky or on the earth so they are
visible from afar
 94
Slogans, caricatures, and symbols (7):
Simple messages or symbols that are written,
painted, printed, mimed, gestured, drawn, spo-
ken, etc.
 95
Letters of opposition or support (2): Letters
signed by individuals and/or organizations
taking a stance on an issue
 96
Declarations by organizations and institu-
tions (3): Statements that not only express a
viewpoint, but place blame on a wrongdoer
and/or declare an intent to take action
 97
Signed public statements (4): Statement for
the general public signed by individuals and/
or organizations
 98
Declarations of indictment and intention (5):
Statements that declare an intention rather
than merely statements of expression
 99
Group or mass petitions (6): Written requests
with a specific purpose that have a large num-
ber of signatories
 100
Public written advertisements: Advertise-
ments that provide a service, send a message,
or highlight a particular issue
 101
of Human Language
Distributing dissenting leaflets and pam-
phlets (9): Distributing written material to
express dissent or opposition
 102
Newspapers and journals (10): Expressing
dissent or opposition through articles in news-
papers, journals, and other publications 103
Public speeches (1): Prepared or spontaneous
addresses delivered as an act of opposition,
protest, support, etc.
 104
Call and response: Chants or slogans punctu-
ated by responses from the listeners
 105
Chanting: Group shouting with rhythmic,
repetitive words and sounds, with or without a
melody
 106
Call-in/phone march: Organizing mass phone
calls to an official or authority
 108
Poetry/spoken word: Writing and performing
poetry as a way to fight back against violence
and the silencing of voices
 109
People’s mic: Using a large group of people to
amplify a message by shouting out the same mes-
sage in waves (similar to call and response) 110
Coded language: Creating a linguistic code to
circumvent censorship
 111
Rude/transgressive language: Swearing or
using offensive language to demonstrate a
point or prevent opponents from talking 112
Counter cat-calling: Responding to cat-calling
in a loud and expressive fashion to draw atten-
tion to public sexual harassment
 317
Self disclosure: Speaking out against egregious
crimes by revealing personal grievances 319
Publishing dissenting literature (9): Writing
educational or training material that relates to
the strategies of resistance
 346
89
Acts of Omission
Noncooperation: Coercive Acts
( HOW ONE DOE S N’T DO S OME TH ING)
Social Noncooperation
O STRACISM OF PERSON S
Social boycott (55): Refusal to engage with a
particular person or group of persons
 113
Selective social boycott (56): Refusal to par-
take in a specific type of social behavior with a
particular person or group of persons
 114
Lysistratic nonaction (57): Refusal of sex until
demands are met
 115
Excommunication (58): Excluding someone
from religious services, ceremonies, and prac-
tices
 116
Interdict (59): Suspension of religious services
in a certain region until some form of political
or social change comes into effect
 117
N O NCO O P E RAT IO N W IT H SOCIAL EVEN TS, CUSTOMS & IN STITUTION S
Suspension of social and sports activities
 Withdrawal from social institutions (64):
(60): Refusal to arrange or participate in social
 Resigning from or refusing to participate in
activities
 118
 certain institutions
 121
Boycott of social affairs (61): Refusal to attend
 Refusal of pledges or oaths: Withholding
certain types of social events
 119
 promises or avowals
 123
Student strike (62): Refusal to attend class		
120
W IT HD RAWAL FROM TH E SOCIAL SYSTEM
Stay-at-home (65): Organized act of staying at
 Sanctuary (68): Fleeing to a place where one
home; usually, but not always, during working
 cannot be removed due to religious, legal,
hours
 124
 moral, or social restrictions
 127
Total personal noncooperation (66): A prison-
 Collective disappearance (69): Temporary
er’s complete refusal to take any action, includ-
 abandonment of a place by a population 128
ing eating, drinking, or even moving one’s
body
 125
 Protest emigration (hijrat) (70): Voluntary exile
from a place
 129
“Flight” of workers (67): Workers stop working
and leave home, but do not yet impose
 Ghost town: Large portions of the population
demands
 126
 stay home instead of going to work or school		
320
90
Economic Noncooperation: Boycotts
ACTION BY CON SUMERS
Consumers’ boycott (71): Consumer refusal to
 Refusal to rent (75): Refusal to rent a building
purchase certain goods or a certain class of
 or land in order to protest against the owner or
goods
 130
 landlord
 134
Nonconsumption of boycotted goods (72):
 National consumers’ boycott (76): Boycott
Refusal to use or consume boycotted goods
 that extends to an entire nation and the prod-
that one has already purchased
 131
 ucts produced there
 135
Policy of austerity (73): Giving up personal
 International consumers’ boycott (77): Boy-
luxuries either for symbolic reasons or as a
 cott of a nation’s goods/services that is enforced
boycott
 132
 by multiple nations
 136
Rent withholding (74): Refusal to pay rent to a
 Coin hoarding: An effort to not spend a partic-
landlord or property owner
 133
 ular form of currency in hopes that it will take
large amounts of cash out of circulation 321
ACT IO N BY WORKERS & PRODUCERS
Producers’ boycott (79): Producers refuse to
 Workmen’s boycott (78): Workers refuse to
create, sell, or deliver their products
 137
 use supplies or tools that were produced under
certain circumstances
 322
ACTION BY MIDDLEMEN
Suppliers’ and handlers’ boycott (80): ‘Mid-
dlemen’ refuse to accommodate the shipping
and processing of a product
 138
ACT IO N BY OWN ERS & MAN AGEMEN T
Traders’ boycott (81): Retail outlets refuse to
buy, sell, or deliver a particular product 139
Refusal to let or sell property (82): Refusal of
a property owner to rent or sell property to
specific persons or groups
 140
Lockout (83): Employer temporarily closes
the workplace in order to put pressure on
employees
 141
Refusal of industrial assistance (84): Organi-
zation refuses to produce economic/technical
services to opponent
 142
Merchants’ “general strike” (85): Merchants
or retailers close down their businesses to slow
or stop the economic flow
 143
91
ACT IO N BY HO LDERS OF FIN AN CIAL RESOURCES
Withdrawal of bank deposits (86): Terminating
 Severance of funds and credit (89): Cutting
financial cooperation
 144
 off sources of money for opponents
 147
Refusal to pay fees, dues, and assessments
 Revenue refusal (90): Usually a refusal to pay
(87): Refusing to pay a monetary sum to a cer-
 proceeds to the government (i.e., taxes or
tain establishment, government, etc. as an act
 licenses)
 148
of protest or appeal
 145
Divestment: Institutional separation from cor-
Refusal to pay debts or interest (88): Not sub-
 porations whose actions investors find objec-
mitting what is owed
 146
 tionable
 323
ACT I ON BY GOVERN MEN TS
Domestic embargo (92): Government that
 International buyers’ embargo (95): Govern-
boycotts a particular institution or opponent
 ment disallows buying a product from a
within its borders
 149
 country
 152
Blacklisting of traders (93): Prohibiting trade
with particular firms or individuals
 150
International sellers’ embargo (94): Govern-
ment disallows selling a product to a country		
151
International trade embargo (96): Combina-
tion of international sellers’ embargo and inter-
national buyers’ embargo
 153
Economic Noncooperation: Strikes
SYMB OLIC STRIKES
Protest strike (97): Strike performed for a set
 Quickie walkout (lightning strike) (98): Short,
period of time with the purpose of making a
 unplanned strikes often undertaken when
statement rather than achieving a specific
 more formal strikes are illegal or impractical		
goal
 154
 155
AGRICULTURAL STRIKES
Peasant strike (99): Refusal of peasants to
 Farm workers’ strike (100): Strike by hired
cooperate with landlords
 156
 agricultural workers
 157
92
ST RIKES BY SPECIAL GROUPS
Refusal of impressed labor (101): Refusal of
 Professional strike (104): Strike undertaken by
slaves and coerced laborers to work
 158
 members of a specific profession
 161
Prisoners’ strike (102): Refusal by prisoners to
 Statewide strikes: A statewide strike that
participate in labor
 159
 incorporates refusing to go to school, attend
work, open shops, etc.
 324
Craft strike (103): Strike by workers of the
same craft; usually associated with a craft
union
 160
O RD INARY IN DUSTRIAL STRIKES
Establishment strike (105): All of a company’s
employees participate in the strike
 162
Industry strike (106): All the workers in a local-
ity and given industry participate in the strike		
163
Sympathetic strike (107): Unions that are unin-
volved in an issue also strike in solidarity with
others
 164
RESTRICTED STRIKES
Detailed strike (108): Organized strike in which
an increasing number of laborers stop working
over a set time period
 165
Bumper strike (109): Unions strike one firm at
a time
 166
Slowdown strike (110): Working slowly and
inefficiently to undermine the employer 167
Working-to-rule strike (111): Meticulously fol-
lowing all rules and regulations of a job without
putting in any extra effort in order to slow down
efficiency
 168
“Reporting ‘sick’ (sick-in)” (112): Large number
of workers falsely claim to be ill
 169
Strike by resignation (113): En masse resigna-
tion by personnel
 170
Limited strike (114): Workers refuse certain
amounts of work—overtime or certain days 171
Selective strike (115): Workers refuse to per-
form certain tasks, usually due to moral or
political qualms
 172
MULTI-IN DUSTRY STRIKES
Generalized strike (116): Strike carried out in
 General strike (117): Strike carried out by work-
several industries at once, but not by a majority
 ers of various professions and industries in a
of workers
 173
 region in order to bring it to an economic
standstill
 174
93
CO MBINAT IO N OF STRIKES & ECON OMIC CLOSURES
Hartal (118): Voluntarily abstaining from eco-
 Economic shutdown (119): General strike and
nomic activity for a set amount of time as a
 merchants’ general strike
 176
symbolic gesture
 175
Political Noncooperation
RE JECTION OF AUTH ORITY
Withholding or withdrawal of allegiance
 Literature and speeches advocating resis-
(120): Refusal to recognize the validity of an
 tance (122): Text that denounces an authority
authority or follow its orders
 177
 and promotes defiance
 179
Refusal of public support (121): Public with-
holds support requested or demanded by a
leader or public figure
 178
CIT IZE NS’ NO NCOOPERATIONBoycott of legislative bodies (123): Refusal to
serve in legislative branches of government		
180
Boycott of elections (124): Refusal to partici-
pate in an election—frequently by an opposi-
tion faction
 181
Boycott of government employment and
positions (125): Refusal to fulfill the obligations
of a government post
 182
Boycott of government depts., agencies, and
other bodies (126): Refusal to interact with or
cooperate with a government agency or similar
organization
 183
Withdrawal from government educational
institutions (127): Mass removal of one’s chil-
dren from public or government-run schools		
184
WITH GOVERN MEN T
Boycott of government-supported organiza-
tions (128): Refusal to interact with or cooper-
ate with organizations which collaborate with
the government
 185
Refusal of assistance to enforcement agents
(129): Refusal to supply police, military, or sim-
ilar enforcement mechanisms with information
or aid
 186
Removal of own signs and placemarks (130):
Removal or alteration of signs that assist with
direction in order to misdirect or impede
authorities
 187
Refusal to accept appointed officials (131):
Refusal to recognize authority of an official and
responding with noncooperation when they
attempt to fulfill their duties
 188
Refusal to dissolve existing institutions (132):
Refusal to accept government orders to dis-
solve an organization or institution
 189
94
CIT IZE NS’ ALTERN ATIVESReluctant and slow compliance (133): Obedi-
ence that is carried out reluctantly or slowly in
order to impede the process
 190
Nonobedience in absence of direct supervi-
sion (134): Disobeying when no authority or
command figure is present
 191
Popular nonobedience (135): Widespread
refusal to follow laws and regulations in a way
that is not open or confrontational
 192
Disguised disobedience (136): Action that
appears to be compliant, but is in fact subver-
sive in some way
 193
TO OB EDIEN CE
Refusal of an assemblage or meeting to
disperse (137): Continuing to gather after being
told by an authority to leave
 194
Sit-down (138): Sitting down and refusing to
leave a space
 195
Noncooperation with conscription and depor-
tation (139): Form of conscientious objection—
refusal to register or appear for conscription or
to accede to deportation
 196
Hiding, escape, and false identities (140):
Fleeing or hiding one’s identity to escape
persecution or to protest a government
policy
 197
ACT IO N BY GOVERNLegislative obstruction: Members of the law-
making branch of government refusing to par-
ticipate in a legislative process as an act of
protest (i.e., delaying a quorum)
 122
Selective refusal of assistance by govern-
ment aides (142): Refusal by government
employees to carry out certain orders
 198
Blocking of lines of command and informa-
tion (143): Deliberate interference in the chain
of command and communication
 199
Stalling and obstruction (144): Subtly working
inefficiently to slow government action 200
MEN T PERSON N EL
General administrative noncooperation (145):
Refusal to cooperate with a regime, usually
one that has just come into power by illegiti-
mate means, by the majority of government
employees
 201
Judicial noncooperation (146): Judges, jurors,
and other judicial agents refuse to carry out
their orders
 202
Deliberate inefficiency and selective nonco-
operation by enforcement agents (147): Lim-
ited disobedience by police or similar enforce-
ment agents
 203
Mutiny (148): Open and outright disobedience
by enforcement agents
 204
Inter-agency noncooperation: Stopping
support for other departments/ministries/
agencies
 205
95
D O ME ST IC GOVERN MEN TAL ACTION
Quasi-legal evasions and delays (149): Enact-
 Noncooperation by constituent governmen-
ing laws that obstruct orders from a higher
 tal units (150): Local governmental entities
level of government, but does not directly
 defying central authorities
 207
contradict them
 206
INT E RNAT ION AL GOVERN MEN TAL ACTION
Changes in diplomatic and other representa-
tions (151): Altering diplomatic staff/structures
to show disapproval of a foreign government’s
policies or actions
 208
Delay and cancellation of diplomatic events
(152): Cancelling planned events to show dis-
approval of a foreign government
 209
Withholding of diplomatic recognition (153):
Refusing to recognize the legitimacy of a coun-
try’s government
 210
Severance of diplomatic relations (154): Ter-
minating the relationship with another nation		
211
Withdrawal from international organizations
(155): Leaving an international group
 212
Refusal of membership to international bod-
ies (156): Refusing an invitation to join an inter-
national group
 213
Expulsion from international organizations
(157): Expelling a government from an inter-
state organization
 214
Refraining: Positive Acts
( HOW ONE DOE S N’T DO S OME TH ING)
Suspension: Protesters temporarily cease an
action, which is typically done in order to
reward the opponent
 215
Halting: Protesters terminate an action, which
is typically done to reward the opponent 216
Active abstention from a planned action:
Not going through with a threatened action,
typically done to reward opponents for an
action
 217
96
Acts of Commission
(HOW ONE DOE S S OME TH ING)
Disruptive Commission
P SYCHOLOGICALProtest disrobings (22): Removal of clothing
in public as an expression of religious and/or
political dissent
 30
Protest sex: Engaging in public sexual activity
as protest
 37
Self-exposure to the elements (158): Exposing
oneself to harsh environmental conditions
(cold, heat, etc.) in order to pressure opponents
into complying with requests
 218
Fast of moral pressure (159a): Fasting to exert
moral burdens on opponent
 219
Hunger strike (159b): Fasting with intent of
coercion
 220
IN TERVEN TION
Satyagrahic fast (159c): Fast preceded by
spiritual preparation with the intent to convert
the opponent
 221
Nonviolent harassment (161): Heightening
pressures and agitations against perceived
wrong doers
 222
Self-mutilation (or body art) : Harming one’s
body as an act of protest/appeal; a highly con-
troversial method of nonviolent action
 223
Distributed Denial of Service (DDoS): Over-
loading a website by sending too many requests
for access
 225
Relay hunger strike: Strike in which partici-
pants take turns fasting in order to elongate
the duration of the fast
 300
P HYSICAL INGluing: Preventing public transport from run-
ning or blocking entry to buildings by super-glu-
ing oneself to objects
 79
Sit-in (162): Sitting in at a space in order to
obstruct further access to the space, which
typically will disturb routine activity
 226
Nonviolent communication jamming: Hin-
dering opponents’ ability to use technological
communication devices
 227
Tripods: A large three-legged blockade device
that resembles a camera tripod
 228
TERVEN TION
Die-in: Pretending to die in a public space en
masse; often used to signify the consequences
of an opponents’ destructive policies, beliefs,
or actions
 229
Book blocs: Participants of a human blockade
use book-shaped constructed items as a type
of shield and/or to spread a message
 230
Boat blockade: Using boats to block other
boats/entrances
 231
Mill-in (166): Campaigners gather but remain
mobile as they do so
 232
Tree-sits: Barricading oneself to a public
tree
 233
97
Throwing food (pie-in): Using food as a nonvi-
olent weapon
 234
Destruction of public property: Destruction of
public property; a highly controversial method
of nonviolent action
 235
Nonviolent air raids (169): Using nonviolent air
transportation devices to drop significant items
or pamphlets
 236
Blocking demonstration: Using bodies and
other items to impede the view of another
demonstration
 237
Nonviolent interjection (171): Physical interces-
sion of someone’s path, blocking their ability to
execute their task in an attempt to make them
reconsider (i.e., bulldozers or soldiers)
 238
Nonviolent obstruction (172): Physical block-
ing of a space, creating a big enough obstacle
to actually prevent passage
 239
Inflatables: Use of inflatable items to obstruct,
impede, and/or add an interactive element to
a protest
 240
Internal accompaniment: A person, potentially
privileged, going with someone in a place that
they might be at risk of harm
 259
Chaining self/other: Using handcuffs, chains,
bike locks, etc. to fasten an individual to an
object/person that is significant to the pro-
test
 325
Destruction of government documents: Pur-
posefully destroying government issued doc-
uments as a form of protest
 327
Destruction of direct government objects:
The destruction or damaging of government
property to prevent direct violence
 328
Covering public property/art: The act of eras-
ing or covering opposing messages or symbols
to prevent the public from continuing to see/
experience them
 329
Motor vehicle blockade: Using trucks, tractors,
taxis, and cars to block traffic
 330
Rebel clowning: Incorporating the practice of
clowning into forms of organizing and protest/
political disobedience
 331
Suspending from bridges: Hanging from the
side of a bridge, usually using ropes
 332
Protest camp: Physical camps that are set up
to delay, obstruct or prevent the focus of an
opponent’s protest by physically blocking it
with the camp.
 333
Solidarity accompaniment: A third-party’s phys-
ical presence reduces the chance of violence
toward a vulnerable community/individual 334
Preventing flights or public transportation
from leaving: Preventing planes, or other forms
of transportation from departing
 335
98
SOCIAL IN TERVENOverloading of facilities (175): Deliberately
overloading the capacity of a building or pro-
cess so that operations efficiency is slowed
down or stopped
 241
Stall-in (176): Customers/clients taking longer
than normal to do something in order to slow
down efficiency
 242
Speak-in (177): Campaigners interrupt gather-
ing or event to raise an issue and speak for a
prolonged period of time
 243
Public filibuster: Similar to a senatorial filibus-
ter—an elongated speech intended to delay
some aspect of a legislative process—but
within the public realm
 244
Infiltration: Covertly sneaking into the oppo-
nent’s gathering/event
 245
Guerrilla theater (178): Disruptive form of the-
ater in which activists put on a surprise public
performance that is designed to shock the
audience
 246
Electoral guerrilla theater: Form of guerrilla
theater in which someone runs for office as
satirical social commentary
 247
Forum theater: Theater that addresses an issue
but invites the audience at various points to
participate and modify the performance 248
Image theater: Like forum theater but partici-
pants stand still as if they are an image cap-
tured in reality
 249
Détournment/subvertising: Modifying a wide-
ly-known piece of art and giving it a new mean-
ing that aligns with the activist’s message 250
Spreading rumors: Spreading unverified infor-
mation in order to elicit a public response 251
TION
Hoax: Writing and spreading a false news
story that is intended to be believed by the
public
 252
Identity correction: Theatrically and publicly
exposing an opponent’s true, typically mali-
cious, intentions/beliefs
 253
Invisible theater: Theater that does not seek
to be recognized as such; public performance
presented as reality
 254
Legislative theater: Form of political theater
that asks the audience for suggestions of
potential policy or legislative solutions to the
problem presented in the performance 255
Media-jacking: Subverting your opponent’s
media event/platform to benefit your own
cause
 256
Creative disruption: Disrupting an event or talk
in a unique way
 257
Alternative communication system (180):
Using media to act outside of mainstream or
monopolized communication (i.e., under-
ground radio)
 258
Protective presence: A group placing itself in
a position to deter the risk of harm to some-
one
 260
Third party witness: Acting as a non-involved
third party by observing and potentially report-
ing on certain events
 261
Shouting down: Making it difficult for someone
to speak by using a consistently higher vol-
ume
 336
99
E CON OMIC INReverse strike (181): Protesters or workers
work harder than usual or continue to work
when they are not supposed to
 263
Stay-in strike (182): Workers stop working and
refuse to leave a workspace until employers
agree with requests
 264
Politically motivated counterfeiting (185):
Distributing counterfeit money to upset the
economy
 265
Preclusive purchasing (186): Deliberately buy-
ing certain commodities to prevent the oppo-
nent from having access to them
 266
TERVEN TION
Seizure of assets (187): Impeding/confiscating
the opponent’s access to bank accounts, inter-
est payments, copyrights/patents, or similar
assets
 267
Dumping (188): Selling a commodity at low-
er-than-market price to pressure a rival or
opponent
 268
Business whistleblowing: Individuals publicly
expose illegal and/or immoral practices of
banks, businesses, corporations, etc.
 337
Land auction disruption: Preventing the sale
of land from taking place
 338
P O LITICAL INWhistleblowing — social, economic, political
(outing): Leaking or alerting the public to infor-
mation that is incriminating or elicits negative
publicity
 269
Overloading of administrative systems (193):
Providing an excessive amount of information
(which may or may not be relevant) to admin-
istrative systems in order to reduce progress
and efficiency; can now be done digitally (for
example, DDOS attacks)
 270
Disclosing identities of secret agents (194):
Publicly exposing the identity of a secret agent,
which makes it difficult for them to continue
their mission
 271
Seeking imprisonment (195): Willful attempt
to be arrested and jailed
 272
TERVEN TION
Civil disobedience of “neutral” laws (196):
Deliberate, open, and peaceful violation of
“morally neutral” laws, decrees and regula-
tions; acceptance of punishment
 273
Work-on without collaboration (197): Protest-
ers choosing to continue behaviors from a
previous administration/hierarchy
 274
Guerrilla legal work: Secretly building a legal
case, typically through covert means
 275
Parliamentary/legislature/council disruption:
Stopping or slowing down legislative proceed-
ings by extra-legal or extra-regulatory means;
disruption often happens through shouting,
singing, chanting
 339
Absurd political candidates/parties: Creating
outlandish or satiric political opponents to dis-
tract the main candidates or pull them towards
a more centrist viewpoint
 340
100
Creative Intervention/Prefigurative Actions
(HOW ONE M A KE S OR CR E ATE S S OME TH ING)
P O L ITICAL & LEGALMock elections (17): A purely symbolic elec-
tion, either public or private, designed to edu-
cate or protest
 276
Social disobedience (63): Defiance of social
norms, which is modeled after civil disobedi-
ence but without the legal violation
 277
Dual sovereignty and parallel government
(198): Creating a new government with its
own political institutions and organizational
structures
 278
ECON OMICBlack markets: Secret trading areas, often
selling illegal goods
 107
Refusal of a government’s money (91): Refus-
ing a particular currency
 281
Alternative markets (190): Creating alterna-
tive, illegal channels for buying and selling
goods and services
 282
Alternative economic institutions (192): Cre-
ating new economic institutions (i.e., consum-
ers’ or producers’ cooperatives) that are used
to exert power/influence
 283
Copyleft/distributing copyrighted materials:
Ignoring copyright laws and posting or distrib-
uting protected texts/materials
 284
Nonviolent land seizure (183): Nonviolent
occupations in which the campaigners expect
that the ownership of the land or facility will
shift to them when they win the struggle 285
ACTION S
Civil disobedience of “illegitimate” laws (141):
Open violation of unjust laws
 279
Reverse trial (160): Courtroom trials in which
the accused—through words and body lan-
guage—assumes the role of prosecutor, put-
ting on trial the law or policy in question 280
Citizen inspection: Illegal searches and sei-
zures performed by private citizens
 341
ACTION S
Selective patronage “buycotts” (189): Activists
promote and reward businesses who comply
with the core principles of the campaign 286
Conditional favorable loans: Activists provide
loans to businesses, under the agreement that
the business will promote practices and ideas
that comply with the activists’ cause
 303
Reverse pay-per-view: Providing money in
exchange for views (of a documentary, short
video, etc.)
 305
Property expropriation: Protesters seize aban-
doned or bankrupt private property to use for
their cause
 342
Patent alternatives: Violating patent laws for pos-
itive reasons (e.g., life-saving medication) 343
Socially responsible standards: A set of prin-
ciples and benchmarks for corporations devel-
oped by civil society; enforced through buy-
cotts, boycotts, legislation, and shareholder
resolutions
 344
101
PH YSICALTree planting: Planting trees in areas where it
is prohibited to do so or where they would not
otherwise be planted
 60
Liberated zones: Claiming and blocking off a
street or other public domain for performances
or other movements
 262
Nonviolent invasion (170): Entering an off-lim-
its area to protest against government control
of the land in question
 288
Nonviolent occupation (173): Continuing to
remain in an area after a nonviolent invasion		
289
Stand-in (163): Continuing to stand and obstruct
activity in a place where one is being refused
service or entrance
 290
SOCIAL ACTIONSymbolic reclamations (29): Creative use or
claiming of appropriated things or symbols
(cultivating a garden on public land, etc.) 39
Establishing new social patterns (174): The
creation of new social institutions, planned and
unplanned
 287
Pray-in (167): Praying in a public or private
space in an obstructive manner, including in
churches
 293
New signs and names (27): Revising/removing
street signs or installing new ones
 294
Alternative transportation systems (191):
Creating an alternative transportation system
in addition to boycotting the pre-existing
one
 295
ACTION S
Ride-in (164): Disobeying enforced segrega-
tion on public transport; most famously used
by “freedom riders” during the US Civil Rights
Movement
 291
Wade-in (165): Those who are excluded from
a public water space enter it anyways; widely
used to combat segregation
 292
Nonviolent raids (168): Protestors march to an
area and declare rightful ownership
 296
Defiance of blockades (184): Refusal to obey
international blockades and providing food or
other supplies to a blockaded place.
 297
Kiss-in: Using public displays of physical affec-
tion as protest
 299
Critical mass (cycling): Mass protests by cyclists
reclaiming the streets from motor vehicles 345
S
Women becoming religious leaders without
official approval: Women assuming leadership
in the Church without being expressly allowed
to do so
 301
Distributing free VPNs and alternative
apps: Distributing internet tools that allow
people to evade government censors or ISP
monitoring
 302
Living memorial: Memorials that continue to
grow or that honor those who are still living		
318
Marriage inclusion: Marriages performed with-
out state approval (for example, same-sex,
cross-religion, cross-caste, interracial, etc.)		
326
102
P SYCH OLOGICAL ACTION S
Self-imposed transparency: Deliberate orga-
 Awards as encouragement: Giving awards to
nizational transparency about ongoing affairs
 those who tentatively align with the move-
and issues
 298
 ment’s goals to encourage their actions 306
Flowers in guns: Placing a flower into the
mouth of a gun to symbolize the need for
peace over violence
 304
For further updates, check the Nonviolent Tactics Database available at
tactics.nonviolenceinternational.net
For inquiries and suggestions for new tactics/methods or forms of nonviolent action
and civil resistance, please send emails to info@nonviolenceinternational.net
103
Acknowledgements
We are indebted to Gene Sharp for his lifetime of contributions to the field of nonviolent action and his
collection and classification of nonviolent methods upon which we build. He blessed my efforts to expand
his collection of nonviolent tactics and write this monograph.
I want to thank those who inspired me early in my career including Charlie Walker, Bob Helvey, Sulak Sivaraksa,
George Willoughby, Lynne Shivers, George Lakey, and Bill Moyer.
The work of collecting and cataloguing tactics from all over the world has been the collective effort of many
interns and volunteers at Nonviolence International.
That list includes Nicholas Anders, Seth Barry-Hinton, India Zietsman, Lara al Qasem, Matt Barr, Netty Brinckerhoff,
Nick Shedd, Kimmy Baagelaar, Kimberle Maro, Sarah Bausmith, Michael Konen, Nicholas Scrimenti, Sarah Knorr,
Rachel Lewis, Daniel Jarrad, Liam Glen, Ryan Flynn, Shelby Rogers, Emily Hill, Jilian Maulella, Emily Mattioli, Pjotr
Tabachnikoff, Annalisa Bell, Connor Paul, Alyssa Scott, and Tiffany Schwartz.
This monograph has taken three years to formulate and write. Those who have provided their time and advice
include Nadine Bloch, Kurt Schock, Véronique Dudouet, Stephen Zunes and Doug Bond.
I want to thank the editors Maciej Bartkowski, Steve Chase, Amber French, and Julia Constantine and the entire
enterprise at the International Center on Nonviolent Conflict (ICNC) led by Hardy Merriman and Peter Ackerman.
ICNC’s contribution to the field of civil resistance continues to be enormous and I hope it will do so for many
decades to come.
I want to thank Mubarak Awad, Jonathan Kuttab, Betty Sitka, Paul Magno, and David Hart who have served on
the staff or board of Nonviolence International and who have tolerated and supported me for so many years.
Finally, I want to thank my family, especially my loving parents, John and Fran Beer who so profoundly shaped
my interests and values, my siblings Jenny and Matthew, who have provided advice, more of which I wish I had
followed. My spouse Latanja and kids, Kian and Skye, are such blessings in my life and they have been so
patient and understanding. This monograph and the Nonviolent Tactics Database would not exist without them.
About the Author
Michael A. Beer serves as the Director of Nonviolence International, an inno-
vative and respected Washington, DC based nonprofit promoting nonviolent
approaches to international conflicts. Since 1991 he has worked with NVI to
serve marginalized people who seek to use nonviolent tactics often in difficult
and dangerous environments. This includes diaspora activists, multinational
coalitions, global social movements, as well as within countries including:
Myanmar, Tibet, Indonesia, Russia, Thailand, Palestine, Cambodia, East Timor,
Iran, India, Kosovo, Zimbabwe, Sudan, and the United States. Michael Beer has a special expertise
in supporting movements against dictators and in support of global organizing for justice, environ-
ment, and peace. Michael co-parents two teenagers with his patient life partner, Latanja.
104
A Note from the Author
The power of nonviolent social change is immense and is shaping the world every single
day. There is not a minute that goes by where there is not a street protest, hunger strike,
blockade, boycott or work stoppage happening somewhere in the world. I have been
studying and collecting nonviolent tactics, examples, and photos for more than 30 years
and have only scratched the surface of the large range of activities that people are engag-
ing in to change the world.
I invite you to contribute to the ongoing collection and classification of nonviolent tactics
on the website: https://tactics.nonviolenceinternational.net. You can find a form on the
website where I urge you to share your input. We also welcome editors willing to help
manage the website and its content.
If you would like to have me or my Nonviolence International colleagues such as Mubarak
Awad speak to your class or group about the book and the vast creative universe of
nonviolent tactics, please contact me at info@nonviolenceinternational.net.
I hope you enjoy the book and I look forward to being in conversation with you.
Sincerely,
Michael A. Beer
Director, Nonviolence International
The ICNC Monograph Series aims to bridge research and practice. Drawing on
scholarly literature and high quality analytical and empirical analyses, monographs
enrich public discourse by expanding scientific knowledge in the field of civil
resistance and providing recommendations for practitioners such as activists,
organizers, journalists, and members of INGOs and the policy community.
Monographs are available for free download at:
https://www.nonviolent-conflict.org/
Hard copies are also available for purchase.
OTHER VOLUMES IN THE ICNC MONOGRAPH SERIES
The Role of External Support in Nonviolent Campaigns:
Poisoned Chalice or Holy Grail?
by Erica Chenoweth and Maria J. Stephan, 2021
When Civil Resistance Succeeds: Building Democracy
After Popular Nonviolent Uprisings
by Jonathan Pinckney, 2018
Civil Resistance against Coups: A Comparative
and Historical Perspective
by Stephen Zunes, 2017
People Power Movements and International Human Rights:
Creating a Legal Framework
by Elizabeth A. Wilson, 2017
Making or Breaking Nonviolent Discipline in Civil Resistance Movements
by Jonathan Pinckney, 2016
(Chinese translation also available)
The Tibetan Nonviolent Struggle: A Strategic and Historical Analysis
by Tenzin Dorjee, 2015
(Tibetan translation also available)
The Power of Staying Put: Nonviolent Resistance
Against Armed Groups in Colombia
by Juan Masullo, 2015
(Spanish translation also available)
Civil Resistance Tactics in the 21st Century belongs on the virtual bookshelf
of anyone who is studying or practicing nonviolent action.
n Scholars: Explore updated categories and
tactics that respect and expand on Gene
Sharp's landmark work.
n Teachers & Trainers: Give your participants
a brief overview of the whole range of
nonviolent tactics used around the world,
when and how those tactics work, and how
nonviolent tactics differ from, or combine
with, other types of civil resistance.
n Activists: Use this concise guide to expand
your toolbox and sharpen your analytical
tools for selecting powerful strategies for
your campaigns.
This book dovetails with two huge online
sources (Nonviolence International’s
Nonviolent Tactics Database: https://tactics.
nonviolenceinternational.net and Organizing &
Training Archive: http://nonviolence.rutgers.
edu) so that you can move seamlessly between
strategy and implementation.
“Michael Beer has brought us up to date with this impressive
monograph that includes a revised, expanded, and recategorized
list of civil resistance methods which is a must-read for both
scholars and activists.” — S TE P HE N ZUN E S
“The rich array of approaches and cultural practices covered in
this study is fascinating. It usefully spells out the ‘frontiers’ between
nonviolent action and overlapping practices. This monograph
will be both highly valuable for both activists and researchers.”
— VÉRO NIQU E D UD O UE T
ICNC’s Monograph Series publishes original studies
that stimulate and enrich policy discussions around
civil resistance and nonviolent movements.
This report underwent peer review.
PUBLISHED BY ICNC PRESS
International Center on Nonviolent Conflict
600 New Hampshire Ave NW, Suite 710
Washington, D.C. 20037
www.nonviolent-conflict.org
ISBN 978-1-943271-40-5
9 781943 271405
90000>

\end{document}
